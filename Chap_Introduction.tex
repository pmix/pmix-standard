%%%%%%%%%%%%%%%%%%%%%%%%%%%%%%%%%%%%%%%%%%%%%%%%%
% Chapter: Introduction
%%%%%%%%%%%%%%%%%%%%%%%%%%%%%%%%%%%%%%%%%%%%%%%%%
\chapter{Introduction}
\label{chap:intro}

The \ac{PMI} has been used for quite some time as a means of exchanging wireup information needed for inter-process communication.
Two versions (PMI-1 and PMI-2) have been released as part of the MPICH effort, with PMI-2 demonstrating better scaling properties than its PMI-1 predecessor. However, two significant challenges face the \ac{HPC} community as it continues to move towards machines capable of exaflop and higher performance levels:

\begin{itemize}
\item the physical scale of the machines, and the corresponding number of total processes they support, is expected to reach levels approaching  1 million processes executing across 100 thousand nodes. Prior methods for initiating applications relied on exchanging communication endpoint information between the processes, either directly or in some form of hierarchical collective operation. Regardless of the specific mechanism employed, the exchange across such large applications would consume considerable time, with estimates running in excess of 5-10 minutes; and
\item whether it be hybrid applications that combine OpenMP threading operations with MPI, or application-steered workflow computations, the HPC community is experiencing an unprecedented wave of new approaches for computing at exascale levels. One common thread across the proposed methods is an increasing need for orchestration between the application and the \ac{SMS} comprising the scheduler (a.k.a. the \ac{WLM}), the \ac{RM}, global file system, fabric, and other subsystems. The lack of available support for application-to-SMS integration has forced researchers to develop "virtual" environments that hide the SMS behind a customized abstraction layer, but this results in considerable duplication of effort and a lack of portability.
\end{itemize}

\ac{PMIx} represents an attempt to resolve these questions by providing an extended version of the \ac{PMI} definitions specifically designed to support clusters up to exascale and larger sizes.
The overall objective of the project is not to branch the existing definitions -- in fact, \ac{PMIx} fully supports both of the existing PMI-1 and PMI-2 \acp{API} -- but rather to:

\begin{compactalphaenum}
\item add flexibility to the existing \acp{API} by adding an array of key-value ``attribute'' pairs to each \ac{API} signature that allows implementers to customize the behavior of the \ac{API} as future needs emerge without having to alter or create new variants of it;

\item add new APIs that provide extended capabilities such as asynchronous event notification plus dynamic resource allocation and management;

\item establish a collaboration between \ac{SMS} subsystem providers including resource manager, fabric, file system, and programming library developers to define integration points between the various subsystems as well as agreed upon definitions for associated \acp{API}, attribute names, and data types;

\item form a standards-like body for the definitions; and

\item provide a reference implementation of the \ac{PMIx} standard.
\end{compactalphaenum}

Complete information about the \ac{PMIx} standard and affiliated projects can be found at the \ac{PMIx} web site: \url{https://pmix.org}


%%%%%%%%%%%%%%%%%%%%%%%%%%%%%%%%%%%%%%%%%%%%%%%%%
%%%%%%%%%%%%%%%%%%%%%%%%%%%%%%%%%%%%%%%%%%%%%%%%%
\section{Charter}
\label{chap:intro:charter}

The charter of the PMIx community is to:
\begin{itemize}
\item Define a set of agnostic APIs (not affiliated with any specific programming model or code base) to support interactions between application processes and the \ac{SMS}.
\item Develop an open source (non-copy-left licensed) standalone ``reference'' library to facilitate adoption of the \ac{PMIx} standard.
\item Retain transparent backward compatibility with the existing PMI-1 and PMI-2 definitions, any future \ac{PMI} releases, and across all \ac{PMIx} versions.
\item Support the ``Instant On'' initiative for rapid startup of applications at exascale and beyond.
\item Work with the \ac{HPC} community to define and implement new \acp{API} that support evolving programming model requirements for application interactions with the \ac{SMS}.
\end{itemize}

Participation in the \ac{PMIx} community is open to anyone, and not restricted to only code contributors to the reference implementation.


%%%%%%%%%%%%%%%%%%%%%%%%%%%%%%%%%%%%%%%%%%%%%%%%%
%%%%%%%%%%%%%%%%%%%%%%%%%%%%%%%%%%%%%%%%%%%%%%%%%
\section{PMIx Standard Overview}
\label{chap:intro:std_overview}

The \ac{PMIx} Standard defines and describes the interface developed by the \acf{PRL}.
Much of this document is specific to the \acf{PRL}'s design and implementation.
Specifically the standard describes the functionality provided by the \ac{PRL}, and what the \ac{PRL} requires of the clients and \acfp{RM} that use it's interface.

%%%%%%%%%%%
\subsection{Who should use the standard?}

The \ac{PMIx} Standard informs PMIx clients and \acp{RM} of the syntax and semantics of the \ac{PMIx} APIs.

\ac{PMIx} clients (e.g., tools, \ac{MPE} libraries) can use this standard to understand the set of attributes provided by various APIs of the \ac{RPL} and their intended behavior.
Additional information about the rationale for the selection of specific interfaces and attributes is also provided.

\ac{PMIx}-enabled \acp{RM} can use this standard to understand the expected behavior required of them when they support various interfaces/attributes.
Additional, optional features and suggestions on behavior are also included in the discussion to help guide \ac{RM} design and implementation.

%%%%%%%%%%%
\subsection{What is defined in the standard?}

The \ac{PMIx} Standard defines and describes the interface developed by the \acf{PRL}.
It defines the set of attributes that the \ac{PRL} supports; the set of attributes that are required of a \ac{RM} to support, for a given interface; and the set of optional attributes that an \ac{RM} may choose to support, for a given interface.

%%%%%%%%%%%
\subsection{What is \emph{not} defined in the standard?}

No standards body can require an implementer to support something in their standard, and \ac{PMIx} is no different in that regard. While an implementer of the \ac{PMIx} library itself must at least include the standard \ac{PMIx} headers and instantiate each function, they are free to return ``not supported'' for any function they choose not to implement.

This also applies to the host environments. Resource managers and other system management stack components retain the right to decide on support of a particular function. The \ac{PMIx} community continues to look at ways to assist \ac{SMS} implementers in their decisions by highlighting functions that are critical to basic application execution (e.g., \refapi{PMIx_Get}), while leaving flexibility for tailoring a vendor's software for their target market segment.

One area where this can become more complicated is regarding the attributes that provide information to the client process and/or control the behavior of a \ac{PMIx} standard \ac{API}. For example, the \refattr{PMIX_TIMEOUT} attribute can be used to specify the time (in seconds) before the requested operation should time out. The intent of this attribute is to allow the client to avoid ``hanging'' in a request that takes longer than the client wishes to wait, or may never return (e.g., a \refapi{PMIx_Fence} that a blocked participant never enters).

If an application (for example) truly relies on the \refattr{PMIX_TIMEOUT} attribute in a call to \refapi{PMIx_Fence}, it should set the required flag in the \refattr{pmix_info_t} for that attribute. This informs the library and its \ac{SMS} host that it must return an immediate error if this attribute is not supported. By not setting the flag, the library and \ac{SMS} host are allowed to treat the attribute as optional, ignoring it if support is not available.

It is therefore critical that users and application implementers:

\begin{compactalphaenum}
\item consider whether or not a given attribute is required, marking it accordingly; and

\item check the return status on all \ac{PMIx} function calls to ensure support was present and that the request was accepted. Note that for non-blocking \acp{API}, a return of \refconst{PMIX_SUCCESS} only indicates that the request had no obvious errors and is being processed – the eventual callback will return the status of the requested operation itself.
\end{compactalphaenum}

While a \ac{PMIx} library implementer, or an \ac{SMS} component server, may choose to support a particular \ac{PMIx} \ac{API}, they are not required to support every attribute that might apply to it. This would pose a significant barrier to entry for an implementer as there can be a broad range of applicable attributes to a given \ac{API}, at least some of which may rarely be used. The \ac{PMIx} community is attempting to help differentiate the attributes by indicating those that are generally used (and therefore, of higher importance to support) vs those that a ``complete implementation'' would support.

Note that an environment that does not include support for a particular attribute/\ac{API} pair is not ``incomplete'' or of lower quality than one that does include that support. Vendors must decide where to invest their time based on the needs of their target markets, and it is perfectly reasonable for them to perform cost/benefit decisions when considering what functions and attributes to support.

The flip side of that statement is also true: Users who find that their current vendor does not support a function or attribute they require may raise that concern to their vendor and request that the implementation be expanded. Alternatively, users may wish to utilize the \ac{PRRTE} as a ``shim'' between their application and the host environment as it might provide the desired support until the vendor can respond. Finally, in the extreme, one can exploit the portability of PMIx-based applications to change vendors.

%%%%%%%%%%%
\subsection{General Guidance for PMIx Users and Implementors}

The \ac{PMIx} Standard defines the behavior of the \acf{PRL}.
A complete system harnessing the \ac{PMIx} interface requires an agreement between the \ac{PMIx} client, be it a tool or library, and the \ac{PMIx}-enabled \ac{RM}.
The \ac{RPL} acts as an intermediary between these two entities providing a standard API for the exchange of requests and responses.
The degree to which the \ac{PMIx} client and the \ac{PMIx}-enabled \ac{RM} may interact needs to be defined by those developer communities.
The \ac{PMIx} standard can be used to define the specifics of this interaction.

\ac{PMIx} clients (e.g., tools, \ac{MPE} libraries) may find that they depend only on a small subset of interfaces and attributes to work correctly.
\ac{PMIx} clients are strongly advised to define a document itemizing the \ac{PMIx} interfaces and associated attributes that are required for correct operation, and are optional but recommended for full functionality.
The \ac{PMIx} standard cannot define this list for all given \ac{PMIx} clients, but such a list is valuable to \acp{RM} desiring to support these clients.

\ac{PMIx}-enabled \acp{RM} may choose to implement a subset of the \ac{PMIx} standard and/or define attributes beyond those defined herein.
\ac{PMIx}-enabled \acp{RM} are strongly advised to define a document itemizing the \ac{PMIx} interfaces and associated attributes that support, with any annotations about behavior limitations.
The \ac{PMIx} standard cannot define this list for all given \ac{PMIx}-enabled \acp{RM}, but such a list is valuable to \ac{PMIx} clients desiring to support a broad range of \ac{PMIx}-enabled \acp{RM}.



%%%%%%%%%%%%%%%%%%%%%%%%%%%%%%%%%%%%%%%%%%%%%%%%%
%%%%%%%%%%%%%%%%%%%%%%%%%%%%%%%%%%%%%%%%%%%%%%%%%
\section{PMIx Architecture Overview}
\label{chap:intro:arch_overview}

This section presents a brief overview the \ac{PMIx} Architecture~\cite{2017-Castain-EuroMPI}.

\ldots

%%%%%%%%%%%
\subsection{The PMIx Reference Implementation}(PRI)

Note that the definition of the \ac{PMIx} Standard is not contingent upon use of the \ac{PMIx} Reference Implementation --- any implementation that supports the defined \acp{API} is a \ac{PMIx} Standard compliant implementation.
The \ac{PMIx} Reference Implementation is provided solely for the following purposes:
\begin{itemize}
\item Validation of the standard.\\
No proposed change and/or extension to the \ac{PMIx} standard is accepted without an accompanying prototype implementation in the \ac{PMIx} Reference Implementation.
This ensures that the proposal has undergone at least some minimal level of scrutiny and testing before being considered.
\item Ease of adoption.\\
The \ac{PMIx} Reference Implementation is designed to be particularly easy for resource managers (and the \ac{SMS} in general) to adopt, thus facilitating a rapid uptake into that community for application portability.
Both client and server \ac{PMIx} libraries are included, along with examples of client usage and server-side integration.
A list of supported environments and versions is maintained on the \ac{PMIx} web site \url{https://pmix.org/support/faq/what-apis-are-supported-on-my-rm/}
\end{itemize}

The \ac{PMIx} Reference Implementation targets support for the Linux operating system.
A reasonable effort is made to support all major, modern Linux distributions; however, validation is limited to the most recent 2-3 releases of RedHat Enterprise Linux (RHEL), Fedora, CentOS, and SUSE Linux Enterprise Server (SLES).
In addition, development support is maintained for Mac OSX.
Production support for vendor-specific operating systems is included as provided by the vendor.

%%%%%%%%%%%
\subsection{The PMIx Reference RunTime Environment}(PRRTE)

\ldots

The PMIx community has released \ac{PRRTE} --- i.e., a runtime environment
containing the reference implementation and capable of operating within a host \ac{SMS}. \ac{PRRTE}
provides an easy way of exploring \ac{PMIx} capabilities and testing PMIx-based
applications outside of a PMIx-enabled environment. Instructions for downloading,
installing, and using \ac{PRI} and \ac{PRRTE} are available
on the community's web site

%%%%%%%%%%%%%%%%%%%%%%%%%%%%%%%%%%%%%%%%%%%%%%%%%
\section{Organization of this document}

The remainder of this document is structured as follows:

\begin{itemize}
\item Introduction and Overview in \chapterref{chap:intro}
\item Terms and Conventions in \chapterref{chap:terms}
\item Data Structures and Types in \chapterref{chap:struct}
\item \ac{PMIx} Initialization and Finalization in \chapterref{chap:api_init}
\item Key/Value Management in \chapterref{chap:api_kv_mgmt}
\item Process Management in \chapterref{chap:api_proc_mgmt}
\item Job Management in \chapterref{chap:api_job_mgmt}
\item Event Notification in \chapterref{chap:api_event}
\item Data Packing and Unpacking in \chapterref{chap:api_data_mgmt}
\item \ac{PMIx} Server Specific Interfaces in \chapterref{chap:api_server}
\end{itemize}

%%%%%%%%%%%%%%%%%%%%%%%%%%%%%%%%%%%%%%%%%%%%%%%%%
