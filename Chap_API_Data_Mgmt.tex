%%%%%%%%%%%%%%%%%%%%%%%%%%%%%%%%%%%%%%%%%%%%%%%%%
% Chapter: Data Packing and Unpacking
%%%%%%%%%%%%%%%%%%%%%%%%%%%%%%%%%%%%%%%%%%%%%%%%%
\chapter{Data Packing and Unpacking}
\label{chap:api_data_mgmt}

\ldots

%%%%%%%%%%%%%%%%%%%%%%%%%%%%%%%%%%%%%%%%%%%%%%
%%%%%%%%%%%%%%%%%%%%%%%%%%%%%%%%%%%%%%%%%%%%%%
\section{General Routines}
\label{chap:api_init:general}

\ldots

%%%%%%%%%%%
\subsection{\code{PMIx_Data_pack}}
\declareapi{PMIx_Data_pack}

\cspecificstart
\begin{codepar}
/**
 * Top-level interface function to pack one or more values into a
 * buffer.
 *
 * The pack function packs one or more values of a specified type into
 * the specified buffer.  The buffer must have already been
 * initialized via the PMIX_DATA_BUFFER_CREATE or PMIX_DATA_BUFFER_CONSTRUCT
 * call - otherwise, the pack_value function will return an error.
 * Providing an unsupported type flag will likewise be reported as an error.
 *
 * Note that any data to be packed that is not hard type cast (i.e.,
 * not type cast to a specific size) may lose precision when unpacked
 * by a non-homogeneous recipient.  The PACK function will do its best to deal
 * with heterogeneity issues between the packer and unpacker in such
 * cases. Sending a number larger than can be handled by the recipient
 * will return an error code (generated upon unpacking) -
 * the error cannot be detected during packing.
 *
 * @param *buffer A pointer to the buffer into which the value is to
 * be packed.
 *
 * @param *src A void* pointer to the data that is to be packed. Note
 * that strings are to be passed as (char **) - i.e., the caller must
 * pass the address of the pointer to the string as the void*. This
 * allows PMIx to use a single pack function, but still allow
 * the caller to pass multiple strings in a single call.
 *
 * @param num_values An int32_t indicating the number of values that are
 * to be packed, beginning at the location pointed to by src. A string
 * value is counted as a single value regardless of length. The values
 * must be contiguous in memory. Arrays of pointers (e.g., string
 * arrays) should be contiguous, although (obviously) the data pointed
 * to need not be contiguous across array entries.
 *
 * @param type The type of the data to be packed - must be one of the
 * PMIX defined data types.
 *
 * @retval PMIX_SUCCESS The data was packed as requested.
 *
 * @retval PMIX_ERROR(s) An appropriate PMIX error code indicating the
 * problem encountered. This error code should be handled
 * appropriately.
 *
 * @code
 * pmix_data_buffer_t *buffer;
 * int32_t src;
 *
 * PMIX_DATA_BUFFER_CREATE(buffer);
 * status_code = PMIx_Data_pack(buffer, &src, 1, PMIX_INT32);
 * @endcode
 */
pmix_status_t
PMIx_Data_pack(pmix_data_buffer_t *buffer,
               void *src, int32_t num_vals,
               pmix_data_type_t type);
\end{codepar}
\cspecificend


%%%%%%%%%%%
\subsection{\code{PMIx_Data_unpack}}
\declareapi{PMIx_Data_unpack}

\cspecificstart
\begin{codepar}
/**
 * Unpack values from a buffer.
 *
 * The unpack function unpacks the next value (or values) of a
 * specified type from the specified buffer.
 *
 * The buffer must have already been initialized via an PMIX_DATA_BUFFER_CREATE or
 * PMIX_DATA_BUFFER_CONSTRUCT call (and assumedly filled with some data) -
 * otherwise, the unpack_value function will return an
 * error. Providing an unsupported type flag will likewise be reported
 * as an error, as will specifying a data type that DOES NOT match the
 * type of the next item in the buffer. An attempt to read beyond the
 * end of the stored data held in the buffer will also return an
 * error.
 *
 * NOTE: it is possible for the buffer to be corrupted and that
 * PMIx will *think* there is a proper variable type at the
 * beginning of an unpack region - but that the value is bogus (e.g., just
 * a byte field in a string array that so happens to have a value that
 * matches the specified data type flag). Therefore, the data type error check
 * is NOT completely safe. This is true for ALL unpack functions.
 *
 *
 * Unpacking values is a "nondestructive" process - i.e., the values are
 * not removed from the buffer. It is therefore possible for the caller
 * to re-unpack a value from the same buffer by resetting the unpack_ptr.
 *
 * Warning: The caller is responsible for providing adequate memory
 * storage for the requested data. As noted below, the user
 * must provide a parameter indicating the maximum number of values that
 * can be unpacked into the allocated memory. If more values exist in the
 * buffer than can fit into the memory storage, then the function will unpack
 * what it can fit into that location and return an error code indicating
 * that the buffer was only partially unpacked.
 *
 * Note that any data that was not hard type cast (i.e., not type cast
 * to a specific size) when packed may lose precision when unpacked by
 * a non-homogeneous recipient.  PMIx will do its best to deal with
 * heterogeneity issues between the packer and unpacker in such
 * cases. Sending a number larger than can be handled by the recipient
 * will return an error code generated upon unpacking - these errors
 * cannot be detected during packing.
 *
 * @param *buffer A pointer to the buffer from which the value will be
 * extracted.
 *
 * @param *dest A void* pointer to the memory location into which the
 * data is to be stored. Note that these values will be stored
 * contiguously in memory. For strings, this pointer must be to (char
 * **) to provide a means of supporting multiple string
 * operations. The unpack function will allocate memory for each
 * string in the array - the caller must only provide adequate memory
 * for the array of pointers.
 *
 * @param type The type of the data to be unpacked - must be one of
 * the BFROP defined data types.
 *
 * @retval *max_num_values The number of values actually unpacked. In
 * most cases, this should match the maximum number provided in the
 * parameters - but in no case will it exceed the value of this
 * parameter.  Note that if you unpack fewer values than are actually
 * available, the buffer will be in an unpackable state - the function will
 * return an error code to warn of this condition.
 *
 * @note The unpack function will return the actual number of values
 * unpacked in this location.
 *
 * @retval PMIX_SUCCESS The next item in the buffer was successfully
 * unpacked.
 *
 * @retval PMIX_ERROR(s) The unpack function returns an error code
 * under one of several conditions: (a) the number of values in the
 * item exceeds the max num provided by the caller; (b) the type of
 * the next item in the buffer does not match the type specified by
 * the caller; or (c) the unpack failed due to either an error in the
 * buffer or an attempt to read past the end of the buffer.
 *
 * @code
 * pmix_data_buffer_t *buffer;
 * int32_t dest;
 * char **string_array;
 * int32_t num_values;
 *
 * num_values = 1;
 * status_code = PMIx_Data_unpack(buffer, (void*)&dest, &num_values, PMIX_INT32);
 *
 * num_values = 5;
 * string_array = malloc(num_values*sizeof(char *));
 * status_code = PMIx_Data_unpack(buffer, (void*)(string_array), &num_values, PMIX_STRING);
 *
 * @endcode
 */
pmix_status_t
PMIx_Data_unpack(pmix_data_buffer_t *buffer, void *dest,
                 int32_t *max_num_values,
                 pmix_data_type_t type);
\end{codepar}
\cspecificend


%%%%%%%%%%%
\subsection{\code{PMIx_Data_copy}}
\declareapi{PMIx_Data_copy}

\cspecificstart
\begin{codepar}
/**
 * Copy a data value from one location to another.
 *
 * Since registered data types can be complex structures, the system
 * needs some way to know how to copy the data from one location to
 * another (e.g., for storage in the registry). This function, which
 * can call other copy functions to build up complex data types, defines
 * the method for making a copy of the specified data type.
 *
 * @param **dest The address of a pointer into which the
 * address of the resulting data is to be stored.
 *
 * @param *src A pointer to the memory location from which the
 * data is to be copied.
 *
 * @param type The type of the data to be copied - must be one of
 * the PMIx defined data types.
 *
 * @retval PMIX_SUCCESS The value was successfully copied.
 *
 * @retval PMIX_ERROR(s) An appropriate error code.
 *
 */
pmix_status_t
PMIx_Data_copy(void **dest, void *src,
               pmix_data_type_t type);
\end{codepar}
\cspecificend


%%%%%%%%%%%
\subsection{\code{PMIx_Data_print}}
\declareapi{PMIx_Data_print}

\cspecificstart
\begin{codepar}
/**
 * Print a data value.
 *
 * Since registered data types can be complex structures, the system
 * needs some way to know how to print them (i.e., convert them to a string
 * representation). Provided for debug purposes.
 *
 * @retval PMIX_SUCCESS The value was successfully printed.
 *
 * @retval PMIX_ERROR(s) An appropriate error code.
 */
pmix_status_t
PMIx_Data_print(char **output, char *prefix,
                void *src, pmix_data_type_t type);
\end{codepar}
\cspecificend


%%%%%%%%%%%
\subsection{\code{PMIx_Data_copy_payload}}
\declareapi{PMIx_Data_copy_payload}

\cspecificstart
\begin{codepar}
/**
 * Copy a payload from one buffer to another
 *
 * This function will append a copy of the payload in one buffer into
 * another buffer.
 * NOTE: This is NOT a destructive procedure - the
 * source buffer's payload will remain intact, as will any pre-existing
 * payload in the destination's buffer.
 */
pmix_status_t
PMIx_Data_copy_payload(pmix_data_buffer_t *dest,
                       pmix_data_buffer_t *src);
\end{codepar}
\cspecificend


%%%%%%%%%%%%%%%%%%%%%%%%%%%%%%%%%%%%%%%%%%%%%%%%%
