% Welcome to pmix-standard.tex.
% This is the master LaTex file for the PMIx Standard document.
%
% The files in this set include:
%
%    pmix-standard.tex                - this file, the master file
%    Makefile                         - makes the document
%    pmix.sty                         - the main style file
%    Title_Page.tex                   - the title page
%    Chap_Introduction.tex            - unnumbered introductory chapter
%    figs/*.png                       - Figures
%    sources/*.c, *.f                 - C/C++/Fortran example source files
%
% When editing this file:
%
%    1. To change formatting, appearance, or style, please edit pmix.sty.
%
%    2. Custom commands and macros are defined in pmix.sty.
%
%    3. Be kind to other editors -- keep a consistent style by copying-and-pasting to
%       create new content.
%
%    4. We use semantic markup, e.g. (see pmix.sty for a full list):
%         \code{}     % for bold monospace keywords, code, operators, etc.
%
%    5. Other recommendations:
%         Use the convenience macros defined in pmix.sty for the minor headers
%         such as Comments, Syntax, etc.
%
%         To keep items together on the same page, prefer the use of
%         \begin{samepage}.... Avoid \parbox for text blocks as it interrupts line numbering.
%         When possible, avoid \filbreak, \pagebreak, \newpage, \clearpage unless that's
%         what you mean. Use \needspace{} cautiously for troublesome paragraphs.
%
%         Avoid absolute lengths and measures in this file; use relative units when possible.
%         Vertical space can be relative to \baselineskip or ex units. Horizontal space
%         can be relative to \linewidth or em units.
%
%         Prefer \emph{} to italicize terminology, e.g.:
%             This is a \emph{definition}, not a placeholder.
%             This is a \plc{var-name}.
%

% The following says letter size, but the style sheet may change the size
\documentclass[10pt,letterpaper,twoside,makeidx,hidelinks]{scrreprt}

% Text to appear in the footer on even-numbered pages:
\newcommand{\VER}{4.0 (Draft)}
\newcommand{\VERDATE}{1H2019}
\newcommand{\footerText}{PMIx Standard -- Version \VER{} -- \VERDATE}

% Unified style sheet for PMIx documents:
% This is pmix.sty, the preamble and style definitions for the PMIx specification.
%
% This specification file, and latex structure was derived from/inspired by the OpenMP specification. So some similarity between the two latex files is expected.
%
%%%%%%%%%%%%%%%%%%%%%%%%%%%%%%%%%%%%%%%%%%%%%%%%%%%%%%%%%%%%%%%%%%%%%%%%%%%%%%%%%%%%%%%%%%%%%
% Quick list of the environments, commands and macros supported.
% Search below for more details.
%
% Formatting Text:
%   -----------------------
%   \notestart            - "Note:" Callout section
%   \noteheader           - \noteheader is optional "Note:" prefix for text
%     ...
%   \noteend
%   -----------------------
%   \rationalestart       - "Rationale" Callout section
%     ...
%   \rationaleend
%   -----------------------
%   \adviceuserstart      - "Advice to users" Callout section
%     ...
%   \adviceuserend
%   -----------------------
%   \adviceimplstart      - "Advice to PMIx library implementers" Callout section
%     ...
%   \adviceimplend
%   -----------------------
%   \advicermstart      - "Advice to PMIx server hosts" Callout section
%     ...
%   \advicermend
%   -----------------------
%
% Formatting Code:
%   \code{}               - Code text
%   \var{}                - Variable
%   -----------------------
%   \begin{codepar}       - Section of generic code
%     ...                 - use language specific macro if language specific code
%   \end[codepar}
%   -----------------------
%   \cspecificstart       - C specific code block
%     ...
%   \cspecificend
%   -----------------------
%
% Attributes:
%   \refAttributeItem{}   - Cross reference
%   \refattr{}            - Same as above
%   \pasteAttributeItem{} - Paste full description
%
% Structures:
%   \refstruct{}          - Reference a structure
%   \structref{}          - Same as above
%   \specrefstruct{}      - Reference a structure by section number and page
%
% APIs:
%   \refapi{}             - Reference an API function
%   \refconst{}           - Constant reference
%   \refarg{} / \argref{} - Reference an argument to an API function
%
% Cross referencing:
%   \chapterref{}         - Reference a Chapter by number and page
%   \specref{}            - Reference a Section by number and page
%
%%%%%%%%%%%%%%%%%%%%%%%%%%%%%%%%%%%%%%%%%%%%%%%%%%%%%%%%%%%%%%%%%%%%%%%%%%%%%%%%%%%%%%%%%%%%%
\usepackage{comment}            % allow use of \begin{comment}
\usepackage{ifpdf,ifthen}       % allow conditional tests in LaTeX definitions
\usepackage{makecell}           % Allows common formatting in cells with \thread & \makecell

\usepackage[T1]{fontenc}        % Allow us to use underscore freely in the document
\catcode`\_=12                  % Use \sb for subscripts
\usepackage{verbatim}


%%%%%%%%%%%%%%%%%%%%%%%%%%%%%%%%%%%%%%%%%%%%%%%%%%%%%%%%%%%%%%%%%%%%%%%%%%%%%%%%%%%%%%%%%%%%%
% Document data
%
\author{}


%%%%%%%%%%%%%%%%%%%%%%%%%%%%%%%%%%%%%%%%%%%%%%%%%%%%%%%%%%%%%%%%%%%%%%%%%%%%%%%%%%%%%%%%%%%%%
% Fonts

\usepackage{amsmath}
\usepackage{amsfonts}
\usepackage{amssymb}
\usepackage{courier}
\usepackage{helvet}
\usepackage[utf8]{inputenc}
\usepackage{textgreek}

% Main body serif font:
\usepackage{tgtermes}
\usepackage[T1]{fontenc}


%%%%%%%%%%%%%%%%%%%%%%%%%%%%%%%%%%%%%%%%%%%%%%%%%%%%%%%%%%%%%%%%%%%%%%%%%%%%%%%%%%%%%%%%%%%%%
% Graphic elements

\usepackage{graphicx}
\usepackage{framed}    % for making boxes with \begin{framed}
\usepackage{tikz}      % for flow charts, diagrams, arrows


%%%%%%%%%%%%%%%%%%%%%%%%%%%%%%%%%%%%%%%%%%%%%%%%%%%%%%%%%%%%%%%%%%%%%%%%%%%%%%%%%%%%%%%%%%%%%
% Page formatting

\usepackage[paperwidth=7.5in, paperheight=9in,
            top=0.75in, bottom=1.0in, left=1.4in, right=0.6in]{geometry}

\usepackage{changepage}   % allows left/right-page margin readjustments

\setlength{\oddsidemargin}{0.185in}
\setlength{\evensidemargin}{0.185in}
\raggedbottom


%%%%%%%%%%%%%%%%%%%%%%%%%%%%%%%%%%%%%%%%%%%%%%%%%%%%%%%%%%%%%%%%%%%%%%%%%%%%%%%%%%%%%%%%%%%%%
% Paragraph formatting

\usepackage{setspace}     % allows use of \singlespacing, \onehalfspacing
\usepackage{needspace}    % allows use of \needspace to keep lines together
\usepackage{parskip}      % removes paragraph indenting

\raggedright
\usepackage[raggedrightboxes]{ragged2e}  % is this needed?

\lefthyphenmin=60         % only hyphenate if the left part is >= this many chars
\righthyphenmin=60        % only hyphenate if the right part is >= this many chars


%%%%%%%%%%%%%%%%%%%%%%%%%%%%%%%%%%%%%%%%%%%%%%%%%%%%%%%%%%%%%%%%%%%%%%%%%%%%%%%%%%%%%%%%%%%%%%
% Bulleted (itemized) lists
%    Align bullets with section header
%    Align text left
%    Small bullets
%    \compactitem for single-spaced lists (used in the Examples doc)

\usepackage{enumitem}     % for setting margins on lists
\setlist{leftmargin=*}    % don't indent bullet items
\renewcommand{\labelitemi}{{\normalsize$\bullet$}} % bullet size

% There is a \compactitem defined in package parlist (and perhaps others), however,
% we'll define our own version of compactitem in terms of package enumitem that
% we already use:
\newenvironment{compactitem}
{\begin{itemize}[itemsep=-1.2ex]}
{\end{itemize}}

%%%%%%%%%%%%%%%%%%%%%%%%%%%%%%%%%%%%%%%%%%%%%%%%%%%%%%%%%%%%%%%%%%%%%%%%%%%%%%%%%%%%%%%%%%%%%
% Floating version
%\usepackage[showboxes]{textpos}
\usepackage{textpos}

\setlength{\TPHorizModule}{1pt}%
\setlength{\TPVertModule}{\TPHorizModule}%
\TPMargin{1pt}%

\newcommand{\versionMarker}[1]{%
 % y is 8 = \parskip
 \begin{textblock}{50}(-55,8)%
   \textit{PMIx v#1}%
   \raggedright
 \end{textblock}%
}
% Alternative is to make a box inline, but that gets tricky when positioning close
% to codepar's
% \makebox[-7pt][r]{\textit{PMIx #4}\raggedright}

%%%%%%%%%%%%%%%%%%%%%%%%%%%%%%%%%%%%%%%%%%%%%%%%%%%%%%%%%%%%%%%%%%%%%%%%%%%%%%%%%%%%%%%%%%%%%%
% Enumerated list with lowercase alphabet lettering
%    \alphaenum for default-spaced lists
%    \compactalphaenum for single-spaced lists

% There is a \compactitem defined in package parlist (and perhaps others), however,
% we'll define our own version of compactitem in terms of package enumitem that
% we already use:
\newenvironment{alphaenum}
{\begin{enumerate}[label=\alph*)]}
{\end{enumerate}}

\newenvironment{compactalphaenum}
{\begin{enumerate}[label=\alph*),itemsep=-1.2ex]}
{\end{enumerate}}

% Argument list for an interface, for use in a \begin{arglist} section
% \argin      Input argument
% \argout     Output argument
% \arginout   Input/Output argument
% \argreturn  Value returned
%%% Old Method using tables.... line numbers didn't work if a cell wrapped...
%\newlength\argdesclen
%\setlength\argdesclen{\dimexpr \linewidth -13em -4\tabcolsep}
%\newenvironment{arglist}{%
%    \begin{edtable}{tabular}{p{3em}p{10em}p{\argdesclen}}}
%    {\end{edtable}\vspace{.25em}}
%
%\newcommand{\argin}[2]{\textbf{IN} & \code{#1} & #2\\}
%\newcommand{\argout}[2]{\textbf{OUT} & \code{#1} & #2\\}
%\newcommand{\arginout}[2]{\textbf{INOUT} & \code{#1} & #2\\}

\newenvironment{arglist}
{\begin{description}[style=nextline,labelindent=\parindent,leftmargin=*,itemindent=\dimexpr-17pt-\labelsep\relax,itemsep=-1.3ex]}
{\end{description}}

\newcommand{\argin}[2]{\item[IN ~~~~\code{#1}] #2}
\newcommand{\argout}[2]{\item[OUT ~~~\code{#1}] #2}
\newcommand{\arginout}[2]{\item[INOUT ~\code{#1}] #2}

% Constant list
%   \declareconstitem  Declare constant with description
\newenvironment{constantdesc}
{\begin{description}[itemsep=-1.3ex,itemindent=\dimexpr-17pt-\labelsep\relax]}
{\end{description}}

\newcommand{\declareconstitem}[1]{\item[\code{#1}] \index{#1} \label{const:#1} \hspace{1em}}
\newcommand{\declareconstitemvalue}[2]{\item[\code{#1}] \index{#1} \hspace{0.25em} \code{#2}  \hspace{1em}}
\newcommand{\declareconstitemDEP}[2]{\item[\code{#1} (Deprecated in PMIx #2)] \index{#1} \label{const:#1} \hspace{1em}}
\newcommand{\declareconstitemNEW}[1]{\item[\color{magenta}\code{#1}] \index{#1} \label{const:#1} \hspace{1em}}


%%%%%%%%%%%%%%%%%%%%%%%%%%%%%%%%%%%%%%%%%%%%%%%%%%%%%%%%%%%%%%%%%%%%%%%%%%%%%%%%%%%%%%%%%%%%%%
% Tables

% This allows tables to flow across page breaks, headers on each new page, etc.
\usepackage{supertabular}
\usepackage{caption}
\usepackage{longtable}
\usepackage{pdflscape} % for 'landscape' environment

%%%%%%%%%%%%%%%%%%%%%%%%%%%%%%%%%%%%%%%%%%%%%%%%%%%%%%%%%%%%%%%%%%%%%%%%%%%%%%%%%%%%%%%%%%%%%
% Line numbering

\usepackage[pagewise,edtable]{lineno}       % for line numbers on left side of the page
\pagewiselinenumbers
\setlength\linenumbersep{6em}
\renewcommand\linenumberfont{\normalfont\small\sffamily}
\nolinenumbers            % start with line numbers off


%%%%%%%%%%%%%%%%%%%%%%%%%%%%%%%%%%%%%%%%%%%%%%%%%%%%%%%%%%%%%%%%%%%%%%%%%%%%%%%%%%%%%%%%%%%%%
% Footers

\usepackage{fancyhdr}     % makes right/left footers
\pagestyle{fancy}
\fancyhead{} % clear all header fields
\cfoot{}
\renewcommand{\headrulewidth}{0pt}

% Left side on even pages:
% This requires that \footerText be defined in the master document:
\fancyfoot[LE]{\bfseries \thepage \mdseries \hspace{2em} \footerText}
\fancyhfoffset[E]{4em}

% Right side on odd pages:
\fancyfoot[RO]{\mdseries  \leftmark \hspace{2em} \bfseries \thepage}


%%%%%%%%%%%%%%%%%%%%%%%%%%%%%%%%%%%%%%%%%%%%%%%%%%%%%%%%%%%%%%%%%%%%%%%%%%%%%%%%%%%%%%%%%%%%%
% Section header format - we use five levels: \chapter \section \subsection \subsubsection

\usepackage{titlesec}     % format headers with \titleformat{}

% Format and spacing for chapter, section, subsection, and subsubsection headers:

\setcounter{secnumdepth}{5}          % show numbers down to subsubsection level

\titleformat{\chapter}[display]%
{\normalfont\sffamily\upshape\Huge\bfseries\nolinenumbers\fontsize{20}{20}\selectfont}%
{\normalfont\sffamily\scshape\large\bfseries\nolinenumbers \hspace{-0.7in} \MakeUppercase%
    {\chaptertitlename} \thechapter}%
{0em}{}[\vspace{1.0em}\hrule]
% {<left>}{<before-sep>}{<after-sep>}
\titlespacing{\chapter}{0ex}{0em plus 1em minus 1em}{1em plus 1em minus 1em}[10em]

\titleformat{\section}[hang]{\huge\bfseries\sffamily\fontsize{16}{16}\selectfont}{\thesection}{1.0em}{}
% {<left>}{<before-sep>}{<after-sep>}
\titlespacing{\section}{-5em}{2em plus 1em minus 1em}{1em plus 0.5em minus 0em}[10em]

\titleformat{\subsection}[hang]{\LARGE\bfseries\sffamily\fontsize{14}{14}\selectfont}{\thesubsection}{1.0em}{}
\titlespacing{\subsection}{-5em}{2em plus 1em minus 2.0em}{0.75em plus 0.5em minus 0em}[10em]

\titleformat{\subsubsection}[hang]{\needspace{1\baselineskip}%
\Large\bfseries\sffamily\fontsize{12}{12}\selectfont}{\thesubsubsection}{1.0em}{}
\titlespacing{\subsubsection}{-5em}{0.5em plus 1em minus 1em}{0.5em plus 0.5em minus 0em}[10em]


%%%%%%%%%%%%%%%%%%%%%%%%%%%%%%%%%%%%%%%%%%%%%%%%%%%%%%%%%%%%%%%%%%%%%%%%%%%%%%%%%%%%%%%%%%%%%%
% Macros for minor headers: Summary, Syntax, Description, etc.
% These headers are defined in terms of \paragraph

\titleformat{\paragraph}[block]{\large\bfseries\sffamily\fontsize{11}{11}\selectfont}{}{}{}
\titlespacing{\paragraph}{0em}{1.0em plus 0.55em minus 0.5em}{0.0em plus 0.55em minus 0.0em}

% Use one of the convenience macros below, or \littleheader{} for an arbitrary header
\newcommand{\littleheader}[1] {\paragraph*{#1}}

\newcommand{\comments} {\littleheader{Comments}}
\newcommand{\descr} {\littleheader{Description}}
\newcommand{\format} {\littleheader{Format}}
\newcommand{\summary} {\littleheader{Summary}}
\newcommand{\history} {\littleheader{History}}
\newcommand{\priattr} {\littleheader{PRI Attributes}}
\newcommand{\reqattr} {\littleheader{\ac{RM} Required Attributes}}
\newcommand{\optattr} {\littleheader{\ac{RM} Optional Attributes}}

%%%%%%%%%%%%%%%%%%%%%%%%%%%%%%%%%%%%%%%%%%%%%%%%%%%%%%%%%%%%%%%%%%%%%%%%%%%%%%%%%%%%%%%%%%%%%
% Clipboard
%
% \StdCopy{TAG}{BODY}
% \StdPaste{TAG}
%
% Inspired by this thread:
%   https://tex.stackexchange.com/questions/150790/how-to-make-text-be-copied-to-another-part-of-a-document
\makeatletter
\newcommand\StdCopy              [2] {
  \immediate\write\@auxout{\unexpanded{\global\long\@namedef{clipbrd@#1}{#2}}}
}
\newcommand\StdCopyEcho          [2] {
  \StdCopy{#1}{#2}%
  #2
}
\newcommand\StdPaste             [1] {%
  \ifcsname clipbrd@#1\endcsname
    \@nameuse{clipbrd@#1}%
  \else
    ??unknown??
  \fi
}
\makeatother


% Attributes
%   \declareAttribute       Declare an attribute with a description
%   \pasteAttributeItem     Paste the attribute description here
%   \refAttributeItem       Reference the original definition of the attribute
%
\newcommand{\declareAttribute}[4]{%
    \code{#1} ~~\code{#2}~~(\code{#3})%
    \index{#1!Definition|textbf} \label{attr:#1}%
    \StdCopy{str:#1}{\code{#2}}%
    \StdCopy{attr:#1}{\code{#3}}%
    \vspace{-1.3ex}%
      \expandafter\begin{adjustwidth}{.95cm}{}%
      \StdCopyEcho{#1}{#4}%
    \end{adjustwidth}%
  \vspace{-1.3ex}%
}

\newcommand{\declareNewAttribute}[4]{%
   {\color{magenta}\code{#1}} ~~\code{#2}~~(\code{#3})%
    \index{#1!Definition|textbf} \label{attr:#1}%
    \StdCopy{str:#1}{\code{#2}}%
    \StdCopy{attr:#1}{\code{#3}}%
    \vspace{-1.3ex}%
      \expandafter\begin{adjustwidth}{.95cm}{}%
      \StdCopyEcho{#1}{#4}%
    \end{adjustwidth}%
  \vspace{-1.3ex}%
}

\newcommand{\declareDepAttribute}[4]{%
   {\color{green!80!black}\code{#1}} ~~\code{#2}~~(\code{#3})%
    \index{#1!Definition|textbf} \label{attr:#1}%
    \StdCopy{str:#1}{\code{#2}}%
    \StdCopy{attr:#1}{\code{#3}}%
    \vspace{-1.3ex}%
      \expandafter\begin{adjustwidth}{.95cm}{}%
      \StdCopyEcho{#1}{#4}%
    \end{adjustwidth}%
  \vspace{-1.3ex}%
}

\newcommand{\pasteAttributeItemBegin}[1]{
  \refAttributeItem{#1} ~~\StdPaste{str:#1}~~(\StdPaste{attr:#1})
  \vspace{-1.3ex}
   \expandafter
   \begin{adjustwidth}{.95cm}{}
    \StdPaste{#1}
}
\newcommand{\pasteAttributeItemEnd}{
   \end{adjustwidth}
}
\newcommand{\pasteAttributeItem}[1]{
	\pasteAttributeItemBegin{#1}
	\pasteAttributeItemEnd{}
}
\newcommand{\refAttributeItem}[1]{\index{#1} \hyperref[attr:#1]{\code{#1}} }
\newcommand{\refattr}[1]{\refAttributeItem{#1}}

\newcommand{\refPRIAttributeItem}[1]{\index{#1} \hyperref[attr:#1]{\color{red}\code{#1}} }

\newcommand{\pastePRIAttributeItemBegin}[1]{
  \refPRIAttributeItem{#1} ~~\StdPaste{str:#1}~~(\StdPaste{attr:#1})
  \vspace{-1.3ex}
   \expandafter
   \begin{adjustwidth}{.95cm}{}
    \StdPaste{#1}
}
\newcommand{\pastePRIAttributeItemEnd}{
   \end{adjustwidth}
}

\newcommand{\pastePRIAttributeItem}[1]{
    \pastePRIAttributeItemBegin{#1}
    \pastePRIAttributeItemEnd{}
}

\newcommand{\refPRRTEAttributeItem}[1]{\index{#1} \hyperref[attr:#1]{\color{green!60!black}\code{#1}} }

\newcommand{\pastePRRTEAttributeItemBegin}[1]{
  \refPRRTEAttributeItem{#1} ~~\StdPaste{str:#1}~~(\StdPaste{attr:#1})
  \vspace{-1.3ex}
   \expandafter
   \begin{adjustwidth}{.95cm}{}
    \StdPaste{#1}
}
\newcommand{\pastePRRTEAttributeItemEnd}{
   \end{adjustwidth}
}

\newcommand{\pastePRRTEAttributeItem}[1]{
    \pastePRRTEAttributeItemBegin{#1}
    \pastePRRTEAttributeItemEnd{}
}

%%%%%%%%%%%%%%%%%%%%%%%%%%%%%%%%%%%%%%%%%%%%%%%%%%%%%%%%%%%%%%%%%%%%%%%%%%%%%%%%%%%%%%%%%%%%%%
% Code and placeholder semantic tagging.
%
% When possible, prefer semantic tags instead of typographic tags. The
% following semantics tags are defined here:
%
%     \code{}     % for bold monospace keywords, code, operators, etc.
%     \plc{}      % for italic placeholder names, grammar, etc.
%
% For function prototypes or other code snippets, you can use \code{} as
% the outer wrapper, and use \plc{{} inside. Example:
%
%     \code{\#pragma omp directive ( \plc{some-placeholder-identifier} :}
%
% To format text in italics for emphasis (rather than text as a placeholder),
% use the generic \emph{} command. Example:
%
%     This sentence \emph{emphasizes some non-placeholder words}.

% Enable \alltt{} for formatting blocks of code:
\usepackage{alltt}

% This sets the default \code{} font to tt (monospace) and bold:
\newcommand{\code}[1]{{\texttt{\textbf{#1}}}}
\newcommand{\var}[1] {{\textrm{\textmd{\itshape{#1}}}}}


% Environment for a paragraph of literal code, single-spaced, no outline, no indenting:
\newenvironment{codepar}[1]
{\begin{alltt}\bfseries #1}
{\end{alltt}}

\usepackage{setspace}

%%%%%%%%%%%%%%%%%%%%%%%%%%%%%%%%%%%%%%%%%%%%%%%%%%%%%%%%%%%%%%%%%%%%%%%%%%%%%%%%%%%%%%%%%%%%%%
% Macros for the black and blue lines and arrows delineating language-specific
% and notes sections. Example:
%
%   \fortranspecificstart
%   This is text that applies to Fortran.
%   \fortranspecificend

% local parameters for use \linewitharrows and \notelinewitharrows:
\newlength{\sbsz}\setlength{\sbsz}{0.05in}  % size of arrows
\newlength{\sblw}\setlength{\sblw}{1.35pt}  % line width (thickness)
\newlength{\sbtw}                           % text width
\newlength{\sblen}                          % total width of horizontal rule
\newlength{\sbht}                           % height of arrows
\newlength{\sbhadj}                         % vertical adjustment for aligning arrows with the line
\newlength{\sbns}\setlength{\sbns}{7\baselineskip}       % arg for \needspace for downward arrows

% \notelinewitharrows is a helper command that makes a black Note marker:
%     arg 1 = 1 or -1 for up or down arrows
%     arg 2 = solid or dashed or loosely dashed, etc.
\newcommand{\notelinewitharrows}[2]{%
    \needspace{0.1\baselineskip}%
    \vbox{\begin{tikzpicture}%
        \setlength{\sblen}{\linewidth}%
        \setlength{\sbht}{#1\sbsz}\setlength{\sbht}{1.4\sbht}%
        \setlength{\sbhadj}{#1\sblw}\setlength{\sbhadj}{0.25\sbhadj}%
        \filldraw (\sblen, 0) -- (\sblen - \sbsz, \sbht) -- (\sblen - 2\sbsz, 0) -- (\sblen, 0);
        \draw[line width=\sblw, #2] (2\sbsz - \sblw, \sbhadj) -- (\sblen - 2\sbsz + \sblw, \sbhadj);
        \filldraw (0, 0) -- (\sbsz, \sbht) -- (0 + 2\sbsz, 0) -- (0, 0);
    \end{tikzpicture}}}

% \adviceuserline is a helper command that makes a red horizontal line, up or down arrows, and some text:
% arg 1 = 1 or -1 for up or down arrows
% arg 2 = solid or dashed or loosely dashed, etc.
% arg 3 = text
% arg 4 = text width
\newcommand{\adviceuserline}[4]{%
    \needspace{0.1\baselineskip}%
    \vbox to 1\baselineskip {\begin{tikzpicture}%
        \setlength{\sbtw}{#4}%
        \setlength{\sblen}{\linewidth}%
        \setlength{\sbht}{#1\sbsz}\setlength{\sbht}{1.4\sbht}%
        \setlength{\sbhadj}{#1\sblw}\setlength{\sbhadj}{0.25\sbhadj}%
        \filldraw[color=red!80!black] (\sblen, 0) -- (\sblen - \sbsz, \sbht) -- (\sblen - 2\sbsz, 0) -- (\sblen, 0);
        \draw[line width=\sblw, color=red!80!black, #2] (2\sbsz - \sblw, \sbhadj) -- (0.5\sblen - 0.5\sbtw, \sbhadj);
        \draw[line width=\sblw, color=red!80!black, #2] (0.5\sblen + 0.5\sbtw, \sbhadj) -- (\sblen - 2\sbsz + \sblw, \sbhadj);
        \filldraw[color=red!80!black] (0, 0) -- (\sbsz, \sbht) -- (0 + 2\sbsz, 0) -- (0, 0);
        \node[color=red!80!black] at (0.5\sblen, 0) {\large  \textsf{\textup{#3}}};
    \end{tikzpicture}}}

% \adviceimpline is a helper command that makes a green horizontal line, up or down arrows, and some text:
% arg 1 = 1 or -1 for up or down arrows
% arg 2 = solid or dashed or loosely dashed, etc.
% arg 3 = text
% arg 4 = text width
\newcommand{\adviceimpline}[4]{%
    \needspace{0.1\baselineskip}%
    \vbox to 1\baselineskip {\begin{tikzpicture}%
        \setlength{\sbtw}{#4}%
        \setlength{\sblen}{\linewidth}%
        \setlength{\sbht}{#1\sbsz}\setlength{\sbht}{1.4\sbht}%
        \setlength{\sbhadj}{#1\sblw}\setlength{\sbhadj}{0.25\sbhadj}%
        \filldraw[color=green!60!black] (\sblen, 0) -- (\sblen - \sbsz, \sbht) -- (\sblen - 2\sbsz, 0) -- (\sblen, 0);
        \draw[line width=\sblw, color=green!60!black, #2] (2\sbsz - \sblw, \sbhadj) -- (0.5\sblen - 0.5\sbtw, \sbhadj);
        \draw[line width=\sblw, color=green!60!black, #2] (0.5\sblen + 0.5\sbtw, \sbhadj) -- (\sblen - 2\sbsz + \sblw, \sbhadj);
        \filldraw[color=green!60!black] (0, 0) -- (\sbsz, \sbht) -- (0 + 2\sbsz, 0) -- (0, 0);
        \node[color=green!60!black] at (0.5\sblen, 0) {\large  \textsf{\textup{#3}}};
    \end{tikzpicture}}}

% \advicermline is a helper command that makes an orange horizontal line, up or down arrows, and some text:
% arg 1 = 1 or -1 for up or down arrows
% arg 2 = solid or dashed or loosely dashed, etc.
% arg 3 = text
% arg 4 = text width
\newcommand{\advicermline}[4]{%
    \needspace{0.1\baselineskip}%
    \vbox to 1\baselineskip {\begin{tikzpicture}%
        \setlength{\sbtw}{#4}%
        \setlength{\sblen}{\linewidth}%
        \setlength{\sbht}{#1\sbsz}\setlength{\sbht}{1.4\sbht}%
        \setlength{\sbhadj}{#1\sblw}\setlength{\sbhadj}{0.25\sbhadj}%
        \filldraw[color=orange!60!black] (\sblen, 0) -- (\sblen - \sbsz, \sbht) -- (\sblen - 2\sbsz, 0) -- (\sblen, 0);
        \draw[line width=\sblw, color=orange!60!black, #2] (2\sbsz - \sblw, \sbhadj) -- (0.5\sblen - 0.5\sbtw, \sbhadj);
        \draw[line width=\sblw, color=orange!60!black, #2] (0.5\sblen + 0.5\sbtw, \sbhadj) -- (\sblen - 2\sbsz + \sblw, \sbhadj);
        \filldraw[color=orange!60!black] (0, 0) -- (\sbsz, \sbht) -- (0 + 2\sbsz, 0) -- (0, 0);
        \node[color=orange!60!black] at (0.5\sblen, 0) {\large  \textsf{\textup{#3}}};
    \end{tikzpicture}}}

% \ratline is a helper command that makes a purple horizontal line, up or down arrows, and some text:
% arg 1 = 1 or -1 for up or down arrows
% arg 2 = solid or dashed or loosely dashed, etc.
% arg 3 = text
% arg 4 = text width
\newcommand{\ratline}[4]{%
    \needspace{0.1\baselineskip}%
    \vbox to 1\baselineskip {\begin{tikzpicture}%
        \setlength{\sbtw}{#4}%
        \setlength{\sblen}{\linewidth}%
        \setlength{\sbht}{#1\sbsz}\setlength{\sbht}{1.4\sbht}%
        \setlength{\sbhadj}{#1\sblw}\setlength{\sbhadj}{0.25\sbhadj}%
        \filldraw[color=purple!40] (\sblen, 0) -- (\sblen - \sbsz, \sbht) -- (\sblen - 2\sbsz, 0) -- (\sblen, 0);
        \draw[line width=\sblw, color=purple!40, #2] (2\sbsz - \sblw, \sbhadj) -- (0.5\sblen - 0.5\sbtw, \sbhadj);
        \draw[line width=\sblw, color=purple!40, #2] (0.5\sblen + 0.5\sbtw, \sbhadj) -- (\sblen - 2\sbsz + \sblw, \sbhadj);
        \filldraw[color=purple!40] (0, 0) -- (\sbsz, \sbht) -- (0 + 2\sbsz, 0) -- (0, 0);
        \node[color=purple!90] at (0.5\sblen, 0) {\large  \textsf{\textup{#3}}};
    \end{tikzpicture}}}

% \linewitharrows is a helper command that makes a blue horizontal line, up or down arrows, and some text:
% arg 1 = 1 or -1 for up or down arrows
% arg 2 = solid or dashed or loosely dashed, etc.
% arg 3 = text
% arg 4 = text width
\newcommand{\linewitharrows}[4]{%
    \needspace{0.1\baselineskip}%
    \vbox to 1\baselineskip {\begin{tikzpicture}%
        \setlength{\sbtw}{#4}%
        \setlength{\sblen}{\linewidth}%
        \setlength{\sbht}{#1\sbsz}\setlength{\sbht}{1.4\sbht}%
        \setlength{\sbhadj}{#1\sblw}\setlength{\sbhadj}{0.25\sbhadj}%
        \filldraw[color=blue!40] (\sblen, 0) -- (\sblen - \sbsz, \sbht) -- (\sblen - 2\sbsz, 0) -- (\sblen, 0);
        \draw[line width=\sblw, color=blue!40, #2] (2\sbsz - \sblw, \sbhadj) -- (0.5\sblen - 0.5\sbtw, \sbhadj);
        \draw[line width=\sblw, color=blue!40, #2] (0.5\sblen + 0.5\sbtw, \sbhadj) -- (\sblen - 2\sbsz + \sblw, \sbhadj);
        \filldraw[color=blue!40] (0, 0) -- (\sbsz, \sbht) -- (0 + 2\sbsz, 0) -- (0, 0);
        \node[color=blue!90] at (0.5\sblen, 0) {\large  \textsf{\textup{#3}}};
    \end{tikzpicture}}}

\newcommand{\VSPb}{\vspace{0.5ex plus 5ex minus 0.25ex}}
\newcommand{\VSPa}{\vspace{0.25ex plus 5ex minus 0.25ex}}

% C
\newcommand{\cspecificstart}{\needspace{\sbns}\linewitharrows{-1}{solid}{C}{3em}}
\newcommand{\cspecificend}{\linewitharrows{1}{solid}{C}{3em}\VSPa}

% Fortran
\newcommand{\fortranspecificstart}{\VSPb\linewitharrows{-1}{solid}{Fortran}{6em}\VSPa}
\newcommand{\fortranspecificend}{\VSPb\linewitharrows{1}{solid}{Fortran}{6em}\VSPa}

% Python
\newcommand{\pyspecificstart}{\needspace{\sbns}\linewitharrows{-1}{solid}{Python}{6em}}
\newcommand{\pyspecificend}{\linewitharrows{1}{solid}{Python}{6em}\VSPa}

% Note
\newcommand{\notestart}{\VSPb\notelinewitharrows{-1}{solid}\VSPa}
\newcommand{\noteend}{\VSPb\notelinewitharrows{1}{solid}\VSPa}
% convenience macro for formatting the word "Note:" at the beginning of note blocks:
\newcommand{\noteheader}{{\textrm{\textsf{\textbf\textup\normalsize{{{{Note: }}}}}}}}

% Rationale
\newcommand{\rationalestart}{\VSPb\ratline{-1}{dashed}{Rationale}{7em}\VSPa}
\newcommand{\rationaleend}{\VSPb\ratline{1}{dashed}{}{0em}\VSPa}

% Advice to users
\newcommand{\adviceuserstart}{\VSPb\adviceuserline{-1}{solid}{Advice to users}{10em}\VSPa}
\newcommand{\adviceuserend}{\VSPb\adviceuserline{1}{solid}{}{0em}\VSPa}

% Advice to implementers
\newcommand{\adviceimplstart}{\VSPb\adviceimpline{-1}{solid}{Advice to PMIx library implementers}{20em}\VSPa}
\newcommand{\adviceimplend}{\VSPb\adviceimpline{1}{solid}{}{0em}\VSPa}

% Advice to hosts
\newcommand{\advicermstart}{\VSPb\advicermline{-1}{solid}{Advice to PMIx server hosts}{16em}\VSPa}
\newcommand{\advicermend}{\VSPb\advicermline{1}{solid}{}{0em}\VSPa}

% Required attributes
\newcommand{\reqattrstart}{\VSPb\adviceuserline{-1}{dashed}{Required Attributes}{16em}\VSPa}
\newcommand{\reqattrend}{\VSPb\adviceuserline{1}{dashed}{}{0em}\VSPa}

% Optional attributes
\newcommand{\optattrstart}{\VSPb\adviceimpline{-1}{dashed}{Optional Attributes}{16em}\VSPa}
\newcommand{\optattrend}{\VSPb\adviceimpline{1}{dashed}{}{0em}\VSPa}


%%%%%%%%%%%%%%%%%%%%%%%%%%%%%%%%%%%%%%%%%%%%%%%%%%%%%%%%%%%%%%%%%%%%%%%%%%%%%%%%%%%%%%%%%%%%%%
% Glossary formatting

\newcommand{\glossaryterm}[1]{\needspace{1ex}
\begin{adjustwidth}{-0.75in}{0.0in}
\nolinenumbers\parbox[b][-0.95\baselineskip][t]{1.4in}{\flushright \textbf{#1}}
\end{adjustwidth}\linenumbers}

\newcommand{\glossarydefstart}{
\begin{adjustwidth}{0.79in}{0.0in}}

\newcommand{\glossarydefend}{
\end{adjustwidth}\vspace{-1.5\baselineskip}}


%%%%%%%%%%%%%%%%%%%%%%%%%%%%%%%%%%%%%%%%%%%%%%%%%%%%%%%%%%%%%%%%%%%%%%%%%%%%%%%%%%%%%%%%%%%%%
% Indexing and Table of Contents

\usepackage{imakeidx}
\usepackage[nodotinlabels]{titletoc}   % required for its [nodotinlabels] option

% Clickable links in TOC and index:
\usepackage[hyperindex=true,linktocpage=true]{hyperref}
\hypersetup{
  bookmarksnumbered = true,
  bookmarksopen     = false,
  colorlinks  = true, % Colors links instead of red boxes
  urlcolor    = blue, % Color for external links
  linkcolor   = blue  % Color for internal links
}

% \url styled in Roman font.
\urlstyle{rm}

%%%%%%%%%%%%%%%%%%%%%%%%%%%%%%%%%%%%%%%%%%%%%%%%%%%%%%%%%%%%%%%%%%%%%%%%%%%%%%%%%%%%%%%%%%%%%
% Cross reference macros
% This defines:
%     \specref          cross reference label as "Section X on page Y"
%     \refsection       Link this label to a specific section label in the document
%
%     \declarstruct     Mark the declaration of a structure
%     \refstruct        Reference the structure declaration
%
%     \declareapi       Mark the declaration of an API function
%     \refapi           Reference the API declaration
%
%     \declaremacro     Mark the declaration of a user-level macro
%     \refmacro         Reference the macro declaration
%

\newcommand{\chapterref}[1]{Chapter~\ref{#1} on page~\pageref{#1}}
\newcommand{\specref}[1]{Section~\ref{#1} on page~\pageref{#1}}

\newcommand{\refsection}[2]{\hyperref[#1]{#2}}

\newcommand{\declarestruct}[1]{\index{#1!Definition|textbf} \label{struct:#1}}
\newcommand{\refstruct}[1]{\index{#1} \hyperref[struct:#1]{\code{#1} }}
\newcommand{\structref}[1] {\refstruct{#1}}
\newcommand{\specrefstruct}[1]{Section~\ref{struct:#1} on page~\pageref{struct:#1}}

\newcommand{\declareapi}[1]{\index{#1!Definition|textbf} \label{api:#1}}
\newcommand{\refapi}[1]{\index{#1} \hyperref[api:#1]{\code{#1} }}
\newcommand{\argapi}[1] {\refapi{#1}}

\newcommand{\refconst}[1]{\hyperref[const:#1]{\code{#1} }}

\newcommand{\declareattr}[1]{\index{#1!Definition|textbf} \label{attr:#1}}

\newcommand{\refarg}[1] {{\textrm{\textmd{\itshape{#1}}}}}
\newcommand{\argref}[1] {\refarg{#1}}

\newcommand{\declaremacro}[1]{\index{#1!Definition|textbf} \label{macro:#1}}
\newcommand{\refmacro}[1]{\index{#1} \hyperref[macro:#1]{\code{#1} }}

\newcommand{\declareterm}[1]{\index{#1!Definition|textbf} \label{macro:#1}}
\newcommand{\refterm}[1]{\index{#1} \hyperref[macro:#1]{\code{#1} }}

% Place in text for in-text questions during review
\newcommand{\rcomment}[1]{(REVIEW COMMENT: \textbf{#1})}

%%%%%%%%%%%%%%%%%%%%%%%%%%%%%%%%%%%%%%%%%%%%%%%%%%%%%%%%%%%%%%%%%%%%%%%%%%%%%%%%%%%%%%%%%%%%%
% Set default fonts:
\rmfamily\mdseries\upshape\normalsize

%%%%%%%%%%%%%%%%%%%%%%%%%%%%%%%%%%%%%%%%%%%%%%%%%
% Define a divider for splitting implementer vs host attribute requirements/options
\newcommand{\divider}{\noindent\makebox[\linewidth]{\rule{\linewidth}{0.8pt}}}


\newcounter{pycounter}
\newcommand{\pylabel}[1]{\refstepcounter{pycounter} \label{appB:#1}}
\newcommand{\refpy}[1]{\hyperref[appB:#1]{\code{#1} }}

\makeindex[intoc,columns=2]

%%%%%%%%%%%%%%%%%%%
\usepackage{acronym}
\acrodef{PMI}[PMI]{Process Management Interface}
\acrodef{PMIx}[PMIx]{Process Management Interface - Exascale}
\acrodef{HPC}[HPC]{High Performance Computing}
\acrodef{MPI}[MPI]{Message Passing Interface}
\acrodef{MPE}[MPE]{Message Passing Environment}

\acrodef{RM}[RM]{resource manager}
\acrodef{RTE}[RTE]{RunTime Environment}
\acrodef{SMS}[SMS]{system management software stack}
\acrodef{WLM}[WLM]{workload manager}
\acrodef{GDS}[GDS]{global data storage}
\acrodef{BCX}[BCX]{business card exchange}

\acrodef{PID}[PID]{process identifier}
\acrodef{URI}[URI]{uniform resource identifier}
\acrodef{CIDR}[CIDR]{Classless Inter-Domain Routing}
\acrodef{XML}[XML]{eXtensible Markup Language}

\acrodef{RAS}[RAS]{Reliability and Survivability}
\acrodef{API}[API]{Application Programming Interface}
\acrodef{PRRTE}[PRRTE]{PMIx-based Reference RunTime Environment}
\acrodef{PRI}[PRI]{PMIx Reference Implementation}
\acrodef{ECC}[ECC]{Error Check and Correction}
\acrodef{FM}[FM]{Fabric Manager}
\acrodef{IO}[IO]{Input/Output}
\acrodef{MPMD}{Multiple Program Multiple Data}
\acrodef{PU}{Processing Unit}
\acrodef{HWLOC}{Hardware Locality}
\acrodef{OS}{Operating System}
\acrodef{PGCID}{Process Group Context IDentifier}

%%%%%%%%%%%%%%%%%%%


\begin{document}
%
% Title page
%
    \pagenumbering{roman}
    %%%%%%%%%%%%%%%%%%%%%%%%%%%%%%%%%%%%%%%%%%%%%%%%%
% Title page
%%%%%%%%%%%%%%%%%%%%%%%%%%%%%%%%%%%%%%%%%%%%%%%%%

  \begin{titlepage}
    \begin{flushleft}
     \hspace{-6em} \includegraphics[width=0.4\textwidth]{figs/pmix-logo.png}
    \end{flushleft}

    \begin{adjustwidth}{-0.75in}{0in}
    \begin{center}
      \Huge
      \textsf{Process Management Interface\\for Exascale (PMIx) Standard}

      \vspace{1.0in}
	  \huge
      \textbf{Version \VER{}}

      \vspace{0.15in}
	  \Large
      \textbf{\VERDATE}

    \end{center}
    \end{adjustwidth}

    \vspace{1.2in}

\par
This document describes the Process Management Interface for Exascale (PMIx) Standard, version \VER{}.

\par
\textbf{Comments:}
Please provide comments on the PMIx Standard by filing issues on the document repository \url{https://github.com/pmix/pmix-standard/issues} or by sending them to the PMIx Community mailing list at \url{https://groups.google.com/forum/#!forum/pmix}.
Comments should include the version of the PMIx standard you are commenting about, and the page, section, and line numbers that you are referencing.
Please note that messages sent to the mailing list from an unsubscribed e-mail address will be ignored.

\vfill

\begin{adjustwidth}{0pt}{1em}\setlength{\parskip}{0.25\baselineskip}%
Copyright \textsuperscript{\textcopyright} 2018-2020 PMIx \acf{ASC}.\\
Permission to copy without fee all or part of this material is granted,
provided the PMIx \ac{ASC} copyright notice and
the title of this document appear, and notice is given that copying is by
permission of PMIx \ac{ASC}.
\end{adjustwidth}

  \end{titlepage}

%%%%%%%%%%%%%%%%%%%%%%%%%%%%%%%%%%%%%%%%%%%%%%%%%
% Blank page
%%%%%%%%%%%%%%%%%%%%%%%%%%%%%%%%%%%%%%%%%%%%%%%%%
\clearpage
\thispagestyle{empty}
\phantom{a}
\begin{center}
\emph{This page intentionally left blank}
\end{center}

\vfill



%
% Table of contents
%
    \setcounter{page}{0}
    \setcounter{tocdepth}{2}

    \begin{spacing}{1.3}
        \RedeclareSectionCommand[tocnumwidth=2.6em]{section}
        \RedeclareSectionCommand[tocnumwidth=3.7em,tocindent=4.1em]{subsection}
        \tableofcontents
    \end{spacing}

%
% Introductory materials
%
    % Uncomment the next line to enable line numbering on the main body text:
    \linenumbers\pagewiselinenumbers
    \newpage\pagenumbering{arabic}
    \setcounter{chapter}{0}  % start chapter numbering here

%
% Chapters
%
    % Introduction to PMIx
    %  - Overview, Goals, Arch.
    %%%%%%%%%%%%%%%%%%%%%%%%%%%%%%%%%%%%%%%%%%%%%%%%%
% Chapter: Introduction
%%%%%%%%%%%%%%%%%%%%%%%%%%%%%%%%%%%%%%%%%%%%%%%%%
\chapter{Introduction}
\label{chap:intro}

The \ac{PMI} has been used for quite some time as a means of exchanging wireup information needed for inter-process communication.
Two versions (PMI-1 and PMI-2) have been released as part of the MPICH effort, with PMI-2 demonstrating better scaling properties than its PMI-1 predecessor. However, two significant challenges face the \ac{HPC} community as it continues to move towards machines capable of exaflop and higher performance levels:

\begin{itemize}
\item the physical scale of the machines, and the corresponding number of total processes they support, is expected to reach levels approaching  1 million processes executing across 100 thousand nodes. Prior methods for initiating applications relied on exchanging communication endpoint information between the processes, either directly or in some form of hierarchical collective operation. Regardless of the specific mechanism employed, the exchange across such large applications would consume considerable time, with estimates running in excess of 5-10 minutes; and
\item whether it be hybrid applications that combine OpenMP threading operations with MPI, or application-steered workflow computations, the HPC community is experiencing an unprecedented wave of new approaches for computing at exascale levels. One common thread across the proposed methods is an increasing need for orchestration between the application and the \ac{SMS} comprising the scheduler (a.k.a. the \ac{WLM}), the \ac{RM}, global file system, fabric, and other subsystems. The lack of available support for application-to-SMS integration has forced researchers to develop "virtual" environments that hide the SMS behind a customized abstraction layer, but this results in considerable duplication of effort and a lack of portability.
\end{itemize}

\ac{PMIx} represents an attempt to resolve these questions by providing an extended version of the \ac{PMI} definitions specifically designed to support clusters up to exascale and larger sizes.
The overall objective of the project is not to branch the existing definitions -- in fact, PMIx fully supports both of the existing PMI-1 and PMI-2 APIs -- but rather to:

\begin{compactalphaenum}
\item augment those APIs to eliminate some current restrictions that impact scalability,
\item extend the breadth of the \ac{PMI} definitions to providing an abstraction layer for \ac{SMS} interactions,
\item establish a standards-like body for maintaining the definitions, and
\item provide a reference implementation of the PMIx standard that demonstrates the desired level of scalability and features.
\end{compactalphaenum}

Complete information about the \ac{PMIx} standard and affiliated projects can be found at the \ac{PMIx} web site: \url{https://pmix.org}


%%%%%%%%%%%%%%%%%%%%%%%%%%%%%%%%%%%%%%%%%%%%%%%%%
%%%%%%%%%%%%%%%%%%%%%%%%%%%%%%%%%%%%%%%%%%%%%%%%%
\section{Charter}
\label{chap:intro:charter}

The charter of the PMIx community is to:
\begin{itemize}
\item Define a set of agnostic APIs (not affiliated with any specific programming model or code base) to support interactions between application processes and the \ac{SMS}.
\item Develop an open source (non-copy-left licensed) standalone ``reference'' library to facilitate adoption of the \ac{PMIx} standard.
\item Retain transparent backward compatibility with the existing PMI-1 and PMI-2 definitions, any future \ac{PMI} releases, and across all \ac{PMIx} versions.
\item Support the ``Instant On'' initiative for rapid startup of applications at exascale and beyond.
\item Work with the \ac{HPC} community to define and implement new APIs that support evolving programming model requirements for application interactions with the \ac{SMS}.
\end{itemize}

Participation in the \ac{PMIx} community is open to anyone, and not restricted to only code contributors to the reference implementation.


%%%%%%%%%%%%%%%%%%%%%%%%%%%%%%%%%%%%%%%%%%%%%%%%%
%%%%%%%%%%%%%%%%%%%%%%%%%%%%%%%%%%%%%%%%%%%%%%%%%
\section{PMIx Standard Overview}
\label{chap:intro:std_overview}

\ldots

%%%%%%%%%%%
\subsection{Who should use the standard?}

\ldots

%%%%%%%%%%%
\subsection{What is defined in the standard?}

\ldots

%%%%%%%%%%%
\subsection{What is \emph{not} defined in the standard?}

The \ac{PMIx} Standard does not include anything, either stated or implied, regarding implementation.
It instead focuses exclusively on defining APIs and associated attribute key strings, and describing the expected behavior of those entities.
How that behavior is realized is entirely at the discretion of the implementer.

As previously noted, system environments and \ac{PMIx} library implementers are free to return ``not supported'' for any request. Thus, users should design their applications accordingly.


%%%%%%%%%%%%%%%%%%%%%%%%%%%%%%%%%%%%%%%%%%%%%%%%%
%%%%%%%%%%%%%%%%%%%%%%%%%%%%%%%%%%%%%%%%%%%%%%%%%
\section{PMIx Architecture Overview}
\label{chap:intro:arch_overview}

This section presents a brief overview the \ac{PMIx} Architecture~\cite{2017-Castain-EuroMPI}.

\ldots

%%%%%%%%%%%
\subsection{The PMIx Reference Implementation}

Note that the definition of the \ac{PMIx} Standard is not contingent upon use of the \ac{PMIx} Reference Implementation.
Any implementation that supports the defined APIs is a \ac{PMIx} Standard compliant implementation, and some environments have chosen to pursue their own custom implementation.
The \ac{PMIx} Reference Implementation is provided solely for the following purposes:
\begin{itemize}
\item Validation of the standard.\\
No proposed change and/or extension to the \ac{PMIx} standard is accepted without an accompanying prototype implementation in the \ac{PMIx} Reference Implementation.
This ensures that the proposal has undergone at least some minimal level of scrutiny and testing before being considered.
\item Ease of adoption.\\
The \ac{PMIx} Reference Implementation is designed to be particularly easy for resource managers (and the \ac{SMS} in general) to adopt, thus facilitating a rapid uptake into that community for application portability.
Both client and server \ac{PMIx} libraries are included, along with examples of client usage and server-side integration.
A list of supported environments and versions is provided on the \ac{PMIx} web site \url{www.pmix.org}
\end{itemize}

The \ac{PMIx} Reference Implementation targets support for the Linux operating system.
A reasonable effort is made to support all major, modern Linux distributions; however, validation is limited to the most recent 2-3 releases of RedHat Enterprise Linux (RHEL), Fedora, CentOS, and SUSE Linux Enterprise Server (SLES).
In addition, development support is maintained for Mac OSX.
Production support for vendor-specific operating systems is included as provided by the vendor.

%%%%%%%%%%%
\subsection{The PMIx Reference Server}

\ldots


%%%%%%%%%%%%%%%%%%%%%%%%%%%%%%%%%%%%%%%%%%%%%%%%%
\section{Organization of this document}

The remainder of this document is structured as follows:

\begin{itemize}
\item Introduction and Overview in \chapterref{chap:intro}
\item Terms and Conventions in \chapterref{chap:terms}
\item Data Structures and Types in \chapterref{chap:struct}
\item \ac{PMIx} Initialization and Finalization in \chapterref{chap:api_init}
\item Key/Value Management in \chapterref{chap:api_kv_mgmt}
\item Process Management in \chapterref{chap:api_proc_mgmt}
\item Job Management in \chapterref{chap:api_job_mgmt}
\item Event Notification in \chapterref{chap:api_event}
\item Data Packing and Unpacking in \chapterref{chap:api_data_mgmt}
\item \ac{PMIx} Server Specific Interfaces in \chapterref{chap:api_server}
\end{itemize}

%%%%%%%%%%%%%%%%%%%%%%%%%%%%%%%%%%%%%%%%%%%%%%%%%


    % PMIx Terms and Conventions
    %%%%%%%%%%%%%%%%%%%%%%%%%%%%%%%%%%%%%%%%%%%%%%%%%
% Chapter: Terms and Conventions
%%%%%%%%%%%%%%%%%%%%%%%%%%%%%%%%%%%%%%%%%%%%%%%%%
\chapter{PMIx Terms and Conventions}
\label{chap:terms}

Define ``attributes'' and how they are used, intent is to allow for definition of flexible APIs that can change behavior based on attributes instead of modifying function signature.
Include description of data types.

This document borrows freely from other standards (most notably from the \ac{MPI} and OpenMP standards) in its use of notation and conventions in an attempt to reduce confusion.

%%%%%%%%%%%
\section{Notional Conventions}

Some sections of this document describe programming language specific examples or APIs.
Text that applies only to programs for which the base language is C is show as follows:

\cspecificstart
C specific text...
\begin{codepar}
int foo = 42;
\end{codepar}
\cspecificend

Some text is for information only, and is not part of the normative specification.
These take three forms, described in their examples below:

\notestart
\noteheader
General text...
\noteend

\rationalestart
Throughout this document, the rationale for the design choices made in the interface specification is set off in this section.
Some readers may wish to skip these sections, while readers interested in interface design may want to read them carefully.
\rationaleend

\adviceuserstart
Throughout this document, material aimed at users and that illustrates usage is set off in this section.
Some readers may wish to skip these sections, while readers interested in programming in \ac{MPI} may want to read them carefully.
\adviceuserend

\adviceimplstart
Throughout this document, material that is primarily commentary to implementers is set off in this section.
Some readers may wish to skip these sections, while readers interested in \ac{PMIx} implementations may want to read them carefully. 
\adviceimplend

%%%%%%%%%%%
\section{Semantics}

The following terms will be taken to mean:

\begin{itemize}
\item \emph{shall} and \emph{will} indicate that the specified behavior is \emph{required} of all conforming implementations
\item \emph{should} and \emph{may} indicate behaviors that a quality implementation would include, but are not required of all conforming implementations
\end{itemize}

%%%%%%%%%%%
\section{Naming Conventions}

\ldots

%%%%%%%%%%%
\section{Procedure Conventions}

While current \ac{PMIx} Reference Implementation is solely based on the C programming language, it is not the intent of the \ac{PMIx} Standard to preclude the use of other languages.
Accordingly, the procedure specifications in the \ac{PMIx} Standard are written in a language-independent syntax with the arguments marked as IN, OUT, or INOUT.
The meanings of these are:
\begin{itemize}
\item IN:
The call may use the input value but does not update the argument from the perspective of the caller at any time during the call?s execution, 
\item OUT:
The call may update the argument but does not use its input value
\item INOUT:
The call may both use and update the argument. 
\end{itemize}

%%%%%%%%%%%%%%%%%%%%%%%%%%%%%%%%%%%%%%%%%%%%%%%%%


    % Data Structures, Types, Constants
    %  - Includes: Reserved attributes, Keys
    %%%%%%%%%%%%%%%%%%%%%%%%%%%%%%%%%%%%%%%%%%%%%%%%%
% Chapter: Data Structures
%%%%%%%%%%%%%%%%%%%%%%%%%%%%%%%%%%%%%%%%%%%%%%%%%
\chapter{Data Structures and Types}
\label{chap:struct}

This chapter defines \ac{PMIx} standard data structures (along with macros for convenient use), types, and constants.
These apply to all consumers of the \ac{PMIx} interface.
Where necessary for clarification, the description of, for example, an attribute may be copied from this chapter into a section where it is used.

A PMIx implementation may define additional attributes beyond those specified in this document.

\adviceimplstart
Structures, types, and macros in the \ac{PMIx} Standard are defined in terms of the C-programming language. Implementers wishing to support other languages should provide the equivalent definitions in a language-appropriate manner.

If a PMIx implementation chooses to define additional attributes they should avoid using the \code{"PMIX"} prefix in their name or starting the attribute string with a \code{"pmix"} prefix.
This helps the end user distinguish between what is defined by the PMIx standard and what is specific to that PMIx implementation, and avoids potential conflicts with attributes defined by the Standard.
\adviceimplend

\adviceuserstart
Use of increment/decrement operations on indices inside \ac{PMIx} macros is discouraged due to unpredictable behavior. For example, the following sequence:

\begin{codepar}
PMIX_INFO_LOAD(&array[n++], "mykey", &mystring, PMIX_STRING);
PMIX_INFO_LOAD(&array[n++], "mykey2", &myint, PMIX_INT);
\end{codepar}

will load the given key-values into incorrect locations if the macro is implemented as:

\begin{codepar}
define PMIX_INFO_LOAD(m, k, v, t)                      \textbackslash
  do \{                                                 \textbackslash
    if (NULL != (k)) \{                                 \textbackslash
      pmix_strncpy((m)->key, (k), PMIX_MAX_KEYLEN);    \textbackslash
    \}                                                  \textbackslash
    (m)->flags = 0;                                    \textbackslash
    pmix_value_load(&((m)->value), (v), (t));          \textbackslash
  \} while (0)
\end{codepar}

since the index is cited more than once in the macro. The \ac{PMIx} standard only governs the existence and syntax of macros - it does not specify their implementation. Given the freedom of implementation, a safer call sequence might be as follows:

\begin{codepar}
PMIX_INFO_LOAD(&array[n], "mykey", &mystring, PMIX_STRING);
++n;
PMIX_INFO_LOAD(&array[n], "mykey2", &myint, PMIX_INT);
++n;
\end{codepar}

Users are also advised to use the macros for creating, loading, and releasing
\ac{PMIx} structures to avoid potential issues with release of memory. For
example, pointing a \refstruct{pmix_envar_t} element at a static string
variable and then using \refmacro{PMIX_ENVAR_DESTRUCT} to clear it would
generate an error as the static string had not been allocated.

\adviceuserend

%%%%%%%%%%%%%%%%%%%%%%%%%%%%%%%%%%%%%%%%%%%%%%%%%
%%%%%%%%%%%%%%%%%%%%%%%%%%%%%%%%%%%%%%%%%%%%%%%%%
\section{Constants}
\label{chap:struct:const}

\ac{PMIx} defines a few values that are used throughout the standard to set the size of fixed arrays or as a means of identifying values with special meaning.
The community makes every attempt to minimize the number of such definitions.
The constants defined in this section may be used before calling any \ac{PMIx} library initialization routine.
Additional constants associated with specific data structures or types are defined in the section describing that data structure or type.

\begin{constantdesc}
%
\declareconstitem{PMIX_MAX_NSLEN}
Maximum namespace string length as an integer.
\end{constantdesc}

\adviceimplstart
\refconst{PMIX_MAX_NSLEN} should have a minimum value of 63 characters. Namespace arrays in \ac{PMIx} defined structures must reserve
a space of size \refconst{PMIX_MAX_NSLEN}+1 to allow room for the \code{NULL} terminator
\adviceimplend

\begin{constantdesc}
%
\declareconstitem{PMIX_MAX_KEYLEN}
Maximum key string length as an integer.
\end{constantdesc}

\adviceimplstart
\refconst{PMIX_MAX_KEYLEN} should have a minimum value of 63 characters. Key arrays in \ac{PMIx} defined structures must reserve
a space of size \refconst{PMIX_MAX_KEYLEN}+1 to allow room for the \code{NULL} terminator
\adviceimplend

\begin{constantdesc}
%
\declareconstitemNEW{PMIX_APP_WILDCARD}
A value to indicate that the user wants the data for the given key from every application that posted that key, or that the given value applies to all applications within the given namespace.
\end{constantdesc}


%%%%%%%%%%%%%%%%%%%%%%%%%%%%%%%%%%%%%%%%%%%%%%%%%
\subsection{PMIx Return Status Constants}
\label{api:struct:errors}
\declarestruct{pmix_status_t}

The \refstruct{pmix_status_t} structure is an \code{int} type for return status. The tables shown in this section define the possible values for \refstruct{pmix_status_t}.
PMIx errors are required to always be negative, with \code{0} reserved for \refconst{PMIX_SUCCESS}. Values in the list that were deprecated in later standards are denoted as such. Values added to the list in this version of the standard are shown in \textbf{\color{magenta}magenta}.

\adviceimplstart
A PMIx implementation must define all of the constants defined in this section, even if they will never return the specific value to the caller.
\adviceimplend

\adviceuserstart
Other than \refconst{PMIX_SUCCESS} (which is required to be zero), the actual value of any \ac{PMIx} error constant is left to the \ac{PMIx} library implementer. Thus, users are advised to always refer to constant by name, and not a specific implementation's value, for portability between implementations and compatibility across library versions.
\adviceuserend

The following values are general constants used in a variety of places.

\begin{constantdesc}
%
\declareconstitem{PMIX_SUCCESS}
Success.
%
\declareconstitem{PMIX_ERROR}
General Error.
%
\declareconstitemNEW{PMIX_ERR_EXISTS}
Requested operation would overwrite an existing value - typically returned
when an operation would overwrite an existing file or directory.
%
\declareconstitemNEW{PMIX_ERR_EXISTS_OUTSIDE_SCOPE}
The requested key exists, but was posted in a \emph{scope} (see Section \ref{api:nres:scope}) that does not include the requester
%
\declareconstitem{PMIX_ERR_INVALID_CRED}
Invalid security credentials.
%
\declareconstitem{PMIX_ERR_WOULD_BLOCK}
Operation would block.
%
\declareconstitem{PMIX_ERR_UNKNOWN_DATA_TYPE}
The data type specified in an input to the \ac{PMIx} library is not recognized
by the implementation.
%
\declareconstitem{PMIX_ERR_TYPE_MISMATCH}
The data type found in an object does not match the expected data type
as specified in the \ac{API} call - e.g., a request to unpack a
\refconst{PMIX_BOOL} value from a buffer that does not contain a value of
that type in the current unpack location.
%
\declareconstitem{PMIX_ERR_UNPACK_INADEQUATE_SPACE}
Inadequate space to unpack data - the number of values in the buffer exceeds
the specified number to unpack.
%
\declareconstitem{PMIX_ERR_UNPACK_READ_PAST_END_OF_BUFFER}
Unpacking past the end of the provided buffer - the number of values in the
buffer is less than the specified number to unpack, or a request was made to
unpack a buffer beyond the buffer's end.
%
\declareconstitem{PMIX_ERR_UNPACK_FAILURE}
The unpack operation failed for an unspecified reason.
%
\declareconstitem{PMIX_ERR_PACK_FAILURE}
The pack operation failed for an unspecified reason.
%
\declareconstitem{PMIX_ERR_NO_PERMISSIONS}
The user lacks permissions to execute the specified operation.
%
\declareconstitem{PMIX_ERR_TIMEOUT}
Either a user-specified or system-internal timeout expired.
%
\declareconstitem{PMIX_ERR_UNREACH}
The specified target server or client process is not reachable - i.e., a
suitable connection either has not been or can not be made.
%
\declareconstitem{PMIX_ERR_BAD_PARAM}
One or more incorrect parameters (e.g., passing an attribute with a value of the wrong type), or multiple parameters containing conflicting directives (e.g., multiple instances of the same attribute with different values, or different attributes specifying conflicting behaviors), were passed to a \ac{PMIx} \ac{API}.
%
\declareconstitemNEW{PMIX_ERR_EMPTY}
An array or list was given that has no members in it - i.e., the object is empty.
%
\declareconstitem{PMIX_ERR_RESOURCE_BUSY}
Resource busy - typically seen when an attempt to establish a connection
to another process (e.g., a \ac{PMIx} server) cannot be made due to a
communication failure.
%
\declareconstitem{PMIX_ERR_OUT_OF_RESOURCE}
Resource exhausted.
%
\declareconstitem{PMIX_ERR_INIT}
Error during initialization.
%
\declareconstitem{PMIX_ERR_NOMEM}
Out of memory.
%
\declareconstitem{PMIX_ERR_NOT_FOUND}
The requested information was not found.
%
\declareconstitem{PMIX_ERR_NOT_SUPPORTED}
The requested operation is not supported by either the \ac{PMIx} implementation
or the host environment.
%
\declareconstitemNEW{PMIX_ERR_PARAM_VALUE_NOT_SUPPORTED}
The requested operation is supported by the \ac{PMIx} implementation and (if applicable) the host environment. However, at least one supplied parameter was given an unsupported value, and the operation cannot therefore be executed as requested.
%
\declareconstitem{PMIX_ERR_COMM_FAILURE}
Communication failure - a message failed to be sent or received, but the
connection remains intact.
%
\declareconstitemNEW{PMIX_ERR_LOST_CONNECTION}
Lost connection between server and client or tool.
%
\declareconstitem{PMIX_ERR_INVALID_OPERATION}
The requested operation is supported by the implementation and host environment, but fails to meet a requirement (e.g., requesting to \textit{disconnect} from processes without first \textit{connecting} to them, inclusion of conflicting directives, or a request to perform an operation that conflicts with an ongoing one).
%
\declareconstitem{PMIX_OPERATION_IN_PROGRESS}
A requested operation is already in progress - the duplicate request
shall therefore be ignored.
%
\declareconstitem{PMIX_OPERATION_SUCCEEDED}
The requested operation was performed atomically - no callback function will be executed.
%
\declareconstitemNEW{PMIX_ERR_PARTIAL_SUCCESS}
The operation is considered successful but not all elements of the operation were concluded (e.g., some members of a group construct operation chose not to participate).
%
\end{constantdesc}


%%%%%%%%%%%%%%%%%%%%%%%%%%%%%%%%%%%%%%%%%%%%%%%%%
\subsubsection{User-Defined Error and Event Constants}
\label{api:struct:usererrors}

\ac{PMIx} establishes a boundary for constants defined in the \ac{PMIx} standard. Negative values larger (i.e., more negative) than this (and any positive values greater than zero) are guaranteed not to conflict with \ac{PMIx} values.

\begin{constantdesc}
%
\declareconstitem{PMIX_EXTERNAL_ERR_BASE}
A starting point for user-level defined error and event constants.
Negative values that are more negative than the defined constant are guaranteed not to conflict with \ac{PMIx} values.
Definitions should always be based on the \refconst{PMIX_EXTERNAL_ERR_BASE} constant and not a specific value as the value of the constant may change.
%
\end{constantdesc}



%%%%%%%%%%%%%%%%%%%%%%%%%%%%%%%%%%%%%%%%%%%%%%%%%
%%%%%%%%%%%%%%%%%%%%%%%%%%%%%%%%%%%%%%%%%%%%%%%%%
\section{Data Types}

This section defines various data types used by the \ac{PMIx} APIs. The version of the standard in which a particular data type was introduced is shown in the margin.

%%%%%%%%%%%%%%%%%%%%%%%%%%%%%%%%%%%%%%%%%%%%%%%%%
\subsection{Key Structure}
\declarestruct{pmix_key_t}

The \refstruct{pmix_key_t} structure is a statically defined character array of length \refconst{PMIX_MAX_KEYLEN}+1, thus supporting keys of maximum length \refconst{PMIX_MAX_KEYLEN} while preserving space for a mandatory \code{NULL} terminator.

\versionMarker{2.0}
\cspecificstart
\begin{codepar}
typedef char pmix_key_t[PMIX_MAX_KEYLEN+1];
\end{codepar}
\cspecificend

Characters in the key must be standard alphanumeric values supported by common utilities such as \textit{strcmp}.

\adviceuserstart
References to keys in \ac{PMIx} v1 were defined simply as an array of characters of size \code{PMIX_MAX_KEYLEN+1}. The \refstruct{pmix_key_t} type definition was introduced in version 2 of the standard. The two definitions are code-compatible and thus do not represent a break in backward compatibility.

Passing a \refstruct{pmix_key_t} value to the standard \textit{sizeof} utility can result in compiler warnings of incorrect returned value. Users are advised to avoid using \textit{sizeof(pmix_key_t)} and instead rely on the \refconst{PMIX_MAX_KEYLEN} constant.
\adviceuserend

%%%%%%%%%%%%%%%%%%%%%%%%%%%%%%%%%%%%%%%%%%%%%%%%%
\subsubsection{Key support macros}

The following macros are provided for convenience when working with \ac{PMIx} keys.

\littleheader{Check key macro}
\declaremacro{PMIX_CHECK_KEY}

Compare the key in a \refstruct{pmix_info_t} to a given value.

\versionMarker{3.0}
\cspecificstart
\begin{codepar}
PMIX_CHECK_KEY(a, b)
\end{codepar}
\cspecificend

\begin{arglist}
\argin{a}{Pointer to the structure whose key is to be checked (pointer to \refstruct{pmix_info_t})}
\argin{b}{String value to be compared against (\code{char*})}
\end{arglist}

Returns \code{true} if the key matches the given value

\littleheader{Check reserved key macro}
\declaremacro{PMIX_CHECK_RESERVED_KEY}

Check if the given key is a \ac{PMIx} \emph{reserved} key as described in Chapter \ref{chap:api_rsvd_keys}.

\versionMarker{4.0}
\cspecificstart
\begin{codepar}
PMIX_CHECK_RESERVED_KEY(a)
\end{codepar}
\cspecificend

\begin{arglist}
\argin{a}{String value to be checked (\code{char*})}
\end{arglist}

Returns \code{true} if the key is reserved by the Standard.

\littleheader{Load key macro}
\declaremacro{PMIX_LOAD_KEY}

Load a key into a \refstruct{pmix_info_t}.

\versionMarker{4.0}
\cspecificstart
\begin{codepar}
PMIX_LOAD_KEY(a, b)
\end{codepar}
\cspecificend

\begin{arglist}
\argin{a}{Pointer to the structure whose key is to be loaded (pointer to \refstruct{pmix_info_t})}
\argin{b}{String value to be loaded (\code{char*})}
\end{arglist}

No return value.

%%%%%%%%%%%%%%%%%%%%%%%%%%%%%%%%%%%%%%%%%%%%%%%%%
\subsection{Namespace Structure}
\declarestruct{pmix_nspace_t}

The \refstruct{pmix_nspace_t} structure is a statically defined character array of length \refconst{PMIX_MAX_NSLEN}+1, thus supporting namespaces of maximum length \refconst{PMIX_MAX_NSLEN} while preserving space for a mandatory \code{NULL} terminator.

\versionMarker{2.0}
\cspecificstart
\begin{codepar}
typedef char pmix_nspace_t[PMIX_MAX_NSLEN+1];
\end{codepar}
\cspecificend

Characters in the namespace must be standard alphanumeric values supported by common utilities such as \textit{strcmp}.

\adviceuserstart
References to namespace values in \ac{PMIx} v1 were defined simply as an array of characters of size \code{PMIX_MAX_NSLEN+1}. The \refstruct{pmix_nspace_t} type definition was introduced in version 2 of the standard. The two definitions are code-compatible and thus do not represent a break in backward compatibility.

Passing a \refstruct{pmix_nspace_t} value to the standard \textit{sizeof} utility can result in compiler warnings of incorrect returned value. Users are advised to avoid using \textit{sizeof(pmix_nspace_t)} and instead rely on the \refconst{PMIX_MAX_NSLEN} constant.
\adviceuserend

%%%%%%%%%%%%%%%%%%%%%%%%%%%%%%%%%%%%%%%%%%%%%%%%%
\subsubsection{Namespace support macros}

The following macros are provided for convenience when working with \ac{PMIx} namespace structures.

\littleheader{Check namespace macro}
\declaremacro{PMIX_CHECK_NSPACE}

Compare the string in a \refstruct{pmix_nspace_t} to a given value.

\versionMarker{3.0}
\cspecificstart
\begin{codepar}
PMIX_CHECK_NSPACE(a, b)
\end{codepar}
\cspecificend

\begin{arglist}
\argin{a}{Pointer to the structure whose value is to be checked (pointer to \refstruct{pmix_nspace_t})}
\argin{b}{String value to be compared against (\code{char*})}
\end{arglist}

Returns \code{true} if the namespace matches the given value

\littleheader{Check invalid namespace macro}
\declaremacro{PMIX_NSPACE_INVALID}

Check the string in a \refstruct{pmix_nspace_t}

\versionMarker{4.1}
\cspecificstart
\begin{codepar}
PMIX_NSPACE_INVALID(a)
\end{codepar}
\cspecificend

\begin{arglist}
\argin{a}{Pointer to the structure whose value is to be checked (pointer to \refstruct{pmix_nspace_t})}
\end{arglist}

Returns \code{true} if the namespace is invalid (i.e., starts with a \code{NULL} resulting in a zero-length string value)

\littleheader{Load namespace macro}
\declaremacro{PMIX_LOAD_NSPACE}

Load a namespace into a \refstruct{pmix_nspace_t}.

\versionMarker{4.0}
\cspecificstart
\begin{codepar}
PMIX_LOAD_NSPACE(a, b)
\end{codepar}
\cspecificend

\begin{arglist}
\argin{a}{Pointer to the target structure (pointer to \refstruct{pmix_nspace_t})}
\argin{b}{String value to be loaded - if \code{NULL} is given, then the target structure will be initialized to zero's (\code{char*})}
\end{arglist}

No return value.


%%%%%%%%%%%%%%%%%%%%%%%%%%%%%%%%%%%%%%%%%%%%%%%%%
\subsection{Rank Structure}
\declarestruct{pmix_rank_t}

The \refstruct{pmix_rank_t} structure is a \code{uint32_t} type for rank values.

\versionMarker{1.0}
\cspecificstart
\begin{codepar}
typedef uint32_t pmix_rank_t;
\end{codepar}
\cspecificend

The following constants can be used to set a variable of the type \refstruct{pmix_rank_t}. All definitions were introduced in version 1 of the standard unless otherwise marked. Valid rank values start at zero.

\begin{constantdesc}
%
\declareconstitem{PMIX_RANK_UNDEF}
A value to request job-level data where the information itself is not associated with any specific rank, or when passing a \refstruct{pmix_proc_t} identifier to an operation that only references the namespace field of that structure.
%
\declareconstitem{PMIX_RANK_WILDCARD}
A value to indicate that the user wants the data for the given key from every rank that posted that key.
%
\declareconstitem{PMIX_RANK_LOCAL_NODE}
Special rank value used to define groups of ranks.
This constant defines the group of all ranks on a local node.
%
\declareconstitem{PMIX_RANK_LOCAL_PEERS}
Special rank value used to define groups of ranks.
This constant defines the group of all ranks on a local node within the same namespace as the current process.
%
\declareconstitem{PMIX_RANK_INVALID}
An invalid rank value.
%
\declareconstitem{PMIX_RANK_VALID}
Define an upper boundary for valid rank values.
%
\end{constantdesc}


%%%%%%%%%%%%%%%%%%%%%%%%%%%%%%%%%%%%%%%%%%%%%%%%%
\subsubsection{Rank support macros}

The following macros are provided for convenience when working with \ac{PMIx} ranks.

\littleheader{Check rank macro}
\declaremacro{PMIX_CHECK_RANK}

Check two ranks for equality, taking into account wildcard values

\versionMarker{4.0}
\cspecificstart
\begin{codepar}
PMIX_CHECK_RANK(a, b)
\end{codepar}
\cspecificend

\begin{arglist}
\argin{a}{Rank to be checked (\refstruct{pmix_rank_t})}
\argin{b}{Rank to be checked (\refstruct{pmix_rank_t})}
\end{arglist}

Returns \code{true} if the ranks are equal, or at least one of the ranks is \refconst{PMIX_RANK_WILDCARD}

\littleheader{Check rank is valid macro}
\declaremacro{PMIX_RANK_IS_VALID}

Check is the given rank is a valid value

\versionMarker{4.1}
\cspecificstart
\begin{codepar}
PMIX_RANK_IS_VALID(a)
\end{codepar}
\cspecificend

\begin{arglist}
\argin{a}{Rank to be checked (\refstruct{pmix_rank_t})}
\end{arglist}

Returns \code{true} if the given rank is valid (i.e., less than \refconst{PMIX_RANK_VALID})

%%%%%%%%%%%%%%%%%%%%%%%%%%%%%%%%%%%%%%%%%%%%%%%%%
\subsection{Process Structure}
\declarestruct{pmix_proc_t}

The \refstruct{pmix_proc_t} structure is used to identify a single process in the PMIx universe.
It contains a reference to the namespace and the \refstruct{pmix_rank_t} within that namespace.

\versionMarker{1.0}
\cspecificstart
\begin{codepar}
typedef struct pmix_proc \{
    pmix_nspace_t nspace;
    pmix_rank_t rank;
\} pmix_proc_t;
\end{codepar}
\cspecificend

%%%%%%%%%%%%%%%%%%%%%%%%%%%%%%%%%%%%%%%%%%%%%%%%%
\subsubsection{Process structure support macros}
The following macros are provided to support the \refstruct{pmix_proc_t} structure.

\littleheader{Initialize the proc structure}
\declaremacro{PMIX_PROC_CONSTRUCT}

Initialize the \refstruct{pmix_proc_t} fields.

\versionMarker{1.0}
\cspecificstart
\begin{codepar}
PMIX_PROC_CONSTRUCT(m)
\end{codepar}
\cspecificend

\begin{arglist}
\argin{m}{Pointer to the structure to be initialized (pointer to \refstruct{pmix_proc_t})}
\end{arglist}

\littleheader{Destruct the proc structure}
\declaremacro{PMIX_PROC_DESTRUCT}

Destruct the \refstruct{pmix_proc_t} fields.

\cspecificstart
\begin{codepar}
PMIX_PROC_DESTRUCT(m)
\end{codepar}
\cspecificend

\begin{arglist}
\argin{m}{Pointer to the structure to be destructed (pointer to \refstruct{pmix_proc_t})}
\end{arglist}

There is nothing to release here as the fields in \refstruct{pmix_proc_t} are either a statically-declared array (the namespace) or a single value (the rank). However, the macro is provided for symmetry in the code and for future-proofing should some allocated field be included some day.

\littleheader{Create a proc array}
\declaremacro{PMIX_PROC_CREATE}

Allocate and initialize an array of \refstruct{pmix_proc_t} structures.

\versionMarker{1.0}
\cspecificstart
\begin{codepar}
PMIX_PROC_CREATE(m, n)
\end{codepar}
\cspecificend

\begin{arglist}
\arginout{m}{Address where the pointer to the array of \refstruct{pmix_proc_t} structures shall be stored (handle)}
\argin{n}{Number of structures to be allocated (\code{size_t})}
\end{arglist}


\littleheader{Free a proc structure}
\declaremacro{PMIX_PROC_RELEASE}

Release a \refstruct{pmix_proc_t} structure.

\versionMarker{4.0}
\cspecificstart
\begin{codepar}
PMIX_PROC_RELEASE(m)
\end{codepar}
\cspecificend

\begin{arglist}
\argin{m}{Pointer to a \refstruct{pmix_proc_t} structure (handle)}
\end{arglist}

\littleheader{Free a proc array}
\declaremacro{PMIX_PROC_FREE}

Release an array of \refstruct{pmix_proc_t} structures.

\versionMarker{1.0}
\cspecificstart
\begin{codepar}
PMIX_PROC_FREE(m, n)
\end{codepar}
\cspecificend

\begin{arglist}
\argin{m}{Pointer to the array of \refstruct{pmix_proc_t} structures (handle)}
\argin{n}{Number of structures in the array (\code{size_t})}
\end{arglist}

\littleheader{Load a proc structure}
\declaremacro{PMIX_PROC_LOAD}

Load values into a \refstruct{pmix_proc_t}.

\versionMarker{2.0}
\cspecificstart
\begin{codepar}
PMIX_PROC_LOAD(m, n, r)
\end{codepar}
\cspecificend

\begin{arglist}
\argin{m}{Pointer to the structure to be loaded (pointer to \refstruct{pmix_proc_t})}
\argin{n}{Namespace to be loaded (\refstruct{pmix_nspace_t})}
\argin{r}{Rank to be assigned (\refstruct{pmix_rank_t})}
\end{arglist}

No return value. Deprecated in favor of \refmacro{PMIX_LOAD_PROCID}

\littleheader{Compare identifiers}
\declaremacro{PMIX_CHECK_PROCID}

Compare two \refstruct{pmix_proc_t} identifiers.

\versionMarker{3.0}
\cspecificstart
\begin{codepar}
PMIX_CHECK_PROCID(a, b)
\end{codepar}
\cspecificend

\begin{arglist}
\argin{a}{Pointer to a structure whose ID is to be compared (pointer to \refstruct{pmix_proc_t})}
\argin{b}{Pointer to a structure whose ID is to be compared (pointer to \refstruct{pmix_proc_t})}
\end{arglist}

Returns \code{true} if the two structures contain matching namespaces and:

\begin{itemize}
    \item the ranks are the same value
    \item one of the ranks is \refconst{PMIX_RANK_WILDCARD}
\end{itemize}

\littleheader{Check if a process identifier is valid}
\declaremacro{PMIX_PROCID_INVALID}

Check for invalid namespace or rank value

\versionMarker{4.1}
\cspecificstart
\begin{codepar}
PMIX_PROCID_INVALID(a)
\end{codepar}
\cspecificend

\begin{arglist}
\argin{a}{Pointer to a structure whose ID is to be checked (pointer to \refstruct{pmix_proc_t})}
\end{arglist}

Returns \code{true} if the process identifier contains either an empty (i.e., invalid) \refarg{nspace} field or a \refarg{rank} field of \refconst{PMIX_RANK_INVALID}

\littleheader{Load a procID structure}
\declaremacro{PMIX_LOAD_PROCID}

Load values into a \refstruct{pmix_proc_t}.

\versionMarker{4.0}
\cspecificstart
\begin{codepar}
PMIX_LOAD_PROCID(m, n, r)
\end{codepar}
\cspecificend

\begin{arglist}
\argin{m}{Pointer to the structure to be loaded (pointer to \refstruct{pmix_proc_t})}
\argin{n}{Namespace to be loaded (\refstruct{pmix_nspace_t})}
\argin{r}{Rank to be assigned (\refstruct{pmix_rank_t})}
\end{arglist}

\littleheader{Transfer a procID structure}
\declaremacro{PMIX_XFER_PROCID}

Transfer contents of one \refstruct{pmix_proc_t} value to another \refstruct{pmix_proc_t}.

\versionMarker{4.1}
\cspecificstart
\begin{codepar}
PMIX_XFER_PROCID(m, n)
\end{codepar}
\cspecificend

\begin{arglist}
\argin{m}{Pointer to the target structure (pointer to \refstruct{pmix_proc_t})}
\argin{n}{Pointer to the source structure (pointer to \refstruct{pmix_proc_t})}
\end{arglist}

\littleheader{Construct a multi-cluster namespace}
\declaremacro{PMIX_MULTICLUSTER_NSPACE_CONSTRUCT}

Construct a multi-cluster identifier containing a cluster ID and a namespace.

\versionMarker{4.0}
\cspecificstart
\begin{codepar}
PMIX_MULTICLUSTER_NSPACE_CONSTRUCT(m, n, r)
\end{codepar}
\cspecificend

\begin{arglist}
\argin{m}{\refstruct{pmix_nspace_t} structure that will contain the multi-cluster identifier (\refstruct{pmix_nspace_t})}
\argin{n}{Cluster identifier (\code{char*})}
\argin{n}{Namespace to be loaded (\refstruct{pmix_nspace_t})}
\end{arglist}

Combined length of the cluster identifier and namespace must be less than \refconst{PMIX_MAX_NSLEN}-2.

\littleheader{Parse a multi-cluster namespace}
\declaremacro{PMIX_MULTICLUSTER_NSPACE_PARSE}

Parse a multi-cluster identifier into its cluster ID and namespace parts.

\versionMarker{4.0}
\cspecificstart
\begin{codepar}
PMIX_MULTICLUSTER_NSPACE_PARSE(m, n, r)
\end{codepar}
\cspecificend

\begin{arglist}
\argin{m}{\refstruct{pmix_nspace_t} structure containing the multi-cluster identifier (pointer to \refstruct{pmix_nspace_t})}
\argin{n}{Location where the cluster ID is to be stored (\refstruct{pmix_nspace_t})}
\argin{n}{Location where the namespace is to be stored (\refstruct{pmix_nspace_t})}
\end{arglist}


%%%%%%%%%%%%%%%%%%%%%%%%%%%%%%%%%%%%%%%%%%%%%%%%%
\subsection{Process State Structure}
\label{api:struct:processstate}
\declarestruct{pmix_proc_state_t}

\versionMarker{2.0}
The \refstruct{pmix_proc_state_t} structure is a \code{uint8_t} type for process state values. The following constants can be used to set a variable of the type \refstruct{pmix_proc_state_t}.

\adviceuserstart
The fine-grained nature of the following constants may exceed the ability of an \ac{RM} to provide updated process state values during the process lifetime. This is particularly true of states for short-lived processes.
\adviceuserend

\begin{constantdesc}
%
\declareconstitem{PMIX_PROC_STATE_UNDEF}
Undefined process state.
%
\declareconstitem{PMIX_PROC_STATE_PREPPED}
Process is ready to be launched.
%
\declareconstitem{PMIX_PROC_STATE_LAUNCH_UNDERWAY}
Process launch is underway.
%
\declareconstitem{PMIX_PROC_STATE_RESTART}
Process is ready for restart.
%
\declareconstitem{PMIX_PROC_STATE_TERMINATE}
Process is marked for termination.
%
\declareconstitem{PMIX_PROC_STATE_RUNNING}
Process has been locally \code{fork}'ed by the \ac{RM}.
%
\declareconstitem{PMIX_PROC_STATE_CONNECTED}
Process has connected to PMIx server.
%
\declareconstitem{PMIX_PROC_STATE_UNTERMINATED}
Define a ``boundary'' between the terminated states and \refconst{PMIX_PROC_STATE_CONNECTED} so users can easily and quickly determine if a process is still running or not.
Any value less than this constant means that the process has not terminated.
%
\declareconstitem{PMIX_PROC_STATE_TERMINATED}
Process has terminated and is no longer running.
%
\declareconstitem{PMIX_PROC_STATE_ERROR}
Define a boundary so users can easily and quickly determine if a process abnormally terminated.
Any value above this constant means that the process has terminated abnormally.
%
\declareconstitem{PMIX_PROC_STATE_KILLED_BY_CMD}
Process was killed by a command.
%
\declareconstitem{PMIX_PROC_STATE_ABORTED}
Process was aborted by a call to \refapi{PMIx_Abort}.
%
\declareconstitem{PMIX_PROC_STATE_FAILED_TO_START}
Process failed to start.
%
\declareconstitem{PMIX_PROC_STATE_ABORTED_BY_SIG}
Process aborted by a signal.
%
\declareconstitem{PMIX_PROC_STATE_TERM_WO_SYNC}
Process exited without calling \refapi{PMIx_Finalize}.
%
\declareconstitem{PMIX_PROC_STATE_COMM_FAILED}
Process communication has failed.
%
\declareconstitemNEW{PMIX_PROC_STATE_SENSOR_BOUND_EXCEEDED}
Process exceeded a specified sensor limit.
%
\declareconstitem{PMIX_PROC_STATE_CALLED_ABORT}
Process called \refapi{PMIx_Abort}.
%
\declareconstitemNEW{PMIX_PROC_STATE_HEARTBEAT_FAILED}
Frocess failed to send heartbeat within specified time limit.
%
\declareconstitem{PMIX_PROC_STATE_MIGRATING}
Process failed and is waiting for resources before restarting.
%
\declareconstitem{PMIX_PROC_STATE_CANNOT_RESTART}
Process failed and cannot be restarted.
%
\declareconstitem{PMIX_PROC_STATE_TERM_NON_ZERO}
Process exited with a non-zero status.
%
\declareconstitem{PMIX_PROC_STATE_FAILED_TO_LAUNCH}
Unable to launch process.
%
\end{constantdesc}


%%%%%%%%%%%%%%%%%%%%%%%%%%%%%%%%%%%%%%%%%%%%%%%%%
\subsection{Process Information Structure}
\declarestruct{pmix_proc_info_t}

The \refstruct{pmix_proc_info_t} structure defines a set of information about a specific process including it's name, location, and state.

\versionMarker{2.0}
\cspecificstart
\begin{codepar}
typedef struct pmix_proc_info \{
    /** Process structure */
    pmix_proc_t proc;
    /** Hostname where process resides */
    char *hostname;
    /** Name of the executable */
    char *executable_name;
    /** Process ID on the host */
    pid_t pid;
    /** Exit code of the process. Default: 0 */
    int exit_code;
    /** Current state of the process */
    pmix_proc_state_t state;
\} pmix_proc_info_t;
\end{codepar}
\cspecificend


%%%%%%%%%%%%%%%%%%%%%%%%%%%%%%%%%%%%%%%%%%%%%%%%%
\subsubsection{Process information structure support macros}

The following macros are provided to support the \refstruct{pmix_proc_info_t} structure.

%%%%
\littleheader{Initialize the process information structure}
\declaremacro{PMIX_PROC_INFO_CONSTRUCT}

Initialize the \refstruct{pmix_proc_info_t} fields.

\versionMarker{2.0}
\cspecificstart
\begin{codepar}
PMIX_PROC_INFO_CONSTRUCT(m)
\end{codepar}
\cspecificend

\begin{arglist}
\argin{m}{Pointer to the structure to be initialized (pointer to \refstruct{pmix_proc_info_t})}
\end{arglist}

%%%%
\littleheader{Destruct the process information structure}
\declaremacro{PMIX_PROC_INFO_DESTRUCT}

Destruct the \refstruct{pmix_proc_info_t} fields.

\versionMarker{2.0}
\cspecificstart
\begin{codepar}
PMIX_PROC_INFO_DESTRUCT(m)
\end{codepar}
\cspecificend

\begin{arglist}
\argin{m}{Pointer to the structure to be destructed (pointer to \refstruct{pmix_proc_info_t})}
\end{arglist}

%%%%
\littleheader{Create a process information array}
\declaremacro{PMIX_PROC_INFO_CREATE}

Allocate and initialize a \refstruct{pmix_proc_info_t} array.

\versionMarker{2.0}
\cspecificstart
\begin{codepar}
PMIX_PROC_INFO_CREATE(m, n)
\end{codepar}
\cspecificend

\begin{arglist}
\arginout{m}{Address where the pointer to the array of \refstruct{pmix_proc_info_t} structures shall be stored (handle)}
\argin{n}{Number of structures to be allocated (\code{size_t})}
\end{arglist}

%%%%
\littleheader{Free a process information structure}
\declaremacro{PMIX_PROC_INFO_RELEASE}

Release a \refstruct{pmix_proc_info_t} structure.

\versionMarker{2.0}
\cspecificstart
\begin{codepar}
PMIX_PROC_INFO_RELEASE(m)
\end{codepar}
\cspecificend

\begin{arglist}
\argin{m}{Pointer to a \refstruct{pmix_proc_info_t} structure (handle)}
\end{arglist}

%%%%
\littleheader{Free a process information array}
\declaremacro{PMIX_PROC_INFO_FREE}

Release an array of \refstruct{pmix_proc_info_t} structures.

\versionMarker{2.0}
\cspecificstart
\begin{codepar}
PMIX_PROC_INFO_FREE(m, n)
\end{codepar}
\cspecificend

\begin{arglist}
\argin{m}{Pointer to the array of \refstruct{pmix_proc_info_t} structures (handle)}
\argin{n}{Number of structures in the array (\code{size_t})}
\end{arglist}


%%%%%%%%%%%%%%%%%%%%%%%%%%%%%%%%%%%%%%%%%%%%%%%%%
\subsection{Job State Structure}
\label{api:struct:jobstate}
\declarestruct{pmix_job_state_t}

\versionMarker{4.0}
The \refstruct{pmix_job_state_t} structure is a \code{uint8_t} type for job state values. The following constants can be used to set a variable of the type \refstruct{pmix_job_state_t}.

\adviceuserstart
The fine-grained nature of the following constants may exceed the ability of an \ac{RM} to provide updated job state values during the job lifetime. This is particularly true for short-lived jobs.
\adviceuserend

\begin{constantdesc}
%
\declareconstitemNEW{PMIX_JOB_STATE_UNDEF}
Undefined job state.
%
\declareconstitemNEW{PMIX_JOB_STATE_AWAITING_ALLOC}
Job is waiting for resources to be allocated to it.
%
\declareconstitemNEW{PMIX_JOB_STATE_LAUNCH_UNDERWAY}
Job launch is underway.
%
\declareconstitemNEW{PMIX_JOB_STATE_RUNNING}
All processes in the job have been spawned and are executing.
%
\declareconstitemNEW{PMIX_JOB_STATE_SUSPENDED}
All processes in the job have been suspended.
%
\declareconstitemNEW{PMIX_JOB_STATE_CONNECTED}
All processes in the job have connected to their \ac{PMIx} server.
%
\declareconstitemNEW{PMIX_JOB_STATE_UNTERMINATED}
Define a ``boundary'' between the terminated states and \refconst{PMIX_JOB_STATE_TERMINATED} so users can easily and quickly determine if a job is still running or not.
Any value less than this constant means that the job has not terminated.
%
\declareconstitemNEW{PMIX_JOB_STATE_TERMINATED}
All processes in the job have terminated and are no longer running - typically will be accompanied by the job exit status in response to a query.
%
\declareconstitemNEW{PMIX_JOB_STATE_TERMINATED_WITH_ERROR}
Define a boundary so users can easily and quickly determine if a job abnormally terminated - typically will be accompanied by a job-related error code in response to a query
Any value above this constant means that the job terminated abnormally.
%
\end{constantdesc}


%%%%%%%%%%%%%%%%%%%%%%%%%%%%%%%%%%%%%%%%%%%%%%%%%
\subsection{Value Structure}
\declarestruct{pmix_value_t}

The \refstruct{pmix_value_t} structure is used to represent the value passed to \refapi{PMIx_Put} and retrieved by \refapi{PMIx_Get}, as well as many of the other \ac{PMIx} functions.

A collection of values may be specified under a single key by passing a \refstruct{pmix_value_t} containing an array of type \refstruct{pmix_data_array_t}, with each array element containing its own object. All members shown below were introduced in version 1 of the standard unless otherwise marked.

\versionMarker{1.0}
\cspecificstart
\begin{codepar}
typedef struct pmix_value \{
    pmix_data_type_t type;
    union \{
        bool flag;
        uint8_t byte;
        char *string;
        size_t size;
        pid_t pid;
        int integer;
        int8_t int8;
        int16_t int16;
        int32_t int32;
        int64_t int64;
        unsigned int uint;
        uint8_t uint8;
        uint16_t uint16;
        uint32_t uint32;
        uint64_t uint64;
        float fval;
        double dval;
        struct timeval tv;
        time_t time;                    // version 2.0
        pmix_status_t status;           // version 2.0
        pmix_rank_t rank;               // version 2.0
        pmix_proc_t *proc;              // version 2.0
        pmix_byte_object_t bo;
        pmix_persistence_t persist;     // version 2.0
        pmix_scope_t scope;             // version 2.0
        pmix_data_range_t range;        // version 2.0
        pmix_proc_state_t state;        // version 2.0
        pmix_proc_info_t *pinfo;        // version 2.0
        pmix_data_array_t *darray;      // version 2.0
        void *ptr;                      // version 2.0
        pmix_alloc_directive_t adir;    // version 2.0
    \} data;
\} pmix_value_t;
\end{codepar}
\cspecificend

%%%%%%%%%%%%%%%%%%%%%%%%%%%%%%%%%%%%%%%%%%%%%%%%%
\subsubsection{Value structure support macros}
The following macros are provided to support the \refstruct{pmix_value_t} structure.

\littleheader{Initialize the value structure}
\declaremacro{PMIX_VALUE_CONSTRUCT}

Initialize the \refstruct{pmix_value_t} fields.

\versionMarker{1.0}
\cspecificstart
\begin{codepar}
PMIX_VALUE_CONSTRUCT(m)
\end{codepar}
\cspecificend

\begin{arglist}
\argin{m}{Pointer to the structure to be initialized (pointer to \refstruct{pmix_value_t})}
\end{arglist}

\littleheader{Destruct the value structure}
\declaremacro{PMIX_VALUE_DESTRUCT}

Destruct the \refstruct{pmix_value_t} fields.

\versionMarker{1.0}
\cspecificstart
\begin{codepar}
PMIX_VALUE_DESTRUCT(m)
\end{codepar}
\cspecificend

\begin{arglist}
\argin{m}{Pointer to the structure to be destructed (pointer to \refstruct{pmix_value_t})}
\end{arglist}

%%%%%%%%%%%
\littleheader{Create a value array}
\declaremacro{PMIX_VALUE_CREATE}

Allocate and initialize an array of \refstruct{pmix_value_t} structures.

\versionMarker{1.0}
\cspecificstart
\begin{codepar}
PMIX_VALUE_CREATE(m, n)
\end{codepar}
\cspecificend

\begin{arglist}
\arginout{m}{Address where the pointer to the array of \refstruct{pmix_value_t} structures shall be stored (handle)}
\argin{n}{Number of structures to be allocated (\code{size_t})}
\end{arglist}


%%%%%%%%%%%
\littleheader{Free a value structure}
\declaremacro{PMIX_VALUE_RELEASE}

Release a \refstruct{pmix_value_t} structure.

\versionMarker{4.0}
\cspecificstart
\begin{codepar}
PMIX_VALUE_RELEASE(m)
\end{codepar}
\cspecificend

\begin{arglist}
\argin{m}{Pointer to a \refstruct{pmix_value_t} structure (handle)}
\end{arglist}

%%%%%%%%%%%
\littleheader{Free a value array}
\declaremacro{PMIX_VALUE_FREE}

Release an array of \refstruct{pmix_value_t} structures.

\versionMarker{1.0}
\cspecificstart
\begin{codepar}
PMIX_VALUE_FREE(m, n)
\end{codepar}
\cspecificend

\begin{arglist}
\argin{m}{Pointer to the array of \refstruct{pmix_value_t} structures (handle)}
\argin{n}{Number of structures in the array (\code{size_t})}
\end{arglist}

%%%%%%%%%%%
\littleheader{Load a value structure}
\declaremacro{PMIX_VALUE_LOAD}

Load data into a \refstruct{pmix_value_t} structure.

\versionMarker{2.0}
\cspecificstart
\begin{codepar}
PMIX_VALUE_LOAD(v, d, t);
\end{codepar}
\cspecificend

\begin{arglist}
\argin{v}{The \refstruct{pmix_value_t} into which the data is to be loaded (pointer to \refstruct{pmix_value_t})}
\argin{d}{Pointer to the data value to be loaded (handle)}
\argin{t}{Type of the provided data value (\refstruct{pmix_data_type_t})}
\end{arglist}

This macro simplifies the loading of data into a \refstruct{pmix_value_t} by correctly assigning values to the structure's fields.

\adviceuserstart
The data will be copied into the \refstruct{pmix_value_t} - thus, any data stored in the source value can be modified or free'd without affecting the copied data once the macro has completed.
\adviceuserend

%%%%%%%%%%%
\littleheader{Unload a value structure}
\declaremacro{PMIX_VALUE_UNLOAD}

Unload data from a \refstruct{pmix_value_t} structure.

\versionMarker{2.2}
\cspecificstart
\begin{codepar}
PMIX_VALUE_UNLOAD(r, v, d, t);
\end{codepar}
\cspecificend

\begin{arglist}
\argout{r}{Status code indicating result of the operation {\refstruct{pmix_status_t}}}
\argin{v}{The \refstruct{pmix_value_t} from which the data is to be unloaded (pointer to \refstruct{pmix_value_t})}
\arginout{d}{Pointer to the location where the data value is to be returned (handle)}
\arginout{t}{Pointer to return the data type of the unloaded value (handle)}
\end{arglist}

This macro simplifies the unloading of data from a \refstruct{pmix_value_t}.

\adviceuserstart
Memory will be allocated and the data will be in the \refstruct{pmix_value_t} returned - the source \refstruct{pmix_value_t} will not be altered.
\adviceuserend

%%%%%%%%%%%
\littleheader{Transfer data between value structures}
\declaremacro{PMIX_VALUE_XFER}

Transfer the data value between two \refstruct{pmix_value_t} structures.

\versionMarker{2.0}
\cspecificstart
\begin{codepar}
PMIX_VALUE_XFER(r, d, s);
\end{codepar}
\cspecificend

\begin{arglist}
\argout{r}{Status code indicating success or failure of the transfer (\refstruct{pmix_status_t})}
\argin{d}{Pointer to the \refstruct{pmix_value_t} destination (handle)}
\argin{s}{Pointer to the \refstruct{pmix_value_t} source (handle)}
\end{arglist}

This macro simplifies the transfer of data between two \refstruct{pmix_value_t} structures, ensuring that all fields are properly copied.

\adviceuserstart
The data will be copied into the destination \refstruct{pmix_value_t} - thus, any data stored in the source value can be modified or free'd without affecting the copied data once the macro has completed.
\adviceuserend

%%%%%%%%%%%
\littleheader{Retrieve a numerical value from a value struct}
\declaremacro{PMIX_VALUE_GET_NUMBER}

Retrieve a numerical value from a \refstruct{pmix_value_t} structure.

\versionMarker{3.0}
\cspecificstart
\begin{codepar}
PMIX_VALUE_GET_NUMBER(s, m, n, t)
\end{codepar}
\cspecificend

\begin{arglist}
\argout{s}{Status code for the request (\refstruct{pmix_status_t})}
\argin{m}{Pointer to the\refstruct{pmix_value_t} structure (handle)}
\argout{n}{Variable to be set to the value (match expected type)}
\argin{t}{Type of number expected in \refarg{m} (\refstruct{pmix_data_type_t})}
\end{arglist}

Sets the provided variable equal to the numerical value contained in the given \refstruct{pmix_value_t}, returning success if the data type of the value matches the expected type and \refconst{PMIX_ERR_BAD_PARAM} if it doesn't

%%%%%%%%%%%%%%%%%%%%%%%%%%%%%%%%%%%%%%%%%%%%%%%%%
\subsection{Info Structure}
\label{chap:struct:info}
\declarestruct{pmix_info_t}

The \refstruct{pmix_info_t} structure defines a key/value pair with associated directive. All fields were defined in version 1.0 unless otherwise marked.

\versionMarker{1.0}
\cspecificstart
\begin{codepar}
typedef struct pmix_info_t \{
    pmix_key_t key;
    pmix_info_directives_t flags;    // version 2.0
    pmix_value_t value;
\} pmix_info_t;
\end{codepar}
\cspecificend

%%%%%%%%%%%
\subsubsection{Info structure support macros}
The following macros are provided to support the \refstruct{pmix_info_t} structure.

\littleheader{Initialize the info structure}
\declaremacro{PMIX_INFO_CONSTRUCT}

Initialize the \refstruct{pmix_info_t} fields.

\versionMarker{1.0}
\cspecificstart
\begin{codepar}
PMIX_INFO_CONSTRUCT(m)
\end{codepar}
\cspecificend

\begin{arglist}
\argin{m}{Pointer to the structure to be initialized (pointer to \refstruct{pmix_info_t})}
\end{arglist}

\littleheader{Destruct the info structure}
\declaremacro{PMIX_INFO_DESTRUCT}

Destruct the \refstruct{pmix_info_t} fields.

\versionMarker{1.0}
\cspecificstart
\begin{codepar}
PMIX_INFO_DESTRUCT(m)
\end{codepar}
\cspecificend

\begin{arglist}
\argin{m}{Pointer to the structure to be destructed (pointer to \refstruct{pmix_info_t})}
\end{arglist}

%%%%%%%%%%%
\littleheader{Create an info array}
\declaremacro{PMIX_INFO_CREATE}

Allocate and initialize an array of info structures.

\versionMarker{1.0}
\cspecificstart
\begin{codepar}
PMIX_INFO_CREATE(m, n)
\end{codepar}
\cspecificend

\begin{arglist}
\arginout{m}{Address where the pointer to the array of \refstruct{pmix_info_t} structures shall be stored (handle)}
\argin{n}{Number of structures to be allocated (\code{size_t})}
\end{arglist}


%%%%%%%%%%%
\littleheader{Free an info array}
\declaremacro{PMIX_INFO_FREE}

Release an array of \refstruct{pmix_info_t} structures.

\versionMarker{1.0}
\cspecificstart
\begin{codepar}
PMIX_INFO_FREE(m, n)
\end{codepar}
\cspecificend

\begin{arglist}
\argin{m}{Pointer to the array of \refstruct{pmix_info_t} structures (handle)}
\argin{n}{Number of structures in the array (\code{size_t})}
\end{arglist}

%%%%%%%%%%%
\littleheader{Load key and value data into a info struct}
\declaremacro{PMIX_INFO_LOAD}

\versionMarker{1.0}
\cspecificstart
\begin{codepar}
PMIX_INFO_LOAD(v, k, d, t);
\end{codepar}
\cspecificend

\begin{arglist}
\argin{v}{Pointer to the \refstruct{pmix_info_t} into which the key and data are to be loaded (pointer to \refstruct{pmix_info_t})}
\argin{k}{String key to be loaded - must be less than or equal to \refconst{PMIX_MAX_KEYLEN} in length (handle)}
\argin{d}{Pointer to the data value to be loaded (handle)}
\argin{t}{Type of the provided data value (\refstruct{pmix_data_type_t})}
\end{arglist}

This macro simplifies the loading of key and data into a \refstruct{pmix_info_t} by correctly assigning values to the structure's fields.

\adviceuserstart
Both key and data will be copied into the \refstruct{pmix_info_t} - thus, the key and any data stored in the source value can be modified or free'd without affecting the copied data once the macro has completed.
\adviceuserend

%%%%%%%%%%%
\littleheader{Copy data between info structures}
\declaremacro{PMIX_INFO_XFER}

Copy all data (including key, value, and directives) between two \refstruct{pmix_info_t} structures.

\versionMarker{2.0}
\cspecificstart
\begin{codepar}
PMIX_INFO_XFER(d, s);
\end{codepar}
\cspecificend

\begin{arglist}
\argin{d}{Pointer to the destination \refstruct{pmix_info_t} (pointer to \refstruct{pmix_info_t})}
\argin{s}{Pointer to the source \refstruct{pmix_info_t} (pointer to \refstruct{pmix_info_t})}
\end{arglist}

This macro simplifies the transfer of data between two\refstruct{pmix_info_t} structures.

\adviceuserstart
All data (including key, value, and directives) will be copied into the destination \refstruct{pmix_info_t} - thus, the source \refstruct{pmix_info_t} may be free'd without affecting the copied data once the macro has completed.
\adviceuserend


%%%%%%%%%%%
\littleheader{Test a boolean info struct}
\declaremacro{PMIX_INFO_TRUE}

A special macro for checking if a boolean \refstruct{pmix_info_t} is \code{true}.

\versionMarker{2.0}
\cspecificstart
\begin{codepar}
PMIX_INFO_TRUE(m)
\end{codepar}
\cspecificend

\begin{arglist}
\argin{m}{Pointer to a \refstruct{pmix_info_t} structure (handle)}
\end{arglist}

A \refstruct{pmix_info_t} structure is considered to be of type \refconst{PMIX_BOOL} and value \code{true} if:

\begin{compactitemize}
    \item the structure reports a type of \refconst{PMIX_UNDEF}, or
    \item the structure reports a type of \refconst{PMIX_BOOL} and the data flag is \code{true}
\end{compactitemize}

%%%%%%%%%%%
\subsubsection{Info structure list macros}
Constructing an array of \refstruct{pmix_info_t} is a fairly common operation. The following macros are provided to simplify this construction.

%%%%%%%%%%%
\littleheader{Start a list of \refstruct{pmix_info_t} structures}
\declaremacro{PMIX_INFO_LIST_START}

Initialize a list of \refstruct{pmix_info_t} structures. The actual list is opaque to the caller and is implementation-dependent.

\versionMarker{4.0}
\cspecificstart
\begin{codepar}
PMIX_INFO_LIST_START(m)
\end{codepar}
\cspecificend

\begin{arglist}
\argin{m}{A \code{void*} pointer (handle)}
\end{arglist}

Note that the pointer will be initialized to an opaque structure whose elements are implementation-dependent. The caller must not modify or dereference the object.

%%%%%%%%%%%
\littleheader{Add a \refstruct{pmix_info_t} structure to a list}
\declaremacro{PMIX_INFO_LIST_ADD}

Add a \refstruct{pmix_info_t} structure containing the specified value to the provided list.

\versionMarker{4.0}
\cspecificstart
\begin{codepar}
PMIX_INFO_LIST_ADD(rc, m, k, d, t)
\end{codepar}
\cspecificend

\begin{arglist}
\arginout{rc}{Return status for the operation (\refstruct{pmix_status_t})}
\argin{m}{A \code{void*} pointer initialized via \refmacro{PMIX_INFO_LIST_START} (handle)}
\argin{k}{String key to be loaded - must be less than or equal to \refconst{PMIX_MAX_KEYLEN} in length (handle)}
\argin{d}{Pointer to the data value to be loaded (handle)}
\argin{t}{Type of the provided data value (\refstruct{pmix_data_type_t})}
\end{arglist}

\adviceuserstart
Both key and data will be copied into the \refstruct{pmix_info_t} on the list - thus, the key and any data stored in the source value can be modified or free'd without affecting the copied data once the macro has completed.
\adviceuserend

%%%%%%%%%%%
\littleheader{Transfer a \refstruct{pmix_info_t} structure to a list}
\declaremacro{PMIX_INFO_LIST_XFER}

Transfer the information in a \refstruct{pmix_info_t} structure to the provided list.

\versionMarker{4.0}
\cspecificstart
\begin{codepar}
PMIX_INFO_LIST_XFER(rc, m, s)
\end{codepar}
\cspecificend

\begin{arglist}
\arginout{rc}{Return status for the operation (\refstruct{pmix_status_t})}
\argin{m}{A \code{void*} pointer initialized via \refmacro{PMIX_INFO_LIST_START} (handle)}
\argin{s}{Pointer to the source \refstruct{pmix_info_t} (pointer to \refstruct{pmix_info_t})}
\end{arglist}

\adviceuserstart
All data (including key, value, and directives) will be copied into the destination \refstruct{pmix_info_t} on the list - thus, the source \refstruct{pmix_info_t} may be free'd without affecting the copied data once the macro has completed.
\adviceuserend

%%%%%%%%%%%
\littleheader{Convert a \refstruct{pmix_info_t} list to an array}
\declaremacro{PMIX_INFO_LIST_CONVERT}

Transfer the information in the provided \refstruct{pmix_info_t} list to a \refstruct{pmix_data_array_t} array

\versionMarker{4.0}
\cspecificstart
\begin{codepar}
PMIX_INFO_LIST_CONVERT(rc, m, d)
\end{codepar}
\cspecificend

\begin{arglist}
\arginout{rc}{Return status for the operation (\refstruct{pmix_status_t})}
\argin{m}{A \code{void*} pointer initialized via \refmacro{PMIX_INFO_LIST_START} (handle)}
\argin{d}{Pointer to an instantiated \refstruct{pmix_data_array_t} structure where the \refstruct{pmix_info_t} array is to be stored (pointer to \refstruct{pmix_data_array_t})}
\end{arglist}

%%%%%%%%%%%
\littleheader{Release a \refstruct{pmix_info_t} list}
\declaremacro{PMIX_INFO_LIST_RELEASE}

Release the provided \refstruct{pmix_info_t} list

\versionMarker{4.0}
\cspecificstart
\begin{codepar}
PMIX_INFO_LIST_RELEASE(m)
\end{codepar}
\cspecificend

\begin{arglist}
\argin{m}{A \code{void*} pointer initialized via \refmacro{PMIX_INFO_LIST_START} (handle)}
\end{arglist}

Information contained in the \refstruct{pmix_info_t} on the list shall be released in addition to whatever backing storage the implementation may have allocated to support construction of the list.


%%%%%%%%%%%%%%%%%%%%%%%%%%%%%%%%%%%%%%%%%%%%%%%%%
\subsection{Info Type Directives}
\declarestruct{pmix_info_directives_t}
\label{api:struct:infodirs}

\versionMarker{2.0}
The \refstruct{pmix_info_directives_t} structure is a \code{uint32_t} type that defines the behavior of command directives via \refstruct{pmix_info_t} arrays.
By default, the values in the \refstruct{pmix_info_t} array passed to a PMIx are \emph{optional}.

\adviceuserstart
A PMIx implementation or PMIx-enabled \ac{RM} may ignore any \refstruct{pmix_info_t} value passed to a \ac{PMIx} \ac{API} that it does not support or does not recognize if it is not explicitly marked as \refconst{PMIX_INFO_REQD}.
This is because the values specified default to optional, meaning they can be ignored in such circumstances.
This may lead to unexpected behavior when porting between environments or \ac{PMIx} implementations if the user is relying on the behavior specified by the \refstruct{pmix_info_t} value.
Users relying on the behavior defined by the \refstruct{pmix_info_t} are advised to set the \refconst{PMIX_INFO_REQD} flag using the \refmacro{PMIX_INFO_REQUIRED} macro.
\adviceuserend

\adviceimplstart
The top 16-bits of the \refstruct{pmix_info_directives_t} are reserved for internal use by \ac{PMIx} library implementers - the \ac{PMIx} standard will \textit{not} specify their intent, leaving them for customized use by implementers. Implementers are advised to use the provided \refmacro{PMIX_INFO_IS_REQUIRED} macro for testing this flag, and must return \refconst{PMIX_ERR_NOT_SUPPORTED} as soon as possible to the caller if the required behavior is not supported.
\adviceimplend

The following constants were introduced in version 2.0 (unless otherwise marked) and can be used to set a variable of the type \refstruct{pmix_info_directives_t}.

\begin{constantdesc}
%
\declareconstitem{PMIX_INFO_REQD}
The behavior defined in the \refstruct{pmix_info_t} array is required, and not optional. This is a bit-mask value.
%
\declareconstitemNEW{PMIX_INFO_REQD_PROCESSED}
Mark that this required attribute has been processed. A required attribute can be handled at any level - the \ac{PMIx} client library might take care of it, or it may be resolved by the \ac{PMIx} server library, or it may pass up to the host environment for handling. If a level does not recognize or support the required attribute, it is required to pass it upwards to give the next level an opportunity to process it. Thus, the host environment (or the server library if the host does not support the given operation) must know if a lower level has handled the requirement so it can return a \refconst{PMIX_ERR_NOT_SUPPORTED} error status if the host itself cannot meet the request. Upon processing the request, the level must therefore mark the attribute with this directive to alert any subsequent levels that the requirement has been met.
%
\declareconstitem{PMIX_INFO_ARRAY_END}
Mark that this \refstruct{pmix_info_t} struct is at the end of an array created by the \refmacro{PMIX_INFO_CREATE} macro. This is a bit-mask value.
%
\declareconstitemNEW{PMIX_INFO_DIR_RESERVED}
A bit-mask identifying the bits reserved for internal use by implementers - these currently are set as \code{0xffff0000}.
%
\end{constantdesc}

\advicermstart
Host environments are advised to use the provided \refmacro{PMIX_INFO_IS_REQUIRED} macro for testing this flag and must return \refconst{PMIX_ERR_NOT_SUPPORTED} as soon as possible to the caller if the required behavior is not supported.
\advicermend


\subsubsection{Info Directive support macros}

The following macros are provided to support the setting and testing of \refstruct{pmix_info_t} directives.

%%%%
\littleheader{Mark an info structure as required}
\declaremacro{PMIX_INFO_REQUIRED}

Set the \refconst{PMIX_INFO_REQD} flag in a \refstruct{pmix_info_t} structure.

\versionMarker{2.0}
\cspecificstart
\begin{codepar}
PMIX_INFO_REQUIRED(info);
\end{codepar}
\cspecificend

\begin{arglist}
\argin{info}{Pointer to the \refstruct{pmix_info_t} (pointer to \refstruct{pmix_info_t})}
\end{arglist}

This macro simplifies the setting of the \refconst{PMIX_INFO_REQD} flag in \refstruct{pmix_info_t} structures.

%%%%
\littleheader{Mark an info structure as optional}
\declaremacro{PMIX_INFO_OPTIONAL}

Unsets the \refconst{PMIX_INFO_REQD} flag in a \refstruct{pmix_info_t} structure.

\versionMarker{2.0}
\cspecificstart
\begin{codepar}
PMIX_INFO_OPTIONAL(info);
\end{codepar}
\cspecificend

\begin{arglist}
\argin{info}{Pointer to the \refstruct{pmix_info_t} (pointer to \refstruct{pmix_info_t})}
\end{arglist}

This macro simplifies marking a \refstruct{pmix_info_t} structure as \textit{optional}.

%%%%%%%%%%%
\littleheader{Test an info structure for \textit{required} directive}
\declaremacro{PMIX_INFO_IS_REQUIRED}

Test the \refconst{PMIX_INFO_REQD} flag in a \refstruct{pmix_info_t} structure, returning \code{true} if the flag is set.

\versionMarker{2.0}
\cspecificstart
\begin{codepar}
PMIX_INFO_IS_REQUIRED(info);
\end{codepar}
\cspecificend

\begin{arglist}
\argin{info}{Pointer to the \refstruct{pmix_info_t} (pointer to \refstruct{pmix_info_t})}
\end{arglist}

This macro simplifies the testing of the required flag in \refstruct{pmix_info_t} structures.

%%%%%%%%%%%
\littleheader{Test an info structure for \textit{optional} directive}
\declaremacro{PMIX_INFO_IS_OPTIONAL}

Test a \refstruct{pmix_info_t} structure, returning \code{true} if the structure is \textit{optional}.

\versionMarker{2.0}
\cspecificstart
\begin{codepar}
PMIX_INFO_IS_OPTIONAL(info);
\end{codepar}
\cspecificend

\begin{arglist}
\argin{info}{Pointer to the \refstruct{pmix_info_t} (pointer to \refstruct{pmix_info_t})}
\end{arglist}

Test the \refconst{PMIX_INFO_REQD} flag in a \refstruct{pmix_info_t} structure, returning \code{true} if the flag is \textit{not} set.

%%%%%%%%%%%
\littleheader{Mark a required attribute as processed}
\declaremacro{PMIX_INFO_PROCESSED}

Mark that a required \refstruct{pmix_info_t} structure has been processed.

\versionMarker{4.0}
\cspecificstart
\begin{codepar}
PMIX_INFO_PROCESSED(info);
\end{codepar}
\cspecificend

\begin{arglist}
\argin{info}{Pointer to the \refstruct{pmix_info_t} (pointer to \refstruct{pmix_info_t})}
\end{arglist}

Set the \refconst{PMIX_INFO_REQD_PROCESSED} flag in a \refstruct{pmix_info_t} structure indicating that is has been processed.

%%%%%%%%%%%
\littleheader{Test if a required attribute has been processed}
\declaremacro{PMIX_INFO_WAS_PROCESSED}

Test that a required \refstruct{pmix_info_t} structure has been processed.

\versionMarker{4.0}
\cspecificstart
\begin{codepar}
PMIX_INFO_WAS_PROCESSED(info);
\end{codepar}
\cspecificend

\begin{arglist}
\argin{info}{Pointer to the \refstruct{pmix_info_t} (pointer to \refstruct{pmix_info_t})}
\end{arglist}

Test the \refconst{PMIX_INFO_REQD_PROCESSED} flag in a \refstruct{pmix_info_t} structure.

%%%%%%%%%%%
\littleheader{Test an info structure for \textit{end of array} directive}
\declaremacro{PMIX_INFO_IS_END}

Test a \refstruct{pmix_info_t} structure, returning \code{true} if the structure is at the end of an array created by the \refmacro{PMIX_INFO_CREATE} macro.

\versionMarker{2.2}
\cspecificstart
\begin{codepar}
PMIX_INFO_IS_END(info);
\end{codepar}
\cspecificend

\begin{arglist}
\argin{info}{Pointer to the \refstruct{pmix_info_t} (pointer to \refstruct{pmix_info_t})}
\end{arglist}

This macro simplifies the testing of the end-of-array flag in \refstruct{pmix_info_t} structures.

%%%%%%%%%%%%%%%%%%%%%%%%%%%%%%%%%%%%%%%%%%%%%%%%%
\subsection{Environmental Variable Structure}
\declarestruct{pmix_envar_t}

\versionMarker{3.0}
Define a structure for specifying environment variable modifications.
Standard environment variables (e.g., \code{PATH}, \code{LD_LIBRARY_PATH}, and \code{LD_PRELOAD})
take multiple arguments separated by delimiters. Unfortunately, the delimiters
depend upon the variable itself - some use semi-colons, some colons, etc. Thus,
the operation requires not only the name of the variable to be modified and
the value to be inserted, but also the separator to be used when composing
the aggregate value.

\cspecificstart
\begin{codepar}
typedef struct \{
    char *envar;
    char *value;
    char separator;
\} pmix_envar_t;
\end{codepar}
\cspecificend

%%%%%%%%%%%%%%%%%%%%%%%%%%%%%%%%%%%%%%%%%%%%%%%%%
\subsubsection{Environmental variable support macros}

The following macros are provided to support the \refstruct{pmix_envar_t} structure.

\littleheader{Initialize the envar structure}
\declaremacro{PMIX_ENVAR_CONSTRUCT}

Initialize the \refstruct{pmix_envar_t} fields.

\versionMarker{3.0}
\cspecificstart
\begin{codepar}
PMIX_ENVAR_CONSTRUCT(m)
\end{codepar}
\cspecificend

\begin{arglist}
\argin{m}{Pointer to the structure to be initialized (pointer to \refstruct{pmix_envar_t})}
\end{arglist}

\littleheader{Destruct the envar structure}
\declaremacro{PMIX_ENVAR_DESTRUCT}

Clear the \refstruct{pmix_envar_t} fields.

\versionMarker{3.0}
\cspecificstart
\begin{codepar}
PMIX_ENVAR_DESTRUCT(m)
\end{codepar}
\cspecificend

\begin{arglist}
\argin{m}{Pointer to the structure to be destructed (pointer to \refstruct{pmix_envar_t})}
\end{arglist}


\littleheader{Create an envar array}
\declaremacro{PMIX_ENVAR_CREATE}

Allocate and initialize an array of \refstruct{pmix_envar_t} structures.

\versionMarker{3.0}
\cspecificstart
\begin{codepar}
PMIX_ENVAR_CREATE(m, n)
\end{codepar}
\cspecificend

\begin{arglist}
\arginout{m}{Address where the pointer to the array of \refstruct{pmix_envar_t} structures shall be stored (handle)}
\argin{n}{Number of structures to be allocated (\code{size_t})}
\end{arglist}


\littleheader{Free an envar array}
\declaremacro{PMIX_ENVAR_FREE}

Release an array of \refstruct{pmix_envar_t} structures.

\versionMarker{3.0}
\cspecificstart
\begin{codepar}
PMIX_ENVAR_FREE(m, n)
\end{codepar}
\cspecificend

\begin{arglist}
\argin{m}{Pointer to the array of \refstruct{pmix_envar_t} structures (handle)}
\argin{n}{Number of structures in the array (\code{size_t})}
\end{arglist}

\littleheader{Load an envar structure}
\declaremacro{PMIX_ENVAR_LOAD}

Load values into a \refstruct{pmix_envar_t}.

\versionMarker{2.0}
\cspecificstart
\begin{codepar}
PMIX_ENVAR_LOAD(m, e, v, s)
\end{codepar}
\cspecificend

\begin{arglist}
\argin{m}{Pointer to the structure to be loaded (pointer to \refstruct{pmix_envar_t})}
\argin{e}{Environmental variable name (\code{char*})}
\argin{v}{Value of variable (\code{char*})}
\argin{v}{Separator character (\code{char})}
\end{arglist}


%%%%%%%%%%%%%%%%%%%%%%%%%%%%%%%%%%%%%%%%%%%%%%%%%
\subsection{Byte Object Type}
\declarestruct{pmix_byte_object_t}

The \refstruct{pmix_byte_object_t} structure describes a raw byte sequence.

\versionMarker{1.0}
\cspecificstart
\begin{codepar}
typedef struct pmix_byte_object \{
    char *bytes;
    size_t size;
\} pmix_byte_object_t;
\end{codepar}
\cspecificend

%%%%%%%%%%%%%%%%%%%%%%%%%%%%%%%%%%%%%%%%%%%%%%%%%
\subsubsection{Byte object support macros}
The following macros support the \refstruct{pmix_byte_object_t} structure.

\littleheader{Initialize the byte object structure}
\declaremacro{PMIX_BYTE_OBJECT_CONSTRUCT}

Initialize the \refstruct{pmix_byte_object_t} fields.

\versionMarker{2.0}
\cspecificstart
\begin{codepar}
PMIX_BYTE_OBJECT_CONSTRUCT(m)
\end{codepar}
\cspecificend

\begin{arglist}
\argin{m}{Pointer to the structure to be initialized (pointer to \refstruct{pmix_byte_object_t})}
\end{arglist}

\littleheader{Destruct the byte object structure}
\declaremacro{PMIX_BYTE_OBJECT_DESTRUCT}

Clear the \refstruct{pmix_byte_object_t} fields.

\versionMarker{2.0}
\cspecificstart
\begin{codepar}
PMIX_BYTE_OBJECT_DESTRUCT(m)
\end{codepar}
\cspecificend

\begin{arglist}
\argin{m}{Pointer to the structure to be destructed (pointer to \refstruct{pmix_byte_object_t})}
\end{arglist}

\littleheader{Create a byte object structure}
\declaremacro{PMIX_BYTE_OBJECT_CREATE}

Allocate and intitialize an array of \refstruct{pmix_byte_object_t} structures.

\versionMarker{2.0}
\cspecificstart
\begin{codepar}
PMIX_BYTE_OBJECT_CREATE(m, n)
\end{codepar}
\cspecificend

\begin{arglist}
\arginout{m}{Address where the pointer to the array of \refstruct{pmix_byte_object_t} structures shall be stored (handle)}
\argin{n}{Number of structures to be allocated (\code{size_t})}
\end{arglist}

\littleheader{Free a byte object array}
\declaremacro{PMIX_BYTE_OBJECT_FREE}

Release an array of \refstruct{pmix_byte_object_t} structures.

\versionMarker{2.0}
\cspecificstart
\begin{codepar}
PMIX_BYTE_OBJECT_FREE(m, n)
\end{codepar}
\cspecificend

\begin{arglist}
\argin{m}{Pointer to the array of \refstruct{pmix_byte_object_t} structures (handle)}
\argin{n}{Number of structures in the array (\code{size_t})}
\end{arglist}

\littleheader{Load a byte object structure}
\declaremacro{PMIX_BYTE_OBJECT_LOAD}

Load values into a \refstruct{pmix_byte_object_t}.

\versionMarker{2.0}
\cspecificstart
\begin{codepar}
PMIX_BYTE_OBJECT_LOAD(b, d, s)
\end{codepar}
\cspecificend

\begin{arglist}
\argin{b}{Pointer to the structure to be loaded (pointer to \refstruct{pmix_byte_object_t})}
\argin{d}{Pointer to the data to be loaded (\code{char*})}
\argin{s}{Number of bytes in the data array (\code{size_t})}
\end{arglist}


%%%%%%%%%%%%%%%%%%%%%%%%%%%%%%%%%%%%%%%%%%%%%%%%%
\subsection{Data Array Structure}
\declarestruct{pmix_data_array_t}

The \refstruct{pmix_data_array_t} structure defines an array data structure.

\versionMarker{2.0}
\cspecificstart
\begin{codepar}
typedef struct pmix_data_array \{
    pmix_data_type_t type;
    size_t size;
    void *array;
\} pmix_data_array_t;
\end{codepar}
\cspecificend

%%%%%%%%%%%%%%%%%%%%%%%%%%%%%%%%%%%%%%%%%%%%%%%%%
\subsubsection{Data array support macros}
The following macros support the \refstruct{pmix_data_array_t} structure.

\littleheader{Initialize a data array structure}
\declaremacro{PMIX_DATA_ARRAY_CONSTRUCT}

Initialize the \refstruct{pmix_data_array_t} fields, allocating memory for the array of the indicated type.

\versionMarker{2.2}
\cspecificstart
\begin{codepar}
PMIX_DATA_ARRAY_CONSTRUCT(m, n, t)
\end{codepar}
\cspecificend

\begin{arglist}
\argin{m}{Pointer to the structure to be initialized (pointer to \refstruct{pmix_data_array_t})}
\argin{n}{Number of elements in the array (\code{size_t})}
\argin{t}{\ac{PMIx} data type of the array elements (\refstruct{pmix_data_type_t})}
\end{arglist}


\littleheader{Destruct a data array structure}
\declaremacro{PMIX_DATA_ARRAY_DESTRUCT}

Destruct the \refstruct{pmix_data_array_t}, releasing the memory in the array.

\versionMarker{2.2}
\cspecificstart
\begin{codepar}
PMIX_DATA_ARRAY_CONSTRUCT(m)
\end{codepar}
\cspecificend

\begin{arglist}
\argin{m}{Pointer to the structure to be destructed (pointer to \refstruct{pmix_data_array_t})}
\end{arglist}


\littleheader{Create a data array structure}
\declaremacro{PMIX_DATA_ARRAY_CREATE}

Allocate memory for the \refstruct{pmix_data_array_t} object itself, and then allocate memory for the array of the indicated type.

\versionMarker{2.2}
\cspecificstart
\begin{codepar}
PMIX_DATA_ARRAY_CREATE(m, n, t)
\end{codepar}
\cspecificend

\begin{arglist}
\arginout{m}{Variable to be set to the address of the structure (pointer to \refstruct{pmix_data_array_t})}
\argin{n}{Number of elements in the array (\code{size_t})}
\argin{t}{\ac{PMIx} data type of the array elements (\refstruct{pmix_data_type_t})}
\end{arglist}


\littleheader{Free a data array structure}
\declaremacro{PMIX_DATA_ARRAY_FREE}

Release the memory in the array, and then release the \refstruct{pmix_data_array_t} object itself.

\versionMarker{2.2}
\cspecificstart
\begin{codepar}
PMIX_DATA_ARRAY_FREE(m)
\end{codepar}
\cspecificend

\begin{arglist}
\argin{m}{Pointer to the structure to be released (pointer to \refstruct{pmix_data_array_t})}
\end{arglist}

%%%%%%%%%%%%%%%%%%%%%%%%%%%%%%%%%%%%%%%%%%%%%%%%%
\subsection{Argument Array Macros}

The following macros support the construction and release of \code{NULL}-terminated argv arrays of strings.

%%%%
\littleheader{Argument array extension}
\declaremacro{PMIX_ARGV_APPEND}

Append a string to a NULL-terminated, argv-style array of strings.

\cspecificstart
\begin{codepar}
PMIX_ARGV_APPEND(r, a, b);
\end{codepar}
\cspecificend

\begin{arglist}
\argout{r}{Status code indicating success or failure of the operation (\refstruct{pmix_status_t})}
\arginout{a}{Argument list (pointer to NULL-terminated array of strings)}
\argin{b}{Argument to append to the list (string)}
\end{arglist}

This function helps the caller build the \code{argv} portion of \refstruct{pmix_app_t} structure, arrays of keys for querying, or other places where argv-style string arrays are required.

\adviceuserstart
The provided argument is copied into the destination array - thus, the source string can be free'd without affecting the array once the macro has completed.
\adviceuserend

%%%%
\littleheader{Argument array prepend}
\declaremacro{PMIX_ARGV_PREPEND}

Prepend a string to a NULL-terminated, argv-style array of strings.

\cspecificstart
\begin{codepar}
PMIX_ARGV_PREPEND(r, a, b);
\end{codepar}
\cspecificend

\begin{arglist}
\argout{r}{Status code indicating success or failure of the operation (\refstruct{pmix_status_t})}
\arginout{a}{Argument list (pointer to NULL-terminated array of strings)}
\argin{b}{Argument to append to the list (string)}
\end{arglist}

This function helps the caller build the \code{argv} portion of \refstruct{pmix_app_t} structure, arrays of keys for querying, or other places where argv-style string arrays are required.

\adviceuserstart
The provided argument is copied into the destination array - thus, the source string can be free'd without affecting the array once the macro has completed.
\adviceuserend

%%%%%%%%%%%
\littleheader{Argument array extension - unique}
\declaremacro{PMIX_ARGV_APPEND_UNIQUE}

Append a string to a NULL-terminated, argv-style array of strings, but only if the provided argument doesn't already exist somewhere in the array.

\cspecificstart
\begin{codepar}
PMIX_ARGV_APPEND_UNIQUE(r, a, b);
\end{codepar}
\cspecificend

\begin{arglist}
\argout{r}{Status code indicating success or failure of the operation (\refstruct{pmix_status_t})}
\arginout{a}{Argument list (pointer to NULL-terminated array of strings)}
\argin{b}{Argument to append to the list (string)}
\end{arglist}

This function helps the caller build the \code{argv} portion of \refstruct{pmix_app_t} structure, arrays of keys for querying, or other places where argv-style string arrays are required.

\adviceuserstart
The provided argument is copied into the destination array - thus, the source string can be free'd without affecting the array once the macro has completed.
\adviceuserend

%%%%%%%%%%%
\littleheader{Argument array release}
\declaremacro{PMIX_ARGV_FREE}

Free an argv-style array and all of the strings that it contains.

\cspecificstart
\begin{codepar}
PMIX_ARGV_FREE(a);
\end{codepar}
\cspecificend

\begin{arglist}
\argin{a}{Argument list (pointer to NULL-terminated array of strings)}
\end{arglist}

This function releases the array and all of the strings it contains.

%%%%%%%%%%%
\littleheader{Argument array split}
\declaremacro{PMIX_ARGV_SPLIT}

Split a string into a NULL-terminated argv array.

\cspecificstart
\begin{codepar}
PMIX_ARGV_SPLIT(a, b, c);
\end{codepar}
\cspecificend

\begin{arglist}
\argout{a}{Resulting argv-style array (\code{char**})}
\argin{b}{String to be split (\code{char*})}
\argin{c}{Delimiter character (\code{char})}
\end{arglist}

Split an input string into a NULL-terminated argv array. Do not include empty strings in the resulting array.

\adviceuserstart
All strings are inserted into the argv array by value; the newly-allocated array makes no references to the src_string argument (i.e., it can be freed after calling this function without invalidating the output argv array)
\adviceuserend

%%%%%%%%%%%
\littleheader{Argument array join}
\declaremacro{PMIX_ARGV_JOIN}

Join all the elements of an argv array into a single newly-allocated string.

\cspecificstart
\begin{codepar}
PMIX_ARGV_JOIN(a, b, c);
\end{codepar}
\cspecificend

\begin{arglist}
\argout{a}{Resulting string (\code{char*})}
\argin{b}{Argv-style array to be joined (\code{char**})}
\argin{c}{Delimiter character (\code{char})}
\end{arglist}

Join all the elements of an argv array into a single newly-allocated string.

%%%%%%%%%%%
\littleheader{Argument array count}
\declaremacro{PMIX_ARGV_COUNT}

Return the length of a NULL-terminated argv array.

\cspecificstart
\begin{codepar}
PMIX_ARGV_COUNT(r, a);
\end{codepar}
\cspecificend

\begin{arglist}
\argout{r}{Number of strings in the array (integer)}
\argin{a}{Argv-style array (\code{char**})}
\end{arglist}

Count the number of elements in an argv array

%%%%%%%%%%%
\littleheader{Argument array copy}
\declaremacro{PMIX_ARGV_COPY}

Copy an argv array, including copying all of its strings.

\cspecificstart
\begin{codepar}
PMIX_ARGV_COPY(a, b);
\end{codepar}
\cspecificend

\begin{arglist}
\argout{a}{New argv-style array (\code{char**})}
\argin{b}{Argv-style array (\code{char**})}
\end{arglist}

Copy an argv array, including copying all of its strings.


%%%%%%%%%%%%%%%%%%%%%%%%%%%%%%%%%%%%%%%%%%%%%%%%%
\subsection{Set Environment Variable}
\declaremacro{PMIX_SETENV}

%%%%
\summary

Set an environment variable in a \code{NULL}-terminated, env-style array.

\cspecificstart
\begin{codepar}
PMIX_SETENV(r, name, value, env);
\end{codepar}
\cspecificend


\begin{arglist}
\argout{r}{Status code indicating success or failure of the operation (\refstruct{pmix_status_t})}
\argin{name}{Argument name (string)}
\argin{value}{Argument value (string)}
\arginout{env}{Environment array to update (pointer to array of strings)}
\end{arglist}

%%%%
\descr

Similar to \code{setenv} from the C API, this allows the caller to set an environment variable in the specified \code{env} array, which could then be passed to the \refstruct{pmix_app_t} structure or any other destination.

\adviceuserstart
The provided name and value are copied into the destination environment array - thus, the source strings can be free'd without affecting the array once the macro has completed.
\adviceuserend


%%%%%%%%%%%%%%%%%%%%%%%%%%%%%%%%%%%%%%%%%%%%%%%%%
%%%%%%%%%%%%%%%%%%%%%%%%%%%%%%%%%%%%%%%%%%%%%%%%%
\section{Generalized Data Types Used for Packing/Unpacking}
\declarestruct{pmix_data_type_t}

The \refstruct{pmix_data_type_t} structure is a \code{uint16_t} type for identifying the data type for packing/unpacking purposes. New data type values introduced in this version of the Standard are shown in \textbf{\color{magenta}magenta}.

\adviceimplstart
The following constants can be used to set a variable of the type \refstruct{pmix_data_type_t}. Data types in the \ac{PMIx} Standard are defined in terms of the C-programming language. Implementers wishing to support other languages should provide the equivalent definitions in a language-appropriate manner. Additionally, a PMIx implementation may choose to add additional types.
\adviceimplend

\begin{constantdesc}
%
\declareconstitem{PMIX_UNDEF}
Undefined.
%
\declareconstitem{PMIX_BOOL}
Boolean (converted to/from native \code{true}/\code{false}) (\code{bool}).
%
\declareconstitem{PMIX_BYTE}
A byte of data (\code{uint8_t}).
%
\declareconstitem{PMIX_STRING}
\code{NULL} terminated string (\code{char*}).
%
\declareconstitem{PMIX_SIZE}
Size \code{size_t}.
%
\declareconstitem{PMIX_PID}
Operating \ac{PID} (\code{pid_t}).
%
\declareconstitem{PMIX_INT}
Integer (\code{int}).
%
\declareconstitem{PMIX_INT8}
8-byte integer (\code{int8_t}).
%
\declareconstitem{PMIX_INT16}
16-byte integer (\code{int16_t}).
%
\declareconstitem{PMIX_INT32}
32-byte integer (\code{int32_t}).
%
\declareconstitem{PMIX_INT64}
64-byte integer (\code{int64_t}).
%
\declareconstitem{PMIX_UINT}
Unsigned integer (\code{unsigned int}).
%
\declareconstitem{PMIX_UINT8}
Unsigned 8-byte integer (\code{uint8_t}).
%
\declareconstitem{PMIX_UINT16}
Unsigned 16-byte integer (\code{uint16_t}).
%
\declareconstitem{PMIX_UINT32}
Unsigned 32-byte integer (\code{uint32_t}).
%
\declareconstitem{PMIX_UINT64}
Unsigned 64-byte integer (\code{uint64_t}).
%
\declareconstitem{PMIX_FLOAT}
Float (\code{float}).
%
\declareconstitem{PMIX_DOUBLE}
Double (\code{double}).
%
\declareconstitem{PMIX_TIMEVAL}
Time value (\code{struct timeval}).
%
\declareconstitem{PMIX_TIME}
Time (\code{time_t}).
%
\declareconstitem{PMIX_STATUS}
Status code {\refstruct{pmix_status_t}}.
%
\declareconstitem{PMIX_VALUE}
Value (\refstruct{pmix_value_t}).
%
\declareconstitem{PMIX_PROC}
Process (\refstruct{pmix_proc_t}).
%
\declareconstitem{PMIX_APP}
Application context.
%
\declareconstitem{PMIX_INFO}
Info object.
%
\declareconstitem{PMIX_PDATA}
Pointer to data.
%
\declareconstitem{PMIX_BUFFER}
Buffer.
%
\declareconstitem{PMIX_BYTE_OBJECT}
Byte object (\refstruct{pmix_byte_object_t}).
%
\declareconstitem{PMIX_KVAL}
Key/value pair.
%
\declareconstitem{PMIX_PERSIST}
Persistance (\refstruct{pmix_persistence_t}).
%
\declareconstitem{PMIX_POINTER}
Pointer to an object (\code{void*}).
%
\declareconstitem{PMIX_SCOPE}
Scope (\refstruct{pmix_scope_t}).
%
\declareconstitem{PMIX_DATA_RANGE}
Range for data (\refstruct{pmix_data_range_t}).
%
\declareconstitem{PMIX_COMMAND}
PMIx command code (used internally).
%
\declareconstitem{PMIX_INFO_DIRECTIVES}
Directives flag for \refstruct{pmix_info_t} (\refstruct{pmix_info_directives_t}).
%
\declareconstitem{PMIX_DATA_TYPE}
Data type code (\refstruct{pmix_data_type_t}).
%
\declareconstitem{PMIX_PROC_STATE}
Process state (\refstruct{pmix_proc_state_t}).
%
\declareconstitem{PMIX_PROC_INFO}
Process information (\refstruct{pmix_proc_info_t}).
%
\declareconstitem{PMIX_DATA_ARRAY}
Data array (\refstruct{pmix_data_array_t}).
%
\declareconstitem{PMIX_PROC_RANK}
Process rank (\refstruct{pmix_rank_t}).
%
\declareconstitem{PMIX_QUERY}
Query structure (\refstruct{pmix_query_t}).
%
\declareconstitem{PMIX_COMPRESSED_STRING}
String compressed with zlib (\code{char*}).
%
\declareconstitemNEW{PMIX_COMPRESSED_BYTE_OBJECT}
Byte object whose bytes have been compressed with zlib (\code{pmix_byte_object_t}).
%
\declareconstitem{PMIX_ALLOC_DIRECTIVE}
Allocation directive (\refstruct{pmix_alloc_directive_t}).
%
\declareconstitem{PMIX_IOF_CHANNEL}
Input/output forwarding channel (\refstruct{pmix_iof_channel_t}).
%
\declareconstitem{PMIX_ENVAR}
Environmental variable structure (\refstruct{pmix_envar_t}).
%
\declareconstitemNEW{PMIX_COORD}
Structure containing fabric coordinates (\refstruct{pmix_coord_t}).
%
\declareconstitemNEW{PMIX_REGATTR}
Structure supporting attribute registrations (\refstruct{pmix_regattr_t}).
%
\declareconstitemNEW{PMIX_REGEX}
Regular expressions - can be a valid NULL-terminated string or an arbitrary array of bytes.
%
\declareconstitemNEW{PMIX_JOB_STATE}
Job state (\refstruct{pmix_job_state_t}).
%
\declareconstitemNEW{PMIX_LINK_STATE}
Link state (\refstruct{pmix_link_state_t}).
%
\declareconstitemNEW{PMIX_PROC_CPUSET}
Structure containing the binding bitmap of a process (\refstruct{pmix_cpuset_t}).
%
\declareconstitemNEW{PMIX_GEOMETRY}
Geometry structure containing the fabric coordinates of a specified device.(\refstruct{pmix_geometry_t}).
%
\declareconstitemNEW{PMIX_DEVICE_DIST}
Structure containing the minimum and maximum relative distance from the caller to a given fabric device. (\refstruct{pmix_device_distance_t}).
%
\declareconstitemNEW{PMIX_ENDPOINT}
Structure containing an assigned endpoint for a given fabric device. (\refstruct{pmix_endpoint_t}).
%
\declareconstitemNEW{PMIX_TOPO}
Structure containing the topology for a given node. (\refstruct{pmix_topology_t}).
%
\declareconstitemNEW{PMIX_DEVTYPE}
Bitmask containing the types of devices being referenced. (\refstruct{pmix_device_type_t}).
%
\declareconstitemNEW{PMIX_LOCTYPE}
Bitmask describing the relative location of another process. (\refstruct{pmix_locality_t}).
%
\declareconstitemNEW{PMIX_DATA_TYPE_MAX}
A starting point for implementer-specific data types.
Values above this are guaranteed not to conflict with \ac{PMIx} values.
Definitions should always be based on the \refconst{PMIX_DATA_TYPE_MAX} constant and not a specific value as the value of the constant may change.
%
\end{constantdesc}


%%%%%%%%%%%%%%%%%%%%%%%%%%%%%%%%%%%%%%%%%%%%%%%%%
%%%%%%%%%%%%%%%%%%%%%%%%%%%%%%%%%%%%%%%%%%%%%%%%%
\section{General Callback Functions}

PMIx provides blocking and nonblocking versions of most APIs.
In the nonblocking versions, a callback is activated upon completion of the the operation.
This section describes many of those callbacks.

%%%%%%%%%%%%%%%%%%%%%%%%%%%%%%%%%%%%%%%%%%%%%%%%%
\subsection{Release Callback Function}
\declareapi{pmix_release_cbfunc_t}

%%%%
\summary

The \refapi{pmix_release_cbfunc_t} is used by the \refapi{pmix_modex_cbfunc_t} and \refapi{pmix_info_cbfunc_t} operations to indicate that the callback data may be reclaimed/freed by the caller.

%%%%
\format

\versionMarker{1.0}
\cspecificstart
\begin{codepar}
typedef void (*pmix_release_cbfunc_t)
    (void *cbdata);
\end{codepar}
\cspecificend

\begin{arglist}
\arginout{cbdata}{Callback data passed to original API call (memory reference)}
\end{arglist}

%%%%
\descr

Since the data is ``owned'' by the host server, provide a callback function to notify the host server that we are done with the data so it can be released.


%%%%%%%%%%%%%%%%%%%%%%%%%%%%%%%%%%%%%%%%%%%%%%%%%
\subsection{Op Callback Function}
\declareapi{pmix_op_cbfunc_t}

%%%%
\summary

The \refapi{pmix_op_cbfunc_t} is used by operations that simply return a status.

\versionMarker{1.0}
\cspecificstart
\begin{codepar}
typedef void (*pmix_op_cbfunc_t)
    (pmix_status_t status, void *cbdata);
\end{codepar}
\cspecificend

\begin{arglist}
\argin{status}{Status associated with the operation (handle)}
\argin{cbdata}{Callback data passed to original API call (memory reference)}
\end{arglist}

%%%%
\descr

Used by a wide range of \ac{PMIx} API's including \refapi{PMIx_Fence_nb}, \refapi{pmix_server_client_connected2_fn_t}, \refapi{PMIx_server_register_nspace}.
This callback function is used to return a status to an often nonblocking operation.


%%%%%%%%%%%%%%%%%%%%%%%%%%%%%%%%%%%%%%%%%%%%%%%%%
\subsection{Value Callback Function}
\declareapi{pmix_value_cbfunc_t}

%%%%
\summary

The \refapi{pmix_value_cbfunc_t} is used by \refapi{PMIx_Get_nb} to return data.

\versionMarker{1.0}
\cspecificstart
\begin{codepar}
typedef void (*pmix_value_cbfunc_t)
    (pmix_status_t status,
     pmix_value_t *kv, void *cbdata);
\end{codepar}
\cspecificend

\begin{arglist}
\argin{status}{Status associated with the operation (handle)}
\argin{kv}{Key/value pair representing the data (\refstruct{pmix_value_t})}
\argin{cbdata}{Callback data passed to original API call (memory reference)}
\end{arglist}


%%%%
\descr

A callback function for calls to \refapi{PMIx_Get_nb}.
The \refarg{status} indicates if the requested data was found or not.
A pointer to the \refstruct{pmix_value_t} structure containing the found data is returned.
The pointer will be \code{NULL} if the requested data was not found.


%%%%%%%%%%%%%%%%%%%%%%%%%%%%%%%%%%%%%%%%%%%%%%%%%
\subsection{Info Callback Function}
\declareapi{pmix_info_cbfunc_t}

%%%%
\summary

The \refapi{pmix_info_cbfunc_t} is a general information callback used by various APIs.

\versionMarker{2.0}
\cspecificstart
\begin{codepar}
typedef void (*pmix_info_cbfunc_t)
    (pmix_status_t status,
     pmix_info_t info[], size_t ninfo,
     void *cbdata,
     pmix_release_cbfunc_t release_fn,
     void *release_cbdata);
\end{codepar}
\cspecificend

\begin{arglist}
\argin{status}{Status associated with the operation (\refstruct{pmix_status_t})}
\argin{info}{Array of \refstruct{pmix_info_t} returned by the operation (pointer)}
\argin{ninfo}{Number of elements in the \argref{info} array (\code{size_t})}
\argin{cbdata}{Callback data passed to original API call (memory reference)}
\argin{release_fn}{Function to be called when done with the \argref{info} data (function pointer)}
\argin{release_cbdata}{Callback data to be passed to \argref{release_fn} (memory reference)}
\end{arglist}


%%%%
\descr

The \refarg{status} indicates if requested data was found or not.
An array of \refstruct{pmix_info_t} will contain the key/value pairs.

%%%%%%%%%%%
\subsection{Handler registration callback function}
\declareapi{pmix_hdlr_reg_cbfunc_t}

%%%%
\summary

Callback function for calls to register handlers, e.g., event notification and IOF requests.

%%%%
\format

\versionMarker{3.0}
\cspecificstart
\begin{codepar}
typedef void (*pmix_hdlr_reg_cbfunc_t)
    (pmix_status_t status,
     size_t refid,
     void *cbdata);
\end{codepar}
\cspecificend

\begin{arglist}
\argin{status}{\refconst{PMIX_SUCCESS} or an appropriate error constant (\refstruct{pmix_status_t})}
\argin{refid}{reference identifier assigned to the handler by PMIx, used to deregister the handler (\code{size_t})}
\argin{cbdata}{object provided to the registration call (pointer)}
\end{arglist}

%%%%
\descr

Callback function for calls to register handlers, e.g., event notification and IOF requests.


%%%%%%%%%%%%%%%%%%%%%%%%%%%%%%%%%%%%%%%%%%%%%%%%%
%%%%%%%%%%%%%%%%%%%%%%%%%%%%%%%%%%%%%%%%%%%%%%%%%
\section{PMIx Datatype Value String Representations}

Provide a string representation for several types of values.
Note that the provided string is statically defined and must NOT be \code{free}'d.

%%%%
\summary
\declareapi{PMIx_Error_string}

String representation of a \refstruct{pmix_status_t}.

\versionMarker{1.0}
\cspecificstart
\begin{codepar}
const char*
PMIx_Error_string(pmix_status_t status);
\end{codepar}
\cspecificend

%%%%
\summary
\declareapi{PMIx_Proc_state_string}

String representation of a \refstruct{pmix_proc_state_t}.

\versionMarker{2.0}
\cspecificstart
\begin{codepar}
const char*
PMIx_Proc_state_string(pmix_proc_state_t state);
\end{codepar}
\cspecificend

%%%%
\summary
\declareapi{PMIx_Scope_string}

String representation of a \refstruct{pmix_scope_t}.

\versionMarker{2.0}
\cspecificstart
\begin{codepar}
const char*
PMIx_Scope_string(pmix_scope_t scope);
\end{codepar}
\cspecificend

%%%%
\summary
\declareapi{PMIx_Persistence_string}

String representation of a \refstruct{pmix_persistence_t}.

\versionMarker{2.0}
\cspecificstart
\begin{codepar}
const char*
PMIx_Persistence_string(pmix_persistence_t persist);
\end{codepar}
\cspecificend

%%%%
\summary
\declareapi{PMIx_Data_range_string}

String representation of a \refstruct{pmix_data_range_t}.

\versionMarker{2.0}
\cspecificstart
\begin{codepar}
const char*
PMIx_Data_range_string(pmix_data_range_t range);
\end{codepar}
\cspecificend

%%%%
\summary
\declareapi{PMIx_Info_directives_string}

String representation of a \refstruct{pmix_info_directives_t}.

\versionMarker{2.0}
\cspecificstart
\begin{codepar}
const char*
PMIx_Info_directives_string(pmix_info_directives_t directives);
\end{codepar}
\cspecificend

%%%%
\summary
\declareapi{PMIx_Data_type_string}

String representation of a \refstruct{pmix_data_type_t}.

\versionMarker{2.0}
\cspecificstart
\begin{codepar}
const char*
PMIx_Data_type_string(pmix_data_type_t type);
\end{codepar}
\cspecificend

%%%%
\summary
\declareapi{PMIx_Alloc_directive_string}

String representation of a \refstruct{pmix_alloc_directive_t}.

\versionMarker{2.0}
\cspecificstart
\begin{codepar}
const char*
PMIx_Alloc_directive_string(pmix_alloc_directive_t directive);
\end{codepar}
\cspecificend

%%%%
\summary
\declareapi{PMIx_IOF_channel_string}

String representation of a \refstruct{pmix_iof_channel_t}.

\versionMarker{3.0}
\cspecificstart
\begin{codepar}
const char*
PMIx_IOF_channel_string(pmix_iof_channel_t channel);
\end{codepar}
\cspecificend

%%%%
\summary
\declareapi{PMIx_Job_state_string}

String representation of a \refstruct{pmix_job_state_t}.

\versionMarker{4.0}
\cspecificstart
\begin{codepar}
const char*
PMIx_Job_state_string(pmix_job_state_t state);
\end{codepar}
\cspecificend

%%%%
\summary
\declareapi{PMIx_Get_attribute_string}

String representation of a \ac{PMIx} attribute.

\versionMarker{4.0}
\cspecificstart
\begin{codepar}
const char*
PMIx_Get_attribute_string(char *attributename);
\end{codepar}
\cspecificend

%%%%
\summary
\declareapi{PMIx_Get_attribute_name}

Return the \ac{PMIx} attribute name corresponding to the given attribute string.

\versionMarker{4.0}
\cspecificstart
\begin{codepar}
const char*
PMIx_Get_attribute_name(char *attributestring);
\end{codepar}
\cspecificend

%%%%
\summary
\declareapi{PMIx_Link_state_string}

String representation of a \refstruct{pmix_link_state_t}.

\versionMarker{4.0}
\cspecificstart
\begin{codepar}
const char*
PMIx_Link_state_string(pmix_link_state_t state);
\end{codepar}
\cspecificend

%%%%
\summary
\declareapi{PMIx_Device_type_string}

String representation of a \refstruct{pmix_device_type_t}.

\versionMarker{4.0}
\cspecificstart
\begin{codepar}
const char*
PMIx_Device_type_string(pmix_device_type_t type);
\end{codepar}
\cspecificend


%%%%%%%%%%%%%%%%%%%%%%%%%%%%%%%%%%%%%%%%%%%%%%%%%


    % Initialization & Finalization
    %  - Client, Server, Tool interfaces
    %%%%%%%%%%%%%%%%%%%%%%%%%%%%%%%%%%%%%%%%%%%%%%%%%
% Chapter: Initialization & Finalization
%%%%%%%%%%%%%%%%%%%%%%%%%%%%%%%%%%%%%%%%%%%%%%%%%
\chapter{Initialization and Finalization}
\label{chap:api_init}

% RALPH

The \ac{PMIx} library is required to be initialized and finalized around the usage of most of the \acp{API}.
The \acp{API} that may be used outside of the initialized and finalized region are noted.
All other \acp{API} must be used inside this region.

There are three sets of initialization and finalization functions depending upon the role of the process in the \ac{PMIx} universe.
Each of these functional sets are described in this chapter. Note that a process can only call \textit{one} of the
init/finalize functional pairs - e.g., a process that calls the client init function cannot also call the tool or server
init functions, and must call the corresponding client finalize.

\adviceuserstart
Processes that initialize as a server or tool automatically are given access to all client \acp{API}. Server initialization
includes setting up the infrastructure to support local clients - thus, it necessarily includes overhead and an increased
memory footprint. Tool initialization automatically searches for a server to which it can connect --- if declared as a
\textit{launcher}, the \ac{PMIx} library sets up the required ``hooks'' for other tools (e.g., debuggers) to attach to it.
\adviceuserend


%%%%%%%%%%%%%%%%%%%%%%%%%%%%%%%%%%%%%%%%%%%%%%
%%%%%%%%%%%%%%%%%%%%%%%%%%%%%%%%%%%%%%%%%%%%%%
\section{Query}
\label{chap:api_init:general}

The APIs defined in this section can be used by any PMIx process, regardless of their role in the PMIx universe.

%%%%%%%%%%%
\subsection{\code{PMIx_Initialized}}
\declareapi{PMIx_Initialized}

%%%%
\format

\cspecificstart
\begin{codepar}
int PMIx_Initialized(void)
\end{codepar}
\cspecificend

A value of \code{1} (true) will be returned if the PMIx library has been initialized, and \code{0} (false) otherwise.

\rationalestart
The return value is an integer for historical reasons as that was the signature of prior PMI libraries.
\rationaleend

%%%%
\descr

Check to see if the PMIx library has been initialized using any of the init functions:
\refapi{PMIx_Init}, \refapi{PMIx_server_init}, or \refapi{PMIx_tool_init}.

%%%%%%%%%%%
\subsection{\code{PMIx_Get_version}}
\declareapi{PMIx_Get_version}

%%%%
\summary

Get the PMIx version information.

%%%%
\format

\cspecificstart
\begin{codepar}
const char* PMIx_Get_version(void)
\end{codepar}
\cspecificend

%%%%
\descr

Get the \ac{PMIx} version string.
Note that the provided string is statically defined and must \textit{not} be free'd.

%%%%%%%%%%%%%%%%%%%%%%%%%%%%%%%%%%%%%%%%%%%%%%
%%%%%%%%%%%%%%%%%%%%%%%%%%%%%%%%%%%%%%%%%%%%%%
\section{Client Initialization and Finalization}
\label{chap:api_init:client}

Initialization and finalization routines for \ac{PMIx} clients.

%%%%%%%%%%%
\subsection{\code{PMIx_Init}}
\declareapi{PMIx_Init}

%%%%
\summary

Initialize the \ac{PMIx} client.

%%%%
\format

\cspecificstart
\begin{codepar}
pmix_status_t
PMIx_Init(pmix_proc_t *proc,
          pmix_info_t info[], size_t ninfo)
\end{codepar}
\cspecificend

\begin{arglist}
\arginout{proc}{proc structure (handle)}
\argin{info}{Array of \refattr{pmix_info_t} structures (array of handles)}
\argin{ninfo}{Number of element in the \refarg{info} array (\code{size_t})}
\end{arglist}

Returns \refconst{PMIX_SUCCESS} or a negative value corresponding to a \ac{PMIx} error constant.

\priattr
The following attributes are supported in the \ac{PRI}:

\pasteAttributeItem{PMIX_EVENT_BASE}
\pasteAttributeItemBegin{PMIX_GDS_MODULE} This attribute controls only the selection of GDS module for internal use by the process. Module selection for interacting with the server is performed dynamically during the connection process.
\pasteAttributeItemEnd{}

%%%%
\descr

Initialize the \ac{PMIx} client, returning the process identifier assigned to this client's application in the provided \refstruct{pmix_proc_t} struct.
Passing a value of \code{NULL} for this parameter is allowed if the user wishes solely to initialize the \ac{PMIx} system and does not require return of the identifier at that time.

When called, the \ac{PMIx} client shall check for the required connection information of the local \ac{PMIx} server and establish the connection.
If the information is not found, or the server connection fails, then an appropriate error constant shall be returned.

If successful, the function shall return \refconst{PMIX_SUCCESS} and fill the \refarg{proc} structure (if provided) with the server-assigned namespace and rank of the process within the application.
In addition, all startup information provided by the resource manager shall be made available to the client process via subsequent calls to \refapi{PMIx_Get}.

The \ac{PMIx} client library shall be reference counted, and so multiple calls to \refapi{PMIx_Init} are allowed by the standard.
Thus, one way for an application process to obtain its namespace and rank is to simply call \refapi{PMIx_Init} with a non-NULL \refarg{proc} parameter.
Note that each call to \refapi{PMIx_Init} must be balanced with a call to \refapi{PMIx_Finalize} to maintain the reference count.

Each call to \refapi{PMIx_Init} may contain an array of \refstruct{pmix_info_t} structures passing directives to the \ac{PMIx} client library as per the above attributes.

Multiple calls to \refapi{PMIx_Init} shall not include conflicting directives.
The \refapi{PMIx_Init} function will return an error when directives that conflict with prior directives are encountered.


%%%%%%%%%%%
\subsection{\code{PMIx_Finalize}}
\declareapi{PMIx_Finalize}

%%%%
\summary

Finalize the PMIx client library.

%%%%
\format

\cspecificstart
\begin{codepar}
pmix_status_t
PMIx_Finalize(const pmix_info_t info[], size_t ninfo)
\end{codepar}
\cspecificend

\begin{arglist}
\argin{info}{Array of \refattr{pmix_info_t} structures (array of handles)}
\argin{ninfo}{Number of element in the \refarg{info} array (\code{size_t})}
\end{arglist}

Returns \refconst{PMIX_SUCCESS} or a negative value corresponding to a PMIx error constant.

\priattr
The following attributes are supported in the \ac{PRI}:

\pasteAttributeItem{PMIX_EMBED_BARRIER}

%%%%
\descr

Decrement the \ac{PMIx} client library reference count.
When the reference count reaches zero, the library will finalize the \ac{PMIx} client, closing the connection with the local \ac{PMIx} server and releasing all internally allocated memory.


%%%%%%%%%%%%%%%%%%%%%%%%%%%%%%%%%%%%%%%%%%%%%%
%%%%%%%%%%%%%%%%%%%%%%%%%%%%%%%%%%%%%%%%%%%%%%
\section{Tool Initialization and Finalization}
\label{chap:api_init:tool}

Initialization and finalization routines for \ac{PMIx} tools.

%%%%%%%%%%%
\subsection{\code{PMIx_tool_init}}
\declareapi{PMIx_tool_init}

%%%%
\summary

Initialize the \ac{PMIx} library for operating as a tool.

%%%%
\format

\cspecificstart
\begin{codepar}
pmix_status_t
PMIx_tool_init(pmix_proc_t *proc,
               pmix_info_t info[], size_t ninfo)
\end{codepar}
\cspecificend

\begin{arglist}
\arginout{proc}{\refstruct{pmix_proc_t} structure (handle)}
\argin{info}{Array of \refattr{pmix_info_t} structures (array of handles)}
\argin{ninfo}{Number of element in the \refarg{info} array (\code{size_t})}
\end{arglist}

Returns \refconst{PMIX_SUCCESS} or a negative value corresponding to a PMIx error constant.

\priattr
The following attributes are supported in the \ac{PRI}:

\pasteAttributeItem{PMIX_GDS_MODULE}
\pasteAttributeItem{PMIX_TOOL_NSPACE}
\pasteAttributeItem{PMIX_TOOL_RANK}
\pasteAttributeItem{PMIX_TOOL_DO_NOT_CONNECT}
\pasteAttributeItem{PMIX_CONNECT_TO_SYSTEM}
\pasteAttributeItem{PMIX_CONNECT_SYSTEM_FIRST}
\pasteAttributeItem{PMIX_SERVER_PIDINFO}
\pasteAttributeItem{PMIX_SERVER_URI}
\pasteAttributeItem{PMIX_TCP_URI}
\pasteAttributeItem{PMIX_CONNECT_RETRY_DELAY}
\pasteAttributeItem{PMIX_CONNECT_MAX_RETRIES}

%%%%
\descr

Initialize the \ac{PMIx} tool, returning the process identifier assigned to this tool in the provided \refstruct{pmix_proc_t} struct. The \refarg{info} array is used to pass user requests pertaining to the init and subsequent operations. Passing a \code{NULL} value for the array pointer is supported if no directives are desired.

If called with the \refattr{PMIX_TOOL_DO_NOT_CONNECT} attribute, the \ac{PMIx} tool library will fully initialize but not attempt to connect to a \ac{PMIx} server. The tool can connect to a server at a later point in time, if desired. In all other cases, the tool library will attempt to connect to according to the following precedence chain:

\begin{itemize}
    \item if \refattr{PMIX_SERVER_URI} or \refattr{PMIX_TCP_URI} is given, then connection will be attempted to the server at the specified \ac{URI}. Note that it is an error for both of these attributes to be specified. \refattr{PMIX_SERVER_URI} is the preferred method as it is more generalized --- \refattr{PMIX_TCP_URI} is provided for those cases where the user specifically wants to use a TCP transport for the connection and wants to error out if it isn't available or cannot succeed. The \ac{PMIx} library will return an error if connection fails --- it will not proceed to check for other connection options as the user specified a particular one to use
    \item if \refattr{PMIX_SERVER_PIDINFO} was provided, then the tool will search under the directory provided by the PMIX\_SERVER\_TMPDIR environmental variable for a rendezvous file created by the process corresponding to that \ac{PID}. The \ac{PMIx} library will return an error if the rendezvous file cannot be found, or the connection is refused by the server
    \item if \refattr{PMIX_CONNECT_TO_SYSTEM} is given, then the tool will search for a system-level rendezvous file created by a \ac{PMIx} server in the directory specified by the PMIX\_SYSTEM\_TMPDIR environmental variable. If found, then the tool will attempt to connect to it. An error is returned if the rendezvous file cannot be found or the connection is refused.
    \item if \refattr{PMIX_CONNECT_SYSTEM_FIRST} is given, then the tool will search for a system-level rendezvous file created by a \ac{PMIx} server in the directory specified by the PMIX\_SYSTEM\_TMPDIR environmental variable. If found, then the tool will attempt to connect to it. In this case, no error will be returned if the rendezvous file is not found or connection is refused --- the library will silently continue to the next option
    \item by default, the tool will search the directory tree under the directory provided by the PMIX\_SERVER\_TMPDIR environmental variable for rendezvous files of \ac{PMIx} servers, attempting to connect to each it finds until one accepts the connection. If no rendezvous files are found, or all contacted servers refuse connection, then the library will return an error.
\end{itemize}

If successful, the function will return \refconst{PMIX_SUCCESS} and will fill the provided structure (if provided) with the server-assigned namespace and rank of the tool. Note that each connection attempt in the above precedence chain will retry (with delay between each retry) a number of times according to the values of the corresponding attributes. Default is no retries.

Note that the \ac{PMIx} tool library is referenced counted, and so multiple calls to \refapi{PMIx_tool_init} are allowed.
Thus, one way to obtain the namespace and rank of the process is to simply call \refapi{PMIx_tool_init} with a non-NULL parameter.


%%%%%%%%%%%
\subsection{\code{PMIx_tool_finalize}}
\declareapi{PMIx_tool_finalize}

%%%%
\summary

Finalize the \ac{PMIx} library for a tool connection.

%%%%
\format

\cspecificstart
\begin{codepar}
pmix_status_t
PMIx_tool_finalize(void)
\end{codepar}
\cspecificend

Returns \refconst{PMIX_SUCCESS} or a negative value corresponding to a PMIx error constant.

%%%%
\descr

Finalize the PMIx tool library, closing the connection to the server.
An error code will be returned if, for some reason, the connection cannot be cleanly terminated --- in this case, the connection is dropped.


%%%%%%%%%%%%%%%%%%%%%%%%%%%%%%%%%%%%%%%%%%%%%%
%%%%%%%%%%%%%%%%%%%%%%%%%%%%%%%%%%%%%%%%%%%%%%
\section{Server Initialization and Finalization}
\label{chap:api_init:server}

Initialization and finalization routines for \ac{PMIx} servers.

%%%%%%%%%%%
\subsection{\code{PMIx_server_init}}
\declareapi{PMIx_server_init}

%%%%
\summary

Initialize the \ac{PMIx} server.

%%%%
\format

\cspecificstart
\begin{codepar}
pmix_status_t
PMIx_server_init(pmix_server_module_t *module,
                 pmix_info_t info[], size_t ninfo)
\end{codepar}
\cspecificend

\begin{arglist}
\arginout{module}{\refstruct{pmix_server_module_t} structure (handle)}
\argin{info}{Array of \refattr{pmix_info_t} structures (array of handles)}
\argin{ninfo}{Number of elements in the \refarg{info} array (\code{size_t})}
\end{arglist}

Returns \refconst{PMIX_SUCCESS} or a negative value corresponding to a PMIx error constant.

\priattr
The following attributes are supported in the \ac{PRI}:

\pasteAttributeItem{PMIX_SERVER_NSPACE}
\pasteAttributeItem{PMIX_SERVER_RANK}
\pasteAttributeItem{PMIX_SERVER_TMPDIR}
\pasteAttributeItem{PMIX_SYSTEM_TMPDIR}
\pasteAttributeItem{PMIX_SERVER_TOOL_SUPPORT}
\pasteAttributeItem{PMIX_SERVER_SYSTEM_SUPPORT}
\pasteAttributeItem{PMIX_TCP_IF_INCLUDE}
\pasteAttributeItem{PMIX_TCP_IPV4_PORT}
\pasteAttributeItem{PMIX_TCP_IPV6_PORT}
\pasteAttributeItem{PMIX_TCP_DISABLE_IPV4}
\pasteAttributeItem{PMIX_TCP_DISABLE_IPV6}
\pasteAttributeItem{PMIX_SERVER_REMOTE_CONNECTIONS}
\pasteAttributeItem{PMIX_TCP_REPORT_URI}
\pasteAttributeItem{PMIX_USOCK_DISABLE}


%%%%
\descr

Initialize the server support library, and provide a pointer to a \refapi{pmix_server_module_t} structure containing the caller's callback functions.
The array of \refstruct{pmix_info_t} structs is used to pass additional info that may be required by the server when initializing.
For example, it may include the \refconst{PMIX_SERVER_TOOL_SUPPORT} key, thereby indicating that the daemon is willing to accept connection requests from tools.

\adviceuserstart
Providing a value of \code{NULL} for the \refarg{module} argument is not permitted - the host must support at least one server callback function.
\adviceuserend

%%%%%%%%%%%
\subsection{\code{PMIx_server_finalize}}
\declareapi{PMIx_server_finalize}

%%%%
\summary

Finalize the PMIx server library.

%%%%
\format

\cspecificstart
\begin{codepar}
pmix_status_t
PMIx_server_finalize(void)
\end{codepar}
\cspecificend

Returns \refconst{PMIX_SUCCESS} or a negative value corresponding to a PMIx error constant.

%%%%
\descr

Finalize the server support library, terminating all connections to attached tools and any local clients.
All memory usage is released.

%%%%%%%%%%%%%%%%%%%%%%%%%%%%%%%%%%%%%%%%%%%%%%%%%


    % Key/Value Management
    %  - put, get, commit, fence, (un)publish, lookup
    %%%%%%%%%%%%%%%%%%%%%%%%%%%%%%%%%%%%%%%%%%%%%%%%%
% Chapter: Key/Value Management
%%%%%%%%%%%%%%%%%%%%%%%%%%%%%%%%%%%%%%%%%%%%%%%%%
\chapter{Key/Value Management}
\label{chap:api_kv_mgmt}

Management of key-value pairs in \ac{PMIx} is a distributed responsibility. While the stated objective of the \ac{PMIx} community is to eliminate collective operations, it is recognized that the traditional method of posting/exchanging data must be supported until that objective can be met. This method relies on processes to discover and post their local information which is collected by the local PMIx server library. Global exchange of the posted information is then executed via a collective operation performed by the host \ac{SMS} servers. The \refapi{PMIx_Put} and \refapi{PMIx_Commit} \acp{API}, plus an attribute directing \refapi{PMIx_Fence} to globally collect the data posted by processes, are provided for this purpose.

%%%%%%%%%%%%%%%%%%%%%%%%%%%%%%%%%%%%%%%%%%%%%%
%%%%%%%%%%%%%%%%%%%%%%%%%%%%%%%%%%%%%%%%%%%%%%
\section{Setting and Accessing Key/Value Pairs}
\label{chap:api_kv_mgmt:access}


%%%%%%%%%%%
\subsection{\code{PMIx_Put}}
\declareapi{PMIx_Put}

%%%%
\summary

Push a key/value pair into the client's namespace.

%%%%
\format

\versionMarker{1.0}
\cspecificstart
\begin{codepar}
pmix_status_t
PMIx_Put(pmix_scope_t scope,
         const pmix_key_t key,
         pmix_value_t *val)
\end{codepar}
\cspecificend

\begin{arglist}
\argin{scope}{Distribution scope of the provided value (handle)}
\argin{key}{key (\refstruct{pmix_key_t})}
\argin{value}{Reference to a \refstruct{pmix_value_t} structure (handle)}
\end{arglist}

Returns \refconst{PMIX_SUCCESS} or a negative value corresponding to a PMIx error constant.

%%%%
\descr

Push a value into the client's namespace.
The client's \ac{PMIx} library will cache the information locally until \refapi{PMIx_Commit} is called.

The provided \refarg{scope} is passed to the local PMIx server, which will distribute the data to other processes according to the provided scope.
The \refstruct{pmix_scope_t} values are defined in \specrefstruct{pmix_scope_t}.
Specific implementations may support different scope values, but all implementations must support at least \code{PMIX_GLOBAL}.

The \refstruct{pmix_value_t} structure supports both string and binary values.
PMIx implementations will support heterogeneous environments by properly converting binary values between host architectures, and will copy the provided \refarg{value} into internal memory.

\adviceimplstart
The PMIx server library will properly pack/unpack data to accommodate heterogeneous environments. The host \ac{SMS} is not involved in this action. The \refarg{value} argument must be copied - the caller is free to release it following return from the function.
\adviceimplend

\adviceuserstart
The value is copied by the PMIx client library. Thus, the application is free to release and/or modify the value once the call to \refapi{PMIx_Put} has completed.

Note that keys starting with a string of ``\code{pmix}'' are exclusively reserved for the \ac{PMIx} standard and must not be used in calls to \refapi{PMIx_Put}. Thus, applications should never use a defined ``PMIX_'' attribute as the key in a call to \refapi{PMIx_Put}.
\adviceuserend


%%%%%%%%%%%
\subsection{\code{PMIx_Get}}
\declareapi{PMIx_Get}

%%%%
\summary

Retrieve a key/value pair from the client's namespace.

%%%%
\format

\versionMarker{1.0}
\cspecificstart
\begin{codepar}
pmix_status_t
PMIx_Get(const pmix_proc_t *proc, const pmix_key_t key,
         const pmix_info_t info[], size_t ninfo,
         pmix_value_t **val)
\end{codepar}
\cspecificend

\begin{arglist}
\argin{proc}{process reference (handle)}
\argin{key}{key to retrieve (\refstruct{pmix_key_t})}
\argin{info}{Array of info structures (array of handles)}
\argin{ninfo}{Number of element in the \refarg{info} array (integer)}
\argout{val}{value (handle)}
\end{arglist}

Returns \refconst{PMIX_SUCCESS} or a negative value corresponding to a PMIx error constant.

\reqattrstart
The following attributes are required to be supported by all \ac{PMIx} libraries:

\pastePRIAttributeItem{PMIX_OPTIONAL}
\pastePRIAttributeItem{PMIX_IMMEDIATE}
\pastePRIAttributeItem{PMIX_DATA_SCOPE}
\pastePRIAttributeItem{PMIX_SESSION_INFO}
\pastePRIAttributeItem{PMIX_JOB_INFO}
\pastePRIAttributeItem{PMIX_APP_INFO}
\pastePRIAttributeItem{PMIX_NODE_INFO}
\pastePRIAttributeItemBegin{PMIX_GET_STATIC_VALUES}
and indicate that the address provided for the return value points to a statically defined memory location. Returned non-pointer values should therefore be copied directly into the provided memory. Pointers in the returned value should point directly to values in the key-value store. User is responsible for \emph{not} releasing memory on any returned pointer value. Note that a return status of \refconst{PMIX_ERR_GET_MALLOC_REQD} indicates that direct pointers could not be supported - thus, the returned data contains allocated memory that the user must release.
\pastePRIAttributeItemEnd

\reqattrend

\optattrstart
The following attributes are optional for host environments:

\pastePRRTEAttributeItem{PMIX_TIMEOUT}

\optattrend

\adviceimplstart
We recommend that implementation of the \refattr{PMIX_TIMEOUT} attribute be left to the host environment due to race condition considerations between delivery of the data by the host environment versus internal timeout in the \ac{PMIx} server library. Implementers that choose to support \refattr{PMIX_TIMEOUT} directly in the \ac{PMIx} server library must take care to resolve the race condition and should avoid passing \refattr{PMIX_TIMEOUT} to the host environment so that multiple competing timeouts are not created.
\adviceimplend

%%%%
\descr

Retrieve information for the specified \refarg{key} as published by the process identified in the given \refstruct{pmix_proc_t}, returning a pointer to the value in the given address.

This is a blocking operation - the caller will block until either the specified data becomes available from the specified rank in the \refarg{proc} structure or the operation times out should the \refattr{PMIX_TIMEOUT} attribute have been given.
The caller is responsible for freeing all memory associated with the returned \refarg{value} when no longer required.

The \refarg{info} array is used to pass user requests regarding the get operation.

\adviceuserstart
Information provided by the \ac{PMIx} server at time of process start is accessed by providing the namespace of the job with the rank set to \refconst{PMIX_RANK_WILDCARD}. The list of data referenced in this way is maintained on the \ac{PMIx} web site at \url{https://pmix.org/support/faq/wildcard-rank-access/} but includes items such as the number of processes in the namespace (\refattr{PMIX_JOB_SIZE}), total available slots in the allocation (\refattr{PMIX_UNIV_SIZE}), and the number of nodes in the allocation (\refattr{PMIX_NUM_NODES}).

Data posted by a process via \refapi{PMIx_Put} needs to be retrieved by specifying the rank of the posting process. All other information is retrievable using a rank of \refconst{PMIX_RANK_WILDCARD} when the information being retrieved refers to something non-rank specific (e.g., number of processes on a node, number of processes in a job), and using the rank of the relevant process when requesting information that is rank-specific (e.g., the \ac{URI} of the process, or the node upon which it is executing). Each subsection of Section \ref{api:struct:attributes} indicates the appropriate rank value for referencing the defined attribute.
\adviceuserend

%%%%%%%%%%%
\subsection{\code{PMIx_Get_nb}}
\declareapi{PMIx_Get_nb}

%%%%
\summary

Nonblocking \refapi{PMIx_Get} operation.

%%%%
\format

\versionMarker{1.0}
\cspecificstart
\begin{codepar}
pmix_status_t
PMIx_Get_nb(const pmix_proc_t *proc, const char key[],
            const pmix_info_t info[], size_t ninfo,
            pmix_value_cbfunc_t cbfunc, void *cbdata)
\end{codepar}
\cspecificend

\begin{arglist}
\argin{proc}{process reference (handle)}
\argin{key}{key to retrieve (string)}
\argin{info}{Array of info structures (array of handles)}
\argin{ninfo}{Number of elements in the \refarg{info} array (integer)}
\argin{cbfunc}{Callback function (function reference)}
\argin{cbdata}{Data to be passed to the callback function (memory reference)}
\end{arglist}

Returns one of the following:

\begin{itemize}
    \item \refconst{PMIX_SUCCESS}, indicating that the request is being processed by the host environment - result will be returned in the provided \refarg{cbfunc}. Note that the library must not invoke the callback function prior to returning from the \ac{API}.
    \item \refconst{PMIX_OPERATION_SUCCEEDED}, indicating that the request was immediately processed and returned \textit{success} - the \refarg{cbfunc} will \textit{not} be called
    \item a PMIx error constant indicating either an error in the input or that the request was immediately processed and failed - the \refarg{cbfunc} will \textit{not} be called
\end{itemize}

If executed, the status returned in the provided callback function will be one of the following constants:

\begin{itemize}
\item \refconst{PMIX_SUCCESS} The requested data has been returned
\item \refconst{PMIX_ERR_NOT_FOUND} The requested data was not available
\item a non-zero \ac{PMIx} error constant indicating a reason for the request's failure
\end{itemize}

\reqattrstart
The following attributes are required to be supported by all \ac{PMIx} libraries:

\pastePRIAttributeItem{PMIX_OPTIONAL}
\pastePRIAttributeItem{PMIX_IMMEDIATE}
\pastePRIAttributeItem{PMIX_DATA_SCOPE}
\pastePRIAttributeItem{PMIX_SESSION_INFO}
\pastePRIAttributeItem{PMIX_JOB_INFO}
\pastePRIAttributeItem{PMIX_APP_INFO}
\pastePRIAttributeItem{PMIX_NODE_INFO}
\pastePRIAttributeItemBegin{PMIX_GET_STATIC_VALUES}
and indicate that user takes responsibility for properly releasing memory on the returned value (i.e., free'ing the value structure but not the pointer fields). Note that a return status of \refconst{PMIX_ERR_GET_MALLOC_REQD} indicates that direct pointers could not be supported - thus, the returned data contains allocated memory that the user must release.
\pastePRIAttributeItemEnd

\reqattrend

\optattrstart
The following attributes are optional for host environments that support this operation:

\pastePRRTEAttributeItem{PMIX_TIMEOUT}

\optattrend

\adviceimplstart
We recommend that implementation of the \refattr{PMIX_TIMEOUT} attribute be left to the host environment due to race condition considerations between delivery of the data by the host environment versus internal timeout in the \ac{PMIx} server library. Implementers that choose to support \refattr{PMIX_TIMEOUT} directly in the \ac{PMIx} server library must take care to resolve the race condition and should avoid passing \refattr{PMIX_TIMEOUT} to the host environment so that multiple competing timeouts are not created.
\adviceimplend

%%%%
\descr

The callback function will be executed once the specified data becomes available from the identified process and retrieved by the local server.
The \argref{info} array is used as described by the \refapi{PMIx_Get} routine.

\adviceuserstart
Information provided by the \ac{PMIx} server at time of process start is accessed by providing the namespace of the job with the rank set to \refconst{PMIX_RANK_WILDCARD}. Attributes referenced in this way are identified in \ref{api:struct:attributes} but includes items such as the number of processes in the namespace (\refattr{PMIX_JOB_SIZE}), total available slots in the allocation (\refattr{PMIX_UNIV_SIZE}), and the number of nodes in the allocation (\refattr{PMIX_NUM_NODES}).

In general, data posted by a process via \refapi{PMIx_Put} and data that refers directly to a process-related value needs to be retrieved by specifying the rank of the posting process. All other information is retrievable using a rank of \refconst{PMIX_RANK_WILDCARD}, as illustrated in \ref{chap:api_kv:getex}. See \ref{api:struct:attributes:retrieval} for an explanation regarding use of the \emph{level} attributes.
\adviceuserend


%%%%%%%%%%%
\subsection{\code{PMIx_Store_internal}}
\declareapi{PMIx_Store_internal}

%%%%
\summary

Store some data locally for retrieval by other areas of the proc.

%%%%
\format

\versionMarker{1.0}
\cspecificstart
\begin{codepar}
pmix_status_t
PMIx_Store_internal(const pmix_proc_t *proc,
                    const pmix_key_t key,
                    pmix_value_t *val);
\end{codepar}
\cspecificend

\begin{arglist}
\argin{proc}{process reference (handle)}
\argin{key}{key to retrieve (string)}
\argin{val}{Value to store (handle)}
\end{arglist}

Returns \refconst{PMIX_SUCCESS} or a negative value corresponding to a PMIx error constant.

%%%%
\descr

Store some data locally for retrieval by other areas of the proc.
This is data that has only internal scope - it will never be ``pushed'' externally.

%%%%%%%%%%%
\subsection{Accessing information: examples}
\label{chap:api_kv:getex}

This section provides examples illustrating methods for accessing information at various levels. The intent of the examples is not to provide comprehensive coding guidance, but rather to illustrate how \refapi{PMIx_Get} can be used to obtain information on a \refterm{session}, \refterm{job}, \refterm{application}, process, and node.

\subsubsection{Session-level information}

The \refapi{PMIx_Get} \ac{API} does not include an argument for specifying the \refterm{session} associated with the information being requested. Information regarding the session containing the requestor can be obtained by the following methods:

\begin{itemize}
\item for session-level attributes (e.g., \refattr{PMIX_UNIV_SIZE}), specifying the requestor's namespace and a rank of \refconst{PMIX_RANK_WILDCARD}; or
\item for non-specific attributes (e.g., \refattr{PMIX_NUM_NODES}), including the \refattr{PMIX_SESSION_INFO} attribute to indicate that the session-level information for that attribute is being requested
\end{itemize}

Example requests are shown below:

\cspecificstart
\begin{codepar}
pmix_info_t info;
pmix_value_t *value;
pmix_status_t rc;
pmix_proc_t myproc, wildcard;

/* initialize the client library */
PMIx_Init(&myproc, NULL, 0);

/* get the #slots in our session */
PMIX_PROC_LOAD(&wildcard, myproc.nspace, PMIX_RANK_WILDCARD);
rc = PMIx_Get(&wildcard, PMIX_UNIV_SIZE, NULL, 0, &value);

/* get the #nodes in our session */
PMIX_INFO_LOAD(&info, PMIX_SESSION_INFO, NULL, PMIX_BOOL);
rc = PMIx_Get(&wildcard, PMIX_NUM_NODES, &info, 1, &value);
\end{codepar}
\cspecificend

Information regarding a different session can be requested by either specifying the namespace and a rank of \refconst{PMIX_RANK_WILDCARD} for a process in the target session, or adding the \refattr{PMIX_SESSION_ID} attribute identifying the target session. In the latter case, the \refarg{proc} argument to \refapi{PMIx_Get} will be ignored:

\cspecificstart
\begin{codepar}
pmix_info_t info[2];
pmix_value_t *value;
pmix_status_t rc;
pmix_proc_t myproc;
uint32_t sid;

/* initialize the client library */
PMIx_Init(&myproc, NULL, 0);

/* get the #nodes in a different session */
sid = 12345;
PMIX_INFO_LOAD(&info[0], PMIX_SESSION_INFO, NULL, PMIX_BOOL);
PMIX_INFO_LOAD(&info[1], PMIX_SESSION_ID, &sid, PMIX_UINT32);
rc = PMIx_Get(&myproc, PMIX_NUM_NODES, info, 2, &value);
\end{codepar}
\cspecificend

\subsubsection{Job-level information}

Information regarding a job can be obtained by the following methods:

\begin{itemize}
\item for job-level attributes (e.g., \refattr{PMIX_JOB_SIZE} or \refattr{PMIX_JOB_NUM_APPS}), specifying the namespace of the job and a rank of \refconst{PMIX_RANK_WILDCARD} for the \refarg{proc} argument to \refapi{PMIx_Get}; or
\item for non-specific attributes (e.g., \refattr{PMIX_NUM_NODES}), including the \refattr{PMIX_JOB_INFO} attribute to indicate that the job-level information for that attribute is being requested
\end{itemize}

Example requests are shown below:

\cspecificstart
\begin{codepar}
pmix_info_t info;
pmix_value_t *value;
pmix_status_t rc;
pmix_proc_t myproc, wildcard;

/* initialize the client library */
PMIx_Init(&myproc, NULL, 0);

/* get the #apps in our job */
PMIX_PROC_LOAD(&wildcard, myproc.nspace, PMIX_RANK_WILDCARD);
rc = PMIx_Get(&wildcard, PMIX_JOB_NUM_APPS, NULL, 0, &value);

/* get the #nodes in our job */
PMIX_INFO_LOAD(&info, PMIX_JOB_INFO, NULL, PMIX_BOOL);
rc = PMIx_Get(&wildcard, PMIX_NUM_NODES, &info, 1, &value);
\end{codepar}
\cspecificend


\subsubsection{Application-level information}

Information regarding an application can be obtained by the following methods:

\begin{itemize}
\item for application-level attributes (e.g., \refattr{PMIX_APP_SIZE}), specifying the namespace and rank of a process within that application;
\item for application-level attributes (e.g., \refattr{PMIX_APP_SIZE}), including the \refattr{PMIX_APPNUM} attribute specifying the application whose information is being requested. In this case, the namespace field of the \refarg{proc} argument is used to reference the \refterm{job} containing the application - the \refterm{rank} field is ignored;
\item or application-level attributes (e.g., \refattr{PMIX_APP_SIZE}), including the \refattr{PMIX_APPNUM} and \refattr{PMIX_NSPACE} or \refattr{PMIX_JOBID} attributes specifying the job/application whose information is being requested. In this case, the \refarg{proc} argument is ignored;
\item for non-specific attributes (e.g., \refattr{PMIX_NUM_NODES}), including the \refattr{PMIX_APP_INFO} attribute to indicate that the application-level information for that attribute is being requested
\end{itemize}

Example requests are shown below:

\cspecificstart
\begin{codepar}
pmix_info_t info;
pmix_value_t *value;
pmix_status_t rc;
pmix_proc_t myproc, otherproc;
uint32_t appsize, appnum;

/* initialize the client library */
PMIx_Init(&myproc, NULL, 0);

/* get the #processes in our application */
rc = PMIx_Get(&myproc, PMIX_APP_SIZE, NULL, 0, &value);
appsize = value->data.uint32;

/* get the #nodes in an application containing "otherproc".
 * Note that the rank of a process in the other application
 * must be obtained first - a simple method is shown here */

/* assume for this example that we are in the first application
 * and we want the #nodes in the second application - use the
 * rank of the first process in that application, remembering
 * that ranks start at zero */
PMIX_PROC_LOAD(&otherproc, myproc.nspace, appsize);

PMIX_INFO_LOAD(&info, PMIX_APP_INFO, NULL, PMIX_BOOL);
rc = PMIx_Get(&otherproc, PMIX_NUM_NODES, &info, 1, &value);

/* alternatively, we can directly ask for the #nodes in
 * the second application in our job, again remembering that
 * application numbers start with zero */
appnum = 1;
PMIX_INFO_LOAD(&appinfo[0], PMIX_APP_INFO, NULL, PMIX_BOOL);
PMIX_INFO_LOAD(&appinfo[1], PMIX_APPNUM, &appnum, PMIX_UINT32);
rc = PMIx_Get(&myproc, PMIX_NUM_NODES, appinfo, 2, &value);

\end{codepar}
\cspecificend

\subsubsection{Process-level information}

Process-level information is accessed by providing the namespace and rank of the target process. In the absence of any directive as to the level of information being requested, the \ac{PMIx} library will always return the process-level value.

\subsubsection{Node-level information}

Information regarding a node within the system can be obtained by the following methods:

\begin{itemize}
\item for node-level attributes (e.g., \refattr{PMIX_NODE_SIZE}), specifying the namespace and rank of a process executing on the target node;
\item for node-level attributes (e.g., \refattr{PMIX_NODE_SIZE}), including the \refattr{PMIX_NODEID} or \refattr{PMIX_HOSTNAME} attribute specifying the node whose information is being requested. In this case, the \refarg{proc} argument's values are ignored; or
\item for non-specific attributes (e.g., \refattr{PMIX_MAX_PROCS}), including the \refattr{PMIX_NODE_INFO} attribute to indicate that the node-level information for that attribute is being requested
\end{itemize}

Example requests are shown below:

\cspecificstart
\begin{codepar}
pmix_info_t info[2];
pmix_value_t *value;
pmix_status_t rc;
pmix_proc_t myproc, otherproc;
uint32_t nodeid;

/* initialize the client library */
PMIx_Init(&myproc, NULL, 0);

/* get the #procs on our node */
rc = PMIx_Get(&myproc, PMIX_NODE_SIZE, NULL, 0, &value);

/* get the #slots on another node */
PMIX_INFO_LOAD(&info[0], PMIX_NODE_INFO, NULL, PMIX_BOOL);
PMIX_INFO_LOAD(&info[1], PMIX_HOSTNAME, "remotehost", PMIX_STRING);
rc = PMIx_Get(&myproc, PMIX_MAX_PROCS, info, 2, &value);

\end{codepar}
\cspecificend

\adviceuserstart
An explanation of the use of \refapi{PMIx_Get} versus \refapi{PMIx_Query_info_nb} is provided in \ref{chap:api_job_mgmt:query}.
\adviceuserend

%%%%%%%%%%%%%%%%%%%%%%%%%%%%%%%%%%%%%%%%%%%%%%
%%%%%%%%%%%%%%%%%%%%%%%%%%%%%%%%%%%%%%%%%%%%%%
\section{Exchanging Key/Value Pairs}
\label{chap:api_kv_mgmt:exchange}

The APIs defined in this section push key/value pairs from the client to the local \ac{PMIx} server, and circulate the data between \ac{PMIx} servers for subsequent retrieval by the local clients.

%%%%%%%%%%%
\subsection{\code{PMIx_Commit}}
\declareapi{PMIx_Commit}

%%%%
\summary

Push all previously \refapi{PMIx_Put} values to the local PMIx server.

%%%%
\format

\versionMarker{1.0}
\cspecificstart
\begin{codepar}
pmix_status_t PMIx_Commit(void)
\end{codepar}
\cspecificend

Returns \refconst{PMIX_SUCCESS} or a negative value corresponding to a PMIx error constant.

%%%%
\descr

This is an asynchronous operation.
The \ac{PRI} will immediately return to the caller while the data is transmitted to the local server in the background.

\adviceuserstart
The local PMIx server will cache the information locally - i.e., the committed data will not be circulated during \refapi{PMIx_Commit}.
Availability of the data upon completion of \refapi{PMIx_Commit} is therefore implementation-dependent.
\adviceuserend


%%%%%%%%%%%
\subsection{\code{PMIx_Fence}}
\declareapi{PMIx_Fence}

%%%%
\summary

Execute a blocking barrier across the processes identified in the specified array, collecting information posted via \refapi{PMIx_Put} as directed.

%%%%
\format

\versionMarker{1.0}
\cspecificstart
\begin{codepar}
pmix_status_t
PMIx_Fence(const pmix_proc_t procs[], size_t nprocs,
           const pmix_info_t info[], size_t ninfo)
\end{codepar}
\cspecificend

\begin{arglist}
\argin{procs}{Array of \refstruct{pmix_proc_t} structures (array of handles)}
\argin{nprocs}{Number of element in the \refarg{procs} array (integer)}
\argin{info}{Array of info structures (array of handles)}
\argin{ninfo}{Number of element in the \refarg{info} array (integer)}
\end{arglist}

Returns \refconst{PMIX_SUCCESS} or a negative value corresponding to a PMIx error constant.

\reqattrstart
The following attributes are required to be supported by all \ac{PMIx} libraries:

\pastePRIAttributeItem{PMIX_COLLECT_DATA}

\reqattrend

\optattrstart
The following attributes are optional for host environments:

\pastePRRTEAttributeItem{PMIX_TIMEOUT}
\pasteAttributeItem{PMIX_COLLECTIVE_ALGO}
\pasteAttributeItem{PMIX_COLLECTIVE_ALGO_REQD}

\optattrend

\adviceimplstart
We recommend that implementation of the \refattr{PMIX_TIMEOUT} attribute be left to the host environment due to race condition considerations between completion of the operation versus internal timeout in the \ac{PMIx} server library. Implementers that choose to support \refattr{PMIX_TIMEOUT} directly in the \ac{PMIx} server library must take care to resolve the race condition and should avoid passing \refattr{PMIX_TIMEOUT} to the host environment so that multiple competing timeouts are not created.
\adviceimplend

%%%%
\descr

Passing a \code{NULL} pointer as the \refarg{procs} parameter indicates that the fence is to span all processes in the client's namespace.
Each provided \refstruct{pmix_proc_t} struct can pass \refconst{PMIX_RANK_WILDCARD} to indicate that all processes in the given namespace are participating.

The \refarg{info} array is used to pass user requests regarding the fence operation.

Note that for scalability reasons, the default behavior for \refapi{PMIx_Fence} is to not collect the data.

\adviceimplstart
\refapi{PMIx_Fence} and its non-blocking form are both \emph{collective} operations. Accordingly, the \ac{PMIx} server library is required to aggregate participation by local clients, passing the request to the host environment once all local participants have executed the \ac{API}.
\adviceimplend

\advicermstart
The host will receive a single call for each collective operation. It is the responsibility of the host to identify the nodes containing participating processes, execute the collective across all participating nodes, and notify the local \ac{PMIx} server library upon completion of the global collective.
\advicermend

%%%%%%%%%%%
\subsection{\code{PMIx_Fence_nb}}
\declareapi{PMIx_Fence_nb}

%%%%
\summary

Execute a nonblocking \refapi{PMIx_Fence} across the processes identified in the specified array of processes, collecting information posted via \refapi{PMIx_Put} as directed.

%%%%
\format

\versionMarker{1.0}
\cspecificstart
\begin{codepar}
pmix_status_t
PMIx_Fence_nb(const pmix_proc_t procs[], size_t nprocs,
              const pmix_info_t info[], size_t ninfo,
              pmix_op_cbfunc_t cbfunc, void *cbdata)
\end{codepar}
\cspecificend

\begin{arglist}
\argin{procs}{Array of \refstruct{pmix_proc_t} structures (array of handles)}
\argin{nprocs}{Number of element in the \refarg{procs} array (integer)}
\argin{info}{Array of info structures (array of handles)}
\argin{ninfo}{Number of element in the \refarg{info} array (integer)}
\argin{cbfunc}{Callback function (function reference)}
\argin{cbdata}{Data to be passed to the callback function (memory reference)}
\end{arglist}

Returns one of the following:

\begin{itemize}
    \item \refconst{PMIX_SUCCESS}, indicating that the request is being processed by the host environment - result will be returned in the provided \refarg{cbfunc}. Note that the library must not invoke the callback function prior to returning from the \ac{API}.
    \item \refconst{PMIX_OPERATION_SUCCEEDED}, indicating that the request was immediately processed and returned \textit{success} - the \refarg{cbfunc} will \textit{not} be called. This can occur if the collective involved only processes on the local node.
    \item a PMIx error constant indicating either an error in the input or that the request was immediately processed and failed - the \refarg{cbfunc} will \textit{not} be called
\end{itemize}


\reqattrstart
The following attributes are required to be supported by all \ac{PMIx} libraries:

\pastePRIAttributeItem{PMIX_COLLECT_DATA}

\reqattrend

\optattrstart
The following attributes are optional for host environments that support this operation:

\pastePRRTEAttributeItem{PMIX_TIMEOUT}
\pasteAttributeItem{PMIX_COLLECTIVE_ALGO}
\pasteAttributeItem{PMIX_COLLECTIVE_ALGO_REQD}

\optattrend

\adviceimplstart
We recommend that implementation of the \refattr{PMIX_TIMEOUT} attribute be left to the host environment due to race condition considerations between completion of the operation versus internal timeout in the \ac{PMIx} server library. Implementers that choose to support \refattr{PMIX_TIMEOUT} directly in the \ac{PMIx} server library must take care to resolve the race condition and should avoid passing \refattr{PMIX_TIMEOUT} to the host environment so that multiple competing timeouts are not created.

Note that \ac{PMIx} libraries may choose to implement an optimization for the case where only the calling process is involved in the fence operation by immediately returning \refconst{PMIX_OPERATION_SUCCEEDED} from the client's call in lieu of passing the fence operation to a \ac{PMIx} server. Fence operations involving more than just the calling process must be communicated to the \ac{PMIx} server for proper execution of the included barrier behavior.

Similarly, fence operations that involve only processes that are clients of the same \ac{PMIx} server may be resolved by that server without referral to its host environment as no inter-node coordination is required.
\adviceimplend

%%%%
\descr

Nonblocking \refapi{PMIx_Fence} routine.
Note that the function will return an error if a \code{NULL} callback function is given.

Note that for scalability reasons, the default behavior for \refapi{PMIx_Fence_nb} is to not collect the data.

See the \refapi{PMIx_Fence} description for further details.

%%%%%%%%%%%%%%%%%%%%%%%%%%%%%%%%%%%%%%%%%%%%%%
%%%%%%%%%%%%%%%%%%%%%%%%%%%%%%%%%%%%%%%%%%%%%%
\section{Publish and Lookup Data}
\label{chap:api_kv_mgmt:publish}

The APIs defined in this section publish data from one client that can be later exchanged and looked up by another client.

\adviceimplstart
\ac{PMIx} libraries that support any of the functions in this section are required to support \textit{all} of them.
\adviceimplend

\advicermstart
Host environments that support any of the functions in this section are required to support \textit{all} of them.
\advicermend

%%%%%%%%%%%
\subsection{\code{PMIx_Publish}}
\declareapi{PMIx_Publish}

%%%%
\summary

Publish data for later access via \refapi{PMIx_Lookup}.

%%%%
\format

\versionMarker{1.0}
\cspecificstart
\begin{codepar}
pmix_status_t
PMIx_Publish(const pmix_info_t info[], size_t ninfo)
\end{codepar}
\cspecificend

\begin{arglist}
\argin{info}{Array of info structures (array of handles)}
\argin{ninfo}{Number of element in the \refarg{info} array (integer)}
\end{arglist}

Returns \refconst{PMIX_SUCCESS} or a negative value corresponding to a PMIx error constant.

\reqattrstart
\ac{PMIx} libraries are not required to directly support any attributes for this function. However, any provided attributes must be passed to the host \ac{SMS} daemon for processing, and the \ac{PMIx} library is \textit{required} to add the \refPRIAttributeItem{PMIX_USERID} and the \refPRIAttributeItem{PMIX_GRPID} attributes of the client process that published the info.

\reqattrend

\optattrstart
The following attributes are optional for host environments that support this operation:

\pastePRRTEAttributeItem{PMIX_TIMEOUT}
\pastePRRTEAttributeItem{PMIX_RANGE}
\pastePRRTEAttributeItem{PMIX_PERSISTENCE}

\optattrend

\adviceimplstart
We recommend that implementation of the \refattr{PMIX_TIMEOUT} attribute be left to the host environment due to race condition considerations between completion of the operation versus internal timeout in the \ac{PMIx} server library. Implementers that choose to support \refattr{PMIX_TIMEOUT} directly in the \ac{PMIx} server library must take care to resolve the race condition and should avoid passing \refattr{PMIX_TIMEOUT} to the host environment so that multiple competing timeouts are not created.
\adviceimplend

%%%%
\descr

Publish the data in the \refarg{info} array for subsequent lookup.
By default, the data will be published into the \refconst{PMIX_RANGE_SESSION} range and with \refconst{PMIX_PERSIST_APP} persistence.
Changes to those values, and any additional directives, can be included in the \refstruct{pmix_info_t} array. Attempts to access the data by processes outside of the provided data range will be rejected. The persistence parameter instructs the server as to how long the data is to be retained.

The blocking form will block until the server confirms that the data has been sent to the \ac{PMIx} server and that it has obtained confirmation from its host \ac{SMS} daemon that the data is ready to be looked up. Data is copied into the backing key-value data store, and therefore the \refarg{info} array can be released upon return from the blocking function call.

\adviceuserstart
Publishing duplicate keys is permitted provided they are published to different ranges.
\adviceuserend

\adviceimplstart
Implementations should, to the best of their ability, detect duplicate keys being posted on the same data range and protect the
user from unexpected behavior by returning the \refconst{PMIX_ERR_DUPLICATE_KEY} error.
\adviceimplend

%%%%%%%%%%%
\subsection{\code{PMIx_Publish_nb}}
\declareapi{PMIx_Publish_nb}

%%%%
\summary

Nonblocking \refapi{PMIx_Publish} routine.

%%%%
\format

\versionMarker{1.0}
\cspecificstart
\begin{codepar}
pmix_status_t
PMIx_Publish_nb(const pmix_info_t info[], size_t ninfo,
                pmix_op_cbfunc_t cbfunc, void *cbdata)
\end{codepar}
\cspecificend

\begin{arglist}
\argin{info}{Array of info structures (array of handles)}
\argin{ninfo}{Number of element in the \refarg{info} array (integer)}
\argin{cbfunc}{Callback function \refapi{pmix_op_cbfunc_t} (function reference)}
\argin{cbdata}{Data to be passed to the callback function (memory reference)}
\end{arglist}

Returns one of the following:

\begin{itemize}
    \item \refconst{PMIX_SUCCESS}, indicating that the request is being processed by the host environment - result will be returned in the provided \refarg{cbfunc}. Note that the library must not invoke the callback function prior to returning from the \ac{API}.
    \item \refconst{PMIX_OPERATION_SUCCEEDED}, indicating that the request was immediately processed and returned \textit{success} - the \refarg{cbfunc} will \textit{not} be called
    \item a PMIx error constant indicating either an error in the input or that the request was immediately processed and failed - the \refarg{cbfunc} will \textit{not} be called
\end{itemize}

\reqattrstart
\ac{PMIx} libraries are not required to directly support any attributes for this function. However, any provided attributes must be passed to the host \ac{SMS} daemon for processing, and the \ac{PMIx} library is \textit{required} to add the \refPRIAttributeItem{PMIX_USERID} and the \refPRIAttributeItem{PMIX_GRPID} attributes of the client process that published the info.

\reqattrend

\optattrstart
The following attributes are optional for host environments that support this operation:

\pastePRRTEAttributeItem{PMIX_TIMEOUT}
\pastePRRTEAttributeItem{PMIX_RANGE}
\pastePRRTEAttributeItem{PMIX_PERSISTENCE}

\optattrend

\adviceimplstart
We recommend that implementation of the \refattr{PMIX_TIMEOUT} attribute be left to the host environment due to race condition considerations between completion of the operation versus internal timeout in the \ac{PMIx} server library. Implementers that choose to support \refattr{PMIX_TIMEOUT} directly in the \ac{PMIx} server library must take care to resolve the race condition and should avoid passing \refattr{PMIX_TIMEOUT} to the host environment so that multiple competing timeouts are not created.
\adviceimplend

%%%%
\descr

Nonblocking \refapi{PMIx_Publish} routine. The non-blocking form will return immediately, executing the callback when the \ac{PMIx} server receives confirmation from its host \ac{SMS} daemon.

Note that the function will return an error if a \code{NULL} callback function is given, and that the \refarg{info} array must be maintained until the callback is provided.


%%%%%%%%%%%
\subsection{\code{PMIx_Lookup}}
\declareapi{PMIx_Lookup}

%%%%
\summary

Lookup information published by this or another process with \refapi{PMIx_Publish} or \refapi{PMIx_Publish_nb}.

%%%%
\format

\versionMarker{1.0}
\cspecificstart
\begin{codepar}
pmix_status_t
PMIx_Lookup(pmix_pdata_t data[], size_t ndata,
            const pmix_info_t info[], size_t ninfo)
\end{codepar}
\cspecificend

\begin{arglist}
\arginout{data}{Array of publishable data structures (array of handles)}
\argin{ndata}{Number of elements in the \refarg{data} array (integer)}
\argin{info}{Array of info structures (array of handles)}
\argin{ninfo}{Number of elements in the \refarg{info} array (integer)}
\end{arglist}

Returns \refconst{PMIX_SUCCESS} or a negative value corresponding to a PMIx error constant.

\reqattrstart
\ac{PMIx} libraries are not required to directly support any attributes for this function. However, any provided attributes must be passed to the host \ac{SMS} daemon for processing, and the \ac{PMIx} library is \textit{required} to add the \refPRIAttributeItem{PMIX_USERID} and the \refPRIAttributeItem{PMIX_GRPID} attributes of the client process that is requesting the info.

\reqattrend

\optattrstart
The following attributes are optional for host environments that support this operation:

\pastePRRTEAttributeItem{PMIX_TIMEOUT}
\pastePRRTEAttributeItem{PMIX_RANGE}
\pastePRRTEAttributeItem{PMIX_WAIT}

\optattrend

\adviceimplstart
We recommend that implementation of the \refattr{PMIX_TIMEOUT} attribute be left to the host environment due to race condition considerations between completion of the operation versus internal timeout in the \ac{PMIx} server library. Implementers that choose to support \refattr{PMIX_TIMEOUT} directly in the \ac{PMIx} server library must take care to resolve the race condition and should avoid passing \refattr{PMIX_TIMEOUT} to the host environment so that multiple competing timeouts are not created.
\adviceimplend

%%%%
\descr

Lookup information published by this or another process.
By default, the search will be conducted across the \refconst{PMIX_RANGE_SESSION} range.
Changes to the range, and any additional directives, can be provided in the \refstruct{pmix_info_t} array. Data is returned provided the following conditions are met:

\begin{itemize}
    \item the requesting process resides within the range specified by the publisher. For example, data published to \refconst{PMIX_RANGE_LOCAL} can only be discovered by a process executing on the same node
    \item the provided key matches the published key within that data range
    \item the data was published by a process with corresponding user and/or group IDs as the one looking up the data. There currently is no option to override this behavior - such an option may become available later via an appropriate \refstruct{pmix_info_t} directive.
\end{itemize}

The \argref{data} parameter consists of an array of \refstruct{pmix_pdata_t} struct with the keys specifying the requested information.
Data will be returned for each key in the associated \refarg{value} struct.
Any key that cannot be found will return with a data type of \refconst{PMIX_UNDEF}.
The function will return \refconst{PMIX_SUCCESS} if any values can be found, so the caller must check each data element to ensure it was returned.

The proc field in each \refstruct{pmix_pdata_t} struct will contain the namespace/rank of the process that published the data.

\adviceuserstart
Although this is a blocking function, it will not wait by default for the requested data to be published.
Instead, it will block for the time required by the server to lookup its current data and return any found items.
Thus, the caller is responsible for ensuring that data is published prior to executing a lookup, using \refattr{PMIX_WAIT} to instruct the server to wait for the data to be published, or for retrying until the requested data is found.
\adviceuserend

%%%%%%%%%%%
\subsection{\code{PMIx_Lookup_nb}}
\declareapi{PMIx_Lookup_nb}

%%%%
\summary

Nonblocking version of \refapi{PMIx_Lookup}.

%%%%
\format

\versionMarker{1.0}
\cspecificstart
\begin{codepar}
pmix_status_t
PMIx_Lookup_nb(char **keys,
               const pmix_info_t info[], size_t ninfo,
               pmix_lookup_cbfunc_t cbfunc, void *cbdata)
\end{codepar}
\cspecificend

\begin{arglist}
\argin{keys}{Array to be provided to the callback (array of strings)}
\argin{info}{Array of info structures (array of handles)}
\argin{ninfo}{Number of element in the \refarg{info} array (integer)}
\argin{cbfunc}{Callback function (handle)}
\argin{cbdata}{Callback data to be provided to the callback function (pointer)}
\end{arglist}

Returns one of the following:

\begin{itemize}
    \item \refconst{PMIX_SUCCESS}, indicating that the request is being processed by the host environment - result will be returned in the provided \refarg{cbfunc}. Note that the library must not invoke the callback function prior to returning from the \ac{API}.
    \item a PMIx error constant indicating an error in the input - the \refarg{cbfunc} will \textit{not} be called
\end{itemize}


\reqattrstart
\ac{PMIx} libraries are not required to directly support any attributes for this function. However, any provided attributes must be passed to the host \ac{SMS} daemon for processing, and the \ac{PMIx} library is \textit{required} to add the \refPRIAttributeItem{PMIX_USERID} and the \refPRIAttributeItem{PMIX_GRPID} attributes of the client process that is requesting the info.

\reqattrend

\optattrstart
The following attributes are optional for host environments that support this operation:

\pastePRRTEAttributeItem{PMIX_TIMEOUT}
\pastePRRTEAttributeItem{PMIX_RANGE}
\pastePRRTEAttributeItem{PMIX_WAIT}

\optattrend

\adviceimplstart
We recommend that implementation of the \refattr{PMIX_TIMEOUT} attribute be left to the host environment due to race condition considerations between completion of the operation versus internal timeout in the \ac{PMIx} server library. Implementers that choose to support \refattr{PMIX_TIMEOUT} directly in the \ac{PMIx} server library must take care to resolve the race condition and should avoid passing \refattr{PMIX_TIMEOUT} to the host environment so that multiple competing timeouts are not created.
\adviceimplend


%%%%
\descr

Non-blocking form of the \refapi{PMIx_Lookup} function.
Data for the provided NULL-terminated \refarg{keys} array will be returned in the provided callback function.
As with \refapi{PMIx_Lookup}, the default behavior is to not wait for data to be published.
The \refarg{info} array can be used to modify the behavior as previously described by \refapi{PMIx_Lookup}. Both the \refarg{info} and \refarg{keys} arrays must be maintained until the callback is provided.



%%%%%%%%%%%
\subsection{\code{PMIx_Unpublish}}
\declareapi{PMIx_Unpublish}

%%%%
\summary

Unpublish data posted by this process using the given keys.

%%%%
\format

\versionMarker{1.0}
\cspecificstart
\begin{codepar}
pmix_status_t
PMIx_Unpublish(char **keys,
               const pmix_info_t info[], size_t ninfo)
\end{codepar}
\cspecificend

\begin{arglist}
\argin{info}{Array of info structures (array of handles)}
\argin{ninfo}{Number of element in the \refarg{info} array (integer)}
\end{arglist}

Returns \refconst{PMIX_SUCCESS} or a negative value corresponding to a PMIx error constant.

\reqattrstart
\ac{PMIx} libraries are not required to directly support any attributes for this function. However, any provided attributes must be passed to the host \ac{SMS} daemon for processing, and the \ac{PMIx} library is \textit{required} to add the \refPRIAttributeItem{PMIX_USERID} and the \refPRIAttributeItem{PMIX_GRPID} attributes of the client process that is requesting the operation.

\reqattrend

\optattrstart
The following attributes are optional for host environments that support this operation:

\pastePRRTEAttributeItem{PMIX_TIMEOUT}
\pastePRRTEAttributeItem{PMIX_RANGE}

\optattrend

\adviceimplstart
We recommend that implementation of the \refattr{PMIX_TIMEOUT} attribute be left to the host environment due to race condition considerations between completion of the operation versus internal timeout in the \ac{PMIx} server library. Implementers that choose to support \refattr{PMIX_TIMEOUT} directly in the \ac{PMIx} server library must take care to resolve the race condition and should avoid passing \refattr{PMIX_TIMEOUT} to the host environment so that multiple competing timeouts are not created.
\adviceimplend


%%%%
\descr

Unpublish data posted by this process using the given \refarg{keys}.
The function will block until the data has been removed by the server (i.e., it is safe to publish that key again).
A value of \code{NULL} for the \refarg{keys} parameter instructs the server to remove all data published by this process.

By default, the range is assumed to be \refconst{PMIX_RANGE_SESSION}.
Changes to the range, and any additional directives, can be provided in the \refarg{info} array.


%%%%%%%%%%%
\subsection{\code{PMIx_Unpublish_nb}}
\declareapi{PMIx_Unpublish_nb}

%%%%
\summary

Nonblocking version of \refapi{PMIx_Unpublish}.

%%%%
\format

\versionMarker{1.0}
\cspecificstart
\begin{codepar}
pmix_status_t
PMIx_Unpublish_nb(char **keys,
                  const pmix_info_t info[], size_t ninfo,
                  pmix_op_cbfunc_t cbfunc, void *cbdata)
\end{codepar}
\cspecificend

\begin{arglist}
\argin{keys}{(array of strings)}
\argin{info}{Array of info structures (array of handles)}
\argin{ninfo}{Number of element in the \refarg{info} array (integer)}
\argin{cbfunc}{Callback function \refapi{pmix_op_cbfunc_t} (function reference)}
\argin{cbdata}{Data to be passed to the callback function (memory reference)}
\end{arglist}

Returns one of the following:

\begin{itemize}
    \item \refconst{PMIX_SUCCESS}, indicating that the request is being processed by the host environment - result will be returned in the provided \refarg{cbfunc}. Note that the library must not invoke the callback function prior to returning from the \ac{API}.
    \item \refconst{PMIX_OPERATION_SUCCEEDED}, indicating that the request was immediately processed and returned \textit{success} - the \refarg{cbfunc} will \textit{not} be called
    \item a PMIx error constant indicating either an error in the input or that the request was immediately processed and failed - the \refarg{cbfunc} will \textit{not} be called
\end{itemize}

\reqattrstart
\ac{PMIx} libraries are not required to directly support any attributes for this function. However, any provided attributes must be passed to the host \ac{SMS} daemon for processing, and the \ac{PMIx} library is \textit{required} to add the \refPRIAttributeItem{PMIX_USERID} and the \refPRIAttributeItem{PMIX_GRPID} attributes of the client process that is requesting the operation.

\reqattrend

\optattrstart
The following attributes are optional for host environments that support this operation:

\pastePRRTEAttributeItem{PMIX_TIMEOUT}
\pastePRRTEAttributeItem{PMIX_RANGE}

\optattrend

\adviceimplstart
We recommend that implementation of the \refattr{PMIX_TIMEOUT} attribute be left to the host environment due to race condition considerations between completion of the operation versus internal timeout in the \ac{PMIx} server library. Implementers that choose to support \refattr{PMIX_TIMEOUT} directly in the \ac{PMIx} server library must take care to resolve the race condition and should avoid passing \refattr{PMIX_TIMEOUT} to the host environment so that multiple competing timeouts are not created.
\adviceimplend

%%%%
\descr

Non-blocking form of the \refapi{PMIx_Unpublish} function.
The callback function will be executed once the server confirms removal of the specified data. The \refarg{info} array must be maintained until the callback is provided.



%%%%%%%%%%%%%%%%%%%%%%%%%%%%%%%%%%%%%%%%%%%%%%%%%


    % Process Management
    %  - spawn, (dis)connect, resolve_peers
    %%%%%%%%%%%%%%%%%%%%%%%%%%%%%%%%%%%%%%%%%%%%%%%%%
% Chapter: Process Management
%%%%%%%%%%%%%%%%%%%%%%%%%%%%%%%%%%%%%%%%%%%%%%%%%
\chapter{Process Management}
\label{chap:api_proc_mgmt}

\ldots

%%%%%%%%%%%%%%%%%%%%%%%%%%%%%%%%%%%%%%%%%%%%%%
%%%%%%%%%%%%%%%%%%%%%%%%%%%%%%%%%%%%%%%%%%%%%%
\section{Abort}
\label{chap:api_proc_mgmt:abort}

\ldots

%%%%%%%%%%%
\subsection{\code{PMIx_Abort}}
\declareapi{PMIx_Abort}

%%%%
\summary

Abort the specified process.

%%%%
\format

\cspecificstart
\begin{codepar}
pmix_status_t
PMIx_Abort(int status, const char msg[],
           pmix_proc_t procs[], size_t nprocs)
\end{codepar}
\cspecificend

\begin{arglist}
\argin{status}{Error code to return to invoking environment (integer)}
\argin{msg}{String message to be returned to user (string)}
\argin{procs}{Array of \refstruct{pmix_proc_t} structures (array of handles)}
\argin{nprocs}{Number of elements in the \refarg{procs} array (integer)}
\end{arglist}

Returns \refconst{PMIX_SUCCESS} or a negative value corresponding to a PMIx error constant.

%%%%
\descr

Request that the host resource manager print the provided message and abort the provided array of \refarg{procs}.
A Unix or POSIX environment should handle the provided status as a return error code from the main program that launched the application.
A \code{NULL} for the \refarg{procs} array indicates that all processes in the caller's namespace are to be aborted, including itself.
Passing a \code{NULL} \refarg{msg} parameter is allowed.

\adviceuserstart
The response to this request is somewhat dependent on the specific \acl{RM} and its configuration (e.g., some resource managers will not abort the application if the provided status is zero unless specifically configured to do so, and some cannot abort subsets of processes in an application), and thus lies outside the control of PMIx itself.
However, the PMIx client library shall inform the \ac{RM} of the request that the specified \refarg{procs} be aborted, regardless of the value of the provided status.

Note that race conditions caused by multiple processes calling \refapi{PMIx_Abort} are left to the server implementation to resolve with regard to which status is returned and what messages (if any) are printed.
\adviceuserend


%%%%%%%%%%%%%%%%%%%%%%%%%%%%%%%%%%%%%%%%%%%%%%
%%%%%%%%%%%%%%%%%%%%%%%%%%%%%%%%%%%%%%%%%%%%%%
\section{Process Creation}
\label{chap:api_proc_mgmt:spawn}

\ldots

%%%%%%%%%%%
\subsection{\code{PMIx_Spawn}}
\declareapi{PMIx_Spawn}

%%%%
\summary

Spawn a new job.

%%%%
\format

\cspecificstart
\begin{codepar}
pmix_status_t
PMIx_Spawn(const pmix_info_t job_info[], size_t ninfo,
           const pmix_app_t apps[], size_t napps,
           char nspace[])
\end{codepar}
\cspecificend

\begin{arglist}
\argin{job_info}{Array of info structures (array of handles)}
\argin{ninfo}{Number of elements in the \refarg{job_info} array (integer)}
\argin{apps}{Array of \refstruct{pmix_app_t} structures (array of handles)}
\argin{napps}{Number of elements in the \refarg{apps} array (integer)}
\argout{nspace}{Namespace of the new job (string)}
\end{arglist}

Returns \refconst{PMIX_SUCCESS} or a negative value corresponding to a PMIx error constant.

%%%%
\descr

Spawn a new job.
The assigned namespace of the spawned applications is returned in the \refarg{nspace} parameter.
A \code{NULL} value in that location indicates that the caller doesn't wish to have the namespace returned.
The \refarg{nspace} array must be at least of size one more than \refconst{PMIX_MAX_NSLEN}.
Behavior of individual resource managers may differ, but it is expected that failure of any application process to start will result in termination/cleanup of \emph{all} processes in the newly spawned job and return of an error code to the caller.

By default, the spawned processes will be PMIx ``connected'' to the parent process upon successful launch (see \refapi{PMIx_Connect} description for details).
Note that this only means that the parent process (a) will be given a copy of the new job's
information so it can query job-level info without incurring any communication penalties, and (b) will receive notification of errors from process in the child job.

Job-level directives can be specified in the \refarg{job_info} array.
This can include:
\begin{attributedesc}
%
\declareattritem{PMIX_NON_PMI} (string)
Processes in the spawned job will not be calling \refapi{PMIx_Init}.
%
\declareattritem{PMIX_TIMEOUT} (string)
Declare the spawn as having failed if the launched processes do not call \refapi{PMIx_Init} within the specified time.
%
\declareattritem{PMIX_NOTIFY_COMPLETION} (string)
Notify the parent process when the child job terminates, either normally or with error.
%
\end{attributedesc}


%%%%%%%%%%%
\subsection{\code{PMIx_Spawn_nb}}
\declareapi{PMIx_Spawn_nb}

%%%%
\summary

Nonblocking version of the \refapi{PMIx_Spawn} routine.

%%%%
\format

\cspecificstart
\begin{codepar}
pmix_status_t
PMIx_Spawn_nb(const pmix_info_t job_info[], size_t ninfo,
              const pmix_app_t apps[], size_t napps,
              pmix_spawn_cbfunc_t cbfunc, void *cbdata)
\end{codepar}
\cspecificend

\begin{arglist}
\argin{job_info}{Array of info structures (array of handles)}
\argin{ninfo}{Number of elements in the \refarg{job_info} array (integer)}
\argin{apps}{Array of \refstruct{pmix_app_t} structures (array of handles)}
\argin{cbfunc}{Callback function \refapi{pmix_spawn_cbfunc_t} (function reference)}
\argin{cbdata}{Data to be passed to the callback function (memory reference)}
\end{arglist}

Returns \refconst{PMIX_SUCCESS} or a negative value corresponding to a PMIx error constant.

%%%%
\descr

Nonblocking version of the \refapi{PMIx_Spawn} routine.


%%%%%%%%%%%%%%%%%%%%%%%%%%%%%%%%%%%%%%%%%%%%%%
%%%%%%%%%%%%%%%%%%%%%%%%%%%%%%%%%%%%%%%%%%%%%%
\section{Connecting and Disconnecting Processes}
\label{chap:api_proc_mgmt:connect}

\ldots

%%%%%%%%%%%
\subsection{\code{PMIx_Connect}}
\declareapi{PMIx_Connect}

%%%%
\summary

Connect namespaces.

%%%%
\format

\cspecificstart
\begin{codepar}
pmix_status_t
PMIx_Connect(const pmix_proc_t procs[], size_t nprocs,
             const pmix_info_t info[], size_t ninfo)
\end{codepar}
\cspecificend

\begin{arglist}
\argin{procs}{Array of proc structures (array of handles)}
\argin{nprocs}{Number of elements in the \refarg{procs} array (integer)}
\argin{info}{Array of info structures (array of handles)}
\argin{ninfo}{Number of elements in the \refarg{info} array (integer)}
\end{arglist}

Returns \refconst{PMIX_SUCCESS} or a negative value corresponding to a PMIx error constant.

%%%%
\descr

Record the specified processes as ``connected''.
This means that the resource manager should treat the failure of any process in the specified group as a reportable event, and take appropriate action.
Note that different resource managers may respond to failures in different manners.

The callback function is to be called once all participating processes have called connect.
The server is required to return any job-level info for the connecting processes that might not already have (i.e., if the connect request involves \refarg{procs} from different namespaces, then each \refarg{proc} shall receive the job-level info from those namespaces other than their own.

A process can only engage in \emph{one} connect operation involving the identical set of processes at a time.
However, a process \emph{can} be simultaneously engaged in multiple connect operations, each involving a different set of processes.

As in the case of the fence operation, the info array can be used to pass user-level directives regarding the algorithm to be used for the collective operation involved in the ``connect'', timeout constraints, and other options available from the host RM.


%%%%%%%%%%%
\subsection{\code{PMIx_Connect_nb}}
\declareapi{PMIx_Connect_nb}

%%%%
\summary

Nonblocking \refapi{PMIx_Connect_nb} routine.

%%%%
\format

\cspecificstart
\begin{codepar}
pmix_status_t
PMIx_Connect_nb(const pmix_proc_t procs[], size_t nprocs,
                const pmix_info_t info[], size_t ninfo,
                pmix_op_cbfunc_t cbfunc, void *cbdata)
\end{codepar}
\cspecificend

\begin{arglist}
\argin{procs}{Array of proc structures (array of handles)}
\argin{nprocs}{Number of elements in the \refarg{procs} array (integer)}
\argin{info}{Array of info structures (array of handles)}
\argin{ninfo}{Number of element in the \refarg{info} array (integer)}
\argin{cbfunc}{Callback function \refapi{pmix_op_cbfunc_t} (function reference)}
\argin{cbdata}{Data to be passed to the callback function (memory reference)}
\end{arglist}

Returns \refconst{PMIX_SUCCESS} or a negative value corresponding to a PMIx error constant.

%%%%
\descr

Nonblocking \refapi{PMIx_Connect_nb} routine.


%%%%%%%%%%%
\subsection{\code{PMIx_Disconnect}}
\declareapi{PMIx_Disconnect}

%%%%
\summary

Disconnect a previously connected set of processes.

%%%%
\format

\cspecificstart
\begin{codepar}
pmix_status_t
PMIx_Disconnect(const pmix_proc_t procs[], size_t nprocs,
                const pmix_info_t info[], size_t ninfo);
\end{codepar}
\cspecificend

\begin{arglist}
\argin{procs}{Array of proc structures (array of handles)}
\argin{nprocs}{Number of elements in the \refarg{procs} array (integer)}
\argin{info}{Array of info structures (array of handles)}
\argin{ninfo}{Number of element in the \refarg{info} array (integer)}
\end{arglist}

Returns \refconst{PMIX_SUCCESS} or a negative value corresponding to a PMIx error constant.

%%%%
\descr

Disconnect a previously connected set of processes.
An error will be returned if the specified set of \refarg{procs} was not previously ``connected''.
As with \refapi{PMIx_Connect}, a process may be involved in multiple simultaneous disconnect operations.
However, a process is not allowed to reconnect to a set of \refarg{procs} that has not fully completed disconnect (i.e., you have to fully disconnect before you can reconnect to the \emph{same} group of processes.
The \refarg{info} array is used as in \refapi{PMIx_Connect}.


%%%%%%%%%%%
\subsection{\code{PMIx_Disconnect_nb}}
\declareapi{PMIx_Disconnect_nb}

%%%%
\summary

Nonblocking \refapi{PMIx_Disconnect} routine.

%%%%
\format

\cspecificstart
\begin{codepar}
pmix_status_t
PMIx_Disconnect_nb(const pmix_proc_t ranges[], size_t nprocs,
                   const pmix_info_t info[], size_t ninfo,
                   pmix_op_cbfunc_t cbfunc, void *cbdata);
\end{codepar}
\cspecificend

\begin{arglist}
\argin{procs}{Array of proc structures (array of handles)}
\argin{nprocs}{Number of elements in the \refarg{procs} array (integer)}
\argin{info}{Array of info structures (array of handles)}
\argin{ninfo}{Number of element in the \refarg{info} array (integer)}
\argin{cbfunc}{Callback function \refapi{pmix_op_cbfunc_t} (function reference)}
\argin{cbdata}{Data to be passed to the callback function (memory reference)}
\end{arglist}

Returns \refconst{PMIX_SUCCESS} or a negative value corresponding to a PMIx error constant.

%%%%
\descr

Nonblocking \refapi{PMIx_Disconnect} routine.


%%%%%%%%%%%%%%%%%%%%%%%%%%%%%%%%%%%%%%%%%%%%%%
%%%%%%%%%%%%%%%%%%%%%%%%%%%%%%%%%%%%%%%%%%%%%%
\section{Query}
\label{chap:api_proc_mgmt:query}

\ldots

%%%%%%%%%%%
\subsection{\code{PMIx_Resolve_peers}}
\declareapi{PMIx_Resolve_peers}

%%%%
\summary

Access an array of processes within the specified namespace on a node.

%%%%
\format

\cspecificstart
\begin{codepar}
pmix_status_t
PMIx_Resolve_peers(const char *nodename, const char *nspace,
                   pmix_proc_t **procs, size_t *nprocs)
\end{codepar}
\cspecificend

\begin{arglist}
\argin{nodename}{Name of the node to query (string)}
\argin{nspace}{namespace (string)}
\argout{procs}{Array of process structures (array of handles)}
\argout{nprocs}{Number of elements in the \refarg{procs} array (integer)}
\end{arglist}

Returns \refconst{PMIX_SUCCESS} or a negative value corresponding to a PMIx error constant.

%%%%
\descr

Given a \refarg{nodename}, return an array of processes within the specified \refarg{nspace}
on that node.
If the \refarg{nspace} is \code{NULL}, then all processes on the node will be returned.
If the specified node does not currently host any processes, then the returned array will be \code{NULL}, and \refarg{nprocs} will be \code{0}.
The caller is responsible for releasing the \refarg{procs} array when done with it.
The \refapi{PMIX_PROC_FREE} macro is provided for this purpose.



%%%%%%%%%%%
\subsection{\code{PMIx_Resolve_nodes}}
\declareapi{PMIx_Resolve_nodes}

%%%%
\summary

Return a list of nodes hosting processes.

%%%%
\format

\cspecificstart
\begin{codepar}
pmix_status_t
PMIx_Resolve_nodes(const char *nspace, char **nodelist)
\end{codepar}
\cspecificend

\begin{arglist}
\argin{nspace}{Namespace (string)}
\argout{nodelist}{Comma-delimited list of nodenames (string)}
\end{arglist}

Returns \refconst{PMIX_SUCCESS} or a negative value corresponding to a PMIx error constant.

%%%%
\descr

Given a \refarg{nspace}, return the list of nodes hosting processes within that namespace.
The returned string will contain a comma-delimited list of nodenames.
The caller is responsible for releasing the string when done with it.


%%%%%%%%%%%
\subsection{\code{PMIx_Query_info_nb}}
\declareapi{PMIx_Query_info_nb}
\declareapi{pmix_info_cbfunc_t}

%%%%
\summary

Query information about the system in general.

%%%%
\format

\cspecificstart
\begin{codepar}
typedef void (*pmix_info_cbfunc_t)(pmix_status_t status,
                                   pmix_info_t *info, size_t ninfo,
                                   void *cbdata,
                                   pmix_release_cbfunc_t release_fn,
                                   void *release_cbdata);

pmix_status_t
PMIx_Query_info_nb(pmix_query_t queries[], size_t nqueries,
                   pmix_info_cbfunc_t cbfunc, void *cbdata)
\end{codepar}
\cspecificend

\begin{arglist}
\argin{queries}{Array of query structures (array of handles)}
\argin{nqueries}{Number of elements in the \refarg{queries} array (integer)}
\argin{cbfunc}{Callback function \refapi{pmix_info_cbfunc_t} (function reference)}
\argin{cbdata}{Data to be passed to the callback function (memory reference)}
\end{arglist}

\begin{constantdesc}
\item \refconst{PMIX_SUCCESS} All data has been returned
\item \refconst{PMIX_ERR_NOT_FOUND} None of the requested data was available
\item \refconst{PMIX_ERR_PARTIAL_SUCCESS} Some of the data has been returned
\item \refconst{PMIX_ERR_NOT_SUPPORTED} The host \ac{RM} does not support this function
\end{constantdesc}

%%%%
\descr

Query information about the system in general.
This can include a list of active namespaces, network topology, etc.
Also can be used to query node-specific info such as the list of peers executing on a given node.
We assume that the host \ac{RM} will exercise appropriate access control on the information.

NOTE: There is no blocking form of this API as the structures passed to query info differ from those for receiving the results.

The \refarg{status} argument to the callback function indicates if requested data was found or not.
An array of \refstruct{pmix_info_t} will contain the key/value pairs.

%%%%%%%%%%%%%%%%%%%%%%%%%%%%%%%%%%%%%%%%%%%%%%%%%


    % Job Allocation Management
    %  - Allocation request, process monitoring
    %%%%%%%%%%%%%%%%%%%%%%%%%%%%%%%%%%%%%%%%%%%%%%%%%
% Chapter: Job Allocation Management
%%%%%%%%%%%%%%%%%%%%%%%%%%%%%%%%%%%%%%%%%%%%%%%%%
\chapter{Job Management and Reporting}
\label{chap:api_job_mgmt}

The job management \acp{API} provide an application with the ability to orchestrate its operation in partnership with the \ac{SMS}.
Members of this category include the \refapi{PMIx_Allocation_request}, \refapi{PMIx_Job_control}, and \refapi{PMIx_Process_monitor} \acp{API}.

%%%%%%%%%%%%%%%%%%%%%%%%%%%%%%%%%%%%%%%%%%%%%%%%%
%%%%%%%%%%%%%%%%%%%%%%%%%%%%%%%%%%%%%%%%%%%%%%%%%
\section{Allocation Requests}
\label{chap:api_job_mgmt:alloc}

This section defines functionality to request new allocations from the \ac{RM}, and request modifications to existing allocations.
These are primarily used in the following scenarios:
\begin{itemize}
\item \textit{Evolving} applications that dynamically request and return resources as they execute.
\item \textit{Malleable} environments where the scheduler redirects resources away from executing applications for higher priority jobs or load balancing.
\item \textit{Resilient} applications that need to request replacement resources in the face of failures.
\item \textit{Rigid} jobs where the user has requested a static allocation of resources for a fixed period of time, but realizes that they underestimated their required time while executing.
\end{itemize}
\ac{PMIx} attempts to address this range of use-cases with a flexible \ac{API}.

%%%%%%%%%%%%%%%%%%%%%%%%%%%%%%%%%%%%%%%%%%%%%%%%%
\subsection{\code{PMIx_Allocation_request}}
\declareapi{PMIx_Allocation_request}

%%%%
\summary

Request an allocation operation from the host resource manager.

%%%%
\format

\copySignature{PMIx_Allocation_request}{3.0}{
pmix_status_t \\
PMIx_Allocation_request(pmix_alloc_directive_t directive, \\
\hspace*{24\sigspace}pmix_info_t info[], size_t ninfo, \\
\hspace*{24\sigspace}pmix_info_t *results[], size_t *nresults);
}

\begin{arglist}
\argin{directive}{Allocation directive (\refstruct{pmix_alloc_directive_t})}
\argin{info}{Array of \refstruct{pmix_info_t} structures (array of handles)}
\argin{ninfo}{Number of elements in the \refarg{info} array (integer)}
\arginout{results}{Address where a pointer to an array of \refstruct{pmix_info_t} containing the results of the request can be returned (memory reference)}
\arginout{nresults}{Address where the number of elements in \refarg{results} can be returned (handle)}
\end{arglist}

Returns one of the following:

\begin{itemize}
    \item \refconst{PMIX_SUCCESS}, indicating that the request was processed and returned \textit{success}
    \item a PMIx error constant indicating either an error in the input or that the request was refused
\end{itemize}

\reqattrstart
\ac{PMIx} libraries are not required to directly support any attributes for this function. However, any provided attributes must be passed to the host \ac{SMS} daemon for processing, and the \ac{PMIx} library is \textit{required} to add the \refAttributeItem{PMIX_USERID} and the \refAttributeItem{PMIX_GRPID} attributes of the client process making the request.

Host environments that implement support for this operation are required to support the following attributes:

\pasteAttributeItem{PMIX_ALLOC_REQ_ID}
\pasteAttributeItem{PMIX_ALLOC_NUM_NODES}
\pasteAttributeItem{PMIX_ALLOC_NUM_CPUS}
\pasteAttributeItem{PMIX_ALLOC_TIME}

\reqattrend

\optattrstart
The following attributes are optional for host environments that support this operation:

\pasteAttributeItem{PMIX_ALLOC_NODE_LIST}
\pasteAttributeItem{PMIX_ALLOC_NUM_CPU_LIST}
\pasteAttributeItem{PMIX_ALLOC_CPU_LIST}
\pasteAttributeItem{PMIX_ALLOC_MEM_SIZE}
\pasteAttributeItem{PMIX_ALLOC_FABRIC}
\pasteAttributeItem{PMIX_ALLOC_FABRIC_ID}
\pasteAttributeItem{PMIX_ALLOC_BANDWIDTH}
\pasteAttributeItem{PMIX_ALLOC_FABRIC_QOS}
\pasteAttributeItem{PMIX_ALLOC_FABRIC_TYPE}
\pasteAttributeItem{PMIX_ALLOC_FABRIC_PLANE}
\pasteAttributeItem{PMIX_ALLOC_FABRIC_ENDPTS}
\pasteAttributeItem{PMIX_ALLOC_FABRIC_ENDPTS_NODE}
\pasteAttributeItem{PMIX_ALLOC_FABRIC_SEC_KEY}

\optattrend

%%%%
\descr

Request an allocation operation from the host resource manager.
Several broad categories are envisioned, including the ability to:

\begin{compactitem}
%
\item Request allocation of additional resources, including memory, bandwidth, and compute.
This should be accomplished in a non-blocking manner so that the application can continue to progress while waiting for resources to become available.
Note that the new allocation will be disjoint from (i.e., not affiliated with) the allocation of the requestor - thus the termination of one allocation will not impact the other.
%
\item Extend the reservation on currently allocated resources, subject to scheduling availability and priorities.
This includes extending the time limit on current resources, and/or requesting additional resources be allocated to the requesting job.
Any additional allocated resources will be considered as part of the current allocation, and thus will be released at the same time.
%
\item Return no-longer-required resources to the scheduler.
This includes the ``loan'' of resources back to the scheduler with a promise to return them upon subsequent request.
\end{compactitem}

If successful, the returned results for a request for additional resources must include the host resource manager's identifier (\refattr{PMIX_ALLOC_ID}) that the requester can use to specify the resources in, for example, a call to \refapi{PMIx_Spawn}.

%%%%%%%%%%%%%%%%%%%%%%%%%%%%%%%%%%%%%%%%%%%%%%%%%
\subsection{\code{PMIx_Allocation_request_nb}}
\declareapi{PMIx_Allocation_request_nb}

%%%%
\summary

Request an allocation operation from the host resource manager.

%%%%
\format

\copySignature{PMIx_Allocation_request_nb}{2.0}{
pmix_status_t \\
PMIx_Allocation_request_nb(pmix_alloc_directive_t directive, \\
\hspace*{27\sigspace}pmix_info_t info[], size_t ninfo, \\
\hspace*{27\sigspace}pmix_info_cbfunc_t cbfunc, void *cbdata);
}

\begin{arglist}
\argin{directive}{Allocation directive (\refstruct{pmix_alloc_directive_t})}
\argin{info}{Array of \refstruct{pmix_info_t} structures (array of handles)}
\argin{ninfo}{Number of elements in the \refarg{info} array (integer)}
\argin{cbfunc}{Callback function \refapi{pmix_info_cbfunc_t} (function reference)}
\argin{cbdata}{Data to be passed to the callback function (memory reference)}
\end{arglist}

Returns one of the following:

\begin{itemize}
    \item \refconst{PMIX_SUCCESS}, indicating that the request is being processed by the host environment - result will be returned in the provided \refarg{cbfunc}. Note that the library must not invoke the callback function prior to returning from the \ac{API}.
    \item \refconst{PMIX_OPERATION_SUCCEEDED}, indicating that the request was immediately processed and returned \textit{success} - the \refarg{cbfunc} will \textit{not} be called
    \item a PMIx error constant indicating either an error in the input or that the request was immediately processed and failed - the \refarg{cbfunc} will \textit{not} be called
\end{itemize}

\reqattrstart
\ac{PMIx} libraries are not required to directly support any attributes for this function. However, any provided attributes must be passed to the host \ac{SMS} daemon for processing, and the \ac{PMIx} library is \textit{required} to add the \refAttributeItem{PMIX_USERID} and the \refAttributeItem{PMIX_GRPID} attributes of the client process making the request.

Host environments that implement support for this operation are required to support the following attributes:

\pasteAttributeItem{PMIX_ALLOC_REQ_ID}
\pasteAttributeItem{PMIX_ALLOC_NUM_NODES}
\pasteAttributeItem{PMIX_ALLOC_NUM_CPUS}
\pasteAttributeItem{PMIX_ALLOC_TIME}

\reqattrend

\optattrstart
The following attributes are optional for host environments that support this operation:

\pasteAttributeItem{PMIX_ALLOC_NODE_LIST}
\pasteAttributeItem{PMIX_ALLOC_NUM_CPU_LIST}
\pasteAttributeItem{PMIX_ALLOC_CPU_LIST}
\pasteAttributeItem{PMIX_ALLOC_MEM_SIZE}
\pasteAttributeItem{PMIX_ALLOC_FABRIC}
\pasteAttributeItem{PMIX_ALLOC_FABRIC_ID}
\pasteAttributeItem{PMIX_ALLOC_BANDWIDTH}
\pasteAttributeItem{PMIX_ALLOC_FABRIC_QOS}
\pasteAttributeItem{PMIX_ALLOC_FABRIC_TYPE}
\pasteAttributeItem{PMIX_ALLOC_FABRIC_PLANE}
\pasteAttributeItem{PMIX_ALLOC_FABRIC_ENDPTS}
\pasteAttributeItem{PMIX_ALLOC_FABRIC_ENDPTS_NODE}
\pasteAttributeItem{PMIX_ALLOC_FABRIC_SEC_KEY}

\optattrend

%%%%
\descr

Non-blocking form of the \refapi{PMIx_Allocation_request} \ac{API}.


%%%%%%%%%%%%%%%%%%%%%%%%%%%%%%%%%%%%%%%%%%%%%%%%%
\subsection{Job Allocation attributes}
\label{api:struct:attributes:joballoc}

Attributes used to describe the job allocation - these are values passed to and/or returned by the \refapi{PMIx_Allocation_request_nb} and \refapi{PMIx_Allocation_request} \acp{API} and are not accessed using the \refapi{PMIx_Get} \ac{API}.

%
\declareAttribute{PMIX_ALLOC_REQ_ID}{"pmix.alloc.reqid"}{char*}{
User-provided string identifier for this allocation request which can later be used to query status of the request.
}
%
\declareAttributeNEW{PMIX_ALLOC_ID}{"pmix.alloc.id"}{char*}{
A string identifier (provided by the host environment) for the resulting allocation which can later be used to reference the allocated resources in, for example, a call to \refapi{PMIx_Spawn}.
}
%
\declareAttributeNEW{PMIX_ALLOC_QUEUE}{"pmix.alloc.queue"}{char*}{
Name of the \ac{WLM} queue to which the allocation request is to be directed, or the queue being referenced in a query.
}
%
\declareAttribute{PMIX_ALLOC_NUM_NODES}{"pmix.alloc.nnodes"}{uint64_t}{
The number of nodes being requested in an allocation request.
}
%
\declareAttribute{PMIX_ALLOC_NODE_LIST}{"pmix.alloc.nlist"}{char*}{
Regular expression of the specific nodes being requested in an allocation request.
}
%
\declareAttribute{PMIX_ALLOC_NUM_CPUS}{"pmix.alloc.ncpus"}{uint64_t}{
Number of \acp{PU} being requested in an allocation request.
}
%
\declareAttribute{PMIX_ALLOC_NUM_CPU_LIST}{"pmix.alloc.ncpulist"}{char*}{
Regular expression of the number of \acp{PU} for each node being requested in an allocation request.
}
%
\declareAttribute{PMIX_ALLOC_CPU_LIST}{"pmix.alloc.cpulist"}{char*}{
Regular expression of the specific \acp{PU}  being requested in an allocation request.
}
%
\declareAttribute{PMIX_ALLOC_MEM_SIZE}{"pmix.alloc.msize"}{float}{
Number of Megabytes[base2] of memory (per process) being requested in an allocation request.
}
%
\declareAttribute{PMIX_ALLOC_FABRIC}{"pmix.alloc.net"}{array}{
Array of \refstruct{pmix_info_t} describing requested fabric resources. This must include at least: \refattr{PMIX_ALLOC_FABRIC_ID}, \refattr{PMIX_ALLOC_FABRIC_TYPE}, and \refattr{PMIX_ALLOC_FABRIC_ENDPTS}, plus whatever other descriptors are desired.
}
%
\declareAttribute{PMIX_ALLOC_FABRIC_ID}{"pmix.alloc.netid"}{char*}{
The key to be used when accessing this requested fabric allocation. The fabric allocation will be returned/stored as a \refstruct{pmix_data_array_t} of \refstruct{pmix_info_t} whose first element is composed of this key and the allocated resource description.
The type of the included value depends upon the fabric support. For example, a \ac{TCP} allocation might consist of a comma-delimited string of socket ranges such as \code{"32000-32100,\allowbreak 33005,38123-38146"}. Additional array entries will consist of any provided resource request directives, along with their assigned values. Examples include: \refattr{PMIX_ALLOC_FABRIC_TYPE} - the type of resources provided; \refattr{PMIX_ALLOC_FABRIC_PLANE} - if applicable, what plane the resources were assigned from; \refattr{PMIX_ALLOC_FABRIC_QOS} - the assigned QoS; \refattr{PMIX_ALLOC_BANDWIDTH} - the allocated bandwidth; \refattr{PMIX_ALLOC_FABRIC_SEC_KEY} - a security key for the requested fabric allocation. NOTE: the array contents may differ from those requested, especially if \refconst{PMIX_INFO_REQD} was not set in the request.
}
%
\declareAttribute{PMIX_ALLOC_BANDWIDTH}{"pmix.alloc.bw"}{float}{
Fabric bandwidth (in Megabits[base2]/sec) for the job being requested in an allocation request.
}
%
\declareAttribute{PMIX_ALLOC_FABRIC_QOS}{"pmix.alloc.netqos"}{char*}{
Fabric quality of service level for the job being requested in an allocation request.
}
%
\declareAttribute{PMIX_ALLOC_TIME}{"pmix.alloc.time"}{uint32_t}{
Total session time (in seconds) being requested in an allocation request.
}
%
\declareAttribute{PMIX_ALLOC_FABRIC_TYPE}{"pmix.alloc.nettype"}{char*}{
Type of desired transport (e.g., \var{``tcp''}, \var{``udp''}) being requested in an allocation request.
}
%
\declareAttribute{PMIX_ALLOC_FABRIC_PLANE}{"pmix.alloc.netplane"}{char*}{
ID string for the \refterm{fabric plane} to be used for the requested allocation.
}
%
\declareAttribute{PMIX_ALLOC_FABRIC_ENDPTS}{"pmix.alloc.endpts"}{size_t}{
Number of endpoints to allocate per \refterm{process} in the job.
}
%
\declareAttribute{PMIX_ALLOC_FABRIC_ENDPTS_NODE}{"pmix.alloc.endpts.nd"}{size_t}{
Number of endpoints to allocate per \refterm{node} for the job.
}
%
\declareAttribute{PMIX_ALLOC_FABRIC_SEC_KEY}{"pmix.alloc.nsec"}{pmix_byte_object_t}{
Request that the allocation include a fabric security key for the spawned job.
}


%%%%%%%%%%%%%%%%%%%%%%%%%%%%%%%%%%%%%%%%%%%%%%%%%
\subsection{Job Allocation Directives}
\declarestruct{pmix_alloc_directive_t}

\versionMarker{2.0}
The \refstruct{pmix_alloc_directive_t} structure is a \code{uint8_t} type that defines the behavior of allocation requests.
The following constants can be used to set a variable of the type \refstruct{pmix_alloc_directive_t}. All definitions were introduced in version 2 of the standard unless otherwise marked.

\begin{constantdesc}
%
\declareconstitem{PMIX_ALLOC_NEW}
A new allocation is being requested.
The resulting allocation will be disjoint (i.e., not connected in a job sense) from the requesting allocation.
%
\declareconstitem{PMIX_ALLOC_EXTEND}
Extend the existing allocation, either in time or as additional resources.
%
\declareconstitem{PMIX_ALLOC_RELEASE}
Release part of the existing allocation.
Attributes in the accompanying \refstruct{pmix_info_t} array may be used to specify permanent release of the identified resources, or ``lending'' of those resources for some period of time.
%
\declareconstitem{PMIX_ALLOC_REAQUIRE}
Reacquire resources that were previously ``lent'' back to the scheduler.
%
\declareconstitem{PMIX_ALLOC_EXTERNAL}
A value boundary above which implementers are free to define their own directive values.
%
\end{constantdesc}



%%%%%%%%%%%%%%%%%%%%%%%%%%%%%%%%%%%%%%%%%%%%%%%%%
%%%%%%%%%%%%%%%%%%%%%%%%%%%%%%%%%%%%%%%%%%%%%%%%%
\section{Job Control}
\label{chap:api_job_mgmt:jctrl}

This section defines \acp{API} that enable the application and host environment to coordinate the response to failures and other events.
This can include requesting termination of the entire job or a subset of processes within a job, but can
also be used in combination with other \ac{PMIx} capabilities (e.g., allocation support and event notification) for more nuanced responses. For example, an application notified of an incipient over-temperature condition on a node could use the \refapi{PMIx_Allocation_request_nb} interface to request replacement nodes while simultaneously using the \refapi{PMIx_Job_control_nb} interface to direct that a checkpoint event be delivered to all processes in the application. If replacement resources are not available, the application might use the \refapi{PMIx_Job_control_nb} interface to request that the job continue at a lower power setting, perhaps sufficient to avoid the over-temperature failure.

The job control \acp{API} can also be used by an application to register itself as available for preemption when operating in an environment such as a cloud or where incentives, financial or otherwise, are provided to jobs willing to be preempted. Registration can include attributes indicating how many resources are being offered for preemption (e.g., all or only some portion), whether the application will require time to prepare for preemption, etc. Jobs that
request a warning will receive an event notifying them of an impending preemption (possibly including information as to the resources that will be taken away, how much time the application will be given prior to being preempted, whether the preemption will be a suspension or full termination, etc.) so they have an opportunity to save
their work. Once the application is ready, it calls the provided event completion callback function to indicate that
the SMS is free to suspend or terminate it, and can include directives regarding any desired restart.

%%%%%%%%%%%%%%%%%%%%%%%%%%%%%%%%%%%%%%%%%%%%%%%%%
\subsection{\code{PMIx_Job_control}}
\declareapi{PMIx_Job_control}

%%%%
\summary

Request a job control action.

%%%%
\format

\copySignature{PMIx_Job_control}{3.0}{
pmix_status_t \\
PMIx_Job_control(const pmix_proc_t targets[], size_t ntargets, \\
\hspace*{17\sigspace}const pmix_info_t directives[], size_t ndirs, \\
\hspace*{17\sigspace}pmix_info_t *results[], size_t *nresults);
}

\begin{arglist}
\argin{targets}{Array of proc structures (array of handles)}
\argin{ntargets}{Number of elements in the \refarg{targets} array (integer)}
\argin{directives}{Array of info structures (array of handles)}
\argin{ndirs}{Number of elements in the \refarg{directives} array (integer)}
\arginout{results}{Address where a pointer to an array of \refstruct{pmix_info_t} containing the results of the request can be returned (memory reference)}
\arginout{nresults}{Address where the number of elements in \refarg{results} can be returned (handle)}
\end{arglist}

Returns one of the following:

\begin{itemize}
    \item \refconst{PMIX_SUCCESS}, indicating that the request was processed by the host environment and returned \textit{success}. Details of the result will be returned in the \refarg{results} array
    \item a \ac{PMIx} error constant indicating either an error in the input or that the request was refused
\end{itemize}

\reqattrstart
\ac{PMIx} libraries are not required to directly support any attributes for this function. However, any provided attributes must be passed to the host \ac{SMS} daemon for processing, and the \ac{PMIx} library is \textit{required} to add the \refAttributeItem{PMIX_USERID} and the \refAttributeItem{PMIX_GRPID} attributes of the client process making the request.

Host environments that implement support for this operation are required to support the following attributes:

\pasteAttributeItem{PMIX_JOB_CTRL_ID}
\pasteAttributeItem{PMIX_JOB_CTRL_PAUSE}
\pasteAttributeItem{PMIX_JOB_CTRL_RESUME}
\pasteAttributeItem{PMIX_JOB_CTRL_KILL}
\pasteAttributeItem{PMIX_JOB_CTRL_SIGNAL}
\pasteAttributeItem{PMIX_JOB_CTRL_TERMINATE}
\pasteAttributeItem{PMIX_REGISTER_CLEANUP}
\pasteAttributeItem{PMIX_REGISTER_CLEANUP_DIR}
\pasteAttributeItem{PMIX_CLEANUP_RECURSIVE}
\pasteAttributeItem{PMIX_CLEANUP_EMPTY}
\pasteAttributeItem{PMIX_CLEANUP_IGNORE}
\pasteAttributeItem{PMIX_CLEANUP_LEAVE_TOPDIR}

\reqattrend

\optattrstart
The following attributes are optional for host environments that support this operation:

\pasteAttributeItem{PMIX_JOB_CTRL_CANCEL}
\pasteAttributeItem{PMIX_JOB_CTRL_RESTART}
\pasteAttributeItem{PMIX_JOB_CTRL_CHECKPOINT}
\pasteAttributeItem{PMIX_JOB_CTRL_CHECKPOINT_EVENT}
\pasteAttributeItem{PMIX_JOB_CTRL_CHECKPOINT_SIGNAL}
\pasteAttributeItem{PMIX_JOB_CTRL_CHECKPOINT_TIMEOUT}
\pasteAttributeItem{PMIX_JOB_CTRL_CHECKPOINT_METHOD}
\pasteAttributeItem{PMIX_JOB_CTRL_PROVISION}
\pasteAttributeItem{PMIX_JOB_CTRL_PROVISION_IMAGE}
\pasteAttributeItem{PMIX_JOB_CTRL_PREEMPTIBLE}

\optattrend

%%%%
\descr

Request a job control action.
The \refarg{targets} array identifies the processes to which the requested job control action is to be applied. All \refterm{clones} of an identified process are to have the requested action applied to them.
A \code{NULL} value can be used to indicate all processes in the caller's namespace.
The use of \refconst{PMIX_RANK_WILDCARD} can also be used to indicate that all processes in the given namespace are to be included.

The directives are provided as \refstruct{pmix_info_t} structures in the \refarg{directives} array.
The returned \refarg{status} indicates whether or not the request was granted, and information as to the reason for any denial of the request shall be returned in the \refarg{results} array.

%%%%%%%%%%%%%%%%%%%%%%%%%%%%%%%%%%%%%%%%%%%%%%%%%
\subsection{\code{PMIx_Job_control_nb}}
\declareapi{PMIx_Job_control_nb}

%%%%
\summary

Request a job control action.

%%%%
\format

\copySignature{PMIx_Job_control_nb}{2.0}{
pmix_status_t \\
PMIx_Job_control_nb(const pmix_proc_t targets[], size_t ntargets, \\
\hspace*{20\sigspace}const pmix_info_t directives[], size_t ndirs, \\
\hspace*{20\sigspace}pmix_info_cbfunc_t cbfunc, void *cbdata);
}

\begin{arglist}
\argin{targets}{Array of proc structures (array of handles)}
\argin{ntargets}{Number of elements in the \refarg{targets} array (integer)}
\argin{directives}{Array of info structures (array of handles)}
\argin{ndirs}{Number of elements in the \refarg{directives} array (integer)}
\argin{cbfunc}{Callback function \refapi{pmix_info_cbfunc_t} (function reference)}
\argin{cbdata}{Data to be passed to the callback function (memory reference)}
\end{arglist}

Returns one of the following:

\begin{itemize}
    \item \refconst{PMIX_SUCCESS}, indicating that the request is being processed by the host environment - result will be returned in the provided \refarg{cbfunc}. Note that the library must not invoke the callback function prior to returning from the \ac{API}.
    \item \refconst{PMIX_OPERATION_SUCCEEDED}, indicating that the request was immediately processed and returned \textit{success} - the \refarg{cbfunc} will \textit{not} be called
    \item a PMIx error constant indicating either an error in the input or that the request was immediately processed and failed - the \refarg{cbfunc} will \textit{not} be called
\end{itemize}

\reqattrstart
\ac{PMIx} libraries are not required to directly support any attributes for this function. However, any provided attributes must be passed to the host \ac{SMS} daemon for processing, and the \ac{PMIx} library is \textit{required} to add the \refAttributeItem{PMIX_USERID} and the \refAttributeItem{PMIX_GRPID} attributes of the client process making the request.

Host environments that implement support for this operation are required to support the following attributes:

\pasteAttributeItem{PMIX_JOB_CTRL_ID}
\pasteAttributeItem{PMIX_JOB_CTRL_PAUSE}
\pasteAttributeItem{PMIX_JOB_CTRL_RESUME}
\pasteAttributeItem{PMIX_JOB_CTRL_KILL}
\pasteAttributeItem{PMIX_JOB_CTRL_SIGNAL}
\pasteAttributeItem{PMIX_JOB_CTRL_TERMINATE}
\pasteAttributeItem{PMIX_REGISTER_CLEANUP}
\pasteAttributeItem{PMIX_REGISTER_CLEANUP_DIR}
\pasteAttributeItem{PMIX_CLEANUP_RECURSIVE}
\pasteAttributeItem{PMIX_CLEANUP_EMPTY}
\pasteAttributeItem{PMIX_CLEANUP_IGNORE}
\pasteAttributeItem{PMIX_CLEANUP_LEAVE_TOPDIR}

\reqattrend

\optattrstart
The following attributes are optional for host environments that support this operation:

\pasteAttributeItem{PMIX_JOB_CTRL_CANCEL}
\pasteAttributeItem{PMIX_JOB_CTRL_RESTART}
\pasteAttributeItem{PMIX_JOB_CTRL_CHECKPOINT}
\pasteAttributeItem{PMIX_JOB_CTRL_CHECKPOINT_EVENT}
\pasteAttributeItem{PMIX_JOB_CTRL_CHECKPOINT_SIGNAL}
\pasteAttributeItem{PMIX_JOB_CTRL_CHECKPOINT_TIMEOUT}
\pasteAttributeItem{PMIX_JOB_CTRL_CHECKPOINT_METHOD}
\pasteAttributeItem{PMIX_JOB_CTRL_PROVISION}
\pasteAttributeItem{PMIX_JOB_CTRL_PROVISION_IMAGE}
\pasteAttributeItem{PMIX_JOB_CTRL_PREEMPTIBLE}

\optattrend

%%%%
\descr

Non-blocking form of the \refapi{PMIx_Job_control} \ac{API}.
The \refarg{targets} array identifies the processes to which the requested job control action is to be applied. All \refterm{clones} of an identified process are to have the requested action applied to them.
A \code{NULL} value can be used to indicate all processes in the caller's namespace.
The use of \refconst{PMIX_RANK_WILDCARD} can also be used to indicate that all processes in the given namespace are to be included.

The directives are provided as \refstruct{pmix_info_t} structures in the \refarg{directives} array.
The callback function provides a \refarg{status} to indicate whether or not the request was granted, and to provide some information as to the reason for any denial in the \refapi{pmix_info_cbfunc_t} array of \refstruct{pmix_info_t} structures.

%%%%%%%%%%%%%%%%%%%%%%%%%%%%%%%%%%%%%%%%%%%%%%%%%
\subsection{Job control constants}
\label{api:struct:constants:jobcontrol}

The following constants are specifically defined for return by the job control \acp{API}:

\begin{constantdesc}

%
\declareconstitemNEW{PMIX_ERR_CONFLICTING_CLEANUP_DIRECTIVES}
Conflicting directives given for job/process cleanup.

\end{constantdesc}

%%%%%%%%%%%%%%%%%%%%%%%%%%%%%%%%%%%%%%%%%%%%%%%%%
\subsection{Job control events}
\label{api:struct:events:jobcontrol}

The following job control events may be available for registration, depending upon implementation and host environment support:

\begin{constantdesc}
%
\declareconstitem{PMIX_JCTRL_CHECKPOINT}
Monitored by \ac{PMIx} client to trigger a checkpoint operation.
%
\declareconstitem{PMIX_JCTRL_CHECKPOINT_COMPLETE}
Sent by a \ac{PMIx} client and monitored by a \ac{PMIx} server to notify that requested checkpoint operation has completed.
%
\declareconstitem{PMIX_JCTRL_PREEMPT_ALERT}
Monitored by a \ac{PMIx} client to detect that an \ac{RM} intends to preempt the job.
%
\declareconstitem{PMIX_ERR_PROC_RESTART}
Error in process restart.
%
\declareconstitem{PMIX_ERR_PROC_CHECKPOINT}
Error in process checkpoint.
%
\declareconstitem{PMIX_ERR_PROC_MIGRATE}
Error in process migration.
%
\end{constantdesc}

%%%%%%%%%%%%%%%%%%%%%%%%%%%%%%%%%%%%%%%%%%%%%%%%%
\subsection{Job control attributes}
\label{api:struct:attributes:jobcontrol}

Attributes used to request control operations on an executing application - these are values passed to the job control \acp{API} and are not accessed using the \refapi{PMIx_Get} \ac{API}.

%
\declareAttribute{PMIX_JOB_CTRL_ID}{"pmix.jctrl.id"}{char*}{
Provide a string identifier for this request. The user can provide an identifier for the requested operation, thus allowing them to later request status of the operation or to terminate it. The host, therefore, shall track it with the request for future reference.
}
%
\declareAttribute{PMIX_JOB_CTRL_PAUSE}{"pmix.jctrl.pause"}{bool}{
Pause the specified processes.
}
%
\declareAttribute{PMIX_JOB_CTRL_RESUME}{"pmix.jctrl.resume"}{bool}{
Resume (``un-pause'') the specified processes.
}
%
\declareAttribute{PMIX_JOB_CTRL_CANCEL}{"pmix.jctrl.cancel"}{char*}{
Cancel the specified request - the provided request ID must match the \refattr{PMIX_JOB_CTRL_ID} provided to a previous call to \refapi{PMIx_Job_control}. An ID of \code{NULL} implies cancel all requests from this requestor.
}
%
\declareAttribute{PMIX_JOB_CTRL_KILL}{"pmix.jctrl.kill"}{bool}{
Forcibly terminate the specified processes and cleanup.
}
%
\declareAttribute{PMIX_JOB_CTRL_RESTART}{"pmix.jctrl.restart"}{char*}{
Restart the specified processes using the given checkpoint ID.
}
%
\declareAttribute{PMIX_JOB_CTRL_CHECKPOINT}{"pmix.jctrl.ckpt"}{char*}{
Checkpoint the specified processes and assign the given ID to it.
}
%
\declareAttribute{PMIX_JOB_CTRL_CHECKPOINT_EVENT}{"pmix.jctrl.ckptev"}{bool}{
Use event notification to trigger a process checkpoint.
}
%
\declareAttribute{PMIX_JOB_CTRL_CHECKPOINT_SIGNAL}{"pmix.jctrl.ckptsig"}{int}{
Use the given signal to trigger a process checkpoint.
}
%
\declareAttribute{PMIX_JOB_CTRL_CHECKPOINT_TIMEOUT}{"pmix.jctrl.ckptsig"}{int}{
Time in seconds to wait for a checkpoint to complete.
}
%
\declareAttribute{PMIX_JOB_CTRL_CHECKPOINT_METHOD}{"pmix.jctrl.ckmethod"}{pmix_data_array_t}{
Array of \refstruct{pmix_info_t} declaring each method and value supported by this application.
}
%
\declareAttribute{PMIX_JOB_CTRL_SIGNAL}{"pmix.jctrl.sig"}{int}{
Send given signal to specified processes.
}
%
\declareAttribute{PMIX_JOB_CTRL_PROVISION}{"pmix.jctrl.pvn"}{char*}{
Regular expression identifying nodes that are to be provisioned.
}
%
\declareAttribute{PMIX_JOB_CTRL_PROVISION_IMAGE}{"pmix.jctrl.pvnimg"}{char*}{
Name of the image that is to be provisioned.
}
%
\declareAttribute{PMIX_JOB_CTRL_PREEMPTIBLE}{"pmix.jctrl.preempt"}{bool}{
Indicate that the job can be pre-empted.
}
%
\declareAttribute{PMIX_JOB_CTRL_TERMINATE}{"pmix.jctrl.term"}{bool}{
Politely terminate the specified processes.
}
%
\declareAttribute{PMIX_REGISTER_CLEANUP}{"pmix.reg.cleanup"}{char*}{
Comma-delimited list of files to be removed upon process termination.
}
%
\declareAttribute{PMIX_REGISTER_CLEANUP_DIR}{"pmix.reg.cleanupdir"}{char*}{
Comma-delimited list of directories to be removed upon process termination.
}
%
\declareAttribute{PMIX_CLEANUP_RECURSIVE}{"pmix.clnup.recurse"}{bool}{
Recursively cleanup all subdirectories under the specified one(s).
}
%
\declareAttribute{PMIX_CLEANUP_EMPTY}{"pmix.clnup.empty"}{bool}{
Only remove empty subdirectories.
}
%
\declareAttribute{PMIX_CLEANUP_IGNORE}{"pmix.clnup.ignore"}{char*}{
Comma-delimited list of filenames that are not to be removed.
}
%
\declareAttribute{PMIX_CLEANUP_LEAVE_TOPDIR}{"pmix.clnup.lvtop"}{bool}{
When recursively cleaning subdirectories, do not remove the top-level directory (the one given in the cleanup request).
}


%%%%%%%%%%%%%%%%%%%%%%%%%%%%%%%%%%%%%%%%%%%%%%%%%
%%%%%%%%%%%%%%%%%%%%%%%%%%%%%%%%%%%%%%%%%%%%%%%%%
\section{Process and Job Monitoring}
\label{chap:api_job_mgmt:monitor}

In addition to external faults, a common problem encountered in \ac{HPC} applications is a failure to make
progress due to some internal conflict in the computation. These situations can
result in a significant waste of resources as the \ac{SMS} is unaware of the problem, and thus cannot terminate the
job. Various watchdog methods have been developed for detecting this situation, including requiring a periodic ``heartbeat''
from the application and monitoring a specified file for changes in size and/or modification time.

The following \acp{API} allow applications to request monitoring, directing what is to be monitored, the frequency of the associated check, whether or not the application is to be notified (via the event notification subsystem) of stall detection, and other characteristics of the operation.

%%%%%%%%%%%%%%%%%%%%%%%%%%%%%%%%%%%%%%%%%%%%%%%%%
\subsection{\code{PMIx_Process_monitor}}
\declareapi{PMIx_Process_monitor}

%%%%
\summary

Request that application processes be monitored.

%%%%
\format

\copySignature{PMIx_Process_monitor}{3.0}{
pmix_status_t \\
PMIx_Process_monitor(const pmix_info_t *monitor, \\
\hspace*{21\sigspace}pmix_status_t error, \\
\hspace*{21\sigspace}const pmix_info_t directives[], size_t ndirs, \\
\hspace*{21\sigspace}pmix_info_t *results[], size_t *nresults);
}

\begin{arglist}
\argin{monitor}{info (handle)}
\argin{error}{status (integer)}
\argin{directives}{Array of info structures (array of handles)}
\argin{ndirs}{Number of elements in the \refarg{directives} array (integer)}
\arginout{results}{Address where a pointer to an array of \refstruct{pmix_info_t} containing the results of the request can be returned (memory reference)}
\arginout{nresults}{Address where the number of elements in \refarg{results} can be returned (handle)}
\end{arglist}

Returns one of the following:

\begin{itemize}
    \item \refconst{PMIX_SUCCESS}, indicating that the request was processed and returned \textit{success}. Details of the result will be returned in the \refarg{results} array
    \item a PMIx error constant indicating either an error in the input or that the request was refused
\end{itemize}

\optattrstart
The following attributes may be implemented by a \ac{PMIx} library or by the host environment. If supported by the \ac{PMIx} server library, then the library must not pass the supported attributes to the host environment. All attributes not directly supported by the server library must be passed to the host environment if it supports this operation, and the library is \textit{required} to add the \refAttributeItem{PMIX_USERID} and the \refAttributeItem{PMIX_GRPID} attributes of the requesting process:

\pasteAttributeItem{PMIX_MONITOR_ID}
\pasteAttributeItem{PMIX_MONITOR_CANCEL}
\pasteAttributeItem{PMIX_MONITOR_APP_CONTROL}
\pasteAttributeItem{PMIX_MONITOR_HEARTBEAT}
\pasteAttributeItem{PMIX_MONITOR_HEARTBEAT_TIME}
\pasteAttributeItem{PMIX_MONITOR_HEARTBEAT_DROPS}
\pasteAttributeItem{PMIX_MONITOR_FILE}
\pasteAttributeItem{PMIX_MONITOR_FILE_SIZE}
\pasteAttributeItem{PMIX_MONITOR_FILE_ACCESS}
\pasteAttributeItem{PMIX_MONITOR_FILE_MODIFY}
\pasteAttributeItem{PMIX_MONITOR_FILE_CHECK_TIME}
\pasteAttributeItem{PMIX_MONITOR_FILE_DROPS}
\pasteAttributeItem{PMIX_SEND_HEARTBEAT}

\optattrend

%%%%
\descr

Request that application processes be monitored via several possible methods.
For example, that the server monitor this process for periodic heartbeats as an indication that the process has not become ``wedged''.
When a monitor detects the specified alarm condition, it will generate an event notification using the provided error code and passing along any available relevant information.
It is up to the caller to register a corresponding event handler.

The \refarg{monitor} argument is an attribute indicating the type of monitor being requested.
For example, \refattr{PMIX_MONITOR_FILE} to indicate that the requestor is asking that a file be monitored.

The \refarg{error} argument is the status code to be used when generating an event notification alerting that the monitor has been triggered.
The range of the notification defaults to \refconst{PMIX_RANGE_NAMESPACE}.
This can be changed by providing a \refattr{PMIX_RANGE} directive.

The \refarg{directives} argument characterizes the monitoring request (e.g., monitor file size) and frequency of checking to be done

The returned \refarg{status} indicates whether or not the request was granted, and information as to the reason for any denial of the request shall be returned in the \refarg{results} array.

%%%%%%%%%%%%%%%%%%%%%%%%%%%%%%%%%%%%%%%%%%%%%%%%%
\subsection{\code{PMIx_Process_monitor_nb}}
\declareapi{PMIx_Process_monitor_nb}

%%%%
\summary

Request that application processes be monitored.

%%%%
\format

\copySignature{PMIx_Process_monitor_nb}{2.0}{
pmix_status_t \\
PMIx_Process_monitor_nb(const pmix_info_t *monitor, \\
\hspace*{24\sigspace}pmix_status_t error, \\
\hspace*{24\sigspace}const pmix_info_t directives[], \\
\hspace*{24\sigspace}size_t ndirs, \\
\hspace*{24\sigspace}pmix_info_cbfunc_t cbfunc, void *cbdata);
}

\begin{arglist}
\argin{monitor}{info (handle)}
\argin{error}{status (integer)}
\argin{directives}{Array of info structures (array of handles)}
\argin{ndirs}{Number of elements in the \refarg{directives} array (integer)}
\argin{cbfunc}{Callback function \refapi{pmix_info_cbfunc_t} (function reference)}
\argin{cbdata}{Data to be passed to the callback function (memory reference)}
\end{arglist}

Returns one of the following:

\begin{itemize}
    \item \refconst{PMIX_SUCCESS}, indicating that the request is being processed by the host environment - result will be returned in the provided \refarg{cbfunc}. Note that the library must not invoke the callback function prior to returning from the \ac{API}.
    \item \refconst{PMIX_OPERATION_SUCCEEDED}, indicating that the request was immediately processed and returned \textit{success} - the \refarg{cbfunc} will \textit{not} be called.
    \item a PMIx error constant indicating either an error in the input or that the request was immediately processed and failed - the \refarg{cbfunc} will \textit{not} be called.
\end{itemize}

\optattrstart
The following attributes may be implemented by a \ac{PMIx} library or by the host environment. If supported by the \ac{PMIx} server library, then the library must not pass the supported attributes to the host environment. All attributes not directly supported by the server library must be passed to the host environment if it supports this operation, and the library is \textit{required} to add the \refAttributeItem{PMIX_USERID} and the \refAttributeItem{PMIX_GRPID} attributes of the requesting process:

\pasteAttributeItem{PMIX_MONITOR_ID}
\pasteAttributeItem{PMIX_MONITOR_CANCEL}
\pasteAttributeItem{PMIX_MONITOR_APP_CONTROL}
\pasteAttributeItem{PMIX_MONITOR_HEARTBEAT}
\pasteAttributeItem{PMIX_MONITOR_HEARTBEAT_TIME}
\pasteAttributeItem{PMIX_MONITOR_HEARTBEAT_DROPS}
\pasteAttributeItem{PMIX_MONITOR_FILE}
\pasteAttributeItem{PMIX_MONITOR_FILE_SIZE}
\pasteAttributeItem{PMIX_MONITOR_FILE_ACCESS}
\pasteAttributeItem{PMIX_MONITOR_FILE_MODIFY}
\pasteAttributeItem{PMIX_MONITOR_FILE_CHECK_TIME}
\pasteAttributeItem{PMIX_MONITOR_FILE_DROPS}
\pasteAttributeItem{PMIX_SEND_HEARTBEAT}

\optattrend

%%%%
\descr

Non-blocking form of the \refapi{PMIx_Process_monitor} \ac{API}. The \refarg{cbfunc} function provides a \refarg{status} to indicate whether or not the request was granted, and to provide some information as to the reason for any denial in the \refapi{pmix_info_cbfunc_t} array of \refstruct{pmix_info_t} structures.

%%%%%%%%%%%%%%%%%%%%%%%%%%%%%%%%%%%%%%%%%%%%%%%%%
\subsection{\code{PMIx_Heartbeat}}
\declaremacro{PMIx_Heartbeat}

%%%%
\summary

Send a heartbeat to the \ac{PMIx} server library

%%%%
\format

\copySignature{PMIx_Heartbeat}{2.0}{
PMIx_Heartbeat();
}


%%%%
\descr

A simplified macro wrapping \refapi{PMIx_Process_monitor_nb} that sends a heartbeat to the \ac{PMIx} server library.

%%%%%%%%%%%%%%%%%%%%%%%%%%%%%%%%%%%%%%%%%%%%%%%%%
\subsection{Monitoring events}
\label{api:struct:events:monitor}

The following monitoring events may be available for registration, depending upon implementation and host environment support:

\begin{constantdesc}
%
\declareconstitem{PMIX_MONITOR_HEARTBEAT_ALERT}
Heartbeat failed to arrive within specified window. The process that triggered this alert will be identified in the event.
%
\declareconstitem{PMIX_MONITOR_FILE_ALERT}
File failed its monitoring detection criteria. The file that triggered this alert will be identified in the event.
%
\end{constantdesc}

%%%%%%%%%%%
\subsection{Monitoring attributes}
\label{api:struct:attributes:monitor}

Attributes used to control monitoring of an executing application- these are values passed to the \refapi{PMIx_Process_monitor_nb} \ac{API} and are not accessed using the \refapi{PMIx_Get} \ac{API}.

%
\declareAttribute{PMIX_MONITOR_ID}{"pmix.monitor.id"}{char*}{
Provide a string identifier for this request.
}
%
\declareAttribute{PMIX_MONITOR_CANCEL}{"pmix.monitor.cancel"}{char*}{
Identifier to be canceled (\code{NULL} means cancel all monitoring for this process).
}
%
\declareAttribute{PMIX_MONITOR_APP_CONTROL}{"pmix.monitor.appctrl"}{bool}{
The application desires to control the response to a monitoring event - i.e., the application is requesting that the host environment not take immediate action in response to the event (e.g., terminating the job).
}
%
\declareAttribute{PMIX_MONITOR_HEARTBEAT}{"pmix.monitor.mbeat"}{void}{
Register to have the PMIx server monitor the requestor for heartbeats.
}
%
\declareAttribute{PMIX_SEND_HEARTBEAT}{"pmix.monitor.beat"}{void}{
Send heartbeat to local PMIx server.
}
%
\declareAttribute{PMIX_MONITOR_HEARTBEAT_TIME}{"pmix.monitor.btime"}{uint32_t}{
Time in seconds before declaring heartbeat missed.
}
%
\declareAttribute{PMIX_MONITOR_HEARTBEAT_DROPS}{"pmix.monitor.bdrop"}{uint32_t}{
Number of heartbeats that can be missed before generating the event.
}
%
\declareAttribute{PMIX_MONITOR_FILE}{"pmix.monitor.fmon"}{char*}{
Register to monitor file for signs of life.
}
%
\declareAttribute{PMIX_MONITOR_FILE_SIZE}{"pmix.monitor.fsize"}{bool}{
Monitor size of given file is growing to determine if the application is running.
}
%
\declareAttribute{PMIX_MONITOR_FILE_ACCESS}{"pmix.monitor.faccess"}{char*}{
Monitor time since last access of given file to determine if the application is running.
}
%
\declareAttribute{PMIX_MONITOR_FILE_MODIFY}{"pmix.monitor.fmod"}{char*}{
Monitor time since last modified of given file to determine if the application is running.
}
%
\declareAttribute{PMIX_MONITOR_FILE_CHECK_TIME}{"pmix.monitor.ftime"}{uint32_t}{
Time in seconds between checking the file.
}
%
\declareAttribute{PMIX_MONITOR_FILE_DROPS}{"pmix.monitor.fdrop"}{uint32_t}{
Number of file checks that can be missed before generating the event.
}

%%%%%%%%%%%%%%%%%%%%%%%%%%%%%%%%%%%%%%%%%%%%%%%%%
%%%%%%%%%%%%%%%%%%%%%%%%%%%%%%%%%%%%%%%%%%%%%%%%%
\section{Logging}
\label{chap:api_job_mgmt:logging}

The logging interface supports posting information by applications and SMS elements to persistent storage. This function is \textit{not} intended for output of computational results, but rather for reporting status and saving state information such as inserting computation progress reports into the application's \ac{SMS} job log or error reports to the local syslog.

%%%%%%%%%%%%%%%%%%%%%%%%%%%%%%%%%%%%%%%%%%%%%%%%%
\subsection{\code{PMIx_Log}}
\declareapi{PMIx_Log}

%%%%
\summary

Log data to a data service.

%%%%
\format

\copySignature{PMIx_Log}{3.0}{
pmix_status_t \\
PMIx_Log(const pmix_info_t data[], size_t ndata, \\
\hspace*{9\sigspace}const pmix_info_t directives[], size_t ndirs);
}

\begin{arglist}
\argin{data}{Array of info structures (array of handles)}
\argin{ndata}{Number of elements in the \refarg{data} array (\code{size_t})}
\argin{directives}{Array of info structures (array of handles)}
\argin{ndirs}{Number of elements in the \refarg{directives} array (\code{size_t})}
\end{arglist}

Return codes are one of the following:

\begin{constantdesc}
    \item \refconst{PMIX_SUCCESS} The logging request was successful.
    \item \refconst{PMIX_ERR_BAD_PARAM} The logging request contains at least one incorrect entry.
    \item \refconst{PMIX_ERR_NOT_SUPPORTED} The \ac{PMIx} implementation or host environment does not support this function.
    \item other appropriate \ac{PMIx} error code
\end{constantdesc}

\reqattrstart
If the \ac{PMIx} library does not itself perform this operation, then it is required to pass any attributes provided by the client to the host environment for processing. In addition, it must include the following attributes in the passed \refarg{info} array:

\pasteAttributeItem{PMIX_USERID}
\pasteAttributeItem{PMIX_GRPID}

Host environments or \ac{PMIx} libraries that implement support for this operation are required to support the following attributes:

\pasteAttributeItem{PMIX_LOG_STDERR}
\pasteAttributeItem{PMIX_LOG_STDOUT}
\pasteAttributeItem{PMIX_LOG_SYSLOG}
\pasteAttributeItem{PMIX_LOG_LOCAL_SYSLOG}
\pasteAttributeItem{PMIX_LOG_GLOBAL_SYSLOG}
\pasteAttributeItem{PMIX_LOG_SYSLOG_PRI}
\pasteAttributeItem{PMIX_LOG_ONCE}

\reqattrend

\optattrstart
The following attributes are optional for host environments or \ac{PMIx} libraries that support this operation:

\pasteAttributeItem{PMIX_LOG_SOURCE}
\pasteAttributeItem{PMIX_LOG_TIMESTAMP}
\pasteAttributeItem{PMIX_LOG_GENERATE_TIMESTAMP}
\pasteAttributeItem{PMIX_LOG_TAG_OUTPUT}
\pasteAttributeItem{PMIX_LOG_TIMESTAMP_OUTPUT}
\pasteAttributeItem{PMIX_LOG_XML_OUTPUT}
\pasteAttributeItem{PMIX_LOG_EMAIL}
\pasteAttributeItem{PMIX_LOG_EMAIL_ADDR}
\pasteAttributeItem{PMIX_LOG_EMAIL_SENDER_ADDR}
\pasteAttributeItem{PMIX_LOG_EMAIL_SERVER}
\pasteAttributeItem{PMIX_LOG_EMAIL_SRVR_PORT}
\pasteAttributeItem{PMIX_LOG_EMAIL_SUBJECT}
\pasteAttributeItem{PMIX_LOG_EMAIL_MSG}
\pasteAttributeItem{PMIX_LOG_JOB_RECORD}
\pasteAttributeItem{PMIX_LOG_GLOBAL_DATASTORE}

\optattrend

%%%%
\descr

Log data subject to the services offered by the host environment. The data to be logged is provided in the \refarg{data} array. The (optional) \refarg{directives} can be used to direct the choice of logging channel.

\adviceuserstart
It is strongly recommended that the \refapi{PMIx_Log} API not be used by applications for streaming data as it is not a ``performant'' transport and can perturb the application since it involves the local \ac{PMIx} server and host \ac{SMS} daemon. Note that a return of \refconst{PMIX_SUCCESS} only denotes that the data was successfully handed to the appropriate system call (for local channels) or the host environment and does not indicate receipt at the final destination.
\adviceuserend

%%%%%%%%%%%%%%%%%%%%%%%%%%%%%%%%%%%%%%%%%%%%%%%%%
\subsection{\code{PMIx_Log_nb}}
\declareapi{PMIx_Log_nb}

%%%%
\summary

Log data to a data service.

%%%%
\format

\copySignature{PMIx_Log_nb}{2.0}{
pmix_status_t \\
PMIx_Log_nb(const pmix_info_t data[], size_t ndata, \\
\hspace*{12\sigspace}const pmix_info_t directives[], size_t ndirs, \\
\hspace*{12\sigspace}pmix_op_cbfunc_t cbfunc, void *cbdata);
}

\begin{arglist}
\argin{data}{Array of info structures (array of handles)}
\argin{ndata}{Number of elements in the \refarg{data} array (\code{size_t})}
\argin{directives}{Array of info structures (array of handles)}
\argin{ndirs}{Number of elements in the \refarg{directives} array (\code{size_t})}
\argin{cbfunc}{Callback function \refapi{pmix_op_cbfunc_t} (function reference)}
\argin{cbdata}{Data to be passed to the callback function (memory reference)}
\end{arglist}

Return codes are one of the following:

\begin{constantdesc}
\item \refconst{PMIX_SUCCESS} The logging request is valid and is being processed. The resulting status from the operation will be provided in the callback function. Note that the library must not invoke the callback function prior to returning from the \ac{API}.
\item \refconst{PMIX_OPERATION_SUCCEEDED}, indicating that the request was immediately processed and returned \textit{success} - the \refarg{cbfunc} will \textit{not} be called
\item \refconst{PMIX_ERR_BAD_PARAM} The logging request contains at least one incorrect entry that prevents it from being processed. The callback function will not be called.
\item \refconst{PMIX_ERR_NOT_SUPPORTED} The \ac{PMIx} implementation does not support this function. The callback function will not be called.
\item other appropriate \ac{PMIx} error code - the callback function will not be called.
\end{constantdesc}

\reqattrstart
If the \ac{PMIx} library does not itself perform this operation, then it is required to pass any attributes provided by the client to the host environment for processing. In addition, it must include the following attributes in the passed \refarg{info} array:

\pasteAttributeItem{PMIX_USERID}
\pasteAttributeItem{PMIX_GRPID}

Host environments or \ac{PMIx} libraries that implement support for this operation are required to support the following attributes:

\pasteAttributeItem{PMIX_LOG_STDERR}
\pasteAttributeItem{PMIX_LOG_STDOUT}
\pasteAttributeItem{PMIX_LOG_SYSLOG}
\pasteAttributeItem{PMIX_LOG_LOCAL_SYSLOG}
\pasteAttributeItem{PMIX_LOG_GLOBAL_SYSLOG}
\pasteAttributeItem{PMIX_LOG_SYSLOG_PRI}
\pasteAttributeItem{PMIX_LOG_ONCE}

\reqattrend

\optattrstart
The following attributes are optional for host environments or \ac{PMIx} libraries that support this operation:

\pasteAttributeItem{PMIX_LOG_SOURCE}
\pasteAttributeItem{PMIX_LOG_TIMESTAMP}
\pasteAttributeItem{PMIX_LOG_GENERATE_TIMESTAMP}
\pasteAttributeItem{PMIX_LOG_TAG_OUTPUT}
\pasteAttributeItem{PMIX_LOG_TIMESTAMP_OUTPUT}
\pasteAttributeItem{PMIX_LOG_XML_OUTPUT}
\pasteAttributeItem{PMIX_LOG_EMAIL}
\pasteAttributeItem{PMIX_LOG_EMAIL_ADDR}
\pasteAttributeItem{PMIX_LOG_EMAIL_SENDER_ADDR}
\pasteAttributeItem{PMIX_LOG_EMAIL_SERVER}
\pasteAttributeItem{PMIX_LOG_EMAIL_SRVR_PORT}
\pasteAttributeItem{PMIX_LOG_EMAIL_SUBJECT}
\pasteAttributeItem{PMIX_LOG_EMAIL_MSG}
\pasteAttributeItem{PMIX_LOG_JOB_RECORD}
\pasteAttributeItem{PMIX_LOG_GLOBAL_DATASTORE}

\optattrend

%%%%
\descr

Log data subject to the services offered by the host environment. The data to be logged is provided in the \refarg{data} array. The (optional) \refarg{directives} can be used to direct the choice of logging channel.
The callback function will be executed when the log operation has been completed. The \refarg{data} and \refarg{directives} arrays must be maintained until the callback is provided.

\adviceuserstart
It is strongly recommended that the \refapi{PMIx_Log_nb} API not be used by applications for streaming data as it is not a ``performant'' transport and can perturb the application since it involves the local \ac{PMIx} server and host \ac{SMS} daemon. Note that a return of \refconst{PMIX_SUCCESS} only denotes that the data was successfully handed to the appropriate system call (for local channels) or the host environment and does not indicate receipt at the final destination.
\adviceuserend


%%%%%%%%%%%%%%%%%%%%%%%%%%%%%%%%%%%%%%%%%%%%%%%%%
\subsection{Log attributes}
\label{api:struct:attributes:log}

Attributes used to describe \refapi{PMIx_Log} behavior - these are values passed to the \refapi{PMIx_Log} \ac{API} and therefore are not accessed using the \refapi{PMIx_Get} \ac{API}.

%
\declareAttribute{PMIX_LOG_SOURCE}{"pmix.log.source"}{pmix_proc_t*}{
ID of source of the log request.
}
%
\declareAttribute{PMIX_LOG_STDERR}{"pmix.log.stderr"}{char*}{
Log string to \code{stderr}.
}
%
\declareAttribute{PMIX_LOG_STDOUT}{"pmix.log.stdout"}{char*}{
Log string to \code{stdout}.
}
%
\declareAttribute{PMIX_LOG_SYSLOG}{"pmix.log.syslog"}{char*}{
Log data to syslog.
Defaults to \code{ERROR} priority.  Will log to global syslog if available, otherwise to local syslog.
}
%
\declareAttribute{PMIX_LOG_LOCAL_SYSLOG}{"pmix.log.lsys"}{char*}{
Log data to local syslog.
Defaults to \code{ERROR} priority.
}
%
\declareAttribute{PMIX_LOG_GLOBAL_SYSLOG}{"pmix.log.gsys"}{char*}{
Forward data to system ``gateway'' and log msg to that syslog
Defaults to \code{ERROR} priority.
}
%
\declareAttribute{PMIX_LOG_SYSLOG_PRI}{"pmix.log.syspri"}{int}{
Syslog priority level.
}
%
\declareAttribute{PMIX_LOG_TIMESTAMP}{"pmix.log.tstmp"}{time_t}{
Timestamp for log report.
}
%
\declareAttribute{PMIX_LOG_GENERATE_TIMESTAMP}{"pmix.log.gtstmp"}{bool}{
Generate timestamp for log.
}
%
\declareAttribute{PMIX_LOG_TAG_OUTPUT}{"pmix.log.tag"}{bool}{
Label the output stream with the channel name (e.g., ``stdout'').
}
%
\declareAttribute{PMIX_LOG_TIMESTAMP_OUTPUT}{"pmix.log.tsout"}{bool}{
Print timestamp in output string.
}
%
\declareAttribute{PMIX_LOG_XML_OUTPUT}{"pmix.log.xml"}{bool}{
Print the output stream in \ac{XML} format.
}
%
\declareAttribute{PMIX_LOG_ONCE}{"pmix.log.once"}{bool}{
Only log this once with whichever channel can first support it, taking the channels in priority order.
}
%
\declareAttribute{PMIX_LOG_MSG}{"pmix.log.msg"}{pmix_byte_object_t}{
Message blob to be sent somewhere.
}
%
\declareAttribute{PMIX_LOG_EMAIL}{"pmix.log.email"}{pmix_data_array_t}{
Log via email based on \refstruct{pmix_info_t} containing directives.
}
%
\declareAttribute{PMIX_LOG_EMAIL_ADDR}{"pmix.log.emaddr"}{char*}{
Comma-delimited list of email addresses that are to receive the message.
}
%
\declareAttribute{PMIX_LOG_EMAIL_SENDER_ADDR}{"pmix.log.emfaddr"}{char*}{
Return email address of sender.
}
%
\declareAttribute{PMIX_LOG_EMAIL_SUBJECT}{"pmix.log.emsub"}{char*}{
Subject line for email.
}
%
\declareAttribute{PMIX_LOG_EMAIL_MSG}{"pmix.log.emmsg"}{char*}{
Message to be included in email.
}
%
\declareAttribute{PMIX_LOG_EMAIL_SERVER}{"pmix.log.esrvr"}{char*}{
Hostname (or \ac{IP} address) of SMTP server.
}
%
\declareAttribute{PMIX_LOG_EMAIL_SRVR_PORT}{"pmix.log.esrvrprt"}{int32_t}{
Port the email server is listening to.
}
%
\declareAttribute{PMIX_LOG_GLOBAL_DATASTORE}{"pmix.log.gstore"}{bool}{
Store the log data in a global data store (e.g., database).
}
%
\declareAttribute{PMIX_LOG_JOB_RECORD}{"pmix.log.jrec"}{bool}{
Log the provided information to the host environment's job record.
}

%%%%%%%%%%%%%%%%%%%%%%%%%%%%%%%%%%%%%%%%%%%%%%%%%


    % Event Handling
    %  - (de)register_event, notify_event
    %%%%%%%%%%%%%%%%%%%%%%%%%%%%%%%%%%%%%%%%%%%%%%%%%
% Chapter: Events
%%%%%%%%%%%%%%%%%%%%%%%%%%%%%%%%%%%%%%%%%%%%%%%%%
\chapter{Event Notification}
\label{chap:api_event}

This chapter defines the \ac{PMIx} event notification system.
These interfaces are designed to support the reporting of events to/from clients and servers, and between library layers within a single process.

%%%%%%%%%%%%%%%%%%%%%%%%%%%%%%%%%%%%%%%%%%%%%%%%%
%%%%%%%%%%%%%%%%%%%%%%%%%%%%%%%%%%%%%%%%%%%%%%%%%
\section{Notification and Management}
\label{chap:api_event:notify}

\ac{PMIx} event notification provides an asynchronous out-of-band mechanism for communicating events between application processes and/or elements of the \ac{SMS}. Its uses span a wide range including fault notification, coordination between multiple programming libraries within a single process, and workflow orchestration for non-synchronous programming models. Events can be divided into two distinct classes:

\begin{itemize}
\item \textit{Job-specific events} directly relate to a job executing within the session, such as a debugger attachment, process failure within a related job, or events generated by an application process. Events in this category are to be immediately delivered to the \ac{PMIx} server library for relay to the related local processes.

\item \textit{Environment events} indirectly relate to a job but do not specifically target the job itself. This category includes \ac{SMS}-generated events such as \ac{ECC} errors, temperature excursions, and other non-job conditions that might directly affect a session's resources, but would never include an event generated by an application process. Note that although these do potentially impact the session's jobs, they are not directly tied to those jobs. Thus, events in this category are to be delivered to the \ac{PMIx} server library only upon request.
\end{itemize}

Both \ac{SMS} elements and applications can register for events of either type.

\adviceimplstart
Race conditions can cause the registration to come after events of possible interest (e.g., a memory \ac{ECC} event that occurs after start of execution but prior to registration, or an application process generating an event prior to another process registering to receive it). \ac{SMS} vendors are \textit{requested} to cache environment events for some time to mitigate this situation, but are not \textit{required} to do so. However, \ac{PMIx} implementers are \textit{required} to cache all events received by the \ac{PMIx} server library and to deliver them to registering clients in the same order in which they were received
\adviceimplend

\adviceuserstart
Applications must be aware that they may not receive environment events that occur prior to registration, depending upon the capabilities of the host \ac{SMS}.
\adviceuserend

The generator of an event can specify the \textit{target range} for delivery of that event. Thus, the generator can choose to limit notification to processes on the local node, processes within the same job as the generator, processes within the same allocation, other threads within the same process, only the \ac{SMS} (i.e., not to any application processes), all application processes, or to a custom range based on specific process identifiers. Only processes within the given range that register for the provided event code will be notified. In addition, the generator can use attributes to direct that the event not be delivered to any default event handlers, or to any multi-code handler (as defined below).

Event notifications provide the process identifier of the source of the event plus the event code and any additional information provided by the generator. When an event notification is received by a process, the registered handlers are scanned for their event code(s), with matching handlers assembled into an \textit{event chain} for servicing. Note that users can also specify a \textit{source range} when registering an event (using the same range designators described above) to further limit when they are to be invoked. When assembled, PMIx event chains are ordered based on both the specificity of the event handler and user directives at time of handler registration. By default, handlers are grouped into three categories based on the number of event codes that can trigger the callback:
\begin{itemize}
%
\item \textit{single-code} handlers are serviced first as they are the most specific. These are handlers that are registered against one specific event code.
%
\item \textit{multi-code} handlers are serviced once all single-code handlers have completed. The handler will be included in the chain upon receipt of an event matching any of the provided codes.
%
\item \textit{default} handlers are serviced once all multi-code handlers have completed. These handlers are always included in the chain unless the generator specifically excludes them.
%
\end{itemize}

Users can specify the callback order of a handler within its category at the time of registration. Ordering can be specified either by providing the relevant returned event handler registration ID or using event handler names, if the user specified an event handler name when registering the corresponding event. Thus, users can specify that a given handler be executed before or after another handler should both handlers appear in an event chain (the ordering is ignored if the other handler isn't included). Note that ordering does not imply immediate relationships. For example, multiple handlers registered to be serviced after event handler \textit{A} will all be executed after \textit{A}, but are not guaranteed to be executed in any particular order amongst themselves.

In addition, one event handler can be declared as the \textit{first} handler to be executed in the chain. This handler will \textit{always} be called prior to any other handler, regardless of category, provided the incoming event matches both the specified range and event code. Only one handler can be so designated --- attempts to designate additional handlers as \textit{first} will return an error. Deregistration of the declared \textit{first} handler will re-open the position for subsequent assignment.

Similarly, one event handler can be declared as the \textit{last} handler to be executed in the chain. This handler will \textit{always} be called after all other handlers have executed, regardless of category, provided the incoming event matches both the specified range and event code. Note that this handler will not be called if the chain is terminated by an earlier handler. Only one handler can be designated as \textit{last} --- attempts to designate additional handlers as \textit{last} will return an error. Deregistration of the declared \textit{last} handler will re-open the position for subsequent assignment.

\adviceuserstart
Note that the \textit{last} handler is called \textit{after} all registered default handlers that match the specified range of the incoming event unless a handler prior to it terminates the chain. Thus, if the application intends to define a \textit{last} handler, it should ensure that no default handler aborts the process before it.
\adviceuserend

Upon completing its work and prior to returning, each handler \textit{must} call the event handler completion function provided when it was invoked (including a status code plus any information to be passed to later handlers) so that the chain can continue being progressed. \ac{PMIx} automatically aggregates the status and any results of each handler (as provided in the completion callback) with status from all prior handlers so that each step in the chain has full knowledge of what preceded it. An event handler can terminate all further progress along the chain by passing the \refconst{PMIX_EVENT_ACTION_COMPLETE} status to the completion callback function.

\subsection{Events versus status constants}
\label{api:event:evssc}

Return status constants (see Section \ref{api:struct:errors}) represent values that can be returned from or passed into \ac{PMIx}
\acp{API}. These are distinct from \ac{PMIx} \emph{events} in that they are
not values that can be registered against event handlers. In general, the two
types of constants are distinguished by inclusion of an "ERR" in the name of
error constants versus an "EVENT" in events, though there are exceptions (e.g,
the \refconst{PMIX_SUCCESS} constant).


%%%%%%%%%%%%%%%%%%%%%%%%%%%%%%%%%%%%%%%%%%%%%%%%%
\subsection{\code{PMIx_Register_event_handler}}
\declareapi{PMIx_Register_event_handler}

%%%%
\summary

Register an event handler.

%%%%
\format

\copySignature{PMIx_Register_event_handler}{2.0}{
pmix_status_t \\
PMIx_Register_event_handler(pmix_status_t codes[], size_t ncodes, \\
\hspace*{28\sigspace}pmix_info_t info[], size_t ninfo, \\
\hspace*{28\sigspace}pmix_notification_fn_t evhdlr, \\
\hspace*{28\sigspace}pmix_hdlr_reg_cbfunc_t cbfunc, \\
\hspace*{28\sigspace}void *cbdata);
}

\begin{arglist}
\argin{codes}{Array of status codes (array of \refstruct{pmix_status_t})}
\argin{ncodes}{Number of elements in the \refarg{codes} array (\code{size_t})}
\argin{info}{Array of info structures (array of handles)}
\argin{ninfo}{Number of elements in the \refarg{info} array (\code{size_t})}
\argin{evhdlr}{Event handler to be called \refapi{pmix_notification_fn_t} (function reference)}
\argin{cbfunc}{Callback function \refapi{pmix_hdlr_reg_cbfunc_t} (function reference)}
\argin{cbdata}{Data to be passed to the cbfunc callback function (memory reference)}
\end{arglist}


If \refarg{cbfunc} is \code{NULL}, the function call will be treated as a \emph{blocking} call. In this case, the returned status will be either (a) the event handler reference identifier if the value is greater than or equal to zero, or (b) a negative error code indicative of the reason for the failure.

If the \refarg{cbfunc} is non-\code{NULL}, the function call will be treated as a \emph{non-blocking} call and will return the following:

\begin{itemize}
\item \refconst{PMIX_SUCCESS} indicating that the request has been accepted for processing and the provided callback function will be executed upon completion of the operation. Note that the library must not invoke the callback function prior to returning from the \ac{API}. The result of the registration operation shall be returned in the provided callback function along with the assigned event handler identifier.
\item \refconst{PMIX_ERR_EVENT_REGISTRATION} indicating that the registration
has failed for an undetermined reason.
\item a non-zero \ac{PMIx} error constant indicating a reason for the request to have been rejected. In this case, the provided callback function will not be executed.
\end{itemize}

The callback function must not be executed prior to returning from the \ac{API}, and no events corresponding to this registration may be delivered prior to the completion of the registration callback function (\refarg{cbfunc}).

\reqattrstart
The following attributes are required to be supported by all \ac{PMIx} libraries:

\pasteAttributeItem{PMIX_EVENT_HDLR_NAME}
\pasteAttributeItem{PMIX_EVENT_HDLR_FIRST}
\pasteAttributeItem{PMIX_EVENT_HDLR_LAST}
\pasteAttributeItem{PMIX_EVENT_HDLR_FIRST_IN_CATEGORY}
\pasteAttributeItem{PMIX_EVENT_HDLR_LAST_IN_CATEGORY}
\pasteAttributeItem{PMIX_EVENT_HDLR_BEFORE}
\pasteAttributeItem{PMIX_EVENT_HDLR_AFTER}
\pasteAttributeItem{PMIX_EVENT_HDLR_PREPEND}
\pasteAttributeItem{PMIX_EVENT_HDLR_APPEND}
\pasteAttributeItem{PMIX_EVENT_CUSTOM_RANGE}
\pasteAttributeItem{PMIX_RANGE}
\pasteAttributeItem{PMIX_EVENT_RETURN_OBJECT}

\divider

Host environments that implement support for \ac{PMIx} event notification are required to support the following attributes when registering handlers - these attributes are used to direct that the handler should be invoked only when the event affects the indicated process(es):

\pasteAttributeItem{PMIX_EVENT_AFFECTED_PROC}
\pasteAttributeItem{PMIX_EVENT_AFFECTED_PROCS}

\reqattrend


%%%%
\descr

Register an event handler to report events. Note that the codes being registered do \textit{not} need to be \ac{PMIx} error constants --- any integer value can be registered. This allows for registration of non-PMIx events such as those defined by a particular \ac{SMS} vendor or by an application itself.

\adviceuserstart
In order to avoid potential conflicts, users are advised to only define codes that lie outside the range of the \ac{PMIx} standard's error codes. Thus, \ac{SMS} vendors and application developers should constrain their definitions to positive values or negative values beyond the \refconst{PMIX_EXTERNAL_ERR_BASE} boundary.
\adviceuserend


\adviceuserstart
As previously stated, upon completing its work, and prior to returning, each handler \textit{must} call the event handler completion function provided when it was invoked (including a status code plus any information to be passed to later handlers) so that the chain can continue being progressed. An event handler can terminate all further progress along the chain by passing the \refconst{PMIX_EVENT_ACTION_COMPLETE} status to the completion callback function. Note that the parameters passed to the event handler (e.g., the \refarg{info} and \refarg{results} arrays) will cease to be valid once the completion function has been called - thus, any information in the incoming parameters that will be referenced following the call to the completion function must be copied.
\adviceuserend

%%%%%%%%%%%%%%%%%%%%%%%%%%%%%%%%%%%%%%%%%%%%%%%%%
\subsection{Event registration constants}
\label{api:struct:constants:event}

\begin{constantdesc}
%
\declareconstitem{PMIX_ERR_EVENT_REGISTRATION}
Error in event registration.
%
\end{constantdesc}

%%%%%%%%%%%%%%%%%%%%%%%%%%%%%%%%%%%%%%%%%%%%%%%%%
\subsection{System events}
\label{api:struct:sys:event}

\begin{constantdesc}
%
\declareconstitemNEW{PMIX_EVENT_SYS_BASE}
Mark the beginning of a dedicated range of constants for system event reporting.
%
\declareconstitemNEW{PMIX_EVENT_NODE_DOWN}
A node has gone down - the identifier of the affected node will be included in the notification.
%
\declareconstitemNEW{PMIX_EVENT_NODE_OFFLINE}
A node has been marked as \emph{offline} - the identifier of the affected node will be included in the notification.
%
\declareconstitemNEW{PMIX_EVENT_SYS_OTHER}
Mark the end of a dedicated range of constants for system event reporting.
%
\end{constantdesc}

\littleheader{Detect system event constant}
\declaremacro{PMIX_SYSTEM_EVENT}

Test a given event constant to see if it falls within the dedicated range of constants for system event reporting.

\copySignature{PMIX_SYSTEM_EVENT}{2.2}{
PMIX_SYSTEM_EVENT(a)
}

\begin{arglist}
\argin{a}{Error constant to be checked (\refstruct{pmix_status_t})}
\end{arglist}

Returns \code{true} if the provided values falls within the dedicated range of events for system event reporting.

%%%%%%%%%%%%%%%%%%%%%%%%%%%%%%%%%%%%%%%%%%%%%%%%%
\subsection{Event handler registration and notification attributes}
\label{api:struct:attributes:event}

Attributes to support event registration and notification.

%
\declareAttribute{PMIX_EVENT_HDLR_NAME}{"pmix.evname"}{char*}{
String name identifying this handler.
}
%
\declareAttribute{PMIX_EVENT_HDLR_FIRST}{"pmix.evfirst"}{bool}{
Invoke this event handler before any other handlers.
}
%
\declareAttribute{PMIX_EVENT_HDLR_LAST}{"pmix.evlast"}{bool}{
Invoke this event handler after all other handlers have been called.
}
%
\declareAttribute{PMIX_EVENT_HDLR_FIRST_IN_CATEGORY}{"pmix.evfirstcat"}{bool}{
Invoke this event handler before any other handlers in this category.
}
%
\declareAttribute{PMIX_EVENT_HDLR_LAST_IN_CATEGORY}{"pmix.evlastcat"}{bool}{
Invoke this event handler after all other handlers in this category have been called.
}
%
\declareAttribute{PMIX_EVENT_HDLR_BEFORE}{"pmix.evbefore"}{char*}{
Put this event handler immediately before the one specified in the \code{(char*)} value.
}
%
\declareAttribute{PMIX_EVENT_HDLR_AFTER}{"pmix.evafter"}{char*}{
Put this event handler immediately after the one specified in the \code{(char*)} value.
}
%
\declareAttribute{PMIX_EVENT_HDLR_PREPEND}{"pmix.evprepend"}{bool}{
Prepend this handler to the precedence list within its category.
}
%
\declareAttribute{PMIX_EVENT_HDLR_APPEND}{"pmix.evappend"}{bool}{
Append this handler to the precedence list within its category.
}
%
\declareAttribute{PMIX_EVENT_CUSTOM_RANGE}{"pmix.evrange"}{pmix_data_array_t*}{
Array of \refstruct{pmix_proc_t} defining range of event notification.
}
%
\declareAttribute{PMIX_EVENT_AFFECTED_PROC}{"pmix.evproc"}{pmix_proc_t}{
The single process that was affected.
}
%
\declareAttribute{PMIX_EVENT_AFFECTED_PROCS}{"pmix.evaffected"}{pmix_data_array_t*}{
Array of \refstruct{pmix_proc_t} defining affected processes.
}
%
\declareAttribute{PMIX_EVENT_NON_DEFAULT}{"pmix.evnondef"}{bool}{
Event is not to be delivered to default event handlers.
}
%
\declareAttribute{PMIX_EVENT_RETURN_OBJECT}{"pmix.evobject"}{void *}{
Object to be returned whenever the registered callback function \code{cbfunc} is invoked.
The object will only be returned to the process that registered it.
}
%
\declareAttribute{PMIX_EVENT_DO_NOT_CACHE}{"pmix.evnocache"}{bool}{
Instruct the \ac{PMIx} server not to cache the event.
}
%
\declareAttribute{PMIX_EVENT_PROXY}{"pmix.evproxy"}{pmix_proc_t*}{
\ac{PMIx} server that sourced the event.
}
%
\declareAttribute{PMIX_EVENT_TEXT_MESSAGE}{"pmix.evtext"}{char*}{
Text message suitable for output by recipient - e.g., describing the cause of the event.
}
%
\declareAttributeNEW{PMIX_EVENT_TIMESTAMP}{"pmix.evtstamp"}{time_t}{
System time when the associated event occurred.
}

%%%%%%%%%%%%%%%%%%%%%%%%%%%%%%%%%%%%%%%%%%%%%%%%%
\subsubsection{Fault tolerance event attributes}
\label{api:struct:attributes:ft}

The following attributes may be used by the host environment when providing an event notification as qualifiers indicating the action it intends to take in response to the event:

%
\declareAttribute{PMIX_EVENT_TERMINATE_SESSION}{"pmix.evterm.sess"}{bool}{
The \ac{RM} intends to terminate this session.
}
%
\declareAttribute{PMIX_EVENT_TERMINATE_JOB}{"pmix.evterm.job"}{bool}{
The \ac{RM} intends to terminate this job.
}
%
\declareAttribute{PMIX_EVENT_TERMINATE_NODE}{"pmix.evterm.node"}{bool}{
The \ac{RM} intends to terminate all processes on this node.
}
%
\declareAttribute{PMIX_EVENT_TERMINATE_PROC}{"pmix.evterm.proc"}{bool}{
The \ac{RM} intends to terminate just this process.
}
%
\declareAttribute{PMIX_EVENT_ACTION_TIMEOUT}{"pmix.evtimeout"}{int}{
The time in seconds before the \ac{RM} will execute the indicated operation.
}

%%%%%%%%%%%%%%%%%%%%%%%%%%%%%%%%%%%%%%%%%%%%%%%%%
\subsubsection{Hybrid programming event attributes}
\label{api:struct:attributes:hybrid}

The following attributes may be used by programming models to coordinate their use of common resources within a process in conjunction with the \refconst{PMIX_OPENMP_PARALLEL_ENTERED} event:
%
\pasteAttributeItem{PMIX_MODEL_PHASE_NAME}
\pasteAttributeItem{PMIX_MODEL_PHASE_TYPE}

%%%%%%%%%%%%%%%%%%%%%%%%%%%%%%%%%%%%%%%%%%%%%%%%%
\subsection{Notification Function}
\declareapi{pmix_notification_fn_t}

%%%%
\summary

The \refapi{pmix_notification_fn_t} is called by \ac{PMIx} to deliver notification of an event.

\adviceuserstart
The \ac{PMIx} \textit{ad hoc} v1.0 Standard defined an error notification function with an identical name, but different signature than the v2.0 Standard described below. The \textit{ad hoc} v1.0 version was removed from the v2.0 Standard is not included in this document to avoid confusion.
\adviceuserend


\copySignature{pmix_notification_fn_t}{2.0}{
typedef void (*pmix_notification_fn_t) \\
\hspace*{4\sigspace}(size_t evhdlr_registration_id, \\
\hspace*{5\sigspace}pmix_status_t status, \\
\hspace*{5\sigspace}const pmix_proc_t *source, \\
\hspace*{5\sigspace}pmix_info_t info[], size_t ninfo, \\
\hspace*{5\sigspace}pmix_info_t results[], size_t nresults, \\
\hspace*{5\sigspace}pmix_event_notification_cbfunc_fn_t cbfunc, \\
\hspace*{5\sigspace}void *cbdata);
}

\begin{arglist}
\argin{evhdlr_registration_id}{Registration number of the handler being called (\code{size_t})}
\argin{status}{Status associated with the operation (\refstruct{pmix_status_t})}
\argin{source}{Identifier of the process that generated the event (\refstruct{pmix_proc_t})}. If the source is the \ac{SMS}, then the nspace will be empty and the rank will be PMIX_RANK_UNDEF
\argin{info}{Information describing the event (\refstruct{pmix_info_t})}. This argument will be NULL if no additional information was provided by the event generator.
\argin{ninfo}{Number of elements in the info array (\code{size_t})}
\argin{results}{Aggregated results from prior event handlers servicing this event (\refstruct{pmix_info_t})}. This argument will be \code{NULL} if this is the first handler servicing the event, or if no prior handlers provided results.
\argin{nresults}{Number of elements in the results array (\code{size_t})}
\argin{cbfunc}{\refapi{pmix_event_notification_cbfunc_fn_t} callback function to be executed upon completion of the handler's operation and prior to handler return (function reference)}.
\argin{cbdata}{Callback data to be passed to cbfunc (memory reference)}
\end{arglist}

%%%%
\descr

Note that different \acp{RM} may provide differing levels of support for event notification to application processes. Thus, the \refarg{info} array may be \code{NULL} or may contain detailed information of the event. It is the responsibility of the application to parse any provided info array for defined key-values if it so desires.

\adviceuserstart
Possible uses of the \refarg{info} array include:

\begin{itemize}
%
\item for the host \ac{RM} to alert the process as to planned actions, such as aborting the session, in response to the reported event
%
\item provide a timeout for alternative action to occur, such as for the application to request an alternate response to the event
%
\end{itemize}

For example, the \ac{RM} might alert the application to the failure of a node that resulted in termination of several processes, and indicate that the overall session will be aborted unless the application requests an alternative behavior in the next 5 seconds. The application then has time to respond with a checkpoint request, or a request to recover from the failure by obtaining replacement nodes and restarting from some earlier checkpoint.

Support for these options is left to the discretion of the host \ac{RM}. Info keys are included in the common definitions above but may be augmented by environment vendors.
\adviceuserend

\advicermstart
On the server side, the notification function is used to inform the \ac{PMIx} server library's host of a detected event in the \ac{PMIx} server library. Events generated by \ac{PMIx} clients are communicated to the \ac{PMIx} server library, but will be relayed to the host via the \refapi{pmix_server_notify_event_fn_t} function pointer, if provided.
\advicermend


%%%%%%%%%%%%%%%%%%%%%%%%%%%%%%%%%%%%%%%%%%%%%%%%%
\subsection{\code{PMIx_Deregister_event_handler}}
\declareapi{PMIx_Deregister_event_handler}

%%%%
\summary

Deregister an event handler.

%%%%
\format

\copySignature{PMIx_Deregister_event_handler}{2.0}{
pmix_status_t \\
PMIx_Deregister_event_handler(size_t evhdlr_ref, \\
\hspace*{30\sigspace}pmix_op_cbfunc_t cbfunc, \\
\hspace*{30\sigspace}void *cbdata);
}

\begin{arglist}
\argin{evhdlr_ref}{Event handler ID returned by registration (\code{size_t})}
\argin{cbfunc}{Callback function to be executed upon completion of operation \refapi{pmix_op_cbfunc_t} (function reference)}
\argin{cbdata}{Data to be passed to the cbfunc callback function (memory reference)}
\end{arglist}

If \refarg{cbfunc} is \code{NULL}, the function will be treated as a \emph{blocking} call and the result of the operation returned in the status code.

If \refarg{cbfunc} is non-\code{NULL}, the function will be treated as a \emph{non-blocking} call and return one of the following:

\begin{itemize}
\item \refconst{PMIX_SUCCESS}, indicating that the request is being processed - result will be returned in the provided \refarg{cbfunc}. Note that the library must not invoke the callback function prior to returning from the \ac{API}.
\item \refconst{PMIX_OPERATION_SUCCEEDED}, indicating that the request was immediately processed and returned \textit{success} - the \refarg{cbfunc} will \textit{not} be called
\item a PMIx error constant indicating either an error in the input or that the request was immediately processed and failed - the \refarg{cbfunc} will \textit{not} be called
\end{itemize}

The returned status code will be one of the following:

\begin{itemize}
\item \refconst{PMIX_SUCCESS} The event handler was successfully deregistered.
\item \refconst{PMIX_ERR_BAD_PARAM} The provided \refarg{evhdlr_ref} was unrecognized.
\item \refconst{PMIX_ERR_NOT_SUPPORTED} The \ac{PMIx} implementation does not support event notification.
\end{itemize}

%%%%
\descr

Deregister an event handler. Note that no events corresponding to the referenced registration may be delivered following completion of the deregistration operation (either return from the \ac{API} with \refconst{PMIX_OPERATION_SUCCEEDED} or execution of the \refarg{cbfunc}).

%%%%%%%%%%%%%%%%%%%%%%%%%%%%%%%%%%%%%%%%%%%%%%%%%
\subsection{\code{PMIx_Notify_event}}
\declareapi{PMIx_Notify_event}

%%%%
\summary

Report an event for notification via any
registered event handler.

%%%%
\format

\copySignature{PMIx_Notify_event}{2.0}{
pmix_status_t \\
PMIx_Notify_event(pmix_status_t status, \\
\hspace*{18\sigspace}const pmix_proc_t *source, \\
\hspace*{18\sigspace}pmix_data_range_t range, \\
\hspace*{18\sigspace}pmix_info_t info[], size_t ninfo, \\
\hspace*{18\sigspace}pmix_op_cbfunc_t cbfunc, void *cbdata);
}

\begin{arglist}
\argin{status}{Status code of the event (\refstruct{pmix_status_t})}
\argin{source}{Pointer to a \refstruct{pmix_proc_t} identifying the original reporter of the event (handle)}
\argin{range}{Range across which this notification shall be delivered (\refstruct{pmix_data_range_t})}
\argin{info}{Array of \refstruct{pmix_info_t} structures containing any further info provided by the originator of the event (array of handles)}
\argin{ninfo}{Number of elements in the \refarg{info} array (\code{size_t})}
\argin{cbfunc}{Callback function to be executed upon completion of operation \refapi{pmix_op_cbfunc_t} (function reference)}
\argin{cbdata}{Data to be passed to the cbfunc callback function (memory reference)}
\end{arglist}

If \refarg{cbfunc} is \code{NULL}, the function will be treated as a \emph{blocking} call and the result of the operation returned in the status code.

If \refarg{cbfunc} is non-\code{NULL}, the function will be treated as a \emph{non-blocking} call and return one of the following:

\begin{itemize}
\item \refconst{PMIX_SUCCESS} The notification request is valid and is being processed. The callback function will be called when the process-local operation is complete and will provide the resulting status of that operation. Note that this does \textit{not} reflect the success or failure of delivering the event to any recipients. The callback function must not be executed prior to returning from the \ac{API}.
\item \refconst{PMIX_OPERATION_SUCCEEDED}, indicating that the request was immediately processed and returned \textit{success} - the \refarg{cbfunc} will \textit{not} be called
\item \refconst{PMIX_ERR_BAD_PARAM} The request contains at least one incorrect entry that prevents it from being processed. The callback function will \textit{not} be called.
\item \refconst{PMIX_ERR_NOT_SUPPORTED} The \ac{PMIx} implementation does not support event notification, or in the case of a \ac{PMIx} server calling the API, the range extended beyond the local node and the host \ac{SMS} environment does not support event notification. The callback function will \textit{not} be called.
\end{itemize}

\reqattrstart
The following attributes are required to be supported by all \ac{PMIx} libraries:

\pasteAttributeItem{PMIX_EVENT_NON_DEFAULT}
\pasteAttributeItem{PMIX_EVENT_CUSTOM_RANGE}
\pasteAttributeItem{PMIX_EVENT_DO_NOT_CACHE}
\pasteAttributeItem{PMIX_EVENT_PROXY}
\pasteAttributeItem{PMIX_EVENT_TEXT_MESSAGE}

\divider

Host environments that implement support for \ac{PMIx} event notification are required to provide the following attributes for all events generated by the environment:

\pasteAttributeItem{PMIX_EVENT_AFFECTED_PROC}
\pasteAttributeItem{PMIX_EVENT_AFFECTED_PROCS}

\reqattrend

\optattrstart
Host environments that support \ac{PMIx} event notification may offer notifications for environmental events impacting the job and for \ac{SMS} events relating to the job. The following attributes may optionally be included to indicate the host environment's intended response to the event:

\pasteAttributeItem{PMIX_EVENT_TERMINATE_SESSION}
\pasteAttributeItem{PMIX_EVENT_TERMINATE_JOB}
\pasteAttributeItem{PMIX_EVENT_TERMINATE_NODE}
\pasteAttributeItem{PMIX_EVENT_TERMINATE_PROC}
\pasteAttributeItem{PMIX_EVENT_ACTION_TIMEOUT}

\optattrend

%%%%
\descr

Report an event for notification via any registered event handler. This function can be called by any \ac{PMIx} process, including application processes, \ac{PMIx} servers, and \ac{SMS} elements. The \ac{PMIx} server calls this \ac{API} to report events it detected itself so that the host \ac{SMS} daemon distribute and handle them, and to pass events given to it by its host down to any attached client processes for processing. Examples might include notification of the failure of another process, detection of an impending node failure due to rising temperatures, or an intent to preempt the application. Events may be locally generated or come from anywhere in the system.

Host \ac{SMS} daemons call the \ac{API} to pass events down to its embedded \ac{PMIx} server both for transmittal to local client processes and for the host's own internal processing where the host has registered its own event handlers. The \ac{PMIx} server library is not allowed to echo any event given to it by its host via this \ac{API} back to the host through the \refapi{pmix_server_notify_event_fn_t} server module function. The host is required to deliver the event to all \ac{PMIx} servers where the targeted processes either are currently running, or (if they haven't started yet) might be running at some point in the future as the events are required to be cached by the \ac{PMIx} server library.

Client application processes can call this function to notify the \ac{SMS} and/or other application processes of an event it encountered. Note that processes are not constrained to report status values defined in the official \ac{PMIx} standard --- any integer value can be used. Thus, applications are free to define their own internal events and use the notification system for their own internal purposes.

\adviceuserstart
The callback function will be called upon completion of the
\code{notify_event} function's actions. At that time, any messages required for executing the operation (e.g., to send the notification to the local \ac{PMIx} server) will
have been queued, but may not yet have been transmitted. The caller is required to maintain the input
data until the callback function has been executed --- the sole purpose of the callback function is to indicate when the input data is no longer required.
\adviceuserend

%%%%%%%%%%%%%%%%%%%%%%%%%%%%%%%%%%%%%%%%%%%%%%%%%
\subsection{Notification Handler Completion Callback Function}
\declareapi{pmix_event_notification_cbfunc_fn_t}

%%%%
\summary

The \refapi{pmix_event_notification_cbfunc_fn_t} is called by event handlers to indicate completion of their operations.

\copySignature{pmix_event_notification_cbfunc_fn_t}{2.0}{
typedef void (*pmix_event_notification_cbfunc_fn_t) \\
\hspace*{4\sigspace}(pmix_status_t status, \\
\hspace*{5\sigspace}pmix_info_t *results, size_t nresults, \\
\hspace*{5\sigspace}pmix_op_cbfunc_t cbfunc, void *thiscbdata, \\
\hspace*{5\sigspace}void *notification_cbdata);
}

\begin{arglist}
\argin{status}{Status returned by the event handler's operation (\refstruct{pmix_status_t})}
\argin{results}{Results from this event handler's operation on the event (\refstruct{pmix_info_t})}
\argin{nresults}{Number of elements in the results array (\code{size_t})}
\argin{cbfunc}{\refapi{pmix_op_cbfunc_t} function to be executed when \ac{PMIx} completes processing the callback (function reference)}
\argin{thiscbdata}{Callback data that was passed in to the handler (memory reference)}
\argin{cbdata}{Callback data to be returned when \ac{PMIx} executes cbfunc (memory reference)}
\end{arglist}

%%%%
\descr

Define a callback by which an event handler can notify the \ac{PMIx} library that it has completed its response to the notification. The handler is \textit{required} to execute this callback so the library can determine if additional handlers need to be called. The handler shall return \refconst{PMIX_EVENT_ACTION_COMPLETE} if no further action is required. The return status of each event handler and any returned \refstruct{pmix_info_t} structures will be added to the \refarg{results} array of \refstruct{pmix_info_t} passed to any subsequent event handlers to help guide their operation.

If non-\code{NULL}, the provided callback function will be called to allow the event handler to release the provided info array and execute any other required cleanup operations.

%%%%%%%%%%%%%%%%%%%%%%%%%%%%%%%%%%%%%%%%%%%%%%%%%
\subsubsection{Completion Callback Function Status Codes}

The following status code may be returned indicating various actions taken by other event handlers.

\begin{constantdesc}
%
\declareconstitem{PMIX_EVENT_NO_ACTION_TAKEN}
Event handler: No action taken.
%
\declareconstitem{PMIX_EVENT_PARTIAL_ACTION_TAKEN}
Event handler: Partial action taken.
%
\declareconstitem{PMIX_EVENT_ACTION_DEFERRED}
Event handler: Action deferred.
%
\declareconstitem{PMIX_EVENT_ACTION_COMPLETE}
Event handler: Action complete.
%
\end{constantdesc}

%%%%%%%%%%%%%%%%%%%%%%%%%%%%%%%%%%%%%%%%%%%%%%%%%


    % Data Packing & Unpacking
    %  - (un)pack, copy
    %%%%%%%%%%%%%%%%%%%%%%%%%%%%%%%%%%%%%%%%%%%%%%%%%
% Chapter: Data Packing and Unpacking
%%%%%%%%%%%%%%%%%%%%%%%%%%%%%%%%%%%%%%%%%%%%%%%%%
\chapter{Data Packing and Unpacking}
\label{chap:api_data_mgmt}

\ldots

%%%%%%%%%%%%%%%%%%%%%%%%%%%%%%%%%%%%%%%%%%%%%%
%%%%%%%%%%%%%%%%%%%%%%%%%%%%%%%%%%%%%%%%%%%%%%
\section{General Routines}
\label{chap:api_init:general}

\ldots

%%%%%%%%%%%
\subsection{\code{PMIx_Data_pack}}
\declareapi{PMIx_Data_pack}

\cspecificstart
\begin{codepar}
/**
 * Top-level interface function to pack one or more values into a
 * buffer.
 *
 * The pack function packs one or more values of a specified type into
 * the specified buffer.  The buffer must have already been
 * initialized via the PMIX_DATA_BUFFER_CREATE or PMIX_DATA_BUFFER_CONSTRUCT
 * call - otherwise, the pack_value function will return an error.
 * Providing an unsupported type flag will likewise be reported as an error.
 *
 * Note that any data to be packed that is not hard type cast (i.e.,
 * not type cast to a specific size) may lose precision when unpacked
 * by a non-homogeneous recipient.  The PACK function will do its best to deal
 * with heterogeneity issues between the packer and unpacker in such
 * cases. Sending a number larger than can be handled by the recipient
 * will return an error code (generated upon unpacking) -
 * the error cannot be detected during packing.
 *
 * @param *buffer A pointer to the buffer into which the value is to
 * be packed.
 *
 * @param *src A void* pointer to the data that is to be packed. Note
 * that strings are to be passed as (char **) - i.e., the caller must
 * pass the address of the pointer to the string as the void*. This
 * allows PMIx to use a single pack function, but still allow
 * the caller to pass multiple strings in a single call.
 *
 * @param num_values An int32_t indicating the number of values that are
 * to be packed, beginning at the location pointed to by src. A string
 * value is counted as a single value regardless of length. The values
 * must be contiguous in memory. Arrays of pointers (e.g., string
 * arrays) should be contiguous, although (obviously) the data pointed
 * to need not be contiguous across array entries.
 *
 * @param type The type of the data to be packed - must be one of the
 * PMIX defined data types.
 *
 * @retval PMIX_SUCCESS The data was packed as requested.
 *
 * @retval PMIX_ERROR(s) An appropriate PMIX error code indicating the
 * problem encountered. This error code should be handled
 * appropriately.
 *
 * @code
 * pmix_data_buffer_t *buffer;
 * int32_t src;
 *
 * PMIX_DATA_BUFFER_CREATE(buffer);
 * status_code = PMIx_Data_pack(buffer, &src, 1, PMIX_INT32);
 * @endcode
 */
pmix_status_t
PMIx_Data_pack(pmix_data_buffer_t *buffer,
               void *src, int32_t num_vals,
               pmix_data_type_t type);
\end{codepar}
\cspecificend


%%%%%%%%%%%
\subsection{\code{PMIx_Data_unpack}}
\declareapi{PMIx_Data_unpack}

\cspecificstart
\begin{codepar}
/**
 * Unpack values from a buffer.
 *
 * The unpack function unpacks the next value (or values) of a
 * specified type from the specified buffer.
 *
 * The buffer must have already been initialized via an PMIX_DATA_BUFFER_CREATE or
 * PMIX_DATA_BUFFER_CONSTRUCT call (and assumedly filled with some data) -
 * otherwise, the unpack_value function will return an
 * error. Providing an unsupported type flag will likewise be reported
 * as an error, as will specifying a data type that DOES NOT match the
 * type of the next item in the buffer. An attempt to read beyond the
 * end of the stored data held in the buffer will also return an
 * error.
 *
 * NOTE: it is possible for the buffer to be corrupted and that
 * PMIx will *think* there is a proper variable type at the
 * beginning of an unpack region - but that the value is bogus (e.g., just
 * a byte field in a string array that so happens to have a value that
 * matches the specified data type flag). Therefore, the data type error check
 * is NOT completely safe. This is true for ALL unpack functions.
 *
 *
 * Unpacking values is a "nondestructive" process - i.e., the values are
 * not removed from the buffer. It is therefore possible for the caller
 * to re-unpack a value from the same buffer by resetting the unpack_ptr.
 *
 * Warning: The caller is responsible for providing adequate memory
 * storage for the requested data. As noted below, the user
 * must provide a parameter indicating the maximum number of values that
 * can be unpacked into the allocated memory. If more values exist in the
 * buffer than can fit into the memory storage, then the function will unpack
 * what it can fit into that location and return an error code indicating
 * that the buffer was only partially unpacked.
 *
 * Note that any data that was not hard type cast (i.e., not type cast
 * to a specific size) when packed may lose precision when unpacked by
 * a non-homogeneous recipient.  PMIx will do its best to deal with
 * heterogeneity issues between the packer and unpacker in such
 * cases. Sending a number larger than can be handled by the recipient
 * will return an error code generated upon unpacking - these errors
 * cannot be detected during packing.
 *
 * @param *buffer A pointer to the buffer from which the value will be
 * extracted.
 *
 * @param *dest A void* pointer to the memory location into which the
 * data is to be stored. Note that these values will be stored
 * contiguously in memory. For strings, this pointer must be to (char
 * **) to provide a means of supporting multiple string
 * operations. The unpack function will allocate memory for each
 * string in the array - the caller must only provide adequate memory
 * for the array of pointers.
 *
 * @param type The type of the data to be unpacked - must be one of
 * the BFROP defined data types.
 *
 * @retval *max_num_values The number of values actually unpacked. In
 * most cases, this should match the maximum number provided in the
 * parameters - but in no case will it exceed the value of this
 * parameter.  Note that if you unpack fewer values than are actually
 * available, the buffer will be in an unpackable state - the function will
 * return an error code to warn of this condition.
 *
 * @note The unpack function will return the actual number of values
 * unpacked in this location.
 *
 * @retval PMIX_SUCCESS The next item in the buffer was successfully
 * unpacked.
 *
 * @retval PMIX_ERROR(s) The unpack function returns an error code
 * under one of several conditions: (a) the number of values in the
 * item exceeds the max num provided by the caller; (b) the type of
 * the next item in the buffer does not match the type specified by
 * the caller; or (c) the unpack failed due to either an error in the
 * buffer or an attempt to read past the end of the buffer.
 *
 * @code
 * pmix_data_buffer_t *buffer;
 * int32_t dest;
 * char **string_array;
 * int32_t num_values;
 *
 * num_values = 1;
 * status_code = PMIx_Data_unpack(buffer, (void*)&dest, &num_values, PMIX_INT32);
 *
 * num_values = 5;
 * string_array = malloc(num_values*sizeof(char *));
 * status_code = PMIx_Data_unpack(buffer, (void*)(string_array), &num_values, PMIX_STRING);
 *
 * @endcode
 */
pmix_status_t
PMIx_Data_unpack(pmix_data_buffer_t *buffer, void *dest,
                 int32_t *max_num_values,
                 pmix_data_type_t type);
\end{codepar}
\cspecificend


%%%%%%%%%%%
\subsection{\code{PMIx_Data_copy}}
\declareapi{PMIx_Data_copy}

\cspecificstart
\begin{codepar}
/**
 * Copy a data value from one location to another.
 *
 * Since registered data types can be complex structures, the system
 * needs some way to know how to copy the data from one location to
 * another (e.g., for storage in the registry). This function, which
 * can call other copy functions to build up complex data types, defines
 * the method for making a copy of the specified data type.
 *
 * @param **dest The address of a pointer into which the
 * address of the resulting data is to be stored.
 *
 * @param *src A pointer to the memory location from which the
 * data is to be copied.
 *
 * @param type The type of the data to be copied - must be one of
 * the PMIx defined data types.
 *
 * @retval PMIX_SUCCESS The value was successfully copied.
 *
 * @retval PMIX_ERROR(s) An appropriate error code.
 *
 */
pmix_status_t
PMIx_Data_copy(void **dest, void *src,
               pmix_data_type_t type);
\end{codepar}
\cspecificend


%%%%%%%%%%%
\subsection{\code{PMIx_Data_print}}
\declareapi{PMIx_Data_print}

\cspecificstart
\begin{codepar}
/**
 * Print a data value.
 *
 * Since registered data types can be complex structures, the system
 * needs some way to know how to print them (i.e., convert them to a string
 * representation). Provided for debug purposes.
 *
 * @retval PMIX_SUCCESS The value was successfully printed.
 *
 * @retval PMIX_ERROR(s) An appropriate error code.
 */
pmix_status_t
PMIx_Data_print(char **output, char *prefix,
                void *src, pmix_data_type_t type);
\end{codepar}
\cspecificend


%%%%%%%%%%%
\subsection{\code{PMIx_Data_copy_payload}}
\declareapi{PMIx_Data_copy_payload}

\cspecificstart
\begin{codepar}
/**
 * Copy a payload from one buffer to another
 *
 * This function will append a copy of the payload in one buffer into
 * another buffer.
 * NOTE: This is NOT a destructive procedure - the
 * source buffer's payload will remain intact, as will any pre-existing
 * payload in the destination's buffer.
 */
pmix_status_t
PMIx_Data_copy_payload(pmix_data_buffer_t *dest,
                       pmix_data_buffer_t *src);
\end{codepar}
\cspecificend


%%%%%%%%%%%%%%%%%%%%%%%%%%%%%%%%%%%%%%%%%%%%%%%%%


    % Security credentials
    %%%%%%%%%%%%%%%%%%%%%%%%%%%%%%%%%%%%%%%%%%%%%%%%%
% Chapter: Security
%%%%%%%%%%%%%%%%%%%%%%%%%%%%%%%%%%%%%%%%%%%%%%%%%
\chapter{Security}
\label{chap:api_security}

Applications and tools often interact with each other in ways that require verification of the identity of the user making the request, and access to that user's relevant authorizations. This is particularly important when tools connect directly to a system-level \ac{PMIx} server that may be operating at a privileged level. A variety of system management software packages provide this service, but the lack of standardized interfaces makes portability problematic.

This section defines two \ac{PMIx} client-side \acp{API} for this purpose. These are most likely to be used by user-space applications/tools, but are not restricted to that realm.


%%%%%%%%%%%%%%%%%%%%%%%%%%%%%%%%%%%%%%%%%%%%%%
%%%%%%%%%%%%%%%%%%%%%%%%%%%%%%%%%%%%%%%%%%%%%%
\section{Obtaining Credentials}
\label{chap:api_security:obtain}

The \ac{API} for obtaining a credential is a non-blocking operation since the host environment may have to contact a remote credential service. The definition takes into account the potential that the returned credential could be sent via some mechanism to another application that resides in an environment using a different security mechanism. Thus, provision is made for the system to return additional information (e.g., the identity of the issuing agent) outside of the credential itself and visible to the application.

%%%%%%%%%%%
\subsection{\code{PMIx_Get_credential}}
\declareapi{PMIx_Get_credential}

%%%%
\summary

Request a credential from the \ac{PMIx} server library or the host environment

%%%%
\format

\versionMarker{3.0}
\cspecificstart
\begin{codepar}
pmix_status_t
PMIx_Get_credential(const pmix_info_t info[], size_t ninfo,
                    pmix_credential_cbfunc_t cbfunc, void *cbdata)
\end{codepar}
\cspecificend

\begin{arglist}
\argin{info}{Array of \refattr{pmix_info_t} structures (array of handles)}
\argin{ninfo}{Number of elements in the \refarg{info} array (\code{size_t})}
\argin{cbfunc}{Callback function to return credential (\refapi{pmix_credential_cbfunc_t} function reference)}
\argin{cbdata}{Data to be passed to the callback function (memory reference)}
\end{arglist}

Returns one of the following:

\begin{itemize}
    \item \refconst{PMIX_SUCCESS}, indicating that the request has been communicated to the local \ac{PMIx} server - result will be returned in the provided \refarg{cbfunc}
    \item a \ac{PMIx} error constant indicating either an error in the input or that the request is unsupported - the \refarg{cbfunc} will \textit{not} be called
\end{itemize}

\reqattrstart
\ac{PMIx} libraries that choose not to support this operation \textit{must} return \refconst{PMIX_ERR_NOT_SUPPORTED} when the function is called. Implementations that support the operation may choose to internally execute integration for some security environments (e.g., directly contacting a \textit{munge} server) - there are no identified required attributes for this \ac{API}.

However, if the \ac{PMIx} implementation provides support for this \ac{API} and the request cannot be processed by the library itself, then any attributes that are provided by the client must be passed to the host environment for processing. In addition, the following attributes are required to be included in the \refarg{info} array passed from the \ac{PMIx} library to the host environment:

\pastePRIAttributeItem{PMIX_USERID}
\pastePRIAttributeItem{PMIX_GRPID}

\reqattrend

\optattrstart
The following attributes are optional for host environments that support this operation:

\pasteAttributeItem{PMIX_TIMEOUT}

\optattrend

\adviceimplstart
We recommend that implementation of the \refattr{PMIX_TIMEOUT} attribute be left to the host environment due to race condition considerations between completion of the operation versus internal timeout in the \ac{PMIx} server library. Implementers that choose to support \refattr{PMIX_TIMEOUT} directly in the \ac{PMIx} server library must take care to resolve the race condition and should avoid passing \refattr{PMIX_TIMEOUT} to the host environment so that multiple competing timeouts are not created.
\adviceimplend

%%%%
\descr

Request a credential from the \ac{PMIx} server library or the host environment

%%%%%%%%%%%%%%%%%%%%%%%%%%%%%%%%%%%%%%%%%%%%%%
%%%%%%%%%%%%%%%%%%%%%%%%%%%%%%%%%%%%%%%%%%%%%%
\section{Validating Credentials}
\label{chap:api_security:validate}

The \ac{API} for validating a credential is a non-blocking operation since the host environment may have to contact a remote credential service. Provision is made for the system to return additional information regarding possible authorization limitations beyond simple authentication.

%%%%%%%%%%%
\subsection{\code{PMIx_Validate_credential}}
\declareapi{PMIx_Validate_credential}

%%%%
\summary

Request validation of a credential by the \ac{PMIx} server library or the host environment

%%%%
\format

\versionMarker{3.0}
\cspecificstart
\begin{codepar}
pmix_status_t
PMIx_Validate_credential(const pmix_byte_object_t *cred,
                         const pmix_info_t info[], size_t ninfo,
                         pmix_validation_cbfunc_t cbfunc,
                         void *cbdata)
\end{codepar}
\cspecificend

\begin{arglist}
\argin{cred}{Pointer to \refstruct{pmix_byte_object_t} containing the credential (handle)}
\argin{info}{Array of \refstruct{pmix_info_t} structures (array of handles)}
\argin{ninfo}{Number of elements in the \refarg{info} array (\code{size_t})}
\argin{cbfunc}{Callback function to return result (\refapi{pmix_validation_cbfunc_t} function reference)}
\argin{cbdata}{Data to be passed to the callback function (memory reference)}
\end{arglist}

Returns one of the following:

\begin{itemize}
    \item \refconst{PMIX_SUCCESS}, indicating that the request has been communicated to the local \ac{PMIx} server - result will be returned in the provided \refarg{cbfunc}
    \item a \ac{PMIx} error constant indicating either an error in the input or that the request is unsupported - the \refarg{cbfunc} will \textit{not} be called
\end{itemize}

\reqattrstart
\ac{PMIx} libraries that choose not to support this operation \textit{must} return \refconst{PMIX_ERR_NOT_SUPPORTED} when the function is called. Implementations that support the operation may choose to internally execute integration for some security environments (e.g., directly contacting a \textit{munge} server) - there are no identified required attributes for this \ac{API}.

However, if the \ac{PMIx} implementation provides support for this \ac{API} and the request cannot be processed by the library itself, then any attributes that are provided by the client must be passed to the host environment for processing. In addition, the following attributes are required to be included in the \refarg{info} array passed from the \ac{PMIx} library to the host environment:

\pastePRIAttributeItem{PMIX_USERID}
\pastePRIAttributeItem{PMIX_GRPID}

\reqattrend

\optattrstart
The following attributes are optional for host environments that support this operation:

\pasteAttributeItem{PMIX_TIMEOUT}

\optattrend

\adviceimplstart
We recommend that implementation of the \refattr{PMIX_TIMEOUT} attribute be left to the host environment due to race condition considerations between completion of the operation versus internal timeout in the \ac{PMIx} server library. Implementers that choose to support \refattr{PMIX_TIMEOUT} directly in the \ac{PMIx} server library must take care to resolve the race condition and should avoid passing \refattr{PMIX_TIMEOUT} to the host environment so that multiple competing timeouts are not created.
\adviceimplend


%%%%
\descr

Request validation of a credential by the \ac{PMIx} server library or the host environment.



%%%%%%%%%%%%%%%%%%%%%%%%%%%%%%%%%%%%%%%%%%%%%%%%%


    % PMIx Server Specific Interfaces
    %  - setup_fork, (de)register_nspace, pmix_server_module_t
    %%%%%%%%%%%%%%%%%%%%%%%%%%%%%%%%%%%%%%%%%%%%%%%%%
% Chapter: API Server
%%%%%%%%%%%%%%%%%%%%%%%%%%%%%%%%%%%%%%%%%%%%%%%%%
\chapter{Server Specific Interfaces}
\label{chap:api_server}

\ldots


%%%%%%%%%%%
\subsection{\code{PMIx_generate_regex}}
\declareapi{PMIx_generate_regex}

%%%%
\summary

Generate a regular expression representation of the input string.

%%%%
\format

\cspecificstart
\begin{codepar}
pmix_status_t PMIx_generate_regex(const char *input, char **regex)
\end{codepar}
\cspecificend

\begin{arglist}
\argin{input}{String to process (string)}
\argout{regex}{Regular expression representation of \refarg{input} (string)}
\end{arglist}

Returns \refconst{PMIX_SUCCESS} or a negative value corresponding to a PMIx error constant.

%%%%
\descr

Given a semicolon-separated list of \refarg{input} values, generate a regular expression that can be passed down to the \ac{PMIx} client for parsing.
The caller is responsible for free'ing the resulting string.

If values have leading zero's, then that is preserved.
You have to add back any prefix/suffix for node names.

% JJH Format this
% * If values have leading zero's, then that is preserved. You
% * have to add back any prefix/suffix for node names
% * odin[009-015,017-023,076-086]
% *
% *     "pmix:odin[009-015,017-023,076-086]"
% *
% * Note that the "pmix" at the beginning of each regex indicates
% * that the PMIx native parser is to be used by the client for
% * parsing the provided regex. Other parsers may be supported - see
% * the pmix_client.h header for a list.


%%%%%%%%%%%
\subsection{\code{PMIx_generate_ppn}}
\declareapi{PMIx_generate_ppn}

%%%%
\summary

Generate a regular expression representation of the input string.

%%%%
\format

\cspecificstart
\begin{codepar}
pmix_status_t PMIx_generate_ppn(const char *input, char **ppn)
\end{codepar}
\cspecificend

\begin{arglist}
\argin{input}{String to process (string)}
\argout{regex}{Regular expression representation of \refarg{input} (string)}
\end{arglist}

Returns \refconst{PMIX_SUCCESS} or a negative value corresponding to a PMIx error constant.

%%%%
\descr

The input is expected to consist of a comma-separated list of ranges.

% JJH Format this
% * of ranges. Thus, an input of:
% *     "1-4;2-5;8,10,11,12;6,7,9"
% * would generate a regex of
% *     "[pmix:2x(3);8,10-12;6-7,9]"
% *
% * Note that the "pmix" at the beginning of each regex indicates
% * that the PMIx native parser is to be used by the client for
% * parsing the provided regex. Other parsers may be supported - see
% * the pmix_client.h header for a list.
% */


%%%%%%%%%%%
\subsection{\code{PMIx_server_register_nspace}}
\declareapi{PMIx_server_register_nspace}

%%%%
\summary

Setup the data about a particular namespace so it can be passed to any child process upon startup.

%%%%
\format

\cspecificstart
\begin{codepar}
pmix_status_t PMIx_server_register_nspace(const char nspace[], int nlocalprocs,
                                          pmix_info_t info[], size_t ninfo,
                                          pmix_op_cbfunc_t cbfunc, void *cbdata)
\end{codepar}
\cspecificend

\begin{arglist}
\argin{nspace}{namespace (string)}
\argin{nlocalprocs}{number of local processes (integer)}
\argin{info}{Array of info structures (array of handles)}
\argin{ninfo}{Number of elements in the \refarg{info} array (integer)}
\argin{cbfunc}{Callback function \refapi{pmix_op_cbfunc_t} (function reference)}
\argin{cbdata}{Data to be passed to the callback function (memory reference)}
\end{arglist}

Returns \refconst{PMIX_SUCCESS} or a negative value corresponding to a PMIx error constant.

%%%%
\descr

The PMIx connection procedure provides an opportunity for the host PMIx server to pass job-related info down to a child process.
This might include the number of processes in the job, relative local ranks of the processes within the job, and other information of use to the process.
The server is free to determine which, if any, of the supported elements it will provide (See \refsection{chap:struct}{Data Structures and Types} for values).

The PMIx server must register \emph{all} namespaces that will participate in collective operations with local processes.
This means that the server must register a namespace even if it will not host any local procs from within that nspace \emph{if} any local process might at some point perform a collective operation involving one or more processes from that namespace.
This is necessary so that the collective operation can know when it is locally complete.

The caller must also provide the number of local processes that will be launched within this namespace.
This is required for the PMIx server library to correctly handle collectives as a collective operation call can occur before all the processes have been started.


%%%%%%%%%%%
\subsection{\code{PMIx_server_deregister_nspace}}
\declareapi{PMIx_server_deregister_nspace}

%%%%
\summary

Deregister a namespace.

%%%%
\format

\cspecificstart
\begin{codepar}
void PMIx_server_deregister_nspace(const char nspace[],
                                   pmix_op_cbfunc_t cbfunc, void *cbdata)
\end{codepar}
\cspecificend

\begin{arglist}
\argin{nspace}{Namespace (string)}
\argin{cbfunc}{Callback function \refapi{pmix_op_cbfunc_t} (function reference)}
\argin{cbdata}{Data to be passed to the callback function (memory reference)}
\end{arglist}

%%%%
\descr

Deregister the specified \refarg{nspace} and purge all objects relating to it, including any client information from that namespace.
This is intended to support persistent PMIx servers by providing an opportunity for the host \ac{RM} to tell the PMIx server library to release all memory for a completed job.



%%%%%%%%%%%
\subsection{\code{PMIx_server_register_client}}
\declareapi{PMIx_server_register_client}

%%%%
\summary

Register a client process with the PMIx server library.

%%%%
\format

\cspecificstart
\begin{codepar}
pmix_status_t PMIx_server_register_client(const pmix_proc_t *proc,
                                          uid_t uid, gid_t gid,
                                          void *server_object,
                                          pmix_op_cbfunc_t cbfunc, void *cbdata)
\end{codepar}
\cspecificend

\begin{arglist}
\argin{proc}{\refstruct{pmix_proc_t} structure (handle)}
\argin{uid}{user id (integer)}
\argin{gid}{group id (integer)}
\argin{server_object}{(memory reference)}
\argin{cbfunc}{Callback function \refapi{pmix_op_cbfunc_t} (function reference)}
\argin{cbdata}{Data to be passed to the callback function (memory reference)}
\end{arglist}

Returns \refconst{PMIX_SUCCESS} or a negative value corresponding to a PMIx error constant.

%%%%
\descr

Register a client process with the PMIx server library.
The expected user ID and group ID of the child process helps the server library to properly authenticate clients as they connect by requiring the two values to match.

The host server can also, if it desires, provide an object it wishes to be returned when a server function is called that relates to a specific process.
For example, the host server may have an object that tracks the specific client.
Passing the object to the library allows the library to provide that object to the host server during subsequent calls related to that client, such as a ``pmix_server_client_connected_fn'' function.  This allows the host server to access the object without performing a lookup based the client's namespace and rank.


%%%%%%%%%%%
\subsection{\code{PMIx_server_deregister_client}}
\declareapi{PMIx_server_deregister_client}

%%%%
\summary

Deregister a client and purge all data relating to it.

%%%%
\format

\cspecificstart
\begin{codepar}
void PMIx_server_deregister_client(const pmix_proc_t *proc,
                                   pmix_op_cbfunc_t cbfunc, void *cbdata)
\end{codepar}
\cspecificend

\begin{arglist}
\argin{proc}{\refstruct{pmix_proc_t} structure (handle)}
\argin{cbfunc}{Callback function \refapi{pmix_op_cbfunc_t} (function reference)}
\argin{cbdata}{Data to be passed to the callback function (memory reference)}
\end{arglist}


%%%%
\descr

The \refapi{PMIx_server_deregister_nspace} API will automatically delete all client information for that namespace.
This API is therefore intended solely for use in exception cases.


%%%%%%%%%%%
\subsection{\code{PMIx_server_setup_fork}}
\declareapi{PMIx_server_setup_fork}

%%%%
\summary

Setup the environment of a child process to be forked by the host.

%%%%
\format

\cspecificstart
\begin{codepar}
pmix_status_t PMIx_server_setup_fork(const pmix_proc_t *proc, char ***env)
\end{codepar}
\cspecificend

\begin{arglist}
\argin{proc}{\refstruct{pmix_proc_t} structure (handle)}
\argin{env}{Environment array (array of strings)}
\end{arglist}

Returns \refconst{PMIX_SUCCESS} or a negative value corresponding to a PMIx error constant.

%%%%
\descr

Setup the environment of a child process to be forked by the host so it can correctly interact with the PMIx server.
The PMIx client needs some setup information so it can properly connect back to the server.
This function will set appropriate environmental variables for this purpose.


%%%%%%%%%%%
\subsection{\code{PMIx_server_dmodex_request}}
\declareapi{PMIx_server_dmodex_request}
\declareapi{pmix_dmodex_response_fn_t}

%%%%
\summary

Define a function by which the host server can request modex data from the local PMIx server.

%%%%
\format

\cspecificstart
\begin{codepar}
typedef void (*pmix_dmodex_response_fn_t)(pmix_status_t status,
                                          char *data, size_t sz,
                                          void *cbdata);

pmix_status_t PMIx_server_dmodex_request(const pmix_proc_t *proc,
                                         pmix_dmodex_response_fn_t cbfunc,
                                         void *cbdata)
\end{codepar}
\cspecificend

\begin{arglist}
\argin{proc}{\refstruct{pmix_proc_t} structure (handle)}
\argin{cbfunc}{Callback function \refapi{pmix_dmodex_response_fn_t} (function reference)}
\argin{cbdata}{Data to be passed to the callback function (memory reference)}
\end{arglist}

Returns \refconst{PMIX_SUCCESS} or a negative value corresponding to a PMIx error constant.

%%%%
\descr

Define a function by which the host server can request modex data from the local PMIx server.
This is used to support the direct modex operation (i.e., where data is cached locally on each PMIx server for its own local clients, and is obtained on-demand for remote requests.
Upon receiving a request from a remote server, the host server will call this function to pass the request into the PMIx server.
The PMIx server will return a blob (once it becomes available) via the \refarg{cbfunc} - the host server shall send the blob back to the original requestor.

The callback function used by the PMIx server to return direct modex requests to the host server.
The PMIx server will free the data blob upon return from the response function.


%%%%%%%%%%%
\subsection{\code{PMIx_server_setup_application}}
\declareapi{PMIx_server_setup_application}
\declareapi{pmix_setup_application_cbfunc_t}

%%%%
\summary

Provide a function by which the resource manager can request any application-specific environmental variables prior to launch of an application.
 
%%%%
\format

\cspecificstart
\begin{codepar}
typedef void (*pmix_setup_application_cbfunc_t)(pmix_status_t status,
                                                pmix_info_t info[], size_t ninfo,
                                                void *provided_cbdata,
                                                pmix_op_cbfunc_t cbfunc, void *cbdata)

pmix_status_t PMIx_server_setup_application(const char nspace[],
                                            pmix_info_t info[], size_t ninfo,
                                            pmix_setup_application_cbfunc_t cbfunc,
                                            void *cbdata)
\end{codepar}
\cspecificend

\begin{arglist}
\argin{nspace}{namespace (string)}
\argin{info}{Array of info structures (array of handles)}
\argin{ninfo}{Number of elements in the \refarg{info} array (integer)}
\argin{cbfunc}{Callback function \refapi{pmix_setup_application_cbfunc_t} (function reference)}
\argin{cbdata}{Data to be passed to the callback function (memory reference)}
\end{arglist}

Returns \refconst{PMIX_SUCCESS} or a negative value corresponding to a PMIx error constant.

%%%%
\descr

Provide a function by which the resource manager can request any application-specific environmental variables prior to launch of an application.
For example, network libraries may opt to provide security credentials for the application.
This is defined as a non-blocking operation in case network libraries need to perform some action before responding.
The returned env will be distributed along with the application

In the callback function, the returned \refarg{info} array is owned by the PMIx server library and will be free'd when the provided \refarg{cbfunc} is called.


%%%%%%%%%%%
\subsection{\code{PMIx_server_setup_local_support}}
\declareapi{PMIx_server_setup_local_support}

%%%%
\summary

Provide a function by which the local PMIx server can perform any application-specific operations prior to spawning local clients of a given application.

%%%%
\format

\cspecificstart
\begin{codepar}
pmix_status_t PMIx_server_setup_local_support(const char nspace[],
                                              pmix_info_t info[], size_t ninfo,
                                              pmix_op_cbfunc_t cbfunc, void *cbdata);
\end{codepar}
\cspecificend

\begin{arglist}
\argin{nspace}{Namespace (string)}
\argin{info}{Array of info structures (array of handles)}
\argin{ninfo}{Number of elements in the \refarg{info} array (integer)}
\argin{cbfunc}{Callback function \refapi{pmix_op_cbfunc_t} (function reference)}
\argin{cbdata}{Data to be passed to the callback function (memory reference)}
\end{arglist}

Returns \refconst{PMIX_SUCCESS} or a negative value corresponding to a PMIx error constant.

%%%%
\descr

Provide a function by which the local PMIx server can perform any application-specific operations prior to spawning local clients of a given application.
For example, a network library might need to setup the local driver for ``instant on'' addressing.


%%%%%%%%%%%
\section{Server Function Pointers}

The PMIx Server will set the function pointers in the \refapi{pmix_server_module_t} structure that they then pass to \refapi{PMIx_server_init}.
That module structure and associated function references is defined in this section.

%%%%%%%%%%%
\subsection{\code{pmix_server_module_t} Module}
\declareapi{pmix_server_module_t}

%%%%
\summary

List of function pointers that a PMIx server passes to \refapi{PMIx_server_init} during startup.

%%%%
\format

\cspecificstart
\begin{codepar}
typedef struct pmix_server_module_2_0_0_t {
    /* v1x interfaces */
    pmix_server_client_connected_fn_t   client_connected;
    pmix_server_client_finalized_fn_t   client_finalized;
    pmix_server_abort_fn_t              abort;
    pmix_server_fencenb_fn_t            fence_nb;
    pmix_server_dmodex_req_fn_t         direct_modex;
    pmix_server_publish_fn_t            publish;
    pmix_server_lookup_fn_t             lookup;
    pmix_server_unpublish_fn_t          unpublish;
    pmix_server_spawn_fn_t              spawn;
    pmix_server_connect_fn_t            connect;
    pmix_server_disconnect_fn_t         disconnect;
    pmix_server_register_events_fn_t    register_events;
    pmix_server_deregister_events_fn_t  deregister_events;
    pmix_server_listener_fn_t           listener;
    /* v2x interfaces */
    pmix_server_notify_event_fn_t       notify_event;
    pmix_server_query_fn_t              query;
    pmix_server_tool_connection_fn_t    tool_connected;
    pmix_server_log_fn_t                log;
    pmix_server_alloc_fn_t              allocate;
    pmix_server_job_control_fn_t        job_control;
    pmix_server_monitor_fn_t            monitor;
} pmix_server_module_t;
\end{codepar}
\cspecificend

%%%%
\descr

NOTE: for performance purposes, the host server is required to return as quickly as possible from all functions.
Execution of the function is thus to be done asynchronously so as to allow the PMIx server support library to handle multiple client requests as quickly and scalably as possible.

All data passed to the host server functions is ``owned'' by the PMIX server support library and MUST NOT be free'd.
Data returned by the host server via callback function is owned by the host server, which is free to release it upon return from the callback.



%%%%%%%%%%%
\subsection{\code{pmix_server_client_connected_fn_t}}
\declareapi{pmix_server_client_connected_fn_t}

%%%%
\summary

Notify the host server that a client connected to this server.

%%%%
\format

\cspecificstart
\begin{codepar}
typedef pmix_status_t (*pmix_server_client_connected_fn_t)(
                             const pmix_proc_t *proc, void* server_object,
                             pmix_op_cbfunc_t cbfunc, void *cbdata)
\end{codepar}
\cspecificend

\begin{arglist}
\argin{proc}{\refstruct{pmix_proc_t} structure (handle)}
\argin{server_object}{object reference (memory reference)}
\argin{cbfunc}{Callback function \refapi{pmix_op_cbfunc_t} (function reference)}
\argin{cbdata}{Data to be passed to the callback function (memory reference)}
\end{arglist}

Returns \refconst{PMIX_SUCCESS} or a negative value corresponding to a PMIx error constant.

%%%%
\descr

Notify the host server that a client has called PMIx_Init or PMIx_Tool_init.
\rcomment{I am guessing a bit on whether PMIx_Tool_init causes a call to pmix_server_client_connected_fn_t}
Note that the client will be in a blocked state until the host server executes the callback function, thus allowing the PMIx server support library to release 
the client.  
The server_object parameter will be the value of the server_object parameter passed to   
\refapi{PMIx_server_register_client} previously by the host server.  If provided, an implementation of \refapi{pmix_server_client_connected_fn_t} 
is only required to
call the callback function designated.  A host server can choose to not be notified when clients connect by setting \refapi{client_connected} to \code{NULL}. 

It is possible that only a subset of the clients in a namespace call PMIx_init.   The server's \refapi{pmix_server_client_connected_fn_t} implemenation 
should not depend on being called once per rank in a namespace or delaying calling the callback function until all ranks have connected.  
However, if a rank makes any PMIx calls, it must first call \refapi{PMIx_Init} and 
therefore the server's \refapi{mpix_server_client_connected_fn_t} will be called before any other server functions specific to the rank.

\adviceimplstart
 The \refapi{PMIx_server_client_connected_fn_t} implementation provided in the \refapi{pmix_server_module_2_0_0_t} is an opportunity for a host server 
 to update the status of the ranks it manages.  It is also a convenient and well defined time to perform initialization necessary to 
 support further calls into the server related to that rank. 
 \adviceimplend

%%%%%%%%%%%
\subsection{\code{pmix_server_client_finalized_fn_t}}
\declareapi{pmix_server_client_finalized_fn_t}

%%%%
\summary

Notify the host server that a client called \refapi{PMIx_Finalize}.

%%%%
\format

\cspecificstart
\begin{codepar}
typedef pmix_status_t (*pmix_server_client_finalized_fn_t)(
                             const pmix_proc_t *proc, void* server_object,
                             pmix_op_cbfunc_t cbfunc, void *cbdata)
\end{codepar}
\cspecificend

\begin{arglist}
\argin{proc}{\refstruct{pmix_proc_t} structure (handle)}
\argin{server_object}{object reference (memory reference)}
\argin{cbfunc}{Callback function \refapi{pmix_op_cbfunc_t} (function reference)}
\argin{cbdata}{Data to be passed to the callback function (memory reference)}
\end{arglist}

Returns \refconst{PMIX_SUCCESS} or a negative value corresponding to a PMIx error constant.

%%%%
\descr

Notify the host server that a client called \refapi{PMIx_Finalize}.
Note that the client will be in a blocked state until the host server executes the callback function, thus allowing the PMIx server support library to release the client.
The server_object parameter will be the value of the server_object parameter passed to   
\refapi{PMIx_server_register_client} previously by the host server.  If provided, an implementation of \refapi{pmix_server_client_finalized_fn_t} 
is only required to
call the callback function designated.  A host server can choose to not be notified when clients finalize by setting \refapi{client_finalized} to \code{NULL}. 

Note that the host server is only being informed that the client has called \refapi{PMIx_Finalize}.  The client might not have exited.  If a client 
exits without calling \reefapi{PMIx_Finalize}, the server support library will not call the \refapi{PMIx_server_client_finalized_fn_t} implementation.

\adviceimplstart
 The \refapi{PMIx_server_client_finalized_fn_t} implementation provided in the \refapi{pmix_server_module_2_0_0_t} is an opportunity for a host server
 to update the status of the tasks it manages.  It is also a convenient and well defined time to release resources used to support that client.   
 \adviceimplend


%%%%%%%%%%%
\subsection{\code{pmix_server_abort_fn_t}}
\declareapi{pmix_server_abort_fn_t}

%%%%
\summary

Notify PMIx Server that a local client called \refapi{PMIx_Abort}.

%%%%
\format

\cspecificstart
\begin{codepar}
typedef pmix_status_t (*pmix_server_abort_fn_t)(
                             const pmix_proc_t *proc, void *server_object,
                             int status, const char msg[],
                             pmix_proc_t procs[], size_t nprocs,
                             pmix_op_cbfunc_t cbfunc, void *cbdata)
\end{codepar}
\cspecificend


\begin{arglist}
\argin{proc}{\refstruct{pmix_proc_t} structure (handle)}
\argin{server_object}{object reference (memory reference)}
\argin{status}{exit status (integer)}
\argin{msg}{exit status message (string)}
\argin{procs}{Array of \refstruct{pmix_proc_t} structures (array of handles)}
\argin{nprocs}{Number of elements in the \refarg{procs} array (integer)}
\argin{cbfunc}{Callback function \refapi{pmix_op_cbfunc_t} (function reference)}
\argin{cbdata}{Data to be passed to the callback function (memory reference)}
\end{arglist}

Returns \refconst{PMIX_SUCCESS} or a negative value corresponding to a PMIx error constant.

%%%%
\descr

A local client called \refapi{PMIx_Abort}.
Note that the client will be in a blocked state until the host server executes the callback function, thus allowing the PMIx server support library to release the client.
The array of \refarg{procs} indicates which processes are to be terminated.
A \code{NULL} indicates that all processes in the client's namespace are to be terminated.


%%%%%%%%%%%
\subsection{\code{pmix_server_fencenb_fn_t}}
\declareapi{pmix_server_fencenb_fn_t}

%%%%
\summary

At least one client called either \refapi{PMIx_Fence} or \refapi{PMIx_Fence_nb}.

%%%%
\format

\cspecificstart
\begin{codepar}
typedef pmix_status_t (*pmix_server_fencenb_fn_t)(
                             const pmix_proc_t procs[], size_t nprocs,
                             const pmix_info_t info[], size_t ninfo,
                             char *data, size_t ndata,
                             pmix_modex_cbfunc_t cbfunc, void *cbdata)
\end{codepar}
\cspecificend

\begin{arglist}
\argin{procs}{Array of \refstruct{pmix_proc_t} structures (array of handles)}
\argin{nprocs}{Number of elements in the \refarg{procs} array (integer)}
\argin{info}{Array of info structures (array of handles)}
\argin{ninfo}{Number of elements in the \refarg{info} array (integer)}
\argin{data}{(string)}
\argin{ndata}{(integer)}
\argin{cbfunc}{Callback function \refapi{pmix_modex_cbfunc_t} (function reference)}
\argin{cbdata}{Data to be passed to the callback function (memory reference)}
\end{arglist}

Returns \refconst{PMIX_SUCCESS} or a negative value corresponding to a PMIx error constant.

%%%%
\descr

At least one client called either \refapi{PMIx_Fence} or \refapi{PMIx_Fence_nb}.
In either case, the host server will be called via a non-blocking function to execute the specified operation once all participating local processes have contributed.
All processes in the specified \refarg{procs} array are required to participate in the \refapi{PMIx_Fence}/\refapi{PMIx_Fence_nb} operation.
The callback is to be executed once each daemon hosting at least one participant has called the host server's \refapi{pmix_server_fencenb_fn_t} function.

The provided data is to be collectively shared with all PMIx servers involved in the fence operation, and returned in the modex \refarg{cbfunc}.
A \code{NULL} data value indicates that the local processes had no data to contribute.

The array of \refarg{info} structs is used to pass user-requested options to the server.
This can include directives as to the algorithm to be used to execute the fence operation.
The directives are optional \emph{unless} the \emph{mandatory} flag has been set - in such cases, the host \ac{RM} is required to return an error if the directive cannot be met.


%%%%%%%%%%%
\subsection{\code{pmix_server_dmodex_req_fn_t}}
\declareapi{pmix_server_dmodex_req_fn_t}

%%%%
\summary

Used by the PMIx server to request its local host contact the PMIx server on the remote node that hosts the specified proc to obtain and return a direct modex blob for that proc.

%%%%
\format

\cspecificstart
\begin{codepar}
typedef pmix_status_t (*pmix_server_dmodex_req_fn_t)(
                             const pmix_proc_t *proc,
                             const pmix_info_t info[], size_t ninfo,
                             pmix_modex_cbfunc_t cbfunc, void *cbdata)
\end{codepar}
\cspecificend

\begin{arglist}
\argin{proc}{\refstruct{pmix_proc_t} structure (handle)}
\argin{info}{Array of info structures (array of handles)}
\argin{ninfo}{Number of elements in the \refarg{info} array (integer)}
\argin{cbfunc}{Callback function \refapi{pmix_modex_cbfunc_t} (function reference)}
\argin{cbdata}{Data to be passed to the callback function (memory reference)}
\end{arglist}

Returns \refconst{PMIX_SUCCESS} or a negative value corresponding to a PMIx error constant.

%%%%
\descr

Used by the PMIx server to request its local host contact the PMIx server on the remote node that hosts the specified proc to obtain and return a direct modex blob for that proc.

The array of \refarg{info} structs is used to pass user-requested options to the server.
This can include a timeout to preclude an indefinite wait for data that may never become available.
The directives are optional \emph{unless} the \emph{mandatory} flag has been set - in such cases, the host \ac{RM} is required to return an error if the directive cannot be met.


%%%%%%%%%%%
\subsection{\code{pmix_server_publish_fn_t}}
\declareapi{pmix_server_publish_fn_t}

%%%%
\summary

Publish data per the PMIx API specification.

%%%%
\format

\cspecificstart
\begin{codepar}
typedef pmix_status_t (*pmix_server_publish_fn_t)(
                             const pmix_proc_t *proc,
                             const pmix_info_t info[], size_t ninfo,
                             pmix_op_cbfunc_t cbfunc, void *cbdata)
\end{codepar}
\cspecificend

\begin{arglist}
\argin{proc}{\refstruct{pmix_proc_t} structure (handle)}
\argin{info}{Array of info structures (array of handles)}
\argin{ninfo}{Number of elements in the \refarg{info} array (integer)}
\argin{cbfunc}{Callback function \refapi{pmix_op_cbfunc_t} (function reference)}
\argin{cbdata}{Data to be passed to the callback function (memory reference)}
\end{arglist}

Returns \refconst{PMIX_SUCCESS} or a negative value corresponding to a PMIx error constant.

%%%%
\descr

Publish data per the PMIx API specification.
The callback is to be executed upon completion of the operation.
The default data range is expected to be \refconst{PMIX_SESSION}, and the default persistence \refconst{PMIX_PERSIST_SESSION}.
These values can be modified by including the respective \refstruct{pmix_info_t} struct in the \refarg{info} array.

Note that the host server is not required to guarantee support for any specific range - i.e., the server does not need to return an error if the data store doesn't support range-based isolation.
However, the server must return an error (a) if the key is duplicative within the storage range, and (b) if the server does not allow overwriting of published info by the original publisher - it is left to the discretion of the host server to allow info-key-based flags to modify this behavior.

The persistence indicates how long the server should retain the data.

The identifier of the publishing process is also provided and is expected to be returned on any subsequent lookup request.


%%%%%%%%%%%
\subsection{\code{pmix_server_lookup_fn_t}}
\declareapi{pmix_server_lookup_fn_t}

%%%%
\summary

Lookup published data.

%%%%
\format

\cspecificstart
\begin{codepar}
typedef pmix_status_t (*pmix_server_lookup_fn_t)(
                             const pmix_proc_t *proc, char **keys,
                             const pmix_info_t info[], size_t ninfo,
                             pmix_lookup_cbfunc_t cbfunc, void *cbdata)
\end{codepar}
\cspecificend

\begin{arglist}
\argin{proc}{\refstruct{pmix_proc_t} structure (handle)}
\argin{keys}{(array of strings)}
\argin{info}{Array of info structures (array of handles)}
\argin{ninfo}{Number of elements in the \refarg{info} array (integer)}
\argin{cbfunc}{Callback function \refapi{pmix_lookup_cbfunc_t} (function reference)}
\argin{cbdata}{Data to be passed to the callback function (memory reference)}
\end{arglist}

Returns \refconst{PMIX_SUCCESS} or a negative value corresponding to a PMIx error constant.

%%%%
\descr

Lookup published data.
The host server will be passed a NULL-terminated array of string keys.

The array of \refarg{info} structs is used to pass user-requested options to the server.
This can include a wait flag to indicate that the server should wait for all data to become available before executing the callback function, or should immediately callback with whatever data is available.
In addition, a timeout can be specified on the wait to preclude an indefinite wait for data that may never be published.


%%%%%%%%%%%
\subsection{\code{pmix_server_unpublish_fn_t}}
\declareapi{pmix_server_unpublish_fn_t}

%%%%
\summary

Delete data from the data store.

%%%%
\format

\cspecificstart
\begin{codepar}
typedef pmix_status_t (*pmix_server_unpublish_fn_t)(
                             const pmix_proc_t *proc, char **keys,
                             const pmix_info_t info[], size_t ninfo,
                             pmix_op_cbfunc_t cbfunc, void *cbdata)
\end{codepar}
\cspecificend

\begin{arglist}
\argin{proc}{\refstruct{pmix_proc_t} structure (handle)}
\argin{keys}{(array of strings)}
\argin{info}{Array of info structures (array of handles)}
\argin{ninfo}{Number of elements in the \refarg{info} array (integer)}
\argin{cbfunc}{Callback function \refapi{pmix_op_cbfunc_t} (function reference)}
\argin{cbdata}{Data to be passed to the callback function (memory reference)}
\end{arglist}

Returns \refconst{PMIX_SUCCESS} or a negative value corresponding to a PMIx error constant.

%%%%
\descr

Delete data from the data store.
The host server will be passed a NULL-terminated array of string keys, plus potential directives such as the data range within which the keys should be deleted.
The callback is to be executed upon completion of the delete procedure.


%%%%%%%%%%%
\subsection{\code{pmix_server_spawn_fn_t}}
\declareapi{pmix_server_spawn_fn_t}

%%%%
\summary

Spawn a set of applications/processes as per the PMIx API.

%%%%
\format

\cspecificstart
\begin{codepar}
typedef pmix_status_t (*pmix_server_spawn_fn_t)(
                             const pmix_proc_t *proc,
                             const pmix_info_t job_info[], size_t ninfo,
                             const pmix_app_t apps[], size_t napps,
                             pmix_spawn_cbfunc_t cbfunc, void *cbdata)
\end{codepar}
\cspecificend

\begin{arglist}
\argin{proc}{\refstruct{pmix_proc_t} structure (handle)}
\argin{job_info}{Array of info structures (array of handles)}
\argin{ninfo}{Number of elements in the \refarg{jobinfo} array (integer)}
\argin{apps}{Array of \refstruct{pmix_app_t} structures (array of handles)}
\argin{napps}{Number of elements in the \refarg{apps} array (integer)}
\argin{cbfunc}{Callback function \refapi{pmix_spawn_cbfunc_t} (function reference)}
\argin{cbdata}{Data to be passed to the callback function (memory reference)}
\end{arglist}

Returns \refconst{PMIX_SUCCESS} or a negative value corresponding to a PMIx error constant.

%%%%
\descr

Spawn a set of applications/processes as per the PMIx API.
Note that applications are not required to be MPI or any other programming model.
Thus, the host server cannot make any assumptions as to their required support.
The callback function is to be executed once all processes have been started.
An error in starting any application or process in this request shall cause all applications and processes in the request to be terminated, and an error returned to the originating caller.

Note that a timeout can be specified in the job_info array to indicate that failure to start the requested job within the given time should result in termination to avoid hangs.


%%%%%%%%%%%
\subsection{\code{pmix_server_connect_fn_t}}
\declareapi{pmix_server_connect_fn_t}

%%%%
\summary

Record the specified processes as ``connected''.

%%%%
\format

\cspecificstart
\begin{codepar}
typedef pmix_status_t (*pmix_server_connect_fn_t)(
                             const pmix_proc_t procs[], size_t nprocs,
                             const pmix_info_t info[], size_t ninfo,
                             pmix_op_cbfunc_t cbfunc, void *cbdata)
\end{codepar}
\cspecificend

\begin{arglist}
\argin{procs}{Array of \refstruct{pmix_proc_t} structures (array of handles)}
\argin{nprocs}{Number of elements in the \refarg{procs} array (integer)}
\argin{info}{Array of info structures (array of handles)}
\argin{ninfo}{Number of elements in the \refarg{info} array (integer)}
\argin{cbfunc}{Callback function \refapi{pmix_op_cbfunc_t} (function reference)}
\argin{cbdata}{Data to be passed to the callback function (memory reference)}
\end{arglist}

Returns \refconst{PMIX_SUCCESS} or a negative value corresponding to a PMIx error constant.

%%%%
\descr

Record the specified processes as ``connected''.
This means that the resource manager should treat the failure of any process in the specified group as a reportable event, and take appropriate action.
The callback function is to be called once all participating processes have called connect.
Note that a process can only engage in \textbf{one} connect operation involving the identical set of processes at a time.
However, a process \emph{can} be simultaneously engaged in multiple connect operations, each involving a different set of processes.

Note also that this is a collective operation within the client library, and thus the client will be blocked until all processes participate.
Thus, the \refarg{info} array can be used to pass user directives, including a timeout.
The directives are optional \emph{unless} the \emph{mandatory} flag has been set - in such cases, the host RM is required to return an error if the directive cannot be met.


%%%%%%%%%%%
\subsection{\code{pmix_server_disconnect_fn_t}}
\declareapi{pmix_server_disconnect_fn_t}

%%%%
\summary

Disconnect a previously connected set of processes.

%%%%
\format

\cspecificstart
\begin{codepar}
typedef pmix_status_t (*pmix_server_disconnect_fn_t)(
                             const pmix_proc_t procs[], size_t nprocs,
                             const pmix_info_t info[], size_t ninfo,
                             pmix_op_cbfunc_t cbfunc, void *cbdata)
\end{codepar}
\cspecificend

\begin{arglist}
\argin{procs}{Array of \refstruct{pmix_proc_t} structures (array of handles)}
\argin{nprocs}{Number of elements in the \refarg{procs} array (integer)}
\argin{info}{Array of info structures (array of handles)}
\argin{ninfo}{Number of elements in the \refarg{info} array (integer)}
\argin{cbfunc}{Callback function \refapi{pmix_op_cbfunc_t} (function reference)}
\argin{cbdata}{Data to be passed to the callback function (memory reference)}
\end{arglist}

Returns \refconst{PMIX_SUCCESS} or a negative value corresponding to a PMIx error constant.

%%%%
\descr

Disconnect a previously connected set of processes.
An error should be returned if the specified set of processes was not previously ``connected''.
As above, a process may be involved in multiple simultaneous disconnect operations.
However, a process is not allowed to reconnect to a set of ranges that has not fully completed disconnect (i.e., you have to fully disconnect before you can reconnect to the same group of processes).

Note also that this is a collective operation within the client library, and thus the client will be blocked until all processes participate.
Thus, the \refarg{info} array can be used to pass user directives, including a timeout.
The directives are optional \emph{unless} the \emph{mandatory} flag has been set - in such cases, the host RM is required to return an error if the directive cannot be met.


%%%%%%%%%%%
\subsection{\code{pmix_server_register_events_fn_t}}
\declareapi{pmix_server_register_events_fn_t}

%%%%
\summary

Register to receive notifications for the specified events.

%%%%
\format

\cspecificstart
\begin{codepar}
 typedef pmix_status_t (*pmix_server_register_events_fn_t)(
                              pmix_status_t *codes, size_t ncodes,
                              const pmix_info_t info[], size_t ninfo,
                              pmix_op_cbfunc_t cbfunc, void *cbdata)
\end{codepar}
\cspecificend

\begin{arglist}
\argin{codes}{Array of \refstruct{pmix_status_t} structures (array of handles)}
\argin{ncodes}{Number of elements in the \refarg{codes} array (integer)}
\argin{info}{Array of info structures (array of handles)}
\argin{ninfo}{Number of elements in the \refarg{info} array (integer)}
\argin{cbfunc}{Callback function \refapi{pmix_op_cbfunc_t} (function reference)}
\argin{cbdata}{Data to be passed to the callback function (memory reference)}
\end{arglist}

Returns \refconst{PMIX_SUCCESS} or a negative value corresponding to a PMIx error constant.

%%%%
\descr

Register to receive notifications for the specified events.
The resource manager is \emph{required} to pass along to the local PMIx server all events that directly relate to a registered namespace.
However, the RM may have access to events beyond those (e.g., environmental events).
The PMIx server will register to receive environmental events that match specific PMIx event codes.
If the host RM supports such notifications, it will need to translate its own internal event codes to fit into a corresponding PMIx event code - any specific info beyond that can be passed in via the \refstruct{pmix_info_t} upon notification.

The \refarg{info} array included in this API is reserved for possible future directives to further steer notification.



%%%%%%%%%%%
\subsection{\code{pmix_server_deregister_events_fn_t}}
\declareapi{pmix_server_deregister_events_fn_t}

%%%%
\summary

Deregister to receive notifications for the specified events.

%%%%
\format

\cspecificstart
\begin{codepar}
 typedef pmix_status_t (*pmix_server_deregister_events_fn_t)(
                              pmix_status_t *codes, size_t ncodes,
                              pmix_op_cbfunc_t cbfunc, void *cbdata)
\end{codepar}
\cspecificend

\begin{arglist}
\argin{codes}{Array of \refstruct{pmix_status_t} structures (array of handles)}
\argin{ncodes}{Number of elements in the \refarg{codes} array (integer)}
\argin{cbfunc}{Callback function \refapi{pmix_op_cbfunc_t} (function reference)}
\argin{cbdata}{Data to be passed to the callback function (memory reference)}
\end{arglist}

Returns \refconst{PMIX_SUCCESS} or a negative value corresponding to a PMIx error constant.

%%%%
\descr

Deregister to receive notifications for the specified environmental events for which the PMIx server has previously registered.
The host RM remains required to notify of any job-related events.


%%%%%%%%%%%
\subsection{\code{pmix_server_notify_event_fn_t}}
\declareapi{pmix_server_notify_event_fn_t}

%%%%
\summary

Notify the specified processes of an event.

%%%%
\format

\cspecificstart
\begin{codepar}
typedef pmix_status_t (*pmix_server_notify_event_fn_t)(pmix_status_t code,
                                                       const pmix_proc_t *source,
                                                       pmix_data_range_t range,
                                                       pmix_info_t info[], size_t ninfo,
                                                       pmix_op_cbfunc_t cbfunc, void *cbdata);
\end{codepar}
\cspecificend

\begin{arglist}
\argin{code}{\refstruct{pmix_status_t} structure (handle)}
\argin{source}{\refstruct{pmix_proc_t} (handle)}
\argin{range}{\refstruct{pmix_data_range_t} (handle)}
\argin{info}{Array of info structures (array of handles)}
\argin{ninfo}{Number of elements in the \refarg{info} array (integer)}
\argin{cbfunc}{Callback function \refapi{pmix_op_cbfunc_t} (function reference)}
\argin{cbdata}{Data to be passed to the callback function (memory reference)}
\end{arglist}

Returns \refconst{PMIX_SUCCESS} or a negative value corresponding to a PMIx error constant.

%%%%
\descr

Notify the specified processes of an event generated either by the PMIx server itself, or by one of its local clients.
The process generating the event is provided in the source parameter.


%%%%%%%%%%%
\subsection{\code{pmix_connection_cbfunc_t}}
\declareapi{pmix_connection_cbfunc_t}

%%%%
\summary

Callback function for incoming connection requests from local clients.

%%%%
\format

\cspecificstart
\begin{codepar}
typedef void (*pmix_connection_cbfunc_t)(
                    int incoming_sd, void *cbdata)
\end{codepar}
\cspecificend

\begin{arglist}
\argin{incoming_sd}{(integer)}
\argin{cbdata}{ (memory reference)}
\end{arglist}

Returns \refconst{PMIX_SUCCESS} or a negative value corresponding to a PMIx error constant.

%%%%
\descr

Callback function for incoming connection requests from local clients.


%%%%%%%%%%%
\subsection{\code{pmix_server_listener_fn_t}}
\declareapi{pmix_server_listener_fn_t}

%%%%
\summary

Register a socket the host server can monitor for connection requests.

%%%%
\format

\cspecificstart
\begin{codepar}
typedef pmix_status_t (*pmix_server_listener_fn_t)(
                             int listening_sd,
                             pmix_connection_cbfunc_t cbfunc,
                             void *cbdata)
\end{codepar}
\cspecificend

\begin{arglist}
\argin{incoming_sd}{(integer)}
\argin{cbfunc}{Callback function \refapi{pmix_connection_cbfunc_t} (function reference)}
\argin{cbdata}{ (memory reference)}
\end{arglist}

Returns \refconst{PMIX_SUCCESS} or a negative value corresponding to a PMIx error constant.

%%%%
\descr

Register a socket the host server can monitor for connection requests, harvest them, and then call our internal callback function for further processing.
A listener thread is essential to efficiently harvesting connection requests from large numbers of local clients such as occur when running on large SMPs.
The host server listener is required to call accept on the incoming connection request, and then passing the resulting soct to the provided cbfunc.
A NULL for this function will cause the internal PMIx server to spawn its own listener thread.


%%%%%%%%%%%
\subsection{\code{pmix_server_query_fn_t}}
\declareapi{pmix_server_query_fn_t}

%%%%
\summary

Query information from the resource manager.

%%%%
\format

\cspecificstart
\begin{codepar}
typedef pmix_status_t (*pmix_server_query_fn_t)(
                             pmix_proc_t *proct,
                             pmix_query_t *queries, size_t nqueries,
                             pmix_info_cbfunc_t cbfunc,
                             void *cbdata)
\end{codepar}
\cspecificend

\begin{arglist}
\argin{proct}{\refstruct{pmix_proc_t} structure (handle)}
\argin{queries}{Array of \refstruct{pmix_query_t} structures (array of handles)}
\argin{nqueries}{Number of elements in the \refarg{queries} array (integer)}
\argin{cbfunc}{Callback function \refapi{pmix_info_cbfunc_t} (function reference)}
\argin{cbdata}{Data to be passed to the callback function (memory reference)}
\end{arglist}

Returns \refconst{PMIX_SUCCESS} or a negative value corresponding to a PMIx error constant.

%%%%
\descr

Query information from the resource manager.
The query will include the nspace/rank of the process that is requesting the info, an array of \refstruct{pmix_query_t} describing the request, and a callback function/data for the return.


%%%%%%%%%%%
\subsection{\code{pmix_tool_connection_cbfunc_t}}
\declareapi{pmix_tool_connection_cbfunc_t}

%%%%
\summary

Callback function for incoming tool connections.

%%%%
\format

\cspecificstart
\begin{codepar}
typedef void (*pmix_tool_connection_cbfunc_t)(
                    pmix_status_t status,
                    pmix_proc_t *proc, void *cbdata)
\end{codepar}
\cspecificend

\begin{arglist}
\argin{status}{\refstruct{pmix_status_t} structure (handle)}
\argin{proc}{\refstruct{pmix_proc_t} structure (handle)}
\argin{cbdata}{Data to be passed (memory reference)}
\end{arglist}

%%%%
\descr

Callback function for incoming tool connections.
The host RM shall provide an nspace/rank for the connecting tool.
We assume that a \code{rank=0} will be the normal assignment, but allow for the future possibility of a parallel set of tools connecting, and thus each proc requiring a rank.


%%%%%%%%%%%
\subsection{\code{pmix_server_tool_connection_fn_t}}
\declareapi{pmix_server_tool_connection_fn_t}

%%%%
\summary

Register that a tool has connected to the server.

%%%%
\format

\cspecificstart
\begin{codepar}
typedef void (*pmix_server_tool_connection_fn_t)(
                    pmix_info_t *info, size_t ninfo,
                    pmix_tool_connection_cbfunc_t cbfunc,
                    void *cbdata)
\end{codepar}
\cspecificend

\begin{arglist}
\argin{info}{Array of info structures (array of handles)}
\argin{ninfo}{Number of elements in the \refarg{info} array (integer)}
\argin{cbfunc}{Callback function \refapi{pmix_tool_connection_cbfunc_t} (function reference)}
\argin{cbdata}{Data to be passed to the callback function (memory reference)}
\end{arglist}


%%%%
\descr

Register that a tool has connected to the server, and request that the tool be assigned an nspace/rank for further interactions.
The optional \refstruct{pmix_info_t} array can be used to pass qualifiers for the connection request:

\begin{constantdesc}
%
\declareconstitem{PMIX_USERID} effective userid of the tool
%
\declareconstitem{PMIX_GRPID} effective groupid of the tool
%
\declareconstitem{PMIX_FWD_STDOUT} forward any stdout to this tool
%
\declareconstitem{PMIX_FWD_STDERR} forward any stderr to this tool
%
\declareconstitem{PMIX_FWD_STDIN} forward stdin from this tool to any processes spawned on its behalf
%
\end{constantdesc}


%%%%%%%%%%%
\subsection{\code{pmix_server_log_fn_t}}
\declareapi{pmix_server_log_fn_t}

%%%%
\summary

Log data on behalf of a client.

%%%%
\format

\cspecificstart
\begin{codepar}
typedef void (*pmix_server_log_fn_t)(
                    const pmix_proc_t *client,
                    const pmix_info_t data[], size_t ndata,
                    const pmix_info_t directives[], size_t ndirs,
                    pmix_op_cbfunc_t cbfunc, void *cbdata)
\end{codepar}
\cspecificend

\begin{arglist}
\argin{client}{\refstruct{pmix_proc_t} structure (handle)}
\argin{data}{Array of info structures (array of handles)}
\argin{ndata}{Number of elements in the \refarg{data} array (integer)}
\argin{directives}{Array of info structures (array of handles)}
\argin{ndirs}{Number of elements in the \refarg{directives} array (integer)}
\argin{cbfunc}{Callback function \refapi{pmix_op_cbfunc_t} (function reference)}
\argin{cbdata}{Data to be passed to the callback function (memory reference)}
\end{arglist}


%%%%
\descr

Log data on behalf of a client.


%%%%%%%%%%%
\subsection{\code{pmix_server_alloc_fn_t}}
\declareapi{pmix_server_alloc_fn_t}

%%%%
\summary

Request allocation modifications on behalf of a client.

%%%%
\format

\cspecificstart
\begin{codepar}
typedef pmix_status_t (*pmix_server_alloc_fn_t)(
                             const pmix_proc_t *client,
                             pmix_alloc_directive_t directive,
                             const pmix_info_t data[], size_t ndata,
                             pmix_info_cbfunc_t cbfunc, void *cbdata)
\end{codepar}
\cspecificend

\begin{arglist}
\argin{client}{\refstruct{pmix_proc_t} structure (handle)}
\argin{directive}{(handle)}
\argin{data}{Array of info structures (array of handles)}
\argin{ndata}{Number of elements in the \refarg{data} array (integer)}
\argin{cbfunc}{Callback function \refapi{pmix_info_cbfunc_t} (function reference)}
\argin{cbdata}{Data to be passed to the callback function (memory reference)}
\end{arglist}

Returns \refconst{PMIX_SUCCESS} or a negative value corresponding to a PMIx error constant.

%%%%
\descr

Request allocation modifications on behalf of a client.


%%%%%%%%%%%
\subsection{\code{pmix_server_job_control_fn_t}}
\declareapi{pmix_server_job_control_fn_t}

%%%%
\summary

Execute a job control action on behalf of a client.

%%%%
\format

\cspecificstart
\begin{codepar}
typedef pmix_status_t (*pmix_server_job_control_fn_t)(
                             const pmix_proc_t *requestor,
                             const pmix_proc_t targets[], size_t ntargets,
                             const pmix_info_t directives[], size_t ndirs,
                             pmix_info_cbfunc_t cbfunc, void *cbdata)
\end{codepar}
\cspecificend

\begin{arglist}
\argin{requestor}{\refstruct{pmix_proc_t} structure (handle)}
\argin{targets}{Array of proc structures (array of handles)}
\argin{ntargets}{Number of elements in the \refarg{targets} array (integer)}
\argin{directives}{Array of info structures (array of handles)}
\argin{ndirs}{Number of elements in the \refarg{info} array (integer)}
\argin{cbfunc}{Callback function \refapi{pmix_op_cbfunc_t} (function reference)}
\argin{cbdata}{Data to be passed to the callback function (memory reference)}
\end{arglist}

Returns \refconst{PMIX_SUCCESS} or a negative value corresponding to a PMIx error constant.

%%%%
\descr

Execute a job control action on behalf of a client.


%%%%%%%%%%%
\subsection{\code{pmix_server_monitor_fn_t}}
\declareapi{pmix_server_monitor_fn_t}

%%%%
\summary

Request that a client be monitored for activity.

%%%%
\format

\cspecificstart
\begin{codepar}
/* Request that a client be monitored for activity */
typedef pmix_status_t (*pmix_server_monitor_fn_t)(
                             const pmix_proc_t *requestor,
                             const pmix_info_t *monitor, pmix_status_t error,
                             const pmix_info_t directives[], size_t ndirs,
                             pmix_info_cbfunc_t cbfunc, void *cbdata);
\end{codepar}
\cspecificend

\begin{arglist}
\argin{requestor}{\refstruct{pmix_proc_t} structure (handle)}
\argin{monitor}{\refstruct{pmix_proc_t} structure (handle)}
\argin{error}{(integer)}
\argin{directives}{Array of info structures (array of handles)}
\argin{ndirs}{Number of elements in the \refarg{info} array (integer)}
\argin{cbfunc}{Callback function \refapi{pmix_op_cbfunc_t} (function reference)}
\argin{cbdata}{Data to be passed to the callback function (memory reference)}
\end{arglist}

Returns \refconst{PMIX_SUCCESS} or a negative value corresponding to a PMIx error constant.

%%%%
\descr

Request that a client be monitored for activity.

%%%%%%%%%%%%%%%%%%%%%%%%%%%%%%%%%%%%%%%%%%%%%%%%%


    % PMIx Process Sets and Groups
    %%%%%%%%%%%%%%%%%%%%%%%%%%%%%%%%%%%%%%%%%%%%%%%%%
% Chapter: Process Sets and Groups
%%%%%%%%%%%%%%%%%%%%%%%%%%%%%%%%%%%%%%%%%%%%%%%%%
\chapter{Process Sets and Groups}
\label{chap:api_sets_groups}

\ac{PMIx} supports two slightly related, but functionally different concepts
known as \emph{process sets} and \emph{process groups}. This chapter defines
these two concepts and describes how they are utilized, along with their
corresponding \acp{API}.


%%%%%%%%%%%%%%%%%%%%%%%%%%%%%%%%%%%%%%%%%%%%%%%%%
%%%%%%%%%%%%%%%%%%%%%%%%%%%%%%%%%%%%%%%%%%%%%%%%%
\section{Process Sets}
\label{chap:api_sets_groups:sets}

A \ac{PMIx} \emph{Process Set} is a user-provided or host environment assigned
label associated with a given set of application processes. Processes can
belong to multiple process \emph{sets} at a time. Users may define a \ac{PMIx}
process set at time of application execution. For example, if using the command line parallel launcher "prun", one could specify process sets as follows:

\cspecificstart
\begin{codepar}
\$ prun -n 4 --pset ocean myoceanapp : -n 3 --pset ice myiceapp
\end{codepar}
\cspecificend

In this example, the processes in the first application will be labeled with a \refattr{PMIX_PSET_NAMES} attribute with a value of \emph{ocean} while those in the second application will be labeled with an \emph{ice} value. During the execution, application processes could lookup the process set attribute for any process using \refapi{PMIx_Get}. Alternatively, other executing applications could utilize the \refapi{PMIx_Query_info} \acp{API} to obtain the number of declared process sets in the system, a list of their names, and other information about them. In other words, the \emph{process set} identifier provides a label by which an application can derive information about a process and its application - it does \emph{not}, however, confer any operational function.

Host environments can create or delete process sets at any time through the
\refapi{PMIx_server_define_process_set} and
\refapi{PMIx_server_delete_process_set} \acp{API}. \ac{PMIx} servers shall
notify all local clients of process set operations via the
\refconst{PMIX_PROCESS_SET_DEFINE} or \refconst{PMIX_PROCESS_SET_DELETE}
events.

Process \emph{sets} differ from process \emph{groups} in several key ways:

\begin{itemize}
    \item Process \emph{sets} have no implied relationship between their members - i.e., a process in a process set has no concept of a ``pset rank'' as it would in a process \emph{group}.
    %
    \item Process \emph{set} identifiers are set by the host environment or by the user at time of application submission for execution -
    there are no \ac{PMIx} \acp{API} provided by which an application can define a process set or
    change a process \emph{set} membership. In contrast, \ac{PMIx} process
    \emph{groups} can only be defined dynamically by the application.
    %
    \item Process \emph{sets} are immutable - members cannot be added or removed once the set has been defined. In contrast, \ac{PMIx} process \emph{groups} can dynamically change their membership using the appropriate \acp{API}.
    %
    \item Process \emph{groups} can be used in calls to \ac{PMIx} operations. Members of process \emph{groups} that are involved in an operation are translated by their \ac{PMIx} server into their \emph{native} identifier prior to the operation being passed to the host environment. For example, an application can define a process group to consist of ranks 0 and 1 from the host-assigned namespace of \emph{210456}, identified by the group id of \emph{foo}. If the application subsequently calls the \refapi{PMIx_Fence} \ac{API} with a process identifier of \code{\{foo, PMIX_RANK_WILDCARD\}}, the \ac{PMIx} server will replace that identifier with an array consisting of \code{\{210456, 0\}} and \code{\{210456, 1\}} - the host-assigned identifiers of the participating processes - prior to processing the request.
    %
    \item Process \emph{groups} can request that the host environment assign a unique \code{size_t} \ac{PGCID} to the group at time of group construction. An \ac{MPI} library may, for example, use the \ac{PGCID} as the \ac{MPI} communicator identifier for the group.
    %
\end{itemize}

The two concepts do, however, overlap in that they both
involve collections of processes. Users desiring to create a process group
based on a process set could, for example, obtain the membership array of the
process set and use that as input to \refapi{PMIx_Group_construct}, perhaps
including the process set name as the group identifier for clarity. Note that
no linkage between the set and group of the same name is implied nor
maintained - e.g., changes in process group membership can not be
reflected in the process set using the same identifier.

\advicermstart
The host environment is responsible for ensuring:

\begin{itemize}
    \item consistent knowledge of process set membership across all involved
    \ac{PMIx} servers; and
    \item that process set names do not conflict with system-assigned namespaces within the scope of the set.
\end{itemize}

\advicermend


%%%%%%%%%%%%%%%%%%%%%%%%%%%%%%%%%%%%%%%%%%%%%%%%%
\subsection{Process Set Constants}

\versionMarker{4.0}
The \ac{PMIx} server is required to send a notification to all local clients upon creation or deletion of process sets. Client processes wishing to receive such
notifications must register for the corresponding event:

\begin{constantdesc}
%
\declareconstitemNEW{PMIX_PROCESS_SET_DEFINE}
The host environment has defined a new process set - the event will include the process set name (\refattr{PMIX_PSET_NAME}) and the membership (\refattr{PMIX_PSET_MEMBERS}).
%
\declareconstitemNEW{PMIX_PROCESS_SET_DELETE}
The host environment has deleted a process set - the event will include the process set name (\refattr{PMIX_PSET_NAME}).
%
\end{constantdesc}


%%%%%%%%%%%
\subsection{Process Set Attributes}

\versionMarker{4.0}
Several attributes are provided for querying the system regarding process sets using the \refapi{PMIx_Query_info} \acp{API}.

%
\declareAttributeNEW{PMIX_QUERY_NUM_PSETS}{"pmix.qry.psetnum"}{size_t}{
Return the number of process sets defined in the specified range (defaults
to \refconst{PMIX_RANGE_SESSION}).
}
%
\declareAttributeNEW{PMIX_QUERY_PSET_NAMES}{"pmix.qry.psets"}{pmix_data_array_t*}{
Return a \refstruct{pmix_data_array_t} containing an array of strings of the
process set names defined in the specified range (defaults to \refconst{PMIX_RANGE_SESSION}).
}
%
\declareAttributeNEW{PMIX_QUERY_PSET_MEMBERSHIP}{"pmix.qry.pmems"}{pmix_data_array_t*}{
Return an array of \refstruct{pmix_proc_t} containing
the members of the specified process set.
}
%

\vspace{\baselineskip}
The \refconst{PMIX_PROCESS_SET_DEFINE} event shall include the name of the newly defined process set and its members:
%
\declareAttributeNEW{PMIX_PSET_NAME}{"pmix.pset.nm"}{char*}{
The name of the newly defined process set.
}
%
\declareAttributeNEW{PMIX_PSET_MEMBERS}{"pmix.pset.mems"}{pmix_data_array_t*}{
An array of \refstruct{pmix_proc_t} containing
the members of the newly defined process set.
}

\vspace{\baselineskip}
In addition, a process can request (via \refapi{PMIx_Get}) the process sets to which a given process (including itself) belongs:

%
\declareAttributeNEW{PMIX_PSET_NAMES}{"pmix.pset.nms"}{pmix_data_array_t*}{
Returns an array of \code{char*} string names of the process sets in which the given process is a member.
}

%%%%%%%%%%%%%%%%%%%%%%%%%%%%%%%%%%%%%%%%%%%%%%%%%
%%%%%%%%%%%%%%%%%%%%%%%%%%%%%%%%%%%%%%%%%%%%%%%%%
\section{Process Groups}
\label{chap:api_sets_groups:groups}

\ac{PMIx} \emph{Groups} are defined as a collection of processes desiring a common, unique identifier for operational purposes such as passing events or participating in \ac{PMIx} fence operations. As with processes that assemble via \refapi{PMIx_Connect}, each member of the group is provided with both the job-level information of any other namespace represented in the group, and the contact information for all group members.

However, members of \ac{PMIx} Groups are \emph{loosely coupled} as opposed to \emph{tightly connected} when constructed via \refapi{PMIx_Connect}. Thus, \emph{groups} differ from \refapi{PMIx_Connect} assemblages in several key areas, as detailed in the following sections.

\subsection{Relation to the host environment}

Calls to \ac{PMIx} Group \acp{API} are first processed within the local \ac{PMIx} server. When constructed, the server creates a tracker that associates the specified processes with the user-provided group identifier, and assigns a new \emph{group rank} based on their relative position in the array of processes provided in the call to \refapi{PMIx_Group_construct}. Members of the group can subsequently utilize the group identifier in \ac{PMIx} function calls to address the group’s members, using either \refconst{PMIX_RANK_WILDCARD} to refer to all of them or the group-level rank of specific members. The \ac{PMIx} server will translate the specified processes into their \ac{RM}-assigned identifiers prior to passing the request up to its host. Thus, the host environment has no visibility into the group’s existence or membership.

In contrast, calls to \refapi{PMIx_Connect} are relayed to the host environment. This means that the host \ac{RM} should treat the failure of any process in the specified assemblage as a reportable event and take appropriate action. However, the environment is not required to define a new identifier for the connected assemblage or any of its member processes, nor does it define a new rank for each process within that assemblage. In addition, the \ac{PMIx} server does not provide any tracking support for the assemblage. Thus, the caller is responsible for addressing members of the connected assemblage using their \ac{RM}-provided identifiers.

\adviceuserstart
User-provided group identifiers must be distinct from both other group identifiers within the system and namespaces provided by the \ac{RM} so as to avoid collisions between group identifiers and \ac{RM}-assigned namespaces. This can usually be accomplished through the use of an application-specific prefix – e.g., ``myapp-foo''
\adviceuserend


\subsection{Construction procedure}

\refapi{PMIx_Connect} calls require that every process call the \ac{API} before completing – i.e., it is modeled upon the bulk synchronous traditional \ac{MPI} connect/accept methodology. Thus, a given application thread can only be involved in one connect/accept operation at a time, and is blocked in that operation until all specified processes participate. In addition, there is no provision for replacing processes in the assemblage due to failure to participate, nor a mechanism by which a process might decline participation.

In contrast, \ac{PMIx} Groups are designed to be more flexible in their construction procedure by relaxing these constraints. While a standard blocking form of constructing groups is provided, the event notification system is utilized to provide a designated \emph{group leader} with the ability to replace participants that fail to participate within a given timeout period. This provides a mechanism by which the application can, if desired, replace members on-the-fly or allow the group to proceed with partial membership. In such cases, the final group membership is returned to all participants upon completion of the operation.

Additionally, \ac{PMIx} supports dynamic definition of group membership based on an invite/join model. A process can asynchronously initiate construction of a group of any processes via the \refapi{PMIx_Group_invite} function call. Invitations are delivered via a \ac{PMIx} event (using the \refconst{PMIX_GROUP_INVITED} event) to the invited processes which can then either accept or decline the invitation using the \refapi{PMIx_Group_join} \ac{API}. The initiating process tracks responses by registering for the events generated by the call to \refapi{PMIx_Group_join}, timeouts, or process terminations, optionally replacing processes that decline the invitation, fail to respond in time, or terminate without responding. Upon completion of the operation, the final list of participants is communicated to each member of the new group.

\subsection{Destruct procedure}

Members of a \ac{PMIx} Group may depart the group at any time via the \refapi{PMIx_Group_leave} \ac{API}. Other members are notified of the departure via the \refconst{PMIX_GROUP_LEFT} event to distinguish such events from those reporting process termination. This leaves the remaining members free to continue group operations. The \refapi{PMIx_Group_destruct} operation offers a collective method akin to \refapi{PMIx_Disconnect} for deconstructing the entire group.

In contrast, processes that assemble via \refapi{PMIx_Connect} must all depart the assemblage together – i.e., no member can depart the assemblage while leaving the remaining members in it. Even the non-blocking form of \refapi{PMIx_Disconnect} retains this requirement in that members remain a part of the assemblage until all members have called \refapi{PMIx_Disconnect_nb}

Note that applications supporting dynamic group behaviors such as asynchronous departure take responsibility for ensuring global consistency in the group definition prior to executing group collective operations - i.e., it is the application's responsibility to either ensure that knowledge of the current group membership is globally consistent across the participants, or to register for appropriate events to deal with the lack of consistency during the operation.

\adviceuserstart
The reliance on \ac{PMIx} events in the \ac{PMIx} Group concept dictates that processes utilizing these \acp{API} must register for the corresponding events. Failure to do so will likely lead to operational failures. Users are recommended to utilize the \refattr{PMIX_TIMEOUT} directive (or retain an internal timer) on calls to \ac{PMIx} Group \acp{API} (especially the blocking form of those functions) as processes that have not registered for required events will never respond.
\adviceuserend

%%%%%%%%%%%%%%%%%%%%%%%%%%%%%%%%%%%%%%%%%%%%%%%%%
\subsection{Process Group Events}

\versionMarker{4.0}
Asynchronous process group operations rely heavily on \ac{PMIx} events.  The following events have been defined for that purpose.

\begin{constantdesc}
%
\declareconstitemNEW{PMIX_GROUP_INVITED}
The process has been invited to join a \ac{PMIx} Group - the identifier of the group and the ID's of other invited (or already joined) members will be included in the notification.
%
\declareconstitemNEW{PMIX_GROUP_LEFT}
A process has asynchronously left a \ac{PMIx} Group - the process identifier of the departing process will in included in the notification.
%
\declareconstitemNEW{PMIX_GROUP_MEMBER_FAILED}
A member of a \ac{PMIx} Group has abnormally terminated (i.e., without formally leaving the group prior to termination) - the process identifier of the failed process will be included in the notification.
%
\declareconstitemNEW{PMIX_GROUP_INVITE_ACCEPTED}
A process has accepted an invitation to join a \ac{PMIx} Group - the identifier of the group being joined will be included in the notification.
%
\declareconstitemNEW{PMIX_GROUP_INVITE_DECLINED}
A process has declined an invitation to join a \ac{PMIx} Group - the identifier of the declined group will be included in the notification.
%
\declareconstitemNEW{PMIX_GROUP_INVITE_FAILED}
An invited process failed or terminated prior to responding to the invitation - the identifier of the failed process will be included in the notification.
%
\declareconstitemNEW{PMIX_GROUP_MEMBERSHIP_UPDATE}
The membership of a \ac{PMIx} group has changed - the identifiers of the revised membership will be included in the notification.
%
\declareconstitemNEW{PMIX_GROUP_CONSTRUCT_ABORT}
Any participant in a \ac{PMIx} group construct operation that returns \refconst{PMIX_GROUP_CONSTRUCT_ABORT} from the \emph{leader failed} event handler will cause all participants to receive an event notifying them of that status. Similarly, the leader may elect to abort the procedure by either returning this error code from the handler assigned to the \refconst{PMIX_GROUP_INVITE_ACCEPTED} or \refconst{PMIX_GROUP_INVITE_DECLINED} codes, or by generating an event for the abort code. Abort events will be sent to all invited or existing members of the group.
%
\declareconstitemNEW{PMIX_GROUP_CONSTRUCT_COMPLETE}
The group construct operation has completed - the final membership will be included in the notification.
%
\declareconstitemNEW{PMIX_GROUP_LEADER_FAILED}
The current \emph{leader} of a group including this process has abnormally terminated - the group identifier will be included in the notification.
%
\declareconstitemNEW{PMIX_GROUP_LEADER_SELECTED}
A new \emph{leader} of a group including this process has been selected - the identifier of the new leader will be included in the notification.
%
\declareconstitemNEW{PMIX_GROUP_CONTEXT_ID_ASSIGNED}
A new \ac{PGCID} has been assigned by the host environment to a group that includes this process - the group identifier will be included in the notification.
%
\end{constantdesc}

%%%%%%%%%%%%%%%%%%%%%%%%%%%%%%%%%%%%%%%%%%%%%%%%%
\subsection{Process Group Attributes}

\versionMarker{4.0}
Attributes for querying the system regarding process groups include:

%
\declareAttributeNEW{PMIX_QUERY_NUM_GROUPS}{"pmix.qry.pgrpnum"}{size_t}{
Return the number of process groups defined in the specified range (defaults
to session). OPTIONAL QUALIFERS: \refattr{PMIX_RANGE}.
}
%
\declareAttributeNEW{PMIX_QUERY_GROUP_NAMES}{"pmix.qry.pgrp"}{pmix_data_array_t*}{
Return a \refstruct{pmix_data_array_t} containing an array of string names of
the process groups defined in the specified range (defaults to session). OPTIONAL QUALIFERS: \refattr{PMIX_RANGE}.
}
%
\declareAttributeNEW{PMIX_QUERY_GROUP_MEMBERSHIP}{"pmix.qry.pgrpmems"}{pmix_data_array_t*}{
Return a \refstruct{pmix_data_array_t} of \refstruct{pmix_proc_t} containing
the members of the specified process group. REQUIRED QUALIFIERS: \refattr{PMIX_GROUP_ID}.
}
%

\vspace{\baselineskip}
The following attributes are used as directives in \ac{PMIx} Group operations:

\declareAttributeNEW{PMIX_GROUP_ID}{"pmix.grp.id"}{char*}{
User-provided group identifier - as the group identifier may be used in
\ac{PMIx} operations, the user is required to ensure that the provided ID is unique within the scope of the host environment (e.g., by including some user-specific or application-specific prefix or suffix to the string).
}
%
\declareAttributeNEW{PMIX_GROUP_LEADER}{"pmix.grp.ldr"}{bool}{
This process is the leader of the group.
}
%
\declareAttributeNEW{PMIX_GROUP_OPTIONAL}{"pmix.grp.opt"}{bool}{
Participation is optional - do not return an error if any of the specified processes terminate without having joined. The default is \code{false}.
}
%
\declareAttributeNEW{PMIX_GROUP_NOTIFY_TERMINATION}{"pmix.grp.notterm"}{bool}{
Notify remaining members when another member terminates without first leaving the group.
}
%
\declareAttributeNEW{PMIX_GROUP_FT_COLLECTIVE}{"pmix.grp.ftcoll"}{bool}{
Adjust internal tracking on-the-fly for terminated processes during a \ac{PMIx} group collective operation.
}
%
\declareAttributeNEW{PMIX_GROUP_MEMBERSHIP}{"pmix.grp.mbrs"}{pmix_data_array_t*}{
Array \refstruct{pmix_proc_t} identifiers identifying the members of the specified group.
}
%
\declareAttributeNEW{PMIX_GROUP_ASSIGN_CONTEXT_ID}{"pmix.grp.actxid"}{bool}{
Requests that the \ac{RM} assign a new context identifier to the newly created group. The identifier is an unsigned, \code{size_t} value that the \ac{RM} guarantees to be unique across the range specified in the request. Thus, the value serves as a means of identifying the group within that range. If no range is specified, then the request defaults to \refconst{PMIX_RANGE_SESSION}.
}
%
\declareAttributeNEW{PMIX_GROUP_LOCAL_ONLY}{"pmix.grp.lcl"}{bool}{
Group operation only involves local processes. \ac{PMIx} implementations are \textit{required} to automatically scan an array of group members for local vs remote processes - if only local processes are detected, the implementation need not execute a global collective for the operation unless a context ID has been requested from the host environment. This can result in significant time savings. This attribute can be used to optimize the operation by indicating whether or not only local processes are represented, thus allowing the implementation to bypass the scan.
}

\vspace{\baselineskip}
The following attributes are used to return information at the conclusion of a \ac{PMIx} Group operation and/or in event notifications:

%
\declareAttributeNEW{PMIX_GROUP_CONTEXT_ID}{"pmix.grp.ctxid"}{size_t}{
Context identifier assigned to the group by the host \ac{RM}.
}
%
\declareAttributeNEW{PMIX_GROUP_ENDPT_DATA}{"pmix.grp.endpt"}{pmix_byte_object_t}{
Data collected during group construction to ensure communication between group members is supported upon completion of the operation.
}

\vspace{\baselineskip}
In addition, a process can request (via \refapi{PMIx_Get}) the process groups to which a given process (including itself) belongs:

%
\declareAttributeNEW{PMIX_GROUP_NAMES}{"pmix.pgrp.nm"}{pmix_data_array_t*}{
Returns an array of \code{char*} string names of the process groups in which the given process is a member.
}

%%%%%%%%%%%%%%%%%%%%%%%%%%%%%%%%%%%%%%%%%%%%%%%%%
\subsection{\code{PMIx_Group_construct}}
\declareapi{PMIx_Group_construct}

%%%%
\summary

Construct a \ac{PMIx} process group.

%%%%
\format

\copySignature{PMIx_Group_construct}{4.0}{
pmix_status_t \\
PMIx_Group_construct(const char grp[], \\
\hspace*{21\sigspace}const pmix_proc_t procs[], size_t nprocs, \\
\hspace*{21\sigspace}const pmix_info_t directives[], \\
\hspace*{21\sigspace}size_t ndirs, \\
\hspace*{21\sigspace}pmix_info_t **results, \\
\hspace*{21\sigspace}size_t *nresults);
}

\begin{arglist}
\argin{grp}{\code{NULL}-terminated character array of maximum size \refconst{PMIX_MAX_NSLEN} containing the group identifier (string)}
\argin{procs}{Array of \refstruct{pmix_proc_t} structures containing the \ac{PMIx} identifiers of the member processes (array of handles)}
\argin{nprocs}{Number of elements in the \refarg{procs} array (\code{size_t})}
\argin{directives}{Array of \refstruct{pmix_info_t} structures (array of handles)}
\argin{ndirs}{Number of elements in the \refarg{directives} array (\code{size_t})}
\arginout{results}{Pointer to a location where the array of \refstruct{pmix_info_t} describing the results of the operation is to be returned (pointer to handle)}
\arginout{nresults}{Pointer to a \code{size_t} location where the number of elements in \refarg{results} is to be returned (memory reference)}
\end{arglist}

Returns one of the following:

\begin{itemize}
    \item \refconst{PMIX_SUCCESS}, indicating that the request has been successfully completed
    \item \refconst{PMIX_ERR_NOT_SUPPORTED} The \ac{PMIx} library and/or the host \ac{RM} does not support this operation
    \item a \ac{PMIx} error constant indicating either an error in the input or that the request failed to be completed
\end{itemize}

\reqattrstart
The following attributes are \textit{required} to be supported by all \ac{PMIx} libraries that support this operation:

\pasteAttributeItem{PMIX_GROUP_LEADER}
\pasteAttributeItem{PMIX_GROUP_OPTIONAL}
\pasteAttributeItem{PMIX_GROUP_LOCAL_ONLY}
\pasteAttributeItem{PMIX_GROUP_FT_COLLECTIVE}

Host environments that support this operation are \textit{required} to support the following attributes:

\pasteAttributeItem{PMIX_GROUP_ASSIGN_CONTEXT_ID}
\pasteAttributeItem{PMIX_GROUP_NOTIFY_TERMINATION}

\reqattrend

\optattrstart
The following attributes are optional for host environments that support this operation:

\pasteAttributeItem{PMIX_TIMEOUT}

\optattrend

%%%%
\descr

Construct a new group composed of the specified processes and identified with the provided group identifier. The group identifier is a user-defined, \code{NULL}-terminated character array of length less than or equal to \refconst{PMIX_MAX_NSLEN}. Only characters accepted by standard string comparison functions (e.g., \emph{strncmp}) are supported. Processes may engage in multiple simultaneous group construct operations so long as each is provided with a unique group ID. The \refarg{directives} array can be used to pass user-level directives regarding timeout constraints and other options available from the \ac{PMIx} server.

If the \refattr{PMIX_GROUP_NOTIFY_TERMINATION} attribute is provided and has a value of \code{true}, then either the construct leader (if \refattr{PMIX_GROUP_LEADER} is provided) or all participants who register for the \refconst{PMIX_GROUP_MEMBER_FAILED} event will receive events whenever a process fails or terminates prior to calling \refapi{PMIx_Group_construct} – i.e. if a \emph{group leader} is declared, \textit{only} that process will receive the event. In the absence of a declared leader, \textit{all} specified group members will receive the event.

The event will contain the identifier of the process that failed to join plus any other information that the host \ac{RM} provided. This provides an opportunity for the leader or the collective members to react to the event – e.g., to decide to proceed with a smaller group or to abort the operation. The decision is communicated to the \ac{PMIx} library in the results array at the end of the event handler. This allows \ac{PMIx} to properly adjust accounting for procedure completion. When construct is complete, the participating \ac{PMIx} servers will be alerted to any change in participants and each group member will receive an updated group membership (marked with the \refattr{PMIX_GROUP_MEMBERSHIP} attribute) as part of the \refarg{results} array returned by this \ac{API}.

Failure of the declared leader at any time will cause a \refconst{PMIX_GROUP_LEADER_FAILED} event to be delivered to all participants so they can optionally declare a new leader. A new leader is identified by providing the \refattr{PMIX_GROUP_LEADER} attribute in the results array in the return of the event handler. Only one process is allowed to return that attribute, thereby declaring itself as the new leader. Results of the leader selection will be communicated to all participants via a \refconst{PMIX_GROUP_LEADER_SELECTED} event identifying the new leader. If no leader was selected, then the \refstruct{pmix_info_t} provided to that event handler will include that information so the participants can take appropriate action.

Any participant that returns \refconst{PMIX_GROUP_CONSTRUCT_ABORT} from either the \refconst{PMIX_GROUP_MEMBER_FAILED} or the \refconst{PMIX_GROUP_LEADER_FAILED} event handler will cause the construct process to abort, returning from the call with a \refconst{PMIX_GROUP_CONSTRUCT_ABORT} status.

If the \refattr{PMIX_GROUP_NOTIFY_TERMINATION} attribute is not provided or has a value of \code{false}, then the \refapi{PMIx_Group_construct} operation will simply return an error whenever a proposed group member fails or terminates prior to calling \refapi{PMIx_Group_construct}.

Providing the \refattr{PMIX_GROUP_OPTIONAL} attribute with a value of \code{true} directs the \ac{PMIx} library to consider participation by any specified group member as non-required - thus, the operation will return \refconst{PMIX_SUCCESS} if all members participate, or \refconst{PMIX_ERR_PARTIAL_SUCCESS} if some members fail to participate. The \refarg{results} array will contain the final group membership in the latter case. Note that this use-case can cause the operation to hang if the \refattr{PMIX_TIMEOUT} attribute is not specified and one or more group members fail to call \refapi{PMIx_Group_construct} while continuing to execute. Also, note that no leader or member failed events will be generated during the operation.

Processes in a group under construction are not allowed to leave the group until group construction is complete. Upon completion of the construct procedure, each group member will have access to the job-level information of all namespaces represented in the group plus any information posted via \refapi{PMIx_Put} (subject to the usual scoping directives) for every group member.

\adviceimplstart
At the conclusion of the construct operation, the \ac{PMIx} library is \emph{required} to ensure that job-related information from each participating namespace plus any information posted by group members via \refapi{PMIx_Put} (subject to scoping directives) is available to each member via calls to \refapi{PMIx_Get}.
\adviceimplend

\advicermstart
The collective nature of this \ac{API} generally results in use of a fence-like operation by the backend host environment. Host environments that utilize the array of process participants as a \emph{signature} for such operations may experience potential conflicts should both a \refapi{PMIx_Group_construct} and a \refapi{PMIx_Fence} operation involving the same participants be simultaneously executed. As \ac{PMIx} allows for such use-cases, it is therefore the responsibility of the host environment to resolve any potential conflicts.
\advicermend

%%%%%%%%%%%%%%%%%%%%%%%%%%%%%%%%%%%%%%%%%%%%%%%%%
\subsection{\code{PMIx_Group_construct_nb}}
\declareapi{PMIx_Group_construct_nb}

%%%%
\summary

Non-blocking form of \refapi{PMIx_Group_construct}.

%%%%
\format

\copySignature{PMIx_Group_construct_nb}{4.0}{
pmix_status_t \\
PMIx_Group_construct_nb(const char grp[], \\
\hspace*{24\sigspace}const pmix_proc_t procs[], size_t nprocs, \\
\hspace*{24\sigspace}const pmix_info_t directives[], \\
\hspace*{24\sigspace}size_t ndirs, \\
\hspace*{24\sigspace}pmix_info_cbfunc_t cbfunc, void *cbdata);
}

\begin{arglist}
\argin{grp}{\code{NULL}-terminated character array of maximum size \refconst{PMIX_MAX_NSLEN} containing the group identifier (string)}
\argin{procs}{Array of \refstruct{pmix_proc_t} structures containing the \ac{PMIx} identifiers of the member processes (array of handles)}
\argin{nprocs}{Number of elements in the \refarg{procs} array (\code{size_t})}
\argin{directives}{Array of \refstruct{pmix_info_t} structures (array of handles)}
\argin{ndirs}{Number of elements in the \refarg{directives} array (\code{size_t})}
\argin{cbfunc}{Callback function \refapi{pmix_info_cbfunc_t} (function reference)}
\argin{cbdata}{Data to be passed to the callback function (memory reference)}
\end{arglist}

Returns one of the following:

\begin{itemize}
\item \refconst{PMIX_SUCCESS} indicating that the request has been accepted for processing and the provided callback function will be executed upon completion of the operation. Note that the library \emph{must not} invoke the callback function prior to returning from the \ac{API}.
\item \refconst{PMIX_OPERATION_SUCCEEDED}, indicating that the request was immediately processed and returned \textit{success} - the \refarg{cbfunc} will \textit{not} be called.
\item \refconst{PMIX_ERR_NOT_SUPPORTED} The \ac{PMIx} library does not support this operation - the \refarg{cbfunc} will \textit{not} be called.
\item a non-zero \ac{PMIx} error constant indicating a reason for the request to have been rejected - the \refarg{cbfunc} will \textit{not} be called.
\end{itemize}

If executed, the status returned in the provided callback function will be one of the following constants:

\begin{itemize}
\item \refconst{PMIX_SUCCESS} The operation succeeded and all specified members participated.
\item \refconst{PMIX_ERR_PARTIAL_SUCCESS} The operation succeeded but not all specified members participated - the final group membership is included in the callback function.
\item \refconst{PMIX_ERR_NOT_SUPPORTED} While the \ac{PMIx} server supports this operation, the host \ac{RM} does not.
\item a non-zero \ac{PMIx} error constant indicating a reason for the request's failure.
\end{itemize}

\reqattrstart
\ac{PMIx} libraries that choose not to support this operation \textit{must} return \refconst{PMIX_ERR_NOT_SUPPORTED} when the function is called.

The following attributes are \textit{required} to be supported by all \ac{PMIx} libraries that support this operation:

\pasteAttributeItem{PMIX_GROUP_LEADER}
\pasteAttributeItem{PMIX_GROUP_OPTIONAL}
\pasteAttributeItem{PMIX_GROUP_LOCAL_ONLY}
\pasteAttributeItem{PMIX_GROUP_FT_COLLECTIVE}

Host environments that support this operation are \textit{required} to provide the following attributes:

\pasteAttributeItem{PMIX_GROUP_ASSIGN_CONTEXT_ID}
\pasteAttributeItem{PMIX_GROUP_NOTIFY_TERMINATION}

\reqattrend

\optattrstart
The following attributes are optional for host environments that support this operation:

\pasteAttributeItem{PMIX_TIMEOUT}

\optattrend

%%%%
\descr

Non-blocking version of the \refapi{PMIx_Group_construct} operation. The callback function will be called once all group members have called either \refapi{PMIx_Group_construct} or \refapi{PMIx_Group_construct_nb}.

%%%%%%%%%%%%%%%%%%%%%%%%%%%%%%%%%%%%%%%%%%%%%%%%%
\subsection{\code{PMIx_Group_destruct}}
\declareapi{PMIx_Group_destruct}

%%%%
\summary

Destruct a \ac{PMIx} process group.

%%%%
\format

\copySignature{PMIx_Group_destruct}{4.0}{
pmix_status_t \\
PMIx_Group_destruct(const char grp[], \\
\hspace*{20\sigspace}const pmix_info_t directives[], \\
\hspace*{20\sigspace}size_t ndirs);
}

\begin{arglist}
\argin{grp}{\code{NULL}-terminated character array of maximum size \refconst{PMIX_MAX_NSLEN} containing the identifier of the group to be destructed (string)}
\argin{directives}{Array of \refstruct{pmix_info_t} structures (array of handles)}
\argin{ndirs}{Number of elements in the \refarg{directives} array (\code{size_t})}
\end{arglist}

Returns one of the following:

\begin{itemize}
    \item \refconst{PMIX_SUCCESS}, indicating that the request has been successfully completed
    \item \refconst{PMIX_ERR_NOT_SUPPORTED} The \ac{PMIx} library and/or the host \ac{RM} does not support this operation
    \item a \ac{PMIx} error constant indicating either an error in the input or that the request failed to be completed
\end{itemize}

\reqattrstart
For implementations and host environments that support the operation, there are no identified required
attributes for this \ac{API}.
\reqattrend

\optattrstart
The following attributes are optional for host environments that support this operation:

\pasteAttributeItem{PMIX_TIMEOUT}

\optattrend

%%%%
\descr

Destruct a group identified by the provided group identifier. Processes may engage in multiple simultaneous group destruct operations so long as each involves a unique group ID. The \refarg{directives} array can be used to pass user-level directives regarding timeout constraints and other options available from the \ac{PMIx} server.

The destruct \ac{API} will return an error if any group process fails or terminates prior to calling \refapi{PMIx_Group_destruct} or its non-blocking version unless the \refattr{PMIX_GROUP_NOTIFY_TERMINATION} attribute was provided (with a value of \code{false}) at time of group construction. If notification was requested, then the \refconst{PMIX_GROUP_MEMBER_FAILED} event will be delivered for each process that fails to call destruct and the destruct tracker updated to account for the lack of participation. The \refapi{PMIx_Group_destruct} operation will subsequently return \refconst{PMIX_SUCCESS} when the remaining processes have all called destruct – i.e., the event will serve in place of return of an error.

\advicermstart
The collective nature of this \ac{API} generally results in use of a fence-like operation by the backend host environment. Host environments that utilize the array of process participants as a \emph{signature} for such operations may experience potential conflicts should both a \refapi{PMIx_Group_destruct} and a \refapi{PMIx_Fence} operation involving the same participants be simultaneously executed. As \ac{PMIx} allows for such use-cases, it is therefore the responsibility of the host environment to resolve any potential conflicts.
\advicermend

%%%%%%%%%%%%%%%%%%%%%%%%%%%%%%%%%%%%%%%%%%%%%%%%%
\subsection{\code{PMIx_Group_destruct_nb}}
\declareapi{PMIx_Group_destruct_nb}

%%%%
\summary

Non-blocking form of \refapi{PMIx_Group_destruct}.

%%%%
\format

\copySignature{PMIx_Group_destruct_nb}{4.0}{
pmix_status_t \\
PMIx_Group_destruct_nb(const char grp[], \\
\hspace*{23\sigspace}const pmix_info_t directives[], \\
\hspace*{23\sigspace}size_t ndirs, \\
\hspace*{23\sigspace}pmix_op_cbfunc_t cbfunc, void *cbdata);
}

\begin{arglist}
\argin{grp}{\code{NULL}-terminated character array of maximum size \refconst{PMIX_MAX_NSLEN} containing the identifier of the group to be destructed (string)}
\argin{directives}{Array of \refstruct{pmix_info_t} structures (array of handles)}
\argin{ndirs}{Number of elements in the \refarg{directives} array (\code{size_t})}
\argin{cbfunc}{Callback function \refapi{pmix_op_cbfunc_t} (function reference)}
\argin{cbdata}{Data to be passed to the callback function (memory reference)}
\end{arglist}

Returns one of the following:

\begin{itemize}
    \item \refconst{PMIX_SUCCESS}, indicating that the request is being processed - result will be returned in the provided \refarg{cbfunc}. Note that the library \emph{must not} invoke the callback function prior to returning from the \ac{API}.
    \item \refconst{PMIX_OPERATION_SUCCEEDED}, indicating that the request was immediately processed and returned \textit{success} - the \refarg{cbfunc} will \textit{not} be called
    \item \refconst{PMIX_ERR_NOT_SUPPORTED} The \ac{PMIx} library does not support this operation - the \refarg{cbfunc} will \textit{not} be called.
    \item a \ac{PMIx} error constant indicating either an error in the input or that the request was immediately processed and failed - the \refarg{cbfunc} will \textit{not} be called.
\end{itemize}

If executed, the status returned in the provided callback function will be one of the following constants:

\begin{itemize}
\item \refconst{PMIX_SUCCESS} The operation was successfully completed.
\item \refconst{PMIX_ERR_NOT_SUPPORTED} While the \ac{PMIx} server supports this operation, the host \ac{RM} does not.
\item a non-zero \ac{PMIx} error constant indicating a reason for the request's failure.
\end{itemize}

\reqattrstart
\ac{PMIx} libraries that choose not to support this operation \textit{must} return \refconst{PMIX_ERR_NOT_SUPPORTED} when the function is called. For implementations and host environments that support the operation, there are no identified required
attributes for this \ac{API}.
\reqattrend

\optattrstart
The following attributes are optional for host environments that support this operation:

\pasteAttributeItem{PMIX_TIMEOUT}

\optattrend

%%%%
\descr

Non-blocking version of the \refapi{PMIx_Group_destruct} operation. The callback function will be called once all members of the group have executed either \refapi{PMIx_Group_destruct} or \refapi{PMIx_Group_destruct_nb}.

%%%%%%%%%%%%%%%%%%%%%%%%%%%%%%%%%%%%%%%%%%%%%%%%%
\subsection{\code{PMIx_Group_invite}}
\declareapi{PMIx_Group_invite}

%%%%
\summary

Asynchronously construct a \ac{PMIx} process group.

%%%%
\format

\copySignature{PMIx_Group_invite}{4.0}{
pmix_status_t \\
PMIx_Group_invite(const char grp[], \\
\hspace*{18\sigspace}const pmix_proc_t procs[], size_t nprocs, \\
\hspace*{18\sigspace}const pmix_info_t directives[], size_t ndirs, \\
\hspace*{18\sigspace}pmix_info_t **results, size_t *nresult);
}

\begin{arglist}
\argin{grp}{\code{NULL}-terminated character array of maximum size \refconst{PMIX_MAX_NSLEN} containing the group identifier (string)}
\argin{procs}{Array of \refstruct{pmix_proc_t} structures containing the \ac{PMIx} identifiers of the processes to be invited (array of handles)}
\argin{nprocs}{Number of elements in the \refarg{procs} array (\code{size_t})}
\argin{directives}{Array of \refstruct{pmix_info_t} structures (array of handles)}
\argin{ndirs}{Number of elements in the \refarg{directives} array (\code{size_t})}
\arginout{results}{Pointer to a location where the array of \refstruct{pmix_info_t} describing the results of the operation is to be returned (pointer to handle)}
\arginout{nresults}{Pointer to a \code{size_t} location where the number of elements in \refarg{results} is to be returned (memory reference)}
\end{arglist}

Returns one of the following:

\begin{itemize}
    \item \refconst{PMIX_SUCCESS}, indicating that the request has been successfully completed.
    \item \refconst{PMIX_ERR_NOT_SUPPORTED} The \ac{PMIx} library and/or the host \ac{RM} does not support this operation.
    \item a \ac{PMIx} error constant indicating either an error in the input or that the request failed to be completed.
\end{itemize}

\reqattrstart
The following attributes are \textit{required} to be supported by all \ac{PMIx} libraries that support this operation:

\pasteAttributeItem{PMIX_GROUP_OPTIONAL}
\pasteAttributeItem{PMIX_GROUP_FT_COLLECTIVE}

Host environments that support this operation are \textit{required} to provide the following attributes:

\pasteAttributeItem{PMIX_GROUP_ASSIGN_CONTEXT_ID}
\pasteAttributeItem{PMIX_GROUP_NOTIFY_TERMINATION}
\reqattrend

\optattrstart
The following attributes are optional for host environments that support this operation:

\pasteAttributeItem{PMIX_TIMEOUT}

\optattrend

%%%%
\descr

Explicitly invite the specified processes to join a group. The process making the \refapi{PMIx_Group_invite} call is automatically declared to be the \emph{group leader}. Each invited process will be notified of the invitation via the \refconst{PMIX_GROUP_INVITED} event - the processes being invited must therefore register for the \refconst{PMIX_GROUP_INVITED} event in order to be notified of the invitation. Note that the \ac{PMIx} event notification system caches events - thus, no ordering of invite versus event registration is required.

The invitation event will include the identity of the inviting process plus the name of the group. When ready to respond, each invited process provides a response using either the blocking or non-blocking form of \refapi{PMIx_Group_join}. This will notify the inviting process that the invitation was either accepted (via the \refconst{PMIX_GROUP_INVITE_ACCEPTED} event) or declined (via the \refconst{PMIX_GROUP_INVITE_DECLINED} event). The \refconst{PMIX_GROUP_INVITE_ACCEPTED} event is captured by the \ac{PMIx} client library of the inviting process – i.e., the application itself does not need to register for this event. The library will track the number of accepting processes and alert the inviting process (by returning from the blocking form of \refapi{PMIx_Group_invite} or calling the callback function of the non-blocking form) when group construction completes.

The inviting process should, however, register for the \refconst{PMIX_GROUP_INVITE_DECLINED} if the application allows invited processes to decline the invitation. This provides an opportunity for the application to either invite a replacement, declare ``abort'', or choose to remove the declining process from the final group. The inviting process should also register to receive \refconst{PMIX_GROUP_INVITE_FAILED} events whenever a process fails or terminates prior to responding to the invitation. Actions taken by the inviting process in response to these events must be communicated at the end of the event handler by returning the corresponding result so that the \ac{PMIx} library can adjust accordingly.

Upon completion of the operation, all members of the new group will receive access to the job-level information of each other’s namespaces plus any information posted via \refapi{PMIx_Put} by the other members.

The inviting process is automatically considered the leader of the asynchronous group construction procedure and will receive all failure or termination events for invited members prior to completion. The inviting process is required to provide a \refconst{PMIX_GROUP_CONSTRUCT_COMPLETE} event once the group has been fully assembled – this event is used by the \ac{PMIx} library as a trigger to release participants from their call to \refapi{PMIx_Group_join} and provides information (e.g., the final group membership) to be returned in the \refarg{results} array.

Failure of the inviting process at any time will cause a \refconst{PMIX_GROUP_LEADER_FAILED} event to be delivered to all participants so they can optionally declare a new leader. A new leader is identified by providing the \refattr{PMIX_GROUP_LEADER} attribute in the results array in the return of the event handler. Only one process is allowed to return that attribute, declaring itself as the new leader. Results of the leader selection will be communicated to all participants via a \refconst{PMIX_GROUP_LEADER_SELECTED} event identifying the new leader. If no leader was selected, then the status code provided in the event handler will provide an error value so the participants can take appropriate action.

\adviceuserstart
Applications are not allowed to use the group in any operations until group construction is complete. This is required in order to ensure consistent knowledge of group membership across all participants.
\adviceuserend


%%%%%%%%%%%%%%%%%%%%%%%%%%%%%%%%%%%%%%%%%%%%%%%%%
\subsection{\code{PMIx_Group_invite_nb}}
\declareapi{PMIx_Group_invite_nb}

%%%%
\summary

Non-blocking form of \refapi{PMIx_Group_invite}.

%%%%
\format

\copySignature{PMIx_Group_invite_nb}{4.0}{
pmix_status_t \\
PMIx_Group_invite_nb(const char grp[], \\
\hspace*{21\sigspace}const pmix_proc_t procs[], size_t nprocs, \\
\hspace*{21\sigspace}const pmix_info_t directives[], size_t ndirs, \\
\hspace*{21\sigspace}pmix_info_cbfunc_t cbfunc, void *cbdata);
}

\begin{arglist}
\argin{grp}{\code{NULL}-terminated character array of maximum size \refconst{PMIX_MAX_NSLEN} containing the group identifier (string)}
\argin{procs}{Array of \refstruct{pmix_proc_t} structures containing the \ac{PMIx} identifiers of the processes to be invited (array of handles)}
\argin{nprocs}{Number of elements in the \refarg{procs} array (\code{size_t})}
\argin{directives}{Array of \refstruct{pmix_info_t} structures (array of handles)}
\argin{ndirs}{Number of elements in the \refarg{directives} array (\code{size_t})}
\argin{cbfunc}{Callback function \refapi{pmix_info_cbfunc_t} (function reference)}
\argin{cbdata}{Data to be passed to the callback function (memory reference)}
\end{arglist}

Returns one of the following:

\begin{itemize}
    \item \refconst{PMIX_SUCCESS}, indicating that the request is being processed - result will be returned in the provided \refarg{cbfunc}. Note that the library \emph{must not} invoke the callback function prior to returning from the \ac{API}.
    \item \refconst{PMIX_OPERATION_SUCCEEDED}, indicating that the request was immediately processed and returned \textit{success} - the \refarg{cbfunc} will \textit{not} be called.
    \item \refconst{PMIX_ERR_NOT_SUPPORTED} The \ac{PMIx} library does not support this operation - the \refarg{cbfunc} will \textit{not} be called.
    \item a PMIx error constant indicating either an error in the input or that the request was immediately processed and failed - the \refarg{cbfunc} will \textit{not} be called.
\end{itemize}

If executed, the status returned in the provided callback function will be one of the following constants:

\begin{itemize}
\item \refconst{PMIX_SUCCESS} The operation succeeded and all specified members participated.
\item \refconst{PMIX_ERR_PARTIAL_SUCCESS} The operation succeeded but not all specified members participated - the final group membership is included in the callback function.
\item \refconst{PMIX_ERR_NOT_SUPPORTED} While the \ac{PMIx} server supports this operation, the host \ac{RM} does not.
\item a non-zero \ac{PMIx} error constant indicating a reason for the request's failure.
\end{itemize}

\reqattrstart
The following attributes are \textit{required} to be supported by all \ac{PMIx} libraries that support this operation:

\pasteAttributeItem{PMIX_GROUP_OPTIONAL}
\pasteAttributeItem{PMIX_GROUP_FT_COLLECTIVE}

Host environments that support this operation are \textit{required} to provide the following attributes:

\pasteAttributeItem{PMIX_GROUP_ASSIGN_CONTEXT_ID}
\pasteAttributeItem{PMIX_GROUP_NOTIFY_TERMINATION}

\reqattrend

\optattrstart
The following attributes are optional for host environments that support this operation:

\pasteAttributeItem{PMIX_TIMEOUT}

\optattrend

%%%%
\descr

Non-blocking version of the \refapi{PMIx_Group_invite} operation. The callback function will be called once all invited members of the group (or their substitutes) have executed either \refapi{PMIx_Group_join} or \refapi{PMIx_Group_join_nb}.

%%%%%%%%%%%%%%%%%%%%%%%%%%%%%%%%%%%%%%%%%%%%%%%%%
\subsection{\code{PMIx_Group_join}}
\declareapi{PMIx_Group_join}

%%%%
\summary

Accept an invitation to join a \ac{PMIx} process group.

%%%%
\format

\copySignature{PMIx_Group_join}{4.0}{
pmix_status_t \\
PMIx_Group_join(const char grp[], \\
\hspace*{16\sigspace}const pmix_proc_t *leader, \\
\hspace*{16\sigspace}pmix_group_opt_t opt, \\
\hspace*{16\sigspace}const pmix_info_t directives[], size_t ndirs, \\
\hspace*{16\sigspace}pmix_info_t **results, size_t *nresult);
}

\begin{arglist}
\argin{grp}{\code{NULL}-terminated character array of maximum size \refconst{PMIX_MAX_NSLEN} containing the group identifier (string)}
\argin{leader}{Process that generated the invitation (handle)}
\argin{opt}{Accept or decline flag (\refstruct{pmix_group_opt_t})}
\argin{directives}{Array of \refstruct{pmix_info_t} structures (array of handles)}
\argin{ndirs}{Number of elements in the \refarg{directives} array (\code{size_t})}
\arginout{results}{Pointer to a location where the array of \refstruct{pmix_info_t} describing the results of the operation is to be returned (pointer to handle)}
\arginout{nresults}{Pointer to a \code{size_t} location where the number of elements in \refarg{results} is to be returned (memory reference)}
\end{arglist}

Returns one of the following:

\begin{itemize}
    \item \refconst{PMIX_SUCCESS}, indicating that the request has been successfully completed.
    \item \refconst{PMIX_ERR_NOT_SUPPORTED} The \ac{PMIx} library and/or the host \ac{RM} does not support this operation.
    \item a \ac{PMIx} error constant indicating either an error in the input or that the request failed to be completed.
\end{itemize}

\reqattrstart
There are no identified required attributes for implementers.

\reqattrend

\optattrstart
The following attributes are optional for host environments that support this operation:

\pasteAttributeItem{PMIX_TIMEOUT}

\optattrend

%%%%
\descr

Respond to an invitation to join a group that is being asynchronously constructed. The process must have registered for the \refconst{PMIX_GROUP_INVITED} event in order to be notified of the invitation. When called, the event information will include the \refstruct{pmix_proc_t} identifier of the process that generated the invitation along with the identifier of the group being constructed. When ready to respond, the process provides a response using either form of \refapi{PMIx_Group_join}.

\adviceuserstart
Since the process is alerted to the invitation in a \ac{PMIx} event handler, the process \emph{must not} use the blocking form of this call unless it first ``thread shifts'' out of the handler and into its own thread context. Likewise, while it is safe to call the non-blocking form of the \ac{API} from the event handler, the process \emph{must not} block in the handler while waiting for the callback function to be called.
\adviceuserend

Calling this function causes the inviting process (aka the \emph{group leader}) to be notified that the process has either accepted or declined the request. The blocking form of the \ac{API} will return once the group has been completely constructed or the group’s construction has failed (as described below) – likewise, the callback function of the non-blocking form will be executed upon the same conditions.

Failure of the leader during the call to \refapi{PMIx_Group_join} will cause a \refconst{PMIX_GROUP_LEADER_FAILED} event to be delivered to all invited participants so they can optionally declare a new leader. A new leader is identified by providing the \refattr{PMIX_GROUP_LEADER} attribute in the results array in the return of the event handler. Only one process is allowed to return that attribute, declaring itself as the new leader. Results of the leader selection will be communicated to all participants via a \refconst{PMIX_GROUP_LEADER_SELECTED} event identifying the new leader. If no leader was selected, then the status code provided in the event handler will provide an error value so the participants can take appropriate action.

Any participant that returns \refconst{PMIX_GROUP_CONSTRUCT_ABORT} from the leader failed event handler will cause all participants to receive an event notifying them of that status. Similarly, the leader may elect to abort the procedure by either returning \refconst{PMIX_GROUP_CONSTRUCT_ABORT} from the handler assigned to the \refconst{PMIX_GROUP_INVITE_ACCEPTED} or \refconst{PMIX_GROUP_INVITE_DECLINED} codes, or by generating an event for the abort code. Abort events will be sent to all invited participants.


%%%%%%%%%%%%%%%%%%%%%%%%%%%%%%%%%%%%%%%%%%%%%%%%%
\subsection{\code{PMIx_Group_join_nb}}
\declareapi{PMIx_Group_join_nb}

%%%%
\summary

Non-blocking form of \refapi{PMIx_Group_join}

%%%%
\format

\copySignature{PMIx_Group_join_nb}{4.0}{
pmix_status_t \\
PMIx_Group_join_nb(const char grp[], \\
\hspace*{19\sigspace}const pmix_proc_t *leader, \\
\hspace*{19\sigspace}pmix_group_opt_t opt, \\
\hspace*{19\sigspace}const pmix_info_t directives[], size_t ndirs, \\
\hspace*{19\sigspace}pmix_info_cbfunc_t cbfunc, void *cbdata);
}

\begin{arglist}
\argin{grp}{\code{NULL}-terminated character array of maximum size \refconst{PMIX_MAX_NSLEN} containing the group identifier (string)}
\argin{leader}{Process that generated the invitation (handle)}
\argin{opt}{Accept or decline flag (\refstruct{pmix_group_opt_t})}
\argin{directives}{Array of \refstruct{pmix_info_t} structures (array of handles)}
\argin{ndirs}{Number of elements in the \refarg{directives} array (\code{size_t})}
\argin{cbfunc}{Callback function \refapi{pmix_info_cbfunc_t} (function reference)}
\argin{cbdata}{Data to be passed to the callback function (memory reference)}
\end{arglist}

Returns one of the following:

\begin{itemize}
    \item \refconst{PMIX_SUCCESS}, indicating that the request is being processed - result will be returned in the provided \refarg{cbfunc}. Note that the library \emph{must not} invoke the callback function prior to returning from the \ac{API}.
    \item \refconst{PMIX_OPERATION_SUCCEEDED}, indicating that the request was immediately processed and returned \textit{success} - the \refarg{cbfunc} will \textit{not} be called.
    \item \refconst{PMIX_ERR_NOT_SUPPORTED} The \ac{PMIx} library does not support this operation - the \refarg{cbfunc} will \textit{not} be called.
    \item a PMIx error constant indicating either an error in the input or that the request was immediately processed and failed - the \refarg{cbfunc} will \textit{not} be called.
\end{itemize}

If executed, the status returned in the provided callback function will be one of the following constants:

\begin{itemize}
\item \refconst{PMIX_SUCCESS} The operation succeeded and group membership is in the callback function parameters.
\item \refconst{PMIX_ERR_NOT_SUPPORTED} While the \ac{PMIx} server supports this operation, the host \ac{RM} does not.
\item a non-zero \ac{PMIx} error constant indicating a reason for the request's failure.
\end{itemize}


\reqattrstart
There are no identified required attributes for implementers.

\reqattrend

\optattrstart
The following attributes are optional for host environments that support this operation:

\pasteAttributeItem{PMIX_TIMEOUT}

\optattrend

%%%%
\descr

Non-blocking version of the \refapi{PMIx_Group_join} operation. The callback function will be called once all invited members of the group (or their substitutes) have executed either \refapi{PMIx_Group_join} or \refapi{PMIx_Group_join_nb}.

%%%%%%%%%%%%%%%%%%%%%%%%%%%%%%%%%%%%%%%%%%%%%%%%%
\subsubsection{Group accept/decline directives}
\declarestruct{pmix_group_opt_t}

\versionMarker{4.0}
The \refstruct{pmix_group_opt_t} type is a \code{uint8_t} value used with the \refapi{PMIx_Group_join} \ac{API} to indicate \emph{accept} or \emph{decline} of the invitation - these are provided for readability of user code:

\begin{constantdesc}
%
\declareconstitem{PMIX_GROUP_DECLINE}
Decline the invitation.
%
\declareconstitem{PMIX_GROUP_ACCEPT}
Accept the invitation.
%
\end{constantdesc}


%%%%%%%%%%%%%%%%%%%%%%%%%%%%%%%%%%%%%%%%%%%%%%%%%
\subsection{\code{PMIx_Group_leave}}
\declareapi{PMIx_Group_leave}

%%%%
\summary

Leave a \ac{PMIx} process group.

%%%%
\format

\copySignature{PMIx_Group_leave}{4.0}{
pmix_status_t \\
PMIx_Group_leave(const char grp[], \\
\hspace*{17\sigspace}const pmix_info_t directives[], \\
\hspace*{17\sigspace}size_t ndirs);
}

\begin{arglist}
\argin{grp}{\code{NULL}-terminated character array of maximum size \refconst{PMIX_MAX_NSLEN} containing the group identifier (string)}
\argin{directives}{Array of \refstruct{pmix_info_t} structures (array of handles)}
\argin{ndirs}{Number of elements in the \refarg{directives} array (\code{size_t})}
\end{arglist}

Returns one of the following:

\begin{itemize}
    \item \refconst{PMIX_SUCCESS}, indicating that the request has been communicated to the local \ac{PMIx} server.
    \item \refconst{PMIX_ERR_NOT_SUPPORTED} The \ac{PMIx} library and/or the host \ac{RM} does not support this operation.
    \item a \ac{PMIx} error constant indicating either an error in the input or that the request is unsupported.
\end{itemize}

\reqattrstart
There are no identified required attributes for implementers.
\reqattrend


%%%%
\descr

Calls to \refapi{PMIx_Group_leave} (or its non-blocking form) will cause a \refconst{PMIX_GROUP_LEFT} event to be generated notifying all members of the group of the caller’s departure. The function will return (or the non-blocking function will execute the specified callback function) once the event has been locally generated and is not indicative of remote receipt.

\adviceuserstart
The \refapi{PMIx_Group_leave} API is intended solely for asynchronous departures of individual processes from a group as it is not a scalable operation – i.e., when a process determines it should no longer be a part of a defined group, but the remainder of the group retains a valid reason to continue in existence. Developers are advised to use \refapi{PMIx_Group_destruct} (or its non-blocking form) for all other scenarios as it represents a more scalable operation.
\adviceuserend

%%%%%%%%%%%%%%%%%%%%%%%%%%%%%%%%%%%%%%%%%%%%%%%%%
\subsection{\code{PMIx_Group_leave_nb}}
\declareapi{PMIx_Group_leave_nb}

%%%%
\summary

Non-blocking form of \refapi{PMIx_Group_leave}.

%%%%
\format

\copySignature{PMIx_Group_leave_nb}{4.0}{
pmix_status_t \\
PMIx_Group_leave_nb(const char grp[], \\
\hspace*{20\sigspace}const pmix_info_t directives[], \\
\hspace*{20\sigspace}size_t ndirs, \\
\hspace*{20\sigspace}pmix_op_cbfunc_t cbfunc, \\
\hspace*{20\sigspace}void *cbdata);
}

\begin{arglist}
\argin{grp}{\code{NULL}-terminated character array of maximum size \refconst{PMIX_MAX_NSLEN} containing the group identifier (string)}
\argin{directives}{Array of \refstruct{pmix_info_t} structures (array of handles)}
\argin{ndirs}{Number of elements in the \refarg{directives} array (\code{size_t})}
\argin{cbfunc}{Callback function \refapi{pmix_op_cbfunc_t} (function reference)}
\argin{cbdata}{Data to be passed to the callback function (memory reference)}
\end{arglist}

Returns one of the following:

\begin{itemize}
    \item \refconst{PMIX_SUCCESS}, indicating that the request is being processed - result will be returned in the provided \refarg{cbfunc}. Note that the library \emph{must not} invoke the callback function prior to returning from the \ac{API}.
    \item \refconst{PMIX_OPERATION_SUCCEEDED}, indicating that the request was immediately processed and returned \textit{success} - the \refarg{cbfunc} will \textit{not} be called.
    \item \refconst{PMIX_ERR_NOT_SUPPORTED} The \ac{PMIx} library does not support this operation - the \refarg{cbfunc} will \textit{not} be called.
    \item a PMIx error constant indicating either an error in the input or that the request was immediately processed and failed - the \refarg{cbfunc} will \textit{not} be called.
\end{itemize}

If executed, the status returned in the provided callback function will be one of the following constants:

\begin{itemize}
\item \refconst{PMIX_SUCCESS} The operation succeeded - i.e., the \refconst{PMIX_GROUP_LEFT} event was generated.
\item \refconst{PMIX_ERR_NOT_SUPPORTED} While the \ac{PMIx} library supports this operation, the host \ac{RM} does not.
\item a non-zero \ac{PMIx} error constant indicating a reason for the request's failure.
\end{itemize}


\reqattrstart
There are no identified required attributes for implementers.

\reqattrend

%%%%
\descr

Non-blocking version of the \refapi{PMIx_Group_leave} operation. The callback function will be called once the event has been locally generated and is not indicative of remote receipt.


%%%%%%%%%%%%%%%%%%%%%%%%%%%%%%%%%%%%%%%%%%%%%%%%%


    % PMIx Network Coordinates
    %%%%%%%%%%%%%%%%%%%%%%%%%%%%%%%%%%%%%%%%%%%%%%%%%
% Chapter: Network Coordinates
%%%%%%%%%%%%%%%%%%%%%%%%%%%%%%%%%%%%%%%%%%%%%%%%%
\chapter{Network Coordinates}
\label{chap:network_coords}

As the drive for performance continues, interest has grown in optimizing collective communication patterns by structuring them to follow network topology. For example, one might aggregate the contribution from all processes on a node, then again across all nodes on a common switch, and finally across all switches. Creating such optimized patterns therefore relies on detailed knowledge of the network location of each participant.

\ac{PMIx} supports these efforts by defining datatypes and attributes by which network coordinates for processes and devices can be obtained from the host \ac{SMS}. When used in conjunction with the \ac{PMIx} \emph{instant on} methods, this results in the ability of a process to obtain the network coordinate of all other processes without incurring additional overhead associated with the publish/exchange of that information.

%%%%%%%%%%%
\section{Network Coordinate Datatypes}

Several datatype definitions have been created to support network coordinates.

%%%%%%%%%%
\subsection{Network Coordinate Structure}
\declarestruct{pmix_coord_t}

The \refstruct{pmix_coord_t} structure describes the network coordinates of a specified process in a given view

\versionMarker{4.0}
\cspecificstart
\begin{codepar}
typedef struct pmix_coord \{
    pmix_coord_view_t view;
    uint32_t *coord;
    size_t dims;
\} pmix_coord_t;
\end{codepar}
\cspecificend

All coordinate values shall be expressed as unsigned integers due to their units being defined in network devices and not physical distances. The coordinate is therefore an indicator of connectivity and not relative communication distance.

\adviceimplstart
Note that the \refstruct{pmix_coord_t} structure does not imply nor mandate any requirement on how the coordinate data is to be stored within the \ac{PMIx} library. Implementers are free to store the coordinate in whatever format they choose.
\adviceimplend

A network coordinate is usually associated with a given network device - e.g., a particular \ac{NIC} on a node. Thus, while the network coordinate of a device must be unique in a given view, the coordinate may be shared by multiple processes on a node. If the node contains multiple network devices, then either the device closest to the binding location of a process shall be used as its coordinate, or (if the process is unbound or its binding is not known) all devices on the node shall be reported as a \refstruct{pmix_data_array_t} of \refstruct{pmix_coord_t} structures.

Nodes with multiple network devices can also have those devices configured as multiple \refterm{network planes}. In such cases, a given process (even if bound to a specific location) may be associated with a coordinate on each plane. The resulting set of network coordinates shall be reported as a \refstruct{pmix_data_array_t} of \refstruct{pmix_coord_t} structures. The caller may request a coordinate from a specific network plane by passing the \refattr{PMIX_NETWORK_PLANE} attribute as a directive/qualifier to the \refapi{PMIx_Get} or \refapi{PMIx_Query_info_nb} call.

\subsection{Network Coordinate Views}
\declarestruct{pmix_coord_view_t}

\versionMarker{4.0}
\cspecificstart
\begin{codepar}
typedef uint8_t pmix_coord_view_t;
#define PMIX_COORD_VIEW_UNDEF       0x00
#define PMIX_COORD_LOGICAL_VIEW     0x01
#define PMIX_COORD_PHYSICAL_VIEW    0x02
\end{codepar}
\cspecificend

Network coordinates can be reported based on different \emph{views} according to user preference at the time of request. The following views have been defined:

\begin{constantdesc}
%
\declareconstitemNEW{PMIX_COORD_VIEW_UNDEF}
The coordinate view has not been defined.
%
\declareconstitemNEW{PMIX_COORD_LOGICAL_VIEW}
The coordinates are provided in a \emph{logical} view, typically given in Cartesian (x,y,z) dimensions, that describes the data flow in the network as defined by the arrangement of the hierarchical addressing scheme, network segmentation, routing domains, and other similar factors employed by that network.
%
\declareconstitemNEW{PMIX_COORD_PHYSICAL_VIEW}
The coordinates are provided in a \emph{physical} view based on the actual wiring diagram of the network - i.e., values along each axis reflect the relative position of that interface on the specific network cabling.
%
\end{constantdesc}

\adviceimplstart
\ac{PMIx} library implementers are advised to avoid declaring the above constants as actual \code{enum} values in order to allow host environments to add support for possibly proprietary coordinate views.
\adviceimplend

If the requester does not specify a view, coordinates shall default to the \emph{logical} view.


\subsection{Network Coordinate Error Constants}
\label{api:netcoord:errors}

The following error constants are used by \ac{PMIx} to notify registered processes of events that affect network coordinates.

\begin{constantdesc}

%
\declareconstitemNEW{PMIX_NETWORK_COORDS_UPDATED}
Network coordinates have been updated - the affected networks/planes are identified in the notification. Coordinates of processes and devices on those affected components should be refreshed prior to next use.

\end{constantdesc}


%%%%%%%%%%%
\subsection{Network Descriptive Attributes}
\label{api:struct:attributes:netinfo}

These attributes are used to describe information about network resources as assigned by the \ac{RM}, and thus are referenced using the process rank except where noted.

%
\declareNewAttribute{PMIX_NETWORK_COORDINATE}{"pmix.net.coord"}{pmix_data_array_t}{
Network coordinate(s) of the specified process in the view and/or plane provided by the requester. If only one \ac{NIC} has been assigned to the specified process, then the array will contain only one address. Otherwise, the array will contain the coordinates of all \acp{NIC} available to the process in order of least to greatest distance from the process (\acp{NIC} equally distant from the process will be listed in arbitrary order).
}

%
\declareNewAttribute{PMIX_NETWORK_VIEW}{"pmix.net.view"}{pmix_coord_view_t}{
Network coordinate view to be used for the requested data - see \refstruct{pmix_coord_view_t} for the list of accepted values.
}

%
\declareNewAttribute{PMIX_NETWORK_DIMS}{"pmix.net.dims"}{uint32_t}{
Request number of dimensions in the specified network plane/view. If no plane is specified, then the dimensions of all planes in the system will be returned as a \refstruct{pmix_data_array_t} containing an array of \code{uint32_t} values. Default is to provide dimensions in \emph{logical} view.
}

%
\declareNewAttribute{PMIX_NETWORK_PLANE}{"pmix.net.plane"}{char*}{
ID string of a network plane (e.g., CIDR for Ethernet). When used as a modifier in a request for information, specifies the plane whose information is to be returned. When used directly in a request, returns a \refstruct{pmix_data_array_t} of string identifiers for all network planes in the system.
}

%
\declareNewAttribute{PMIX_NETWORK_NIC}{"pmix.net.nic"}{char*}{
ID string of a network interface card (NIC). When used as a modifier in a request for information, specifies the \ac{NIC} whose information is to be returned. When used directly in a request, returns a \refstruct{pmix_data_array_t} of string identifiers for all \acp{NIC} in the specified network plane. If no plane is specified, then the \ac{NIC} identifiers of each plane in the system will be returned in an array where each element is in turn an array of strings containing the network plane ID followed by the identifiers of the \acp{NIC} attached to that plane.
}

%
\declareNewAttribute{PMIX_NETWORK_ENDPT}{"pmix.net.endpt"}{pmix_data_array_t}{
Network endpoints for a specified process. As multiple endpoints may be assigned to a given process (e.g., in the case where multiple \acp{NIC} are associated with a socket to which the process is bound), the returned values will be provided in a \refstruct{pmix_data_array_t} - the returned data type of the individual values in the array varies by fabric provider.
}

%
\declareNewAttribute{PMIX_NETWORK_SHAPE}{"pmix.net.shape"}{pmix_data_array_t*}{
The size of each dimension in the specified network plane/view, returned in a \refstruct{pmix_data_array_t} containing an array of \code{uint32_t} values. The size is defined as the number of elements present in that dimension - e.g., the number of \acp{NIC} in one dimension of a physical view of a network plane. If no plane is specified, then the shape of each plane in the system will be returned in an array of network shapes. Default is to provide the shape in \emph{logical} view.
}


%%%%%%%%%%%%%%%%%%%%%%%%%%%%%%%%%%%%%%%%%%%%%%%%%


%
% Appendix
%
    \setcounter{chapter}{0}  % restart chapter numbering with "letter A"
    \renewcommand{\thechapter}{\Alph{chapter}}%
    \appendix

    % Support funcitons outside of the standard
%    %%%%%%%%%%%%%%%%%%%%%%%%%%%%%%%%%%%%%%%%%%%%%%%%%
% Appendix: Support functions
%%%%%%%%%%%%%%%%%%%%%%%%%%%%%%%%%%%%%%%%%%%%%%%%%
\chapter{PMIx Support Functions}
\label{app:support}

This chapter describes some additional support APIs that are provided in the \ac{PRI} headers, but are not part of the core \ac{PMIx} standard specification as the macros reference internal \ac{PRI} functions exposed in the \ac{PRI}'s external headers.


%%%%%%%%%%%
\section{Data Structure Support}

This section describes some additional support macros focused on the data structures defined in \chapterref{chap:struct}.


%%%%%%%%%%%
\subsection{Argument array extension}
\declaremacro{PMIX_ARGV_APPEND}

%%%%
\summary

Append a string to a NULL-terminated, argv-style array of strings.

\cspecificstart
\begin{codepar}
PMIX_ARGV_APPEND(r, a, b);
\end{codepar}
\cspecificend

\begin{arglist}
\argout{r}{Status code indicating success or failure of the operation (\refstruct{pmix_status_t})}
\arginout{a}{Argument list (pointer to NULL-terminated array of strings)}
\argin{b}{Argument to append to the list (string)}
\end{arglist}

%%%%
\descr

This function helps the caller build the \code{argv} portion of \refstruct{pmix_app_t} structure, arrays of keys for querying, or other places where argv-style string arrays are required in the way that the \ac{PRI} expects it to be constructed.

\adviceuserstart
The provided argument is copied into the destination array - thus, the source string can be free'd without affecting the array once the macro has completed.
\adviceuserend

%%%%%%%%%%%
\section{Environment Manipulation Support}

This section describes some additional support APIs focused on environment manipulation.

%%%%%%%%%%%
\subsection{Set an environment variable}
\declaremacro{PMIX_SETENV}

%%%%
\summary

Set an environment variable in a NULL-terminated, env-style array

\cspecificstart
\begin{codepar}
PMIX_SETENV(r, name, value, env);
\end{codepar}
\cspecificend


\begin{arglist}
\argout{r}{Status code indicating success or failure of the operation (\refstruct{pmix_status_t})}
\argin{name}{Argument name (string)}
\argin{value}{Argument value (string)}
\arginout{env}{Environment array to update (pointer to array of strings)}
\end{arglist}

%%%%
\descr

Similar to \code{setenv} from the C API, this allows the caller to set an environment variable in the specified \code{env} array, which could then be passed to the \refstruct{pmix_app_t} structure or any other destination.

\adviceuserstart
The provided name and value are copied into the destination environment array - thus, the source strings can be free'd without affecting the array once the macro has completed.
\adviceuserend


%%%%%%%%%%%%%%%%%%%%%%%%%%%%%%%%%%%%%%%%%%%%%%%%%


    % Revisions, Acknowledgements
    %%%%%%%%%%%%%%%%%%%%%%%%%%%%%%%%%%%%%%%%%%%%%%%%%
% Chapter: Acknowledgements
%%%%%%%%%%%%%%%%%%%%%%%%%%%%%%%%%%%%%%%%%%%%%%%%%
\chapter{Acknowledgements}
\label{chap:acknowledgements}

This document represents the work of many people who have contributed to the PMIx community.
Without the hard work and dedication of these people this document would not have been possible.
The sections below list some of the active participants and organizations in the various PMIx standard iterations.

%%%%%%%%%% Version 2.0
\section{Version 2.0}

The following list includes some of the active participants in the PMIx v2 standardization process.

\begin{itemize}
\item Ralph H. Castain, Annapurna Dasari, Christopher A. Holguin, Andrew Friedley, Michael Klemm and Terry Wilmarth
\item Joshua Hursey, David Solt, Alexander Eichenberger, Geoff Paulsen, and Sameh Sharkawi
\item Aurelien Bouteiller and George Bosilca
\item Artem Polyakov, Igor Ivanov and Boris Karasev
\item Gilles Gouaillardet
\item Michael A Raymond and Jim Stoffel
\item Dirk Schubert
\item Moe Jette
\item Takahiro Kawashima and Shinji Sumimoto
\item Howard Pritchard
\item David Beer
\item Brice Goglin
\item Geoffroy Vallee, Swen Boehm, Thomas Naughton and David Bernholdt
\item Adam Moody and Martin Schulz
\item Ryan Grant and Stephen Olivier
\item Michael Karo
\end{itemize}

The following institutions supported this effort through time and travel support for the people listed above.

\begin{itemize}
\item Intel Corporation
\item IBM, Inc.
\item University of Tennessee, Knoxville
\item The Exascale Computing Project, an initiative of the US Department of Energy
\item National Science Foundation
\item Mellanox, Inc.
\item Research Organization for Information Science and Technology
\item HPE Co.
\item Allinea (ARM)
\item SchedMD, Inc.
\item Fujitsu Limited
\item Los Alamos National Laboratory
\item Adaptive Solutions, Inc.
\item INRIA
\item Oak Ridge National Laboratory
\item Lawrence Livermore National Laboratory
\item Sandia National Laboratory
\item Altair
\end{itemize}


%%%%%%%%%% Version 1.0
\section{Version 1.0}

The following list includes some of the active participants in the PMIx v1 standardization process.

\begin{itemize}
\item Ralph H. Castain, Annapurna Dasari and Christopher A. Holguin
\item Joshua Hursey and David Solt
\item Aurelien Bouteiller and George Bosilca
\item Artem Polyakov, Elena Shipunova, Igor Ivanov, and Joshua Ladd
\item Gilles Gouaillardet
\item Gary Brown
\item Moe Jette
\end{itemize}

The following institutions supported this effort through time and travel support for the people listed above.

\begin{itemize}
\item Intel Corporation
\item IBM, Inc.
\item University of Tennessee, Knoxville
\item Mellanox, Inc.
\item Research Organization for Information Science and Technology
\item Adaptive Solutions, Inc.
\item SchedMD, Inc.
\end{itemize}


%
% Bibliography
%
	\nolinenumbers
	\bibliography{pmix-standard}{}
	\addcontentsline{toc}{chapter}{Bibliography}
	\bibliographystyle{plain}

%
% Index
%
	\nolinenumbers
	\printindex

\end{document}

%%%%%%%%%%%%%%%%%%%%%%%%%%%%%%%%%%%%%%%%%%%%%%%%%
