% Welcome to pmix-standard.tex.
% This is the master LaTex file for the PMIx Standard document.
%
% The files in this set include:
%
%    pmix-standard.tex                - this file, the master file
%    Makefile                         - makes the document
%    pmix.sty                         - the main style file
%    Title_Page.tex                   - the title page
%    Chap_Introduction.tex            - unnumbered introductory chapter
%    figs/*.png                       - Figures
%    sources/*.c, *.f                 - C/C++/Fortran example source files
%
% When editing this file:
%
%    1. To change formatting, appearance, or style, please edit pmix.sty.
%
%    2. Custom commands and macros are defined in pmix.sty.
%
%    3. Be kind to other editors -- keep a consistent style by copying-and-pasting to
%       create new content.
%
%    4. We use semantic markup, e.g. (see pmix.sty for a full list):
%         \code{}     % for bold monospace keywords, code, operators, etc.
%
%    5. Other recommendations:
%         Use the convenience macros defined in pmix.sty for the minor headers
%         such as Comments, Syntax, etc.
%
%         To keep items together on the same page, prefer the use of
%         \begin{samepage}.... Avoid \parbox for text blocks as it interrupts line numbering.
%         When possible, avoid \filbreak, \pagebreak, \newpage, \clearpage unless that's
%         what you mean. Use \needspace{} cautiously for troublesome paragraphs.
%
%         Avoid absolute lengths and measures in this file; use relative units when possible.
%         Vertical space can be relative to \baselineskip or ex units. Horizontal space
%         can be relative to \linewidth or em units.
%
%         Prefer \emph{} to italicize terminology, e.g.:
%             This is a \emph{definition}, not a placeholder.
%             This is a \plc{var-name}.
%

% The following says letter size, but the style sheet may change the size
\documentclass[10pt,letterpaper,twoside,makeidx,hidelinks]{scrreprt}

%%%%%%%%%%%%%%%%%%%
% Release Managers:
% - Set is_unofficial_draft = false to remove the watermark
% - Set VER to the correct version number
% - Set VERDATE to the Month+Year of the release
%%%%%%%%%%%%%%%%%%%
\usepackage[us,24hr]{datetime}
\usepackage{ifthen}
\newboolean{is_unofficial_draft}
\setboolean{is_unofficial_draft}{true}

% Text to appear in the footer on even-numbered pages:
\ifthenelse{\boolean{is_unofficial_draft}}
  {\newcommand{\VER}{5.0 (Draft)}
   \newcommand{\VERDATE}{\emph{Created on \today}}
  }
  {\newcommand{\VER}{5.0}
   \newcommand{\VERDATE}{Month Year}
  }
\newcommand{\footerText}{PMIx Standard -- Version \VER{} -- \VERDATE}

% Unified style sheet for PMIx documents:
% This is pmix.sty, the preamble and style definitions for the PMIx specification.
%
% This specification file, and latex structure was derived from/inspired by the OpenMP specification. So some similarity between the two latex files is expected.
%
%%%%%%%%%%%%%%%%%%%%%%%%%%%%%%%%%%%%%%%%%%%%%%%%%%%%%%%%%%%%%%%%%%%%%%%%%%%%%%%%%%%%%%%%%%%%%
% Quick list of the environments, commands and macros supported.
% Search below for more details.
%
% Formatting Text:
%   -----------------------
%   \notestart            - "Note:" Callout section
%   \noteheader           - \noteheader is optional "Note:" prefix for text
%     ...
%   \noteend
%   -----------------------
%   \rationalestart       - "Rationale" Callout section
%     ...
%   \rationaleend
%   -----------------------
%   \adviceuserstart      - "Advice to users" Callout section
%     ...
%   \adviceuserend
%   -----------------------
%   \adviceimplstart      - "Advice to PMIx library implementers" Callout section
%     ...
%   \adviceimplend
%   -----------------------
%   \advicermstart      - "Advice to PMIx server hosts" Callout section
%     ...
%   \advicermend
%   -----------------------
%
% Formatting Code:
%   \code{}               - Code text
%   \var{}                - Variable
%   -----------------------
%   \begin{codepar}       - Section of generic code
%     ...                 - use language specific macro if language specific code
%   \end[codepar}
%   -----------------------
%   \cspecificstart       - C specific code block
%     ...
%   \cspecificend
%   -----------------------
%
% Attributes:
%   \refAttributeItem{}   - Cross reference
%   \refattr{}            - Same as above
%   \pasteAttributeItem{} - Paste full description
%
% Structures:
%   \refstruct{}          - Reference a structure
%   \structref{}          - Same as above
%   \specrefstruct{}      - Reference a structure by section number and page
%
% APIs:
%   \refapi{}             - Reference an API function
%   \refconst{}           - Constant reference
%   \refarg{} / \argref{} - Reference an argument to an API function
%
% Cross referencing:
%   \chapterref{}         - Reference a Chapter by number and page
%   \specref{}            - Reference a Section by number and page
%
%%%%%%%%%%%%%%%%%%%%%%%%%%%%%%%%%%%%%%%%%%%%%%%%%%%%%%%%%%%%%%%%%%%%%%%%%%%%%%%%%%%%%%%%%%%%%
\usepackage{comment}            % allow use of \begin{comment}
\usepackage{ifpdf,ifthen}       % allow conditional tests in LaTeX definitions
\usepackage{makecell}           % Allows common formatting in cells with \thread & \makecell

\usepackage[T1]{fontenc}        % Allow us to use underscore freely in the document
\catcode`\_=12                  % Use \sb for subscripts
\usepackage{verbatim}


%%%%%%%%%%%%%%%%%%%%%%%%%%%%%%%%%%%%%%%%%%%%%%%%%%%%%%%%%%%%%%%%%%%%%%%%%%%%%%%%%%%%%%%%%%%%%
% Document data
%
\author{}


%%%%%%%%%%%%%%%%%%%%%%%%%%%%%%%%%%%%%%%%%%%%%%%%%%%%%%%%%%%%%%%%%%%%%%%%%%%%%%%%%%%%%%%%%%%%%
% Fonts

\usepackage{amsmath}
\usepackage{amsfonts}
\usepackage{amssymb}
\usepackage{courier}
\usepackage{helvet}
\usepackage[utf8]{inputenc}
\usepackage{textgreek}

% Main body serif font:
\usepackage{tgtermes}
\usepackage[T1]{fontenc}


%%%%%%%%%%%%%%%%%%%%%%%%%%%%%%%%%%%%%%%%%%%%%%%%%%%%%%%%%%%%%%%%%%%%%%%%%%%%%%%%%%%%%%%%%%%%%
% Graphic elements

\usepackage{graphicx}
\usepackage{framed}    % for making boxes with \begin{framed}
\usepackage{tikz}      % for flow charts, diagrams, arrows


%%%%%%%%%%%%%%%%%%%%%%%%%%%%%%%%%%%%%%%%%%%%%%%%%%%%%%%%%%%%%%%%%%%%%%%%%%%%%%%%%%%%%%%%%%%%%
% Page formatting

\usepackage[paperwidth=7.5in, paperheight=9in,
            top=0.75in, bottom=1.0in, left=1.4in, right=0.6in]{geometry}

\usepackage{changepage}   % allows left/right-page margin readjustments

\setlength{\oddsidemargin}{0.185in}
\setlength{\evensidemargin}{0.185in}
\raggedbottom


%%%%%%%%%%%%%%%%%%%%%%%%%%%%%%%%%%%%%%%%%%%%%%%%%%%%%%%%%%%%%%%%%%%%%%%%%%%%%%%%%%%%%%%%%%%%%
% Paragraph formatting

\usepackage{setspace}     % allows use of \singlespacing, \onehalfspacing
\usepackage{needspace}    % allows use of \needspace to keep lines together
\usepackage{parskip}      % removes paragraph indenting

\raggedright
\usepackage[raggedrightboxes]{ragged2e}  % is this needed?

\lefthyphenmin=60         % only hyphenate if the left part is >= this many chars
\righthyphenmin=60        % only hyphenate if the right part is >= this many chars


%%%%%%%%%%%%%%%%%%%%%%%%%%%%%%%%%%%%%%%%%%%%%%%%%%%%%%%%%%%%%%%%%%%%%%%%%%%%%%%%%%%%%%%%%%%%%%
% Bulleted (itemized) lists
%    Align bullets with section header
%    Align text left
%    Small bullets
%    \compactitem for single-spaced lists (used in the Examples doc)

\usepackage{enumitem}     % for setting margins on lists
\setlist{leftmargin=*}    % don't indent bullet items
\renewcommand{\labelitemi}{{\normalsize$\bullet$}} % bullet size

% There is a \compactitem defined in package parlist (and perhaps others), however,
% we'll define our own version of compactitem in terms of package enumitem that
% we already use:
\newenvironment{compactitem}
{\begin{itemize}[itemsep=-1.2ex]}
{\end{itemize}}

%%%%%%%%%%%%%%%%%%%%%%%%%%%%%%%%%%%%%%%%%%%%%%%%%%%%%%%%%%%%%%%%%%%%%%%%%%%%%%%%%%%%%%%%%%%%%
% Floating version
%\usepackage[showboxes]{textpos}
\usepackage{textpos}

\setlength{\TPHorizModule}{1pt}%
\setlength{\TPVertModule}{\TPHorizModule}%
\TPMargin{1pt}%

\newcommand{\versionMarker}[1]{%
 % y is 8 = \parskip
 \begin{textblock}{50}(-55,8)%
   \textit{PMIx v#1}%
   \raggedright
 \end{textblock}%
}
% Alternative is to make a box inline, but that gets tricky when positioning close
% to codepar's
% \makebox[-7pt][r]{\textit{PMIx #4}\raggedright}

%%%%%%%%%%%%%%%%%%%%%%%%%%%%%%%%%%%%%%%%%%%%%%%%%%%%%%%%%%%%%%%%%%%%%%%%%%%%%%%%%%%%%%%%%%%%%%
% Enumerated list with lowercase alphabet lettering
%    \alphaenum for default-spaced lists
%    \compactalphaenum for single-spaced lists

% There is a \compactitem defined in package parlist (and perhaps others), however,
% we'll define our own version of compactitem in terms of package enumitem that
% we already use:
\newenvironment{alphaenum}
{\begin{enumerate}[label=\alph*)]}
{\end{enumerate}}

\newenvironment{compactalphaenum}
{\begin{enumerate}[label=\alph*),itemsep=-1.2ex]}
{\end{enumerate}}

% Argument list for an interface, for use in a \begin{arglist} section
% \argin      Input argument
% \argout     Output argument
% \arginout   Input/Output argument
% \argreturn  Value returned
%%% Old Method using tables.... line numbers didn't work if a cell wrapped...
%\newlength\argdesclen
%\setlength\argdesclen{\dimexpr \linewidth -13em -4\tabcolsep}
%\newenvironment{arglist}{%
%    \begin{edtable}{tabular}{p{3em}p{10em}p{\argdesclen}}}
%    {\end{edtable}\vspace{.25em}}
%
%\newcommand{\argin}[2]{\textbf{IN} & \code{#1} & #2\\}
%\newcommand{\argout}[2]{\textbf{OUT} & \code{#1} & #2\\}
%\newcommand{\arginout}[2]{\textbf{INOUT} & \code{#1} & #2\\}

\newenvironment{arglist}
{\begin{description}[style=nextline,labelindent=\parindent,leftmargin=*,itemindent=\dimexpr-17pt-\labelsep\relax,itemsep=-1.3ex]}
{\end{description}}

\newcommand{\argin}[2]{\item[IN ~~~~\code{#1}] #2}
\newcommand{\argout}[2]{\item[OUT ~~~\code{#1}] #2}
\newcommand{\arginout}[2]{\item[INOUT ~\code{#1}] #2}

% Constant list
%   \declareconstitem  Declare constant with description
\newenvironment{constantdesc}
{\begin{description}[itemsep=-1.3ex,itemindent=\dimexpr-17pt-\labelsep\relax]}
{\end{description}}

\newcommand{\declareconstitem}[1]{\item[\code{#1}] \index{#1} \label{const:#1} \hspace{1em}}
\newcommand{\declareconstitemvalue}[2]{\item[\code{#1}] \index{#1} \hspace{0.25em} \code{#2}  \hspace{1em}}
\newcommand{\declareconstitemDEP}[2]{\item[\code{#1} (Deprecated in PMIx #2)] \index{#1} \label{const:#1} \hspace{1em}}
\newcommand{\declareconstitemNEW}[1]{\item[\color{magenta}\code{#1}] \index{#1} \label{const:#1} \hspace{1em}}


%%%%%%%%%%%%%%%%%%%%%%%%%%%%%%%%%%%%%%%%%%%%%%%%%%%%%%%%%%%%%%%%%%%%%%%%%%%%%%%%%%%%%%%%%%%%%%
% Tables

% This allows tables to flow across page breaks, headers on each new page, etc.
\usepackage{supertabular}
\usepackage{caption}
\usepackage{longtable}
\usepackage{pdflscape} % for 'landscape' environment

%%%%%%%%%%%%%%%%%%%%%%%%%%%%%%%%%%%%%%%%%%%%%%%%%%%%%%%%%%%%%%%%%%%%%%%%%%%%%%%%%%%%%%%%%%%%%
% Line numbering

\usepackage[pagewise,edtable]{lineno}       % for line numbers on left side of the page
\pagewiselinenumbers
\setlength\linenumbersep{6em}
\renewcommand\linenumberfont{\normalfont\small\sffamily}
\nolinenumbers            % start with line numbers off


%%%%%%%%%%%%%%%%%%%%%%%%%%%%%%%%%%%%%%%%%%%%%%%%%%%%%%%%%%%%%%%%%%%%%%%%%%%%%%%%%%%%%%%%%%%%%
% Footers

\usepackage{fancyhdr}     % makes right/left footers
\pagestyle{fancy}
\fancyhead{} % clear all header fields
\cfoot{}
\renewcommand{\headrulewidth}{0pt}

% Left side on even pages:
% This requires that \footerText be defined in the master document:
\fancyfoot[LE]{\bfseries \thepage \mdseries \hspace{2em} \footerText}
\fancyhfoffset[E]{4em}

% Right side on odd pages:
\fancyfoot[RO]{\mdseries  \leftmark \hspace{2em} \bfseries \thepage}


%%%%%%%%%%%%%%%%%%%%%%%%%%%%%%%%%%%%%%%%%%%%%%%%%%%%%%%%%%%%%%%%%%%%%%%%%%%%%%%%%%%%%%%%%%%%%
% Section header format - we use five levels: \chapter \section \subsection \subsubsection

\usepackage{titlesec}     % format headers with \titleformat{}

% Format and spacing for chapter, section, subsection, and subsubsection headers:

\setcounter{secnumdepth}{5}          % show numbers down to subsubsection level

\titleformat{\chapter}[display]%
{\normalfont\sffamily\upshape\Huge\bfseries\nolinenumbers\fontsize{20}{20}\selectfont}%
{\normalfont\sffamily\scshape\large\bfseries\nolinenumbers \hspace{-0.7in} \MakeUppercase%
    {\chaptertitlename} \thechapter}%
{0em}{}[\vspace{1.0em}\hrule]
% {<left>}{<before-sep>}{<after-sep>}
\titlespacing{\chapter}{0ex}{0em plus 1em minus 1em}{1em plus 1em minus 1em}[10em]

\titleformat{\section}[hang]{\huge\bfseries\sffamily\fontsize{16}{16}\selectfont}{\thesection}{1.0em}{}
% {<left>}{<before-sep>}{<after-sep>}
\titlespacing{\section}{-5em}{2em plus 1em minus 1em}{1em plus 0.5em minus 0em}[10em]

\titleformat{\subsection}[hang]{\LARGE\bfseries\sffamily\fontsize{14}{14}\selectfont}{\thesubsection}{1.0em}{}
\titlespacing{\subsection}{-5em}{2em plus 1em minus 2.0em}{0.75em plus 0.5em minus 0em}[10em]

\titleformat{\subsubsection}[hang]{\needspace{1\baselineskip}%
\Large\bfseries\sffamily\fontsize{12}{12}\selectfont}{\thesubsubsection}{1.0em}{}
\titlespacing{\subsubsection}{-5em}{0.5em plus 1em minus 1em}{0.5em plus 0.5em minus 0em}[10em]


%%%%%%%%%%%%%%%%%%%%%%%%%%%%%%%%%%%%%%%%%%%%%%%%%%%%%%%%%%%%%%%%%%%%%%%%%%%%%%%%%%%%%%%%%%%%%%
% Macros for minor headers: Summary, Syntax, Description, etc.
% These headers are defined in terms of \paragraph

\titleformat{\paragraph}[block]{\large\bfseries\sffamily\fontsize{11}{11}\selectfont}{}{}{}
\titlespacing{\paragraph}{0em}{1.0em plus 0.55em minus 0.5em}{0.0em plus 0.55em minus 0.0em}

% Use one of the convenience macros below, or \littleheader{} for an arbitrary header
\newcommand{\littleheader}[1] {\paragraph*{#1}}

\newcommand{\comments} {\littleheader{Comments}}
\newcommand{\descr} {\littleheader{Description}}
\newcommand{\format} {\littleheader{Format}}
\newcommand{\summary} {\littleheader{Summary}}
\newcommand{\history} {\littleheader{History}}
\newcommand{\priattr} {\littleheader{PRI Attributes}}
\newcommand{\reqattr} {\littleheader{\ac{RM} Required Attributes}}
\newcommand{\optattr} {\littleheader{\ac{RM} Optional Attributes}}

%%%%%%%%%%%%%%%%%%%%%%%%%%%%%%%%%%%%%%%%%%%%%%%%%%%%%%%%%%%%%%%%%%%%%%%%%%%%%%%%%%%%%%%%%%%%%
% Clipboard
%
% \StdCopy{TAG}{BODY}
% \StdPaste{TAG}
%
% Inspired by this thread:
%   https://tex.stackexchange.com/questions/150790/how-to-make-text-be-copied-to-another-part-of-a-document
\makeatletter
\newcommand\StdCopy              [2] {
  \immediate\write\@auxout{\unexpanded{\global\long\@namedef{clipbrd@#1}{#2}}}
}
\newcommand\StdCopyEcho          [2] {
  \StdCopy{#1}{#2}%
  #2
}
\newcommand\StdPaste             [1] {%
  \ifcsname clipbrd@#1\endcsname
    \@nameuse{clipbrd@#1}%
  \else
    ??unknown??
  \fi
}
\makeatother


% Attributes
%   \declareAttribute       Declare an attribute with a description
%   \pasteAttributeItem     Paste the attribute description here
%   \refAttributeItem       Reference the original definition of the attribute
%
\newcommand{\declareAttribute}[4]{%
    \code{#1} ~~\code{#2}~~(\code{#3})%
    \index{#1!Definition|textbf} \label{attr:#1}%
    \StdCopy{str:#1}{\code{#2}}%
    \StdCopy{attr:#1}{\code{#3}}%
    \vspace{-1.3ex}%
      \expandafter\begin{adjustwidth}{.95cm}{}%
      \StdCopyEcho{#1}{#4}%
    \end{adjustwidth}%
  \vspace{-1.3ex}%
}

\newcommand{\declareNewAttribute}[4]{%
   {\color{magenta}\code{#1}} ~~\code{#2}~~(\code{#3})%
    \index{#1!Definition|textbf} \label{attr:#1}%
    \StdCopy{str:#1}{\code{#2}}%
    \StdCopy{attr:#1}{\code{#3}}%
    \vspace{-1.3ex}%
      \expandafter\begin{adjustwidth}{.95cm}{}%
      \StdCopyEcho{#1}{#4}%
    \end{adjustwidth}%
  \vspace{-1.3ex}%
}

\newcommand{\declareDepAttribute}[4]{%
   {\color{green!80!black}\code{#1}} ~~\code{#2}~~(\code{#3})%
    \index{#1!Definition|textbf} \label{attr:#1}%
    \StdCopy{str:#1}{\code{#2}}%
    \StdCopy{attr:#1}{\code{#3}}%
    \vspace{-1.3ex}%
      \expandafter\begin{adjustwidth}{.95cm}{}%
      \StdCopyEcho{#1}{#4}%
    \end{adjustwidth}%
  \vspace{-1.3ex}%
}

\newcommand{\pasteAttributeItemBegin}[1]{
  \refAttributeItem{#1} ~~\StdPaste{str:#1}~~(\StdPaste{attr:#1})
  \vspace{-1.3ex}
   \expandafter
   \begin{adjustwidth}{.95cm}{}
    \StdPaste{#1}
}
\newcommand{\pasteAttributeItemEnd}{
   \end{adjustwidth}
}
\newcommand{\pasteAttributeItem}[1]{
	\pasteAttributeItemBegin{#1}
	\pasteAttributeItemEnd{}
}
\newcommand{\refAttributeItem}[1]{\index{#1} \hyperref[attr:#1]{\code{#1}} }
\newcommand{\refattr}[1]{\refAttributeItem{#1}}

\newcommand{\refPRIAttributeItem}[1]{\index{#1} \hyperref[attr:#1]{\color{red}\code{#1}} }

\newcommand{\pastePRIAttributeItemBegin}[1]{
  \refPRIAttributeItem{#1} ~~\StdPaste{str:#1}~~(\StdPaste{attr:#1})
  \vspace{-1.3ex}
   \expandafter
   \begin{adjustwidth}{.95cm}{}
    \StdPaste{#1}
}
\newcommand{\pastePRIAttributeItemEnd}{
   \end{adjustwidth}
}

\newcommand{\pastePRIAttributeItem}[1]{
    \pastePRIAttributeItemBegin{#1}
    \pastePRIAttributeItemEnd{}
}

\newcommand{\refPRRTEAttributeItem}[1]{\index{#1} \hyperref[attr:#1]{\color{green!60!black}\code{#1}} }

\newcommand{\pastePRRTEAttributeItemBegin}[1]{
  \refPRRTEAttributeItem{#1} ~~\StdPaste{str:#1}~~(\StdPaste{attr:#1})
  \vspace{-1.3ex}
   \expandafter
   \begin{adjustwidth}{.95cm}{}
    \StdPaste{#1}
}
\newcommand{\pastePRRTEAttributeItemEnd}{
   \end{adjustwidth}
}

\newcommand{\pastePRRTEAttributeItem}[1]{
    \pastePRRTEAttributeItemBegin{#1}
    \pastePRRTEAttributeItemEnd{}
}

%%%%%%%%%%%%%%%%%%%%%%%%%%%%%%%%%%%%%%%%%%%%%%%%%%%%%%%%%%%%%%%%%%%%%%%%%%%%%%%%%%%%%%%%%%%%%%
% Code and placeholder semantic tagging.
%
% When possible, prefer semantic tags instead of typographic tags. The
% following semantics tags are defined here:
%
%     \code{}     % for bold monospace keywords, code, operators, etc.
%     \plc{}      % for italic placeholder names, grammar, etc.
%
% For function prototypes or other code snippets, you can use \code{} as
% the outer wrapper, and use \plc{{} inside. Example:
%
%     \code{\#pragma omp directive ( \plc{some-placeholder-identifier} :}
%
% To format text in italics for emphasis (rather than text as a placeholder),
% use the generic \emph{} command. Example:
%
%     This sentence \emph{emphasizes some non-placeholder words}.

% Enable \alltt{} for formatting blocks of code:
\usepackage{alltt}

% This sets the default \code{} font to tt (monospace) and bold:
\newcommand{\code}[1]{{\texttt{\textbf{#1}}}}
\newcommand{\var}[1] {{\textrm{\textmd{\itshape{#1}}}}}


% Environment for a paragraph of literal code, single-spaced, no outline, no indenting:
\newenvironment{codepar}[1]
{\begin{alltt}\bfseries #1}
{\end{alltt}}

\usepackage{setspace}

%%%%%%%%%%%%%%%%%%%%%%%%%%%%%%%%%%%%%%%%%%%%%%%%%%%%%%%%%%%%%%%%%%%%%%%%%%%%%%%%%%%%%%%%%%%%%%
% Macros for the black and blue lines and arrows delineating language-specific
% and notes sections. Example:
%
%   \fortranspecificstart
%   This is text that applies to Fortran.
%   \fortranspecificend

% local parameters for use \linewitharrows and \notelinewitharrows:
\newlength{\sbsz}\setlength{\sbsz}{0.05in}  % size of arrows
\newlength{\sblw}\setlength{\sblw}{1.35pt}  % line width (thickness)
\newlength{\sbtw}                           % text width
\newlength{\sblen}                          % total width of horizontal rule
\newlength{\sbht}                           % height of arrows
\newlength{\sbhadj}                         % vertical adjustment for aligning arrows with the line
\newlength{\sbns}\setlength{\sbns}{7\baselineskip}       % arg for \needspace for downward arrows

% \notelinewitharrows is a helper command that makes a black Note marker:
%     arg 1 = 1 or -1 for up or down arrows
%     arg 2 = solid or dashed or loosely dashed, etc.
\newcommand{\notelinewitharrows}[2]{%
    \needspace{0.1\baselineskip}%
    \vbox{\begin{tikzpicture}%
        \setlength{\sblen}{\linewidth}%
        \setlength{\sbht}{#1\sbsz}\setlength{\sbht}{1.4\sbht}%
        \setlength{\sbhadj}{#1\sblw}\setlength{\sbhadj}{0.25\sbhadj}%
        \filldraw (\sblen, 0) -- (\sblen - \sbsz, \sbht) -- (\sblen - 2\sbsz, 0) -- (\sblen, 0);
        \draw[line width=\sblw, #2] (2\sbsz - \sblw, \sbhadj) -- (\sblen - 2\sbsz + \sblw, \sbhadj);
        \filldraw (0, 0) -- (\sbsz, \sbht) -- (0 + 2\sbsz, 0) -- (0, 0);
    \end{tikzpicture}}}

% \adviceuserline is a helper command that makes a red horizontal line, up or down arrows, and some text:
% arg 1 = 1 or -1 for up or down arrows
% arg 2 = solid or dashed or loosely dashed, etc.
% arg 3 = text
% arg 4 = text width
\newcommand{\adviceuserline}[4]{%
    \needspace{0.1\baselineskip}%
    \vbox to 1\baselineskip {\begin{tikzpicture}%
        \setlength{\sbtw}{#4}%
        \setlength{\sblen}{\linewidth}%
        \setlength{\sbht}{#1\sbsz}\setlength{\sbht}{1.4\sbht}%
        \setlength{\sbhadj}{#1\sblw}\setlength{\sbhadj}{0.25\sbhadj}%
        \filldraw[color=red!80!black] (\sblen, 0) -- (\sblen - \sbsz, \sbht) -- (\sblen - 2\sbsz, 0) -- (\sblen, 0);
        \draw[line width=\sblw, color=red!80!black, #2] (2\sbsz - \sblw, \sbhadj) -- (0.5\sblen - 0.5\sbtw, \sbhadj);
        \draw[line width=\sblw, color=red!80!black, #2] (0.5\sblen + 0.5\sbtw, \sbhadj) -- (\sblen - 2\sbsz + \sblw, \sbhadj);
        \filldraw[color=red!80!black] (0, 0) -- (\sbsz, \sbht) -- (0 + 2\sbsz, 0) -- (0, 0);
        \node[color=red!80!black] at (0.5\sblen, 0) {\large  \textsf{\textup{#3}}};
    \end{tikzpicture}}}

% \adviceimpline is a helper command that makes a green horizontal line, up or down arrows, and some text:
% arg 1 = 1 or -1 for up or down arrows
% arg 2 = solid or dashed or loosely dashed, etc.
% arg 3 = text
% arg 4 = text width
\newcommand{\adviceimpline}[4]{%
    \needspace{0.1\baselineskip}%
    \vbox to 1\baselineskip {\begin{tikzpicture}%
        \setlength{\sbtw}{#4}%
        \setlength{\sblen}{\linewidth}%
        \setlength{\sbht}{#1\sbsz}\setlength{\sbht}{1.4\sbht}%
        \setlength{\sbhadj}{#1\sblw}\setlength{\sbhadj}{0.25\sbhadj}%
        \filldraw[color=green!60!black] (\sblen, 0) -- (\sblen - \sbsz, \sbht) -- (\sblen - 2\sbsz, 0) -- (\sblen, 0);
        \draw[line width=\sblw, color=green!60!black, #2] (2\sbsz - \sblw, \sbhadj) -- (0.5\sblen - 0.5\sbtw, \sbhadj);
        \draw[line width=\sblw, color=green!60!black, #2] (0.5\sblen + 0.5\sbtw, \sbhadj) -- (\sblen - 2\sbsz + \sblw, \sbhadj);
        \filldraw[color=green!60!black] (0, 0) -- (\sbsz, \sbht) -- (0 + 2\sbsz, 0) -- (0, 0);
        \node[color=green!60!black] at (0.5\sblen, 0) {\large  \textsf{\textup{#3}}};
    \end{tikzpicture}}}

% \advicermline is a helper command that makes an orange horizontal line, up or down arrows, and some text:
% arg 1 = 1 or -1 for up or down arrows
% arg 2 = solid or dashed or loosely dashed, etc.
% arg 3 = text
% arg 4 = text width
\newcommand{\advicermline}[4]{%
    \needspace{0.1\baselineskip}%
    \vbox to 1\baselineskip {\begin{tikzpicture}%
        \setlength{\sbtw}{#4}%
        \setlength{\sblen}{\linewidth}%
        \setlength{\sbht}{#1\sbsz}\setlength{\sbht}{1.4\sbht}%
        \setlength{\sbhadj}{#1\sblw}\setlength{\sbhadj}{0.25\sbhadj}%
        \filldraw[color=orange!60!black] (\sblen, 0) -- (\sblen - \sbsz, \sbht) -- (\sblen - 2\sbsz, 0) -- (\sblen, 0);
        \draw[line width=\sblw, color=orange!60!black, #2] (2\sbsz - \sblw, \sbhadj) -- (0.5\sblen - 0.5\sbtw, \sbhadj);
        \draw[line width=\sblw, color=orange!60!black, #2] (0.5\sblen + 0.5\sbtw, \sbhadj) -- (\sblen - 2\sbsz + \sblw, \sbhadj);
        \filldraw[color=orange!60!black] (0, 0) -- (\sbsz, \sbht) -- (0 + 2\sbsz, 0) -- (0, 0);
        \node[color=orange!60!black] at (0.5\sblen, 0) {\large  \textsf{\textup{#3}}};
    \end{tikzpicture}}}

% \ratline is a helper command that makes a purple horizontal line, up or down arrows, and some text:
% arg 1 = 1 or -1 for up or down arrows
% arg 2 = solid or dashed or loosely dashed, etc.
% arg 3 = text
% arg 4 = text width
\newcommand{\ratline}[4]{%
    \needspace{0.1\baselineskip}%
    \vbox to 1\baselineskip {\begin{tikzpicture}%
        \setlength{\sbtw}{#4}%
        \setlength{\sblen}{\linewidth}%
        \setlength{\sbht}{#1\sbsz}\setlength{\sbht}{1.4\sbht}%
        \setlength{\sbhadj}{#1\sblw}\setlength{\sbhadj}{0.25\sbhadj}%
        \filldraw[color=purple!40] (\sblen, 0) -- (\sblen - \sbsz, \sbht) -- (\sblen - 2\sbsz, 0) -- (\sblen, 0);
        \draw[line width=\sblw, color=purple!40, #2] (2\sbsz - \sblw, \sbhadj) -- (0.5\sblen - 0.5\sbtw, \sbhadj);
        \draw[line width=\sblw, color=purple!40, #2] (0.5\sblen + 0.5\sbtw, \sbhadj) -- (\sblen - 2\sbsz + \sblw, \sbhadj);
        \filldraw[color=purple!40] (0, 0) -- (\sbsz, \sbht) -- (0 + 2\sbsz, 0) -- (0, 0);
        \node[color=purple!90] at (0.5\sblen, 0) {\large  \textsf{\textup{#3}}};
    \end{tikzpicture}}}

% \linewitharrows is a helper command that makes a blue horizontal line, up or down arrows, and some text:
% arg 1 = 1 or -1 for up or down arrows
% arg 2 = solid or dashed or loosely dashed, etc.
% arg 3 = text
% arg 4 = text width
\newcommand{\linewitharrows}[4]{%
    \needspace{0.1\baselineskip}%
    \vbox to 1\baselineskip {\begin{tikzpicture}%
        \setlength{\sbtw}{#4}%
        \setlength{\sblen}{\linewidth}%
        \setlength{\sbht}{#1\sbsz}\setlength{\sbht}{1.4\sbht}%
        \setlength{\sbhadj}{#1\sblw}\setlength{\sbhadj}{0.25\sbhadj}%
        \filldraw[color=blue!40] (\sblen, 0) -- (\sblen - \sbsz, \sbht) -- (\sblen - 2\sbsz, 0) -- (\sblen, 0);
        \draw[line width=\sblw, color=blue!40, #2] (2\sbsz - \sblw, \sbhadj) -- (0.5\sblen - 0.5\sbtw, \sbhadj);
        \draw[line width=\sblw, color=blue!40, #2] (0.5\sblen + 0.5\sbtw, \sbhadj) -- (\sblen - 2\sbsz + \sblw, \sbhadj);
        \filldraw[color=blue!40] (0, 0) -- (\sbsz, \sbht) -- (0 + 2\sbsz, 0) -- (0, 0);
        \node[color=blue!90] at (0.5\sblen, 0) {\large  \textsf{\textup{#3}}};
    \end{tikzpicture}}}

\newcommand{\VSPb}{\vspace{0.5ex plus 5ex minus 0.25ex}}
\newcommand{\VSPa}{\vspace{0.25ex plus 5ex minus 0.25ex}}

% C
\newcommand{\cspecificstart}{\needspace{\sbns}\linewitharrows{-1}{solid}{C}{3em}}
\newcommand{\cspecificend}{\linewitharrows{1}{solid}{C}{3em}\VSPa}

% Fortran
\newcommand{\fortranspecificstart}{\VSPb\linewitharrows{-1}{solid}{Fortran}{6em}\VSPa}
\newcommand{\fortranspecificend}{\VSPb\linewitharrows{1}{solid}{Fortran}{6em}\VSPa}

% Python
\newcommand{\pyspecificstart}{\needspace{\sbns}\linewitharrows{-1}{solid}{Python}{6em}}
\newcommand{\pyspecificend}{\linewitharrows{1}{solid}{Python}{6em}\VSPa}

% Note
\newcommand{\notestart}{\VSPb\notelinewitharrows{-1}{solid}\VSPa}
\newcommand{\noteend}{\VSPb\notelinewitharrows{1}{solid}\VSPa}
% convenience macro for formatting the word "Note:" at the beginning of note blocks:
\newcommand{\noteheader}{{\textrm{\textsf{\textbf\textup\normalsize{{{{Note: }}}}}}}}

% Rationale
\newcommand{\rationalestart}{\VSPb\ratline{-1}{dashed}{Rationale}{7em}\VSPa}
\newcommand{\rationaleend}{\VSPb\ratline{1}{dashed}{}{0em}\VSPa}

% Advice to users
\newcommand{\adviceuserstart}{\VSPb\adviceuserline{-1}{solid}{Advice to users}{10em}\VSPa}
\newcommand{\adviceuserend}{\VSPb\adviceuserline{1}{solid}{}{0em}\VSPa}

% Advice to implementers
\newcommand{\adviceimplstart}{\VSPb\adviceimpline{-1}{solid}{Advice to PMIx library implementers}{20em}\VSPa}
\newcommand{\adviceimplend}{\VSPb\adviceimpline{1}{solid}{}{0em}\VSPa}

% Advice to hosts
\newcommand{\advicermstart}{\VSPb\advicermline{-1}{solid}{Advice to PMIx server hosts}{16em}\VSPa}
\newcommand{\advicermend}{\VSPb\advicermline{1}{solid}{}{0em}\VSPa}

% Required attributes
\newcommand{\reqattrstart}{\VSPb\adviceuserline{-1}{dashed}{Required Attributes}{16em}\VSPa}
\newcommand{\reqattrend}{\VSPb\adviceuserline{1}{dashed}{}{0em}\VSPa}

% Optional attributes
\newcommand{\optattrstart}{\VSPb\adviceimpline{-1}{dashed}{Optional Attributes}{16em}\VSPa}
\newcommand{\optattrend}{\VSPb\adviceimpline{1}{dashed}{}{0em}\VSPa}


%%%%%%%%%%%%%%%%%%%%%%%%%%%%%%%%%%%%%%%%%%%%%%%%%%%%%%%%%%%%%%%%%%%%%%%%%%%%%%%%%%%%%%%%%%%%%%
% Glossary formatting

\newcommand{\glossaryterm}[1]{\needspace{1ex}
\begin{adjustwidth}{-0.75in}{0.0in}
\nolinenumbers\parbox[b][-0.95\baselineskip][t]{1.4in}{\flushright \textbf{#1}}
\end{adjustwidth}\linenumbers}

\newcommand{\glossarydefstart}{
\begin{adjustwidth}{0.79in}{0.0in}}

\newcommand{\glossarydefend}{
\end{adjustwidth}\vspace{-1.5\baselineskip}}


%%%%%%%%%%%%%%%%%%%%%%%%%%%%%%%%%%%%%%%%%%%%%%%%%%%%%%%%%%%%%%%%%%%%%%%%%%%%%%%%%%%%%%%%%%%%%
% Indexing and Table of Contents

\usepackage{imakeidx}
\usepackage[nodotinlabels]{titletoc}   % required for its [nodotinlabels] option

% Clickable links in TOC and index:
\usepackage[hyperindex=true,linktocpage=true]{hyperref}
\hypersetup{
  bookmarksnumbered = true,
  bookmarksopen     = false,
  colorlinks  = true, % Colors links instead of red boxes
  urlcolor    = blue, % Color for external links
  linkcolor   = blue  % Color for internal links
}

% \url styled in Roman font.
\urlstyle{rm}

%%%%%%%%%%%%%%%%%%%%%%%%%%%%%%%%%%%%%%%%%%%%%%%%%%%%%%%%%%%%%%%%%%%%%%%%%%%%%%%%%%%%%%%%%%%%%
% Cross reference macros
% This defines:
%     \specref          cross reference label as "Section X on page Y"
%     \refsection       Link this label to a specific section label in the document
%
%     \declarstruct     Mark the declaration of a structure
%     \refstruct        Reference the structure declaration
%
%     \declareapi       Mark the declaration of an API function
%     \refapi           Reference the API declaration
%
%     \declaremacro     Mark the declaration of a user-level macro
%     \refmacro         Reference the macro declaration
%

\newcommand{\chapterref}[1]{Chapter~\ref{#1} on page~\pageref{#1}}
\newcommand{\specref}[1]{Section~\ref{#1} on page~\pageref{#1}}

\newcommand{\refsection}[2]{\hyperref[#1]{#2}}

\newcommand{\declarestruct}[1]{\index{#1!Definition|textbf} \label{struct:#1}}
\newcommand{\refstruct}[1]{\index{#1} \hyperref[struct:#1]{\code{#1} }}
\newcommand{\structref}[1] {\refstruct{#1}}
\newcommand{\specrefstruct}[1]{Section~\ref{struct:#1} on page~\pageref{struct:#1}}

\newcommand{\declareapi}[1]{\index{#1!Definition|textbf} \label{api:#1}}
\newcommand{\refapi}[1]{\index{#1} \hyperref[api:#1]{\code{#1} }}
\newcommand{\argapi}[1] {\refapi{#1}}

\newcommand{\refconst}[1]{\hyperref[const:#1]{\code{#1} }}

\newcommand{\declareattr}[1]{\index{#1!Definition|textbf} \label{attr:#1}}

\newcommand{\refarg}[1] {{\textrm{\textmd{\itshape{#1}}}}}
\newcommand{\argref}[1] {\refarg{#1}}

\newcommand{\declaremacro}[1]{\index{#1!Definition|textbf} \label{macro:#1}}
\newcommand{\refmacro}[1]{\index{#1} \hyperref[macro:#1]{\code{#1} }}

\newcommand{\declareterm}[1]{\index{#1!Definition|textbf} \label{macro:#1}}
\newcommand{\refterm}[1]{\index{#1} \hyperref[macro:#1]{\code{#1} }}

% Place in text for in-text questions during review
\newcommand{\rcomment}[1]{(REVIEW COMMENT: \textbf{#1})}

%%%%%%%%%%%%%%%%%%%%%%%%%%%%%%%%%%%%%%%%%%%%%%%%%%%%%%%%%%%%%%%%%%%%%%%%%%%%%%%%%%%%%%%%%%%%%
% Set default fonts:
\rmfamily\mdseries\upshape\normalsize

%%%%%%%%%%%%%%%%%%%%%%%%%%%%%%%%%%%%%%%%%%%%%%%%%
% Define a divider for splitting implementer vs host attribute requirements/options
\newcommand{\divider}{\noindent\makebox[\linewidth]{\rule{\linewidth}{0.8pt}}}


\newcounter{pycounter}
\newcommand{\pylabel}[1]{\refstepcounter{pycounter} \label{appB:#1}}
\newcommand{\refpy}[1]{\hyperref[appB:#1]{\code{#1} }}


% Watermark for the Unofficial Drafts
\ifthenelse{\boolean{is_unofficial_draft}}
  {\usepackage{draftwatermark}
   \SetWatermarkText{\textbf{Unofficial Draft}}
   \SetWatermarkColor[gray]{0.85}
   \SetWatermarkFontSize{90pt}
   \SetWatermarkScale{1}
   \SetWatermarkAngle{45}
  }{}

%%%%%%%%%%%%%%%%%%% Indices
\makeindex[intoc,columns=1,columnseprule=true,columnsep=15pt,title=Index]
\makeindex[intoc,columns=1,columnseprule=true,columnsep=15pt,name=index_api,title=Index of APIs]
\makeindex[intoc,columns=1,columnseprule=true,columnsep=15pt,name=index_macro,title=Index of Support Macros]
\makeindex[intoc,columns=1,columnseprule=true,columnsep=15pt,name=index_const,title=Index of Constants] % 1 column because of long names
\makeindex[intoc,columns=1,columnseprule=true,columnsep=15pt,name=index_attribute,title=Index of Attributes]
\makeindex[intoc,columns=1,columnseprule=true,columnsep=15pt,name=index_struct,title=Index of Data Structures]
\makeindex[intoc,columns=1,columnseprule=true,columnsep=15pt,name=index_envars,title=Index of Environmental Variables]


%%%%%%%%%%%%%%%%%%%
\usepackage{acronym}
\acrodef{PMI}[PMI]{Process Management Interface}
\acrodef{PMIx}[PMIx]{Process Management Interface - Exascale}
\acrodef{HPC}[HPC]{High Performance Computing}
\acrodef{MPI}[MPI]{Message Passing Interface}
\acrodef{MPE}[MPE]{Message Passing Environment}

\acrodef{RM}[RM]{Resource Manager}
\acrodef{RTE}[RTE]{RunTime Environment}
\acrodef{SMS}[SMS]{System Management Software stack}
\acrodef{ASC}[ASC]{Administrative Steering Committee}
\acrodef{WLM}[WLM]{WorkLoad Manager}
\acrodef{GDS}[GDS]{Global Data Storage}
\acrodef{BCX}[BCX]{Business Card Exchange}
\acrodef{DBCX}[DBCX]{Direct Business Card Exchange}

\acrodef{PID}[PID]{Process IDentifier}
\acrodef{URI}[URI]{Uniform Resource Identifier}
\acrodef{CIDR}[CIDR]{Classless Inter-Domain Routing}
\acrodef{XML}[XML]{eXtensible Markup Language}

\acrodef{RAS}[RAS]{Reliability and Survivability}
\acrodef{API}[API]{Application Programming Interface}
\acrodef{ECC}[ECC]{Error Check and Correction}
\acrodef{FM}[FM]{Fabric Manager}
\acrodef{IO}[IO]{Input/Output}
\acrodef{MPMD}{Multiple Program Multiple Data}
\acrodef{PU}{Processing Unit}
\acrodef{HWLOC}{Hardware Locality}
\acrodef{OS}{Operating System}
\acrodef{PGCID}{Process Group Context IDentifier}
\acrodef{NIC}{Network Interface Card}
\acrodef{PCI}{Peripheral Component Interconnect}
\acrodef{MAC}{Media Access Control}
\acrodef{TCP}{Transmission Control Protocol}
\acrodef{UID}{User ID}
\acrodef{GID}{Group ID}
\acrodef{IL}{Intermediate Launcher}
\acrodef{FEA}{Fork/Exec Agent}
\acrodef{FQDN}{Fully Qualified Domain Name}
\acrodef{NUMA}{Non-Uniform Memory Access}
\acrodef{UUID}{Universally Unique IDentifier}
\acrodef{GPU}{Graphics Processing Unit}
\acrodef{DRM}{Direct Rendering Manager}
\acrodef{DMA}{Direct Memory Access}
\acrodef{CUDA}{Compute Unified Device Architecture}
\acrodef{HCA}{Host Channel Adapter}
\acrodef{IP}{Internet Protocol}
\acrodef{DVM}{Distributed Virtual Machine}

%%%%%%%%%%%%%%%%%%%


\begin{document}


%
% Title page
%
    \pagenumbering{roman}
    %%%%%%%%%%%%%%%%%%%%%%%%%%%%%%%%%%%%%%%%%%%%%%%%%
% Title page
%%%%%%%%%%%%%%%%%%%%%%%%%%%%%%%%%%%%%%%%%%%%%%%%%

  \begin{titlepage}
    \begin{flushleft}
     \hspace{-6em} \includegraphics[width=0.4\textwidth]{figs/pmix-logo.png}
    \end{flushleft}

    \begin{adjustwidth}{-0.75in}{0in}
    \begin{center}
      \Huge
      \textsf{Process Management Interface\\for Exascale (PMIx) Standard}

      \vspace{1.0in}
	  \huge
      \textbf{Version \VER{}}

      \vspace{0.15in}
	  \Large
      \textbf{\VERDATE}

    \end{center}
    \end{adjustwidth}

    \vspace{1.2in}

\par
This document describes the Process Management Interface for Exascale (PMIx) Standard, version \VER{}.

\par
\textbf{Comments:}
Please provide comments on the PMIx Standard by filing issues on the document repository \url{https://github.com/pmix/pmix-standard/issues} or by sending them to the PMIx Community mailing list at \url{https://groups.google.com/forum/#!forum/pmix}.
Comments should include the version of the PMIx standard you are commenting about, and the page, section, and line numbers that you are referencing.
Please note that messages sent to the mailing list from an unsubscribed e-mail address will be ignored.

\vfill

\begin{adjustwidth}{0pt}{1em}\setlength{\parskip}{0.25\baselineskip}%
Copyright \textsuperscript{\textcopyright} 2018-2020 PMIx \acf{ASC}.\\
Permission to copy without fee all or part of this material is granted,
provided the PMIx \ac{ASC} copyright notice and
the title of this document appear, and notice is given that copying is by
permission of PMIx \ac{ASC}.
\end{adjustwidth}

  \end{titlepage}

%%%%%%%%%%%%%%%%%%%%%%%%%%%%%%%%%%%%%%%%%%%%%%%%%
% Blank page
%%%%%%%%%%%%%%%%%%%%%%%%%%%%%%%%%%%%%%%%%%%%%%%%%
\clearpage
\thispagestyle{empty}
\phantom{a}
\begin{center}
\emph{This page intentionally left blank}
\end{center}

\vfill



%
% Table of contents
%
    \setcounter{page}{0}
    \setcounter{tocdepth}{3}

    \begin{spacing}{1.3}
        \RedeclareSectionCommand[tocnumwidth=2.6em]{section}
        \RedeclareSectionCommand[tocnumwidth=3.7em,tocindent=4.1em]{subsection}
        \tableofcontents
    \end{spacing}

%
% Introductory materials
%
    % Uncomment the next line to enable line numbering on the main body text:
    \linenumbers\pagewiselinenumbers
    \newpage\pagenumbering{arabic}
    \setcounter{chapter}{0}  % start chapter numbering here

%
% Chapters
%
    % Introduction to PMIx
    %  - Overview, Goals, Arch.
    %%%%%%%%%%%%%%%%%%%%%%%%%%%%%%%%%%%%%%%%%%%%%%%%%
% Chapter: Introduction
%%%%%%%%%%%%%%%%%%%%%%%%%%%%%%%%%%%%%%%%%%%%%%%%%
\chapter{Introduction}
\label{chap:intro}

The \ac{PMI} has been used for quite some time as a means of exchanging wireup information needed for inter-process communication.
Two versions (PMI-1 and PMI-2) have been released as part of the MPICH effort, with PMI-2 demonstrating better scaling properties than its PMI-1 predecessor. However, two significant challenges face the \ac{HPC} community as it continues to move towards machines capable of exaflop and higher performance levels:

\begin{itemize}
\item the physical scale of the machines, and the corresponding number of total processes they support, is expected to reach levels approaching  1 million processes executing across 100 thousand nodes. Prior methods for initiating applications relied on exchanging communication endpoint information between the processes, either directly or in some form of hierarchical collective operation. Regardless of the specific mechanism employed, the exchange across such large applications would consume considerable time, with estimates running in excess of 5-10 minutes; and
\item whether it be hybrid applications that combine OpenMP threading operations with MPI, or application-steered workflow computations, the HPC community is experiencing an unprecedented wave of new approaches for computing at exascale levels. One common thread across the proposed methods is an increasing need for orchestration between the application and the \ac{SMS} comprising the scheduler (a.k.a. the \ac{WLM}), the \ac{RM}, global file system, fabric, and other subsystems. The lack of available support for application-to-SMS integration has forced researchers to develop "virtual" environments that hide the SMS behind a customized abstraction layer, but this results in considerable duplication of effort and a lack of portability.
\end{itemize}

\ac{PMIx} represents an attempt to resolve these questions by providing an extended version of the \ac{PMI} definitions specifically designed to support clusters up to exascale and larger sizes.
The overall objective of the project is not to branch the existing definitions -- in fact, PMIx fully supports both of the existing PMI-1 and PMI-2 APIs -- but rather to:

\begin{compactalphaenum}
\item augment those APIs to eliminate some current restrictions that impact scalability,
\item extend the breadth of the \ac{PMI} definitions to providing an abstraction layer for \ac{SMS} interactions,
\item establish a standards-like body for maintaining the definitions, and
\item provide a reference implementation of the PMIx standard that demonstrates the desired level of scalability and features.
\end{compactalphaenum}

Complete information about the \ac{PMIx} standard and affiliated projects can be found at the \ac{PMIx} web site: \url{https://pmix.org}


%%%%%%%%%%%%%%%%%%%%%%%%%%%%%%%%%%%%%%%%%%%%%%%%%
%%%%%%%%%%%%%%%%%%%%%%%%%%%%%%%%%%%%%%%%%%%%%%%%%
\section{Charter}
\label{chap:intro:charter}

The charter of the PMIx community is to:
\begin{itemize}
\item Define a set of agnostic APIs (not affiliated with any specific programming model or code base) to support interactions between application processes and the \ac{SMS}.
\item Develop an open source (non-copy-left licensed) standalone ``reference'' library to facilitate adoption of the \ac{PMIx} standard.
\item Retain transparent backward compatibility with the existing PMI-1 and PMI-2 definitions, any future \ac{PMI} releases, and across all \ac{PMIx} versions.
\item Support the ``Instant On'' initiative for rapid startup of applications at exascale and beyond.
\item Work with the \ac{HPC} community to define and implement new APIs that support evolving programming model requirements for application interactions with the \ac{SMS}.
\end{itemize}

Participation in the \ac{PMIx} community is open to anyone, and not restricted to only code contributors to the reference implementation.


%%%%%%%%%%%%%%%%%%%%%%%%%%%%%%%%%%%%%%%%%%%%%%%%%
%%%%%%%%%%%%%%%%%%%%%%%%%%%%%%%%%%%%%%%%%%%%%%%%%
\section{PMIx Standard Overview}
\label{chap:intro:std_overview}

\ldots

%%%%%%%%%%%
\subsection{Who should use the standard?}

\ldots

%%%%%%%%%%%
\subsection{What is defined in the standard?}

\ldots

%%%%%%%%%%%
\subsection{What is \emph{not} defined in the standard?}

The \ac{PMIx} Standard does not include anything, either stated or implied, regarding implementation.
It instead focuses exclusively on defining APIs and associated attribute key strings, and describing the expected behavior of those entities.
How that behavior is realized is entirely at the discretion of the implementer.

As previously noted, system environments and \ac{PMIx} library implementers are free to return ``not supported'' for any request. Thus, users should design their applications accordingly.


%%%%%%%%%%%%%%%%%%%%%%%%%%%%%%%%%%%%%%%%%%%%%%%%%
%%%%%%%%%%%%%%%%%%%%%%%%%%%%%%%%%%%%%%%%%%%%%%%%%
\section{PMIx Architecture Overview}
\label{chap:intro:arch_overview}

This section presents a brief overview the \ac{PMIx} Architecture~\cite{2017-Castain-EuroMPI}.

\ldots

%%%%%%%%%%%
\subsection{The PMIx Reference Implementation}

Note that the definition of the \ac{PMIx} Standard is not contingent upon use of the \ac{PMIx} Reference Implementation.
Any implementation that supports the defined APIs is a \ac{PMIx} Standard compliant implementation, and some environments have chosen to pursue their own custom implementation.
The \ac{PMIx} Reference Implementation is provided solely for the following purposes:
\begin{itemize}
\item Validation of the standard.\\
No proposed change and/or extension to the \ac{PMIx} standard is accepted without an accompanying prototype implementation in the \ac{PMIx} Reference Implementation.
This ensures that the proposal has undergone at least some minimal level of scrutiny and testing before being considered.
\item Ease of adoption.\\
The \ac{PMIx} Reference Implementation is designed to be particularly easy for resource managers (and the \ac{SMS} in general) to adopt, thus facilitating a rapid uptake into that community for application portability.
Both client and server \ac{PMIx} libraries are included, along with examples of client usage and server-side integration.
A list of supported environments and versions is provided on the \ac{PMIx} web site \url{www.pmix.org}
\end{itemize}

The \ac{PMIx} Reference Implementation targets support for the Linux operating system.
A reasonable effort is made to support all major, modern Linux distributions; however, validation is limited to the most recent 2-3 releases of RedHat Enterprise Linux (RHEL), Fedora, CentOS, and SUSE Linux Enterprise Server (SLES).
In addition, development support is maintained for Mac OSX.
Production support for vendor-specific operating systems is included as provided by the vendor.

%%%%%%%%%%%
\subsection{The PMIx Reference Server}

\ldots


%%%%%%%%%%%%%%%%%%%%%%%%%%%%%%%%%%%%%%%%%%%%%%%%%
\section{Organization of this document}

The remainder of this document is structured as follows:

\begin{itemize}
\item Introduction and Overview in \chapterref{chap:intro}
\item Terms and Conventions in \chapterref{chap:terms}
\item Data Structures and Types in \chapterref{chap:struct}
\item \ac{PMIx} Initialization and Finalization in \chapterref{chap:api_init}
\item Key/Value Management in \chapterref{chap:api_kv_mgmt}
\item Process Management in \chapterref{chap:api_proc_mgmt}
\item Job Management in \chapterref{chap:api_job_mgmt}
\item Event Notification in \chapterref{chap:api_event}
\item Data Packing and Unpacking in \chapterref{chap:api_data_mgmt}
\item \ac{PMIx} Server Specific Interfaces in \chapterref{chap:api_server}
\end{itemize}

%%%%%%%%%%%%%%%%%%%%%%%%%%%%%%%%%%%%%%%%%%%%%%%%%


    % PMIx Terms and Conventions
    %%%%%%%%%%%%%%%%%%%%%%%%%%%%%%%%%%%%%%%%%%%%%%%%%
% Chapter: Terms and Conventions
%%%%%%%%%%%%%%%%%%%%%%%%%%%%%%%%%%%%%%%%%%%%%%%%%
\chapter{PMIx Terms and Conventions}
\label{chap:terms}

Define ``attributes'' and how they are used, intent is to allow for definition of flexible APIs that can change behavior based on attributes instead of modifying function signature.
Include description of data types.

This document borrows freely from other standards (most notably from the \ac{MPI} and OpenMP standards) in its use of notation and conventions in an attempt to reduce confusion.

%%%%%%%%%%%
\section{Notional Conventions}

Some sections of this document describe programming language specific examples or APIs.
Text that applies only to programs for which the base language is C is show as follows:

\cspecificstart
C specific text...
\begin{codepar}
int foo = 42;
\end{codepar}
\cspecificend

Some text is for information only, and is not part of the normative specification.
These take three forms, described in their examples below:

\notestart
\noteheader
General text...
\noteend

\rationalestart
Throughout this document, the rationale for the design choices made in the interface specification is set off in this section.
Some readers may wish to skip these sections, while readers interested in interface design may want to read them carefully.
\rationaleend

\adviceuserstart
Throughout this document, material aimed at users and that illustrates usage is set off in this section.
Some readers may wish to skip these sections, while readers interested in programming in \ac{MPI} may want to read them carefully.
\adviceuserend

\adviceimplstart
Throughout this document, material that is primarily commentary to implementers is set off in this section.
Some readers may wish to skip these sections, while readers interested in \ac{PMIx} implementations may want to read them carefully. 
\adviceimplend

%%%%%%%%%%%
\section{Semantics}

The following terms will be taken to mean:

\begin{itemize}
\item \emph{shall} and \emph{will} indicate that the specified behavior is \emph{required} of all conforming implementations
\item \emph{should} and \emph{may} indicate behaviors that a quality implementation would include, but are not required of all conforming implementations
\end{itemize}

%%%%%%%%%%%
\section{Naming Conventions}

\ldots

%%%%%%%%%%%
\section{Procedure Conventions}

While current \ac{PMIx} Reference Implementation is solely based on the C programming language, it is not the intent of the \ac{PMIx} Standard to preclude the use of other languages.
Accordingly, the procedure specifications in the \ac{PMIx} Standard are written in a language-independent syntax with the arguments marked as IN, OUT, or INOUT.
The meanings of these are:
\begin{itemize}
\item IN:
The call may use the input value but does not update the argument from the perspective of the caller at any time during the call?s execution, 
\item OUT:
The call may update the argument but does not use its input value
\item INOUT:
The call may both use and update the argument. 
\end{itemize}

%%%%%%%%%%%%%%%%%%%%%%%%%%%%%%%%%%%%%%%%%%%%%%%%%


    % Data Structures, Types, Constants
    %%%%%%%%%%%%%%%%%%%%%%%%%%%%%%%%%%%%%%%%%%%%%%%%%
% Chapter: Data Structures
%%%%%%%%%%%%%%%%%%%%%%%%%%%%%%%%%%%%%%%%%%%%%%%%%
\chapter{Data Structures and Types}
\label{chap:struct}

This chapter defines \ac{PMIx} standard data structures (along with macros for convenient use), types, and constants.
These apply to all consumers of the \ac{PMIx} interface.
Where necessary for clarification, the description of, for example, an attribute may be copied from this chapter into a section where it is used.

A PMIx implementation may define additional attributes beyond those specified in this document.

\adviceimplstart
Structures, types, and macros in the \ac{PMIx} Standard are defined in terms of the C-programming language. Implementers wishing to support other languages should provide the equivalent definitions in a language-appropriate manner.

If a PMIx implementation chooses to define additional attributes they should avoid using the \code{"PMIX"} prefix in their name or starting the attribute string with a \code{"pmix"} prefix.
This helps the end user distinguish between what is defined by the PMIx standard and what is specific to that PMIx implementation, and avoids potential conflicts with attributes defined by the Standard.
\adviceimplend

\adviceuserstart
Use of increment/decrement operations on indices inside \ac{PMIx} macros is discouraged due to unpredictable behavior. For example, the following sequence:

\begin{codepar}
PMIX_INFO_LOAD(&array[n++], "mykey", &mystring, PMIX_STRING);
PMIX_INFO_LOAD(&array[n++], "mykey2", &myint, PMIX_INT);
\end{codepar}

will load the given key-values into incorrect locations if the macro is implemented as:

\begin{codepar}
define PMIX_INFO_LOAD(m, k, v, t)                      \textbackslash
  do \{                                                 \textbackslash
    if (NULL != (k)) \{                                 \textbackslash
      pmix_strncpy((m)->key, (k), PMIX_MAX_KEYLEN);    \textbackslash
    \}                                                  \textbackslash
    (m)->flags = 0;                                    \textbackslash
    pmix_value_load(&((m)->value), (v), (t));          \textbackslash
  \} while (0)
\end{codepar}

since the index is cited more than once in the macro. The \ac{PMIx} standard only governs the existence and syntax of macros - it does not specify their implementation. Given the freedom of implementation, a safer call sequence might be as follows:

\begin{codepar}
PMIX_INFO_LOAD(&array[n], "mykey", &mystring, PMIX_STRING);
++n;
PMIX_INFO_LOAD(&array[n], "mykey2", &myint, PMIX_INT);
++n;
\end{codepar}

Users are also advised to use the macros for creating, loading, and releasing
\ac{PMIx} structures to avoid potential issues with release of memory. For
example, pointing a \refstruct{pmix_envar_t} element at a static string
variable and then using \refmacro{PMIX_ENVAR_DESTRUCT} to clear it would
generate an error as the static string had not been allocated.

\adviceuserend

%%%%%%%%%%%%%%%%%%%%%%%%%%%%%%%%%%%%%%%%%%%%%%%%%
%%%%%%%%%%%%%%%%%%%%%%%%%%%%%%%%%%%%%%%%%%%%%%%%%
\section{Constants}
\label{chap:struct:const}

\ac{PMIx} defines a few values that are used throughout the standard to set the size of fixed arrays or as a means of identifying values with special meaning.
The community makes every attempt to minimize the number of such definitions.
The constants defined in this section may be used before calling any \ac{PMIx} library initialization routine.
Additional constants associated with specific data structures or types are defined in the section describing that data structure or type.

\begin{constantdesc}
%
\declareconstitem{PMIX_MAX_NSLEN}
Maximum namespace string length as an integer.
\end{constantdesc}

\adviceimplstart
\refconst{PMIX_MAX_NSLEN} should have a minimum value of 63 characters. Namespace arrays in \ac{PMIx} defined structures must reserve
a space of size \refconst{PMIX_MAX_NSLEN}+1 to allow room for the \code{NULL} terminator
\adviceimplend

\begin{constantdesc}
%
\declareconstitem{PMIX_MAX_KEYLEN}
Maximum key string length as an integer.
\end{constantdesc}

\adviceimplstart
\refconst{PMIX_MAX_KEYLEN} should have a minimum value of 63 characters. Key arrays in \ac{PMIx} defined structures must reserve
a space of size \refconst{PMIX_MAX_KEYLEN}+1 to allow room for the \code{NULL} terminator
\adviceimplend

\begin{constantdesc}
%
\declareconstitemNEW{PMIX_APP_WILDCARD}
A value to indicate that the user wants the data for the given key from every application that posted that key, or that the given value applies to all applications within the given namespace.
\end{constantdesc}


%%%%%%%%%%%%%%%%%%%%%%%%%%%%%%%%%%%%%%%%%%%%%%%%%
\subsection{PMIx Return Status Constants}
\label{api:struct:errors}
\declarestruct{pmix_status_t}

The \refstruct{pmix_status_t} structure is an \code{int} type for return status. The tables shown in this section define the possible values for \refstruct{pmix_status_t}.
PMIx errors are required to always be negative, with \code{0} reserved for \refconst{PMIX_SUCCESS}. Values in the list that were deprecated in later standards are denoted as such. Values added to the list in this version of the standard are shown in \textbf{\color{magenta}magenta}.

\adviceimplstart
A PMIx implementation must define all of the constants defined in this section, even if they will never return the specific value to the caller.
\adviceimplend

\adviceuserstart
Other than \refconst{PMIX_SUCCESS} (which is required to be zero), the actual value of any \ac{PMIx} error constant is left to the \ac{PMIx} library implementer. Thus, users are advised to always refer to constant by name, and not a specific implementation's value, for portability between implementations and compatibility across library versions.
\adviceuserend

The following values are general constants used in a variety of places.

\begin{constantdesc}
%
\declareconstitem{PMIX_SUCCESS}
Success.
%
\declareconstitem{PMIX_ERROR}
General Error.
%
\declareconstitemNEW{PMIX_ERR_EXISTS}
Requested operation would overwrite an existing value - typically returned
when an operation would overwrite an existing file or directory.
%
\declareconstitemNEW{PMIX_ERR_EXISTS_OUTSIDE_SCOPE}
The requested key exists, but was posted in a \emph{scope} (see Section \ref{api:nres:scope}) that does not include the requester
%
\declareconstitem{PMIX_ERR_INVALID_CRED}
Invalid security credentials.
%
\declareconstitem{PMIX_ERR_WOULD_BLOCK}
Operation would block.
%
\declareconstitem{PMIX_ERR_UNKNOWN_DATA_TYPE}
The data type specified in an input to the \ac{PMIx} library is not recognized
by the implementation.
%
\declareconstitem{PMIX_ERR_TYPE_MISMATCH}
The data type found in an object does not match the expected data type
as specified in the \ac{API} call - e.g., a request to unpack a
\refconst{PMIX_BOOL} value from a buffer that does not contain a value of
that type in the current unpack location.
%
\declareconstitem{PMIX_ERR_UNPACK_INADEQUATE_SPACE}
Inadequate space to unpack data - the number of values in the buffer exceeds
the specified number to unpack.
%
\declareconstitem{PMIX_ERR_UNPACK_READ_PAST_END_OF_BUFFER}
Unpacking past the end of the provided buffer - the number of values in the
buffer is less than the specified number to unpack, or a request was made to
unpack a buffer beyond the buffer's end.
%
\declareconstitem{PMIX_ERR_UNPACK_FAILURE}
The unpack operation failed for an unspecified reason.
%
\declareconstitem{PMIX_ERR_PACK_FAILURE}
The pack operation failed for an unspecified reason.
%
\declareconstitem{PMIX_ERR_NO_PERMISSIONS}
The user lacks permissions to execute the specified operation.
%
\declareconstitem{PMIX_ERR_TIMEOUT}
Either a user-specified or system-internal timeout expired.
%
\declareconstitem{PMIX_ERR_UNREACH}
The specified target server or client process is not reachable - i.e., a
suitable connection either has not been or can not be made.
%
\declareconstitem{PMIX_ERR_BAD_PARAM}
One or more incorrect parameters (e.g., passing an attribute with a value of the wrong type), or multiple parameters containing conflicting directives (e.g., multiple instances of the same attribute with different values, or different attributes specifying conflicting behaviors), were passed to a \ac{PMIx} \ac{API}.
%
\declareconstitemNEW{PMIX_ERR_EMPTY}
An array or list was given that has no members in it - i.e., the object is empty.
%
\declareconstitem{PMIX_ERR_RESOURCE_BUSY}
Resource busy - typically seen when an attempt to establish a connection
to another process (e.g., a \ac{PMIx} server) cannot be made due to a
communication failure.
%
\declareconstitem{PMIX_ERR_OUT_OF_RESOURCE}
Resource exhausted.
%
\declareconstitem{PMIX_ERR_INIT}
Error during initialization.
%
\declareconstitem{PMIX_ERR_NOMEM}
Out of memory.
%
\declareconstitem{PMIX_ERR_NOT_FOUND}
The requested information was not found.
%
\declareconstitem{PMIX_ERR_NOT_SUPPORTED}
The requested operation is not supported by either the \ac{PMIx} implementation
or the host environment.
%
\declareconstitemNEW{PMIX_ERR_PARAM_VALUE_NOT_SUPPORTED}
The requested operation is supported by the \ac{PMIx} implementation and (if applicable) the host environment. However, at least one supplied parameter was given an unsupported value, and the operation cannot therefore be executed as requested.
%
\declareconstitem{PMIX_ERR_COMM_FAILURE}
Communication failure - a message failed to be sent or received, but the
connection remains intact.
%
\declareconstitemNEW{PMIX_ERR_LOST_CONNECTION}
Lost connection between server and client or tool.
%
\declareconstitem{PMIX_ERR_INVALID_OPERATION}
The requested operation is supported by the implementation and host environment, but fails to meet a requirement (e.g., requesting to \textit{disconnect} from processes without first \textit{connecting} to them, inclusion of conflicting directives, or a request to perform an operation that conflicts with an ongoing one).
%
\declareconstitem{PMIX_OPERATION_IN_PROGRESS}
A requested operation is already in progress - the duplicate request
shall therefore be ignored.
%
\declareconstitem{PMIX_OPERATION_SUCCEEDED}
The requested operation was performed atomically - no callback function will be executed.
%
\declareconstitemNEW{PMIX_ERR_PARTIAL_SUCCESS}
The operation is considered successful but not all elements of the operation were concluded (e.g., some members of a group construct operation chose not to participate).
%
\end{constantdesc}


%%%%%%%%%%%%%%%%%%%%%%%%%%%%%%%%%%%%%%%%%%%%%%%%%
\subsubsection{User-Defined Error and Event Constants}
\label{api:struct:usererrors}

\ac{PMIx} establishes a boundary for constants defined in the \ac{PMIx} standard. Negative values larger (i.e., more negative) than this (and any positive values greater than zero) are guaranteed not to conflict with \ac{PMIx} values.

\begin{constantdesc}
%
\declareconstitem{PMIX_EXTERNAL_ERR_BASE}
A starting point for user-level defined error and event constants.
Negative values that are more negative than the defined constant are guaranteed not to conflict with \ac{PMIx} values.
Definitions should always be based on the \refconst{PMIX_EXTERNAL_ERR_BASE} constant and not a specific value as the value of the constant may change.
%
\end{constantdesc}



%%%%%%%%%%%%%%%%%%%%%%%%%%%%%%%%%%%%%%%%%%%%%%%%%
%%%%%%%%%%%%%%%%%%%%%%%%%%%%%%%%%%%%%%%%%%%%%%%%%
\section{Data Types}

This section defines various data types used by the \ac{PMIx} APIs. The version of the standard in which a particular data type was introduced is shown in the margin.

%%%%%%%%%%%%%%%%%%%%%%%%%%%%%%%%%%%%%%%%%%%%%%%%%
\subsection{Key Structure}
\declarestruct{pmix_key_t}

The \refstruct{pmix_key_t} structure is a statically defined character array of length \refconst{PMIX_MAX_KEYLEN}+1, thus supporting keys of maximum length \refconst{PMIX_MAX_KEYLEN} while preserving space for a mandatory \code{NULL} terminator.

\versionMarker{2.0}
\cspecificstart
\begin{codepar}
typedef char pmix_key_t[PMIX_MAX_KEYLEN+1];
\end{codepar}
\cspecificend

Characters in the key must be standard alphanumeric values supported by common utilities such as \textit{strcmp}.

\adviceuserstart
References to keys in \ac{PMIx} v1 were defined simply as an array of characters of size \code{PMIX_MAX_KEYLEN+1}. The \refstruct{pmix_key_t} type definition was introduced in version 2 of the standard. The two definitions are code-compatible and thus do not represent a break in backward compatibility.

Passing a \refstruct{pmix_key_t} value to the standard \textit{sizeof} utility can result in compiler warnings of incorrect returned value. Users are advised to avoid using \textit{sizeof(pmix_key_t)} and instead rely on the \refconst{PMIX_MAX_KEYLEN} constant.
\adviceuserend

%%%%%%%%%%%%%%%%%%%%%%%%%%%%%%%%%%%%%%%%%%%%%%%%%
\subsubsection{Key support macros}

The following macros are provided for convenience when working with \ac{PMIx} keys.

\littleheader{Check key macro}
\declaremacro{PMIX_CHECK_KEY}

Compare the key in a \refstruct{pmix_info_t} to a given value.

\versionMarker{3.0}
\cspecificstart
\begin{codepar}
PMIX_CHECK_KEY(a, b)
\end{codepar}
\cspecificend

\begin{arglist}
\argin{a}{Pointer to the structure whose key is to be checked (pointer to \refstruct{pmix_info_t})}
\argin{b}{String value to be compared against (\code{char*})}
\end{arglist}

Returns \code{true} if the key matches the given value

\littleheader{Check reserved key macro}
\declaremacro{PMIX_CHECK_RESERVED_KEY}

Check if the given key is a \ac{PMIx} \emph{reserved} key as described in Chapter \ref{chap:api_rsvd_keys}.

\versionMarker{4.0}
\cspecificstart
\begin{codepar}
PMIX_CHECK_RESERVED_KEY(a)
\end{codepar}
\cspecificend

\begin{arglist}
\argin{a}{String value to be checked (\code{char*})}
\end{arglist}

Returns \code{true} if the key is reserved by the Standard.

\littleheader{Load key macro}
\declaremacro{PMIX_LOAD_KEY}

Load a key into a \refstruct{pmix_info_t}.

\versionMarker{4.0}
\cspecificstart
\begin{codepar}
PMIX_LOAD_KEY(a, b)
\end{codepar}
\cspecificend

\begin{arglist}
\argin{a}{Pointer to the structure whose key is to be loaded (pointer to \refstruct{pmix_info_t})}
\argin{b}{String value to be loaded (\code{char*})}
\end{arglist}

No return value.

%%%%%%%%%%%%%%%%%%%%%%%%%%%%%%%%%%%%%%%%%%%%%%%%%
\subsection{Namespace Structure}
\declarestruct{pmix_nspace_t}

The \refstruct{pmix_nspace_t} structure is a statically defined character array of length \refconst{PMIX_MAX_NSLEN}+1, thus supporting namespaces of maximum length \refconst{PMIX_MAX_NSLEN} while preserving space for a mandatory \code{NULL} terminator.

\versionMarker{2.0}
\cspecificstart
\begin{codepar}
typedef char pmix_nspace_t[PMIX_MAX_NSLEN+1];
\end{codepar}
\cspecificend

Characters in the namespace must be standard alphanumeric values supported by common utilities such as \textit{strcmp}.

\adviceuserstart
References to namespace values in \ac{PMIx} v1 were defined simply as an array of characters of size \code{PMIX_MAX_NSLEN+1}. The \refstruct{pmix_nspace_t} type definition was introduced in version 2 of the standard. The two definitions are code-compatible and thus do not represent a break in backward compatibility.

Passing a \refstruct{pmix_nspace_t} value to the standard \textit{sizeof} utility can result in compiler warnings of incorrect returned value. Users are advised to avoid using \textit{sizeof(pmix_nspace_t)} and instead rely on the \refconst{PMIX_MAX_NSLEN} constant.
\adviceuserend

%%%%%%%%%%%%%%%%%%%%%%%%%%%%%%%%%%%%%%%%%%%%%%%%%
\subsubsection{Namespace support macros}

The following macros are provided for convenience when working with \ac{PMIx} namespace structures.

\littleheader{Check namespace macro}
\declaremacro{PMIX_CHECK_NSPACE}

Compare the string in a \refstruct{pmix_nspace_t} to a given value.

\versionMarker{3.0}
\cspecificstart
\begin{codepar}
PMIX_CHECK_NSPACE(a, b)
\end{codepar}
\cspecificend

\begin{arglist}
\argin{a}{Pointer to the structure whose value is to be checked (pointer to \refstruct{pmix_nspace_t})}
\argin{b}{String value to be compared against (\code{char*})}
\end{arglist}

Returns \code{true} if the namespace matches the given value

\littleheader{Check invalid namespace macro}
\declaremacro{PMIX_NSPACE_INVALID}

Check the string in a \refstruct{pmix_nspace_t}

\versionMarker{4.1}
\cspecificstart
\begin{codepar}
PMIX_NSPACE_INVALID(a)
\end{codepar}
\cspecificend

\begin{arglist}
\argin{a}{Pointer to the structure whose value is to be checked (pointer to \refstruct{pmix_nspace_t})}
\end{arglist}

Returns \code{true} if the namespace is invalid (i.e., starts with a \code{NULL} resulting in a zero-length string value)

\littleheader{Load namespace macro}
\declaremacro{PMIX_LOAD_NSPACE}

Load a namespace into a \refstruct{pmix_nspace_t}.

\versionMarker{4.0}
\cspecificstart
\begin{codepar}
PMIX_LOAD_NSPACE(a, b)
\end{codepar}
\cspecificend

\begin{arglist}
\argin{a}{Pointer to the target structure (pointer to \refstruct{pmix_nspace_t})}
\argin{b}{String value to be loaded - if \code{NULL} is given, then the target structure will be initialized to zero's (\code{char*})}
\end{arglist}

No return value.


%%%%%%%%%%%%%%%%%%%%%%%%%%%%%%%%%%%%%%%%%%%%%%%%%
\subsection{Rank Structure}
\declarestruct{pmix_rank_t}

The \refstruct{pmix_rank_t} structure is a \code{uint32_t} type for rank values.

\versionMarker{1.0}
\cspecificstart
\begin{codepar}
typedef uint32_t pmix_rank_t;
\end{codepar}
\cspecificend

The following constants can be used to set a variable of the type \refstruct{pmix_rank_t}. All definitions were introduced in version 1 of the standard unless otherwise marked. Valid rank values start at zero.

\begin{constantdesc}
%
\declareconstitem{PMIX_RANK_UNDEF}
A value to request job-level data where the information itself is not associated with any specific rank, or when passing a \refstruct{pmix_proc_t} identifier to an operation that only references the namespace field of that structure.
%
\declareconstitem{PMIX_RANK_WILDCARD}
A value to indicate that the user wants the data for the given key from every rank that posted that key.
%
\declareconstitem{PMIX_RANK_LOCAL_NODE}
Special rank value used to define groups of ranks.
This constant defines the group of all ranks on a local node.
%
\declareconstitem{PMIX_RANK_LOCAL_PEERS}
Special rank value used to define groups of ranks.
This constant defines the group of all ranks on a local node within the same namespace as the current process.
%
\declareconstitem{PMIX_RANK_INVALID}
An invalid rank value.
%
\declareconstitem{PMIX_RANK_VALID}
Define an upper boundary for valid rank values.
%
\end{constantdesc}


%%%%%%%%%%%%%%%%%%%%%%%%%%%%%%%%%%%%%%%%%%%%%%%%%
\subsubsection{Rank support macros}

The following macros are provided for convenience when working with \ac{PMIx} ranks.

\littleheader{Check rank macro}
\declaremacro{PMIX_CHECK_RANK}

Check two ranks for equality, taking into account wildcard values

\versionMarker{4.0}
\cspecificstart
\begin{codepar}
PMIX_CHECK_RANK(a, b)
\end{codepar}
\cspecificend

\begin{arglist}
\argin{a}{Rank to be checked (\refstruct{pmix_rank_t})}
\argin{b}{Rank to be checked (\refstruct{pmix_rank_t})}
\end{arglist}

Returns \code{true} if the ranks are equal, or at least one of the ranks is \refconst{PMIX_RANK_WILDCARD}

\littleheader{Check rank is valid macro}
\declaremacro{PMIX_RANK_IS_VALID}

Check is the given rank is a valid value

\versionMarker{4.1}
\cspecificstart
\begin{codepar}
PMIX_RANK_IS_VALID(a)
\end{codepar}
\cspecificend

\begin{arglist}
\argin{a}{Rank to be checked (\refstruct{pmix_rank_t})}
\end{arglist}

Returns \code{true} if the given rank is valid (i.e., less than \refconst{PMIX_RANK_VALID})

%%%%%%%%%%%%%%%%%%%%%%%%%%%%%%%%%%%%%%%%%%%%%%%%%
\subsection{Process Structure}
\declarestruct{pmix_proc_t}

The \refstruct{pmix_proc_t} structure is used to identify a single process in the PMIx universe.
It contains a reference to the namespace and the \refstruct{pmix_rank_t} within that namespace.

\versionMarker{1.0}
\cspecificstart
\begin{codepar}
typedef struct pmix_proc \{
    pmix_nspace_t nspace;
    pmix_rank_t rank;
\} pmix_proc_t;
\end{codepar}
\cspecificend

%%%%%%%%%%%%%%%%%%%%%%%%%%%%%%%%%%%%%%%%%%%%%%%%%
\subsubsection{Process structure support macros}
The following macros are provided to support the \refstruct{pmix_proc_t} structure.

\littleheader{Initialize the proc structure}
\declaremacro{PMIX_PROC_CONSTRUCT}

Initialize the \refstruct{pmix_proc_t} fields.

\versionMarker{1.0}
\cspecificstart
\begin{codepar}
PMIX_PROC_CONSTRUCT(m)
\end{codepar}
\cspecificend

\begin{arglist}
\argin{m}{Pointer to the structure to be initialized (pointer to \refstruct{pmix_proc_t})}
\end{arglist}

\littleheader{Destruct the proc structure}
\declaremacro{PMIX_PROC_DESTRUCT}

Destruct the \refstruct{pmix_proc_t} fields.

\cspecificstart
\begin{codepar}
PMIX_PROC_DESTRUCT(m)
\end{codepar}
\cspecificend

\begin{arglist}
\argin{m}{Pointer to the structure to be destructed (pointer to \refstruct{pmix_proc_t})}
\end{arglist}

There is nothing to release here as the fields in \refstruct{pmix_proc_t} are either a statically-declared array (the namespace) or a single value (the rank). However, the macro is provided for symmetry in the code and for future-proofing should some allocated field be included some day.

\littleheader{Create a proc array}
\declaremacro{PMIX_PROC_CREATE}

Allocate and initialize an array of \refstruct{pmix_proc_t} structures.

\versionMarker{1.0}
\cspecificstart
\begin{codepar}
PMIX_PROC_CREATE(m, n)
\end{codepar}
\cspecificend

\begin{arglist}
\arginout{m}{Address where the pointer to the array of \refstruct{pmix_proc_t} structures shall be stored (handle)}
\argin{n}{Number of structures to be allocated (\code{size_t})}
\end{arglist}


\littleheader{Free a proc structure}
\declaremacro{PMIX_PROC_RELEASE}

Release a \refstruct{pmix_proc_t} structure.

\versionMarker{4.0}
\cspecificstart
\begin{codepar}
PMIX_PROC_RELEASE(m)
\end{codepar}
\cspecificend

\begin{arglist}
\argin{m}{Pointer to a \refstruct{pmix_proc_t} structure (handle)}
\end{arglist}

\littleheader{Free a proc array}
\declaremacro{PMIX_PROC_FREE}

Release an array of \refstruct{pmix_proc_t} structures.

\versionMarker{1.0}
\cspecificstart
\begin{codepar}
PMIX_PROC_FREE(m, n)
\end{codepar}
\cspecificend

\begin{arglist}
\argin{m}{Pointer to the array of \refstruct{pmix_proc_t} structures (handle)}
\argin{n}{Number of structures in the array (\code{size_t})}
\end{arglist}

\littleheader{Load a proc structure}
\declaremacro{PMIX_PROC_LOAD}

Load values into a \refstruct{pmix_proc_t}.

\versionMarker{2.0}
\cspecificstart
\begin{codepar}
PMIX_PROC_LOAD(m, n, r)
\end{codepar}
\cspecificend

\begin{arglist}
\argin{m}{Pointer to the structure to be loaded (pointer to \refstruct{pmix_proc_t})}
\argin{n}{Namespace to be loaded (\refstruct{pmix_nspace_t})}
\argin{r}{Rank to be assigned (\refstruct{pmix_rank_t})}
\end{arglist}

No return value. Deprecated in favor of \refmacro{PMIX_LOAD_PROCID}

\littleheader{Compare identifiers}
\declaremacro{PMIX_CHECK_PROCID}

Compare two \refstruct{pmix_proc_t} identifiers.

\versionMarker{3.0}
\cspecificstart
\begin{codepar}
PMIX_CHECK_PROCID(a, b)
\end{codepar}
\cspecificend

\begin{arglist}
\argin{a}{Pointer to a structure whose ID is to be compared (pointer to \refstruct{pmix_proc_t})}
\argin{b}{Pointer to a structure whose ID is to be compared (pointer to \refstruct{pmix_proc_t})}
\end{arglist}

Returns \code{true} if the two structures contain matching namespaces and:

\begin{itemize}
    \item the ranks are the same value
    \item one of the ranks is \refconst{PMIX_RANK_WILDCARD}
\end{itemize}

\littleheader{Check if a process identifier is valid}
\declaremacro{PMIX_PROCID_INVALID}

Check for invalid namespace or rank value

\versionMarker{4.1}
\cspecificstart
\begin{codepar}
PMIX_PROCID_INVALID(a)
\end{codepar}
\cspecificend

\begin{arglist}
\argin{a}{Pointer to a structure whose ID is to be checked (pointer to \refstruct{pmix_proc_t})}
\end{arglist}

Returns \code{true} if the process identifier contains either an empty (i.e., invalid) \refarg{nspace} field or a \refarg{rank} field of \refconst{PMIX_RANK_INVALID}

\littleheader{Load a procID structure}
\declaremacro{PMIX_LOAD_PROCID}

Load values into a \refstruct{pmix_proc_t}.

\versionMarker{4.0}
\cspecificstart
\begin{codepar}
PMIX_LOAD_PROCID(m, n, r)
\end{codepar}
\cspecificend

\begin{arglist}
\argin{m}{Pointer to the structure to be loaded (pointer to \refstruct{pmix_proc_t})}
\argin{n}{Namespace to be loaded (\refstruct{pmix_nspace_t})}
\argin{r}{Rank to be assigned (\refstruct{pmix_rank_t})}
\end{arglist}

\littleheader{Transfer a procID structure}
\declaremacro{PMIX_XFER_PROCID}

Transfer contents of one \refstruct{pmix_proc_t} value to another \refstruct{pmix_proc_t}.

\versionMarker{4.1}
\cspecificstart
\begin{codepar}
PMIX_XFER_PROCID(m, n)
\end{codepar}
\cspecificend

\begin{arglist}
\argin{m}{Pointer to the target structure (pointer to \refstruct{pmix_proc_t})}
\argin{n}{Pointer to the source structure (pointer to \refstruct{pmix_proc_t})}
\end{arglist}

\littleheader{Construct a multi-cluster namespace}
\declaremacro{PMIX_MULTICLUSTER_NSPACE_CONSTRUCT}

Construct a multi-cluster identifier containing a cluster ID and a namespace.

\versionMarker{4.0}
\cspecificstart
\begin{codepar}
PMIX_MULTICLUSTER_NSPACE_CONSTRUCT(m, n, r)
\end{codepar}
\cspecificend

\begin{arglist}
\argin{m}{\refstruct{pmix_nspace_t} structure that will contain the multi-cluster identifier (\refstruct{pmix_nspace_t})}
\argin{n}{Cluster identifier (\code{char*})}
\argin{n}{Namespace to be loaded (\refstruct{pmix_nspace_t})}
\end{arglist}

Combined length of the cluster identifier and namespace must be less than \refconst{PMIX_MAX_NSLEN}-2.

\littleheader{Parse a multi-cluster namespace}
\declaremacro{PMIX_MULTICLUSTER_NSPACE_PARSE}

Parse a multi-cluster identifier into its cluster ID and namespace parts.

\versionMarker{4.0}
\cspecificstart
\begin{codepar}
PMIX_MULTICLUSTER_NSPACE_PARSE(m, n, r)
\end{codepar}
\cspecificend

\begin{arglist}
\argin{m}{\refstruct{pmix_nspace_t} structure containing the multi-cluster identifier (pointer to \refstruct{pmix_nspace_t})}
\argin{n}{Location where the cluster ID is to be stored (\refstruct{pmix_nspace_t})}
\argin{n}{Location where the namespace is to be stored (\refstruct{pmix_nspace_t})}
\end{arglist}


%%%%%%%%%%%%%%%%%%%%%%%%%%%%%%%%%%%%%%%%%%%%%%%%%
\subsection{Process State Structure}
\label{api:struct:processstate}
\declarestruct{pmix_proc_state_t}

\versionMarker{2.0}
The \refstruct{pmix_proc_state_t} structure is a \code{uint8_t} type for process state values. The following constants can be used to set a variable of the type \refstruct{pmix_proc_state_t}.

\adviceuserstart
The fine-grained nature of the following constants may exceed the ability of an \ac{RM} to provide updated process state values during the process lifetime. This is particularly true of states for short-lived processes.
\adviceuserend

\begin{constantdesc}
%
\declareconstitem{PMIX_PROC_STATE_UNDEF}
Undefined process state.
%
\declareconstitem{PMIX_PROC_STATE_PREPPED}
Process is ready to be launched.
%
\declareconstitem{PMIX_PROC_STATE_LAUNCH_UNDERWAY}
Process launch is underway.
%
\declareconstitem{PMIX_PROC_STATE_RESTART}
Process is ready for restart.
%
\declareconstitem{PMIX_PROC_STATE_TERMINATE}
Process is marked for termination.
%
\declareconstitem{PMIX_PROC_STATE_RUNNING}
Process has been locally \code{fork}'ed by the \ac{RM}.
%
\declareconstitem{PMIX_PROC_STATE_CONNECTED}
Process has connected to PMIx server.
%
\declareconstitem{PMIX_PROC_STATE_UNTERMINATED}
Define a ``boundary'' between the terminated states and \refconst{PMIX_PROC_STATE_CONNECTED} so users can easily and quickly determine if a process is still running or not.
Any value less than this constant means that the process has not terminated.
%
\declareconstitem{PMIX_PROC_STATE_TERMINATED}
Process has terminated and is no longer running.
%
\declareconstitem{PMIX_PROC_STATE_ERROR}
Define a boundary so users can easily and quickly determine if a process abnormally terminated.
Any value above this constant means that the process has terminated abnormally.
%
\declareconstitem{PMIX_PROC_STATE_KILLED_BY_CMD}
Process was killed by a command.
%
\declareconstitem{PMIX_PROC_STATE_ABORTED}
Process was aborted by a call to \refapi{PMIx_Abort}.
%
\declareconstitem{PMIX_PROC_STATE_FAILED_TO_START}
Process failed to start.
%
\declareconstitem{PMIX_PROC_STATE_ABORTED_BY_SIG}
Process aborted by a signal.
%
\declareconstitem{PMIX_PROC_STATE_TERM_WO_SYNC}
Process exited without calling \refapi{PMIx_Finalize}.
%
\declareconstitem{PMIX_PROC_STATE_COMM_FAILED}
Process communication has failed.
%
\declareconstitemNEW{PMIX_PROC_STATE_SENSOR_BOUND_EXCEEDED}
Process exceeded a specified sensor limit.
%
\declareconstitem{PMIX_PROC_STATE_CALLED_ABORT}
Process called \refapi{PMIx_Abort}.
%
\declareconstitemNEW{PMIX_PROC_STATE_HEARTBEAT_FAILED}
Frocess failed to send heartbeat within specified time limit.
%
\declareconstitem{PMIX_PROC_STATE_MIGRATING}
Process failed and is waiting for resources before restarting.
%
\declareconstitem{PMIX_PROC_STATE_CANNOT_RESTART}
Process failed and cannot be restarted.
%
\declareconstitem{PMIX_PROC_STATE_TERM_NON_ZERO}
Process exited with a non-zero status.
%
\declareconstitem{PMIX_PROC_STATE_FAILED_TO_LAUNCH}
Unable to launch process.
%
\end{constantdesc}


%%%%%%%%%%%%%%%%%%%%%%%%%%%%%%%%%%%%%%%%%%%%%%%%%
\subsection{Process Information Structure}
\declarestruct{pmix_proc_info_t}

The \refstruct{pmix_proc_info_t} structure defines a set of information about a specific process including it's name, location, and state.

\versionMarker{2.0}
\cspecificstart
\begin{codepar}
typedef struct pmix_proc_info \{
    /** Process structure */
    pmix_proc_t proc;
    /** Hostname where process resides */
    char *hostname;
    /** Name of the executable */
    char *executable_name;
    /** Process ID on the host */
    pid_t pid;
    /** Exit code of the process. Default: 0 */
    int exit_code;
    /** Current state of the process */
    pmix_proc_state_t state;
\} pmix_proc_info_t;
\end{codepar}
\cspecificend


%%%%%%%%%%%%%%%%%%%%%%%%%%%%%%%%%%%%%%%%%%%%%%%%%
\subsubsection{Process information structure support macros}

The following macros are provided to support the \refstruct{pmix_proc_info_t} structure.

%%%%
\littleheader{Initialize the process information structure}
\declaremacro{PMIX_PROC_INFO_CONSTRUCT}

Initialize the \refstruct{pmix_proc_info_t} fields.

\versionMarker{2.0}
\cspecificstart
\begin{codepar}
PMIX_PROC_INFO_CONSTRUCT(m)
\end{codepar}
\cspecificend

\begin{arglist}
\argin{m}{Pointer to the structure to be initialized (pointer to \refstruct{pmix_proc_info_t})}
\end{arglist}

%%%%
\littleheader{Destruct the process information structure}
\declaremacro{PMIX_PROC_INFO_DESTRUCT}

Destruct the \refstruct{pmix_proc_info_t} fields.

\versionMarker{2.0}
\cspecificstart
\begin{codepar}
PMIX_PROC_INFO_DESTRUCT(m)
\end{codepar}
\cspecificend

\begin{arglist}
\argin{m}{Pointer to the structure to be destructed (pointer to \refstruct{pmix_proc_info_t})}
\end{arglist}

%%%%
\littleheader{Create a process information array}
\declaremacro{PMIX_PROC_INFO_CREATE}

Allocate and initialize a \refstruct{pmix_proc_info_t} array.

\versionMarker{2.0}
\cspecificstart
\begin{codepar}
PMIX_PROC_INFO_CREATE(m, n)
\end{codepar}
\cspecificend

\begin{arglist}
\arginout{m}{Address where the pointer to the array of \refstruct{pmix_proc_info_t} structures shall be stored (handle)}
\argin{n}{Number of structures to be allocated (\code{size_t})}
\end{arglist}

%%%%
\littleheader{Free a process information structure}
\declaremacro{PMIX_PROC_INFO_RELEASE}

Release a \refstruct{pmix_proc_info_t} structure.

\versionMarker{2.0}
\cspecificstart
\begin{codepar}
PMIX_PROC_INFO_RELEASE(m)
\end{codepar}
\cspecificend

\begin{arglist}
\argin{m}{Pointer to a \refstruct{pmix_proc_info_t} structure (handle)}
\end{arglist}

%%%%
\littleheader{Free a process information array}
\declaremacro{PMIX_PROC_INFO_FREE}

Release an array of \refstruct{pmix_proc_info_t} structures.

\versionMarker{2.0}
\cspecificstart
\begin{codepar}
PMIX_PROC_INFO_FREE(m, n)
\end{codepar}
\cspecificend

\begin{arglist}
\argin{m}{Pointer to the array of \refstruct{pmix_proc_info_t} structures (handle)}
\argin{n}{Number of structures in the array (\code{size_t})}
\end{arglist}


%%%%%%%%%%%%%%%%%%%%%%%%%%%%%%%%%%%%%%%%%%%%%%%%%
\subsection{Job State Structure}
\label{api:struct:jobstate}
\declarestruct{pmix_job_state_t}

\versionMarker{4.0}
The \refstruct{pmix_job_state_t} structure is a \code{uint8_t} type for job state values. The following constants can be used to set a variable of the type \refstruct{pmix_job_state_t}.

\adviceuserstart
The fine-grained nature of the following constants may exceed the ability of an \ac{RM} to provide updated job state values during the job lifetime. This is particularly true for short-lived jobs.
\adviceuserend

\begin{constantdesc}
%
\declareconstitemNEW{PMIX_JOB_STATE_UNDEF}
Undefined job state.
%
\declareconstitemNEW{PMIX_JOB_STATE_AWAITING_ALLOC}
Job is waiting for resources to be allocated to it.
%
\declareconstitemNEW{PMIX_JOB_STATE_LAUNCH_UNDERWAY}
Job launch is underway.
%
\declareconstitemNEW{PMIX_JOB_STATE_RUNNING}
All processes in the job have been spawned and are executing.
%
\declareconstitemNEW{PMIX_JOB_STATE_SUSPENDED}
All processes in the job have been suspended.
%
\declareconstitemNEW{PMIX_JOB_STATE_CONNECTED}
All processes in the job have connected to their \ac{PMIx} server.
%
\declareconstitemNEW{PMIX_JOB_STATE_UNTERMINATED}
Define a ``boundary'' between the terminated states and \refconst{PMIX_JOB_STATE_TERMINATED} so users can easily and quickly determine if a job is still running or not.
Any value less than this constant means that the job has not terminated.
%
\declareconstitemNEW{PMIX_JOB_STATE_TERMINATED}
All processes in the job have terminated and are no longer running - typically will be accompanied by the job exit status in response to a query.
%
\declareconstitemNEW{PMIX_JOB_STATE_TERMINATED_WITH_ERROR}
Define a boundary so users can easily and quickly determine if a job abnormally terminated - typically will be accompanied by a job-related error code in response to a query
Any value above this constant means that the job terminated abnormally.
%
\end{constantdesc}


%%%%%%%%%%%%%%%%%%%%%%%%%%%%%%%%%%%%%%%%%%%%%%%%%
\subsection{Value Structure}
\declarestruct{pmix_value_t}

The \refstruct{pmix_value_t} structure is used to represent the value passed to \refapi{PMIx_Put} and retrieved by \refapi{PMIx_Get}, as well as many of the other \ac{PMIx} functions.

A collection of values may be specified under a single key by passing a \refstruct{pmix_value_t} containing an array of type \refstruct{pmix_data_array_t}, with each array element containing its own object. All members shown below were introduced in version 1 of the standard unless otherwise marked.

\versionMarker{1.0}
\cspecificstart
\begin{codepar}
typedef struct pmix_value \{
    pmix_data_type_t type;
    union \{
        bool flag;
        uint8_t byte;
        char *string;
        size_t size;
        pid_t pid;
        int integer;
        int8_t int8;
        int16_t int16;
        int32_t int32;
        int64_t int64;
        unsigned int uint;
        uint8_t uint8;
        uint16_t uint16;
        uint32_t uint32;
        uint64_t uint64;
        float fval;
        double dval;
        struct timeval tv;
        time_t time;                    // version 2.0
        pmix_status_t status;           // version 2.0
        pmix_rank_t rank;               // version 2.0
        pmix_proc_t *proc;              // version 2.0
        pmix_byte_object_t bo;
        pmix_persistence_t persist;     // version 2.0
        pmix_scope_t scope;             // version 2.0
        pmix_data_range_t range;        // version 2.0
        pmix_proc_state_t state;        // version 2.0
        pmix_proc_info_t *pinfo;        // version 2.0
        pmix_data_array_t *darray;      // version 2.0
        void *ptr;                      // version 2.0
        pmix_alloc_directive_t adir;    // version 2.0
    \} data;
\} pmix_value_t;
\end{codepar}
\cspecificend

%%%%%%%%%%%%%%%%%%%%%%%%%%%%%%%%%%%%%%%%%%%%%%%%%
\subsubsection{Value structure support macros}
The following macros are provided to support the \refstruct{pmix_value_t} structure.

\littleheader{Initialize the value structure}
\declaremacro{PMIX_VALUE_CONSTRUCT}

Initialize the \refstruct{pmix_value_t} fields.

\versionMarker{1.0}
\cspecificstart
\begin{codepar}
PMIX_VALUE_CONSTRUCT(m)
\end{codepar}
\cspecificend

\begin{arglist}
\argin{m}{Pointer to the structure to be initialized (pointer to \refstruct{pmix_value_t})}
\end{arglist}

\littleheader{Destruct the value structure}
\declaremacro{PMIX_VALUE_DESTRUCT}

Destruct the \refstruct{pmix_value_t} fields.

\versionMarker{1.0}
\cspecificstart
\begin{codepar}
PMIX_VALUE_DESTRUCT(m)
\end{codepar}
\cspecificend

\begin{arglist}
\argin{m}{Pointer to the structure to be destructed (pointer to \refstruct{pmix_value_t})}
\end{arglist}

%%%%%%%%%%%
\littleheader{Create a value array}
\declaremacro{PMIX_VALUE_CREATE}

Allocate and initialize an array of \refstruct{pmix_value_t} structures.

\versionMarker{1.0}
\cspecificstart
\begin{codepar}
PMIX_VALUE_CREATE(m, n)
\end{codepar}
\cspecificend

\begin{arglist}
\arginout{m}{Address where the pointer to the array of \refstruct{pmix_value_t} structures shall be stored (handle)}
\argin{n}{Number of structures to be allocated (\code{size_t})}
\end{arglist}


%%%%%%%%%%%
\littleheader{Free a value structure}
\declaremacro{PMIX_VALUE_RELEASE}

Release a \refstruct{pmix_value_t} structure.

\versionMarker{4.0}
\cspecificstart
\begin{codepar}
PMIX_VALUE_RELEASE(m)
\end{codepar}
\cspecificend

\begin{arglist}
\argin{m}{Pointer to a \refstruct{pmix_value_t} structure (handle)}
\end{arglist}

%%%%%%%%%%%
\littleheader{Free a value array}
\declaremacro{PMIX_VALUE_FREE}

Release an array of \refstruct{pmix_value_t} structures.

\versionMarker{1.0}
\cspecificstart
\begin{codepar}
PMIX_VALUE_FREE(m, n)
\end{codepar}
\cspecificend

\begin{arglist}
\argin{m}{Pointer to the array of \refstruct{pmix_value_t} structures (handle)}
\argin{n}{Number of structures in the array (\code{size_t})}
\end{arglist}

%%%%%%%%%%%
\littleheader{Load a value structure}
\declaremacro{PMIX_VALUE_LOAD}

Load data into a \refstruct{pmix_value_t} structure.

\versionMarker{2.0}
\cspecificstart
\begin{codepar}
PMIX_VALUE_LOAD(v, d, t);
\end{codepar}
\cspecificend

\begin{arglist}
\argin{v}{The \refstruct{pmix_value_t} into which the data is to be loaded (pointer to \refstruct{pmix_value_t})}
\argin{d}{Pointer to the data value to be loaded (handle)}
\argin{t}{Type of the provided data value (\refstruct{pmix_data_type_t})}
\end{arglist}

This macro simplifies the loading of data into a \refstruct{pmix_value_t} by correctly assigning values to the structure's fields.

\adviceuserstart
The data will be copied into the \refstruct{pmix_value_t} - thus, any data stored in the source value can be modified or free'd without affecting the copied data once the macro has completed.
\adviceuserend

%%%%%%%%%%%
\littleheader{Unload a value structure}
\declaremacro{PMIX_VALUE_UNLOAD}

Unload data from a \refstruct{pmix_value_t} structure.

\versionMarker{2.2}
\cspecificstart
\begin{codepar}
PMIX_VALUE_UNLOAD(r, v, d, t);
\end{codepar}
\cspecificend

\begin{arglist}
\argout{r}{Status code indicating result of the operation {\refstruct{pmix_status_t}}}
\argin{v}{The \refstruct{pmix_value_t} from which the data is to be unloaded (pointer to \refstruct{pmix_value_t})}
\arginout{d}{Pointer to the location where the data value is to be returned (handle)}
\arginout{t}{Pointer to return the data type of the unloaded value (handle)}
\end{arglist}

This macro simplifies the unloading of data from a \refstruct{pmix_value_t}.

\adviceuserstart
Memory will be allocated and the data will be in the \refstruct{pmix_value_t} returned - the source \refstruct{pmix_value_t} will not be altered.
\adviceuserend

%%%%%%%%%%%
\littleheader{Transfer data between value structures}
\declaremacro{PMIX_VALUE_XFER}

Transfer the data value between two \refstruct{pmix_value_t} structures.

\versionMarker{2.0}
\cspecificstart
\begin{codepar}
PMIX_VALUE_XFER(r, d, s);
\end{codepar}
\cspecificend

\begin{arglist}
\argout{r}{Status code indicating success or failure of the transfer (\refstruct{pmix_status_t})}
\argin{d}{Pointer to the \refstruct{pmix_value_t} destination (handle)}
\argin{s}{Pointer to the \refstruct{pmix_value_t} source (handle)}
\end{arglist}

This macro simplifies the transfer of data between two \refstruct{pmix_value_t} structures, ensuring that all fields are properly copied.

\adviceuserstart
The data will be copied into the destination \refstruct{pmix_value_t} - thus, any data stored in the source value can be modified or free'd without affecting the copied data once the macro has completed.
\adviceuserend

%%%%%%%%%%%
\littleheader{Retrieve a numerical value from a value struct}
\declaremacro{PMIX_VALUE_GET_NUMBER}

Retrieve a numerical value from a \refstruct{pmix_value_t} structure.

\versionMarker{3.0}
\cspecificstart
\begin{codepar}
PMIX_VALUE_GET_NUMBER(s, m, n, t)
\end{codepar}
\cspecificend

\begin{arglist}
\argout{s}{Status code for the request (\refstruct{pmix_status_t})}
\argin{m}{Pointer to the\refstruct{pmix_value_t} structure (handle)}
\argout{n}{Variable to be set to the value (match expected type)}
\argin{t}{Type of number expected in \refarg{m} (\refstruct{pmix_data_type_t})}
\end{arglist}

Sets the provided variable equal to the numerical value contained in the given \refstruct{pmix_value_t}, returning success if the data type of the value matches the expected type and \refconst{PMIX_ERR_BAD_PARAM} if it doesn't

%%%%%%%%%%%%%%%%%%%%%%%%%%%%%%%%%%%%%%%%%%%%%%%%%
\subsection{Info Structure}
\label{chap:struct:info}
\declarestruct{pmix_info_t}

The \refstruct{pmix_info_t} structure defines a key/value pair with associated directive. All fields were defined in version 1.0 unless otherwise marked.

\versionMarker{1.0}
\cspecificstart
\begin{codepar}
typedef struct pmix_info_t \{
    pmix_key_t key;
    pmix_info_directives_t flags;    // version 2.0
    pmix_value_t value;
\} pmix_info_t;
\end{codepar}
\cspecificend

%%%%%%%%%%%
\subsubsection{Info structure support macros}
The following macros are provided to support the \refstruct{pmix_info_t} structure.

\littleheader{Initialize the info structure}
\declaremacro{PMIX_INFO_CONSTRUCT}

Initialize the \refstruct{pmix_info_t} fields.

\versionMarker{1.0}
\cspecificstart
\begin{codepar}
PMIX_INFO_CONSTRUCT(m)
\end{codepar}
\cspecificend

\begin{arglist}
\argin{m}{Pointer to the structure to be initialized (pointer to \refstruct{pmix_info_t})}
\end{arglist}

\littleheader{Destruct the info structure}
\declaremacro{PMIX_INFO_DESTRUCT}

Destruct the \refstruct{pmix_info_t} fields.

\versionMarker{1.0}
\cspecificstart
\begin{codepar}
PMIX_INFO_DESTRUCT(m)
\end{codepar}
\cspecificend

\begin{arglist}
\argin{m}{Pointer to the structure to be destructed (pointer to \refstruct{pmix_info_t})}
\end{arglist}

%%%%%%%%%%%
\littleheader{Create an info array}
\declaremacro{PMIX_INFO_CREATE}

Allocate and initialize an array of info structures.

\versionMarker{1.0}
\cspecificstart
\begin{codepar}
PMIX_INFO_CREATE(m, n)
\end{codepar}
\cspecificend

\begin{arglist}
\arginout{m}{Address where the pointer to the array of \refstruct{pmix_info_t} structures shall be stored (handle)}
\argin{n}{Number of structures to be allocated (\code{size_t})}
\end{arglist}


%%%%%%%%%%%
\littleheader{Free an info array}
\declaremacro{PMIX_INFO_FREE}

Release an array of \refstruct{pmix_info_t} structures.

\versionMarker{1.0}
\cspecificstart
\begin{codepar}
PMIX_INFO_FREE(m, n)
\end{codepar}
\cspecificend

\begin{arglist}
\argin{m}{Pointer to the array of \refstruct{pmix_info_t} structures (handle)}
\argin{n}{Number of structures in the array (\code{size_t})}
\end{arglist}

%%%%%%%%%%%
\littleheader{Load key and value data into a info struct}
\declaremacro{PMIX_INFO_LOAD}

\versionMarker{1.0}
\cspecificstart
\begin{codepar}
PMIX_INFO_LOAD(v, k, d, t);
\end{codepar}
\cspecificend

\begin{arglist}
\argin{v}{Pointer to the \refstruct{pmix_info_t} into which the key and data are to be loaded (pointer to \refstruct{pmix_info_t})}
\argin{k}{String key to be loaded - must be less than or equal to \refconst{PMIX_MAX_KEYLEN} in length (handle)}
\argin{d}{Pointer to the data value to be loaded (handle)}
\argin{t}{Type of the provided data value (\refstruct{pmix_data_type_t})}
\end{arglist}

This macro simplifies the loading of key and data into a \refstruct{pmix_info_t} by correctly assigning values to the structure's fields.

\adviceuserstart
Both key and data will be copied into the \refstruct{pmix_info_t} - thus, the key and any data stored in the source value can be modified or free'd without affecting the copied data once the macro has completed.
\adviceuserend

%%%%%%%%%%%
\littleheader{Copy data between info structures}
\declaremacro{PMIX_INFO_XFER}

Copy all data (including key, value, and directives) between two \refstruct{pmix_info_t} structures.

\versionMarker{2.0}
\cspecificstart
\begin{codepar}
PMIX_INFO_XFER(d, s);
\end{codepar}
\cspecificend

\begin{arglist}
\argin{d}{Pointer to the destination \refstruct{pmix_info_t} (pointer to \refstruct{pmix_info_t})}
\argin{s}{Pointer to the source \refstruct{pmix_info_t} (pointer to \refstruct{pmix_info_t})}
\end{arglist}

This macro simplifies the transfer of data between two\refstruct{pmix_info_t} structures.

\adviceuserstart
All data (including key, value, and directives) will be copied into the destination \refstruct{pmix_info_t} - thus, the source \refstruct{pmix_info_t} may be free'd without affecting the copied data once the macro has completed.
\adviceuserend


%%%%%%%%%%%
\littleheader{Test a boolean info struct}
\declaremacro{PMIX_INFO_TRUE}

A special macro for checking if a boolean \refstruct{pmix_info_t} is \code{true}.

\versionMarker{2.0}
\cspecificstart
\begin{codepar}
PMIX_INFO_TRUE(m)
\end{codepar}
\cspecificend

\begin{arglist}
\argin{m}{Pointer to a \refstruct{pmix_info_t} structure (handle)}
\end{arglist}

A \refstruct{pmix_info_t} structure is considered to be of type \refconst{PMIX_BOOL} and value \code{true} if:

\begin{compactitemize}
    \item the structure reports a type of \refconst{PMIX_UNDEF}, or
    \item the structure reports a type of \refconst{PMIX_BOOL} and the data flag is \code{true}
\end{compactitemize}

%%%%%%%%%%%
\subsubsection{Info structure list macros}
Constructing an array of \refstruct{pmix_info_t} is a fairly common operation. The following macros are provided to simplify this construction.

%%%%%%%%%%%
\littleheader{Start a list of \refstruct{pmix_info_t} structures}
\declaremacro{PMIX_INFO_LIST_START}

Initialize a list of \refstruct{pmix_info_t} structures. The actual list is opaque to the caller and is implementation-dependent.

\versionMarker{4.0}
\cspecificstart
\begin{codepar}
PMIX_INFO_LIST_START(m)
\end{codepar}
\cspecificend

\begin{arglist}
\argin{m}{A \code{void*} pointer (handle)}
\end{arglist}

Note that the pointer will be initialized to an opaque structure whose elements are implementation-dependent. The caller must not modify or dereference the object.

%%%%%%%%%%%
\littleheader{Add a \refstruct{pmix_info_t} structure to a list}
\declaremacro{PMIX_INFO_LIST_ADD}

Add a \refstruct{pmix_info_t} structure containing the specified value to the provided list.

\versionMarker{4.0}
\cspecificstart
\begin{codepar}
PMIX_INFO_LIST_ADD(rc, m, k, d, t)
\end{codepar}
\cspecificend

\begin{arglist}
\arginout{rc}{Return status for the operation (\refstruct{pmix_status_t})}
\argin{m}{A \code{void*} pointer initialized via \refmacro{PMIX_INFO_LIST_START} (handle)}
\argin{k}{String key to be loaded - must be less than or equal to \refconst{PMIX_MAX_KEYLEN} in length (handle)}
\argin{d}{Pointer to the data value to be loaded (handle)}
\argin{t}{Type of the provided data value (\refstruct{pmix_data_type_t})}
\end{arglist}

\adviceuserstart
Both key and data will be copied into the \refstruct{pmix_info_t} on the list - thus, the key and any data stored in the source value can be modified or free'd without affecting the copied data once the macro has completed.
\adviceuserend

%%%%%%%%%%%
\littleheader{Transfer a \refstruct{pmix_info_t} structure to a list}
\declaremacro{PMIX_INFO_LIST_XFER}

Transfer the information in a \refstruct{pmix_info_t} structure to the provided list.

\versionMarker{4.0}
\cspecificstart
\begin{codepar}
PMIX_INFO_LIST_XFER(rc, m, s)
\end{codepar}
\cspecificend

\begin{arglist}
\arginout{rc}{Return status for the operation (\refstruct{pmix_status_t})}
\argin{m}{A \code{void*} pointer initialized via \refmacro{PMIX_INFO_LIST_START} (handle)}
\argin{s}{Pointer to the source \refstruct{pmix_info_t} (pointer to \refstruct{pmix_info_t})}
\end{arglist}

\adviceuserstart
All data (including key, value, and directives) will be copied into the destination \refstruct{pmix_info_t} on the list - thus, the source \refstruct{pmix_info_t} may be free'd without affecting the copied data once the macro has completed.
\adviceuserend

%%%%%%%%%%%
\littleheader{Convert a \refstruct{pmix_info_t} list to an array}
\declaremacro{PMIX_INFO_LIST_CONVERT}

Transfer the information in the provided \refstruct{pmix_info_t} list to a \refstruct{pmix_data_array_t} array

\versionMarker{4.0}
\cspecificstart
\begin{codepar}
PMIX_INFO_LIST_CONVERT(rc, m, d)
\end{codepar}
\cspecificend

\begin{arglist}
\arginout{rc}{Return status for the operation (\refstruct{pmix_status_t})}
\argin{m}{A \code{void*} pointer initialized via \refmacro{PMIX_INFO_LIST_START} (handle)}
\argin{d}{Pointer to an instantiated \refstruct{pmix_data_array_t} structure where the \refstruct{pmix_info_t} array is to be stored (pointer to \refstruct{pmix_data_array_t})}
\end{arglist}

%%%%%%%%%%%
\littleheader{Release a \refstruct{pmix_info_t} list}
\declaremacro{PMIX_INFO_LIST_RELEASE}

Release the provided \refstruct{pmix_info_t} list

\versionMarker{4.0}
\cspecificstart
\begin{codepar}
PMIX_INFO_LIST_RELEASE(m)
\end{codepar}
\cspecificend

\begin{arglist}
\argin{m}{A \code{void*} pointer initialized via \refmacro{PMIX_INFO_LIST_START} (handle)}
\end{arglist}

Information contained in the \refstruct{pmix_info_t} on the list shall be released in addition to whatever backing storage the implementation may have allocated to support construction of the list.


%%%%%%%%%%%%%%%%%%%%%%%%%%%%%%%%%%%%%%%%%%%%%%%%%
\subsection{Info Type Directives}
\declarestruct{pmix_info_directives_t}
\label{api:struct:infodirs}

\versionMarker{2.0}
The \refstruct{pmix_info_directives_t} structure is a \code{uint32_t} type that defines the behavior of command directives via \refstruct{pmix_info_t} arrays.
By default, the values in the \refstruct{pmix_info_t} array passed to a PMIx are \emph{optional}.

\adviceuserstart
A PMIx implementation or PMIx-enabled \ac{RM} may ignore any \refstruct{pmix_info_t} value passed to a \ac{PMIx} \ac{API} that it does not support or does not recognize if it is not explicitly marked as \refconst{PMIX_INFO_REQD}.
This is because the values specified default to optional, meaning they can be ignored in such circumstances.
This may lead to unexpected behavior when porting between environments or \ac{PMIx} implementations if the user is relying on the behavior specified by the \refstruct{pmix_info_t} value.
Users relying on the behavior defined by the \refstruct{pmix_info_t} are advised to set the \refconst{PMIX_INFO_REQD} flag using the \refmacro{PMIX_INFO_REQUIRED} macro.
\adviceuserend

\adviceimplstart
The top 16-bits of the \refstruct{pmix_info_directives_t} are reserved for internal use by \ac{PMIx} library implementers - the \ac{PMIx} standard will \textit{not} specify their intent, leaving them for customized use by implementers. Implementers are advised to use the provided \refmacro{PMIX_INFO_IS_REQUIRED} macro for testing this flag, and must return \refconst{PMIX_ERR_NOT_SUPPORTED} as soon as possible to the caller if the required behavior is not supported.
\adviceimplend

The following constants were introduced in version 2.0 (unless otherwise marked) and can be used to set a variable of the type \refstruct{pmix_info_directives_t}.

\begin{constantdesc}
%
\declareconstitem{PMIX_INFO_REQD}
The behavior defined in the \refstruct{pmix_info_t} array is required, and not optional. This is a bit-mask value.
%
\declareconstitemNEW{PMIX_INFO_REQD_PROCESSED}
Mark that this required attribute has been processed. A required attribute can be handled at any level - the \ac{PMIx} client library might take care of it, or it may be resolved by the \ac{PMIx} server library, or it may pass up to the host environment for handling. If a level does not recognize or support the required attribute, it is required to pass it upwards to give the next level an opportunity to process it. Thus, the host environment (or the server library if the host does not support the given operation) must know if a lower level has handled the requirement so it can return a \refconst{PMIX_ERR_NOT_SUPPORTED} error status if the host itself cannot meet the request. Upon processing the request, the level must therefore mark the attribute with this directive to alert any subsequent levels that the requirement has been met.
%
\declareconstitem{PMIX_INFO_ARRAY_END}
Mark that this \refstruct{pmix_info_t} struct is at the end of an array created by the \refmacro{PMIX_INFO_CREATE} macro. This is a bit-mask value.
%
\declareconstitemNEW{PMIX_INFO_DIR_RESERVED}
A bit-mask identifying the bits reserved for internal use by implementers - these currently are set as \code{0xffff0000}.
%
\end{constantdesc}

\advicermstart
Host environments are advised to use the provided \refmacro{PMIX_INFO_IS_REQUIRED} macro for testing this flag and must return \refconst{PMIX_ERR_NOT_SUPPORTED} as soon as possible to the caller if the required behavior is not supported.
\advicermend


\subsubsection{Info Directive support macros}

The following macros are provided to support the setting and testing of \refstruct{pmix_info_t} directives.

%%%%
\littleheader{Mark an info structure as required}
\declaremacro{PMIX_INFO_REQUIRED}

Set the \refconst{PMIX_INFO_REQD} flag in a \refstruct{pmix_info_t} structure.

\versionMarker{2.0}
\cspecificstart
\begin{codepar}
PMIX_INFO_REQUIRED(info);
\end{codepar}
\cspecificend

\begin{arglist}
\argin{info}{Pointer to the \refstruct{pmix_info_t} (pointer to \refstruct{pmix_info_t})}
\end{arglist}

This macro simplifies the setting of the \refconst{PMIX_INFO_REQD} flag in \refstruct{pmix_info_t} structures.

%%%%
\littleheader{Mark an info structure as optional}
\declaremacro{PMIX_INFO_OPTIONAL}

Unsets the \refconst{PMIX_INFO_REQD} flag in a \refstruct{pmix_info_t} structure.

\versionMarker{2.0}
\cspecificstart
\begin{codepar}
PMIX_INFO_OPTIONAL(info);
\end{codepar}
\cspecificend

\begin{arglist}
\argin{info}{Pointer to the \refstruct{pmix_info_t} (pointer to \refstruct{pmix_info_t})}
\end{arglist}

This macro simplifies marking a \refstruct{pmix_info_t} structure as \textit{optional}.

%%%%%%%%%%%
\littleheader{Test an info structure for \textit{required} directive}
\declaremacro{PMIX_INFO_IS_REQUIRED}

Test the \refconst{PMIX_INFO_REQD} flag in a \refstruct{pmix_info_t} structure, returning \code{true} if the flag is set.

\versionMarker{2.0}
\cspecificstart
\begin{codepar}
PMIX_INFO_IS_REQUIRED(info);
\end{codepar}
\cspecificend

\begin{arglist}
\argin{info}{Pointer to the \refstruct{pmix_info_t} (pointer to \refstruct{pmix_info_t})}
\end{arglist}

This macro simplifies the testing of the required flag in \refstruct{pmix_info_t} structures.

%%%%%%%%%%%
\littleheader{Test an info structure for \textit{optional} directive}
\declaremacro{PMIX_INFO_IS_OPTIONAL}

Test a \refstruct{pmix_info_t} structure, returning \code{true} if the structure is \textit{optional}.

\versionMarker{2.0}
\cspecificstart
\begin{codepar}
PMIX_INFO_IS_OPTIONAL(info);
\end{codepar}
\cspecificend

\begin{arglist}
\argin{info}{Pointer to the \refstruct{pmix_info_t} (pointer to \refstruct{pmix_info_t})}
\end{arglist}

Test the \refconst{PMIX_INFO_REQD} flag in a \refstruct{pmix_info_t} structure, returning \code{true} if the flag is \textit{not} set.

%%%%%%%%%%%
\littleheader{Mark a required attribute as processed}
\declaremacro{PMIX_INFO_PROCESSED}

Mark that a required \refstruct{pmix_info_t} structure has been processed.

\versionMarker{4.0}
\cspecificstart
\begin{codepar}
PMIX_INFO_PROCESSED(info);
\end{codepar}
\cspecificend

\begin{arglist}
\argin{info}{Pointer to the \refstruct{pmix_info_t} (pointer to \refstruct{pmix_info_t})}
\end{arglist}

Set the \refconst{PMIX_INFO_REQD_PROCESSED} flag in a \refstruct{pmix_info_t} structure indicating that is has been processed.

%%%%%%%%%%%
\littleheader{Test if a required attribute has been processed}
\declaremacro{PMIX_INFO_WAS_PROCESSED}

Test that a required \refstruct{pmix_info_t} structure has been processed.

\versionMarker{4.0}
\cspecificstart
\begin{codepar}
PMIX_INFO_WAS_PROCESSED(info);
\end{codepar}
\cspecificend

\begin{arglist}
\argin{info}{Pointer to the \refstruct{pmix_info_t} (pointer to \refstruct{pmix_info_t})}
\end{arglist}

Test the \refconst{PMIX_INFO_REQD_PROCESSED} flag in a \refstruct{pmix_info_t} structure.

%%%%%%%%%%%
\littleheader{Test an info structure for \textit{end of array} directive}
\declaremacro{PMIX_INFO_IS_END}

Test a \refstruct{pmix_info_t} structure, returning \code{true} if the structure is at the end of an array created by the \refmacro{PMIX_INFO_CREATE} macro.

\versionMarker{2.2}
\cspecificstart
\begin{codepar}
PMIX_INFO_IS_END(info);
\end{codepar}
\cspecificend

\begin{arglist}
\argin{info}{Pointer to the \refstruct{pmix_info_t} (pointer to \refstruct{pmix_info_t})}
\end{arglist}

This macro simplifies the testing of the end-of-array flag in \refstruct{pmix_info_t} structures.

%%%%%%%%%%%%%%%%%%%%%%%%%%%%%%%%%%%%%%%%%%%%%%%%%
\subsection{Environmental Variable Structure}
\declarestruct{pmix_envar_t}

\versionMarker{3.0}
Define a structure for specifying environment variable modifications.
Standard environment variables (e.g., \code{PATH}, \code{LD_LIBRARY_PATH}, and \code{LD_PRELOAD})
take multiple arguments separated by delimiters. Unfortunately, the delimiters
depend upon the variable itself - some use semi-colons, some colons, etc. Thus,
the operation requires not only the name of the variable to be modified and
the value to be inserted, but also the separator to be used when composing
the aggregate value.

\cspecificstart
\begin{codepar}
typedef struct \{
    char *envar;
    char *value;
    char separator;
\} pmix_envar_t;
\end{codepar}
\cspecificend

%%%%%%%%%%%%%%%%%%%%%%%%%%%%%%%%%%%%%%%%%%%%%%%%%
\subsubsection{Environmental variable support macros}

The following macros are provided to support the \refstruct{pmix_envar_t} structure.

\littleheader{Initialize the envar structure}
\declaremacro{PMIX_ENVAR_CONSTRUCT}

Initialize the \refstruct{pmix_envar_t} fields.

\versionMarker{3.0}
\cspecificstart
\begin{codepar}
PMIX_ENVAR_CONSTRUCT(m)
\end{codepar}
\cspecificend

\begin{arglist}
\argin{m}{Pointer to the structure to be initialized (pointer to \refstruct{pmix_envar_t})}
\end{arglist}

\littleheader{Destruct the envar structure}
\declaremacro{PMIX_ENVAR_DESTRUCT}

Clear the \refstruct{pmix_envar_t} fields.

\versionMarker{3.0}
\cspecificstart
\begin{codepar}
PMIX_ENVAR_DESTRUCT(m)
\end{codepar}
\cspecificend

\begin{arglist}
\argin{m}{Pointer to the structure to be destructed (pointer to \refstruct{pmix_envar_t})}
\end{arglist}


\littleheader{Create an envar array}
\declaremacro{PMIX_ENVAR_CREATE}

Allocate and initialize an array of \refstruct{pmix_envar_t} structures.

\versionMarker{3.0}
\cspecificstart
\begin{codepar}
PMIX_ENVAR_CREATE(m, n)
\end{codepar}
\cspecificend

\begin{arglist}
\arginout{m}{Address where the pointer to the array of \refstruct{pmix_envar_t} structures shall be stored (handle)}
\argin{n}{Number of structures to be allocated (\code{size_t})}
\end{arglist}


\littleheader{Free an envar array}
\declaremacro{PMIX_ENVAR_FREE}

Release an array of \refstruct{pmix_envar_t} structures.

\versionMarker{3.0}
\cspecificstart
\begin{codepar}
PMIX_ENVAR_FREE(m, n)
\end{codepar}
\cspecificend

\begin{arglist}
\argin{m}{Pointer to the array of \refstruct{pmix_envar_t} structures (handle)}
\argin{n}{Number of structures in the array (\code{size_t})}
\end{arglist}

\littleheader{Load an envar structure}
\declaremacro{PMIX_ENVAR_LOAD}

Load values into a \refstruct{pmix_envar_t}.

\versionMarker{2.0}
\cspecificstart
\begin{codepar}
PMIX_ENVAR_LOAD(m, e, v, s)
\end{codepar}
\cspecificend

\begin{arglist}
\argin{m}{Pointer to the structure to be loaded (pointer to \refstruct{pmix_envar_t})}
\argin{e}{Environmental variable name (\code{char*})}
\argin{v}{Value of variable (\code{char*})}
\argin{v}{Separator character (\code{char})}
\end{arglist}


%%%%%%%%%%%%%%%%%%%%%%%%%%%%%%%%%%%%%%%%%%%%%%%%%
\subsection{Byte Object Type}
\declarestruct{pmix_byte_object_t}

The \refstruct{pmix_byte_object_t} structure describes a raw byte sequence.

\versionMarker{1.0}
\cspecificstart
\begin{codepar}
typedef struct pmix_byte_object \{
    char *bytes;
    size_t size;
\} pmix_byte_object_t;
\end{codepar}
\cspecificend

%%%%%%%%%%%%%%%%%%%%%%%%%%%%%%%%%%%%%%%%%%%%%%%%%
\subsubsection{Byte object support macros}
The following macros support the \refstruct{pmix_byte_object_t} structure.

\littleheader{Initialize the byte object structure}
\declaremacro{PMIX_BYTE_OBJECT_CONSTRUCT}

Initialize the \refstruct{pmix_byte_object_t} fields.

\versionMarker{2.0}
\cspecificstart
\begin{codepar}
PMIX_BYTE_OBJECT_CONSTRUCT(m)
\end{codepar}
\cspecificend

\begin{arglist}
\argin{m}{Pointer to the structure to be initialized (pointer to \refstruct{pmix_byte_object_t})}
\end{arglist}

\littleheader{Destruct the byte object structure}
\declaremacro{PMIX_BYTE_OBJECT_DESTRUCT}

Clear the \refstruct{pmix_byte_object_t} fields.

\versionMarker{2.0}
\cspecificstart
\begin{codepar}
PMIX_BYTE_OBJECT_DESTRUCT(m)
\end{codepar}
\cspecificend

\begin{arglist}
\argin{m}{Pointer to the structure to be destructed (pointer to \refstruct{pmix_byte_object_t})}
\end{arglist}

\littleheader{Create a byte object structure}
\declaremacro{PMIX_BYTE_OBJECT_CREATE}

Allocate and intitialize an array of \refstruct{pmix_byte_object_t} structures.

\versionMarker{2.0}
\cspecificstart
\begin{codepar}
PMIX_BYTE_OBJECT_CREATE(m, n)
\end{codepar}
\cspecificend

\begin{arglist}
\arginout{m}{Address where the pointer to the array of \refstruct{pmix_byte_object_t} structures shall be stored (handle)}
\argin{n}{Number of structures to be allocated (\code{size_t})}
\end{arglist}

\littleheader{Free a byte object array}
\declaremacro{PMIX_BYTE_OBJECT_FREE}

Release an array of \refstruct{pmix_byte_object_t} structures.

\versionMarker{2.0}
\cspecificstart
\begin{codepar}
PMIX_BYTE_OBJECT_FREE(m, n)
\end{codepar}
\cspecificend

\begin{arglist}
\argin{m}{Pointer to the array of \refstruct{pmix_byte_object_t} structures (handle)}
\argin{n}{Number of structures in the array (\code{size_t})}
\end{arglist}

\littleheader{Load a byte object structure}
\declaremacro{PMIX_BYTE_OBJECT_LOAD}

Load values into a \refstruct{pmix_byte_object_t}.

\versionMarker{2.0}
\cspecificstart
\begin{codepar}
PMIX_BYTE_OBJECT_LOAD(b, d, s)
\end{codepar}
\cspecificend

\begin{arglist}
\argin{b}{Pointer to the structure to be loaded (pointer to \refstruct{pmix_byte_object_t})}
\argin{d}{Pointer to the data to be loaded (\code{char*})}
\argin{s}{Number of bytes in the data array (\code{size_t})}
\end{arglist}


%%%%%%%%%%%%%%%%%%%%%%%%%%%%%%%%%%%%%%%%%%%%%%%%%
\subsection{Data Array Structure}
\declarestruct{pmix_data_array_t}

The \refstruct{pmix_data_array_t} structure defines an array data structure.

\versionMarker{2.0}
\cspecificstart
\begin{codepar}
typedef struct pmix_data_array \{
    pmix_data_type_t type;
    size_t size;
    void *array;
\} pmix_data_array_t;
\end{codepar}
\cspecificend

%%%%%%%%%%%%%%%%%%%%%%%%%%%%%%%%%%%%%%%%%%%%%%%%%
\subsubsection{Data array support macros}
The following macros support the \refstruct{pmix_data_array_t} structure.

\littleheader{Initialize a data array structure}
\declaremacro{PMIX_DATA_ARRAY_CONSTRUCT}

Initialize the \refstruct{pmix_data_array_t} fields, allocating memory for the array of the indicated type.

\versionMarker{2.2}
\cspecificstart
\begin{codepar}
PMIX_DATA_ARRAY_CONSTRUCT(m, n, t)
\end{codepar}
\cspecificend

\begin{arglist}
\argin{m}{Pointer to the structure to be initialized (pointer to \refstruct{pmix_data_array_t})}
\argin{n}{Number of elements in the array (\code{size_t})}
\argin{t}{\ac{PMIx} data type of the array elements (\refstruct{pmix_data_type_t})}
\end{arglist}


\littleheader{Destruct a data array structure}
\declaremacro{PMIX_DATA_ARRAY_DESTRUCT}

Destruct the \refstruct{pmix_data_array_t}, releasing the memory in the array.

\versionMarker{2.2}
\cspecificstart
\begin{codepar}
PMIX_DATA_ARRAY_CONSTRUCT(m)
\end{codepar}
\cspecificend

\begin{arglist}
\argin{m}{Pointer to the structure to be destructed (pointer to \refstruct{pmix_data_array_t})}
\end{arglist}


\littleheader{Create a data array structure}
\declaremacro{PMIX_DATA_ARRAY_CREATE}

Allocate memory for the \refstruct{pmix_data_array_t} object itself, and then allocate memory for the array of the indicated type.

\versionMarker{2.2}
\cspecificstart
\begin{codepar}
PMIX_DATA_ARRAY_CREATE(m, n, t)
\end{codepar}
\cspecificend

\begin{arglist}
\arginout{m}{Variable to be set to the address of the structure (pointer to \refstruct{pmix_data_array_t})}
\argin{n}{Number of elements in the array (\code{size_t})}
\argin{t}{\ac{PMIx} data type of the array elements (\refstruct{pmix_data_type_t})}
\end{arglist}


\littleheader{Free a data array structure}
\declaremacro{PMIX_DATA_ARRAY_FREE}

Release the memory in the array, and then release the \refstruct{pmix_data_array_t} object itself.

\versionMarker{2.2}
\cspecificstart
\begin{codepar}
PMIX_DATA_ARRAY_FREE(m)
\end{codepar}
\cspecificend

\begin{arglist}
\argin{m}{Pointer to the structure to be released (pointer to \refstruct{pmix_data_array_t})}
\end{arglist}

%%%%%%%%%%%%%%%%%%%%%%%%%%%%%%%%%%%%%%%%%%%%%%%%%
\subsection{Argument Array Macros}

The following macros support the construction and release of \code{NULL}-terminated argv arrays of strings.

%%%%
\littleheader{Argument array extension}
\declaremacro{PMIX_ARGV_APPEND}

Append a string to a NULL-terminated, argv-style array of strings.

\cspecificstart
\begin{codepar}
PMIX_ARGV_APPEND(r, a, b);
\end{codepar}
\cspecificend

\begin{arglist}
\argout{r}{Status code indicating success or failure of the operation (\refstruct{pmix_status_t})}
\arginout{a}{Argument list (pointer to NULL-terminated array of strings)}
\argin{b}{Argument to append to the list (string)}
\end{arglist}

This function helps the caller build the \code{argv} portion of \refstruct{pmix_app_t} structure, arrays of keys for querying, or other places where argv-style string arrays are required.

\adviceuserstart
The provided argument is copied into the destination array - thus, the source string can be free'd without affecting the array once the macro has completed.
\adviceuserend

%%%%
\littleheader{Argument array prepend}
\declaremacro{PMIX_ARGV_PREPEND}

Prepend a string to a NULL-terminated, argv-style array of strings.

\cspecificstart
\begin{codepar}
PMIX_ARGV_PREPEND(r, a, b);
\end{codepar}
\cspecificend

\begin{arglist}
\argout{r}{Status code indicating success or failure of the operation (\refstruct{pmix_status_t})}
\arginout{a}{Argument list (pointer to NULL-terminated array of strings)}
\argin{b}{Argument to append to the list (string)}
\end{arglist}

This function helps the caller build the \code{argv} portion of \refstruct{pmix_app_t} structure, arrays of keys for querying, or other places where argv-style string arrays are required.

\adviceuserstart
The provided argument is copied into the destination array - thus, the source string can be free'd without affecting the array once the macro has completed.
\adviceuserend

%%%%%%%%%%%
\littleheader{Argument array extension - unique}
\declaremacro{PMIX_ARGV_APPEND_UNIQUE}

Append a string to a NULL-terminated, argv-style array of strings, but only if the provided argument doesn't already exist somewhere in the array.

\cspecificstart
\begin{codepar}
PMIX_ARGV_APPEND_UNIQUE(r, a, b);
\end{codepar}
\cspecificend

\begin{arglist}
\argout{r}{Status code indicating success or failure of the operation (\refstruct{pmix_status_t})}
\arginout{a}{Argument list (pointer to NULL-terminated array of strings)}
\argin{b}{Argument to append to the list (string)}
\end{arglist}

This function helps the caller build the \code{argv} portion of \refstruct{pmix_app_t} structure, arrays of keys for querying, or other places where argv-style string arrays are required.

\adviceuserstart
The provided argument is copied into the destination array - thus, the source string can be free'd without affecting the array once the macro has completed.
\adviceuserend

%%%%%%%%%%%
\littleheader{Argument array release}
\declaremacro{PMIX_ARGV_FREE}

Free an argv-style array and all of the strings that it contains.

\cspecificstart
\begin{codepar}
PMIX_ARGV_FREE(a);
\end{codepar}
\cspecificend

\begin{arglist}
\argin{a}{Argument list (pointer to NULL-terminated array of strings)}
\end{arglist}

This function releases the array and all of the strings it contains.

%%%%%%%%%%%
\littleheader{Argument array split}
\declaremacro{PMIX_ARGV_SPLIT}

Split a string into a NULL-terminated argv array.

\cspecificstart
\begin{codepar}
PMIX_ARGV_SPLIT(a, b, c);
\end{codepar}
\cspecificend

\begin{arglist}
\argout{a}{Resulting argv-style array (\code{char**})}
\argin{b}{String to be split (\code{char*})}
\argin{c}{Delimiter character (\code{char})}
\end{arglist}

Split an input string into a NULL-terminated argv array. Do not include empty strings in the resulting array.

\adviceuserstart
All strings are inserted into the argv array by value; the newly-allocated array makes no references to the src_string argument (i.e., it can be freed after calling this function without invalidating the output argv array)
\adviceuserend

%%%%%%%%%%%
\littleheader{Argument array join}
\declaremacro{PMIX_ARGV_JOIN}

Join all the elements of an argv array into a single newly-allocated string.

\cspecificstart
\begin{codepar}
PMIX_ARGV_JOIN(a, b, c);
\end{codepar}
\cspecificend

\begin{arglist}
\argout{a}{Resulting string (\code{char*})}
\argin{b}{Argv-style array to be joined (\code{char**})}
\argin{c}{Delimiter character (\code{char})}
\end{arglist}

Join all the elements of an argv array into a single newly-allocated string.

%%%%%%%%%%%
\littleheader{Argument array count}
\declaremacro{PMIX_ARGV_COUNT}

Return the length of a NULL-terminated argv array.

\cspecificstart
\begin{codepar}
PMIX_ARGV_COUNT(r, a);
\end{codepar}
\cspecificend

\begin{arglist}
\argout{r}{Number of strings in the array (integer)}
\argin{a}{Argv-style array (\code{char**})}
\end{arglist}

Count the number of elements in an argv array

%%%%%%%%%%%
\littleheader{Argument array copy}
\declaremacro{PMIX_ARGV_COPY}

Copy an argv array, including copying all of its strings.

\cspecificstart
\begin{codepar}
PMIX_ARGV_COPY(a, b);
\end{codepar}
\cspecificend

\begin{arglist}
\argout{a}{New argv-style array (\code{char**})}
\argin{b}{Argv-style array (\code{char**})}
\end{arglist}

Copy an argv array, including copying all of its strings.


%%%%%%%%%%%%%%%%%%%%%%%%%%%%%%%%%%%%%%%%%%%%%%%%%
\subsection{Set Environment Variable}
\declaremacro{PMIX_SETENV}

%%%%
\summary

Set an environment variable in a \code{NULL}-terminated, env-style array.

\cspecificstart
\begin{codepar}
PMIX_SETENV(r, name, value, env);
\end{codepar}
\cspecificend


\begin{arglist}
\argout{r}{Status code indicating success or failure of the operation (\refstruct{pmix_status_t})}
\argin{name}{Argument name (string)}
\argin{value}{Argument value (string)}
\arginout{env}{Environment array to update (pointer to array of strings)}
\end{arglist}

%%%%
\descr

Similar to \code{setenv} from the C API, this allows the caller to set an environment variable in the specified \code{env} array, which could then be passed to the \refstruct{pmix_app_t} structure or any other destination.

\adviceuserstart
The provided name and value are copied into the destination environment array - thus, the source strings can be free'd without affecting the array once the macro has completed.
\adviceuserend


%%%%%%%%%%%%%%%%%%%%%%%%%%%%%%%%%%%%%%%%%%%%%%%%%
%%%%%%%%%%%%%%%%%%%%%%%%%%%%%%%%%%%%%%%%%%%%%%%%%
\section{Generalized Data Types Used for Packing/Unpacking}
\declarestruct{pmix_data_type_t}

The \refstruct{pmix_data_type_t} structure is a \code{uint16_t} type for identifying the data type for packing/unpacking purposes. New data type values introduced in this version of the Standard are shown in \textbf{\color{magenta}magenta}.

\adviceimplstart
The following constants can be used to set a variable of the type \refstruct{pmix_data_type_t}. Data types in the \ac{PMIx} Standard are defined in terms of the C-programming language. Implementers wishing to support other languages should provide the equivalent definitions in a language-appropriate manner. Additionally, a PMIx implementation may choose to add additional types.
\adviceimplend

\begin{constantdesc}
%
\declareconstitem{PMIX_UNDEF}
Undefined.
%
\declareconstitem{PMIX_BOOL}
Boolean (converted to/from native \code{true}/\code{false}) (\code{bool}).
%
\declareconstitem{PMIX_BYTE}
A byte of data (\code{uint8_t}).
%
\declareconstitem{PMIX_STRING}
\code{NULL} terminated string (\code{char*}).
%
\declareconstitem{PMIX_SIZE}
Size \code{size_t}.
%
\declareconstitem{PMIX_PID}
Operating \ac{PID} (\code{pid_t}).
%
\declareconstitem{PMIX_INT}
Integer (\code{int}).
%
\declareconstitem{PMIX_INT8}
8-byte integer (\code{int8_t}).
%
\declareconstitem{PMIX_INT16}
16-byte integer (\code{int16_t}).
%
\declareconstitem{PMIX_INT32}
32-byte integer (\code{int32_t}).
%
\declareconstitem{PMIX_INT64}
64-byte integer (\code{int64_t}).
%
\declareconstitem{PMIX_UINT}
Unsigned integer (\code{unsigned int}).
%
\declareconstitem{PMIX_UINT8}
Unsigned 8-byte integer (\code{uint8_t}).
%
\declareconstitem{PMIX_UINT16}
Unsigned 16-byte integer (\code{uint16_t}).
%
\declareconstitem{PMIX_UINT32}
Unsigned 32-byte integer (\code{uint32_t}).
%
\declareconstitem{PMIX_UINT64}
Unsigned 64-byte integer (\code{uint64_t}).
%
\declareconstitem{PMIX_FLOAT}
Float (\code{float}).
%
\declareconstitem{PMIX_DOUBLE}
Double (\code{double}).
%
\declareconstitem{PMIX_TIMEVAL}
Time value (\code{struct timeval}).
%
\declareconstitem{PMIX_TIME}
Time (\code{time_t}).
%
\declareconstitem{PMIX_STATUS}
Status code {\refstruct{pmix_status_t}}.
%
\declareconstitem{PMIX_VALUE}
Value (\refstruct{pmix_value_t}).
%
\declareconstitem{PMIX_PROC}
Process (\refstruct{pmix_proc_t}).
%
\declareconstitem{PMIX_APP}
Application context.
%
\declareconstitem{PMIX_INFO}
Info object.
%
\declareconstitem{PMIX_PDATA}
Pointer to data.
%
\declareconstitem{PMIX_BUFFER}
Buffer.
%
\declareconstitem{PMIX_BYTE_OBJECT}
Byte object (\refstruct{pmix_byte_object_t}).
%
\declareconstitem{PMIX_KVAL}
Key/value pair.
%
\declareconstitem{PMIX_PERSIST}
Persistance (\refstruct{pmix_persistence_t}).
%
\declareconstitem{PMIX_POINTER}
Pointer to an object (\code{void*}).
%
\declareconstitem{PMIX_SCOPE}
Scope (\refstruct{pmix_scope_t}).
%
\declareconstitem{PMIX_DATA_RANGE}
Range for data (\refstruct{pmix_data_range_t}).
%
\declareconstitem{PMIX_COMMAND}
PMIx command code (used internally).
%
\declareconstitem{PMIX_INFO_DIRECTIVES}
Directives flag for \refstruct{pmix_info_t} (\refstruct{pmix_info_directives_t}).
%
\declareconstitem{PMIX_DATA_TYPE}
Data type code (\refstruct{pmix_data_type_t}).
%
\declareconstitem{PMIX_PROC_STATE}
Process state (\refstruct{pmix_proc_state_t}).
%
\declareconstitem{PMIX_PROC_INFO}
Process information (\refstruct{pmix_proc_info_t}).
%
\declareconstitem{PMIX_DATA_ARRAY}
Data array (\refstruct{pmix_data_array_t}).
%
\declareconstitem{PMIX_PROC_RANK}
Process rank (\refstruct{pmix_rank_t}).
%
\declareconstitem{PMIX_QUERY}
Query structure (\refstruct{pmix_query_t}).
%
\declareconstitem{PMIX_COMPRESSED_STRING}
String compressed with zlib (\code{char*}).
%
\declareconstitemNEW{PMIX_COMPRESSED_BYTE_OBJECT}
Byte object whose bytes have been compressed with zlib (\code{pmix_byte_object_t}).
%
\declareconstitem{PMIX_ALLOC_DIRECTIVE}
Allocation directive (\refstruct{pmix_alloc_directive_t}).
%
\declareconstitem{PMIX_IOF_CHANNEL}
Input/output forwarding channel (\refstruct{pmix_iof_channel_t}).
%
\declareconstitem{PMIX_ENVAR}
Environmental variable structure (\refstruct{pmix_envar_t}).
%
\declareconstitemNEW{PMIX_COORD}
Structure containing fabric coordinates (\refstruct{pmix_coord_t}).
%
\declareconstitemNEW{PMIX_REGATTR}
Structure supporting attribute registrations (\refstruct{pmix_regattr_t}).
%
\declareconstitemNEW{PMIX_REGEX}
Regular expressions - can be a valid NULL-terminated string or an arbitrary array of bytes.
%
\declareconstitemNEW{PMIX_JOB_STATE}
Job state (\refstruct{pmix_job_state_t}).
%
\declareconstitemNEW{PMIX_LINK_STATE}
Link state (\refstruct{pmix_link_state_t}).
%
\declareconstitemNEW{PMIX_PROC_CPUSET}
Structure containing the binding bitmap of a process (\refstruct{pmix_cpuset_t}).
%
\declareconstitemNEW{PMIX_GEOMETRY}
Geometry structure containing the fabric coordinates of a specified device.(\refstruct{pmix_geometry_t}).
%
\declareconstitemNEW{PMIX_DEVICE_DIST}
Structure containing the minimum and maximum relative distance from the caller to a given fabric device. (\refstruct{pmix_device_distance_t}).
%
\declareconstitemNEW{PMIX_ENDPOINT}
Structure containing an assigned endpoint for a given fabric device. (\refstruct{pmix_endpoint_t}).
%
\declareconstitemNEW{PMIX_TOPO}
Structure containing the topology for a given node. (\refstruct{pmix_topology_t}).
%
\declareconstitemNEW{PMIX_DEVTYPE}
Bitmask containing the types of devices being referenced. (\refstruct{pmix_device_type_t}).
%
\declareconstitemNEW{PMIX_LOCTYPE}
Bitmask describing the relative location of another process. (\refstruct{pmix_locality_t}).
%
\declareconstitemNEW{PMIX_DATA_TYPE_MAX}
A starting point for implementer-specific data types.
Values above this are guaranteed not to conflict with \ac{PMIx} values.
Definitions should always be based on the \refconst{PMIX_DATA_TYPE_MAX} constant and not a specific value as the value of the constant may change.
%
\end{constantdesc}


%%%%%%%%%%%%%%%%%%%%%%%%%%%%%%%%%%%%%%%%%%%%%%%%%
%%%%%%%%%%%%%%%%%%%%%%%%%%%%%%%%%%%%%%%%%%%%%%%%%
\section{General Callback Functions}

PMIx provides blocking and nonblocking versions of most APIs.
In the nonblocking versions, a callback is activated upon completion of the the operation.
This section describes many of those callbacks.

%%%%%%%%%%%%%%%%%%%%%%%%%%%%%%%%%%%%%%%%%%%%%%%%%
\subsection{Release Callback Function}
\declareapi{pmix_release_cbfunc_t}

%%%%
\summary

The \refapi{pmix_release_cbfunc_t} is used by the \refapi{pmix_modex_cbfunc_t} and \refapi{pmix_info_cbfunc_t} operations to indicate that the callback data may be reclaimed/freed by the caller.

%%%%
\format

\versionMarker{1.0}
\cspecificstart
\begin{codepar}
typedef void (*pmix_release_cbfunc_t)
    (void *cbdata);
\end{codepar}
\cspecificend

\begin{arglist}
\arginout{cbdata}{Callback data passed to original API call (memory reference)}
\end{arglist}

%%%%
\descr

Since the data is ``owned'' by the host server, provide a callback function to notify the host server that we are done with the data so it can be released.


%%%%%%%%%%%%%%%%%%%%%%%%%%%%%%%%%%%%%%%%%%%%%%%%%
\subsection{Op Callback Function}
\declareapi{pmix_op_cbfunc_t}

%%%%
\summary

The \refapi{pmix_op_cbfunc_t} is used by operations that simply return a status.

\versionMarker{1.0}
\cspecificstart
\begin{codepar}
typedef void (*pmix_op_cbfunc_t)
    (pmix_status_t status, void *cbdata);
\end{codepar}
\cspecificend

\begin{arglist}
\argin{status}{Status associated with the operation (handle)}
\argin{cbdata}{Callback data passed to original API call (memory reference)}
\end{arglist}

%%%%
\descr

Used by a wide range of \ac{PMIx} API's including \refapi{PMIx_Fence_nb}, \refapi{pmix_server_client_connected2_fn_t}, \refapi{PMIx_server_register_nspace}.
This callback function is used to return a status to an often nonblocking operation.


%%%%%%%%%%%%%%%%%%%%%%%%%%%%%%%%%%%%%%%%%%%%%%%%%
\subsection{Value Callback Function}
\declareapi{pmix_value_cbfunc_t}

%%%%
\summary

The \refapi{pmix_value_cbfunc_t} is used by \refapi{PMIx_Get_nb} to return data.

\versionMarker{1.0}
\cspecificstart
\begin{codepar}
typedef void (*pmix_value_cbfunc_t)
    (pmix_status_t status,
     pmix_value_t *kv, void *cbdata);
\end{codepar}
\cspecificend

\begin{arglist}
\argin{status}{Status associated with the operation (handle)}
\argin{kv}{Key/value pair representing the data (\refstruct{pmix_value_t})}
\argin{cbdata}{Callback data passed to original API call (memory reference)}
\end{arglist}


%%%%
\descr

A callback function for calls to \refapi{PMIx_Get_nb}.
The \refarg{status} indicates if the requested data was found or not.
A pointer to the \refstruct{pmix_value_t} structure containing the found data is returned.
The pointer will be \code{NULL} if the requested data was not found.


%%%%%%%%%%%%%%%%%%%%%%%%%%%%%%%%%%%%%%%%%%%%%%%%%
\subsection{Info Callback Function}
\declareapi{pmix_info_cbfunc_t}

%%%%
\summary

The \refapi{pmix_info_cbfunc_t} is a general information callback used by various APIs.

\versionMarker{2.0}
\cspecificstart
\begin{codepar}
typedef void (*pmix_info_cbfunc_t)
    (pmix_status_t status,
     pmix_info_t info[], size_t ninfo,
     void *cbdata,
     pmix_release_cbfunc_t release_fn,
     void *release_cbdata);
\end{codepar}
\cspecificend

\begin{arglist}
\argin{status}{Status associated with the operation (\refstruct{pmix_status_t})}
\argin{info}{Array of \refstruct{pmix_info_t} returned by the operation (pointer)}
\argin{ninfo}{Number of elements in the \argref{info} array (\code{size_t})}
\argin{cbdata}{Callback data passed to original API call (memory reference)}
\argin{release_fn}{Function to be called when done with the \argref{info} data (function pointer)}
\argin{release_cbdata}{Callback data to be passed to \argref{release_fn} (memory reference)}
\end{arglist}


%%%%
\descr

The \refarg{status} indicates if requested data was found or not.
An array of \refstruct{pmix_info_t} will contain the key/value pairs.

%%%%%%%%%%%
\subsection{Handler registration callback function}
\declareapi{pmix_hdlr_reg_cbfunc_t}

%%%%
\summary

Callback function for calls to register handlers, e.g., event notification and IOF requests.

%%%%
\format

\versionMarker{3.0}
\cspecificstart
\begin{codepar}
typedef void (*pmix_hdlr_reg_cbfunc_t)
    (pmix_status_t status,
     size_t refid,
     void *cbdata);
\end{codepar}
\cspecificend

\begin{arglist}
\argin{status}{\refconst{PMIX_SUCCESS} or an appropriate error constant (\refstruct{pmix_status_t})}
\argin{refid}{reference identifier assigned to the handler by PMIx, used to deregister the handler (\code{size_t})}
\argin{cbdata}{object provided to the registration call (pointer)}
\end{arglist}

%%%%
\descr

Callback function for calls to register handlers, e.g., event notification and IOF requests.


%%%%%%%%%%%%%%%%%%%%%%%%%%%%%%%%%%%%%%%%%%%%%%%%%
%%%%%%%%%%%%%%%%%%%%%%%%%%%%%%%%%%%%%%%%%%%%%%%%%
\section{PMIx Datatype Value String Representations}

Provide a string representation for several types of values.
Note that the provided string is statically defined and must NOT be \code{free}'d.

%%%%
\summary
\declareapi{PMIx_Error_string}

String representation of a \refstruct{pmix_status_t}.

\versionMarker{1.0}
\cspecificstart
\begin{codepar}
const char*
PMIx_Error_string(pmix_status_t status);
\end{codepar}
\cspecificend

%%%%
\summary
\declareapi{PMIx_Proc_state_string}

String representation of a \refstruct{pmix_proc_state_t}.

\versionMarker{2.0}
\cspecificstart
\begin{codepar}
const char*
PMIx_Proc_state_string(pmix_proc_state_t state);
\end{codepar}
\cspecificend

%%%%
\summary
\declareapi{PMIx_Scope_string}

String representation of a \refstruct{pmix_scope_t}.

\versionMarker{2.0}
\cspecificstart
\begin{codepar}
const char*
PMIx_Scope_string(pmix_scope_t scope);
\end{codepar}
\cspecificend

%%%%
\summary
\declareapi{PMIx_Persistence_string}

String representation of a \refstruct{pmix_persistence_t}.

\versionMarker{2.0}
\cspecificstart
\begin{codepar}
const char*
PMIx_Persistence_string(pmix_persistence_t persist);
\end{codepar}
\cspecificend

%%%%
\summary
\declareapi{PMIx_Data_range_string}

String representation of a \refstruct{pmix_data_range_t}.

\versionMarker{2.0}
\cspecificstart
\begin{codepar}
const char*
PMIx_Data_range_string(pmix_data_range_t range);
\end{codepar}
\cspecificend

%%%%
\summary
\declareapi{PMIx_Info_directives_string}

String representation of a \refstruct{pmix_info_directives_t}.

\versionMarker{2.0}
\cspecificstart
\begin{codepar}
const char*
PMIx_Info_directives_string(pmix_info_directives_t directives);
\end{codepar}
\cspecificend

%%%%
\summary
\declareapi{PMIx_Data_type_string}

String representation of a \refstruct{pmix_data_type_t}.

\versionMarker{2.0}
\cspecificstart
\begin{codepar}
const char*
PMIx_Data_type_string(pmix_data_type_t type);
\end{codepar}
\cspecificend

%%%%
\summary
\declareapi{PMIx_Alloc_directive_string}

String representation of a \refstruct{pmix_alloc_directive_t}.

\versionMarker{2.0}
\cspecificstart
\begin{codepar}
const char*
PMIx_Alloc_directive_string(pmix_alloc_directive_t directive);
\end{codepar}
\cspecificend

%%%%
\summary
\declareapi{PMIx_IOF_channel_string}

String representation of a \refstruct{pmix_iof_channel_t}.

\versionMarker{3.0}
\cspecificstart
\begin{codepar}
const char*
PMIx_IOF_channel_string(pmix_iof_channel_t channel);
\end{codepar}
\cspecificend

%%%%
\summary
\declareapi{PMIx_Job_state_string}

String representation of a \refstruct{pmix_job_state_t}.

\versionMarker{4.0}
\cspecificstart
\begin{codepar}
const char*
PMIx_Job_state_string(pmix_job_state_t state);
\end{codepar}
\cspecificend

%%%%
\summary
\declareapi{PMIx_Get_attribute_string}

String representation of a \ac{PMIx} attribute.

\versionMarker{4.0}
\cspecificstart
\begin{codepar}
const char*
PMIx_Get_attribute_string(char *attributename);
\end{codepar}
\cspecificend

%%%%
\summary
\declareapi{PMIx_Get_attribute_name}

Return the \ac{PMIx} attribute name corresponding to the given attribute string.

\versionMarker{4.0}
\cspecificstart
\begin{codepar}
const char*
PMIx_Get_attribute_name(char *attributestring);
\end{codepar}
\cspecificend

%%%%
\summary
\declareapi{PMIx_Link_state_string}

String representation of a \refstruct{pmix_link_state_t}.

\versionMarker{4.0}
\cspecificstart
\begin{codepar}
const char*
PMIx_Link_state_string(pmix_link_state_t state);
\end{codepar}
\cspecificend

%%%%
\summary
\declareapi{PMIx_Device_type_string}

String representation of a \refstruct{pmix_device_type_t}.

\versionMarker{4.0}
\cspecificstart
\begin{codepar}
const char*
PMIx_Device_type_string(pmix_device_type_t type);
\end{codepar}
\cspecificend


%%%%%%%%%%%%%%%%%%%%%%%%%%%%%%%%%%%%%%%%%%%%%%%%%


    % Client Initialization & Finalization
    %%%%%%%%%%%%%%%%%%%%%%%%%%%%%%%%%%%%%%%%%%%%%%%%%
% Chapter: Initialization & Finalization
%%%%%%%%%%%%%%%%%%%%%%%%%%%%%%%%%%%%%%%%%%%%%%%%%
\chapter{Initialization and Finalization}
\label{chap:api_init}

% RALPH

The \ac{PMIx} library is required to be initialized and finalized around the usage of most of the \acp{API}.
The \acp{API} that may be used outside of the initialized and finalized region are noted.
All other \acp{API} must be used inside this region.

There are three sets of initialization and finalization functions depending upon the role of the process in the \ac{PMIx} universe.
Each of these functional sets are described in this chapter. Note that a process can only call \textit{one} of the
init/finalize functional pairs - e.g., a process that calls the client init function cannot also call the tool or server
init functions, and must call the corresponding client finalize.

\adviceuserstart
Processes that initialize as a server or tool automatically are given access to all client \acp{API}. Server initialization
includes setting up the infrastructure to support local clients - thus, it necessarily includes overhead and an increased
memory footprint. Tool initialization automatically searches for a server to which it can connect --- if declared as a
\textit{launcher}, the \ac{PMIx} library sets up the required ``hooks'' for other tools (e.g., debuggers) to attach to it.
\adviceuserend


%%%%%%%%%%%%%%%%%%%%%%%%%%%%%%%%%%%%%%%%%%%%%%
%%%%%%%%%%%%%%%%%%%%%%%%%%%%%%%%%%%%%%%%%%%%%%
\section{Query}
\label{chap:api_init:general}

The APIs defined in this section can be used by any PMIx process, regardless of their role in the PMIx universe.

%%%%%%%%%%%
\subsection{\code{PMIx_Initialized}}
\declareapi{PMIx_Initialized}

%%%%
\format

\cspecificstart
\begin{codepar}
int PMIx_Initialized(void)
\end{codepar}
\cspecificend

A value of \code{1} (true) will be returned if the PMIx library has been initialized, and \code{0} (false) otherwise.

\rationalestart
The return value is an integer for historical reasons as that was the signature of prior PMI libraries.
\rationaleend

%%%%
\descr

Check to see if the PMIx library has been initialized using any of the init functions:
\refapi{PMIx_Init}, \refapi{PMIx_server_init}, or \refapi{PMIx_tool_init}.

%%%%%%%%%%%
\subsection{\code{PMIx_Get_version}}
\declareapi{PMIx_Get_version}

%%%%
\summary

Get the PMIx version information.

%%%%
\format

\cspecificstart
\begin{codepar}
const char* PMIx_Get_version(void)
\end{codepar}
\cspecificend

%%%%
\descr

Get the \ac{PMIx} version string.
Note that the provided string is statically defined and must \textit{not} be free'd.

%%%%%%%%%%%%%%%%%%%%%%%%%%%%%%%%%%%%%%%%%%%%%%
%%%%%%%%%%%%%%%%%%%%%%%%%%%%%%%%%%%%%%%%%%%%%%
\section{Client Initialization and Finalization}
\label{chap:api_init:client}

Initialization and finalization routines for \ac{PMIx} clients.

%%%%%%%%%%%
\subsection{\code{PMIx_Init}}
\declareapi{PMIx_Init}

%%%%
\summary

Initialize the \ac{PMIx} client.

%%%%
\format

\cspecificstart
\begin{codepar}
pmix_status_t
PMIx_Init(pmix_proc_t *proc,
          pmix_info_t info[], size_t ninfo)
\end{codepar}
\cspecificend

\begin{arglist}
\arginout{proc}{proc structure (handle)}
\argin{info}{Array of \refattr{pmix_info_t} structures (array of handles)}
\argin{ninfo}{Number of element in the \refarg{info} array (\code{size_t})}
\end{arglist}

Returns \refconst{PMIX_SUCCESS} or a negative value corresponding to a \ac{PMIx} error constant.

\priattr
The following attributes are supported in the \ac{PRI}:

\pasteAttributeItem{PMIX_EVENT_BASE}
\pasteAttributeItemBegin{PMIX_GDS_MODULE} This attribute controls only the selection of GDS module for internal use by the process. Module selection for interacting with the server is performed dynamically during the connection process.
\pasteAttributeItemEnd{}

%%%%
\descr

Initialize the \ac{PMIx} client, returning the process identifier assigned to this client's application in the provided \refstruct{pmix_proc_t} struct.
Passing a value of \code{NULL} for this parameter is allowed if the user wishes solely to initialize the \ac{PMIx} system and does not require return of the identifier at that time.

When called, the \ac{PMIx} client shall check for the required connection information of the local \ac{PMIx} server and establish the connection.
If the information is not found, or the server connection fails, then an appropriate error constant shall be returned.

If successful, the function shall return \refconst{PMIX_SUCCESS} and fill the \refarg{proc} structure (if provided) with the server-assigned namespace and rank of the process within the application.
In addition, all startup information provided by the resource manager shall be made available to the client process via subsequent calls to \refapi{PMIx_Get}.

The \ac{PMIx} client library shall be reference counted, and so multiple calls to \refapi{PMIx_Init} are allowed by the standard.
Thus, one way for an application process to obtain its namespace and rank is to simply call \refapi{PMIx_Init} with a non-NULL \refarg{proc} parameter.
Note that each call to \refapi{PMIx_Init} must be balanced with a call to \refapi{PMIx_Finalize} to maintain the reference count.

Each call to \refapi{PMIx_Init} may contain an array of \refstruct{pmix_info_t} structures passing directives to the \ac{PMIx} client library as per the above attributes.

Multiple calls to \refapi{PMIx_Init} shall not include conflicting directives.
The \refapi{PMIx_Init} function will return an error when directives that conflict with prior directives are encountered.


%%%%%%%%%%%
\subsection{\code{PMIx_Finalize}}
\declareapi{PMIx_Finalize}

%%%%
\summary

Finalize the PMIx client library.

%%%%
\format

\cspecificstart
\begin{codepar}
pmix_status_t
PMIx_Finalize(const pmix_info_t info[], size_t ninfo)
\end{codepar}
\cspecificend

\begin{arglist}
\argin{info}{Array of \refattr{pmix_info_t} structures (array of handles)}
\argin{ninfo}{Number of element in the \refarg{info} array (\code{size_t})}
\end{arglist}

Returns \refconst{PMIX_SUCCESS} or a negative value corresponding to a PMIx error constant.

\priattr
The following attributes are supported in the \ac{PRI}:

\pasteAttributeItem{PMIX_EMBED_BARRIER}

%%%%
\descr

Decrement the \ac{PMIx} client library reference count.
When the reference count reaches zero, the library will finalize the \ac{PMIx} client, closing the connection with the local \ac{PMIx} server and releasing all internally allocated memory.


%%%%%%%%%%%%%%%%%%%%%%%%%%%%%%%%%%%%%%%%%%%%%%
%%%%%%%%%%%%%%%%%%%%%%%%%%%%%%%%%%%%%%%%%%%%%%
\section{Tool Initialization and Finalization}
\label{chap:api_init:tool}

Initialization and finalization routines for \ac{PMIx} tools.

%%%%%%%%%%%
\subsection{\code{PMIx_tool_init}}
\declareapi{PMIx_tool_init}

%%%%
\summary

Initialize the \ac{PMIx} library for operating as a tool.

%%%%
\format

\cspecificstart
\begin{codepar}
pmix_status_t
PMIx_tool_init(pmix_proc_t *proc,
               pmix_info_t info[], size_t ninfo)
\end{codepar}
\cspecificend

\begin{arglist}
\arginout{proc}{\refstruct{pmix_proc_t} structure (handle)}
\argin{info}{Array of \refattr{pmix_info_t} structures (array of handles)}
\argin{ninfo}{Number of element in the \refarg{info} array (\code{size_t})}
\end{arglist}

Returns \refconst{PMIX_SUCCESS} or a negative value corresponding to a PMIx error constant.

\priattr
The following attributes are supported in the \ac{PRI}:

\pasteAttributeItem{PMIX_GDS_MODULE}
\pasteAttributeItem{PMIX_TOOL_NSPACE}
\pasteAttributeItem{PMIX_TOOL_RANK}
\pasteAttributeItem{PMIX_TOOL_DO_NOT_CONNECT}
\pasteAttributeItem{PMIX_CONNECT_TO_SYSTEM}
\pasteAttributeItem{PMIX_CONNECT_SYSTEM_FIRST}
\pasteAttributeItem{PMIX_SERVER_PIDINFO}
\pasteAttributeItem{PMIX_SERVER_URI}
\pasteAttributeItem{PMIX_TCP_URI}
\pasteAttributeItem{PMIX_CONNECT_RETRY_DELAY}
\pasteAttributeItem{PMIX_CONNECT_MAX_RETRIES}

%%%%
\descr

Initialize the \ac{PMIx} tool, returning the process identifier assigned to this tool in the provided \refstruct{pmix_proc_t} struct. The \refarg{info} array is used to pass user requests pertaining to the init and subsequent operations. Passing a \code{NULL} value for the array pointer is supported if no directives are desired.

If called with the \refattr{PMIX_TOOL_DO_NOT_CONNECT} attribute, the \ac{PMIx} tool library will fully initialize but not attempt to connect to a \ac{PMIx} server. The tool can connect to a server at a later point in time, if desired. In all other cases, the tool library will attempt to connect to according to the following precedence chain:

\begin{itemize}
    \item if \refattr{PMIX_SERVER_URI} or \refattr{PMIX_TCP_URI} is given, then connection will be attempted to the server at the specified \ac{URI}. Note that it is an error for both of these attributes to be specified. \refattr{PMIX_SERVER_URI} is the preferred method as it is more generalized --- \refattr{PMIX_TCP_URI} is provided for those cases where the user specifically wants to use a TCP transport for the connection and wants to error out if it isn't available or cannot succeed. The \ac{PMIx} library will return an error if connection fails --- it will not proceed to check for other connection options as the user specified a particular one to use
    \item if \refattr{PMIX_SERVER_PIDINFO} was provided, then the tool will search under the directory provided by the PMIX\_SERVER\_TMPDIR environmental variable for a rendezvous file created by the process corresponding to that \ac{PID}. The \ac{PMIx} library will return an error if the rendezvous file cannot be found, or the connection is refused by the server
    \item if \refattr{PMIX_CONNECT_TO_SYSTEM} is given, then the tool will search for a system-level rendezvous file created by a \ac{PMIx} server in the directory specified by the PMIX\_SYSTEM\_TMPDIR environmental variable. If found, then the tool will attempt to connect to it. An error is returned if the rendezvous file cannot be found or the connection is refused.
    \item if \refattr{PMIX_CONNECT_SYSTEM_FIRST} is given, then the tool will search for a system-level rendezvous file created by a \ac{PMIx} server in the directory specified by the PMIX\_SYSTEM\_TMPDIR environmental variable. If found, then the tool will attempt to connect to it. In this case, no error will be returned if the rendezvous file is not found or connection is refused --- the library will silently continue to the next option
    \item by default, the tool will search the directory tree under the directory provided by the PMIX\_SERVER\_TMPDIR environmental variable for rendezvous files of \ac{PMIx} servers, attempting to connect to each it finds until one accepts the connection. If no rendezvous files are found, or all contacted servers refuse connection, then the library will return an error.
\end{itemize}

If successful, the function will return \refconst{PMIX_SUCCESS} and will fill the provided structure (if provided) with the server-assigned namespace and rank of the tool. Note that each connection attempt in the above precedence chain will retry (with delay between each retry) a number of times according to the values of the corresponding attributes. Default is no retries.

Note that the \ac{PMIx} tool library is referenced counted, and so multiple calls to \refapi{PMIx_tool_init} are allowed.
Thus, one way to obtain the namespace and rank of the process is to simply call \refapi{PMIx_tool_init} with a non-NULL parameter.


%%%%%%%%%%%
\subsection{\code{PMIx_tool_finalize}}
\declareapi{PMIx_tool_finalize}

%%%%
\summary

Finalize the \ac{PMIx} library for a tool connection.

%%%%
\format

\cspecificstart
\begin{codepar}
pmix_status_t
PMIx_tool_finalize(void)
\end{codepar}
\cspecificend

Returns \refconst{PMIX_SUCCESS} or a negative value corresponding to a PMIx error constant.

%%%%
\descr

Finalize the PMIx tool library, closing the connection to the server.
An error code will be returned if, for some reason, the connection cannot be cleanly terminated --- in this case, the connection is dropped.


%%%%%%%%%%%%%%%%%%%%%%%%%%%%%%%%%%%%%%%%%%%%%%
%%%%%%%%%%%%%%%%%%%%%%%%%%%%%%%%%%%%%%%%%%%%%%
\section{Server Initialization and Finalization}
\label{chap:api_init:server}

Initialization and finalization routines for \ac{PMIx} servers.

%%%%%%%%%%%
\subsection{\code{PMIx_server_init}}
\declareapi{PMIx_server_init}

%%%%
\summary

Initialize the \ac{PMIx} server.

%%%%
\format

\cspecificstart
\begin{codepar}
pmix_status_t
PMIx_server_init(pmix_server_module_t *module,
                 pmix_info_t info[], size_t ninfo)
\end{codepar}
\cspecificend

\begin{arglist}
\arginout{module}{\refstruct{pmix_server_module_t} structure (handle)}
\argin{info}{Array of \refattr{pmix_info_t} structures (array of handles)}
\argin{ninfo}{Number of elements in the \refarg{info} array (\code{size_t})}
\end{arglist}

Returns \refconst{PMIX_SUCCESS} or a negative value corresponding to a PMIx error constant.

\priattr
The following attributes are supported in the \ac{PRI}:

\pasteAttributeItem{PMIX_SERVER_NSPACE}
\pasteAttributeItem{PMIX_SERVER_RANK}
\pasteAttributeItem{PMIX_SERVER_TMPDIR}
\pasteAttributeItem{PMIX_SYSTEM_TMPDIR}
\pasteAttributeItem{PMIX_SERVER_TOOL_SUPPORT}
\pasteAttributeItem{PMIX_SERVER_SYSTEM_SUPPORT}
\pasteAttributeItem{PMIX_TCP_IF_INCLUDE}
\pasteAttributeItem{PMIX_TCP_IPV4_PORT}
\pasteAttributeItem{PMIX_TCP_IPV6_PORT}
\pasteAttributeItem{PMIX_TCP_DISABLE_IPV4}
\pasteAttributeItem{PMIX_TCP_DISABLE_IPV6}
\pasteAttributeItem{PMIX_SERVER_REMOTE_CONNECTIONS}
\pasteAttributeItem{PMIX_TCP_REPORT_URI}
\pasteAttributeItem{PMIX_USOCK_DISABLE}


%%%%
\descr

Initialize the server support library, and provide a pointer to a \refapi{pmix_server_module_t} structure containing the caller's callback functions.
The array of \refstruct{pmix_info_t} structs is used to pass additional info that may be required by the server when initializing.
For example, it may include the \refconst{PMIX_SERVER_TOOL_SUPPORT} key, thereby indicating that the daemon is willing to accept connection requests from tools.

\adviceuserstart
Providing a value of \code{NULL} for the \refarg{module} argument is not permitted - the host must support at least one server callback function.
\adviceuserend

%%%%%%%%%%%
\subsection{\code{PMIx_server_finalize}}
\declareapi{PMIx_server_finalize}

%%%%
\summary

Finalize the PMIx server library.

%%%%
\format

\cspecificstart
\begin{codepar}
pmix_status_t
PMIx_server_finalize(void)
\end{codepar}
\cspecificend

Returns \refconst{PMIX_SUCCESS} or a negative value corresponding to a PMIx error constant.

%%%%
\descr

Finalize the server support library, terminating all connections to attached tools and any local clients.
All memory usage is released.

%%%%%%%%%%%%%%%%%%%%%%%%%%%%%%%%%%%%%%%%%%%%%%%%%


    % Key/Value Management
    %  - put, get, commit, fence, (un)publish, lookup
    %%%%%%%%%%%%%%%%%%%%%%%%%%%%%%%%%%%%%%%%%%%%%%%%%
% Chapter: Synchronization and Data Access Operations
%%%%%%%%%%%%%%%%%%%%%%%%%%%%%%%%%%%%%%%%%%%%%%%%%
\chapter{Synchronization and Data Access Operations}
\label{chap:api_sync_acc}

Applications may need to synchronize their operations at various points in
their execution. Depending on a variety of factors (e.g., the programming
model and where the synchronization point lies), the application may choose to
execute the operation using \ac{PMIx}. This is particularly useful in
situations where communication by other means is not yet available since
\ac{PMIx} relies on the host environment's infrastructure for such operations.

Synchronization operations also offer an opportunity for processes to exchange
data at a known point in their execution. Where required, this can include
information on communication endpoints for subsequent wireup of various
messaging protocols.

This chapter covers both the synchronization and data retrieval functions
provided under the \ac{PMIx} Standard.

%%%%%%%%%%%%%%%%%%%%%%%%%%%%%%%%%%%%%%%%%%%%%%%%%
%%%%%%%%%%%%%%%%%%%%%%%%%%%%%%%%%%%%%%%%%%%%%%%%%
\section{\code{PMIx_Fence}}
\declareapi{PMIx_Fence}

%%%%
\summary

Execute a blocking barrier across the processes identified in the specified array, collecting information posted via \refapi{PMIx_Put} as directed.

%%%%
\format

\copySignature{PMIx_Fence}{1.0}{
pmix_status_t \\
PMIx_Fence(const pmix_proc_t procs[], size_t nprocs, \\
\hspace*{11\sigspace}const pmix_info_t info[], size_t ninfo);
}

\begin{arglist}
\argin{procs}{Array of \refstruct{pmix_proc_t} structures (array of handles)}
\argin{nprocs}{Number of elements in the \refarg{procs} array (integer)}
\argin{info}{Array of info structures (array of handles)}
\argin{ninfo}{Number of elements in the \refarg{info} array (integer)}
\end{arglist}

\returnsimple

\reqattrstart
The following attributes are required to be supported by all \ac{PMIx} libraries:

\pasteAttributeItem{PMIX_COLLECT_DATA}
\pasteAttributeItem{PMIX_COLLECT_GENERATED_JOB_INFO}

\reqattrend

\optattrstart
The following attributes are optional for \ac{PMIx} implementations:

\pasteAttributeItem{PMIX_ALL_CLONES_PARTICIPATE}


The following attributes are optional for host environments:

\pasteAttributeItem{PMIX_TIMEOUT}

\optattrend

%%%%
\descr

Passing a \code{NULL} pointer as the \refarg{procs} parameter indicates that the fence is to span all processes in the client's namespace.
Each provided \refstruct{pmix_proc_t} struct can pass \refconst{PMIX_RANK_WILDCARD} to indicate that all processes in the given namespace are participating.

The \refarg{info} array is used to pass user directives regarding the behavior of the fence operation. Note that for scalability reasons, the default behavior for \refapi{PMIx_Fence} is to not collect data posted by the operation's participants.

\adviceimplstart
\refapi{PMIx_Fence} and its non-blocking form are both \emph{collective} operations. Accordingly, the \ac{PMIx} server library is required to aggregate participation by local clients, passing the request to the host environment once all local participants have executed the \ac{API}.
\adviceimplend

\advicermstart
The host will receive a single call for each collective operation. It is the responsibility of the host to identify the nodes containing participating processes, execute the collective across all participating nodes, and notify the local \ac{PMIx} server library upon completion of the global collective.
\advicermend


%%%%%%%%%%%%%%%%%%%%%%%%%%%%%%%%%%%%%%%%%%%%%%%%%
%%%%%%%%%%%%%%%%%%%%%%%%%%%%%%%%%%%%%%%%%%%%%%%%%
\section{\code{PMIx_Fence_nb}}
\declareapi{PMIx_Fence_nb}

%%%%
\summary

Execute a nonblocking \refapi{PMIx_Fence} across the processes identified in the specified array of processes, collecting information posted via \refapi{PMIx_Put} as directed.

%%%%
\format

\copySignature{PMIx_Fence_nb}{1.0}{
pmix_status_t \\
PMIx_Fence_nb(const pmix_proc_t procs[], size_t nprocs, \\
\hspace*{14\sigspace}const pmix_info_t info[], size_t ninfo, \\
\hspace*{14\sigspace}pmix_op_cbfunc_t cbfunc, void *cbdata);
}

\begin{arglist}
\argin{procs}{Array of \refstruct{pmix_proc_t} structures (array of handles)}
\argin{nprocs}{Number of elements in the \refarg{procs} array (integer)}
\argin{info}{Array of info structures (array of handles)}
\argin{ninfo}{Number of elements in the \refarg{info} array (integer)}
\argin{cbfunc}{Callback function (function reference)}
\argin{cbdata}{Data to be passed to the callback function (memory reference)}
\end{arglist}

\returnsimplenb

\returnstart
\begin{itemize}
    \item \refconst{PMIX_OPERATION_SUCCEEDED}, indicating that the request was immediately processed and returned \textit{success} - the \refarg{cbfunc} will \textit{not} be called. This can occur if the collective involved only processes on the local node.
\end{itemize}
\returnend

\reqattrstart
The following attributes are required to be supported by all \ac{PMIx} libraries:

\pasteAttributeItem{PMIX_COLLECT_DATA}
\pasteAttributeItem{PMIX_COLLECT_GENERATED_JOB_INFO}

\reqattrend

\optattrstart
The following attributes are optional for \ac{PMIx} implementations:

\pasteAttributeItem{PMIX_ALL_CLONES_PARTICIPATE}


The following attributes are optional for host environments that support this operation:

\pasteAttributeItem{PMIX_TIMEOUT}

\optattrend

%%%%
\descr

Nonblocking version of the \refapi{PMIx_Fence} routine. See the \refapi{PMIx_Fence} description for further details.

%%%%%%%%%%%%%%%%%%%%%%%%%%%%%%%%%%%%%%%%%%%%%%%%%
\subsection{Fence-related attributes}

The following attributes are defined specifically to support the fence operation:

%
\declareAttribute{PMIX_COLLECT_DATA}{"pmix.collect"}{bool}{
Collect all data posted by the participants using \refapi{PMIx_Put} that
has been committed via \refapi{PMIx_Commit}, making the collection locally
available to each participant at the end of the operation. By default, this will include all job-level information that was locally generated by \ac{PMIx} servers unless excluded using the \refattr{PMIX_COLLECT_GENERATED_JOB_INFO} attribute.
}
%
\declareAttributeNEW{PMIX_COLLECT_GENERATED_JOB_INFO}{"pmix.collect.gen"}{bool}{
Collect all job-level information (i.e., reserved keys) that was locally generated by \ac{PMIx} servers. Some job-level information (e.g., distance between processes and fabric devices) is best determined on a distributed basis as it primarily pertains to local processes. Should remote processes need to access the information, it can either be obtained collectively using the \refapi{PMIx_Fence} operation with this directive, or can be retrieved one peer at a time using \refapi{PMIx_Get} without first having performed the job-wide collection.
}
%
\declareAttributeNEW{PMIX_ALL_CLONES_PARTICIPATE}{"pmix.clone.part"}{bool}{
All \refterm{clones} of the calling process must participate in the collective operation.
}


%%%%%%%%%%%%%%%%%%%%%%%%%%%%%%%%%%%%%%%%%%%%%%%%%
%%%%%%%%%%%%%%%%%%%%%%%%%%%%%%%%%%%%%%%%%%%%%%%%%
\section{\code{PMIx_Get}}
\declareapi{PMIx_Get}

%%%%
\summary

Retrieve a key/value pair from the client's namespace.

%%%%
\format

\copySignature{PMIx_Get}{1.0}{
pmix_status_t \\
PMIx_Get(const pmix_proc_t *proc, const pmix_key_t key, \\
\hspace*{9\sigspace}const pmix_info_t info[], size_t ninfo, \\
\hspace*{9\sigspace}pmix_value_t **val);
}

\begin{arglist}
\argin{proc}{Process identifier - a \code{NULL} value may be used in place of the caller's ID (handle)}
\argin{key}{Key to retrieve (\refstruct{pmix_key_t})}
\argin{info}{Array of info structures (array of handles)}
\argin{ninfo}{Number of elements in the \refarg{info} array (integer)}
\argout{val}{value (handle)}
\end{arglist}

A successful return indicates that the requested data has been returned in the manner requested (.e.g., in a provided static memory location ).

\returnstart
\begin{itemize}
\item \refconst{PMIX_ERR_BAD_PARAM} A bad parameter was passed to the function call - e.g., the request included the \refattr{PMIX_GET_STATIC_VALUES} directive, but the provided storage location was \code{NULL}
\item \refconst{PMIX_ERR_EXISTS_OUTSIDE_SCOPE} The requested key exists, but was posted in a \emph{scope} (see Section \ref{api:nres:scope}) that does not include the requester.
\item \refconst{PMIX_ERR_NOT_FOUND} The requested data was not available.
\end{itemize}
\returnend

\reqattrstart
The following attributes are required to be supported by all \ac{PMIx} libraries:

\pasteAttributeItem{PMIX_OPTIONAL}
\pasteAttributeItem{PMIX_IMMEDIATE}
\pasteAttributeItem{PMIX_DATA_SCOPE}
\pasteAttributeItem{PMIX_SESSION_INFO}
\pasteAttributeItem{PMIX_JOB_INFO}
\pasteAttributeItem{PMIX_APP_INFO}
\pasteAttributeItem{PMIX_NODE_INFO}
\pasteAttributeItem{PMIX_GET_STATIC_VALUES}
\pasteAttributeItem{PMIX_GET_POINTER_VALUES}
\pasteAttributeItem{PMIX_GET_REFRESH_CACHE}

\reqattrend

\optattrstart
The following attributes are optional for host environments:

\pasteAttributeItem{PMIX_TIMEOUT}

\optattrend

%%%%
\descr

Retrieve information for the specified \refarg{key} associated with the process identified in the given \refstruct{pmix_proc_t}. See Chapters \ref{chap:api_rsvd_keys} and \ref{chap:nrkeys} for details on rules governing retrieval of information. Information will be returned according to provided directives:

\begin{itemize}
    \item In the absence of any directive, the returned \refstruct{pmix_value_t} shall be an allocated memory object. The caller is responsible for releasing the object when done.
    \item If \refattr{PMIX_GET_POINTER_VALUES} is given, then the function shall return a pointer to a \refstruct{pmix_value_t} in the \ac{PMIx} library's memory that contains the requested information.
    \item If \refattr{PMIX_GET_STATIC_VALUES} is given, then the function shall return the information in the provided \refstruct{pmix_value_t} pointer. In this case, the caller must provide storage for the structure and pass the pointer to that storage in the \refarg{val} parameter. If the implementation cannot return a static value, then the call to \refapi{PMIx_Get} must return the \refconst{PMIX_ERR_NOT_SUPPORTED} status.
\end{itemize}

This is a blocking operation - the caller will block until the retrieval rules of Chapters \ref{chap:api_rsvd_keys} or \ref{chap:nrkeys} are met.

The \refarg{info} array is used to pass user directives regarding the get operation.

%%%%%%%%%%%%%%%%%%%%%%%%%%%%%%%%%%%%%%%%%%%%%%%%%
\subsection{\code{PMIx_Get_nb}}
\declareapi{PMIx_Get_nb}

%%%%
\summary

Nonblocking \refapi{PMIx_Get} operation.

%%%%
\format

\copySignature{PMIx_Get_nb}{1.0}{
pmix_status_t \\
PMIx_Get_nb(const pmix_proc_t *proc, const char key[], \\
\hspace*{12\sigspace}const pmix_info_t info[], size_t ninfo, \\
\hspace*{12\sigspace}pmix_value_cbfunc_t cbfunc, void *cbdata);
}

\begin{arglist}
\argin{proc}{Process identifier - a \code{NULL} value may be used in place of the caller's ID (handle)}
\argin{key}{Key to retrieve (string)}
\argin{info}{Array of info structures (array of handles)}
\argin{ninfo}{Number of elements in the \refarg{info} array (integer)}
\argin{cbfunc}{Callback function (function reference)}
\argin{cbdata}{Data to be passed to the callback function (memory reference)}
\end{arglist}

\returnsimplenb

If executed, the status returned in the provided callback function will be one of the following constants:

\begin{itemize}
\item \refconst{PMIX_SUCCESS} The requested data has been returned.
\item \refconst{PMIX_ERR_EXISTS_OUTSIDE_SCOPE} The requested key exists, but was posted in a \emph{scope} (see Section \ref{api:nres:scope}) that does not include the requester.
\item \refconst{PMIX_ERR_NOT_FOUND} The requested data was not available.
\item a non-zero \ac{PMIx} error constant indicating a reason for the request's failure.
\end{itemize}

\reqattrstart
The following attributes are required to be supported by all \ac{PMIx} libraries:

\pasteAttributeItem{PMIX_OPTIONAL}
\pasteAttributeItem{PMIX_IMMEDIATE}
\pasteAttributeItem{PMIX_DATA_SCOPE}
\pasteAttributeItem{PMIX_SESSION_INFO}
\pasteAttributeItem{PMIX_JOB_INFO}
\pasteAttributeItem{PMIX_APP_INFO}
\pasteAttributeItem{PMIX_NODE_INFO}
\pasteAttributeItem{PMIX_GET_POINTER_VALUES}
\pasteAttributeItem{PMIX_GET_REFRESH_CACHE}

\divider

The following attributes are required for host environments that support this operation:

\pasteAttributeItem{PMIX_WAIT}

\reqattrend

\optattrstart
The following attributes are optional for host environments that support this operation:

\pasteAttributeItem{PMIX_TIMEOUT}

\optattrend

%%%%
\descr

The callback function will be executed once the retrieval rules of Chapters \ref{chap:api_rsvd_keys} or \ref{chap:nrkeys} are met.
See \refapi{PMIx_Get} for a full description. Note that the non-blocking form of this function cannot support the \refattr{PMIX_GET_STATIC_VALUES} attribute as the user cannot pass in the required pointer to storage for the result.


%%%%%%%%%%%%%%%%%%%%%%%%%%%%%%%%%%%%%%%%%%%%%%%%%
\subsection{Retrieval attributes}
\label{chap:api_kg:attr}

The following attributes are defined for use by retrieval \acp{API}:

%
\declareAttribute{PMIX_OPTIONAL}{"pmix.optional"}{bool}{
Look only in the client's local data store for the requested value - do not request data from the \ac{PMIx} server if not found.
}
%
\declareAttribute{PMIX_IMMEDIATE}{"pmix.immediate"}{bool}{
Specified operation should immediately return an error from the \ac{PMIx} server if the requested data cannot be found - do not request it from the host \ac{RM}.
}
%
\declareAttributeNEW{PMIX_GET_POINTER_VALUES}{"pmix.get.pntrs"}{bool}{
Request that any pointers in the returned value point directly to values in the key-value store. The user \emph{must not} release any returned data pointers.
}
%
\declareAttributeNEW{PMIX_GET_STATIC_VALUES}{"pmix.get.static"}{bool}{
Request that the data be returned in the provided storage location. The caller is responsible for destructing the \refstruct{pmix_value_t} using the \refmacro{PMIX_VALUE_DESTRUCT} macro when done.
}
%
\declareAttributeNEW{PMIX_GET_REFRESH_CACHE}{"pmix.get.refresh"}{bool}{
When retrieving data for a remote process, refresh the existing local data cache for the process in case new values have been put and committed by the process since the last refresh. Local process information is assumed to be automatically updated upon posting by the process. A \code{NULL} key will cause all values associated with the process to be refreshed - otherwise, only the indicated key will be updated. A process rank of \refconst{PMIX_RANK_WILDCARD} can be used to update job-related information in dynamic environments. The user is responsible for subsequently updating refreshed values they may have cached in their own local memory.
}
%
\declareAttribute{PMIX_DATA_SCOPE}{"pmix.scope"}{pmix_scope_t}{
Scope of the data to be searched in a \refapi{PMIx_Get} call.
}
%
\declareAttribute{PMIX_TIMEOUT}{"pmix.timeout"}{int}{
Time in seconds before the specified operation should time out (zero indicating infinite) and return the \refconst{PMIX_ERR_TIMEOUT} error.
Care should be taken to avoid race conditions caused by multiple layers (client, server, and host) simultaneously timing the operation.
}
%
\declareAttribute{PMIX_WAIT}{"pmix.wait"}{int}{
Caller requests that the \ac{PMIx} server wait until at least the specified number of values are found (a value of zero indicates \emph{all} and is the default).
}


%%%%%%%%%%%%%%%%%%%%%%%%%%%%%%%%%%%%%%%%%%%%%%%%%

    %%%%%%%%%%%%%%%%%%%%%%%%%%%%%%%%%%%%%%%%%%%%%%%%%
% Chapter: Reserved Keys
%%%%%%%%%%%%%%%%%%%%%%%%%%%%%%%%%%%%%%%%%%%%%%%%%
\chapter{Reserved Keys}
\label{chap:api_rsvd_keys}


\emph{Reserved} keys are keys whose string representation begin with a prefix of
\code{"pmix"}. By definition, reserved keys are provided by the host
environment and the \ac{PMIx} server, and are required to be available at client
start of execution. \ac{PMIx} clients and tools are therefore prohibited from
posting reserved keys.

%% TODO: Should we note that although values are required to be available at
%% client start of execution, that some values can change during execution?

Host environments may opt to define non-standardized reserved keys. 
All reserved keys, whether standardized or non-standardized, follow the same retrieval rules.
Users
are advised to check both the local \ac{PMIx} implementation and host environment documentation
for a list of any non-standardized reserved keys they must avoid, and to learn of any non-standard keys that may require special handling.


%%%%%%%%%%%%%%%%%%%%%%%%%%%%%%%%%%%%%%%%%%%%%%%%%
%%%%%%%%%%%%%%%%%%%%%%%%%%%%%%%%%%%%%%%%%%%%%%%%%
\section{Data realms}
\label{api:struct:attributes:retrieval}

\ac{PMIx} information spans a wide range of sources. In some cases,
there are multiple overlapping sources for the same type of data - e.g., the
session, job, and application can each provide information on the number of
nodes involved in their respective area. In order to resolve the ambiguity,
a \declaretermAlt{data realm}{realm,realms,data realm,data realms}
is used to identify the scope to which the referenced data applies. Thus, a reference
to an attribute that isn't specific to a realm (e.g., the
\refattr{PMIX_NUM_NODES} attribute) must be accompanied by a corresponding
attribute identifying the realm to which the request pertains if it differs
from the default.

\ac{PMIx} defines five \emph{data realms} to resolve the ambiguities, as
captured in the following attributes used in \refapi{PMIx_Get} for retrieving
information from each of the realms:

%
\declareAttribute{PMIX_SESSION_INFO}{"pmix.ssn.info"}{bool}{
Return information regarding the session realm of the target process.
}
%
\declareAttribute{PMIX_JOB_INFO}{"pmix.job.info"}{bool}{
Return information regarding the job realm corresponding to the namespace in
the target process' identifier.
}
%
\declareAttribute{PMIX_APP_INFO}{"pmix.app.info"}{bool}{
Return information regarding the application realm to which the target process
belongs - the namespace of the target process serves to identify the job
containing the target application. If information about an application other
than the one containing the target process is desired, then the attribute array
must contain a \refattr{PMIX_APPNUM} attribute identifying the desired target
application. This is useful in cases where there are multiple applications and the mapping of processes to applications is unclear.
}
%
\declareAttribute{PMIX_PROC_INFO}{"pmix.proc.info"}{bool}{
Return information regarding the target process. This attribute is technically
not required as the \refapi{PMIx_Get} \ac{API} specifically identifies the
target process in its parameters. However, it is included here for
completeness.
}
%
\declareAttribute{PMIX_NODE_INFO}{"pmix.node.info"}{bool}{
Return information from the node realm regarding the node upon which the specified process is executing. If information about
a node other than the one containing the specified process is desired, then
the attribute array must also contain either the \refattr{PMIX_NODEID} or
\refattr{PMIX_HOSTNAME} attribute identifying the desired target. This is useful for requesting information about a specific node even if the identity of processes running on that node are not known.
}

\adviceuserstart
If information about a session other than the one containing the requesting
process is desired, then the attribute array must contain a
\refattr{PMIX_SESSION_ID} attribute identifying the desired target session.
This is required as many environments only guarantee unique namespaces within a
session, and not across sessions.
\adviceuserend

Determining the target within a realm varies between realms and is explained in detail in the realm
descriptions below.  Note that several attributes can be either queried as a key or set as an attribute to
specify the target within a realm.  The attributes \refattr{PMIX_SESSION_ID}, \refattr{PMIX_NSPACE} and \refattr{PMIX_APPNUM} can be used in both ways.

%%%%%%%%%%%%%%%%%%%%%%%%%%%%%%%%%%%%%%%%%%%%%%%%%
\subsection{Session realm attributes}

If information about a session other than the one containing the requesting
process is desired, then the \refarg{info} array passed to \refapi{PMIx_Get}
must contain a \refattr{PMIX_SESSION_ID} attribute identifying the desired
target session. This is required as many environments only guarantee unique
namespaces within a session, and not across sessions.

Note that the \refarg{proc} argument of \refapi{PMIx_Get} is ignored when
referencing session-related information.

The following keys, by default, request session-level information.
They will return information about the caller's session unless a \refattr{PMIX_SESSION_ID} attribute is specified in the \refarg{info} array passed to \refapi{PMIx_Get}:

%
\declareAttribute{PMIX_CLUSTER_ID}{"pmix.clid"}{char*}{
A string name for the cluster this allocation is on.
}
%
\declareAttribute{PMIX_UNIV_SIZE}{"pmix.univ.size"}{uint32_t}{
Maximum number of process that can be simultaneously executing in 
a session. Note that this attribute is equivalent to
the \refattr{PMIX_MAX_PROCS} attribute for the \refterm{session} realm - it 
is included in the \ac{PMIx} Standard for historical reasons.
}
%
\declareAttribute{PMIX_TMPDIR}{"pmix.tmpdir"}{char*}{
Full path to the top-level temporary directory assigned to the session.
}
%
\declareAttribute{PMIX_TDIR_RMCLEAN}{"pmix.tdir.rmclean"}{bool}{
The Resource Manager will remove any directories or files it creates in
\refattr{PMIX_TMPDIR}.
}
%
\declareAttributeNEW{PMIX_HOSTNAME_KEEP_FQDN}{"pmix.fqdn"}{bool}{
\acp{FQDN} are being retained by the \ac{PMIx} library.
}

%
\declareAttribute{PMIX_RM_NAME}{"pmix.rm.name"}{char*}{
String name of the \ac{RM}.
}
%
\declareAttribute{PMIX_RM_VERSION}{"pmix.rm.version"}{char*}{
\ac{RM} version string.
}

\vspace{\baselineskip}

The following session-related keys default to the realms described in their descriptions but can be
retrieved from the session realm by setting the \refattr{PMIX_SESSION_INFO} attribute in the \refarg{info} array passed to \refapi{PMIx_Get}:

\declareAttribute{PMIX_ALLOCATED_NODELIST}{"pmix.alist"}{char*}{
Comma-delimited list or regular expression of all nodes in the specified realm regardless of whether or not they currently host processes. Defaults to the \refterm{job} realm.
}
%
\declareAttributeNEW{PMIX_NUM_ALLOCATED_NODES}{"pmix.num.anodes"}{uint32_t}{
Number of nodes in the specified realm regardless of whether or not they currently host processes. Defaults to the \refterm{job} realm.
}
%
\declareAttribute{PMIX_MAX_PROCS}{"pmix.max.size"}{uint32_t}{
Maximum number of processes that can be simultaneously executed in the specified realm.
Typically, this is a constraint imposed by a scheduler or by user settings in a
hostfile or other resource description. Defaults to the \refterm{job} realm.
}
%
\declareAttribute{PMIX_NODE_LIST}{"pmix.nlist"}{char*}{
Comma-delimited list of nodes currently hosting processes in the specified realm. Defaults to the \refterm{job} realm.
}
%
\declareAttribute{PMIX_NUM_SLOTS}{"pmix.num.slots"}{uint32_t}{
Maximum number of processes that can simultaneously be executing in the specified realm. Note that this attribute is
the equivalent to \refattr{PMIX_MAX_PROCS} -
it is included in the \ac{PMIx} Standard for historical reasons. Defaults to the \refterm{job} realm.
}
%
\declareAttributeNEW{PMIX_NUM_NODES}{"pmix.num.nodes"}{uint32_t}{
Number of nodes currently hosting processes in the specified realm. Defaults to the \refterm{job} realm.
}

%
\declareAttribute{PMIX_NODE_MAP}{"pmix.nmap"}{char*}{
Regular expression of nodes currently hosting processes in the specified realm - see \ref{cptr:api_server:noderegex} for an explanation of its generation. Defaults to the \refterm{job} realm.
}
%
\declareAttributeNEW{PMIX_NODE_MAP_RAW}{"pmix.nmap.raw"}{char*}{
Comma-delimited list of nodes containing procs within the specified realm. Defaults to the \refterm{job} realm.
}
%
\declareAttribute{PMIX_PROC_MAP}{"pmix.pmap"}{char*}{
Regular expression describing processes on each node in the specified realm - see \ref{cptr:api_server:ppnregex} for an explanation of its generation. Defaults to the \refterm{job} realm.
}
%
\declareAttributeNEW{PMIX_PROC_MAP_RAW}{"pmix.pmap.raw"}{char*}{
Semi-colon delimited list of strings, each string containing a comma-delimited list of ranks on the corresponding node within the specified realm. Defaults to the \refterm{job} realm.
}
%
\declareAttribute{PMIX_ANL_MAP}{"pmix.anlmap"}{char*}{
Process map equivalent to \refattr{PMIX_PROC_MAP} expressed in Argonne National Laboratory's PMI-1/PMI-2 notation. Defaults to the \refterm{job} realm.
}

%%%%%%%%%%%%%%%%%%%%%%%%%%%%%%%%%%%%%%%%%%%%%%%%%
\subsection{Job realm attributes}
\label{chap:api_rsvd_keys:jrealm}

Job-related information can be retrieved by requesting a key which defaults
to the job realm or by including the \refattr{PMIX_JOB_INFO} attribute
in the \refarg{info} array passed to \refapi{PMIx_Get}.
For job-related keys the target job is specified by setting the namespace of the target
job in the \refarg{proc} argument and specifying a rank of \refconst{PMIX_RANK_WILDCARD} 
in the \refarg{proc} argument passed to \refapi{PMIx_Get}.

% Removing because it is only true until we add a key that defaults to another
% realm, but can be used for job realms.  Then this attribute becomes required.
%If desired for code clarity, the caller can also
%include the \refattr{PMIX_JOB_INFO} attribute in the \refarg{info} array,
%though this is not required.

If information is requested about a namespace in
a session other than the one containing the requesting process, then the
\refarg{info} array must contain a \refattr{PMIX_SESSION_ID} attribute
identifying the desired target session. This is required as
many environments only guarantee unique namespaces within a session, and not
across sessions.

The following keys, by default, request job-level information:
They will return information about the job indicated in \refarg{proc}:

%
\declareAttribute{PMIX_JOBID}{"pmix.jobid"}{char*}{
Job identifier assigned by the scheduler to the specified job - may be identical to the namespace, but is often a numerical value expressed as a string (e.g., \code{"12345.3"}).
}
%
\declareAttribute{PMIX_NPROC_OFFSET}{"pmix.offset"}{pmix_rank_t}{
Starting global rank of the specified job.  The returned value is the same as the value of \refattr{PMIX_GLOBAL_RANK} of rank 0 of the specified job.
}
%
\pasteAttributeItemBegin{PMIX_MAX_PROCS}In this context, this is the maximum number of processes that can be simultaneously executed in the specified job, which may be a subset of the number allocated to the overall session.
\pasteAttributeItemEnd{}
%
\pasteAttributeItemBegin{PMIX_NUM_SLOTS}In this context, this is the maximum number of
process that can be simultaneously executing within the specified job, which may be a subset of the number
allocated to the overall session. Jobs may reserve a subset of their assigned
maximum processes for dynamic operations such as \refapi{PMIx_Spawn}.
\pasteAttributeItemEnd{}
%
\pasteAttributeItemBegin{PMIX_NUM_NODES}In this context, this is the number of
nodes currently hosting processes in the specified job, which may be a subset
of the nodes allocated to the overall session. Jobs may reserve a subset of
their assigned nodes for dynamic operations such as \refapi{PMIx_Spawn} -
i.e., not all nodes may have executing processes from this job at a given
point in time.
\pasteAttributeItemEnd{}
%
\pasteAttributeItemBegin{PMIX_NODE_MAP}In this context, this is the regular expression of nodes currently hosting processes in the specified job.
\pasteAttributeItemEnd{}
%
\pasteAttributeItemBegin{PMIX_NODE_LIST}In this context, this is the comma-delimited list of nodes currently hosting processes in the specified job.
\pasteAttributeItemEnd{}
%
\pasteAttributeItemBegin{PMIX_PROC_MAP}In this context, this is the regular expression describing processes on each node in the specified job.
\pasteAttributeItemEnd{}
%
\pasteAttributeItemBegin{PMIX_ANL_MAP}In this context, this is the
process mapping in Argonne National Laboratory's PMI-1/PMI-2 notation of the processes in the specified job.
\pasteAttributeItemEnd{}
%
\declareAttributeNEW{PMIX_CMD_LINE}{"pmix.cmd.line"}{char*}{
Command line used to execute the specified job (e.g., "mpirun -n 2 --map-by foo ./myapp : -n 4 ./myapp2").
}
%
\declareAttribute{PMIX_NSDIR}{"pmix.nsdir"}{char*}{
Full path to the temporary directory assigned to the specified job, under \refattr{PMIX_TMPDIR}.
}
%
\declareAttribute{PMIX_JOB_SIZE}{"pmix.job.size"}{uint32_t}{
Total number of processes in the specified job across all contained applications. Note that this value can be different from \refattr{PMIX_MAX_PROCS}. For example, users may choose to subdivide an allocation (running several jobs in parallel within it), and dynamic programming models may support adding and removing processes from a running \refterm{job} on-the-fly. In the latter case, \ac{PMIx} events may be used to notify processes within the job that the job size has changed.
}
%
\declareAttribute{PMIX_JOB_NUM_APPS}{"pmix.job.napps"}{uint32_t}{
Number of applications in the specified job.
}

\declareAttribute{PMIX_LOCAL_PEERS}{"pmix.lpeers"}{char*}{
Comma-delimited list of ranks that are executing on the local node within the specified namespace -- shortcut for \refapi{PMIx_Resolve_peers} for the local node.
}
%
\declareAttribute{PMIX_LOCALLDR}{"pmix.lldr"}{pmix_rank_t}{
Lowest rank within the specified job on the node (defaults to current node in absence of \refattr{PMIX_HOSTNAME} or \refattr{PMIX_NODEID} qualifier).
}
%
\declareAttribute{PMIX_LOCAL_CPUSETS}{"pmix.lcpus"}{pmix_data_array_t}{
A \refstruct{pmix_data_array_t} array of string representations of the \ac{PU} binding bitmaps applied to each local \refterm{peer} on the caller's node upon launch. Each string shall begin with the name of the library that generated it (e.g., "hwloc") followed by a colon and the bitmap string itself. The array shall be in the same order as the processes returned by \refattr{PMIX_LOCAL_PEERS} for that namespace.
}
%
\declareAttribute{PMIX_LOCAL_SIZE}{"pmix.local.size"}{uint32_t}{
Number of processes in the specified job or application on the caller's node. Defaults to job unless the \refattr{PMIX_APP_INFO} and the \refattr{PMIX_APPNUM} qualifiers are given.
}


%%%%%%%%%%%%%%%%%%%%%%%%%%%%%%%%%%%%%%%%%%%%%%%%%
\subsection{Application realm attributes}
\label{chap:api_rsvd_keys:aprealm}

Application-related information can be retrieved by requesting a key which defaults
to the application realm or by including the \refattr{PMIX_APP_INFO} attribute
in the \refarg{info} array passed to \refapi{PMIx_Get}.
If the \refattr{PMIX_APPNUM} qualifier is given, then the
query shall return the corresponding value for the given application within the
namespace specified in the \refarg{proc} argument of the query (a \code{NULL}
value for the \refarg{proc} argument equates to the namespace of the caller).
If the \refattr{PMIX_APPNUM} qualifier is not included, then the retrieval
shall default to the application containing the process specified by \refarg{proc}. 
If the rank specified in \refarg{proc} is \refconst{PMIX_RANK_WILDCARD},
then the application number shall default to that of the calling process if the
namespace is its own job, or a value of zero if the namespace is that of a
different job.

The following keys, by default, request application-level information.
They will return information about the application indicated in \refarg{proc}:

%  Removing this because its not really part of the application realm.  It is probably just
% meant to be used to specify the APPNUM, not to query it.  It you do query it, you are really
% querying an attribute of a particular process.
%\pasteAttributeItem{PMIX_APPNUM}
%
\declareAttribute{PMIX_APPLDR}{"pmix.aldr"}{pmix_rank_t}{
Lowest rank in the specified application.
}
%
\declareAttribute{PMIX_APP_SIZE}{"pmix.app.size"}{uint32_t}{
Number of processes in the specified application, regardless of their execution state - i.e., this number may include processes that either failed to start or have already terminated.
}
%
\declareAttributeNEW{PMIX_APP_ARGV}{"pmix.app.argv"}{char*}{
Consolidated argv passed to the spawn command for the given application (e.g., "./myapp arg1 arg2 arg3").
}
%
\declareAttribute{PMIX_APP_MAP_TYPE}{"pmix.apmap.type"}{char*}{
Type of mapping used to layout the application (e.g., \code{cyclic}).
}
%
\declareAttribute{PMIX_APP_MAP_REGEX}{"pmix.apmap.regex"}{char*}{
Regular expression describing the result of the process mapping.
}

\vspace{\baselineskip}

The following application-related keys default to the realms described in their descriptions but can be
retrieved retrieved from the application realm by setting the \refattr{PMIX_APP_INFO} attribute in the \refarg{info} array passed to \refapi{PMIx_Get}:

\pasteAttributeItemBegin{PMIX_NUM_NODES}In this context, this is the number of
nodes currently hosting processes in the specified application, which may be a
subset of the nodes allocated to the overall session.
\pasteAttributeItemEnd{}
%
\pasteAttributeItemBegin{PMIX_MAX_PROCS}In this context, this is the maximum number of processes that can be executed in the specified application, which may be a subset of the number allocated to the overall session and job.
\pasteAttributeItemEnd{}
%
\pasteAttributeItemBegin{PMIX_NUM_SLOTS}In this context, this is the number of
slots assigned to the specified application, which may be a subset of the slots
allocated to the overall session and job.
\pasteAttributeItemEnd{}
%
\pasteAttributeItemBegin{PMIX_NODE_MAP}In this context, this is the regular expression of nodes currently hosting processes in the specified application.
\pasteAttributeItemEnd{}
%
\pasteAttributeItemBegin{PMIX_NODE_LIST}In this context, this is the comma-delimited list of nodes currently hosting processes in the specified application.
\pasteAttributeItemEnd{}
%
\pasteAttributeItemBegin{PMIX_PROC_MAP}In this context, this is the regular expression describing processes on each node in the specified application.
\pasteAttributeItemEnd{}

%%%%%%%%%%%%%%%%%%%%%%%%%%%%%%%%%%%%%%%%%%%%%%%%%
\subsection{Process realm attributes}
\label{chap:api_rsvd_keys:prealm}


Process-related information can be retrieved by requesting a key which defaults
to the process realm or by including the \refattr{PMIX_PROC_INFO} attribute
in the \refarg{info} array passed to \refapi{PMIx_Get}.
The target process is specified by the namespace
and rank of the \refarg{proc} argument to \refapi{PMIx_Get}. 
For process-related keys (other than \refattr{PMIX_PROCID} and \refattr{PMIX_NSPACE}) 
the target process is specified by setting the namespace and rank
of the target process in the \refarg{proc} argument passed to \refapi{PMIx_Get}.
If information
is requested about a process in a session other than the one containing the
requesting process, then an attribute identifying the target session must be
provided. This is required as many environments only guarantee unique
namespaces within a session, and not across sessions.

The following keys, by default, request process-level information:
They will return information about the process indicated in \refarg{proc}:

%
\declareAttribute{PMIX_APPNUM}{"pmix.appnum"}{uint32_t}{
The application number within the job in which the specified process is a member.
}
%
\declareAttribute{PMIX_RANK}{"pmix.rank"}{pmix_rank_t}{
Process rank within the job, starting from zero.
}

\declareAttribute{PMIX_NSPACE}{"pmix.nspace"}{char*}{
Namespace of the job - may be a numerical value expressed as a string, but is often an
alphanumeric string carrying information solely of use to the system. Required to be unique
within the scope of the host environment.  One cannot retrieve the namespace of an arbitrary process
since that would require already knowing the namespace of that process.  However, a processes own
namespace can be retrieved by passing a NULL value of \refarg{proc} to \refapi{PMIx_Get}.
}

\declareAttribute{PMIX_SESSION_ID}{"pmix.session.id"}{uint32_t}{
Session identifier assigned by the scheduler.
}

\declareAttribute{PMIX_GLOBAL_RANK}{"pmix.grank"}{pmix_rank_t}{
Rank of the specified process spanning across all jobs in this session,
starting with zero. Note that no ordering of the jobs is implied when computing
this value. As jobs can start and end at random times, this is defined as a
continually growing number - i.e., it is not dynamically adjusted as individual
jobs and processes are started or terminated.
}
%
\declareAttribute{PMIX_APP_RANK}{"pmix.apprank"}{pmix_rank_t}{
Rank of the specified process within its application.
}
%
\declareAttribute{PMIX_PARENT_ID}{"pmix.parent"}{pmix_proc_t}{
Process identifier of the parent process of the specified process - typically
used to identify the application process that caused the job containing the
specified process to be spawned (e.g., the process that called \refapi{PMIx_Spawn}).
}
%
\declareAttribute{PMIX_EXIT_CODE}{"pmix.exit.code"}{int}{
Exit code returned when the specified process terminated.
}
%
\declareAttribute{PMIX_PROCID}{"pmix.procid"}{pmix_proc_t}{
The caller's process identifier.  
The value returned is identical to what \refapi{PMIx_Init} or \refapi{PMIx_Tool_init} provides.
The process identifier in the \refapi{PMIx_Get} call is ignored when requesting this key.
}
%
\declareAttribute{PMIX_LOCAL_RANK}{"pmix.lrank"}{uint16_t}{
Rank of the specified process on its node - refers to the numerical location (starting from zero) of the process on its node when counting only those processes from the same job that share the node, ordered by their overall rank within that job.
}
%
\declareAttribute{PMIX_NODE_RANK}{"pmix.nrank"}{uint16_t}{
Rank of the specified process on its node spanning all jobs- refers to the numerical location (starting from zero) of the process on its node when counting all processes (regardless of job) that share the node, ordered by their overall rank within the job. The value represents a snapshot in time when the specified process was started on its node and is not dynamically adjusted as processes from other jobs are started or terminated on the node.
}
%
\declareAttributeNEW{PMIX_PACKAGE_RANK}{"pmix.pkgrank"}{uint16_t}{
Rank of the specified process on the \refterm{package} where this process resides - refers to the numerical location (starting from zero) of the process on its package when counting only those processes from the same job that share the package, ordered by their overall rank within that job. Note that processes that are not bound to \acp{PU} within a single specific package cannot have a package rank.
}
%
\declareAttribute{PMIX_PROC_PID}{"pmix.ppid"}{pid_t}{
Operating system \ac{PID} of specified process.
}
%
\declareAttribute{PMIX_PROCDIR}{"pmix.pdir"}{char*}{
Full path to the subdirectory under \refattr{PMIX_NSDIR} assigned to the specified process.
}
%
\declareAttribute{PMIX_CPUSET}{"pmix.cpuset"}{char*}{
A string representation of the \ac{PU} binding bitmap applied to the process upon launch. The string shall begin with the name of the library that generated it (e.g., "hwloc") followed by a colon and the bitmap string itself.
}
%
\declareAttributeNEW{PMIX_CPUSET_BITMAP}{"pmix.bitmap"}{pmix_cpuset_t*}{
Bitmap applied to the process upon launch.
}
%
\declareAttribute{PMIX_CREDENTIAL}{"pmix.cred"}{char*}{
Security credential assigned to the process.
}
%
\declareAttribute{PMIX_SPAWNED}{"pmix.spawned"}{bool}{
\code{true} if this process resulted from a call to \refapi{PMIx_Spawn}. Lack of inclusion (i.e., a return status of \refconst{PMIX_ERR_NOT_FOUND}) corresponds to a value of \code{false} for this attribute.
}
%
\declareAttributeNEW{PMIX_REINCARNATION}{"pmix.reinc"}{uint32_t}{
Number of times this process has been re-instantiated - i.e, a value of zero indicates that the process has never been restarted.
}

\vspace{\baselineskip}

In addition, process-level information includes functional attributes directly associated with a process - for example, the process-related fabric attributes included in Section \ref{api:fabric:attrs} or the distance attributes of Section \ref{api:netenddist:attrs}.

%%%%%%%%%%%%%%%%%%%%%%%%%%%%%%%%%%%%%%%%%%%%%%%%%
%%%%%%%%%%%%%%%%%%%%%%%%%%%%%%%%%%%%%%%%%%%%%%%%%
\subsection{Node realm keys}
\label{chap:api_rsvd_keys:nrealm}

Node-related information can be retrieved by requesting a key which defaults
to the node realm or by including the \refattr{PMIX_NODE_INFO} attribute
in the \refarg{info} array passed to \refapi{PMIx_Get}.
The target node defaults to the local node unless a different node is specified
in the \refarg{info} array using
either the \refattr{PMIX_HOSTNAME} or \refattr{PMIX_NODEID}.
Some node related keys are an exception to this rule and are 
listed separately at the end of this section.
These special keys can only target the local node and also require that a namespace 
be specified using the \refarg{proc} argument to \refapi{PMIx_Get}.


The following keys, by default, request node-level information.
They will return information about either the local node or the node specified by \refattr{PMIX_HOSTNAME} or \refattr{PMIX_NODEID}:

%
\declareAttribute{PMIX_HOSTNAME}{"pmix.hname"}{char*}{
Name of the host, as returned by the \code{gethostname} utility or its equivalent.
}
%
\declareAttributeNEW{PMIX_HOSTNAME_ALIASES}{"pmix.alias"}{char*}{
Comma-delimited list of names by which the target node is known.
}
%
\declareAttribute{PMIX_NODEID}{"pmix.nodeid"}{uint32_t}{
Node identifier expressed as the node's index (beginning at zero) in an array of nodes within the active session. The value must be unique and directly correlate to the \refattr{PMIX_HOSTNAME} of the node - i.e., users can interchangeably reference the same location using either the \refattr{PMIX_HOSTNAME} or corresponding \refattr{PMIX_NODEID}.
}
%
\declareAttribute{PMIX_NODE_SIZE}{"pmix.node.size"}{uint32_t}{
Number of processes across all jobs that are executing upon the node.
}
%
\declareAttribute{PMIX_AVAIL_PHYS_MEMORY}{"pmix.pmem"}{uint64_t}{
Total available physical memory on a node.
}

\declareAttribute{PMIX_LOCAL_PROCS}{"pmix.lprocs"}{pmix_proc_t array}{
Array of \refstruct{pmix_proc_t} of all processes executing on the local node -- shortcut for \refapi{PMIx_Resolve_peers} for the local node and a \code{NULL} namespace argument. The process identifier is ignored for this attribute.  Unlike other node-realm keys, this key does not allow the caller to specify
a specific node using \refattr{PMIX_HOSTNAME} or \refattr{PMIX_NODEID}. 
}
%

\vspace{\baselineskip}

In addition, node-level information includes functional attributes directly associated with a node - for example, the node-related fabric attributes included in Section \ref{api:fabric:attrs}.

%%%%%%%%%%%%%%%%%%%%%%%%%%%%%%%%%%%%%%%%%%%%%%%%%
%%%%%%%%%%%%%%%%%%%%%%%%%%%%%%%%%%%%%%%%%%%%%%%%%
\section{Retrieval rules for reserved keys}
\label{chap:rkeys:retrules}

The retrieval rules for reserved keys are relatively simple as the keys, if provided by 
an implementation, are
required, by definition, to be available when the client begins execution.
Accordingly, \refapi{PMIx_Get} for a reserved key first checks the local
\ac{PMIx} Client cache (per the data realm rules of the prior section) for the target key. If the information is not found,
then the \refconst{PMIX_ERR_NOT_FOUND} error constant is returned unless the
target process belongs to a different namespace from that of the requester.

In the case where the target and requester's namespaces differ, then the
request is forwarded to the local \ac{PMIx} server. Upon receiving the
request, the server shall check its data storage for the specified namespace.
If it already knows about this namespace, then it shall attempt to lookup the
specified key, returning the value if it is found or the
\refconst{PMIX_ERR_NOT_FOUND} error constant.

If the server does not have a copy of the information for the specified
namespace, then the server shall take one of the following actions:
\begin{enumerate}
    \item If the request included the \refattr{PMIX_IMMEDIATE} attribute, then
    the server will respond to the client with the
    \refconst{PMIX_ERR_NOT_FOUND} status.
    %
    \item If the host has provided the \ac{DBCX} module function interface
    (\refapi{pmix_server_dmodex_req_fn_t}), then the server shall pass the
    request to its host for servicing. The host is responsible for identifying
    a source of information on the specified namespace and retrieving it. The
    host is required to retrieve \emph{all} of the information regarding the target namespace
    and return it to the requesting server in anticipation of follow-on
    requests. If the host cannot retrieve the
    namespace information, then it must respond with the \refconst{PMIX_ERR_NOT_FOUND} error constant unless the \refattr{PMIX_TIMEOUT} is given and reached (in which case, the host must respond with the \refconst{PMIX_ERR_TIMEOUT} constant).

    Once the the \ac{PMIx} server receives the namespace information, the server shall search it (again adhering to the prior data realm rules) for the requested key, returning the value if it is found or the \refconst{PMIX_ERR_NOT_FOUND} error constant.
    %
    \item If the host does not support the \ac{DBCX} interface, then the
    server will respond to the client with the \refconst{PMIX_ERR_NOT_FOUND}
    status
\end{enumerate}


%%%%%%%%%%%%%%%%%%%%%%%%%%%%%%%%%%%%%%%%%%%%%%%%%
\subsection{Accessing information: examples}
\label{chap:api_rsvd_keys:getex}

This section provides examples illustrating methods for accessing information from the various realms. The intent of the examples is not to provide comprehensive coding guidance, but rather to further illustrate the use of \refapi{PMIx_Get} for obtaining information on a \refterm{session}, \refterm{job}, \refterm{application}, \refterm{process}, and \refterm{node}.


%%%%%%%%%%%%%%%%%%%%%%%%%%%%%%%%%%%%%%%%%%%%%%%%%
\subsubsection{Session-level information}

The \refapi{PMIx_Get} \ac{API} does not include an argument for specifying the \refterm{session} associated with the information being requested. Thus, requests for keys that are not specifically for session-level information must be accompanied by the \refattr{PMIX_SESSION_INFO} qualifier.

Example requests are shown below:

\cspecificstart
\begin{codepar}
pmix_info_t info;
pmix_value_t *value;
pmix_status_t rc;
pmix_proc_t myproc, wildcard;

/* initialize the client library */
PMIx_Init(&myproc, NULL, 0);

/* get the #slots in our session */
PMIX_PROC_LOAD(&wildcard, myproc.nspace, PMIX_RANK_WILDCARD);
rc = PMIx_Get(&wildcard, PMIX_UNIV_SIZE, NULL, 0, &value);

/* get the #nodes in our session */
PMIX_INFO_LOAD(&info, PMIX_SESSION_INFO, NULL, PMIX_BOOL);
rc = PMIx_Get(&wildcard, PMIX_NUM_NODES, &info, 1, &value);
\end{codepar}
\cspecificend

Information regarding a different session can be requested by adding the \refattr{PMIX_SESSION_ID} attribute identifying the target session. In this case, the \refarg{proc} argument to \refapi{PMIx_Get} will be ignored:

\cspecificstart
\begin{codepar}
pmix_info_t info[2];
pmix_value_t *value;
pmix_status_t rc;
pmix_proc_t myproc;
uint32_t sid;

/* initialize the client library */
PMIx_Init(&myproc, NULL, 0);

/* get the #nodes in a different session */
sid = 12345;
PMIX_INFO_LOAD(&info[0], PMIX_SESSION_INFO, NULL, PMIX_BOOL);
PMIX_INFO_LOAD(&info[1], PMIX_SESSION_ID, &sid, PMIX_UINT32);
rc = PMIx_Get(NULL, PMIX_NUM_NODES, info, 2, &value);
\end{codepar}
\cspecificend


%%%%%%%%%%%%%%%%%%%%%%%%%%%%%%%%%%%%%%%%%%%%%%%%%
\subsubsection{Job-level information}

Information regarding a job can be obtained by the methods detailed in Section \ref{chap:api_rsvd_keys:jrealm}. Example requests are shown below:

\cspecificstart
\begin{codepar}
pmix_info_t info;
pmix_value_t *value;
pmix_status_t rc;
pmix_proc_t myproc, wildcard;

/* initialize the client library */
PMIx_Init(&myproc, NULL, 0);

/* get the #apps in our job */
PMIX_PROC_LOAD(&wildcard, myproc.nspace, PMIX_RANK_WILDCARD);
rc = PMIx_Get(&wildcard, PMIX_JOB_NUM_APPS, NULL, 0, &value);

/* get the #nodes in our job */
PMIX_INFO_LOAD(&info, PMIX_JOB_INFO, NULL, PMIX_BOOL);
rc = PMIx_Get(&wildcard, PMIX_NUM_NODES, &info, 1, &value);
\end{codepar}
\cspecificend


%%%%%%%%%%%%%%%%%%%%%%%%%%%%%%%%%%%%%%%%%%%%%%%%%
\subsubsection{Application-level information}

Information regarding an application can be obtained by the methods described in Section \ref{chap:api_rsvd_keys:aprealm}. Example requests are shown below:

\cspecificstart
\begin{codepar}
pmix_info_t info;
pmix_value_t *value;
pmix_status_t rc;
pmix_proc_t myproc, otherproc;
uint32_t appsize, appnum;

/* initialize the client library */
PMIx_Init(&myproc, NULL, 0);

/* get the #processes in our application */
rc = PMIx_Get(&myproc, PMIX_APP_SIZE, NULL, 0, &value);
appsize = value->data.uint32;

/* get the #nodes in an application containing "otherproc".
 * For this use-case, assume that we are in the first application
 * and we want the #nodes in the second application - use the
 * rank of the first process in that application, remembering
 * that ranks start at zero */
PMIX_PROC_LOAD(&otherproc, myproc.nspace, appsize);

/* Since "otherproc" refers to a process in the second application,
 * we can simply mark that we want the info for this key from the
 * application realm */
PMIX_INFO_LOAD(&info, PMIX_APP_INFO, NULL, PMIX_BOOL);
rc = PMIx_Get(&otherproc, PMIX_NUM_NODES, &info, 1, &value);

/* alternatively, we can directly ask for the #nodes in
 * the second application in our job, again remembering that
 * application numbers start with zero. Since we are asking
 * for application realm information about a specific appnum
 * within our own namespace, the process identifier can be NULL */
appnum = 1;
PMIX_INFO_LOAD(&appinfo[0], PMIX_APP_INFO, NULL, PMIX_BOOL);
PMIX_INFO_LOAD(&appinfo[1], PMIX_APPNUM, &appnum, PMIX_UINT32);
rc = PMIx_Get(NULL, PMIX_NUM_NODES, appinfo, 2, &value);
\end{codepar}
\cspecificend


%%%%%%%%%%%%%%%%%%%%%%%%%%%%%%%%%%%%%%%%%%%%%%%%%
\subsubsection{Process-level information}

Process-level information is accessed by providing the namespace and rank of the target process. In the absence of any directive as to the level of information being requested, the \ac{PMIx} library will always return the process-level value. See Section \ref{chap:api_rsvd_keys:prealm} for details.


%%%%%%%%%%%%%%%%%%%%%%%%%%%%%%%%%%%%%%%%%%%%%%%%%
\subsubsection{Node-level information}

Information regarding a node within the system can be obtained by the methods described in Section \ref{chap:api_rsvd_keys:nrealm}. Example requests are shown below:

\cspecificstart
\begin{codepar}
pmix_info_t info[2];
pmix_value_t *value;
pmix_status_t rc;
pmix_proc_t myproc, otherproc;
uint32_t nodeid;

/* initialize the client library */
PMIx_Init(&myproc, NULL, 0);

/* get the #procs on our node */
rc = PMIx_Get(&myproc, PMIX_NODE_SIZE, NULL, 0, &value);

/* get the #slots on another node */
PMIX_INFO_LOAD(&info[0], PMIX_NODE_INFO, NULL, PMIX_BOOL);
PMIX_INFO_LOAD(&info[1], PMIX_HOSTNAME, "remotehost", PMIX_STRING);
rc = PMIx_Get(NULL, PMIX_MAX_PROCS, info, 2, &value);

/* get the total #procs on the remote node - note that we don't
 * actually need to include the "PMIX_NODE_INFO" attribute here,
 * but (a) it does no harm and (b) it allowed us to simply reuse
 * the prior info array
rc = PMIx_Get(NULL, PMIX_NODE_SIZE, info, 2, &value);
\end{codepar}
\cspecificend

%%%%%%%%%%%%%%%%%%%%%%%%%%%%%%%%%%%%%%%%%%%%%%%%%

    %%%%%%%%%%%%%%%%%%%%%%%%%%%%%%%%%%%%%%%%%%%%%%%%%
% Chapter: Process-Related Non-Reserved Keys
%%%%%%%%%%%%%%%%%%%%%%%%%%%%%%%%%%%%%%%%%%%%%%%%%
\chapter{Process-Related Non-Reserved Keys}
\label{chap:nrkeys}

\emph{Non-reserved keys} are keys whose string representation begin with a
prefix other than \code{"pmix"}. Such keys are typically defined by an
application when information needs to be exchanged between processes (e.g.,
where connection information is required and the host environment does not
support the \refterm{instant on} option) or where the host environment does not
provide a required piece of data. Beyond the restriction on name prefix,
non-reserved keys are required to be unique across conflicting \emph{scopes} as defined in Section \ref{api:nres:scope} - e,g., a non-reserved key cannot be posted by the same process in both the \refconst{PMIX_LOCAL} and \refconst{PMIX_REMOTE} scopes (note that posting the key in the \refconst{PMIX_GLOBAL} scope would have met the desired objective).

\ac{PMIx} provides support for two methods of exchanging non-reserved keys:

\begin{itemize}
    \item Global, collective exchange of the information prior to retrieval. This is accomplished by executing a barrier operation that includes collection and exchange of the data provided by each process such that each process has access to the full set of data from all participants once the operation has completed. \ac{PMIx} provides the \refapi{PMIx_Fence} function (or its non-blocking equivalent) for this purpose, accompanied by the \refattr{PMIX_COLLECT_DATA} qualifier.
    \item Direct, on-demand retrieval of the information. No barrier or global exchange is conducted in this case. Instead, information is retrieved from the host where that process is executing upon request - i.e., a call to \refapi{PMIx_Get} results in a data exchange with the \ac{PMIx} server on the remote host. Various caching strategies may be employed by the host environment and/or \ac{PMIx} implementation to reduce the number of retrievals. Note that this method requires that the host environment both know the location of the posting process and support direct information retrieval.
\end{itemize}

Both of the above methods are based on retrieval from a specific process -
i.e., the \refarg{proc} argument to \refapi{PMIx_Get} must include both the
namespace and the rank of the process that posted the information. However, in
some cases, non-reserved keys are provided on a globally unique basis and the
retrieving process has no knowledge of the identity of the process posting the
key. This is typically found in legacy applications (where the originating
process identifier is often embedded in the key itself) and in unstructured
applications that lack rank-related behavior. In these cases, the key remains
associated with the namespace of the process that posted it, but is retrieved
by use of the \refconst{PMIX_RANK_UNDEF} rank. In addition, the keys must be
globally exchanged prior to retrieval as there is no way for the host to
otherwise locate the source for the information.

Note that the retrieval rules for non-reserved keys (detailed in Section \ref{chap:nrkeys:retrules}) differ significantly from those used for reserved keys.


%%%%%%%%%%%%%%%%%%%%%%%%%%%%%%%%%%%%%%%%%%%%%%%%%
%%%%%%%%%%%%%%%%%%%%%%%%%%%%%%%%%%%%%%%%%%%%%%%%%
\section{Posting Key/Value Pairs}
\label{chap:api_kv_mgmt:set}

\ac{PMIx} clients can post non-reserved key-value pairs associated with themselves by using \refapi{PMIx_Put}. Alternatively, \ac{PMIx} clients can cache arbitrary key-value pairs accessible only by the caller via the \refapi{PMIx_Store_internal} \ac{API}.


%%%%%%%%%%%%%%%%%%%%%%%%%%%%%%%%%%%%%%%%%%%%%%%%%
\subsection{\code{PMIx_Put}}
\declareapi{PMIx_Put}

%%%%
\summary

Post a key/value pair for distribution.

%%%%
\format

\copySignature{PMIx_Put}{1.0}{
pmix_status_t \\
PMIx_Put(pmix_scope_t scope, \\
\hspace*{9\sigspace}const pmix_key_t key, \\
\hspace*{9\sigspace}pmix_value_t *val);
}

\begin{arglist}
\argin{scope}{Distribution scope of the provided value (handle)}
\argin{key}{key (\refstruct{pmix_key_t})}
\argin{value}{Reference to a \refstruct{pmix_value_t} structure (handle)}
\end{arglist}

Returns \refconst{PMIX_SUCCESS} or a negative value corresponding to a \ac{PMIx} error constant.
If a reserved key is provided in the \refarg{key} argument then \refapi{PMIx_Put} will return \refconst{PMIX_ERR_BAD_PARAM}.

%%%%
\descr

Post a key-value pair for distribution. Depending upon the \ac{PMIx} implementation, the posted value may be locally cached in the client's \ac{PMIx} library until \refapi{PMIx_Commit} is called.

The provided \refarg{scope} determines the ability of other processes to access the posted data, as defined in \specrefstruct{pmix_scope_t}.
Specific implementations may support different scope values, but all implementations must support at least \refconst{PMIX_GLOBAL}.

The \refstruct{pmix_value_t} structure supports both string and binary values.
\ac{PMIx} implementations are required to support heterogeneous environments by properly converting binary values between host architectures, and will copy the provided \refarg{value} into internal memory prior to returning from \refapi{PMIx_Put}.

\adviceuserstart
Note that keys starting with a string of ``\code{pmix}'' must not be used in calls to \refapi{PMIx_Put}. Thus, applications should never use a defined ``PMIX'' attribute as the key in a call to \refapi{PMIx_Put}.
\adviceuserend


%%%%%%%%%%%%%%%%%%%%%%%%%%%%%%%%%%%%%%%%%%%%%%%%%
\subsubsection{Scope of Put Data}
\declarestruct{pmix_scope_t}
\label{api:nres:scope}

\versionMarker{1.0}
The \refstruct{pmix_scope_t} structure is a \code{uint8_t} type that defines the availability of data passed to \refapi{PMIx_Put}.
The following constants can be used to set a variable of the type \refstruct{pmix_scope_t}. All definitions were introduced in version 1 of the standard unless otherwise marked.

Specific implementations may support different scope values, but all implementations must support at least \refconst{PMIX_GLOBAL}.
If a specified scope value is not supported, then the \refapi{PMIx_Put} call must return \refconst{PMIX_ERR_NOT_SUPPORTED}.

\begin{constantdesc}
%
\declareconstitem{PMIX_SCOPE_UNDEF}
Undefined scope.
%
\declareconstitem{PMIX_LOCAL}
The data is intended only for other application processes on the same node.
Data marked in this way will not be included in data packages sent to remote requesters - i.e., it is only available to processes on the local node.
%
\declareconstitem{PMIX_REMOTE}
The data is intended solely for applications processes on remote nodes.
Data marked in this way will not be shared with other processes on the same node - i.e., it is only available to  processes on remote nodes.
%
\declareconstitem{PMIX_GLOBAL}
The data is to be shared with all other requesting processes, regardless of location.
%
\versionMarker{2.0}
\declareconstitem{PMIX_INTERNAL}
The data is intended solely for this process and is not shared with other processes.
%
\end{constantdesc}


%%%%%%%%%%%%%%%%%%%%%%%%%%%%%%%%%%%%%%%%%%%%%%%%%
\subsection{\code{PMIx_Store_internal}}
\declareapi{PMIx_Store_internal}

%%%%
\summary

Store some data locally for retrieval by other areas of the process.

%%%%
\format

\copySignature{PMIx_Store_internal}{1.0}{
pmix_status_t \\
PMIx_Store_internal(const pmix_proc_t *proc, \\
\hspace*{20\sigspace}const pmix_key_t key, \\
\hspace*{20\sigspace}pmix_value_t *val);
}

\begin{arglist}
\argin{proc}{process reference (handle)}
\argin{key}{key to retrieve (string)}
\argin{val}{Value to store (handle)}
\end{arglist}

Returns \refconst{PMIX_SUCCESS} or a negative value corresponding to a PMIx error constant.
If a reserved key is provided in the \refarg{key} argument then \refapi{PMIx_Store_internal} will return \refconst{PMIX_ERR_BAD_PARAM}.

%%%%
\descr

Store some data locally for retrieval by other areas of the process.
This is data that has only internal scope - it will never be posted externally. Typically used to cache data obtained by means outside of \ac{PMIx} so that it can be accessed by various areas of the process.


%%%%%%%%%%%%%%%%%%%%%%%%%%%%%%%%%%%%%%%%%%%%%%%%%
\subsection{\code{PMIx_Commit}}
\declareapi{PMIx_Commit}

%%%%
\summary

Post all previously \refapi{PMIx_Put} values for distribution.

%%%%
\format

\copySignature{PMIx_Commit}{1.0}{
pmix_status_t PMIx_Commit(void);
}

Returns \refconst{PMIX_SUCCESS} or a negative value corresponding to a PMIx error constant.

%%%%
\descr

\ac{PMIx} implementations may choose to locally cache non-reserved keys prior to submitting them for distribution. Accordingly, \ac{PMIx} provides a second \ac{API} specifically to stage all previously posted data for distribution - e.g., by transmitting the entire collection of data posted by the process to a server in one operation. This is an asynchronous operation that will immediately return to the caller while the data is staged in the background.

\adviceuserstart
Users are advised to always include the call to \refapi{PMIx_Commit} in case the local implementation requires it. Note that posted data will not be circulated during \refapi{PMIx_Commit}. Availability of the data by other processes upon completion of \refapi{PMIx_Commit} therefore still relies upon the exchange mechanisms described at the beginning of this chapter.
\adviceuserend


%%%%%%%%%%%%%%%%%%%%%%%%%%%%%%%%%%%%%%%%%%%%%%%%%
\section{Retrieval rules for non-reserved keys}
\label{chap:nrkeys:retrules}

Since non-reserved keys cannot, by definition, have been provided by the host
environment, their retrieval follows significantly different rules than those
defined for reserved keys (as detailed in Section \ref{chap:rkeys:retrules}).
\refapi{PMIx_Get} for a non-reserved key will obey the
following precedence search:

\begin{enumerate}
    \item If the \refattr{PMIX_GET_REFRESH_CACHE} attribute is given, then the
    request is first forwarded to the local \ac{PMIx} server which will then
    update the client's cache. Note that this may not, depending upon
    implementation details, result in any action.

    \item Check the local \ac{PMIx} client cache for the requested key - if not found and either the \refattr{PMIX_OPTIONAL} or \refattr{PMIX_GET_REFRESH_CACHE} attribute was given, the search will stop at this point and return the \refconst{PMIX_ERR_NOT_FOUND} status.

    \item Request the information from the local \ac{PMIx} server. The server
    will check its cache for the specified key within the appropriate scope as
    defined by the process that originally posted the key. If the value exists
    in a scope that contains the requesting process, then the value shall be
    returned. If the value exists, but in a scope that excludes the requesting
    process, then the server shall immediately return the
    \refconst{PMIX_ERR_EXISTS_OUTSIDE_SCOPE}.

    If the value still isn't found and the \refattr{PMIX_IMMEDIATE} attribute
    was given, then the library shall return the \refconst{PMIX_ERR_NOT_FOUND}
    error constant to the requester. Otherwise, the \ac{PMIx} server library
    will take one of the following actions:
    \begin{compactitemize}
        \item If the target process has a rank of \refconst{PMIX_RANK_UNDEF},
        then this indicates that the key being requested is globally unique
        and \emph{not} associated with a specific process. In this case, the
        server shall hold the request until either the data appears at the
        server or, if given, the \refattr{PMIX_TIMEOUT} is reached. In the
        latter case, the server will return the \refconst{PMIX_ERR_TIMEOUT}
        status. Note that the server may, depending on \ac{PMIx}
        implementation, never respond if the caller failed to specify a
        \refattr{PMIX_TIMEOUT} and the requested key fails to arrive at the
        server.

        \item If the target process is \emph{local} (i.e., attached to the
        same \ac{PMIx} server), then the server will hold the request until
        either the target process provides the data or, if given, the
        \refattr{PMIX_TIMEOUT} is reached. In the latter case, the server will
        return the \refconst{PMIX_ERR_TIMEOUT} status. Note that data which is
        posted via \refapi{PMIx_Put} but not staged with \refapi{PMIx_Commit}
        may, depending upon implementation, never appear at the server.

        \item If the target process is \emph{remote} (i.e., not attached to
        the same \ac{PMIx} server), the server will either:
        \begin{compactitemize}
            \item If the host has provided the
            \refapi{pmix_server_dmodex_req_fn_t} module function
            interface, then the server
            shall pass the request to its host for servicing. The host is
            responsible for determining the location of the target process and
            passing the request to the \ac{PMIx} server at that location.

            When the remote data request is received, the target \ac{PMIx}
            server will check its cache for the specified key. If the key is
            not present, the request shall be held until either the target
            process provides the data or, if given, the \refattr{PMIX_TIMEOUT}
            is reached. In the latter case, the server will return the
            \refconst{PMIX_ERR_TIMEOUT} status. The host shall convey the
            result back to the originating \ac{PMIx} server, which will reply
            to the requesting client with the result of the request when the
            host provides it.

            Note that the target server may, depending on \ac{PMIx}
            implementation, never respond if the caller failed to specify a
            \refattr{PMIX_TIMEOUT} and the target process fails to post the
            requested key.

            \item if the host does not support the
            \refapi{pmix_server_dmodex_req_fn_t} interface, then
            the server will immediately respond to the client with the
            \refconst{PMIX_ERR_NOT_FOUND} status
        \end{compactitemize}
    \end{compactitemize}
\end{enumerate}

\adviceimplstart
While there is no requirement that all \ac{PMIx} implementations follow the
client-server paradigm used in the above description, implementers are
required to provide behaviors consistent with the described search pattern.
\adviceimplend

\adviceuserstart
Users are advised to always specify the \refattr{PMIX_TIMEOUT} value when
retrieving non-reserved keys to avoid potential deadlocks should the specified
key not become available.
\adviceuserend

%%%%%%%%%%%%%%%%%%%%%%%%%%%%%%%%%%%%%%%%%%%%%%%%%

    %%%%%%%%%%%%%%%%%%%%%%%%%%%%%%%%%%%%%%%%%%%%%%%%%
% Chapter: Publish/Lookup Operations
%%%%%%%%%%%%%%%%%%%%%%%%%%%%%%%%%%%%%%%%%%%%%%%%%
\chapter{Publish/Lookup Operations}
\label{chap:pub}

Chapter~\ref{chap:api_rsvd_keys} and Chapter~\ref{chap:nrkeys} discussed how reserved and non-reserved keys dealt with
information that either was associated with a specific process (i.e., the
retrieving process knew the identifier of the process that posted it) or
required a synchronization operation prior to retrieval (e.g., the case of
globally unique non-reserved keys). However, another requirement exists for an
asynchronous exchange of data where neither the posting nor the retrieving
process is known in advance. For example, two separate namespaces may need to
rendezvous with each other without knowing in advance the identity of the other
namespace or when that namespace might become active.

The \acp{API} defined in this section focus on resolving that specific
situation by allowing processes to publish data that can subsequently be
retrieved solely by referral to its key. Mechanisms for constraining
availability of the information are also provided as a means for better
targeting of the eventual recipient(s).

Note that no presumption is made regarding how the published information is to be stored, nor as to the entity (host environment or \ac{PMIx} implementation) that shall act as the datastore. The descriptions in the remainder of this chapter shall simply refer to that entity as the \emph{datastore}.


%%%%%%%%%%%%%%%%%%%%%%%%%%%%%%%%%%%%%%%%%%%%%%%%%
%%%%%%%%%%%%%%%%%%%%%%%%%%%%%%%%%%%%%%%%%%%%%%%%%
\section{\code{PMIx_Publish}}
\declareapi{PMIx_Publish}

%%%%
\summary

Publish data for later access via \refapi{PMIx_Lookup}.

%%%%
\format

\copySignature{PMIx_Publish}{1.0}{
pmix_status_t \\
PMIx_Publish(const pmix_info_t info[], size_t ninfo);
}

\begin{arglist}
\argin{info}{Array of info structures containing both data to be published and directives (array of handles)}
\argin{ninfo}{Number of elements in the \refarg{info} array (integer)}
\end{arglist}

Returns \refconst{PMIX_SUCCESS} or a negative value corresponding to a PMIx error constant.

\reqattrstart
There are no required attributes for this \ac{API}. \ac{PMIx} implementations that do not directly support the operation but are hosted by environments that do support it must pass any attributes that are provided by the client to the host environment for processing. In addition, the \ac{PMIx} library is required to add the \refAttributeItem{PMIX_USERID} and the \refAttributeItem{PMIX_GRPID} attributes of the client process that published the information to the \refarg{info} array passed to the host environment.

\reqattrend

\optattrstart
The following attributes are optional for host environments that support this operation:

\pasteAttributeItem{PMIX_TIMEOUT}
\pasteAttributeItem{PMIX_RANGE}
\pasteAttributeItem{PMIX_PERSISTENCE}
\pasteAttributeItem{PMIX_ACCESS_PERMISSIONS}

\optattrend

%%%%
\descr

Publish the data in the \refarg{info} array for subsequent lookup.
By default, the data will be published into the \refconst{PMIX_RANGE_SESSION} range and with \refconst{PMIX_PERSIST_APP} persistence.
Changes to those values, and any additional directives, can be included in the \refstruct{pmix_info_t} array. Attempts to access the data by processes outside of the provided data range shall be rejected. The \refattr{PMIX_PERSISTENCE} attribute instructs the datastore holding the published information as to how long that information is to be retained.

The blocking form of this call will block until it has obtained confirmation from the datastore that the data is available for lookup. The \refarg{info} array can be released upon return from the blocking function call.

Publishing duplicate keys is permitted provided they are published to different
ranges. Duplicate keys being published on the same data range shall return the
\refconst{PMIX_ERR_DUPLICATE_KEY} error.


%%%%%%%%%%%%%%%%%%%%%%%%%%%%%%%%%%%%%%%%%%%%%%%%%
%%%%%%%%%%%%%%%%%%%%%%%%%%%%%%%%%%%%%%%%%%%%%%%%%
\section{\code{PMIx_Publish_nb}}
\declareapi{PMIx_Publish_nb}

%%%%
\summary

Nonblocking \refapi{PMIx_Publish} routine.

%%%%
\format

\copySignature{PMIx_Publish_nb}{1.0}{
pmix_status_t \\
PMIx_Publish_nb(const pmix_info_t info[], size_t ninfo, \\
\hspace*{16\sigspace}pmix_op_cbfunc_t cbfunc, void *cbdata);
}

\begin{arglist}
\argin{info}{Array of info structures containing both data to be published and directives (array of handles)}
\argin{ninfo}{Number of elements in the \refarg{info} array (integer)}
\argin{cbfunc}{Callback function \refapi{pmix_op_cbfunc_t} (function reference)}
\argin{cbdata}{Data to be passed to the callback function (memory reference)}
\end{arglist}

Returns one of the following:

\begin{itemize}
    \item \refconst{PMIX_SUCCESS}, indicating that the request is being processed by the host environment - result will be returned in the provided \refarg{cbfunc}. Note that the library must not invoke the callback function prior to returning from the \ac{API}.
    \item \refconst{PMIX_OPERATION_SUCCEEDED}, indicating that the request was immediately processed and returned \textit{success} - the \refarg{cbfunc} will \textit{not} be called.
    \item a PMIx error constant indicating either an error in the input or that the request was immediately processed and failed - the \refarg{cbfunc} will \textit{not} be called.
\end{itemize}

\reqattrstart
There are no required attributes for this \ac{API}. \ac{PMIx} implementations that do not directly support the operation but are hosted by environments that do support it must pass any attributes that are provided by the client to the host environment for processing. In addition, the \ac{PMIx} library is required to add the \refAttributeItem{PMIX_USERID} and the \refAttributeItem{PMIX_GRPID} attributes of the client process that published the information to the \refarg{info} array passed to the host environment.

\reqattrend

\optattrstart
The following attributes are optional for host environments that support this operation:

\pasteAttributeItem{PMIX_TIMEOUT}
\pasteAttributeItem{PMIX_RANGE}
\pasteAttributeItem{PMIX_PERSISTENCE}
\pasteAttributeItem{PMIX_ACCESS_PERMISSIONS}

\optattrend

%%%%
\descr

Nonblocking \refapi{PMIx_Publish} routine.


%%%%%%%%%%%%%%%%%%%%%%%%%%%%%%%%%%%%%%%%%%%%%%%%%
%%%%%%%%%%%%%%%%%%%%%%%%%%%%%%%%%%%%%%%%%%%%%%%%%
\section{Publish-specific constants}

The following constants are defined for use with the \refapi{PMIx_Publish} \acp{API}:

\begin{constantdesc}
%
\declareconstitemNEW{PMIX_ERR_DUPLICATE_KEY}
The provided key has already been published on the same data range.
%
\end{constantdesc}


%%%%%%%%%%%%%%%%%%%%%%%%%%%%%%%%%%%%%%%%%%%%%%%%%
%%%%%%%%%%%%%%%%%%%%%%%%%%%%%%%%%%%%%%%%%%%%%%%%%
\section{Publish-specific attributes}

The following attributes are defined for use with the \refapi{PMIx_Publish} \acp{API}:

%
\declareAttribute{PMIX_RANGE}{"pmix.range"}{pmix_data_range_t}{
Define constraints on the processes that can access the provided data. Only processes that meet the constraints are allowed to access it.
}
%
\declareAttribute{PMIX_PERSISTENCE}{"pmix.persist"}{pmix_persistence_t}{
Declare how long the datastore shall retain the provided data. The datastore is to delete the data upon reaching the persistence criterion.
}
%
\declareAttributeNEW{PMIX_ACCESS_PERMISSIONS}{"pmix.aperms"}{pmix_data_array_t}{
Define access permissions for the published data. The value shall contain an array of \refstruct{pmix_info_t} structs containing the specified permissions.
}
%
\declareAttributeNEW{PMIX_ACCESS_USERIDS}{"pmix.auids"}{pmix_data_array_t}{
Array of effective \acp{UID} that are allowed to access the published data.
}
%
\declareAttributeNEW{PMIX_ACCESS_GRPIDS}{"pmix.agids"}{pmix_data_array_t}{
Array of effective \acp{GID} that are allowed to access the published data.
}


%%%%%%%%%%%%%%%%%%%%%%%%%%%%%%%%%%%%%%%%%%%%%%%%%
%%%%%%%%%%%%%%%%%%%%%%%%%%%%%%%%%%%%%%%%%%%%%%%%%
\section{Publish-Lookup Datatypes}

The following data types are defined for use with the \refapi{PMIx_Publish} \acp{API}.

%%%%%%%%%%%%%%%%%%%%%%%%%%%%%%%%%%%%%%%%%%%%%%%%%
\subsection{Range of Published Data}
\declarestruct{pmix_data_range_t}

\versionMarker{1.0}
The \refstruct{pmix_data_range_t} structure is a \code{uint8_t} type that defines a range for both data \textit{published} via the \refapi{PMIx_Publish} \ac{API} and generated events.
The following constants can be used to set a variable of the type \refstruct{pmix_data_range_t}.

\begin{constantdesc}
%
\declareconstitem{PMIX_RANGE_UNDEF}
Undefined range.
%
\declareconstitem{PMIX_RANGE_RM}
Data is intended for the host environment, or lookup is restricted to data published by the host environment.
%
\declareconstitem{PMIX_RANGE_LOCAL}
Data is only available to processes on the local node, or lookup is restricted to data published by processes on the local node of the requester.
%
\declareconstitem{PMIX_RANGE_NAMESPACE}
Data is only available to processes in the same namespace, or lookup is restricted to data published by processes in the same namespace as the requester.
%
\declareconstitem{PMIX_RANGE_SESSION}
Data is only available to all processes in the session, or lookup is restricted to data published by other processes in the same session as the requester.
%
\declareconstitem{PMIX_RANGE_GLOBAL}
Data is available to all processes, or lookup is open to data published by anyone.
%
\declareconstitem{PMIX_RANGE_CUSTOM}
Data is available only to processes as specified in the \refstruct{pmix_info_t} associated with this call, or lookup is restricted to data published by processes as specified in the \refstruct{pmix_info_t}.
%
\declareconstitem{PMIX_RANGE_PROC_LOCAL}
Data is only available to this process, or lookup is restricted to data published by this process.
%
\declareconstitem{PMIX_RANGE_INVALID}
Invalid value - typically used to indicate that a range has not yet been set.
%
\end{constantdesc}


%%%%%%%%%%%%%%%%%%%%%%%%%%%%%%%%%%%%%%%%%%%%%%%%%
\subsection{Data Persistence Structure}
\declarestruct{pmix_persistence_t}

\versionMarker{1.0}
The \refstruct{pmix_persistence_t} structure is a \code{uint8_t} type that defines the policy for data published by clients via the \refapi{PMIx_Publish} \ac{API}.
The following constants can be used to set a variable of the type \refstruct{pmix_persistence_t}.

\begin{constantdesc}
%
\declareconstitem{PMIX_PERSIST_INDEF}
Retain data until specifically deleted.
%
\declareconstitem{PMIX_PERSIST_FIRST_READ}
Retain data until the first access, then the data is deleted.
%
\declareconstitem{PMIX_PERSIST_PROC}
Retain data until the publishing process terminates.
%
\declareconstitem{PMIX_PERSIST_APP}
Retain data until the application terminates.
%
\declareconstitem{PMIX_PERSIST_SESSION}
Retain data until the session/allocation terminates.
%
\declareconstitem{PMIX_PERSIST_INVALID}
Invalid value - typically used to indicate that a persistence has not yet been set.
%
\end{constantdesc}


%%%%%%%%%%%%%%%%%%%%%%%%%%%%%%%%%%%%%%%%%%%%%%%%%
%%%%%%%%%%%%%%%%%%%%%%%%%%%%%%%%%%%%%%%%%%%%%%%%%
\section{\code{PMIx_Lookup}}
\declareapi{PMIx_Lookup}

%%%%
\summary

Lookup information published by this or another process with \refapi{PMIx_Publish} or \refapi{PMIx_Publish_nb}.

%%%%
\format

\copySignature{PMIx_Lookup}{1.0}{
pmix_status_t \\
PMIx_Lookup(pmix_pdata_t data[], size_t ndata, \\
\hspace*{12\sigspace}const pmix_info_t info[], size_t ninfo);
}

\begin{arglist}
\arginout{data}{Array of publishable data structures (array of \refstruct{pmix_pdata_t})}
\argin{ndata}{Number of elements in the \refarg{data} array (integer)}
\argin{info}{Array of info structures (array of \refstruct{pmix_info_t})}
\argin{ninfo}{Number of elements in the \refarg{info} array (integer)}
\end{arglist}

Returns one of the following:

\begin{itemize}
\item \refconst{PMIX_SUCCESS} All data was found and has been returned.

\item \refconst{PMIX_ERR_NOT_FOUND} None of the requested data could be found within the requester's range.

\item \refconst{PMIX_ERR_PARTIAL_SUCCESS} Some of the requested data was found.
Any key that cannot be found will return with a data type of \refconst{PMIX_UNDEF} in the associated \refarg{value} struct. Note that the specific reason for a particular piece of missing information (e.g., lack of permissions) cannot be communicated back to the requester in this situation.

\item \refconst{PMIX_ERR_NOT_SUPPORTED} There is no available datastore (either at the host environment or \ac{PMIx} implementation level) on this system that supports this function.

\item \refconst{PMIX_ERR_NO_PERMISSIONS} All of the requested data was found and range restrictions were met for each specified key, but none of the matching data could be returned due to lack of access permissions.

\item a non-zero \ac{PMIx} error constant indicating a reason for the request's failure.
\end{itemize}

\reqattrstart
\ac{PMIx} libraries are not required to directly support any attributes for this function. However, any provided attributes must be passed to the host environment for processing, and the \ac{PMIx} library is required to add the \refAttributeItem{PMIX_USERID} and the \refAttributeItem{PMIX_GRPID} attributes of the client process that is requesting the info.

\reqattrend

\optattrstart
The following attributes are optional for host environments that support this operation:

\pasteAttributeItem{PMIX_TIMEOUT}
\pasteAttributeItem{PMIX_RANGE}
\pasteAttributeItem{PMIX_WAIT}

\optattrend

%%%%
\descr

Lookup information published by this or another process.
By default, the search will be constrained to publishers that fall within the \refconst{PMIX_RANGE_SESSION} range in case duplicate keys exist on different ranges.
Changes to the range (e.g., expanding the search to all potential publishers via the \refconst{PMIX_RANGE_GLOBAL} constant), and any additional directives, can be provided in the \refstruct{pmix_info_t} array. Data is returned per the retrieval rules of Section \ref{chap:pub:retrules}.

The \argref{data} parameter consists of an array of \refstruct{pmix_pdata_t} structures with the keys specifying the requested information.
Data will be returned for each \code{key} field in the associated \code{value} field of this structure as per the above description of return values. The \code{proc} field in each \refstruct{pmix_pdata_t} structure will contain the namespace/rank of the process that published the data.

\adviceuserstart
Although this is a blocking function, it will not wait by default for the requested data to be published.
Instead, it will block for the time required by the datastore to lookup its current data and return any found items.
Thus, the caller is responsible for either ensuring that data is published prior to executing a lookup, using \refattr{PMIX_WAIT} to instruct the datastore to wait for the data to be published, or retrying until the requested data is found.
\adviceuserend


%%%%%%%%%%%%%%%%%%%%%%%%%%%%%%%%%%%%%%%%%%%%%%%%%
%%%%%%%%%%%%%%%%%%%%%%%%%%%%%%%%%%%%%%%%%%%%%%%%%
\section{\code{PMIx_Lookup_nb}}
\declareapi{PMIx_Lookup_nb}

%%%%
\summary

Nonblocking version of \refapi{PMIx_Lookup}.

%%%%
\format

\copySignature{PMIx_Lookup_nb}{1.0}{
pmix_status_t \\
PMIx_Lookup_nb(char **keys, \\
\hspace*{15\sigspace}const pmix_info_t info[], size_t ninfo, \\
\hspace*{15\sigspace}pmix_lookup_cbfunc_t cbfunc, void *cbdata);
}

\begin{arglist}
\argin{keys}{\code{NULL}-terminated array of keys (array of strings)}
\argin{info}{Array of info structures (array of handles)}
\argin{ninfo}{Number of elements in the \refarg{info} array (integer)}
\argin{cbfunc}{Callback function (handle)}
\argin{cbdata}{Callback data to be provided to the callback function (pointer)}
\end{arglist}

Returns one of the following:

\begin{itemize}
    \item \refconst{PMIX_SUCCESS}, indicating that the request is being processed by the host environment - result will be returned in the provided \refarg{cbfunc}. Note that the library must not invoke the callback function prior to returning from the \ac{API}.
    \item a PMIx error constant indicating an error in the input - the \refarg{cbfunc} will \textit{not} be called.
\end{itemize}

If executed, the status returned in the provided callback function will be one of the following constants:

\begin{itemize}
\item \refconst{PMIX_SUCCESS} All data was found and has been returned.

\item \refconst{PMIX_ERR_NOT_FOUND} None of the requested data was available within the requester's range. The \refarg{pdata} array in the callback function shall be \code{NULL} and the \refarg{npdata} parameter set to zero.

\item \refconst{PMIX_ERR_PARTIAL_SUCCESS} Some of the requested data was found.
Only found data will be included in the returned \refarg{pdata} array. Note that the specific reason for a particular piece of missing information (e.g., lack of permissions) cannot be communicated back to the requester in this situation.

\item \refconst{PMIX_ERR_NOT_SUPPORTED} There is no available datastore (either at the host environment or \ac{PMIx} implementation level) on this system that supports this function.

\item \refconst{PMIX_ERR_NO_PERMISSIONS} All of the requested data was found and range restrictions were met for each specified key, but none of the matching data could be returned due to lack of access permissions.

\item a non-zero \ac{PMIx} error constant indicating a reason for the request's failure.
\end{itemize}

\reqattrstart
\ac{PMIx} libraries are not required to directly support any attributes for this function. However, any provided attributes must be passed to the host environment for processing, and the \ac{PMIx} library is required to add the \refAttributeItem{PMIX_USERID} and the \refAttributeItem{PMIX_GRPID} attributes of the client process that is requesting the info.

\reqattrend

\optattrstart
The following attributes are optional for host environments that support this operation:

\pasteAttributeItem{PMIX_TIMEOUT}
\pasteAttributeItem{PMIX_RANGE}
\pasteAttributeItem{PMIX_WAIT}

\optattrend

%%%%
\descr

Non-blocking form of the \refapi{PMIx_Lookup} function.


%%%%%%%%%%%%%%%%%%%%%%%%%%%%%%%%%%%%%%%%%%%%%%%%%
\subsection{Lookup Returned Data Structure}
\declarestruct{pmix_pdata_t}

The \refstruct{pmix_pdata_t} structure is used by \refapi{PMIx_Lookup} to describe the data being accessed.

\copySignature{pmix_pdata_t}{1.0}{
typedef struct pmix_pdata \{ \\
\hspace*{4\sigspace}pmix_proc_t proc; \\
\hspace*{4\sigspace}pmix_key_t key; \\
\hspace*{4\sigspace}pmix_value_t value; \\
\} pmix_pdata_t;
}

where:
\begin{itemize}
    \item \emph{proc} is the process identifier of the data publisher.
    \item \emph{key} is the string key of the published data.
    \item \emph{value} is the value associated with the \emph{key}.
\end{itemize}


%%%%%%%%%%%%%%%%%%%%%%%%%%%%%%%%%%%%%%%%%%%%%%%%%
\subsubsection{Lookup data structure support macros}

The following macros are provided to support the \refstruct{pmix_pdata_t} structure.

\littleheader{Initialize the pdata structure}
\declaremacro{PMIX_PDATA_CONSTRUCT}

Initialize the \refstruct{pmix_pdata_t} fields

\copySignature{PMIX_PDATA_CONSTRUCT}{1.0}{
PMIX_PDATA_CONSTRUCT(m)
}

\begin{arglist}
\argin{m}{Pointer to the structure to be initialized (pointer to \refstruct{pmix_pdata_t})}
\end{arglist}

\littleheader{Destruct the pdata structure}
\declaremacro{PMIX_PDATA_DESTRUCT}

Destruct the \refstruct{pmix_pdata_t} fields

\copySignature{PMIX_PDATA_DESTRUCT}{1.0}{
PMIX_PDATA_DESTRUCT(m)
}

\begin{arglist}
\argin{m}{Pointer to the structure to be destructed (pointer to \refstruct{pmix_pdata_t})}
\end{arglist}

%%%%%%%%%%%
\littleheader{Create a pdata array}
\declaremacro{PMIX_PDATA_CREATE}

Allocate and initialize an array of \refstruct{pmix_pdata_t} structures

\copySignature{PMIX_PDATA_CREATE}{1.0}{
PMIX_PDATA_CREATE(m, n)
}

\begin{arglist}
\arginout{m}{Address where the pointer to the array of \refstruct{pmix_pdata_t} structures shall be stored (handle)}
\argin{n}{Number of structures to be allocated (\code{size_t})}
\end{arglist}


%%%%%%%%%%%
\littleheader{Free a pdata structure}
\declaremacro{PMIX_PDATA_RELEASE}

Release a \refstruct{pmix_pdata_t} structure

\copySignature{PMIX_PDATA_RELEASE}{4.0}{
PMIX_PDATA_RELEASE(m)
}

\begin{arglist}
\argin{m}{Pointer to a \refstruct{pmix_pdata_t} structure (handle)}
\end{arglist}


%%%%%%%%%%%
\littleheader{Free a pdata array}
\declaremacro{PMIX_PDATA_FREE}

Release an array of \refstruct{pmix_pdata_t} structures

\copySignature{PMIX_PDATA_FREE}{1.0}{
PMIX_PDATA_FREE(m, n)
}

\begin{arglist}
\argin{m}{Pointer to the array of \refstruct{pmix_pdata_t} structures (handle)}
\argin{n}{Number of structures in the array (\code{size_t})}
\end{arglist}

%%%%%%%%%%%
\littleheader{Load a lookup data structure}
\declaremacro{PMIX_PDATA_LOAD}

This macro simplifies the loading of key, process identifier, and data into a \refstruct{pmix_pdata_t} by correctly assigning values to the structure's fields.

\copySignature{PMIX_PDATA_LOAD}{1.0}{
PMIX_PDATA_LOAD(m, p, k, d, t);
}

\begin{arglist}
\argin{m}{Pointer to the \refstruct{pmix_pdata_t} structure into which the key and data are to be loaded (pointer to \refstruct{pmix_pdata_t})}
\argin{p}{Pointer to the \refstruct{pmix_proc_t} structure containing the identifier of the process being referenced (pointer to \refstruct{pmix_proc_t})}
\argin{k}{String key to be loaded - must be less than or equal to \refconst{PMIX_MAX_KEYLEN} in length (handle)}
\argin{d}{Pointer to the data value to be loaded (handle)}
\argin{t}{Type of the provided data value (\refstruct{pmix_data_type_t})}
\end{arglist}

\adviceuserstart
Key, process identifier, and data will all be copied into the \refstruct{pmix_pdata_t} - thus, the source information can be modified or free'd without affecting the copied data once the macro has completed.
\adviceuserend

%%%%%%%%%%%
\littleheader{Transfer a lookup data structure}
\declaremacro{PMIX_PDATA_XFER}

This macro simplifies the transfer of key, process identifier, and data value between two\refstruct{pmix_pdata_t} structures.

\copySignature{PMIX_PDATA_XFER}{2.0}{
PMIX_PDATA_XFER(d, s);
}

\begin{arglist}
\argin{d}{Pointer to the destination \refstruct{pmix_pdata_t} (pointer to \refstruct{pmix_pdata_t})}
\argin{s}{Pointer to the source \refstruct{pmix_pdata_t} (pointer to \refstruct{pmix_pdata_t})}
\end{arglist}

\adviceuserstart
Key, process identifier, and data will all be copied into the destination \refstruct{pmix_pdata_t} - thus, the source \refstruct{pmix_pdata_t} may free'd without affecting the copied data once the macro has completed.
\adviceuserend


%%%%%%%%%%%%%%%%%%%%%%%%%%%%%%%%%%%%%%%%%%%%%%%%%
\subsection{Lookup Callback Function}
\declareapi{pmix_lookup_cbfunc_t}

%%%%
\summary

The \refapi{pmix_lookup_cbfunc_t} is used by \refapi{PMIx_Lookup_nb} to return data.

\copySignature{pmix_lookup_cbfunc_t}{1.0}{
typedef void (*pmix_lookup_cbfunc_t) \\
\hspace*{4\sigspace}(pmix_status_t status, \\
\hspace*{5\sigspace}pmix_pdata_t data[], size_t ndata, \\
\hspace*{5\sigspace}void *cbdata);
}

\begin{arglist}
\argin{status}{Status associated with the operation (handle)}
\argin{data}{Array of data returned (\refstruct{pmix_pdata_t})}
\argin{ndata}{Number of elements in the \argref{data} array (\code{size_t})}
\argin{cbdata}{Callback data passed to original API call (memory reference)}
\end{arglist}


%%%%
\descr

A callback function for calls to \refapi{PMIx_Lookup_nb}.
The function will be called upon completion of the \refapi{PMIx_Lookup_nb} \ac{API} with the \refarg{status} indicating the success or failure of the request.
Any retrieved data will be returned in an array of \refstruct{pmix_pdata_t} structs.
The namespace and rank of the process that provided each data element is also returned.

Note that the \refstruct{pmix_pdata_t} structures will be released upon return from the callback function, so the receiver must copy/protect the data prior to returning if it needs to be retained.


%%%%%%%%%%%%%%%%%%%%%%%%%%%%%%%%%%%%%%%%%%%%%%%%%
%%%%%%%%%%%%%%%%%%%%%%%%%%%%%%%%%%%%%%%%%%%%%%%%%
\section{Retrieval rules for published data}
\label{chap:pub:retrules}

The retrieval rules for published data primarily revolve around enforcing data access permissions and range constraints. The datastore shall search its stored information for each specified key according to the following precedence logic:

\begin{enumerate}
    \item If the requester specified the range, then the search shall be constrained to data where the publishing process falls within the specified range.

    \item If the key of the stored information does not match the specified key, then the search will continue.

    \item If the requester's identifier does not fall within the range specified by the publisher, then the search will continue.

    \item If the publisher specified access permissions, the effective \ac{UID} and \ac{GID} of the requester shall be checked against those permissions, with the datastore rejecting the match if the requester fails to meet the requirements.

    \item If all of the above checks pass, then the value is added to the information that is to be returned.
\end{enumerate}

The status returned by the datastore shall be set to:

\begin{itemize}
\item \refconst{PMIX_SUCCESS} All data was found and is included in the returned information.

\item \refconst{PMIX_ERR_NOT_FOUND} None of the requested data could be found within a requester's range.

\item \refconst{PMIX_ERR_PARTIAL_SUCCESS} Some of the requested data was found.
Only found data will be included in the returned information. Note that the specific reason for a particular piece of missing information (e.g., lack of permissions) cannot be communicated back to the requester in this situation.

\item a non-zero \ac{PMIx} error constant indicating a reason for the request's failure.
\end{itemize}

In the case where data was found and range restrictions were met for each specified key, but none of the matching data could be returned due to lack of access permissions, the datastore must return the  \refconst{PMIX_ERR_NO_PERMISSIONS} error.

\adviceuserstart
Note that duplicate keys are allowed to exist on different ranges, and that ranges do overlap each other. Thus, if duplicate keys are published on overlapping ranges, it is possible for the datastore to successfully find multiple responses for a given key should publisher and requester specify sufficiently broad ranges. In this situation, the choice of resolving the duplication is left to the datastore implementation - e.g., it may return the first value found in its search, or the value corresponding to the most limited range of the found values, or it may choose to simply return an error.

Users are advised to avoid this ambiguity by careful selection of key values and ranges - e.g., by creating range-specific keys where necessary.
\adviceuserend


%%%%%%%%%%%%%%%%%%%%%%%%%%%%%%%%%%%%%%%%%%%%%%%%%
%%%%%%%%%%%%%%%%%%%%%%%%%%%%%%%%%%%%%%%%%%%%%%%%%
\section{\code{PMIx_Unpublish}}
\declareapi{PMIx_Unpublish}

%%%%
\summary

Unpublish data posted by this process using the given keys.

%%%%
\format

\copySignature{PMIx_Unpublish}{1.0}{
pmix_status_t \\
PMIx_Unpublish(char **keys, \\
\hspace*{15\sigspace}const pmix_info_t info[], size_t ninfo);
}

\begin{arglist}
\argin{keys}{\code{NULL}-terminated array of keys (array of strings)}
\argin{info}{Array of info structures (array of handles)}
\argin{ninfo}{Number of elements in the \refarg{info} array (integer)}
\end{arglist}

Returns \refconst{PMIX_SUCCESS} or a negative value corresponding to a PMIx error constant.

\reqattrstart
\ac{PMIx} libraries are not required to directly support any attributes for this function. However, any provided attributes must be passed to the host environment for processing, and the \ac{PMIx} library is required to add the \refAttributeItem{PMIX_USERID} and the \refAttributeItem{PMIX_GRPID} attributes of the client process that is requesting the operation.

\reqattrend

\optattrstart
The following attributes are optional for host environments that support this operation:

\pasteAttributeItem{PMIX_TIMEOUT}
\pasteAttributeItem{PMIX_RANGE}

\optattrend

%%%%
\descr

Unpublish data posted by this process using the given \refarg{keys}.
The function will block until the data has been removed by the server (i.e., it is safe to publish that key again within the specified range).
A value of \code{NULL} for the \refarg{keys} parameter instructs the server to remove all data published by this process.

By default, the range is assumed to be \refconst{PMIX_RANGE_SESSION}.
Changes to the range, and any additional directives, can be provided in the \refarg{info} array.


%%%%%%%%%%%%%%%%%%%%%%%%%%%%%%%%%%%%%%%%%%%%%%%%%
%%%%%%%%%%%%%%%%%%%%%%%%%%%%%%%%%%%%%%%%%%%%%%%%%
\section{\code{PMIx_Unpublish_nb}}
\declareapi{PMIx_Unpublish_nb}

%%%%
\summary

Nonblocking version of \refapi{PMIx_Unpublish}.

%%%%
\format

\copySignature{PMIx_Unpublish_nb}{1.0}{
pmix_status_t \\
PMIx_Unpublish_nb(char **keys, \\
\hspace*{18\sigspace}const pmix_info_t info[], size_t ninfo, \\
\hspace*{18\sigspace}pmix_op_cbfunc_t cbfunc, void *cbdata);
}

\begin{arglist}
\argin{keys}{\code{NULL}-terminated array of keys (array of strings)}
\argin{info}{Array of info structures (array of handles)}
\argin{ninfo}{Number of elements in the \refarg{info} array (integer)}
\argin{cbfunc}{Callback function \refapi{pmix_op_cbfunc_t} (function reference)}
\argin{cbdata}{Data to be passed to the callback function (memory reference)}
\end{arglist}

Returns one of the following:

\begin{itemize}
    \item \refconst{PMIX_SUCCESS}, indicating that the request is being processed by the host environment - result will be returned in the provided \refarg{cbfunc}. Note that the library must not invoke the callback function prior to returning from the \ac{API}.
    \item \refconst{PMIX_OPERATION_SUCCEEDED}, indicating that the request was immediately processed and returned \textit{success} - the \refarg{cbfunc} will \textit{not} be called.
    \item a PMIx error constant indicating either an error in the input or that the request was immediately processed and failed - the \refarg{cbfunc} will \textit{not} be called.
\end{itemize}

\reqattrstart
\ac{PMIx} libraries are not required to directly support any attributes for this function. However, any provided attributes must be passed to the host environment for processing, and the \ac{PMIx} library is required to add the \refAttributeItem{PMIX_USERID} and the \refAttributeItem{PMIX_GRPID} attributes of the client process that is requesting the operation.

\reqattrend

\optattrstart
The following attributes are optional for host environments that support this operation:

\pasteAttributeItem{PMIX_TIMEOUT}
\pasteAttributeItem{PMIX_RANGE}

\optattrend

%%%%
\descr

Non-blocking form of the \refapi{PMIx_Unpublish} function.
The callback function will be executed once the server confirms removal of the specified data. The \refarg{info} array must be maintained until the callback is provided.


%%%%%%%%%%%%%%%%%%%%%%%%%%%%%%%%%%%%%%%%%%%%%%%%%


    % Event Handling
    %  - (de)register_event, notify_event
    %%%%%%%%%%%%%%%%%%%%%%%%%%%%%%%%%%%%%%%%%%%%%%%%%
% Chapter: Events
%%%%%%%%%%%%%%%%%%%%%%%%%%%%%%%%%%%%%%%%%%%%%%%%%
\chapter{Event Notification}
\label{chap:api_event}

This chapter defines the \ac{PMIx} event notification system.
These interfaces are designed to support the reporting of events to/from clients and servers, and between library layers within a single process.

%%%%%%%%%%%%%%%%%%%%%%%%%%%%%%%%%%%%%%%%%%%%%%%%%
%%%%%%%%%%%%%%%%%%%%%%%%%%%%%%%%%%%%%%%%%%%%%%%%%
\section{Notification and Management}
\label{chap:api_event:notify}

\ac{PMIx} event notification provides an asynchronous out-of-band mechanism for communicating events between application processes and/or elements of the \ac{SMS}. Its uses span a wide range including fault notification, coordination between multiple programming libraries within a single process, and workflow orchestration for non-synchronous programming models. Events can be divided into two distinct classes:

\begin{itemize}
\item \textit{Job-specific events} directly relate to a job executing within the session, such as a debugger attachment, process failure within a related job, or events generated by an application process. Events in this category are to be immediately delivered to the \ac{PMIx} server library for relay to the related local processes.

\item \textit{Environment events} indirectly relate to a job but do not specifically target the job itself. This category includes \ac{SMS}-generated events such as \ac{ECC} errors, temperature excursions, and other non-job conditions that might directly affect a session's resources, but would never include an event generated by an application process. Note that although these do potentially impact the session's jobs, they are not directly tied to those jobs. Thus, events in this category are to be delivered to the \ac{PMIx} server library only upon request.
\end{itemize}

Both \ac{SMS} elements and applications can register for events of either type.

\adviceimplstart
Race conditions can cause the registration to come after events of possible interest (e.g., a memory \ac{ECC} event that occurs after start of execution but prior to registration, or an application process generating an event prior to another process registering to receive it). \ac{SMS} vendors are \textit{requested} to cache environment events for some time to mitigate this situation, but are not \textit{required} to do so. However, \ac{PMIx} implementers are \textit{required} to cache all events received by the \ac{PMIx} server library and to deliver them to registering clients in the same order in which they were received
\adviceimplend

\adviceuserstart
Applications must be aware that they may not receive environment events that occur prior to registration, depending upon the capabilities of the host \ac{SMS}.
\adviceuserend

The generator of an event can specify the \textit{target range} for delivery of that event. Thus, the generator can choose to limit notification to processes on the local node, processes within the same job as the generator, processes within the same allocation, other threads within the same process, only the \ac{SMS} (i.e., not to any application processes), all application processes, or to a custom range based on specific process identifiers. Only processes within the given range that register for the provided event code will be notified. In addition, the generator can use attributes to direct that the event not be delivered to any default event handlers, or to any multi-code handler (as defined below).

Event notifications provide the process identifier of the source of the event plus the event code and any additional information provided by the generator. When an event notification is received by a process, the registered handlers are scanned for their event code(s), with matching handlers assembled into an \textit{event chain} for servicing. Note that users can also specify a \textit{source range} when registering an event (using the same range designators described above) to further limit when they are to be invoked. When assembled, PMIx event chains are ordered based on both the specificity of the event handler and user directives at time of handler registration. By default, handlers are grouped into three categories based on the number of event codes that can trigger the callback:
\begin{itemize}
%
\item \textit{single-code} handlers are serviced first as they are the most specific. These are handlers that are registered against one specific event code.
%
\item \textit{multi-code} handlers are serviced once all single-code handlers have completed. The handler will be included in the chain upon receipt of an event matching any of the provided codes.
%
\item \textit{default} handlers are serviced once all multi-code handlers have completed. These handlers are always included in the chain unless the generator specifically excludes them.
%
\end{itemize}

Users can specify the callback order of a handler within its category at the time of registration. Ordering can be specified either by providing the relevant returned event handler registration ID or using event handler names, if the user specified an event handler name when registering the corresponding event. Thus, users can specify that a given handler be executed before or after another handler should both handlers appear in an event chain (the ordering is ignored if the other handler isn't included). Note that ordering does not imply immediate relationships. For example, multiple handlers registered to be serviced after event handler \textit{A} will all be executed after \textit{A}, but are not guaranteed to be executed in any particular order amongst themselves.

In addition, one event handler can be declared as the \textit{first} handler to be executed in the chain. This handler will \textit{always} be called prior to any other handler, regardless of category, provided the incoming event matches both the specified range and event code. Only one handler can be so designated --- attempts to designate additional handlers as \textit{first} will return an error. Deregistration of the declared \textit{first} handler will re-open the position for subsequent assignment.

Similarly, one event handler can be declared as the \textit{last} handler to be executed in the chain. This handler will \textit{always} be called after all other handlers have executed, regardless of category, provided the incoming event matches both the specified range and event code. Note that this handler will not be called if the chain is terminated by an earlier handler. Only one handler can be designated as \textit{last} --- attempts to designate additional handlers as \textit{last} will return an error. Deregistration of the declared \textit{last} handler will re-open the position for subsequent assignment.

\adviceuserstart
Note that the \textit{last} handler is called \textit{after} all registered default handlers that match the specified range of the incoming event unless a handler prior to it terminates the chain. Thus, if the application intends to define a \textit{last} handler, it should ensure that no default handler aborts the process before it.
\adviceuserend

Upon completing its work and prior to returning, each handler \textit{must} call the event handler completion function provided when it was invoked (including a status code plus any information to be passed to later handlers) so that the chain can continue being progressed. \ac{PMIx} automatically aggregates the status and any results of each handler (as provided in the completion callback) with status from all prior handlers so that each step in the chain has full knowledge of what preceded it. An event handler can terminate all further progress along the chain by passing the \refconst{PMIX_EVENT_ACTION_COMPLETE} status to the completion callback function.

\subsection{Events versus status constants}
\label{api:event:evssc}

Return status constants (see Section \ref{api:struct:errors}) represent values that can be returned from or passed into \ac{PMIx}
\acp{API}. These are distinct from \ac{PMIx} \emph{events} in that they are
not values that can be registered against event handlers. In general, the two
types of constants are distinguished by inclusion of an "ERR" in the name of
error constants versus an "EVENT" in events, though there are exceptions (e.g,
the \refconst{PMIX_SUCCESS} constant).


%%%%%%%%%%%%%%%%%%%%%%%%%%%%%%%%%%%%%%%%%%%%%%%%%
\subsection{\code{PMIx_Register_event_handler}}
\declareapi{PMIx_Register_event_handler}

%%%%
\summary

Register an event handler.

%%%%
\format

\copySignature{PMIx_Register_event_handler}{2.0}{
pmix_status_t \\
PMIx_Register_event_handler(pmix_status_t codes[], size_t ncodes, \\
\hspace*{28\sigspace}pmix_info_t info[], size_t ninfo, \\
\hspace*{28\sigspace}pmix_notification_fn_t evhdlr, \\
\hspace*{28\sigspace}pmix_hdlr_reg_cbfunc_t cbfunc, \\
\hspace*{28\sigspace}void *cbdata);
}

\begin{arglist}
\argin{codes}{Array of status codes (array of \refstruct{pmix_status_t})}
\argin{ncodes}{Number of elements in the \refarg{codes} array (\code{size_t})}
\argin{info}{Array of info structures (array of handles)}
\argin{ninfo}{Number of elements in the \refarg{info} array (\code{size_t})}
\argin{evhdlr}{Event handler to be called \refapi{pmix_notification_fn_t} (function reference)}
\argin{cbfunc}{Callback function \refapi{pmix_hdlr_reg_cbfunc_t} (function reference)}
\argin{cbdata}{Data to be passed to the cbfunc callback function (memory reference)}
\end{arglist}


If \refarg{cbfunc} is \code{NULL}, the function call will be treated as a \emph{blocking} call. In this case, the returned status will be either (a) the event handler reference identifier if the value is greater than or equal to zero, or (b) a negative error code indicative of the reason for the failure.

If the \refarg{cbfunc} is non-\code{NULL}, the function call will be treated as a \emph{non-blocking} call and will return the following:

\begin{itemize}
\item \refconst{PMIX_SUCCESS} indicating that the request has been accepted for processing and the provided callback function will be executed upon completion of the operation. Note that the library must not invoke the callback function prior to returning from the \ac{API}. The result of the registration operation shall be returned in the provided callback function along with the assigned event handler identifier.
\item \refconst{PMIX_ERR_EVENT_REGISTRATION} indicating that the registration
has failed for an undetermined reason.
\item a non-zero \ac{PMIx} error constant indicating a reason for the request to have been rejected. In this case, the provided callback function will not be executed.
\end{itemize}

The callback function must not be executed prior to returning from the \ac{API}, and no events corresponding to this registration may be delivered prior to the completion of the registration callback function (\refarg{cbfunc}).

\reqattrstart
The following attributes are required to be supported by all \ac{PMIx} libraries:

\pasteAttributeItem{PMIX_EVENT_HDLR_NAME}
\pasteAttributeItem{PMIX_EVENT_HDLR_FIRST}
\pasteAttributeItem{PMIX_EVENT_HDLR_LAST}
\pasteAttributeItem{PMIX_EVENT_HDLR_FIRST_IN_CATEGORY}
\pasteAttributeItem{PMIX_EVENT_HDLR_LAST_IN_CATEGORY}
\pasteAttributeItem{PMIX_EVENT_HDLR_BEFORE}
\pasteAttributeItem{PMIX_EVENT_HDLR_AFTER}
\pasteAttributeItem{PMIX_EVENT_HDLR_PREPEND}
\pasteAttributeItem{PMIX_EVENT_HDLR_APPEND}
\pasteAttributeItem{PMIX_EVENT_CUSTOM_RANGE}
\pasteAttributeItem{PMIX_RANGE}
\pasteAttributeItem{PMIX_EVENT_RETURN_OBJECT}

\divider

Host environments that implement support for \ac{PMIx} event notification are required to support the following attributes when registering handlers - these attributes are used to direct that the handler should be invoked only when the event affects the indicated process(es):

\pasteAttributeItem{PMIX_EVENT_AFFECTED_PROC}
\pasteAttributeItem{PMIX_EVENT_AFFECTED_PROCS}

\reqattrend


%%%%
\descr

Register an event handler to report events. Note that the codes being registered do \textit{not} need to be \ac{PMIx} error constants --- any integer value can be registered. This allows for registration of non-PMIx events such as those defined by a particular \ac{SMS} vendor or by an application itself.

\adviceuserstart
In order to avoid potential conflicts, users are advised to only define codes that lie outside the range of the \ac{PMIx} standard's error codes. Thus, \ac{SMS} vendors and application developers should constrain their definitions to positive values or negative values beyond the \refconst{PMIX_EXTERNAL_ERR_BASE} boundary.
\adviceuserend


\adviceuserstart
As previously stated, upon completing its work, and prior to returning, each handler \textit{must} call the event handler completion function provided when it was invoked (including a status code plus any information to be passed to later handlers) so that the chain can continue being progressed. An event handler can terminate all further progress along the chain by passing the \refconst{PMIX_EVENT_ACTION_COMPLETE} status to the completion callback function. Note that the parameters passed to the event handler (e.g., the \refarg{info} and \refarg{results} arrays) will cease to be valid once the completion function has been called - thus, any information in the incoming parameters that will be referenced following the call to the completion function must be copied.
\adviceuserend

%%%%%%%%%%%%%%%%%%%%%%%%%%%%%%%%%%%%%%%%%%%%%%%%%
\subsection{Event registration constants}
\label{api:struct:constants:event}

\begin{constantdesc}
%
\declareconstitem{PMIX_ERR_EVENT_REGISTRATION}
Error in event registration.
%
\end{constantdesc}

%%%%%%%%%%%%%%%%%%%%%%%%%%%%%%%%%%%%%%%%%%%%%%%%%
\subsection{System events}
\label{api:struct:sys:event}

\begin{constantdesc}
%
\declareconstitemNEW{PMIX_EVENT_SYS_BASE}
Mark the beginning of a dedicated range of constants for system event reporting.
%
\declareconstitemNEW{PMIX_EVENT_NODE_DOWN}
A node has gone down - the identifier of the affected node will be included in the notification.
%
\declareconstitemNEW{PMIX_EVENT_NODE_OFFLINE}
A node has been marked as \emph{offline} - the identifier of the affected node will be included in the notification.
%
\declareconstitemNEW{PMIX_EVENT_SYS_OTHER}
Mark the end of a dedicated range of constants for system event reporting.
%
\end{constantdesc}

\littleheader{Detect system event constant}
\declaremacro{PMIX_SYSTEM_EVENT}

Test a given event constant to see if it falls within the dedicated range of constants for system event reporting.

\copySignature{PMIX_SYSTEM_EVENT}{2.2}{
PMIX_SYSTEM_EVENT(a)
}

\begin{arglist}
\argin{a}{Error constant to be checked (\refstruct{pmix_status_t})}
\end{arglist}

Returns \code{true} if the provided values falls within the dedicated range of events for system event reporting.

%%%%%%%%%%%%%%%%%%%%%%%%%%%%%%%%%%%%%%%%%%%%%%%%%
\subsection{Event handler registration and notification attributes}
\label{api:struct:attributes:event}

Attributes to support event registration and notification.

%
\declareAttribute{PMIX_EVENT_HDLR_NAME}{"pmix.evname"}{char*}{
String name identifying this handler.
}
%
\declareAttribute{PMIX_EVENT_HDLR_FIRST}{"pmix.evfirst"}{bool}{
Invoke this event handler before any other handlers.
}
%
\declareAttribute{PMIX_EVENT_HDLR_LAST}{"pmix.evlast"}{bool}{
Invoke this event handler after all other handlers have been called.
}
%
\declareAttribute{PMIX_EVENT_HDLR_FIRST_IN_CATEGORY}{"pmix.evfirstcat"}{bool}{
Invoke this event handler before any other handlers in this category.
}
%
\declareAttribute{PMIX_EVENT_HDLR_LAST_IN_CATEGORY}{"pmix.evlastcat"}{bool}{
Invoke this event handler after all other handlers in this category have been called.
}
%
\declareAttribute{PMIX_EVENT_HDLR_BEFORE}{"pmix.evbefore"}{char*}{
Put this event handler immediately before the one specified in the \code{(char*)} value.
}
%
\declareAttribute{PMIX_EVENT_HDLR_AFTER}{"pmix.evafter"}{char*}{
Put this event handler immediately after the one specified in the \code{(char*)} value.
}
%
\declareAttribute{PMIX_EVENT_HDLR_PREPEND}{"pmix.evprepend"}{bool}{
Prepend this handler to the precedence list within its category.
}
%
\declareAttribute{PMIX_EVENT_HDLR_APPEND}{"pmix.evappend"}{bool}{
Append this handler to the precedence list within its category.
}
%
\declareAttribute{PMIX_EVENT_CUSTOM_RANGE}{"pmix.evrange"}{pmix_data_array_t*}{
Array of \refstruct{pmix_proc_t} defining range of event notification.
}
%
\declareAttribute{PMIX_EVENT_AFFECTED_PROC}{"pmix.evproc"}{pmix_proc_t}{
The single process that was affected.
}
%
\declareAttribute{PMIX_EVENT_AFFECTED_PROCS}{"pmix.evaffected"}{pmix_data_array_t*}{
Array of \refstruct{pmix_proc_t} defining affected processes.
}
%
\declareAttribute{PMIX_EVENT_NON_DEFAULT}{"pmix.evnondef"}{bool}{
Event is not to be delivered to default event handlers.
}
%
\declareAttribute{PMIX_EVENT_RETURN_OBJECT}{"pmix.evobject"}{void *}{
Object to be returned whenever the registered callback function \code{cbfunc} is invoked.
The object will only be returned to the process that registered it.
}
%
\declareAttribute{PMIX_EVENT_DO_NOT_CACHE}{"pmix.evnocache"}{bool}{
Instruct the \ac{PMIx} server not to cache the event.
}
%
\declareAttribute{PMIX_EVENT_PROXY}{"pmix.evproxy"}{pmix_proc_t*}{
\ac{PMIx} server that sourced the event.
}
%
\declareAttribute{PMIX_EVENT_TEXT_MESSAGE}{"pmix.evtext"}{char*}{
Text message suitable for output by recipient - e.g., describing the cause of the event.
}
%
\declareAttributeNEW{PMIX_EVENT_TIMESTAMP}{"pmix.evtstamp"}{time_t}{
System time when the associated event occurred.
}

%%%%%%%%%%%%%%%%%%%%%%%%%%%%%%%%%%%%%%%%%%%%%%%%%
\subsubsection{Fault tolerance event attributes}
\label{api:struct:attributes:ft}

The following attributes may be used by the host environment when providing an event notification as qualifiers indicating the action it intends to take in response to the event:

%
\declareAttribute{PMIX_EVENT_TERMINATE_SESSION}{"pmix.evterm.sess"}{bool}{
The \ac{RM} intends to terminate this session.
}
%
\declareAttribute{PMIX_EVENT_TERMINATE_JOB}{"pmix.evterm.job"}{bool}{
The \ac{RM} intends to terminate this job.
}
%
\declareAttribute{PMIX_EVENT_TERMINATE_NODE}{"pmix.evterm.node"}{bool}{
The \ac{RM} intends to terminate all processes on this node.
}
%
\declareAttribute{PMIX_EVENT_TERMINATE_PROC}{"pmix.evterm.proc"}{bool}{
The \ac{RM} intends to terminate just this process.
}
%
\declareAttribute{PMIX_EVENT_ACTION_TIMEOUT}{"pmix.evtimeout"}{int}{
The time in seconds before the \ac{RM} will execute the indicated operation.
}

%%%%%%%%%%%%%%%%%%%%%%%%%%%%%%%%%%%%%%%%%%%%%%%%%
\subsubsection{Hybrid programming event attributes}
\label{api:struct:attributes:hybrid}

The following attributes may be used by programming models to coordinate their use of common resources within a process in conjunction with the \refconst{PMIX_OPENMP_PARALLEL_ENTERED} event:
%
\pasteAttributeItem{PMIX_MODEL_PHASE_NAME}
\pasteAttributeItem{PMIX_MODEL_PHASE_TYPE}

%%%%%%%%%%%%%%%%%%%%%%%%%%%%%%%%%%%%%%%%%%%%%%%%%
\subsection{Notification Function}
\declareapi{pmix_notification_fn_t}

%%%%
\summary

The \refapi{pmix_notification_fn_t} is called by \ac{PMIx} to deliver notification of an event.

\adviceuserstart
The \ac{PMIx} \textit{ad hoc} v1.0 Standard defined an error notification function with an identical name, but different signature than the v2.0 Standard described below. The \textit{ad hoc} v1.0 version was removed from the v2.0 Standard is not included in this document to avoid confusion.
\adviceuserend


\copySignature{pmix_notification_fn_t}{2.0}{
typedef void (*pmix_notification_fn_t) \\
\hspace*{4\sigspace}(size_t evhdlr_registration_id, \\
\hspace*{5\sigspace}pmix_status_t status, \\
\hspace*{5\sigspace}const pmix_proc_t *source, \\
\hspace*{5\sigspace}pmix_info_t info[], size_t ninfo, \\
\hspace*{5\sigspace}pmix_info_t results[], size_t nresults, \\
\hspace*{5\sigspace}pmix_event_notification_cbfunc_fn_t cbfunc, \\
\hspace*{5\sigspace}void *cbdata);
}

\begin{arglist}
\argin{evhdlr_registration_id}{Registration number of the handler being called (\code{size_t})}
\argin{status}{Status associated with the operation (\refstruct{pmix_status_t})}
\argin{source}{Identifier of the process that generated the event (\refstruct{pmix_proc_t})}. If the source is the \ac{SMS}, then the nspace will be empty and the rank will be PMIX_RANK_UNDEF
\argin{info}{Information describing the event (\refstruct{pmix_info_t})}. This argument will be NULL if no additional information was provided by the event generator.
\argin{ninfo}{Number of elements in the info array (\code{size_t})}
\argin{results}{Aggregated results from prior event handlers servicing this event (\refstruct{pmix_info_t})}. This argument will be \code{NULL} if this is the first handler servicing the event, or if no prior handlers provided results.
\argin{nresults}{Number of elements in the results array (\code{size_t})}
\argin{cbfunc}{\refapi{pmix_event_notification_cbfunc_fn_t} callback function to be executed upon completion of the handler's operation and prior to handler return (function reference)}.
\argin{cbdata}{Callback data to be passed to cbfunc (memory reference)}
\end{arglist}

%%%%
\descr

Note that different \acp{RM} may provide differing levels of support for event notification to application processes. Thus, the \refarg{info} array may be \code{NULL} or may contain detailed information of the event. It is the responsibility of the application to parse any provided info array for defined key-values if it so desires.

\adviceuserstart
Possible uses of the \refarg{info} array include:

\begin{itemize}
%
\item for the host \ac{RM} to alert the process as to planned actions, such as aborting the session, in response to the reported event
%
\item provide a timeout for alternative action to occur, such as for the application to request an alternate response to the event
%
\end{itemize}

For example, the \ac{RM} might alert the application to the failure of a node that resulted in termination of several processes, and indicate that the overall session will be aborted unless the application requests an alternative behavior in the next 5 seconds. The application then has time to respond with a checkpoint request, or a request to recover from the failure by obtaining replacement nodes and restarting from some earlier checkpoint.

Support for these options is left to the discretion of the host \ac{RM}. Info keys are included in the common definitions above but may be augmented by environment vendors.
\adviceuserend

\advicermstart
On the server side, the notification function is used to inform the \ac{PMIx} server library's host of a detected event in the \ac{PMIx} server library. Events generated by \ac{PMIx} clients are communicated to the \ac{PMIx} server library, but will be relayed to the host via the \refapi{pmix_server_notify_event_fn_t} function pointer, if provided.
\advicermend


%%%%%%%%%%%%%%%%%%%%%%%%%%%%%%%%%%%%%%%%%%%%%%%%%
\subsection{\code{PMIx_Deregister_event_handler}}
\declareapi{PMIx_Deregister_event_handler}

%%%%
\summary

Deregister an event handler.

%%%%
\format

\copySignature{PMIx_Deregister_event_handler}{2.0}{
pmix_status_t \\
PMIx_Deregister_event_handler(size_t evhdlr_ref, \\
\hspace*{30\sigspace}pmix_op_cbfunc_t cbfunc, \\
\hspace*{30\sigspace}void *cbdata);
}

\begin{arglist}
\argin{evhdlr_ref}{Event handler ID returned by registration (\code{size_t})}
\argin{cbfunc}{Callback function to be executed upon completion of operation \refapi{pmix_op_cbfunc_t} (function reference)}
\argin{cbdata}{Data to be passed to the cbfunc callback function (memory reference)}
\end{arglist}

If \refarg{cbfunc} is \code{NULL}, the function will be treated as a \emph{blocking} call and the result of the operation returned in the status code.

If \refarg{cbfunc} is non-\code{NULL}, the function will be treated as a \emph{non-blocking} call and return one of the following:

\begin{itemize}
\item \refconst{PMIX_SUCCESS}, indicating that the request is being processed - result will be returned in the provided \refarg{cbfunc}. Note that the library must not invoke the callback function prior to returning from the \ac{API}.
\item \refconst{PMIX_OPERATION_SUCCEEDED}, indicating that the request was immediately processed and returned \textit{success} - the \refarg{cbfunc} will \textit{not} be called
\item a PMIx error constant indicating either an error in the input or that the request was immediately processed and failed - the \refarg{cbfunc} will \textit{not} be called
\end{itemize}

The returned status code will be one of the following:

\begin{itemize}
\item \refconst{PMIX_SUCCESS} The event handler was successfully deregistered.
\item \refconst{PMIX_ERR_BAD_PARAM} The provided \refarg{evhdlr_ref} was unrecognized.
\item \refconst{PMIX_ERR_NOT_SUPPORTED} The \ac{PMIx} implementation does not support event notification.
\end{itemize}

%%%%
\descr

Deregister an event handler. Note that no events corresponding to the referenced registration may be delivered following completion of the deregistration operation (either return from the \ac{API} with \refconst{PMIX_OPERATION_SUCCEEDED} or execution of the \refarg{cbfunc}).

%%%%%%%%%%%%%%%%%%%%%%%%%%%%%%%%%%%%%%%%%%%%%%%%%
\subsection{\code{PMIx_Notify_event}}
\declareapi{PMIx_Notify_event}

%%%%
\summary

Report an event for notification via any
registered event handler.

%%%%
\format

\copySignature{PMIx_Notify_event}{2.0}{
pmix_status_t \\
PMIx_Notify_event(pmix_status_t status, \\
\hspace*{18\sigspace}const pmix_proc_t *source, \\
\hspace*{18\sigspace}pmix_data_range_t range, \\
\hspace*{18\sigspace}pmix_info_t info[], size_t ninfo, \\
\hspace*{18\sigspace}pmix_op_cbfunc_t cbfunc, void *cbdata);
}

\begin{arglist}
\argin{status}{Status code of the event (\refstruct{pmix_status_t})}
\argin{source}{Pointer to a \refstruct{pmix_proc_t} identifying the original reporter of the event (handle)}
\argin{range}{Range across which this notification shall be delivered (\refstruct{pmix_data_range_t})}
\argin{info}{Array of \refstruct{pmix_info_t} structures containing any further info provided by the originator of the event (array of handles)}
\argin{ninfo}{Number of elements in the \refarg{info} array (\code{size_t})}
\argin{cbfunc}{Callback function to be executed upon completion of operation \refapi{pmix_op_cbfunc_t} (function reference)}
\argin{cbdata}{Data to be passed to the cbfunc callback function (memory reference)}
\end{arglist}

If \refarg{cbfunc} is \code{NULL}, the function will be treated as a \emph{blocking} call and the result of the operation returned in the status code.

If \refarg{cbfunc} is non-\code{NULL}, the function will be treated as a \emph{non-blocking} call and return one of the following:

\begin{itemize}
\item \refconst{PMIX_SUCCESS} The notification request is valid and is being processed. The callback function will be called when the process-local operation is complete and will provide the resulting status of that operation. Note that this does \textit{not} reflect the success or failure of delivering the event to any recipients. The callback function must not be executed prior to returning from the \ac{API}.
\item \refconst{PMIX_OPERATION_SUCCEEDED}, indicating that the request was immediately processed and returned \textit{success} - the \refarg{cbfunc} will \textit{not} be called
\item \refconst{PMIX_ERR_BAD_PARAM} The request contains at least one incorrect entry that prevents it from being processed. The callback function will \textit{not} be called.
\item \refconst{PMIX_ERR_NOT_SUPPORTED} The \ac{PMIx} implementation does not support event notification, or in the case of a \ac{PMIx} server calling the API, the range extended beyond the local node and the host \ac{SMS} environment does not support event notification. The callback function will \textit{not} be called.
\end{itemize}

\reqattrstart
The following attributes are required to be supported by all \ac{PMIx} libraries:

\pasteAttributeItem{PMIX_EVENT_NON_DEFAULT}
\pasteAttributeItem{PMIX_EVENT_CUSTOM_RANGE}
\pasteAttributeItem{PMIX_EVENT_DO_NOT_CACHE}
\pasteAttributeItem{PMIX_EVENT_PROXY}
\pasteAttributeItem{PMIX_EVENT_TEXT_MESSAGE}

\divider

Host environments that implement support for \ac{PMIx} event notification are required to provide the following attributes for all events generated by the environment:

\pasteAttributeItem{PMIX_EVENT_AFFECTED_PROC}
\pasteAttributeItem{PMIX_EVENT_AFFECTED_PROCS}

\reqattrend

\optattrstart
Host environments that support \ac{PMIx} event notification may offer notifications for environmental events impacting the job and for \ac{SMS} events relating to the job. The following attributes may optionally be included to indicate the host environment's intended response to the event:

\pasteAttributeItem{PMIX_EVENT_TERMINATE_SESSION}
\pasteAttributeItem{PMIX_EVENT_TERMINATE_JOB}
\pasteAttributeItem{PMIX_EVENT_TERMINATE_NODE}
\pasteAttributeItem{PMIX_EVENT_TERMINATE_PROC}
\pasteAttributeItem{PMIX_EVENT_ACTION_TIMEOUT}

\optattrend

%%%%
\descr

Report an event for notification via any registered event handler. This function can be called by any \ac{PMIx} process, including application processes, \ac{PMIx} servers, and \ac{SMS} elements. The \ac{PMIx} server calls this \ac{API} to report events it detected itself so that the host \ac{SMS} daemon distribute and handle them, and to pass events given to it by its host down to any attached client processes for processing. Examples might include notification of the failure of another process, detection of an impending node failure due to rising temperatures, or an intent to preempt the application. Events may be locally generated or come from anywhere in the system.

Host \ac{SMS} daemons call the \ac{API} to pass events down to its embedded \ac{PMIx} server both for transmittal to local client processes and for the host's own internal processing where the host has registered its own event handlers. The \ac{PMIx} server library is not allowed to echo any event given to it by its host via this \ac{API} back to the host through the \refapi{pmix_server_notify_event_fn_t} server module function. The host is required to deliver the event to all \ac{PMIx} servers where the targeted processes either are currently running, or (if they haven't started yet) might be running at some point in the future as the events are required to be cached by the \ac{PMIx} server library.

Client application processes can call this function to notify the \ac{SMS} and/or other application processes of an event it encountered. Note that processes are not constrained to report status values defined in the official \ac{PMIx} standard --- any integer value can be used. Thus, applications are free to define their own internal events and use the notification system for their own internal purposes.

\adviceuserstart
The callback function will be called upon completion of the
\code{notify_event} function's actions. At that time, any messages required for executing the operation (e.g., to send the notification to the local \ac{PMIx} server) will
have been queued, but may not yet have been transmitted. The caller is required to maintain the input
data until the callback function has been executed --- the sole purpose of the callback function is to indicate when the input data is no longer required.
\adviceuserend

%%%%%%%%%%%%%%%%%%%%%%%%%%%%%%%%%%%%%%%%%%%%%%%%%
\subsection{Notification Handler Completion Callback Function}
\declareapi{pmix_event_notification_cbfunc_fn_t}

%%%%
\summary

The \refapi{pmix_event_notification_cbfunc_fn_t} is called by event handlers to indicate completion of their operations.

\copySignature{pmix_event_notification_cbfunc_fn_t}{2.0}{
typedef void (*pmix_event_notification_cbfunc_fn_t) \\
\hspace*{4\sigspace}(pmix_status_t status, \\
\hspace*{5\sigspace}pmix_info_t *results, size_t nresults, \\
\hspace*{5\sigspace}pmix_op_cbfunc_t cbfunc, void *thiscbdata, \\
\hspace*{5\sigspace}void *notification_cbdata);
}

\begin{arglist}
\argin{status}{Status returned by the event handler's operation (\refstruct{pmix_status_t})}
\argin{results}{Results from this event handler's operation on the event (\refstruct{pmix_info_t})}
\argin{nresults}{Number of elements in the results array (\code{size_t})}
\argin{cbfunc}{\refapi{pmix_op_cbfunc_t} function to be executed when \ac{PMIx} completes processing the callback (function reference)}
\argin{thiscbdata}{Callback data that was passed in to the handler (memory reference)}
\argin{cbdata}{Callback data to be returned when \ac{PMIx} executes cbfunc (memory reference)}
\end{arglist}

%%%%
\descr

Define a callback by which an event handler can notify the \ac{PMIx} library that it has completed its response to the notification. The handler is \textit{required} to execute this callback so the library can determine if additional handlers need to be called. The handler shall return \refconst{PMIX_EVENT_ACTION_COMPLETE} if no further action is required. The return status of each event handler and any returned \refstruct{pmix_info_t} structures will be added to the \refarg{results} array of \refstruct{pmix_info_t} passed to any subsequent event handlers to help guide their operation.

If non-\code{NULL}, the provided callback function will be called to allow the event handler to release the provided info array and execute any other required cleanup operations.

%%%%%%%%%%%%%%%%%%%%%%%%%%%%%%%%%%%%%%%%%%%%%%%%%
\subsubsection{Completion Callback Function Status Codes}

The following status code may be returned indicating various actions taken by other event handlers.

\begin{constantdesc}
%
\declareconstitem{PMIX_EVENT_NO_ACTION_TAKEN}
Event handler: No action taken.
%
\declareconstitem{PMIX_EVENT_PARTIAL_ACTION_TAKEN}
Event handler: Partial action taken.
%
\declareconstitem{PMIX_EVENT_ACTION_DEFERRED}
Event handler: Action deferred.
%
\declareconstitem{PMIX_EVENT_ACTION_COMPLETE}
Event handler: Action complete.
%
\end{constantdesc}

%%%%%%%%%%%%%%%%%%%%%%%%%%%%%%%%%%%%%%%%%%%%%%%%%


    % Data Packing & Unpacking
    %  - (un)pack, copy
    %%%%%%%%%%%%%%%%%%%%%%%%%%%%%%%%%%%%%%%%%%%%%%%%%
% Chapter: Data Packing and Unpacking
%%%%%%%%%%%%%%%%%%%%%%%%%%%%%%%%%%%%%%%%%%%%%%%%%
\chapter{Data Packing and Unpacking}
\label{chap:api_data_mgmt}

\ldots

%%%%%%%%%%%%%%%%%%%%%%%%%%%%%%%%%%%%%%%%%%%%%%
%%%%%%%%%%%%%%%%%%%%%%%%%%%%%%%%%%%%%%%%%%%%%%
\section{General Routines}
\label{chap:api_init:general}

\ldots

%%%%%%%%%%%
\subsection{\code{PMIx_Data_pack}}
\declareapi{PMIx_Data_pack}

\cspecificstart
\begin{codepar}
/**
 * Top-level interface function to pack one or more values into a
 * buffer.
 *
 * The pack function packs one or more values of a specified type into
 * the specified buffer.  The buffer must have already been
 * initialized via the PMIX_DATA_BUFFER_CREATE or PMIX_DATA_BUFFER_CONSTRUCT
 * call - otherwise, the pack_value function will return an error.
 * Providing an unsupported type flag will likewise be reported as an error.
 *
 * Note that any data to be packed that is not hard type cast (i.e.,
 * not type cast to a specific size) may lose precision when unpacked
 * by a non-homogeneous recipient.  The PACK function will do its best to deal
 * with heterogeneity issues between the packer and unpacker in such
 * cases. Sending a number larger than can be handled by the recipient
 * will return an error code (generated upon unpacking) -
 * the error cannot be detected during packing.
 *
 * @param *buffer A pointer to the buffer into which the value is to
 * be packed.
 *
 * @param *src A void* pointer to the data that is to be packed. Note
 * that strings are to be passed as (char **) - i.e., the caller must
 * pass the address of the pointer to the string as the void*. This
 * allows PMIx to use a single pack function, but still allow
 * the caller to pass multiple strings in a single call.
 *
 * @param num_values An int32_t indicating the number of values that are
 * to be packed, beginning at the location pointed to by src. A string
 * value is counted as a single value regardless of length. The values
 * must be contiguous in memory. Arrays of pointers (e.g., string
 * arrays) should be contiguous, although (obviously) the data pointed
 * to need not be contiguous across array entries.
 *
 * @param type The type of the data to be packed - must be one of the
 * PMIX defined data types.
 *
 * @retval PMIX_SUCCESS The data was packed as requested.
 *
 * @retval PMIX_ERROR(s) An appropriate PMIX error code indicating the
 * problem encountered. This error code should be handled
 * appropriately.
 *
 * @code
 * pmix_data_buffer_t *buffer;
 * int32_t src;
 *
 * PMIX_DATA_BUFFER_CREATE(buffer);
 * status_code = PMIx_Data_pack(buffer, &src, 1, PMIX_INT32);
 * @endcode
 */
pmix_status_t
PMIx_Data_pack(pmix_data_buffer_t *buffer,
               void *src, int32_t num_vals,
               pmix_data_type_t type);
\end{codepar}
\cspecificend


%%%%%%%%%%%
\subsection{\code{PMIx_Data_unpack}}
\declareapi{PMIx_Data_unpack}

\cspecificstart
\begin{codepar}
/**
 * Unpack values from a buffer.
 *
 * The unpack function unpacks the next value (or values) of a
 * specified type from the specified buffer.
 *
 * The buffer must have already been initialized via an PMIX_DATA_BUFFER_CREATE or
 * PMIX_DATA_BUFFER_CONSTRUCT call (and assumedly filled with some data) -
 * otherwise, the unpack_value function will return an
 * error. Providing an unsupported type flag will likewise be reported
 * as an error, as will specifying a data type that DOES NOT match the
 * type of the next item in the buffer. An attempt to read beyond the
 * end of the stored data held in the buffer will also return an
 * error.
 *
 * NOTE: it is possible for the buffer to be corrupted and that
 * PMIx will *think* there is a proper variable type at the
 * beginning of an unpack region - but that the value is bogus (e.g., just
 * a byte field in a string array that so happens to have a value that
 * matches the specified data type flag). Therefore, the data type error check
 * is NOT completely safe. This is true for ALL unpack functions.
 *
 *
 * Unpacking values is a "nondestructive" process - i.e., the values are
 * not removed from the buffer. It is therefore possible for the caller
 * to re-unpack a value from the same buffer by resetting the unpack_ptr.
 *
 * Warning: The caller is responsible for providing adequate memory
 * storage for the requested data. As noted below, the user
 * must provide a parameter indicating the maximum number of values that
 * can be unpacked into the allocated memory. If more values exist in the
 * buffer than can fit into the memory storage, then the function will unpack
 * what it can fit into that location and return an error code indicating
 * that the buffer was only partially unpacked.
 *
 * Note that any data that was not hard type cast (i.e., not type cast
 * to a specific size) when packed may lose precision when unpacked by
 * a non-homogeneous recipient.  PMIx will do its best to deal with
 * heterogeneity issues between the packer and unpacker in such
 * cases. Sending a number larger than can be handled by the recipient
 * will return an error code generated upon unpacking - these errors
 * cannot be detected during packing.
 *
 * @param *buffer A pointer to the buffer from which the value will be
 * extracted.
 *
 * @param *dest A void* pointer to the memory location into which the
 * data is to be stored. Note that these values will be stored
 * contiguously in memory. For strings, this pointer must be to (char
 * **) to provide a means of supporting multiple string
 * operations. The unpack function will allocate memory for each
 * string in the array - the caller must only provide adequate memory
 * for the array of pointers.
 *
 * @param type The type of the data to be unpacked - must be one of
 * the BFROP defined data types.
 *
 * @retval *max_num_values The number of values actually unpacked. In
 * most cases, this should match the maximum number provided in the
 * parameters - but in no case will it exceed the value of this
 * parameter.  Note that if you unpack fewer values than are actually
 * available, the buffer will be in an unpackable state - the function will
 * return an error code to warn of this condition.
 *
 * @note The unpack function will return the actual number of values
 * unpacked in this location.
 *
 * @retval PMIX_SUCCESS The next item in the buffer was successfully
 * unpacked.
 *
 * @retval PMIX_ERROR(s) The unpack function returns an error code
 * under one of several conditions: (a) the number of values in the
 * item exceeds the max num provided by the caller; (b) the type of
 * the next item in the buffer does not match the type specified by
 * the caller; or (c) the unpack failed due to either an error in the
 * buffer or an attempt to read past the end of the buffer.
 *
 * @code
 * pmix_data_buffer_t *buffer;
 * int32_t dest;
 * char **string_array;
 * int32_t num_values;
 *
 * num_values = 1;
 * status_code = PMIx_Data_unpack(buffer, (void*)&dest, &num_values, PMIX_INT32);
 *
 * num_values = 5;
 * string_array = malloc(num_values*sizeof(char *));
 * status_code = PMIx_Data_unpack(buffer, (void*)(string_array), &num_values, PMIX_STRING);
 *
 * @endcode
 */
pmix_status_t
PMIx_Data_unpack(pmix_data_buffer_t *buffer, void *dest,
                 int32_t *max_num_values,
                 pmix_data_type_t type);
\end{codepar}
\cspecificend


%%%%%%%%%%%
\subsection{\code{PMIx_Data_copy}}
\declareapi{PMIx_Data_copy}

\cspecificstart
\begin{codepar}
/**
 * Copy a data value from one location to another.
 *
 * Since registered data types can be complex structures, the system
 * needs some way to know how to copy the data from one location to
 * another (e.g., for storage in the registry). This function, which
 * can call other copy functions to build up complex data types, defines
 * the method for making a copy of the specified data type.
 *
 * @param **dest The address of a pointer into which the
 * address of the resulting data is to be stored.
 *
 * @param *src A pointer to the memory location from which the
 * data is to be copied.
 *
 * @param type The type of the data to be copied - must be one of
 * the PMIx defined data types.
 *
 * @retval PMIX_SUCCESS The value was successfully copied.
 *
 * @retval PMIX_ERROR(s) An appropriate error code.
 *
 */
pmix_status_t
PMIx_Data_copy(void **dest, void *src,
               pmix_data_type_t type);
\end{codepar}
\cspecificend


%%%%%%%%%%%
\subsection{\code{PMIx_Data_print}}
\declareapi{PMIx_Data_print}

\cspecificstart
\begin{codepar}
/**
 * Print a data value.
 *
 * Since registered data types can be complex structures, the system
 * needs some way to know how to print them (i.e., convert them to a string
 * representation). Provided for debug purposes.
 *
 * @retval PMIX_SUCCESS The value was successfully printed.
 *
 * @retval PMIX_ERROR(s) An appropriate error code.
 */
pmix_status_t
PMIx_Data_print(char **output, char *prefix,
                void *src, pmix_data_type_t type);
\end{codepar}
\cspecificend


%%%%%%%%%%%
\subsection{\code{PMIx_Data_copy_payload}}
\declareapi{PMIx_Data_copy_payload}

\cspecificstart
\begin{codepar}
/**
 * Copy a payload from one buffer to another
 *
 * This function will append a copy of the payload in one buffer into
 * another buffer.
 * NOTE: This is NOT a destructive procedure - the
 * source buffer's payload will remain intact, as will any pre-existing
 * payload in the destination's buffer.
 */
pmix_status_t
PMIx_Data_copy_payload(pmix_data_buffer_t *dest,
                       pmix_data_buffer_t *src);
\end{codepar}
\cspecificend


%%%%%%%%%%%%%%%%%%%%%%%%%%%%%%%%%%%%%%%%%%%%%%%%%


    % Process Management
    %  - spawn, (dis)connect, resolve_peers
    %%%%%%%%%%%%%%%%%%%%%%%%%%%%%%%%%%%%%%%%%%%%%%%%%
% Chapter: Process Management
%%%%%%%%%%%%%%%%%%%%%%%%%%%%%%%%%%%%%%%%%%%%%%%%%
\chapter{Process Management}
\label{chap:api_proc_mgmt}

\ldots

%%%%%%%%%%%%%%%%%%%%%%%%%%%%%%%%%%%%%%%%%%%%%%
%%%%%%%%%%%%%%%%%%%%%%%%%%%%%%%%%%%%%%%%%%%%%%
\section{Abort}
\label{chap:api_proc_mgmt:abort}

\ldots

%%%%%%%%%%%
\subsection{\code{PMIx_Abort}}
\declareapi{PMIx_Abort}

%%%%
\summary

Abort the specified process.

%%%%
\format

\cspecificstart
\begin{codepar}
pmix_status_t
PMIx_Abort(int status, const char msg[],
           pmix_proc_t procs[], size_t nprocs)
\end{codepar}
\cspecificend

\begin{arglist}
\argin{status}{Error code to return to invoking environment (integer)}
\argin{msg}{String message to be returned to user (string)}
\argin{procs}{Array of \refstruct{pmix_proc_t} structures (array of handles)}
\argin{nprocs}{Number of elements in the \refarg{procs} array (integer)}
\end{arglist}

Returns \refconst{PMIX_SUCCESS} or a negative value corresponding to a PMIx error constant.

%%%%
\descr

Request that the host resource manager print the provided message and abort the provided array of \refarg{procs}.
A Unix or POSIX environment should handle the provided status as a return error code from the main program that launched the application.
A \code{NULL} for the \refarg{procs} array indicates that all processes in the caller's namespace are to be aborted, including itself.
Passing a \code{NULL} \refarg{msg} parameter is allowed.

\adviceuserstart
The response to this request is somewhat dependent on the specific \acl{RM} and its configuration (e.g., some resource managers will not abort the application if the provided status is zero unless specifically configured to do so, and some cannot abort subsets of processes in an application), and thus lies outside the control of PMIx itself.
However, the PMIx client library shall inform the \ac{RM} of the request that the specified \refarg{procs} be aborted, regardless of the value of the provided status.

Note that race conditions caused by multiple processes calling \refapi{PMIx_Abort} are left to the server implementation to resolve with regard to which status is returned and what messages (if any) are printed.
\adviceuserend


%%%%%%%%%%%%%%%%%%%%%%%%%%%%%%%%%%%%%%%%%%%%%%
%%%%%%%%%%%%%%%%%%%%%%%%%%%%%%%%%%%%%%%%%%%%%%
\section{Process Creation}
\label{chap:api_proc_mgmt:spawn}

\ldots

%%%%%%%%%%%
\subsection{\code{PMIx_Spawn}}
\declareapi{PMIx_Spawn}

%%%%
\summary

Spawn a new job.

%%%%
\format

\cspecificstart
\begin{codepar}
pmix_status_t
PMIx_Spawn(const pmix_info_t job_info[], size_t ninfo,
           const pmix_app_t apps[], size_t napps,
           char nspace[])
\end{codepar}
\cspecificend

\begin{arglist}
\argin{job_info}{Array of info structures (array of handles)}
\argin{ninfo}{Number of elements in the \refarg{job_info} array (integer)}
\argin{apps}{Array of \refstruct{pmix_app_t} structures (array of handles)}
\argin{napps}{Number of elements in the \refarg{apps} array (integer)}
\argout{nspace}{Namespace of the new job (string)}
\end{arglist}

Returns \refconst{PMIX_SUCCESS} or a negative value corresponding to a PMIx error constant.

%%%%
\descr

Spawn a new job.
The assigned namespace of the spawned applications is returned in the \refarg{nspace} parameter.
A \code{NULL} value in that location indicates that the caller doesn't wish to have the namespace returned.
The \refarg{nspace} array must be at least of size one more than \refconst{PMIX_MAX_NSLEN}.
Behavior of individual resource managers may differ, but it is expected that failure of any application process to start will result in termination/cleanup of \emph{all} processes in the newly spawned job and return of an error code to the caller.

By default, the spawned processes will be PMIx ``connected'' to the parent process upon successful launch (see \refapi{PMIx_Connect} description for details).
Note that this only means that the parent process (a) will be given a copy of the new job's
information so it can query job-level info without incurring any communication penalties, and (b) will receive notification of errors from process in the child job.

Job-level directives can be specified in the \refarg{job_info} array.
This can include:
\begin{attributedesc}
%
\declareattritem{PMIX_NON_PMI} (string)
Processes in the spawned job will not be calling \refapi{PMIx_Init}.
%
\declareattritem{PMIX_TIMEOUT} (string)
Declare the spawn as having failed if the launched processes do not call \refapi{PMIx_Init} within the specified time.
%
\declareattritem{PMIX_NOTIFY_COMPLETION} (string)
Notify the parent process when the child job terminates, either normally or with error.
%
\end{attributedesc}


%%%%%%%%%%%
\subsection{\code{PMIx_Spawn_nb}}
\declareapi{PMIx_Spawn_nb}

%%%%
\summary

Nonblocking version of the \refapi{PMIx_Spawn} routine.

%%%%
\format

\cspecificstart
\begin{codepar}
pmix_status_t
PMIx_Spawn_nb(const pmix_info_t job_info[], size_t ninfo,
              const pmix_app_t apps[], size_t napps,
              pmix_spawn_cbfunc_t cbfunc, void *cbdata)
\end{codepar}
\cspecificend

\begin{arglist}
\argin{job_info}{Array of info structures (array of handles)}
\argin{ninfo}{Number of elements in the \refarg{job_info} array (integer)}
\argin{apps}{Array of \refstruct{pmix_app_t} structures (array of handles)}
\argin{cbfunc}{Callback function \refapi{pmix_spawn_cbfunc_t} (function reference)}
\argin{cbdata}{Data to be passed to the callback function (memory reference)}
\end{arglist}

Returns \refconst{PMIX_SUCCESS} or a negative value corresponding to a PMIx error constant.

%%%%
\descr

Nonblocking version of the \refapi{PMIx_Spawn} routine.


%%%%%%%%%%%%%%%%%%%%%%%%%%%%%%%%%%%%%%%%%%%%%%
%%%%%%%%%%%%%%%%%%%%%%%%%%%%%%%%%%%%%%%%%%%%%%
\section{Connecting and Disconnecting Processes}
\label{chap:api_proc_mgmt:connect}

\ldots

%%%%%%%%%%%
\subsection{\code{PMIx_Connect}}
\declareapi{PMIx_Connect}

%%%%
\summary

Connect namespaces.

%%%%
\format

\cspecificstart
\begin{codepar}
pmix_status_t
PMIx_Connect(const pmix_proc_t procs[], size_t nprocs,
             const pmix_info_t info[], size_t ninfo)
\end{codepar}
\cspecificend

\begin{arglist}
\argin{procs}{Array of proc structures (array of handles)}
\argin{nprocs}{Number of elements in the \refarg{procs} array (integer)}
\argin{info}{Array of info structures (array of handles)}
\argin{ninfo}{Number of elements in the \refarg{info} array (integer)}
\end{arglist}

Returns \refconst{PMIX_SUCCESS} or a negative value corresponding to a PMIx error constant.

%%%%
\descr

Record the specified processes as ``connected''.
This means that the resource manager should treat the failure of any process in the specified group as a reportable event, and take appropriate action.
Note that different resource managers may respond to failures in different manners.

The callback function is to be called once all participating processes have called connect.
The server is required to return any job-level info for the connecting processes that might not already have (i.e., if the connect request involves \refarg{procs} from different namespaces, then each \refarg{proc} shall receive the job-level info from those namespaces other than their own.

A process can only engage in \emph{one} connect operation involving the identical set of processes at a time.
However, a process \emph{can} be simultaneously engaged in multiple connect operations, each involving a different set of processes.

As in the case of the fence operation, the info array can be used to pass user-level directives regarding the algorithm to be used for the collective operation involved in the ``connect'', timeout constraints, and other options available from the host RM.


%%%%%%%%%%%
\subsection{\code{PMIx_Connect_nb}}
\declareapi{PMIx_Connect_nb}

%%%%
\summary

Nonblocking \refapi{PMIx_Connect_nb} routine.

%%%%
\format

\cspecificstart
\begin{codepar}
pmix_status_t
PMIx_Connect_nb(const pmix_proc_t procs[], size_t nprocs,
                const pmix_info_t info[], size_t ninfo,
                pmix_op_cbfunc_t cbfunc, void *cbdata)
\end{codepar}
\cspecificend

\begin{arglist}
\argin{procs}{Array of proc structures (array of handles)}
\argin{nprocs}{Number of elements in the \refarg{procs} array (integer)}
\argin{info}{Array of info structures (array of handles)}
\argin{ninfo}{Number of element in the \refarg{info} array (integer)}
\argin{cbfunc}{Callback function \refapi{pmix_op_cbfunc_t} (function reference)}
\argin{cbdata}{Data to be passed to the callback function (memory reference)}
\end{arglist}

Returns \refconst{PMIX_SUCCESS} or a negative value corresponding to a PMIx error constant.

%%%%
\descr

Nonblocking \refapi{PMIx_Connect_nb} routine.


%%%%%%%%%%%
\subsection{\code{PMIx_Disconnect}}
\declareapi{PMIx_Disconnect}

%%%%
\summary

Disconnect a previously connected set of processes.

%%%%
\format

\cspecificstart
\begin{codepar}
pmix_status_t
PMIx_Disconnect(const pmix_proc_t procs[], size_t nprocs,
                const pmix_info_t info[], size_t ninfo);
\end{codepar}
\cspecificend

\begin{arglist}
\argin{procs}{Array of proc structures (array of handles)}
\argin{nprocs}{Number of elements in the \refarg{procs} array (integer)}
\argin{info}{Array of info structures (array of handles)}
\argin{ninfo}{Number of element in the \refarg{info} array (integer)}
\end{arglist}

Returns \refconst{PMIX_SUCCESS} or a negative value corresponding to a PMIx error constant.

%%%%
\descr

Disconnect a previously connected set of processes.
An error will be returned if the specified set of \refarg{procs} was not previously ``connected''.
As with \refapi{PMIx_Connect}, a process may be involved in multiple simultaneous disconnect operations.
However, a process is not allowed to reconnect to a set of \refarg{procs} that has not fully completed disconnect (i.e., you have to fully disconnect before you can reconnect to the \emph{same} group of processes.
The \refarg{info} array is used as in \refapi{PMIx_Connect}.


%%%%%%%%%%%
\subsection{\code{PMIx_Disconnect_nb}}
\declareapi{PMIx_Disconnect_nb}

%%%%
\summary

Nonblocking \refapi{PMIx_Disconnect} routine.

%%%%
\format

\cspecificstart
\begin{codepar}
pmix_status_t
PMIx_Disconnect_nb(const pmix_proc_t ranges[], size_t nprocs,
                   const pmix_info_t info[], size_t ninfo,
                   pmix_op_cbfunc_t cbfunc, void *cbdata);
\end{codepar}
\cspecificend

\begin{arglist}
\argin{procs}{Array of proc structures (array of handles)}
\argin{nprocs}{Number of elements in the \refarg{procs} array (integer)}
\argin{info}{Array of info structures (array of handles)}
\argin{ninfo}{Number of element in the \refarg{info} array (integer)}
\argin{cbfunc}{Callback function \refapi{pmix_op_cbfunc_t} (function reference)}
\argin{cbdata}{Data to be passed to the callback function (memory reference)}
\end{arglist}

Returns \refconst{PMIX_SUCCESS} or a negative value corresponding to a PMIx error constant.

%%%%
\descr

Nonblocking \refapi{PMIx_Disconnect} routine.


%%%%%%%%%%%%%%%%%%%%%%%%%%%%%%%%%%%%%%%%%%%%%%
%%%%%%%%%%%%%%%%%%%%%%%%%%%%%%%%%%%%%%%%%%%%%%
\section{Query}
\label{chap:api_proc_mgmt:query}

\ldots

%%%%%%%%%%%
\subsection{\code{PMIx_Resolve_peers}}
\declareapi{PMIx_Resolve_peers}

%%%%
\summary

Access an array of processes within the specified namespace on a node.

%%%%
\format

\cspecificstart
\begin{codepar}
pmix_status_t
PMIx_Resolve_peers(const char *nodename, const char *nspace,
                   pmix_proc_t **procs, size_t *nprocs)
\end{codepar}
\cspecificend

\begin{arglist}
\argin{nodename}{Name of the node to query (string)}
\argin{nspace}{namespace (string)}
\argout{procs}{Array of process structures (array of handles)}
\argout{nprocs}{Number of elements in the \refarg{procs} array (integer)}
\end{arglist}

Returns \refconst{PMIX_SUCCESS} or a negative value corresponding to a PMIx error constant.

%%%%
\descr

Given a \refarg{nodename}, return an array of processes within the specified \refarg{nspace}
on that node.
If the \refarg{nspace} is \code{NULL}, then all processes on the node will be returned.
If the specified node does not currently host any processes, then the returned array will be \code{NULL}, and \refarg{nprocs} will be \code{0}.
The caller is responsible for releasing the \refarg{procs} array when done with it.
The \refapi{PMIX_PROC_FREE} macro is provided for this purpose.



%%%%%%%%%%%
\subsection{\code{PMIx_Resolve_nodes}}
\declareapi{PMIx_Resolve_nodes}

%%%%
\summary

Return a list of nodes hosting processes.

%%%%
\format

\cspecificstart
\begin{codepar}
pmix_status_t
PMIx_Resolve_nodes(const char *nspace, char **nodelist)
\end{codepar}
\cspecificend

\begin{arglist}
\argin{nspace}{Namespace (string)}
\argout{nodelist}{Comma-delimited list of nodenames (string)}
\end{arglist}

Returns \refconst{PMIX_SUCCESS} or a negative value corresponding to a PMIx error constant.

%%%%
\descr

Given a \refarg{nspace}, return the list of nodes hosting processes within that namespace.
The returned string will contain a comma-delimited list of nodenames.
The caller is responsible for releasing the string when done with it.


%%%%%%%%%%%
\subsection{\code{PMIx_Query_info_nb}}
\declareapi{PMIx_Query_info_nb}
\declareapi{pmix_info_cbfunc_t}

%%%%
\summary

Query information about the system in general.

%%%%
\format

\cspecificstart
\begin{codepar}
typedef void (*pmix_info_cbfunc_t)(pmix_status_t status,
                                   pmix_info_t *info, size_t ninfo,
                                   void *cbdata,
                                   pmix_release_cbfunc_t release_fn,
                                   void *release_cbdata);

pmix_status_t
PMIx_Query_info_nb(pmix_query_t queries[], size_t nqueries,
                   pmix_info_cbfunc_t cbfunc, void *cbdata)
\end{codepar}
\cspecificend

\begin{arglist}
\argin{queries}{Array of query structures (array of handles)}
\argin{nqueries}{Number of elements in the \refarg{queries} array (integer)}
\argin{cbfunc}{Callback function \refapi{pmix_info_cbfunc_t} (function reference)}
\argin{cbdata}{Data to be passed to the callback function (memory reference)}
\end{arglist}

\begin{constantdesc}
\item \refconst{PMIX_SUCCESS} All data has been returned
\item \refconst{PMIX_ERR_NOT_FOUND} None of the requested data was available
\item \refconst{PMIX_ERR_PARTIAL_SUCCESS} Some of the data has been returned
\item \refconst{PMIX_ERR_NOT_SUPPORTED} The host \ac{RM} does not support this function
\end{constantdesc}

%%%%
\descr

Query information about the system in general.
This can include a list of active namespaces, network topology, etc.
Also can be used to query node-specific info such as the list of peers executing on a given node.
We assume that the host \ac{RM} will exercise appropriate access control on the information.

NOTE: There is no blocking form of this API as the structures passed to query info differ from those for receiving the results.

The \refarg{status} argument to the callback function indicates if requested data was found or not.
An array of \refstruct{pmix_info_t} will contain the key/value pairs.

%%%%%%%%%%%%%%%%%%%%%%%%%%%%%%%%%%%%%%%%%%%%%%%%%


    % Job Allocation Management
    %  - Allocation request, process monitoring
    %%%%%%%%%%%%%%%%%%%%%%%%%%%%%%%%%%%%%%%%%%%%%%%%%
% Chapter: Job Allocation Management
%%%%%%%%%%%%%%%%%%%%%%%%%%%%%%%%%%%%%%%%%%%%%%%%%
\chapter{Job Management and Reporting}
\label{chap:api_job_mgmt}

The job management \acp{API} provide an application with the ability to orchestrate its operation in partnership with the \ac{SMS}.
Members of this category include the \refapi{PMIx_Allocation_request}, \refapi{PMIx_Job_control}, and \refapi{PMIx_Process_monitor} \acp{API}.

%%%%%%%%%%%%%%%%%%%%%%%%%%%%%%%%%%%%%%%%%%%%%%%%%
%%%%%%%%%%%%%%%%%%%%%%%%%%%%%%%%%%%%%%%%%%%%%%%%%
\section{Allocation Requests}
\label{chap:api_job_mgmt:alloc}

This section defines functionality to request new allocations from the \ac{RM}, and request modifications to existing allocations.
These are primarily used in the following scenarios:
\begin{itemize}
\item \textit{Evolving} applications that dynamically request and return resources as they execute.
\item \textit{Malleable} environments where the scheduler redirects resources away from executing applications for higher priority jobs or load balancing.
\item \textit{Resilient} applications that need to request replacement resources in the face of failures.
\item \textit{Rigid} jobs where the user has requested a static allocation of resources for a fixed period of time, but realizes that they underestimated their required time while executing.
\end{itemize}
\ac{PMIx} attempts to address this range of use-cases with a flexible \ac{API}.

%%%%%%%%%%%%%%%%%%%%%%%%%%%%%%%%%%%%%%%%%%%%%%%%%
\subsection{\code{PMIx_Allocation_request}}
\declareapi{PMIx_Allocation_request}

%%%%
\summary

Request an allocation operation from the host resource manager.

%%%%
\format

\copySignature{PMIx_Allocation_request}{3.0}{
pmix_status_t \\
PMIx_Allocation_request(pmix_alloc_directive_t directive, \\
\hspace*{24\sigspace}pmix_info_t info[], size_t ninfo, \\
\hspace*{24\sigspace}pmix_info_t *results[], size_t *nresults);
}

\begin{arglist}
\argin{directive}{Allocation directive (\refstruct{pmix_alloc_directive_t})}
\argin{info}{Array of \refstruct{pmix_info_t} structures (array of handles)}
\argin{ninfo}{Number of elements in the \refarg{info} array (integer)}
\arginout{results}{Address where a pointer to an array of \refstruct{pmix_info_t} containing the results of the request can be returned (memory reference)}
\arginout{nresults}{Address where the number of elements in \refarg{results} can be returned (handle)}
\end{arglist}

Returns one of the following:

\begin{itemize}
    \item \refconst{PMIX_SUCCESS}, indicating that the request was processed and returned \textit{success}
    \item a PMIx error constant indicating either an error in the input or that the request was refused
\end{itemize}

\reqattrstart
\ac{PMIx} libraries are not required to directly support any attributes for this function. However, any provided attributes must be passed to the host \ac{SMS} daemon for processing, and the \ac{PMIx} library is \textit{required} to add the \refAttributeItem{PMIX_USERID} and the \refAttributeItem{PMIX_GRPID} attributes of the client process making the request.

Host environments that implement support for this operation are required to support the following attributes:

\pasteAttributeItem{PMIX_ALLOC_REQ_ID}
\pasteAttributeItem{PMIX_ALLOC_NUM_NODES}
\pasteAttributeItem{PMIX_ALLOC_NUM_CPUS}
\pasteAttributeItem{PMIX_ALLOC_TIME}

\reqattrend

\optattrstart
The following attributes are optional for host environments that support this operation:

\pasteAttributeItem{PMIX_ALLOC_NODE_LIST}
\pasteAttributeItem{PMIX_ALLOC_NUM_CPU_LIST}
\pasteAttributeItem{PMIX_ALLOC_CPU_LIST}
\pasteAttributeItem{PMIX_ALLOC_MEM_SIZE}
\pasteAttributeItem{PMIX_ALLOC_FABRIC}
\pasteAttributeItem{PMIX_ALLOC_FABRIC_ID}
\pasteAttributeItem{PMIX_ALLOC_BANDWIDTH}
\pasteAttributeItem{PMIX_ALLOC_FABRIC_QOS}
\pasteAttributeItem{PMIX_ALLOC_FABRIC_TYPE}
\pasteAttributeItem{PMIX_ALLOC_FABRIC_PLANE}
\pasteAttributeItem{PMIX_ALLOC_FABRIC_ENDPTS}
\pasteAttributeItem{PMIX_ALLOC_FABRIC_ENDPTS_NODE}
\pasteAttributeItem{PMIX_ALLOC_FABRIC_SEC_KEY}

\optattrend

%%%%
\descr

Request an allocation operation from the host resource manager.
Several broad categories are envisioned, including the ability to:

\begin{compactitem}
%
\item Request allocation of additional resources, including memory, bandwidth, and compute.
This should be accomplished in a non-blocking manner so that the application can continue to progress while waiting for resources to become available.
Note that the new allocation will be disjoint from (i.e., not affiliated with) the allocation of the requestor - thus the termination of one allocation will not impact the other.
%
\item Extend the reservation on currently allocated resources, subject to scheduling availability and priorities.
This includes extending the time limit on current resources, and/or requesting additional resources be allocated to the requesting job.
Any additional allocated resources will be considered as part of the current allocation, and thus will be released at the same time.
%
\item Return no-longer-required resources to the scheduler.
This includes the ``loan'' of resources back to the scheduler with a promise to return them upon subsequent request.
\end{compactitem}

If successful, the returned results for a request for additional resources must include the host resource manager's identifier (\refattr{PMIX_ALLOC_ID}) that the requester can use to specify the resources in, for example, a call to \refapi{PMIx_Spawn}.

%%%%%%%%%%%%%%%%%%%%%%%%%%%%%%%%%%%%%%%%%%%%%%%%%
\subsection{\code{PMIx_Allocation_request_nb}}
\declareapi{PMIx_Allocation_request_nb}

%%%%
\summary

Request an allocation operation from the host resource manager.

%%%%
\format

\copySignature{PMIx_Allocation_request_nb}{2.0}{
pmix_status_t \\
PMIx_Allocation_request_nb(pmix_alloc_directive_t directive, \\
\hspace*{27\sigspace}pmix_info_t info[], size_t ninfo, \\
\hspace*{27\sigspace}pmix_info_cbfunc_t cbfunc, void *cbdata);
}

\begin{arglist}
\argin{directive}{Allocation directive (\refstruct{pmix_alloc_directive_t})}
\argin{info}{Array of \refstruct{pmix_info_t} structures (array of handles)}
\argin{ninfo}{Number of elements in the \refarg{info} array (integer)}
\argin{cbfunc}{Callback function \refapi{pmix_info_cbfunc_t} (function reference)}
\argin{cbdata}{Data to be passed to the callback function (memory reference)}
\end{arglist}

Returns one of the following:

\begin{itemize}
    \item \refconst{PMIX_SUCCESS}, indicating that the request is being processed by the host environment - result will be returned in the provided \refarg{cbfunc}. Note that the library must not invoke the callback function prior to returning from the \ac{API}.
    \item \refconst{PMIX_OPERATION_SUCCEEDED}, indicating that the request was immediately processed and returned \textit{success} - the \refarg{cbfunc} will \textit{not} be called
    \item a PMIx error constant indicating either an error in the input or that the request was immediately processed and failed - the \refarg{cbfunc} will \textit{not} be called
\end{itemize}

\reqattrstart
\ac{PMIx} libraries are not required to directly support any attributes for this function. However, any provided attributes must be passed to the host \ac{SMS} daemon for processing, and the \ac{PMIx} library is \textit{required} to add the \refAttributeItem{PMIX_USERID} and the \refAttributeItem{PMIX_GRPID} attributes of the client process making the request.

Host environments that implement support for this operation are required to support the following attributes:

\pasteAttributeItem{PMIX_ALLOC_REQ_ID}
\pasteAttributeItem{PMIX_ALLOC_NUM_NODES}
\pasteAttributeItem{PMIX_ALLOC_NUM_CPUS}
\pasteAttributeItem{PMIX_ALLOC_TIME}

\reqattrend

\optattrstart
The following attributes are optional for host environments that support this operation:

\pasteAttributeItem{PMIX_ALLOC_NODE_LIST}
\pasteAttributeItem{PMIX_ALLOC_NUM_CPU_LIST}
\pasteAttributeItem{PMIX_ALLOC_CPU_LIST}
\pasteAttributeItem{PMIX_ALLOC_MEM_SIZE}
\pasteAttributeItem{PMIX_ALLOC_FABRIC}
\pasteAttributeItem{PMIX_ALLOC_FABRIC_ID}
\pasteAttributeItem{PMIX_ALLOC_BANDWIDTH}
\pasteAttributeItem{PMIX_ALLOC_FABRIC_QOS}
\pasteAttributeItem{PMIX_ALLOC_FABRIC_TYPE}
\pasteAttributeItem{PMIX_ALLOC_FABRIC_PLANE}
\pasteAttributeItem{PMIX_ALLOC_FABRIC_ENDPTS}
\pasteAttributeItem{PMIX_ALLOC_FABRIC_ENDPTS_NODE}
\pasteAttributeItem{PMIX_ALLOC_FABRIC_SEC_KEY}

\optattrend

%%%%
\descr

Non-blocking form of the \refapi{PMIx_Allocation_request} \ac{API}.


%%%%%%%%%%%%%%%%%%%%%%%%%%%%%%%%%%%%%%%%%%%%%%%%%
\subsection{Job Allocation attributes}
\label{api:struct:attributes:joballoc}

Attributes used to describe the job allocation - these are values passed to and/or returned by the \refapi{PMIx_Allocation_request_nb} and \refapi{PMIx_Allocation_request} \acp{API} and are not accessed using the \refapi{PMIx_Get} \ac{API}.

%
\declareAttribute{PMIX_ALLOC_REQ_ID}{"pmix.alloc.reqid"}{char*}{
User-provided string identifier for this allocation request which can later be used to query status of the request.
}
%
\declareAttributeNEW{PMIX_ALLOC_ID}{"pmix.alloc.id"}{char*}{
A string identifier (provided by the host environment) for the resulting allocation which can later be used to reference the allocated resources in, for example, a call to \refapi{PMIx_Spawn}.
}
%
\declareAttributeNEW{PMIX_ALLOC_QUEUE}{"pmix.alloc.queue"}{char*}{
Name of the \ac{WLM} queue to which the allocation request is to be directed, or the queue being referenced in a query.
}
%
\declareAttribute{PMIX_ALLOC_NUM_NODES}{"pmix.alloc.nnodes"}{uint64_t}{
The number of nodes being requested in an allocation request.
}
%
\declareAttribute{PMIX_ALLOC_NODE_LIST}{"pmix.alloc.nlist"}{char*}{
Regular expression of the specific nodes being requested in an allocation request.
}
%
\declareAttribute{PMIX_ALLOC_NUM_CPUS}{"pmix.alloc.ncpus"}{uint64_t}{
Number of \acp{PU} being requested in an allocation request.
}
%
\declareAttribute{PMIX_ALLOC_NUM_CPU_LIST}{"pmix.alloc.ncpulist"}{char*}{
Regular expression of the number of \acp{PU} for each node being requested in an allocation request.
}
%
\declareAttribute{PMIX_ALLOC_CPU_LIST}{"pmix.alloc.cpulist"}{char*}{
Regular expression of the specific \acp{PU}  being requested in an allocation request.
}
%
\declareAttribute{PMIX_ALLOC_MEM_SIZE}{"pmix.alloc.msize"}{float}{
Number of Megabytes[base2] of memory (per process) being requested in an allocation request.
}
%
\declareAttribute{PMIX_ALLOC_FABRIC}{"pmix.alloc.net"}{array}{
Array of \refstruct{pmix_info_t} describing requested fabric resources. This must include at least: \refattr{PMIX_ALLOC_FABRIC_ID}, \refattr{PMIX_ALLOC_FABRIC_TYPE}, and \refattr{PMIX_ALLOC_FABRIC_ENDPTS}, plus whatever other descriptors are desired.
}
%
\declareAttribute{PMIX_ALLOC_FABRIC_ID}{"pmix.alloc.netid"}{char*}{
The key to be used when accessing this requested fabric allocation. The fabric allocation will be returned/stored as a \refstruct{pmix_data_array_t} of \refstruct{pmix_info_t} whose first element is composed of this key and the allocated resource description.
The type of the included value depends upon the fabric support. For example, a \ac{TCP} allocation might consist of a comma-delimited string of socket ranges such as \code{"32000-32100,\allowbreak 33005,38123-38146"}. Additional array entries will consist of any provided resource request directives, along with their assigned values. Examples include: \refattr{PMIX_ALLOC_FABRIC_TYPE} - the type of resources provided; \refattr{PMIX_ALLOC_FABRIC_PLANE} - if applicable, what plane the resources were assigned from; \refattr{PMIX_ALLOC_FABRIC_QOS} - the assigned QoS; \refattr{PMIX_ALLOC_BANDWIDTH} - the allocated bandwidth; \refattr{PMIX_ALLOC_FABRIC_SEC_KEY} - a security key for the requested fabric allocation. NOTE: the array contents may differ from those requested, especially if \refconst{PMIX_INFO_REQD} was not set in the request.
}
%
\declareAttribute{PMIX_ALLOC_BANDWIDTH}{"pmix.alloc.bw"}{float}{
Fabric bandwidth (in Megabits[base2]/sec) for the job being requested in an allocation request.
}
%
\declareAttribute{PMIX_ALLOC_FABRIC_QOS}{"pmix.alloc.netqos"}{char*}{
Fabric quality of service level for the job being requested in an allocation request.
}
%
\declareAttribute{PMIX_ALLOC_TIME}{"pmix.alloc.time"}{uint32_t}{
Total session time (in seconds) being requested in an allocation request.
}
%
\declareAttribute{PMIX_ALLOC_FABRIC_TYPE}{"pmix.alloc.nettype"}{char*}{
Type of desired transport (e.g., \var{``tcp''}, \var{``udp''}) being requested in an allocation request.
}
%
\declareAttribute{PMIX_ALLOC_FABRIC_PLANE}{"pmix.alloc.netplane"}{char*}{
ID string for the \refterm{fabric plane} to be used for the requested allocation.
}
%
\declareAttribute{PMIX_ALLOC_FABRIC_ENDPTS}{"pmix.alloc.endpts"}{size_t}{
Number of endpoints to allocate per \refterm{process} in the job.
}
%
\declareAttribute{PMIX_ALLOC_FABRIC_ENDPTS_NODE}{"pmix.alloc.endpts.nd"}{size_t}{
Number of endpoints to allocate per \refterm{node} for the job.
}
%
\declareAttribute{PMIX_ALLOC_FABRIC_SEC_KEY}{"pmix.alloc.nsec"}{pmix_byte_object_t}{
Request that the allocation include a fabric security key for the spawned job.
}


%%%%%%%%%%%%%%%%%%%%%%%%%%%%%%%%%%%%%%%%%%%%%%%%%
\subsection{Job Allocation Directives}
\declarestruct{pmix_alloc_directive_t}

\versionMarker{2.0}
The \refstruct{pmix_alloc_directive_t} structure is a \code{uint8_t} type that defines the behavior of allocation requests.
The following constants can be used to set a variable of the type \refstruct{pmix_alloc_directive_t}. All definitions were introduced in version 2 of the standard unless otherwise marked.

\begin{constantdesc}
%
\declareconstitem{PMIX_ALLOC_NEW}
A new allocation is being requested.
The resulting allocation will be disjoint (i.e., not connected in a job sense) from the requesting allocation.
%
\declareconstitem{PMIX_ALLOC_EXTEND}
Extend the existing allocation, either in time or as additional resources.
%
\declareconstitem{PMIX_ALLOC_RELEASE}
Release part of the existing allocation.
Attributes in the accompanying \refstruct{pmix_info_t} array may be used to specify permanent release of the identified resources, or ``lending'' of those resources for some period of time.
%
\declareconstitem{PMIX_ALLOC_REAQUIRE}
Reacquire resources that were previously ``lent'' back to the scheduler.
%
\declareconstitem{PMIX_ALLOC_EXTERNAL}
A value boundary above which implementers are free to define their own directive values.
%
\end{constantdesc}



%%%%%%%%%%%%%%%%%%%%%%%%%%%%%%%%%%%%%%%%%%%%%%%%%
%%%%%%%%%%%%%%%%%%%%%%%%%%%%%%%%%%%%%%%%%%%%%%%%%
\section{Job Control}
\label{chap:api_job_mgmt:jctrl}

This section defines \acp{API} that enable the application and host environment to coordinate the response to failures and other events.
This can include requesting termination of the entire job or a subset of processes within a job, but can
also be used in combination with other \ac{PMIx} capabilities (e.g., allocation support and event notification) for more nuanced responses. For example, an application notified of an incipient over-temperature condition on a node could use the \refapi{PMIx_Allocation_request_nb} interface to request replacement nodes while simultaneously using the \refapi{PMIx_Job_control_nb} interface to direct that a checkpoint event be delivered to all processes in the application. If replacement resources are not available, the application might use the \refapi{PMIx_Job_control_nb} interface to request that the job continue at a lower power setting, perhaps sufficient to avoid the over-temperature failure.

The job control \acp{API} can also be used by an application to register itself as available for preemption when operating in an environment such as a cloud or where incentives, financial or otherwise, are provided to jobs willing to be preempted. Registration can include attributes indicating how many resources are being offered for preemption (e.g., all or only some portion), whether the application will require time to prepare for preemption, etc. Jobs that
request a warning will receive an event notifying them of an impending preemption (possibly including information as to the resources that will be taken away, how much time the application will be given prior to being preempted, whether the preemption will be a suspension or full termination, etc.) so they have an opportunity to save
their work. Once the application is ready, it calls the provided event completion callback function to indicate that
the SMS is free to suspend or terminate it, and can include directives regarding any desired restart.

%%%%%%%%%%%%%%%%%%%%%%%%%%%%%%%%%%%%%%%%%%%%%%%%%
\subsection{\code{PMIx_Job_control}}
\declareapi{PMIx_Job_control}

%%%%
\summary

Request a job control action.

%%%%
\format

\copySignature{PMIx_Job_control}{3.0}{
pmix_status_t \\
PMIx_Job_control(const pmix_proc_t targets[], size_t ntargets, \\
\hspace*{17\sigspace}const pmix_info_t directives[], size_t ndirs, \\
\hspace*{17\sigspace}pmix_info_t *results[], size_t *nresults);
}

\begin{arglist}
\argin{targets}{Array of proc structures (array of handles)}
\argin{ntargets}{Number of elements in the \refarg{targets} array (integer)}
\argin{directives}{Array of info structures (array of handles)}
\argin{ndirs}{Number of elements in the \refarg{directives} array (integer)}
\arginout{results}{Address where a pointer to an array of \refstruct{pmix_info_t} containing the results of the request can be returned (memory reference)}
\arginout{nresults}{Address where the number of elements in \refarg{results} can be returned (handle)}
\end{arglist}

Returns one of the following:

\begin{itemize}
    \item \refconst{PMIX_SUCCESS}, indicating that the request was processed by the host environment and returned \textit{success}. Details of the result will be returned in the \refarg{results} array
    \item a \ac{PMIx} error constant indicating either an error in the input or that the request was refused
\end{itemize}

\reqattrstart
\ac{PMIx} libraries are not required to directly support any attributes for this function. However, any provided attributes must be passed to the host \ac{SMS} daemon for processing, and the \ac{PMIx} library is \textit{required} to add the \refAttributeItem{PMIX_USERID} and the \refAttributeItem{PMIX_GRPID} attributes of the client process making the request.

Host environments that implement support for this operation are required to support the following attributes:

\pasteAttributeItem{PMIX_JOB_CTRL_ID}
\pasteAttributeItem{PMIX_JOB_CTRL_PAUSE}
\pasteAttributeItem{PMIX_JOB_CTRL_RESUME}
\pasteAttributeItem{PMIX_JOB_CTRL_KILL}
\pasteAttributeItem{PMIX_JOB_CTRL_SIGNAL}
\pasteAttributeItem{PMIX_JOB_CTRL_TERMINATE}
\pasteAttributeItem{PMIX_REGISTER_CLEANUP}
\pasteAttributeItem{PMIX_REGISTER_CLEANUP_DIR}
\pasteAttributeItem{PMIX_CLEANUP_RECURSIVE}
\pasteAttributeItem{PMIX_CLEANUP_EMPTY}
\pasteAttributeItem{PMIX_CLEANUP_IGNORE}
\pasteAttributeItem{PMIX_CLEANUP_LEAVE_TOPDIR}

\reqattrend

\optattrstart
The following attributes are optional for host environments that support this operation:

\pasteAttributeItem{PMIX_JOB_CTRL_CANCEL}
\pasteAttributeItem{PMIX_JOB_CTRL_RESTART}
\pasteAttributeItem{PMIX_JOB_CTRL_CHECKPOINT}
\pasteAttributeItem{PMIX_JOB_CTRL_CHECKPOINT_EVENT}
\pasteAttributeItem{PMIX_JOB_CTRL_CHECKPOINT_SIGNAL}
\pasteAttributeItem{PMIX_JOB_CTRL_CHECKPOINT_TIMEOUT}
\pasteAttributeItem{PMIX_JOB_CTRL_CHECKPOINT_METHOD}
\pasteAttributeItem{PMIX_JOB_CTRL_PROVISION}
\pasteAttributeItem{PMIX_JOB_CTRL_PROVISION_IMAGE}
\pasteAttributeItem{PMIX_JOB_CTRL_PREEMPTIBLE}

\optattrend

%%%%
\descr

Request a job control action.
The \refarg{targets} array identifies the processes to which the requested job control action is to be applied. All \refterm{clones} of an identified process are to have the requested action applied to them.
A \code{NULL} value can be used to indicate all processes in the caller's namespace.
The use of \refconst{PMIX_RANK_WILDCARD} can also be used to indicate that all processes in the given namespace are to be included.

The directives are provided as \refstruct{pmix_info_t} structures in the \refarg{directives} array.
The returned \refarg{status} indicates whether or not the request was granted, and information as to the reason for any denial of the request shall be returned in the \refarg{results} array.

%%%%%%%%%%%%%%%%%%%%%%%%%%%%%%%%%%%%%%%%%%%%%%%%%
\subsection{\code{PMIx_Job_control_nb}}
\declareapi{PMIx_Job_control_nb}

%%%%
\summary

Request a job control action.

%%%%
\format

\copySignature{PMIx_Job_control_nb}{2.0}{
pmix_status_t \\
PMIx_Job_control_nb(const pmix_proc_t targets[], size_t ntargets, \\
\hspace*{20\sigspace}const pmix_info_t directives[], size_t ndirs, \\
\hspace*{20\sigspace}pmix_info_cbfunc_t cbfunc, void *cbdata);
}

\begin{arglist}
\argin{targets}{Array of proc structures (array of handles)}
\argin{ntargets}{Number of elements in the \refarg{targets} array (integer)}
\argin{directives}{Array of info structures (array of handles)}
\argin{ndirs}{Number of elements in the \refarg{directives} array (integer)}
\argin{cbfunc}{Callback function \refapi{pmix_info_cbfunc_t} (function reference)}
\argin{cbdata}{Data to be passed to the callback function (memory reference)}
\end{arglist}

Returns one of the following:

\begin{itemize}
    \item \refconst{PMIX_SUCCESS}, indicating that the request is being processed by the host environment - result will be returned in the provided \refarg{cbfunc}. Note that the library must not invoke the callback function prior to returning from the \ac{API}.
    \item \refconst{PMIX_OPERATION_SUCCEEDED}, indicating that the request was immediately processed and returned \textit{success} - the \refarg{cbfunc} will \textit{not} be called
    \item a PMIx error constant indicating either an error in the input or that the request was immediately processed and failed - the \refarg{cbfunc} will \textit{not} be called
\end{itemize}

\reqattrstart
\ac{PMIx} libraries are not required to directly support any attributes for this function. However, any provided attributes must be passed to the host \ac{SMS} daemon for processing, and the \ac{PMIx} library is \textit{required} to add the \refAttributeItem{PMIX_USERID} and the \refAttributeItem{PMIX_GRPID} attributes of the client process making the request.

Host environments that implement support for this operation are required to support the following attributes:

\pasteAttributeItem{PMIX_JOB_CTRL_ID}
\pasteAttributeItem{PMIX_JOB_CTRL_PAUSE}
\pasteAttributeItem{PMIX_JOB_CTRL_RESUME}
\pasteAttributeItem{PMIX_JOB_CTRL_KILL}
\pasteAttributeItem{PMIX_JOB_CTRL_SIGNAL}
\pasteAttributeItem{PMIX_JOB_CTRL_TERMINATE}
\pasteAttributeItem{PMIX_REGISTER_CLEANUP}
\pasteAttributeItem{PMIX_REGISTER_CLEANUP_DIR}
\pasteAttributeItem{PMIX_CLEANUP_RECURSIVE}
\pasteAttributeItem{PMIX_CLEANUP_EMPTY}
\pasteAttributeItem{PMIX_CLEANUP_IGNORE}
\pasteAttributeItem{PMIX_CLEANUP_LEAVE_TOPDIR}

\reqattrend

\optattrstart
The following attributes are optional for host environments that support this operation:

\pasteAttributeItem{PMIX_JOB_CTRL_CANCEL}
\pasteAttributeItem{PMIX_JOB_CTRL_RESTART}
\pasteAttributeItem{PMIX_JOB_CTRL_CHECKPOINT}
\pasteAttributeItem{PMIX_JOB_CTRL_CHECKPOINT_EVENT}
\pasteAttributeItem{PMIX_JOB_CTRL_CHECKPOINT_SIGNAL}
\pasteAttributeItem{PMIX_JOB_CTRL_CHECKPOINT_TIMEOUT}
\pasteAttributeItem{PMIX_JOB_CTRL_CHECKPOINT_METHOD}
\pasteAttributeItem{PMIX_JOB_CTRL_PROVISION}
\pasteAttributeItem{PMIX_JOB_CTRL_PROVISION_IMAGE}
\pasteAttributeItem{PMIX_JOB_CTRL_PREEMPTIBLE}

\optattrend

%%%%
\descr

Non-blocking form of the \refapi{PMIx_Job_control} \ac{API}.
The \refarg{targets} array identifies the processes to which the requested job control action is to be applied. All \refterm{clones} of an identified process are to have the requested action applied to them.
A \code{NULL} value can be used to indicate all processes in the caller's namespace.
The use of \refconst{PMIX_RANK_WILDCARD} can also be used to indicate that all processes in the given namespace are to be included.

The directives are provided as \refstruct{pmix_info_t} structures in the \refarg{directives} array.
The callback function provides a \refarg{status} to indicate whether or not the request was granted, and to provide some information as to the reason for any denial in the \refapi{pmix_info_cbfunc_t} array of \refstruct{pmix_info_t} structures.

%%%%%%%%%%%%%%%%%%%%%%%%%%%%%%%%%%%%%%%%%%%%%%%%%
\subsection{Job control constants}
\label{api:struct:constants:jobcontrol}

The following constants are specifically defined for return by the job control \acp{API}:

\begin{constantdesc}

%
\declareconstitemNEW{PMIX_ERR_CONFLICTING_CLEANUP_DIRECTIVES}
Conflicting directives given for job/process cleanup.

\end{constantdesc}

%%%%%%%%%%%%%%%%%%%%%%%%%%%%%%%%%%%%%%%%%%%%%%%%%
\subsection{Job control events}
\label{api:struct:events:jobcontrol}

The following job control events may be available for registration, depending upon implementation and host environment support:

\begin{constantdesc}
%
\declareconstitem{PMIX_JCTRL_CHECKPOINT}
Monitored by \ac{PMIx} client to trigger a checkpoint operation.
%
\declareconstitem{PMIX_JCTRL_CHECKPOINT_COMPLETE}
Sent by a \ac{PMIx} client and monitored by a \ac{PMIx} server to notify that requested checkpoint operation has completed.
%
\declareconstitem{PMIX_JCTRL_PREEMPT_ALERT}
Monitored by a \ac{PMIx} client to detect that an \ac{RM} intends to preempt the job.
%
\declareconstitem{PMIX_ERR_PROC_RESTART}
Error in process restart.
%
\declareconstitem{PMIX_ERR_PROC_CHECKPOINT}
Error in process checkpoint.
%
\declareconstitem{PMIX_ERR_PROC_MIGRATE}
Error in process migration.
%
\end{constantdesc}

%%%%%%%%%%%%%%%%%%%%%%%%%%%%%%%%%%%%%%%%%%%%%%%%%
\subsection{Job control attributes}
\label{api:struct:attributes:jobcontrol}

Attributes used to request control operations on an executing application - these are values passed to the job control \acp{API} and are not accessed using the \refapi{PMIx_Get} \ac{API}.

%
\declareAttribute{PMIX_JOB_CTRL_ID}{"pmix.jctrl.id"}{char*}{
Provide a string identifier for this request. The user can provide an identifier for the requested operation, thus allowing them to later request status of the operation or to terminate it. The host, therefore, shall track it with the request for future reference.
}
%
\declareAttribute{PMIX_JOB_CTRL_PAUSE}{"pmix.jctrl.pause"}{bool}{
Pause the specified processes.
}
%
\declareAttribute{PMIX_JOB_CTRL_RESUME}{"pmix.jctrl.resume"}{bool}{
Resume (``un-pause'') the specified processes.
}
%
\declareAttribute{PMIX_JOB_CTRL_CANCEL}{"pmix.jctrl.cancel"}{char*}{
Cancel the specified request - the provided request ID must match the \refattr{PMIX_JOB_CTRL_ID} provided to a previous call to \refapi{PMIx_Job_control}. An ID of \code{NULL} implies cancel all requests from this requestor.
}
%
\declareAttribute{PMIX_JOB_CTRL_KILL}{"pmix.jctrl.kill"}{bool}{
Forcibly terminate the specified processes and cleanup.
}
%
\declareAttribute{PMIX_JOB_CTRL_RESTART}{"pmix.jctrl.restart"}{char*}{
Restart the specified processes using the given checkpoint ID.
}
%
\declareAttribute{PMIX_JOB_CTRL_CHECKPOINT}{"pmix.jctrl.ckpt"}{char*}{
Checkpoint the specified processes and assign the given ID to it.
}
%
\declareAttribute{PMIX_JOB_CTRL_CHECKPOINT_EVENT}{"pmix.jctrl.ckptev"}{bool}{
Use event notification to trigger a process checkpoint.
}
%
\declareAttribute{PMIX_JOB_CTRL_CHECKPOINT_SIGNAL}{"pmix.jctrl.ckptsig"}{int}{
Use the given signal to trigger a process checkpoint.
}
%
\declareAttribute{PMIX_JOB_CTRL_CHECKPOINT_TIMEOUT}{"pmix.jctrl.ckptsig"}{int}{
Time in seconds to wait for a checkpoint to complete.
}
%
\declareAttribute{PMIX_JOB_CTRL_CHECKPOINT_METHOD}{"pmix.jctrl.ckmethod"}{pmix_data_array_t}{
Array of \refstruct{pmix_info_t} declaring each method and value supported by this application.
}
%
\declareAttribute{PMIX_JOB_CTRL_SIGNAL}{"pmix.jctrl.sig"}{int}{
Send given signal to specified processes.
}
%
\declareAttribute{PMIX_JOB_CTRL_PROVISION}{"pmix.jctrl.pvn"}{char*}{
Regular expression identifying nodes that are to be provisioned.
}
%
\declareAttribute{PMIX_JOB_CTRL_PROVISION_IMAGE}{"pmix.jctrl.pvnimg"}{char*}{
Name of the image that is to be provisioned.
}
%
\declareAttribute{PMIX_JOB_CTRL_PREEMPTIBLE}{"pmix.jctrl.preempt"}{bool}{
Indicate that the job can be pre-empted.
}
%
\declareAttribute{PMIX_JOB_CTRL_TERMINATE}{"pmix.jctrl.term"}{bool}{
Politely terminate the specified processes.
}
%
\declareAttribute{PMIX_REGISTER_CLEANUP}{"pmix.reg.cleanup"}{char*}{
Comma-delimited list of files to be removed upon process termination.
}
%
\declareAttribute{PMIX_REGISTER_CLEANUP_DIR}{"pmix.reg.cleanupdir"}{char*}{
Comma-delimited list of directories to be removed upon process termination.
}
%
\declareAttribute{PMIX_CLEANUP_RECURSIVE}{"pmix.clnup.recurse"}{bool}{
Recursively cleanup all subdirectories under the specified one(s).
}
%
\declareAttribute{PMIX_CLEANUP_EMPTY}{"pmix.clnup.empty"}{bool}{
Only remove empty subdirectories.
}
%
\declareAttribute{PMIX_CLEANUP_IGNORE}{"pmix.clnup.ignore"}{char*}{
Comma-delimited list of filenames that are not to be removed.
}
%
\declareAttribute{PMIX_CLEANUP_LEAVE_TOPDIR}{"pmix.clnup.lvtop"}{bool}{
When recursively cleaning subdirectories, do not remove the top-level directory (the one given in the cleanup request).
}


%%%%%%%%%%%%%%%%%%%%%%%%%%%%%%%%%%%%%%%%%%%%%%%%%
%%%%%%%%%%%%%%%%%%%%%%%%%%%%%%%%%%%%%%%%%%%%%%%%%
\section{Process and Job Monitoring}
\label{chap:api_job_mgmt:monitor}

In addition to external faults, a common problem encountered in \ac{HPC} applications is a failure to make
progress due to some internal conflict in the computation. These situations can
result in a significant waste of resources as the \ac{SMS} is unaware of the problem, and thus cannot terminate the
job. Various watchdog methods have been developed for detecting this situation, including requiring a periodic ``heartbeat''
from the application and monitoring a specified file for changes in size and/or modification time.

The following \acp{API} allow applications to request monitoring, directing what is to be monitored, the frequency of the associated check, whether or not the application is to be notified (via the event notification subsystem) of stall detection, and other characteristics of the operation.

%%%%%%%%%%%%%%%%%%%%%%%%%%%%%%%%%%%%%%%%%%%%%%%%%
\subsection{\code{PMIx_Process_monitor}}
\declareapi{PMIx_Process_monitor}

%%%%
\summary

Request that application processes be monitored.

%%%%
\format

\copySignature{PMIx_Process_monitor}{3.0}{
pmix_status_t \\
PMIx_Process_monitor(const pmix_info_t *monitor, \\
\hspace*{21\sigspace}pmix_status_t error, \\
\hspace*{21\sigspace}const pmix_info_t directives[], size_t ndirs, \\
\hspace*{21\sigspace}pmix_info_t *results[], size_t *nresults);
}

\begin{arglist}
\argin{monitor}{info (handle)}
\argin{error}{status (integer)}
\argin{directives}{Array of info structures (array of handles)}
\argin{ndirs}{Number of elements in the \refarg{directives} array (integer)}
\arginout{results}{Address where a pointer to an array of \refstruct{pmix_info_t} containing the results of the request can be returned (memory reference)}
\arginout{nresults}{Address where the number of elements in \refarg{results} can be returned (handle)}
\end{arglist}

Returns one of the following:

\begin{itemize}
    \item \refconst{PMIX_SUCCESS}, indicating that the request was processed and returned \textit{success}. Details of the result will be returned in the \refarg{results} array
    \item a PMIx error constant indicating either an error in the input or that the request was refused
\end{itemize}

\optattrstart
The following attributes may be implemented by a \ac{PMIx} library or by the host environment. If supported by the \ac{PMIx} server library, then the library must not pass the supported attributes to the host environment. All attributes not directly supported by the server library must be passed to the host environment if it supports this operation, and the library is \textit{required} to add the \refAttributeItem{PMIX_USERID} and the \refAttributeItem{PMIX_GRPID} attributes of the requesting process:

\pasteAttributeItem{PMIX_MONITOR_ID}
\pasteAttributeItem{PMIX_MONITOR_CANCEL}
\pasteAttributeItem{PMIX_MONITOR_APP_CONTROL}
\pasteAttributeItem{PMIX_MONITOR_HEARTBEAT}
\pasteAttributeItem{PMIX_MONITOR_HEARTBEAT_TIME}
\pasteAttributeItem{PMIX_MONITOR_HEARTBEAT_DROPS}
\pasteAttributeItem{PMIX_MONITOR_FILE}
\pasteAttributeItem{PMIX_MONITOR_FILE_SIZE}
\pasteAttributeItem{PMIX_MONITOR_FILE_ACCESS}
\pasteAttributeItem{PMIX_MONITOR_FILE_MODIFY}
\pasteAttributeItem{PMIX_MONITOR_FILE_CHECK_TIME}
\pasteAttributeItem{PMIX_MONITOR_FILE_DROPS}
\pasteAttributeItem{PMIX_SEND_HEARTBEAT}

\optattrend

%%%%
\descr

Request that application processes be monitored via several possible methods.
For example, that the server monitor this process for periodic heartbeats as an indication that the process has not become ``wedged''.
When a monitor detects the specified alarm condition, it will generate an event notification using the provided error code and passing along any available relevant information.
It is up to the caller to register a corresponding event handler.

The \refarg{monitor} argument is an attribute indicating the type of monitor being requested.
For example, \refattr{PMIX_MONITOR_FILE} to indicate that the requestor is asking that a file be monitored.

The \refarg{error} argument is the status code to be used when generating an event notification alerting that the monitor has been triggered.
The range of the notification defaults to \refconst{PMIX_RANGE_NAMESPACE}.
This can be changed by providing a \refattr{PMIX_RANGE} directive.

The \refarg{directives} argument characterizes the monitoring request (e.g., monitor file size) and frequency of checking to be done

The returned \refarg{status} indicates whether or not the request was granted, and information as to the reason for any denial of the request shall be returned in the \refarg{results} array.

%%%%%%%%%%%%%%%%%%%%%%%%%%%%%%%%%%%%%%%%%%%%%%%%%
\subsection{\code{PMIx_Process_monitor_nb}}
\declareapi{PMIx_Process_monitor_nb}

%%%%
\summary

Request that application processes be monitored.

%%%%
\format

\copySignature{PMIx_Process_monitor_nb}{2.0}{
pmix_status_t \\
PMIx_Process_monitor_nb(const pmix_info_t *monitor, \\
\hspace*{24\sigspace}pmix_status_t error, \\
\hspace*{24\sigspace}const pmix_info_t directives[], \\
\hspace*{24\sigspace}size_t ndirs, \\
\hspace*{24\sigspace}pmix_info_cbfunc_t cbfunc, void *cbdata);
}

\begin{arglist}
\argin{monitor}{info (handle)}
\argin{error}{status (integer)}
\argin{directives}{Array of info structures (array of handles)}
\argin{ndirs}{Number of elements in the \refarg{directives} array (integer)}
\argin{cbfunc}{Callback function \refapi{pmix_info_cbfunc_t} (function reference)}
\argin{cbdata}{Data to be passed to the callback function (memory reference)}
\end{arglist}

Returns one of the following:

\begin{itemize}
    \item \refconst{PMIX_SUCCESS}, indicating that the request is being processed by the host environment - result will be returned in the provided \refarg{cbfunc}. Note that the library must not invoke the callback function prior to returning from the \ac{API}.
    \item \refconst{PMIX_OPERATION_SUCCEEDED}, indicating that the request was immediately processed and returned \textit{success} - the \refarg{cbfunc} will \textit{not} be called.
    \item a PMIx error constant indicating either an error in the input or that the request was immediately processed and failed - the \refarg{cbfunc} will \textit{not} be called.
\end{itemize}

\optattrstart
The following attributes may be implemented by a \ac{PMIx} library or by the host environment. If supported by the \ac{PMIx} server library, then the library must not pass the supported attributes to the host environment. All attributes not directly supported by the server library must be passed to the host environment if it supports this operation, and the library is \textit{required} to add the \refAttributeItem{PMIX_USERID} and the \refAttributeItem{PMIX_GRPID} attributes of the requesting process:

\pasteAttributeItem{PMIX_MONITOR_ID}
\pasteAttributeItem{PMIX_MONITOR_CANCEL}
\pasteAttributeItem{PMIX_MONITOR_APP_CONTROL}
\pasteAttributeItem{PMIX_MONITOR_HEARTBEAT}
\pasteAttributeItem{PMIX_MONITOR_HEARTBEAT_TIME}
\pasteAttributeItem{PMIX_MONITOR_HEARTBEAT_DROPS}
\pasteAttributeItem{PMIX_MONITOR_FILE}
\pasteAttributeItem{PMIX_MONITOR_FILE_SIZE}
\pasteAttributeItem{PMIX_MONITOR_FILE_ACCESS}
\pasteAttributeItem{PMIX_MONITOR_FILE_MODIFY}
\pasteAttributeItem{PMIX_MONITOR_FILE_CHECK_TIME}
\pasteAttributeItem{PMIX_MONITOR_FILE_DROPS}
\pasteAttributeItem{PMIX_SEND_HEARTBEAT}

\optattrend

%%%%
\descr

Non-blocking form of the \refapi{PMIx_Process_monitor} \ac{API}. The \refarg{cbfunc} function provides a \refarg{status} to indicate whether or not the request was granted, and to provide some information as to the reason for any denial in the \refapi{pmix_info_cbfunc_t} array of \refstruct{pmix_info_t} structures.

%%%%%%%%%%%%%%%%%%%%%%%%%%%%%%%%%%%%%%%%%%%%%%%%%
\subsection{\code{PMIx_Heartbeat}}
\declaremacro{PMIx_Heartbeat}

%%%%
\summary

Send a heartbeat to the \ac{PMIx} server library

%%%%
\format

\copySignature{PMIx_Heartbeat}{2.0}{
PMIx_Heartbeat();
}


%%%%
\descr

A simplified macro wrapping \refapi{PMIx_Process_monitor_nb} that sends a heartbeat to the \ac{PMIx} server library.

%%%%%%%%%%%%%%%%%%%%%%%%%%%%%%%%%%%%%%%%%%%%%%%%%
\subsection{Monitoring events}
\label{api:struct:events:monitor}

The following monitoring events may be available for registration, depending upon implementation and host environment support:

\begin{constantdesc}
%
\declareconstitem{PMIX_MONITOR_HEARTBEAT_ALERT}
Heartbeat failed to arrive within specified window. The process that triggered this alert will be identified in the event.
%
\declareconstitem{PMIX_MONITOR_FILE_ALERT}
File failed its monitoring detection criteria. The file that triggered this alert will be identified in the event.
%
\end{constantdesc}

%%%%%%%%%%%
\subsection{Monitoring attributes}
\label{api:struct:attributes:monitor}

Attributes used to control monitoring of an executing application- these are values passed to the \refapi{PMIx_Process_monitor_nb} \ac{API} and are not accessed using the \refapi{PMIx_Get} \ac{API}.

%
\declareAttribute{PMIX_MONITOR_ID}{"pmix.monitor.id"}{char*}{
Provide a string identifier for this request.
}
%
\declareAttribute{PMIX_MONITOR_CANCEL}{"pmix.monitor.cancel"}{char*}{
Identifier to be canceled (\code{NULL} means cancel all monitoring for this process).
}
%
\declareAttribute{PMIX_MONITOR_APP_CONTROL}{"pmix.monitor.appctrl"}{bool}{
The application desires to control the response to a monitoring event - i.e., the application is requesting that the host environment not take immediate action in response to the event (e.g., terminating the job).
}
%
\declareAttribute{PMIX_MONITOR_HEARTBEAT}{"pmix.monitor.mbeat"}{void}{
Register to have the PMIx server monitor the requestor for heartbeats.
}
%
\declareAttribute{PMIX_SEND_HEARTBEAT}{"pmix.monitor.beat"}{void}{
Send heartbeat to local PMIx server.
}
%
\declareAttribute{PMIX_MONITOR_HEARTBEAT_TIME}{"pmix.monitor.btime"}{uint32_t}{
Time in seconds before declaring heartbeat missed.
}
%
\declareAttribute{PMIX_MONITOR_HEARTBEAT_DROPS}{"pmix.monitor.bdrop"}{uint32_t}{
Number of heartbeats that can be missed before generating the event.
}
%
\declareAttribute{PMIX_MONITOR_FILE}{"pmix.monitor.fmon"}{char*}{
Register to monitor file for signs of life.
}
%
\declareAttribute{PMIX_MONITOR_FILE_SIZE}{"pmix.monitor.fsize"}{bool}{
Monitor size of given file is growing to determine if the application is running.
}
%
\declareAttribute{PMIX_MONITOR_FILE_ACCESS}{"pmix.monitor.faccess"}{char*}{
Monitor time since last access of given file to determine if the application is running.
}
%
\declareAttribute{PMIX_MONITOR_FILE_MODIFY}{"pmix.monitor.fmod"}{char*}{
Monitor time since last modified of given file to determine if the application is running.
}
%
\declareAttribute{PMIX_MONITOR_FILE_CHECK_TIME}{"pmix.monitor.ftime"}{uint32_t}{
Time in seconds between checking the file.
}
%
\declareAttribute{PMIX_MONITOR_FILE_DROPS}{"pmix.monitor.fdrop"}{uint32_t}{
Number of file checks that can be missed before generating the event.
}

%%%%%%%%%%%%%%%%%%%%%%%%%%%%%%%%%%%%%%%%%%%%%%%%%
%%%%%%%%%%%%%%%%%%%%%%%%%%%%%%%%%%%%%%%%%%%%%%%%%
\section{Logging}
\label{chap:api_job_mgmt:logging}

The logging interface supports posting information by applications and SMS elements to persistent storage. This function is \textit{not} intended for output of computational results, but rather for reporting status and saving state information such as inserting computation progress reports into the application's \ac{SMS} job log or error reports to the local syslog.

%%%%%%%%%%%%%%%%%%%%%%%%%%%%%%%%%%%%%%%%%%%%%%%%%
\subsection{\code{PMIx_Log}}
\declareapi{PMIx_Log}

%%%%
\summary

Log data to a data service.

%%%%
\format

\copySignature{PMIx_Log}{3.0}{
pmix_status_t \\
PMIx_Log(const pmix_info_t data[], size_t ndata, \\
\hspace*{9\sigspace}const pmix_info_t directives[], size_t ndirs);
}

\begin{arglist}
\argin{data}{Array of info structures (array of handles)}
\argin{ndata}{Number of elements in the \refarg{data} array (\code{size_t})}
\argin{directives}{Array of info structures (array of handles)}
\argin{ndirs}{Number of elements in the \refarg{directives} array (\code{size_t})}
\end{arglist}

Return codes are one of the following:

\begin{constantdesc}
    \item \refconst{PMIX_SUCCESS} The logging request was successful.
    \item \refconst{PMIX_ERR_BAD_PARAM} The logging request contains at least one incorrect entry.
    \item \refconst{PMIX_ERR_NOT_SUPPORTED} The \ac{PMIx} implementation or host environment does not support this function.
    \item other appropriate \ac{PMIx} error code
\end{constantdesc}

\reqattrstart
If the \ac{PMIx} library does not itself perform this operation, then it is required to pass any attributes provided by the client to the host environment for processing. In addition, it must include the following attributes in the passed \refarg{info} array:

\pasteAttributeItem{PMIX_USERID}
\pasteAttributeItem{PMIX_GRPID}

Host environments or \ac{PMIx} libraries that implement support for this operation are required to support the following attributes:

\pasteAttributeItem{PMIX_LOG_STDERR}
\pasteAttributeItem{PMIX_LOG_STDOUT}
\pasteAttributeItem{PMIX_LOG_SYSLOG}
\pasteAttributeItem{PMIX_LOG_LOCAL_SYSLOG}
\pasteAttributeItem{PMIX_LOG_GLOBAL_SYSLOG}
\pasteAttributeItem{PMIX_LOG_SYSLOG_PRI}
\pasteAttributeItem{PMIX_LOG_ONCE}

\reqattrend

\optattrstart
The following attributes are optional for host environments or \ac{PMIx} libraries that support this operation:

\pasteAttributeItem{PMIX_LOG_SOURCE}
\pasteAttributeItem{PMIX_LOG_TIMESTAMP}
\pasteAttributeItem{PMIX_LOG_GENERATE_TIMESTAMP}
\pasteAttributeItem{PMIX_LOG_TAG_OUTPUT}
\pasteAttributeItem{PMIX_LOG_TIMESTAMP_OUTPUT}
\pasteAttributeItem{PMIX_LOG_XML_OUTPUT}
\pasteAttributeItem{PMIX_LOG_EMAIL}
\pasteAttributeItem{PMIX_LOG_EMAIL_ADDR}
\pasteAttributeItem{PMIX_LOG_EMAIL_SENDER_ADDR}
\pasteAttributeItem{PMIX_LOG_EMAIL_SERVER}
\pasteAttributeItem{PMIX_LOG_EMAIL_SRVR_PORT}
\pasteAttributeItem{PMIX_LOG_EMAIL_SUBJECT}
\pasteAttributeItem{PMIX_LOG_EMAIL_MSG}
\pasteAttributeItem{PMIX_LOG_JOB_RECORD}
\pasteAttributeItem{PMIX_LOG_GLOBAL_DATASTORE}

\optattrend

%%%%
\descr

Log data subject to the services offered by the host environment. The data to be logged is provided in the \refarg{data} array. The (optional) \refarg{directives} can be used to direct the choice of logging channel.

\adviceuserstart
It is strongly recommended that the \refapi{PMIx_Log} API not be used by applications for streaming data as it is not a ``performant'' transport and can perturb the application since it involves the local \ac{PMIx} server and host \ac{SMS} daemon. Note that a return of \refconst{PMIX_SUCCESS} only denotes that the data was successfully handed to the appropriate system call (for local channels) or the host environment and does not indicate receipt at the final destination.
\adviceuserend

%%%%%%%%%%%%%%%%%%%%%%%%%%%%%%%%%%%%%%%%%%%%%%%%%
\subsection{\code{PMIx_Log_nb}}
\declareapi{PMIx_Log_nb}

%%%%
\summary

Log data to a data service.

%%%%
\format

\copySignature{PMIx_Log_nb}{2.0}{
pmix_status_t \\
PMIx_Log_nb(const pmix_info_t data[], size_t ndata, \\
\hspace*{12\sigspace}const pmix_info_t directives[], size_t ndirs, \\
\hspace*{12\sigspace}pmix_op_cbfunc_t cbfunc, void *cbdata);
}

\begin{arglist}
\argin{data}{Array of info structures (array of handles)}
\argin{ndata}{Number of elements in the \refarg{data} array (\code{size_t})}
\argin{directives}{Array of info structures (array of handles)}
\argin{ndirs}{Number of elements in the \refarg{directives} array (\code{size_t})}
\argin{cbfunc}{Callback function \refapi{pmix_op_cbfunc_t} (function reference)}
\argin{cbdata}{Data to be passed to the callback function (memory reference)}
\end{arglist}

Return codes are one of the following:

\begin{constantdesc}
\item \refconst{PMIX_SUCCESS} The logging request is valid and is being processed. The resulting status from the operation will be provided in the callback function. Note that the library must not invoke the callback function prior to returning from the \ac{API}.
\item \refconst{PMIX_OPERATION_SUCCEEDED}, indicating that the request was immediately processed and returned \textit{success} - the \refarg{cbfunc} will \textit{not} be called
\item \refconst{PMIX_ERR_BAD_PARAM} The logging request contains at least one incorrect entry that prevents it from being processed. The callback function will not be called.
\item \refconst{PMIX_ERR_NOT_SUPPORTED} The \ac{PMIx} implementation does not support this function. The callback function will not be called.
\item other appropriate \ac{PMIx} error code - the callback function will not be called.
\end{constantdesc}

\reqattrstart
If the \ac{PMIx} library does not itself perform this operation, then it is required to pass any attributes provided by the client to the host environment for processing. In addition, it must include the following attributes in the passed \refarg{info} array:

\pasteAttributeItem{PMIX_USERID}
\pasteAttributeItem{PMIX_GRPID}

Host environments or \ac{PMIx} libraries that implement support for this operation are required to support the following attributes:

\pasteAttributeItem{PMIX_LOG_STDERR}
\pasteAttributeItem{PMIX_LOG_STDOUT}
\pasteAttributeItem{PMIX_LOG_SYSLOG}
\pasteAttributeItem{PMIX_LOG_LOCAL_SYSLOG}
\pasteAttributeItem{PMIX_LOG_GLOBAL_SYSLOG}
\pasteAttributeItem{PMIX_LOG_SYSLOG_PRI}
\pasteAttributeItem{PMIX_LOG_ONCE}

\reqattrend

\optattrstart
The following attributes are optional for host environments or \ac{PMIx} libraries that support this operation:

\pasteAttributeItem{PMIX_LOG_SOURCE}
\pasteAttributeItem{PMIX_LOG_TIMESTAMP}
\pasteAttributeItem{PMIX_LOG_GENERATE_TIMESTAMP}
\pasteAttributeItem{PMIX_LOG_TAG_OUTPUT}
\pasteAttributeItem{PMIX_LOG_TIMESTAMP_OUTPUT}
\pasteAttributeItem{PMIX_LOG_XML_OUTPUT}
\pasteAttributeItem{PMIX_LOG_EMAIL}
\pasteAttributeItem{PMIX_LOG_EMAIL_ADDR}
\pasteAttributeItem{PMIX_LOG_EMAIL_SENDER_ADDR}
\pasteAttributeItem{PMIX_LOG_EMAIL_SERVER}
\pasteAttributeItem{PMIX_LOG_EMAIL_SRVR_PORT}
\pasteAttributeItem{PMIX_LOG_EMAIL_SUBJECT}
\pasteAttributeItem{PMIX_LOG_EMAIL_MSG}
\pasteAttributeItem{PMIX_LOG_JOB_RECORD}
\pasteAttributeItem{PMIX_LOG_GLOBAL_DATASTORE}

\optattrend

%%%%
\descr

Log data subject to the services offered by the host environment. The data to be logged is provided in the \refarg{data} array. The (optional) \refarg{directives} can be used to direct the choice of logging channel.
The callback function will be executed when the log operation has been completed. The \refarg{data} and \refarg{directives} arrays must be maintained until the callback is provided.

\adviceuserstart
It is strongly recommended that the \refapi{PMIx_Log_nb} API not be used by applications for streaming data as it is not a ``performant'' transport and can perturb the application since it involves the local \ac{PMIx} server and host \ac{SMS} daemon. Note that a return of \refconst{PMIX_SUCCESS} only denotes that the data was successfully handed to the appropriate system call (for local channels) or the host environment and does not indicate receipt at the final destination.
\adviceuserend


%%%%%%%%%%%%%%%%%%%%%%%%%%%%%%%%%%%%%%%%%%%%%%%%%
\subsection{Log attributes}
\label{api:struct:attributes:log}

Attributes used to describe \refapi{PMIx_Log} behavior - these are values passed to the \refapi{PMIx_Log} \ac{API} and therefore are not accessed using the \refapi{PMIx_Get} \ac{API}.

%
\declareAttribute{PMIX_LOG_SOURCE}{"pmix.log.source"}{pmix_proc_t*}{
ID of source of the log request.
}
%
\declareAttribute{PMIX_LOG_STDERR}{"pmix.log.stderr"}{char*}{
Log string to \code{stderr}.
}
%
\declareAttribute{PMIX_LOG_STDOUT}{"pmix.log.stdout"}{char*}{
Log string to \code{stdout}.
}
%
\declareAttribute{PMIX_LOG_SYSLOG}{"pmix.log.syslog"}{char*}{
Log data to syslog.
Defaults to \code{ERROR} priority.  Will log to global syslog if available, otherwise to local syslog.
}
%
\declareAttribute{PMIX_LOG_LOCAL_SYSLOG}{"pmix.log.lsys"}{char*}{
Log data to local syslog.
Defaults to \code{ERROR} priority.
}
%
\declareAttribute{PMIX_LOG_GLOBAL_SYSLOG}{"pmix.log.gsys"}{char*}{
Forward data to system ``gateway'' and log msg to that syslog
Defaults to \code{ERROR} priority.
}
%
\declareAttribute{PMIX_LOG_SYSLOG_PRI}{"pmix.log.syspri"}{int}{
Syslog priority level.
}
%
\declareAttribute{PMIX_LOG_TIMESTAMP}{"pmix.log.tstmp"}{time_t}{
Timestamp for log report.
}
%
\declareAttribute{PMIX_LOG_GENERATE_TIMESTAMP}{"pmix.log.gtstmp"}{bool}{
Generate timestamp for log.
}
%
\declareAttribute{PMIX_LOG_TAG_OUTPUT}{"pmix.log.tag"}{bool}{
Label the output stream with the channel name (e.g., ``stdout'').
}
%
\declareAttribute{PMIX_LOG_TIMESTAMP_OUTPUT}{"pmix.log.tsout"}{bool}{
Print timestamp in output string.
}
%
\declareAttribute{PMIX_LOG_XML_OUTPUT}{"pmix.log.xml"}{bool}{
Print the output stream in \ac{XML} format.
}
%
\declareAttribute{PMIX_LOG_ONCE}{"pmix.log.once"}{bool}{
Only log this once with whichever channel can first support it, taking the channels in priority order.
}
%
\declareAttribute{PMIX_LOG_MSG}{"pmix.log.msg"}{pmix_byte_object_t}{
Message blob to be sent somewhere.
}
%
\declareAttribute{PMIX_LOG_EMAIL}{"pmix.log.email"}{pmix_data_array_t}{
Log via email based on \refstruct{pmix_info_t} containing directives.
}
%
\declareAttribute{PMIX_LOG_EMAIL_ADDR}{"pmix.log.emaddr"}{char*}{
Comma-delimited list of email addresses that are to receive the message.
}
%
\declareAttribute{PMIX_LOG_EMAIL_SENDER_ADDR}{"pmix.log.emfaddr"}{char*}{
Return email address of sender.
}
%
\declareAttribute{PMIX_LOG_EMAIL_SUBJECT}{"pmix.log.emsub"}{char*}{
Subject line for email.
}
%
\declareAttribute{PMIX_LOG_EMAIL_MSG}{"pmix.log.emmsg"}{char*}{
Message to be included in email.
}
%
\declareAttribute{PMIX_LOG_EMAIL_SERVER}{"pmix.log.esrvr"}{char*}{
Hostname (or \ac{IP} address) of SMTP server.
}
%
\declareAttribute{PMIX_LOG_EMAIL_SRVR_PORT}{"pmix.log.esrvrprt"}{int32_t}{
Port the email server is listening to.
}
%
\declareAttribute{PMIX_LOG_GLOBAL_DATASTORE}{"pmix.log.gstore"}{bool}{
Store the log data in a global data store (e.g., database).
}
%
\declareAttribute{PMIX_LOG_JOB_RECORD}{"pmix.log.jrec"}{bool}{
Log the provided information to the host environment's job record.
}

%%%%%%%%%%%%%%%%%%%%%%%%%%%%%%%%%%%%%%%%%%%%%%%%%


    % PMIx Process Sets and Groups
    %%%%%%%%%%%%%%%%%%%%%%%%%%%%%%%%%%%%%%%%%%%%%%%%%
% Chapter: Process Sets and Groups
%%%%%%%%%%%%%%%%%%%%%%%%%%%%%%%%%%%%%%%%%%%%%%%%%
\chapter{Process Sets and Groups}
\label{chap:api_sets_groups}

\ac{PMIx} supports two slightly related, but functionally different concepts
known as \emph{process sets} and \emph{process groups}. This chapter defines
these two concepts and describes how they are utilized, along with their
corresponding \acp{API}.


%%%%%%%%%%%%%%%%%%%%%%%%%%%%%%%%%%%%%%%%%%%%%%%%%
%%%%%%%%%%%%%%%%%%%%%%%%%%%%%%%%%%%%%%%%%%%%%%%%%
\section{Process Sets}
\label{chap:api_sets_groups:sets}

A \ac{PMIx} \emph{Process Set} is a user-provided or host environment assigned
label associated with a given set of application processes. Processes can
belong to multiple process \emph{sets} at a time. Users may define a \ac{PMIx}
process set at time of application execution. For example, if using the command line parallel launcher "prun", one could specify process sets as follows:

\cspecificstart
\begin{codepar}
\$ prun -n 4 --pset ocean myoceanapp : -n 3 --pset ice myiceapp
\end{codepar}
\cspecificend

In this example, the processes in the first application will be labeled with a \refattr{PMIX_PSET_NAMES} attribute with a value of \emph{ocean} while those in the second application will be labeled with an \emph{ice} value. During the execution, application processes could lookup the process set attribute for any process using \refapi{PMIx_Get}. Alternatively, other executing applications could utilize the \refapi{PMIx_Query_info} \acp{API} to obtain the number of declared process sets in the system, a list of their names, and other information about them. In other words, the \emph{process set} identifier provides a label by which an application can derive information about a process and its application - it does \emph{not}, however, confer any operational function.

Host environments can create or delete process sets at any time through the
\refapi{PMIx_server_define_process_set} and
\refapi{PMIx_server_delete_process_set} \acp{API}. \ac{PMIx} servers shall
notify all local clients of process set operations via the
\refconst{PMIX_PROCESS_SET_DEFINE} or \refconst{PMIX_PROCESS_SET_DELETE}
events.

Process \emph{sets} differ from process \emph{groups} in several key ways:

\begin{itemize}
    \item Process \emph{sets} have no implied relationship between their members - i.e., a process in a process set has no concept of a ``pset rank'' as it would in a process \emph{group}.
    %
    \item Process \emph{set} identifiers are set by the host environment or by the user at time of application submission for execution -
    there are no \ac{PMIx} \acp{API} provided by which an application can define a process set or
    change a process \emph{set} membership. In contrast, \ac{PMIx} process
    \emph{groups} can only be defined dynamically by the application.
    %
    \item Process \emph{sets} are immutable - members cannot be added or removed once the set has been defined. In contrast, \ac{PMIx} process \emph{groups} can dynamically change their membership using the appropriate \acp{API}.
    %
    \item Process \emph{groups} can be used in calls to \ac{PMIx} operations. Members of process \emph{groups} that are involved in an operation are translated by their \ac{PMIx} server into their \emph{native} identifier prior to the operation being passed to the host environment. For example, an application can define a process group to consist of ranks 0 and 1 from the host-assigned namespace of \emph{210456}, identified by the group id of \emph{foo}. If the application subsequently calls the \refapi{PMIx_Fence} \ac{API} with a process identifier of \code{\{foo, PMIX_RANK_WILDCARD\}}, the \ac{PMIx} server will replace that identifier with an array consisting of \code{\{210456, 0\}} and \code{\{210456, 1\}} - the host-assigned identifiers of the participating processes - prior to processing the request.
    %
    \item Process \emph{groups} can request that the host environment assign a unique \code{size_t} \ac{PGCID} to the group at time of group construction. An \ac{MPI} library may, for example, use the \ac{PGCID} as the \ac{MPI} communicator identifier for the group.
    %
\end{itemize}

The two concepts do, however, overlap in that they both
involve collections of processes. Users desiring to create a process group
based on a process set could, for example, obtain the membership array of the
process set and use that as input to \refapi{PMIx_Group_construct}, perhaps
including the process set name as the group identifier for clarity. Note that
no linkage between the set and group of the same name is implied nor
maintained - e.g., changes in process group membership can not be
reflected in the process set using the same identifier.

\advicermstart
The host environment is responsible for ensuring:

\begin{itemize}
    \item consistent knowledge of process set membership across all involved
    \ac{PMIx} servers; and
    \item that process set names do not conflict with system-assigned namespaces within the scope of the set.
\end{itemize}

\advicermend


%%%%%%%%%%%%%%%%%%%%%%%%%%%%%%%%%%%%%%%%%%%%%%%%%
\subsection{Process Set Constants}

\versionMarker{4.0}
The \ac{PMIx} server is required to send a notification to all local clients upon creation or deletion of process sets. Client processes wishing to receive such
notifications must register for the corresponding event:

\begin{constantdesc}
%
\declareconstitemNEW{PMIX_PROCESS_SET_DEFINE}
The host environment has defined a new process set - the event will include the process set name (\refattr{PMIX_PSET_NAME}) and the membership (\refattr{PMIX_PSET_MEMBERS}).
%
\declareconstitemNEW{PMIX_PROCESS_SET_DELETE}
The host environment has deleted a process set - the event will include the process set name (\refattr{PMIX_PSET_NAME}).
%
\end{constantdesc}


%%%%%%%%%%%
\subsection{Process Set Attributes}

\versionMarker{4.0}
Several attributes are provided for querying the system regarding process sets using the \refapi{PMIx_Query_info} \acp{API}.

%
\declareAttributeNEW{PMIX_QUERY_NUM_PSETS}{"pmix.qry.psetnum"}{size_t}{
Return the number of process sets defined in the specified range (defaults
to \refconst{PMIX_RANGE_SESSION}).
}
%
\declareAttributeNEW{PMIX_QUERY_PSET_NAMES}{"pmix.qry.psets"}{pmix_data_array_t*}{
Return a \refstruct{pmix_data_array_t} containing an array of strings of the
process set names defined in the specified range (defaults to \refconst{PMIX_RANGE_SESSION}).
}
%
\declareAttributeNEW{PMIX_QUERY_PSET_MEMBERSHIP}{"pmix.qry.pmems"}{pmix_data_array_t*}{
Return an array of \refstruct{pmix_proc_t} containing
the members of the specified process set.
}
%

\vspace{\baselineskip}
The \refconst{PMIX_PROCESS_SET_DEFINE} event shall include the name of the newly defined process set and its members:
%
\declareAttributeNEW{PMIX_PSET_NAME}{"pmix.pset.nm"}{char*}{
The name of the newly defined process set.
}
%
\declareAttributeNEW{PMIX_PSET_MEMBERS}{"pmix.pset.mems"}{pmix_data_array_t*}{
An array of \refstruct{pmix_proc_t} containing
the members of the newly defined process set.
}

\vspace{\baselineskip}
In addition, a process can request (via \refapi{PMIx_Get}) the process sets to which a given process (including itself) belongs:

%
\declareAttributeNEW{PMIX_PSET_NAMES}{"pmix.pset.nms"}{pmix_data_array_t*}{
Returns an array of \code{char*} string names of the process sets in which the given process is a member.
}

%%%%%%%%%%%%%%%%%%%%%%%%%%%%%%%%%%%%%%%%%%%%%%%%%
%%%%%%%%%%%%%%%%%%%%%%%%%%%%%%%%%%%%%%%%%%%%%%%%%
\section{Process Groups}
\label{chap:api_sets_groups:groups}

\ac{PMIx} \emph{Groups} are defined as a collection of processes desiring a common, unique identifier for operational purposes such as passing events or participating in \ac{PMIx} fence operations. As with processes that assemble via \refapi{PMIx_Connect}, each member of the group is provided with both the job-level information of any other namespace represented in the group, and the contact information for all group members.

However, members of \ac{PMIx} Groups are \emph{loosely coupled} as opposed to \emph{tightly connected} when constructed via \refapi{PMIx_Connect}. Thus, \emph{groups} differ from \refapi{PMIx_Connect} assemblages in several key areas, as detailed in the following sections.

\subsection{Relation to the host environment}

Calls to \ac{PMIx} Group \acp{API} are first processed within the local \ac{PMIx} server. When constructed, the server creates a tracker that associates the specified processes with the user-provided group identifier, and assigns a new \emph{group rank} based on their relative position in the array of processes provided in the call to \refapi{PMIx_Group_construct}. Members of the group can subsequently utilize the group identifier in \ac{PMIx} function calls to address the group’s members, using either \refconst{PMIX_RANK_WILDCARD} to refer to all of them or the group-level rank of specific members. The \ac{PMIx} server will translate the specified processes into their \ac{RM}-assigned identifiers prior to passing the request up to its host. Thus, the host environment has no visibility into the group’s existence or membership.

In contrast, calls to \refapi{PMIx_Connect} are relayed to the host environment. This means that the host \ac{RM} should treat the failure of any process in the specified assemblage as a reportable event and take appropriate action. However, the environment is not required to define a new identifier for the connected assemblage or any of its member processes, nor does it define a new rank for each process within that assemblage. In addition, the \ac{PMIx} server does not provide any tracking support for the assemblage. Thus, the caller is responsible for addressing members of the connected assemblage using their \ac{RM}-provided identifiers.

\adviceuserstart
User-provided group identifiers must be distinct from both other group identifiers within the system and namespaces provided by the \ac{RM} so as to avoid collisions between group identifiers and \ac{RM}-assigned namespaces. This can usually be accomplished through the use of an application-specific prefix – e.g., ``myapp-foo''
\adviceuserend


\subsection{Construction procedure}

\refapi{PMIx_Connect} calls require that every process call the \ac{API} before completing – i.e., it is modeled upon the bulk synchronous traditional \ac{MPI} connect/accept methodology. Thus, a given application thread can only be involved in one connect/accept operation at a time, and is blocked in that operation until all specified processes participate. In addition, there is no provision for replacing processes in the assemblage due to failure to participate, nor a mechanism by which a process might decline participation.

In contrast, \ac{PMIx} Groups are designed to be more flexible in their construction procedure by relaxing these constraints. While a standard blocking form of constructing groups is provided, the event notification system is utilized to provide a designated \emph{group leader} with the ability to replace participants that fail to participate within a given timeout period. This provides a mechanism by which the application can, if desired, replace members on-the-fly or allow the group to proceed with partial membership. In such cases, the final group membership is returned to all participants upon completion of the operation.

Additionally, \ac{PMIx} supports dynamic definition of group membership based on an invite/join model. A process can asynchronously initiate construction of a group of any processes via the \refapi{PMIx_Group_invite} function call. Invitations are delivered via a \ac{PMIx} event (using the \refconst{PMIX_GROUP_INVITED} event) to the invited processes which can then either accept or decline the invitation using the \refapi{PMIx_Group_join} \ac{API}. The initiating process tracks responses by registering for the events generated by the call to \refapi{PMIx_Group_join}, timeouts, or process terminations, optionally replacing processes that decline the invitation, fail to respond in time, or terminate without responding. Upon completion of the operation, the final list of participants is communicated to each member of the new group.

\subsection{Destruct procedure}

Members of a \ac{PMIx} Group may depart the group at any time via the \refapi{PMIx_Group_leave} \ac{API}. Other members are notified of the departure via the \refconst{PMIX_GROUP_LEFT} event to distinguish such events from those reporting process termination. This leaves the remaining members free to continue group operations. The \refapi{PMIx_Group_destruct} operation offers a collective method akin to \refapi{PMIx_Disconnect} for deconstructing the entire group.

In contrast, processes that assemble via \refapi{PMIx_Connect} must all depart the assemblage together – i.e., no member can depart the assemblage while leaving the remaining members in it. Even the non-blocking form of \refapi{PMIx_Disconnect} retains this requirement in that members remain a part of the assemblage until all members have called \refapi{PMIx_Disconnect_nb}

Note that applications supporting dynamic group behaviors such as asynchronous departure take responsibility for ensuring global consistency in the group definition prior to executing group collective operations - i.e., it is the application's responsibility to either ensure that knowledge of the current group membership is globally consistent across the participants, or to register for appropriate events to deal with the lack of consistency during the operation.

\adviceuserstart
The reliance on \ac{PMIx} events in the \ac{PMIx} Group concept dictates that processes utilizing these \acp{API} must register for the corresponding events. Failure to do so will likely lead to operational failures. Users are recommended to utilize the \refattr{PMIX_TIMEOUT} directive (or retain an internal timer) on calls to \ac{PMIx} Group \acp{API} (especially the blocking form of those functions) as processes that have not registered for required events will never respond.
\adviceuserend

%%%%%%%%%%%%%%%%%%%%%%%%%%%%%%%%%%%%%%%%%%%%%%%%%
\subsection{Process Group Events}

\versionMarker{4.0}
Asynchronous process group operations rely heavily on \ac{PMIx} events.  The following events have been defined for that purpose.

\begin{constantdesc}
%
\declareconstitemNEW{PMIX_GROUP_INVITED}
The process has been invited to join a \ac{PMIx} Group - the identifier of the group and the ID's of other invited (or already joined) members will be included in the notification.
%
\declareconstitemNEW{PMIX_GROUP_LEFT}
A process has asynchronously left a \ac{PMIx} Group - the process identifier of the departing process will in included in the notification.
%
\declareconstitemNEW{PMIX_GROUP_MEMBER_FAILED}
A member of a \ac{PMIx} Group has abnormally terminated (i.e., without formally leaving the group prior to termination) - the process identifier of the failed process will be included in the notification.
%
\declareconstitemNEW{PMIX_GROUP_INVITE_ACCEPTED}
A process has accepted an invitation to join a \ac{PMIx} Group - the identifier of the group being joined will be included in the notification.
%
\declareconstitemNEW{PMIX_GROUP_INVITE_DECLINED}
A process has declined an invitation to join a \ac{PMIx} Group - the identifier of the declined group will be included in the notification.
%
\declareconstitemNEW{PMIX_GROUP_INVITE_FAILED}
An invited process failed or terminated prior to responding to the invitation - the identifier of the failed process will be included in the notification.
%
\declareconstitemNEW{PMIX_GROUP_MEMBERSHIP_UPDATE}
The membership of a \ac{PMIx} group has changed - the identifiers of the revised membership will be included in the notification.
%
\declareconstitemNEW{PMIX_GROUP_CONSTRUCT_ABORT}
Any participant in a \ac{PMIx} group construct operation that returns \refconst{PMIX_GROUP_CONSTRUCT_ABORT} from the \emph{leader failed} event handler will cause all participants to receive an event notifying them of that status. Similarly, the leader may elect to abort the procedure by either returning this error code from the handler assigned to the \refconst{PMIX_GROUP_INVITE_ACCEPTED} or \refconst{PMIX_GROUP_INVITE_DECLINED} codes, or by generating an event for the abort code. Abort events will be sent to all invited or existing members of the group.
%
\declareconstitemNEW{PMIX_GROUP_CONSTRUCT_COMPLETE}
The group construct operation has completed - the final membership will be included in the notification.
%
\declareconstitemNEW{PMIX_GROUP_LEADER_FAILED}
The current \emph{leader} of a group including this process has abnormally terminated - the group identifier will be included in the notification.
%
\declareconstitemNEW{PMIX_GROUP_LEADER_SELECTED}
A new \emph{leader} of a group including this process has been selected - the identifier of the new leader will be included in the notification.
%
\declareconstitemNEW{PMIX_GROUP_CONTEXT_ID_ASSIGNED}
A new \ac{PGCID} has been assigned by the host environment to a group that includes this process - the group identifier will be included in the notification.
%
\end{constantdesc}

%%%%%%%%%%%%%%%%%%%%%%%%%%%%%%%%%%%%%%%%%%%%%%%%%
\subsection{Process Group Attributes}

\versionMarker{4.0}
Attributes for querying the system regarding process groups include:

%
\declareAttributeNEW{PMIX_QUERY_NUM_GROUPS}{"pmix.qry.pgrpnum"}{size_t}{
Return the number of process groups defined in the specified range (defaults
to session). OPTIONAL QUALIFERS: \refattr{PMIX_RANGE}.
}
%
\declareAttributeNEW{PMIX_QUERY_GROUP_NAMES}{"pmix.qry.pgrp"}{pmix_data_array_t*}{
Return a \refstruct{pmix_data_array_t} containing an array of string names of
the process groups defined in the specified range (defaults to session). OPTIONAL QUALIFERS: \refattr{PMIX_RANGE}.
}
%
\declareAttributeNEW{PMIX_QUERY_GROUP_MEMBERSHIP}{"pmix.qry.pgrpmems"}{pmix_data_array_t*}{
Return a \refstruct{pmix_data_array_t} of \refstruct{pmix_proc_t} containing
the members of the specified process group. REQUIRED QUALIFIERS: \refattr{PMIX_GROUP_ID}.
}
%

\vspace{\baselineskip}
The following attributes are used as directives in \ac{PMIx} Group operations:

\declareAttributeNEW{PMIX_GROUP_ID}{"pmix.grp.id"}{char*}{
User-provided group identifier - as the group identifier may be used in
\ac{PMIx} operations, the user is required to ensure that the provided ID is unique within the scope of the host environment (e.g., by including some user-specific or application-specific prefix or suffix to the string).
}
%
\declareAttributeNEW{PMIX_GROUP_LEADER}{"pmix.grp.ldr"}{bool}{
This process is the leader of the group.
}
%
\declareAttributeNEW{PMIX_GROUP_OPTIONAL}{"pmix.grp.opt"}{bool}{
Participation is optional - do not return an error if any of the specified processes terminate without having joined. The default is \code{false}.
}
%
\declareAttributeNEW{PMIX_GROUP_NOTIFY_TERMINATION}{"pmix.grp.notterm"}{bool}{
Notify remaining members when another member terminates without first leaving the group.
}
%
\declareAttributeNEW{PMIX_GROUP_FT_COLLECTIVE}{"pmix.grp.ftcoll"}{bool}{
Adjust internal tracking on-the-fly for terminated processes during a \ac{PMIx} group collective operation.
}
%
\declareAttributeNEW{PMIX_GROUP_MEMBERSHIP}{"pmix.grp.mbrs"}{pmix_data_array_t*}{
Array \refstruct{pmix_proc_t} identifiers identifying the members of the specified group.
}
%
\declareAttributeNEW{PMIX_GROUP_ASSIGN_CONTEXT_ID}{"pmix.grp.actxid"}{bool}{
Requests that the \ac{RM} assign a new context identifier to the newly created group. The identifier is an unsigned, \code{size_t} value that the \ac{RM} guarantees to be unique across the range specified in the request. Thus, the value serves as a means of identifying the group within that range. If no range is specified, then the request defaults to \refconst{PMIX_RANGE_SESSION}.
}
%
\declareAttributeNEW{PMIX_GROUP_LOCAL_ONLY}{"pmix.grp.lcl"}{bool}{
Group operation only involves local processes. \ac{PMIx} implementations are \textit{required} to automatically scan an array of group members for local vs remote processes - if only local processes are detected, the implementation need not execute a global collective for the operation unless a context ID has been requested from the host environment. This can result in significant time savings. This attribute can be used to optimize the operation by indicating whether or not only local processes are represented, thus allowing the implementation to bypass the scan.
}

\vspace{\baselineskip}
The following attributes are used to return information at the conclusion of a \ac{PMIx} Group operation and/or in event notifications:

%
\declareAttributeNEW{PMIX_GROUP_CONTEXT_ID}{"pmix.grp.ctxid"}{size_t}{
Context identifier assigned to the group by the host \ac{RM}.
}
%
\declareAttributeNEW{PMIX_GROUP_ENDPT_DATA}{"pmix.grp.endpt"}{pmix_byte_object_t}{
Data collected during group construction to ensure communication between group members is supported upon completion of the operation.
}

\vspace{\baselineskip}
In addition, a process can request (via \refapi{PMIx_Get}) the process groups to which a given process (including itself) belongs:

%
\declareAttributeNEW{PMIX_GROUP_NAMES}{"pmix.pgrp.nm"}{pmix_data_array_t*}{
Returns an array of \code{char*} string names of the process groups in which the given process is a member.
}

%%%%%%%%%%%%%%%%%%%%%%%%%%%%%%%%%%%%%%%%%%%%%%%%%
\subsection{\code{PMIx_Group_construct}}
\declareapi{PMIx_Group_construct}

%%%%
\summary

Construct a \ac{PMIx} process group.

%%%%
\format

\copySignature{PMIx_Group_construct}{4.0}{
pmix_status_t \\
PMIx_Group_construct(const char grp[], \\
\hspace*{21\sigspace}const pmix_proc_t procs[], size_t nprocs, \\
\hspace*{21\sigspace}const pmix_info_t directives[], \\
\hspace*{21\sigspace}size_t ndirs, \\
\hspace*{21\sigspace}pmix_info_t **results, \\
\hspace*{21\sigspace}size_t *nresults);
}

\begin{arglist}
\argin{grp}{\code{NULL}-terminated character array of maximum size \refconst{PMIX_MAX_NSLEN} containing the group identifier (string)}
\argin{procs}{Array of \refstruct{pmix_proc_t} structures containing the \ac{PMIx} identifiers of the member processes (array of handles)}
\argin{nprocs}{Number of elements in the \refarg{procs} array (\code{size_t})}
\argin{directives}{Array of \refstruct{pmix_info_t} structures (array of handles)}
\argin{ndirs}{Number of elements in the \refarg{directives} array (\code{size_t})}
\arginout{results}{Pointer to a location where the array of \refstruct{pmix_info_t} describing the results of the operation is to be returned (pointer to handle)}
\arginout{nresults}{Pointer to a \code{size_t} location where the number of elements in \refarg{results} is to be returned (memory reference)}
\end{arglist}

Returns one of the following:

\begin{itemize}
    \item \refconst{PMIX_SUCCESS}, indicating that the request has been successfully completed
    \item \refconst{PMIX_ERR_NOT_SUPPORTED} The \ac{PMIx} library and/or the host \ac{RM} does not support this operation
    \item a \ac{PMIx} error constant indicating either an error in the input or that the request failed to be completed
\end{itemize}

\reqattrstart
The following attributes are \textit{required} to be supported by all \ac{PMIx} libraries that support this operation:

\pasteAttributeItem{PMIX_GROUP_LEADER}
\pasteAttributeItem{PMIX_GROUP_OPTIONAL}
\pasteAttributeItem{PMIX_GROUP_LOCAL_ONLY}
\pasteAttributeItem{PMIX_GROUP_FT_COLLECTIVE}

Host environments that support this operation are \textit{required} to support the following attributes:

\pasteAttributeItem{PMIX_GROUP_ASSIGN_CONTEXT_ID}
\pasteAttributeItem{PMIX_GROUP_NOTIFY_TERMINATION}

\reqattrend

\optattrstart
The following attributes are optional for host environments that support this operation:

\pasteAttributeItem{PMIX_TIMEOUT}

\optattrend

%%%%
\descr

Construct a new group composed of the specified processes and identified with the provided group identifier. The group identifier is a user-defined, \code{NULL}-terminated character array of length less than or equal to \refconst{PMIX_MAX_NSLEN}. Only characters accepted by standard string comparison functions (e.g., \emph{strncmp}) are supported. Processes may engage in multiple simultaneous group construct operations so long as each is provided with a unique group ID. The \refarg{directives} array can be used to pass user-level directives regarding timeout constraints and other options available from the \ac{PMIx} server.

If the \refattr{PMIX_GROUP_NOTIFY_TERMINATION} attribute is provided and has a value of \code{true}, then either the construct leader (if \refattr{PMIX_GROUP_LEADER} is provided) or all participants who register for the \refconst{PMIX_GROUP_MEMBER_FAILED} event will receive events whenever a process fails or terminates prior to calling \refapi{PMIx_Group_construct} – i.e. if a \emph{group leader} is declared, \textit{only} that process will receive the event. In the absence of a declared leader, \textit{all} specified group members will receive the event.

The event will contain the identifier of the process that failed to join plus any other information that the host \ac{RM} provided. This provides an opportunity for the leader or the collective members to react to the event – e.g., to decide to proceed with a smaller group or to abort the operation. The decision is communicated to the \ac{PMIx} library in the results array at the end of the event handler. This allows \ac{PMIx} to properly adjust accounting for procedure completion. When construct is complete, the participating \ac{PMIx} servers will be alerted to any change in participants and each group member will receive an updated group membership (marked with the \refattr{PMIX_GROUP_MEMBERSHIP} attribute) as part of the \refarg{results} array returned by this \ac{API}.

Failure of the declared leader at any time will cause a \refconst{PMIX_GROUP_LEADER_FAILED} event to be delivered to all participants so they can optionally declare a new leader. A new leader is identified by providing the \refattr{PMIX_GROUP_LEADER} attribute in the results array in the return of the event handler. Only one process is allowed to return that attribute, thereby declaring itself as the new leader. Results of the leader selection will be communicated to all participants via a \refconst{PMIX_GROUP_LEADER_SELECTED} event identifying the new leader. If no leader was selected, then the \refstruct{pmix_info_t} provided to that event handler will include that information so the participants can take appropriate action.

Any participant that returns \refconst{PMIX_GROUP_CONSTRUCT_ABORT} from either the \refconst{PMIX_GROUP_MEMBER_FAILED} or the \refconst{PMIX_GROUP_LEADER_FAILED} event handler will cause the construct process to abort, returning from the call with a \refconst{PMIX_GROUP_CONSTRUCT_ABORT} status.

If the \refattr{PMIX_GROUP_NOTIFY_TERMINATION} attribute is not provided or has a value of \code{false}, then the \refapi{PMIx_Group_construct} operation will simply return an error whenever a proposed group member fails or terminates prior to calling \refapi{PMIx_Group_construct}.

Providing the \refattr{PMIX_GROUP_OPTIONAL} attribute with a value of \code{true} directs the \ac{PMIx} library to consider participation by any specified group member as non-required - thus, the operation will return \refconst{PMIX_SUCCESS} if all members participate, or \refconst{PMIX_ERR_PARTIAL_SUCCESS} if some members fail to participate. The \refarg{results} array will contain the final group membership in the latter case. Note that this use-case can cause the operation to hang if the \refattr{PMIX_TIMEOUT} attribute is not specified and one or more group members fail to call \refapi{PMIx_Group_construct} while continuing to execute. Also, note that no leader or member failed events will be generated during the operation.

Processes in a group under construction are not allowed to leave the group until group construction is complete. Upon completion of the construct procedure, each group member will have access to the job-level information of all namespaces represented in the group plus any information posted via \refapi{PMIx_Put} (subject to the usual scoping directives) for every group member.

\adviceimplstart
At the conclusion of the construct operation, the \ac{PMIx} library is \emph{required} to ensure that job-related information from each participating namespace plus any information posted by group members via \refapi{PMIx_Put} (subject to scoping directives) is available to each member via calls to \refapi{PMIx_Get}.
\adviceimplend

\advicermstart
The collective nature of this \ac{API} generally results in use of a fence-like operation by the backend host environment. Host environments that utilize the array of process participants as a \emph{signature} for such operations may experience potential conflicts should both a \refapi{PMIx_Group_construct} and a \refapi{PMIx_Fence} operation involving the same participants be simultaneously executed. As \ac{PMIx} allows for such use-cases, it is therefore the responsibility of the host environment to resolve any potential conflicts.
\advicermend

%%%%%%%%%%%%%%%%%%%%%%%%%%%%%%%%%%%%%%%%%%%%%%%%%
\subsection{\code{PMIx_Group_construct_nb}}
\declareapi{PMIx_Group_construct_nb}

%%%%
\summary

Non-blocking form of \refapi{PMIx_Group_construct}.

%%%%
\format

\copySignature{PMIx_Group_construct_nb}{4.0}{
pmix_status_t \\
PMIx_Group_construct_nb(const char grp[], \\
\hspace*{24\sigspace}const pmix_proc_t procs[], size_t nprocs, \\
\hspace*{24\sigspace}const pmix_info_t directives[], \\
\hspace*{24\sigspace}size_t ndirs, \\
\hspace*{24\sigspace}pmix_info_cbfunc_t cbfunc, void *cbdata);
}

\begin{arglist}
\argin{grp}{\code{NULL}-terminated character array of maximum size \refconst{PMIX_MAX_NSLEN} containing the group identifier (string)}
\argin{procs}{Array of \refstruct{pmix_proc_t} structures containing the \ac{PMIx} identifiers of the member processes (array of handles)}
\argin{nprocs}{Number of elements in the \refarg{procs} array (\code{size_t})}
\argin{directives}{Array of \refstruct{pmix_info_t} structures (array of handles)}
\argin{ndirs}{Number of elements in the \refarg{directives} array (\code{size_t})}
\argin{cbfunc}{Callback function \refapi{pmix_info_cbfunc_t} (function reference)}
\argin{cbdata}{Data to be passed to the callback function (memory reference)}
\end{arglist}

Returns one of the following:

\begin{itemize}
\item \refconst{PMIX_SUCCESS} indicating that the request has been accepted for processing and the provided callback function will be executed upon completion of the operation. Note that the library \emph{must not} invoke the callback function prior to returning from the \ac{API}.
\item \refconst{PMIX_OPERATION_SUCCEEDED}, indicating that the request was immediately processed and returned \textit{success} - the \refarg{cbfunc} will \textit{not} be called.
\item \refconst{PMIX_ERR_NOT_SUPPORTED} The \ac{PMIx} library does not support this operation - the \refarg{cbfunc} will \textit{not} be called.
\item a non-zero \ac{PMIx} error constant indicating a reason for the request to have been rejected - the \refarg{cbfunc} will \textit{not} be called.
\end{itemize}

If executed, the status returned in the provided callback function will be one of the following constants:

\begin{itemize}
\item \refconst{PMIX_SUCCESS} The operation succeeded and all specified members participated.
\item \refconst{PMIX_ERR_PARTIAL_SUCCESS} The operation succeeded but not all specified members participated - the final group membership is included in the callback function.
\item \refconst{PMIX_ERR_NOT_SUPPORTED} While the \ac{PMIx} server supports this operation, the host \ac{RM} does not.
\item a non-zero \ac{PMIx} error constant indicating a reason for the request's failure.
\end{itemize}

\reqattrstart
\ac{PMIx} libraries that choose not to support this operation \textit{must} return \refconst{PMIX_ERR_NOT_SUPPORTED} when the function is called.

The following attributes are \textit{required} to be supported by all \ac{PMIx} libraries that support this operation:

\pasteAttributeItem{PMIX_GROUP_LEADER}
\pasteAttributeItem{PMIX_GROUP_OPTIONAL}
\pasteAttributeItem{PMIX_GROUP_LOCAL_ONLY}
\pasteAttributeItem{PMIX_GROUP_FT_COLLECTIVE}

Host environments that support this operation are \textit{required} to provide the following attributes:

\pasteAttributeItem{PMIX_GROUP_ASSIGN_CONTEXT_ID}
\pasteAttributeItem{PMIX_GROUP_NOTIFY_TERMINATION}

\reqattrend

\optattrstart
The following attributes are optional for host environments that support this operation:

\pasteAttributeItem{PMIX_TIMEOUT}

\optattrend

%%%%
\descr

Non-blocking version of the \refapi{PMIx_Group_construct} operation. The callback function will be called once all group members have called either \refapi{PMIx_Group_construct} or \refapi{PMIx_Group_construct_nb}.

%%%%%%%%%%%%%%%%%%%%%%%%%%%%%%%%%%%%%%%%%%%%%%%%%
\subsection{\code{PMIx_Group_destruct}}
\declareapi{PMIx_Group_destruct}

%%%%
\summary

Destruct a \ac{PMIx} process group.

%%%%
\format

\copySignature{PMIx_Group_destruct}{4.0}{
pmix_status_t \\
PMIx_Group_destruct(const char grp[], \\
\hspace*{20\sigspace}const pmix_info_t directives[], \\
\hspace*{20\sigspace}size_t ndirs);
}

\begin{arglist}
\argin{grp}{\code{NULL}-terminated character array of maximum size \refconst{PMIX_MAX_NSLEN} containing the identifier of the group to be destructed (string)}
\argin{directives}{Array of \refstruct{pmix_info_t} structures (array of handles)}
\argin{ndirs}{Number of elements in the \refarg{directives} array (\code{size_t})}
\end{arglist}

Returns one of the following:

\begin{itemize}
    \item \refconst{PMIX_SUCCESS}, indicating that the request has been successfully completed
    \item \refconst{PMIX_ERR_NOT_SUPPORTED} The \ac{PMIx} library and/or the host \ac{RM} does not support this operation
    \item a \ac{PMIx} error constant indicating either an error in the input or that the request failed to be completed
\end{itemize}

\reqattrstart
For implementations and host environments that support the operation, there are no identified required
attributes for this \ac{API}.
\reqattrend

\optattrstart
The following attributes are optional for host environments that support this operation:

\pasteAttributeItem{PMIX_TIMEOUT}

\optattrend

%%%%
\descr

Destruct a group identified by the provided group identifier. Processes may engage in multiple simultaneous group destruct operations so long as each involves a unique group ID. The \refarg{directives} array can be used to pass user-level directives regarding timeout constraints and other options available from the \ac{PMIx} server.

The destruct \ac{API} will return an error if any group process fails or terminates prior to calling \refapi{PMIx_Group_destruct} or its non-blocking version unless the \refattr{PMIX_GROUP_NOTIFY_TERMINATION} attribute was provided (with a value of \code{false}) at time of group construction. If notification was requested, then the \refconst{PMIX_GROUP_MEMBER_FAILED} event will be delivered for each process that fails to call destruct and the destruct tracker updated to account for the lack of participation. The \refapi{PMIx_Group_destruct} operation will subsequently return \refconst{PMIX_SUCCESS} when the remaining processes have all called destruct – i.e., the event will serve in place of return of an error.

\advicermstart
The collective nature of this \ac{API} generally results in use of a fence-like operation by the backend host environment. Host environments that utilize the array of process participants as a \emph{signature} for such operations may experience potential conflicts should both a \refapi{PMIx_Group_destruct} and a \refapi{PMIx_Fence} operation involving the same participants be simultaneously executed. As \ac{PMIx} allows for such use-cases, it is therefore the responsibility of the host environment to resolve any potential conflicts.
\advicermend

%%%%%%%%%%%%%%%%%%%%%%%%%%%%%%%%%%%%%%%%%%%%%%%%%
\subsection{\code{PMIx_Group_destruct_nb}}
\declareapi{PMIx_Group_destruct_nb}

%%%%
\summary

Non-blocking form of \refapi{PMIx_Group_destruct}.

%%%%
\format

\copySignature{PMIx_Group_destruct_nb}{4.0}{
pmix_status_t \\
PMIx_Group_destruct_nb(const char grp[], \\
\hspace*{23\sigspace}const pmix_info_t directives[], \\
\hspace*{23\sigspace}size_t ndirs, \\
\hspace*{23\sigspace}pmix_op_cbfunc_t cbfunc, void *cbdata);
}

\begin{arglist}
\argin{grp}{\code{NULL}-terminated character array of maximum size \refconst{PMIX_MAX_NSLEN} containing the identifier of the group to be destructed (string)}
\argin{directives}{Array of \refstruct{pmix_info_t} structures (array of handles)}
\argin{ndirs}{Number of elements in the \refarg{directives} array (\code{size_t})}
\argin{cbfunc}{Callback function \refapi{pmix_op_cbfunc_t} (function reference)}
\argin{cbdata}{Data to be passed to the callback function (memory reference)}
\end{arglist}

Returns one of the following:

\begin{itemize}
    \item \refconst{PMIX_SUCCESS}, indicating that the request is being processed - result will be returned in the provided \refarg{cbfunc}. Note that the library \emph{must not} invoke the callback function prior to returning from the \ac{API}.
    \item \refconst{PMIX_OPERATION_SUCCEEDED}, indicating that the request was immediately processed and returned \textit{success} - the \refarg{cbfunc} will \textit{not} be called
    \item \refconst{PMIX_ERR_NOT_SUPPORTED} The \ac{PMIx} library does not support this operation - the \refarg{cbfunc} will \textit{not} be called.
    \item a \ac{PMIx} error constant indicating either an error in the input or that the request was immediately processed and failed - the \refarg{cbfunc} will \textit{not} be called.
\end{itemize}

If executed, the status returned in the provided callback function will be one of the following constants:

\begin{itemize}
\item \refconst{PMIX_SUCCESS} The operation was successfully completed.
\item \refconst{PMIX_ERR_NOT_SUPPORTED} While the \ac{PMIx} server supports this operation, the host \ac{RM} does not.
\item a non-zero \ac{PMIx} error constant indicating a reason for the request's failure.
\end{itemize}

\reqattrstart
\ac{PMIx} libraries that choose not to support this operation \textit{must} return \refconst{PMIX_ERR_NOT_SUPPORTED} when the function is called. For implementations and host environments that support the operation, there are no identified required
attributes for this \ac{API}.
\reqattrend

\optattrstart
The following attributes are optional for host environments that support this operation:

\pasteAttributeItem{PMIX_TIMEOUT}

\optattrend

%%%%
\descr

Non-blocking version of the \refapi{PMIx_Group_destruct} operation. The callback function will be called once all members of the group have executed either \refapi{PMIx_Group_destruct} or \refapi{PMIx_Group_destruct_nb}.

%%%%%%%%%%%%%%%%%%%%%%%%%%%%%%%%%%%%%%%%%%%%%%%%%
\subsection{\code{PMIx_Group_invite}}
\declareapi{PMIx_Group_invite}

%%%%
\summary

Asynchronously construct a \ac{PMIx} process group.

%%%%
\format

\copySignature{PMIx_Group_invite}{4.0}{
pmix_status_t \\
PMIx_Group_invite(const char grp[], \\
\hspace*{18\sigspace}const pmix_proc_t procs[], size_t nprocs, \\
\hspace*{18\sigspace}const pmix_info_t directives[], size_t ndirs, \\
\hspace*{18\sigspace}pmix_info_t **results, size_t *nresult);
}

\begin{arglist}
\argin{grp}{\code{NULL}-terminated character array of maximum size \refconst{PMIX_MAX_NSLEN} containing the group identifier (string)}
\argin{procs}{Array of \refstruct{pmix_proc_t} structures containing the \ac{PMIx} identifiers of the processes to be invited (array of handles)}
\argin{nprocs}{Number of elements in the \refarg{procs} array (\code{size_t})}
\argin{directives}{Array of \refstruct{pmix_info_t} structures (array of handles)}
\argin{ndirs}{Number of elements in the \refarg{directives} array (\code{size_t})}
\arginout{results}{Pointer to a location where the array of \refstruct{pmix_info_t} describing the results of the operation is to be returned (pointer to handle)}
\arginout{nresults}{Pointer to a \code{size_t} location where the number of elements in \refarg{results} is to be returned (memory reference)}
\end{arglist}

Returns one of the following:

\begin{itemize}
    \item \refconst{PMIX_SUCCESS}, indicating that the request has been successfully completed.
    \item \refconst{PMIX_ERR_NOT_SUPPORTED} The \ac{PMIx} library and/or the host \ac{RM} does not support this operation.
    \item a \ac{PMIx} error constant indicating either an error in the input or that the request failed to be completed.
\end{itemize}

\reqattrstart
The following attributes are \textit{required} to be supported by all \ac{PMIx} libraries that support this operation:

\pasteAttributeItem{PMIX_GROUP_OPTIONAL}
\pasteAttributeItem{PMIX_GROUP_FT_COLLECTIVE}

Host environments that support this operation are \textit{required} to provide the following attributes:

\pasteAttributeItem{PMIX_GROUP_ASSIGN_CONTEXT_ID}
\pasteAttributeItem{PMIX_GROUP_NOTIFY_TERMINATION}
\reqattrend

\optattrstart
The following attributes are optional for host environments that support this operation:

\pasteAttributeItem{PMIX_TIMEOUT}

\optattrend

%%%%
\descr

Explicitly invite the specified processes to join a group. The process making the \refapi{PMIx_Group_invite} call is automatically declared to be the \emph{group leader}. Each invited process will be notified of the invitation via the \refconst{PMIX_GROUP_INVITED} event - the processes being invited must therefore register for the \refconst{PMIX_GROUP_INVITED} event in order to be notified of the invitation. Note that the \ac{PMIx} event notification system caches events - thus, no ordering of invite versus event registration is required.

The invitation event will include the identity of the inviting process plus the name of the group. When ready to respond, each invited process provides a response using either the blocking or non-blocking form of \refapi{PMIx_Group_join}. This will notify the inviting process that the invitation was either accepted (via the \refconst{PMIX_GROUP_INVITE_ACCEPTED} event) or declined (via the \refconst{PMIX_GROUP_INVITE_DECLINED} event). The \refconst{PMIX_GROUP_INVITE_ACCEPTED} event is captured by the \ac{PMIx} client library of the inviting process – i.e., the application itself does not need to register for this event. The library will track the number of accepting processes and alert the inviting process (by returning from the blocking form of \refapi{PMIx_Group_invite} or calling the callback function of the non-blocking form) when group construction completes.

The inviting process should, however, register for the \refconst{PMIX_GROUP_INVITE_DECLINED} if the application allows invited processes to decline the invitation. This provides an opportunity for the application to either invite a replacement, declare ``abort'', or choose to remove the declining process from the final group. The inviting process should also register to receive \refconst{PMIX_GROUP_INVITE_FAILED} events whenever a process fails or terminates prior to responding to the invitation. Actions taken by the inviting process in response to these events must be communicated at the end of the event handler by returning the corresponding result so that the \ac{PMIx} library can adjust accordingly.

Upon completion of the operation, all members of the new group will receive access to the job-level information of each other’s namespaces plus any information posted via \refapi{PMIx_Put} by the other members.

The inviting process is automatically considered the leader of the asynchronous group construction procedure and will receive all failure or termination events for invited members prior to completion. The inviting process is required to provide a \refconst{PMIX_GROUP_CONSTRUCT_COMPLETE} event once the group has been fully assembled – this event is used by the \ac{PMIx} library as a trigger to release participants from their call to \refapi{PMIx_Group_join} and provides information (e.g., the final group membership) to be returned in the \refarg{results} array.

Failure of the inviting process at any time will cause a \refconst{PMIX_GROUP_LEADER_FAILED} event to be delivered to all participants so they can optionally declare a new leader. A new leader is identified by providing the \refattr{PMIX_GROUP_LEADER} attribute in the results array in the return of the event handler. Only one process is allowed to return that attribute, declaring itself as the new leader. Results of the leader selection will be communicated to all participants via a \refconst{PMIX_GROUP_LEADER_SELECTED} event identifying the new leader. If no leader was selected, then the status code provided in the event handler will provide an error value so the participants can take appropriate action.

\adviceuserstart
Applications are not allowed to use the group in any operations until group construction is complete. This is required in order to ensure consistent knowledge of group membership across all participants.
\adviceuserend


%%%%%%%%%%%%%%%%%%%%%%%%%%%%%%%%%%%%%%%%%%%%%%%%%
\subsection{\code{PMIx_Group_invite_nb}}
\declareapi{PMIx_Group_invite_nb}

%%%%
\summary

Non-blocking form of \refapi{PMIx_Group_invite}.

%%%%
\format

\copySignature{PMIx_Group_invite_nb}{4.0}{
pmix_status_t \\
PMIx_Group_invite_nb(const char grp[], \\
\hspace*{21\sigspace}const pmix_proc_t procs[], size_t nprocs, \\
\hspace*{21\sigspace}const pmix_info_t directives[], size_t ndirs, \\
\hspace*{21\sigspace}pmix_info_cbfunc_t cbfunc, void *cbdata);
}

\begin{arglist}
\argin{grp}{\code{NULL}-terminated character array of maximum size \refconst{PMIX_MAX_NSLEN} containing the group identifier (string)}
\argin{procs}{Array of \refstruct{pmix_proc_t} structures containing the \ac{PMIx} identifiers of the processes to be invited (array of handles)}
\argin{nprocs}{Number of elements in the \refarg{procs} array (\code{size_t})}
\argin{directives}{Array of \refstruct{pmix_info_t} structures (array of handles)}
\argin{ndirs}{Number of elements in the \refarg{directives} array (\code{size_t})}
\argin{cbfunc}{Callback function \refapi{pmix_info_cbfunc_t} (function reference)}
\argin{cbdata}{Data to be passed to the callback function (memory reference)}
\end{arglist}

Returns one of the following:

\begin{itemize}
    \item \refconst{PMIX_SUCCESS}, indicating that the request is being processed - result will be returned in the provided \refarg{cbfunc}. Note that the library \emph{must not} invoke the callback function prior to returning from the \ac{API}.
    \item \refconst{PMIX_OPERATION_SUCCEEDED}, indicating that the request was immediately processed and returned \textit{success} - the \refarg{cbfunc} will \textit{not} be called.
    \item \refconst{PMIX_ERR_NOT_SUPPORTED} The \ac{PMIx} library does not support this operation - the \refarg{cbfunc} will \textit{not} be called.
    \item a PMIx error constant indicating either an error in the input or that the request was immediately processed and failed - the \refarg{cbfunc} will \textit{not} be called.
\end{itemize}

If executed, the status returned in the provided callback function will be one of the following constants:

\begin{itemize}
\item \refconst{PMIX_SUCCESS} The operation succeeded and all specified members participated.
\item \refconst{PMIX_ERR_PARTIAL_SUCCESS} The operation succeeded but not all specified members participated - the final group membership is included in the callback function.
\item \refconst{PMIX_ERR_NOT_SUPPORTED} While the \ac{PMIx} server supports this operation, the host \ac{RM} does not.
\item a non-zero \ac{PMIx} error constant indicating a reason for the request's failure.
\end{itemize}

\reqattrstart
The following attributes are \textit{required} to be supported by all \ac{PMIx} libraries that support this operation:

\pasteAttributeItem{PMIX_GROUP_OPTIONAL}
\pasteAttributeItem{PMIX_GROUP_FT_COLLECTIVE}

Host environments that support this operation are \textit{required} to provide the following attributes:

\pasteAttributeItem{PMIX_GROUP_ASSIGN_CONTEXT_ID}
\pasteAttributeItem{PMIX_GROUP_NOTIFY_TERMINATION}

\reqattrend

\optattrstart
The following attributes are optional for host environments that support this operation:

\pasteAttributeItem{PMIX_TIMEOUT}

\optattrend

%%%%
\descr

Non-blocking version of the \refapi{PMIx_Group_invite} operation. The callback function will be called once all invited members of the group (or their substitutes) have executed either \refapi{PMIx_Group_join} or \refapi{PMIx_Group_join_nb}.

%%%%%%%%%%%%%%%%%%%%%%%%%%%%%%%%%%%%%%%%%%%%%%%%%
\subsection{\code{PMIx_Group_join}}
\declareapi{PMIx_Group_join}

%%%%
\summary

Accept an invitation to join a \ac{PMIx} process group.

%%%%
\format

\copySignature{PMIx_Group_join}{4.0}{
pmix_status_t \\
PMIx_Group_join(const char grp[], \\
\hspace*{16\sigspace}const pmix_proc_t *leader, \\
\hspace*{16\sigspace}pmix_group_opt_t opt, \\
\hspace*{16\sigspace}const pmix_info_t directives[], size_t ndirs, \\
\hspace*{16\sigspace}pmix_info_t **results, size_t *nresult);
}

\begin{arglist}
\argin{grp}{\code{NULL}-terminated character array of maximum size \refconst{PMIX_MAX_NSLEN} containing the group identifier (string)}
\argin{leader}{Process that generated the invitation (handle)}
\argin{opt}{Accept or decline flag (\refstruct{pmix_group_opt_t})}
\argin{directives}{Array of \refstruct{pmix_info_t} structures (array of handles)}
\argin{ndirs}{Number of elements in the \refarg{directives} array (\code{size_t})}
\arginout{results}{Pointer to a location where the array of \refstruct{pmix_info_t} describing the results of the operation is to be returned (pointer to handle)}
\arginout{nresults}{Pointer to a \code{size_t} location where the number of elements in \refarg{results} is to be returned (memory reference)}
\end{arglist}

Returns one of the following:

\begin{itemize}
    \item \refconst{PMIX_SUCCESS}, indicating that the request has been successfully completed.
    \item \refconst{PMIX_ERR_NOT_SUPPORTED} The \ac{PMIx} library and/or the host \ac{RM} does not support this operation.
    \item a \ac{PMIx} error constant indicating either an error in the input or that the request failed to be completed.
\end{itemize}

\reqattrstart
There are no identified required attributes for implementers.

\reqattrend

\optattrstart
The following attributes are optional for host environments that support this operation:

\pasteAttributeItem{PMIX_TIMEOUT}

\optattrend

%%%%
\descr

Respond to an invitation to join a group that is being asynchronously constructed. The process must have registered for the \refconst{PMIX_GROUP_INVITED} event in order to be notified of the invitation. When called, the event information will include the \refstruct{pmix_proc_t} identifier of the process that generated the invitation along with the identifier of the group being constructed. When ready to respond, the process provides a response using either form of \refapi{PMIx_Group_join}.

\adviceuserstart
Since the process is alerted to the invitation in a \ac{PMIx} event handler, the process \emph{must not} use the blocking form of this call unless it first ``thread shifts'' out of the handler and into its own thread context. Likewise, while it is safe to call the non-blocking form of the \ac{API} from the event handler, the process \emph{must not} block in the handler while waiting for the callback function to be called.
\adviceuserend

Calling this function causes the inviting process (aka the \emph{group leader}) to be notified that the process has either accepted or declined the request. The blocking form of the \ac{API} will return once the group has been completely constructed or the group’s construction has failed (as described below) – likewise, the callback function of the non-blocking form will be executed upon the same conditions.

Failure of the leader during the call to \refapi{PMIx_Group_join} will cause a \refconst{PMIX_GROUP_LEADER_FAILED} event to be delivered to all invited participants so they can optionally declare a new leader. A new leader is identified by providing the \refattr{PMIX_GROUP_LEADER} attribute in the results array in the return of the event handler. Only one process is allowed to return that attribute, declaring itself as the new leader. Results of the leader selection will be communicated to all participants via a \refconst{PMIX_GROUP_LEADER_SELECTED} event identifying the new leader. If no leader was selected, then the status code provided in the event handler will provide an error value so the participants can take appropriate action.

Any participant that returns \refconst{PMIX_GROUP_CONSTRUCT_ABORT} from the leader failed event handler will cause all participants to receive an event notifying them of that status. Similarly, the leader may elect to abort the procedure by either returning \refconst{PMIX_GROUP_CONSTRUCT_ABORT} from the handler assigned to the \refconst{PMIX_GROUP_INVITE_ACCEPTED} or \refconst{PMIX_GROUP_INVITE_DECLINED} codes, or by generating an event for the abort code. Abort events will be sent to all invited participants.


%%%%%%%%%%%%%%%%%%%%%%%%%%%%%%%%%%%%%%%%%%%%%%%%%
\subsection{\code{PMIx_Group_join_nb}}
\declareapi{PMIx_Group_join_nb}

%%%%
\summary

Non-blocking form of \refapi{PMIx_Group_join}

%%%%
\format

\copySignature{PMIx_Group_join_nb}{4.0}{
pmix_status_t \\
PMIx_Group_join_nb(const char grp[], \\
\hspace*{19\sigspace}const pmix_proc_t *leader, \\
\hspace*{19\sigspace}pmix_group_opt_t opt, \\
\hspace*{19\sigspace}const pmix_info_t directives[], size_t ndirs, \\
\hspace*{19\sigspace}pmix_info_cbfunc_t cbfunc, void *cbdata);
}

\begin{arglist}
\argin{grp}{\code{NULL}-terminated character array of maximum size \refconst{PMIX_MAX_NSLEN} containing the group identifier (string)}
\argin{leader}{Process that generated the invitation (handle)}
\argin{opt}{Accept or decline flag (\refstruct{pmix_group_opt_t})}
\argin{directives}{Array of \refstruct{pmix_info_t} structures (array of handles)}
\argin{ndirs}{Number of elements in the \refarg{directives} array (\code{size_t})}
\argin{cbfunc}{Callback function \refapi{pmix_info_cbfunc_t} (function reference)}
\argin{cbdata}{Data to be passed to the callback function (memory reference)}
\end{arglist}

Returns one of the following:

\begin{itemize}
    \item \refconst{PMIX_SUCCESS}, indicating that the request is being processed - result will be returned in the provided \refarg{cbfunc}. Note that the library \emph{must not} invoke the callback function prior to returning from the \ac{API}.
    \item \refconst{PMIX_OPERATION_SUCCEEDED}, indicating that the request was immediately processed and returned \textit{success} - the \refarg{cbfunc} will \textit{not} be called.
    \item \refconst{PMIX_ERR_NOT_SUPPORTED} The \ac{PMIx} library does not support this operation - the \refarg{cbfunc} will \textit{not} be called.
    \item a PMIx error constant indicating either an error in the input or that the request was immediately processed and failed - the \refarg{cbfunc} will \textit{not} be called.
\end{itemize}

If executed, the status returned in the provided callback function will be one of the following constants:

\begin{itemize}
\item \refconst{PMIX_SUCCESS} The operation succeeded and group membership is in the callback function parameters.
\item \refconst{PMIX_ERR_NOT_SUPPORTED} While the \ac{PMIx} server supports this operation, the host \ac{RM} does not.
\item a non-zero \ac{PMIx} error constant indicating a reason for the request's failure.
\end{itemize}


\reqattrstart
There are no identified required attributes for implementers.

\reqattrend

\optattrstart
The following attributes are optional for host environments that support this operation:

\pasteAttributeItem{PMIX_TIMEOUT}

\optattrend

%%%%
\descr

Non-blocking version of the \refapi{PMIx_Group_join} operation. The callback function will be called once all invited members of the group (or their substitutes) have executed either \refapi{PMIx_Group_join} or \refapi{PMIx_Group_join_nb}.

%%%%%%%%%%%%%%%%%%%%%%%%%%%%%%%%%%%%%%%%%%%%%%%%%
\subsubsection{Group accept/decline directives}
\declarestruct{pmix_group_opt_t}

\versionMarker{4.0}
The \refstruct{pmix_group_opt_t} type is a \code{uint8_t} value used with the \refapi{PMIx_Group_join} \ac{API} to indicate \emph{accept} or \emph{decline} of the invitation - these are provided for readability of user code:

\begin{constantdesc}
%
\declareconstitem{PMIX_GROUP_DECLINE}
Decline the invitation.
%
\declareconstitem{PMIX_GROUP_ACCEPT}
Accept the invitation.
%
\end{constantdesc}


%%%%%%%%%%%%%%%%%%%%%%%%%%%%%%%%%%%%%%%%%%%%%%%%%
\subsection{\code{PMIx_Group_leave}}
\declareapi{PMIx_Group_leave}

%%%%
\summary

Leave a \ac{PMIx} process group.

%%%%
\format

\copySignature{PMIx_Group_leave}{4.0}{
pmix_status_t \\
PMIx_Group_leave(const char grp[], \\
\hspace*{17\sigspace}const pmix_info_t directives[], \\
\hspace*{17\sigspace}size_t ndirs);
}

\begin{arglist}
\argin{grp}{\code{NULL}-terminated character array of maximum size \refconst{PMIX_MAX_NSLEN} containing the group identifier (string)}
\argin{directives}{Array of \refstruct{pmix_info_t} structures (array of handles)}
\argin{ndirs}{Number of elements in the \refarg{directives} array (\code{size_t})}
\end{arglist}

Returns one of the following:

\begin{itemize}
    \item \refconst{PMIX_SUCCESS}, indicating that the request has been communicated to the local \ac{PMIx} server.
    \item \refconst{PMIX_ERR_NOT_SUPPORTED} The \ac{PMIx} library and/or the host \ac{RM} does not support this operation.
    \item a \ac{PMIx} error constant indicating either an error in the input or that the request is unsupported.
\end{itemize}

\reqattrstart
There are no identified required attributes for implementers.
\reqattrend


%%%%
\descr

Calls to \refapi{PMIx_Group_leave} (or its non-blocking form) will cause a \refconst{PMIX_GROUP_LEFT} event to be generated notifying all members of the group of the caller’s departure. The function will return (or the non-blocking function will execute the specified callback function) once the event has been locally generated and is not indicative of remote receipt.

\adviceuserstart
The \refapi{PMIx_Group_leave} API is intended solely for asynchronous departures of individual processes from a group as it is not a scalable operation – i.e., when a process determines it should no longer be a part of a defined group, but the remainder of the group retains a valid reason to continue in existence. Developers are advised to use \refapi{PMIx_Group_destruct} (or its non-blocking form) for all other scenarios as it represents a more scalable operation.
\adviceuserend

%%%%%%%%%%%%%%%%%%%%%%%%%%%%%%%%%%%%%%%%%%%%%%%%%
\subsection{\code{PMIx_Group_leave_nb}}
\declareapi{PMIx_Group_leave_nb}

%%%%
\summary

Non-blocking form of \refapi{PMIx_Group_leave}.

%%%%
\format

\copySignature{PMIx_Group_leave_nb}{4.0}{
pmix_status_t \\
PMIx_Group_leave_nb(const char grp[], \\
\hspace*{20\sigspace}const pmix_info_t directives[], \\
\hspace*{20\sigspace}size_t ndirs, \\
\hspace*{20\sigspace}pmix_op_cbfunc_t cbfunc, \\
\hspace*{20\sigspace}void *cbdata);
}

\begin{arglist}
\argin{grp}{\code{NULL}-terminated character array of maximum size \refconst{PMIX_MAX_NSLEN} containing the group identifier (string)}
\argin{directives}{Array of \refstruct{pmix_info_t} structures (array of handles)}
\argin{ndirs}{Number of elements in the \refarg{directives} array (\code{size_t})}
\argin{cbfunc}{Callback function \refapi{pmix_op_cbfunc_t} (function reference)}
\argin{cbdata}{Data to be passed to the callback function (memory reference)}
\end{arglist}

Returns one of the following:

\begin{itemize}
    \item \refconst{PMIX_SUCCESS}, indicating that the request is being processed - result will be returned in the provided \refarg{cbfunc}. Note that the library \emph{must not} invoke the callback function prior to returning from the \ac{API}.
    \item \refconst{PMIX_OPERATION_SUCCEEDED}, indicating that the request was immediately processed and returned \textit{success} - the \refarg{cbfunc} will \textit{not} be called.
    \item \refconst{PMIX_ERR_NOT_SUPPORTED} The \ac{PMIx} library does not support this operation - the \refarg{cbfunc} will \textit{not} be called.
    \item a PMIx error constant indicating either an error in the input or that the request was immediately processed and failed - the \refarg{cbfunc} will \textit{not} be called.
\end{itemize}

If executed, the status returned in the provided callback function will be one of the following constants:

\begin{itemize}
\item \refconst{PMIX_SUCCESS} The operation succeeded - i.e., the \refconst{PMIX_GROUP_LEFT} event was generated.
\item \refconst{PMIX_ERR_NOT_SUPPORTED} While the \ac{PMIx} library supports this operation, the host \ac{RM} does not.
\item a non-zero \ac{PMIx} error constant indicating a reason for the request's failure.
\end{itemize}


\reqattrstart
There are no identified required attributes for implementers.

\reqattrend

%%%%
\descr

Non-blocking version of the \refapi{PMIx_Group_leave} operation. The callback function will be called once the event has been locally generated and is not indicative of remote receipt.


%%%%%%%%%%%%%%%%%%%%%%%%%%%%%%%%%%%%%%%%%%%%%%%%%


    % PMIx fabric support interfaces
    %  - register/deregister fabric, get vertex/index
    %%%%%%%%%%%%%%%%%%%%%%%%%%%%%%%%%%%%%%%%%%%%%%%%%
% Chapter: API Fabric support
%%%%%%%%%%%%%%%%%%%%%%%%%%%%%%%%%%%%%%%%%%%%%%%%%
\chapter{Fabric Support Definitions}
\label{chap:api_fabric}

As the drive for performance continues, interest has grown in both scheduling algorithms that take into account network locality of the allocated resources, and in optimizing collective communication patterns by structuring them to follow fabric topology. Several interfaces have been defined that are specifically intended to support \acp{WLM} (also known as \emph{schedulers}) by providing access to information of potential use to scheduling algorithms - e.g., information on communication costs between different points on the fabric.

In contrast, hierarchical collective operations require each process have global information about both its peers and the fabric. For example, one might aggregate the contribution from all processes on a node, then again across all nodes on a common switch, and finally across all switches. Creating such optimized patterns relies on detailed knowledge of the fabric location of each participant.

\ac{PMIx} supports these efforts by defining datatypes and attributes by which fabric coordinates for processes and devices can be obtained from the host \ac{SMS}. When used in conjunction with the \ac{PMIx} \emph{instant on} methods, this results in the ability of a process to obtain the fabric coordinate of all other processes without incurring additional overhead associated with the publish/exchange of that information.



\section{Fabric Support Constants}
\label{api:sched:consts}

The following constants are defined for use in fabric-related events.

\begin{constantdesc}

%
\declareconstitemNEW{PMIX_FABRIC_UPDATE_PENDING}
The \ac{PMIx} server library has been alerted to a change in the fabric that requires updating of one or more registered \refstruct{pmix_fabric_t} objects.

%
\declareconstitemNEW{PMIX_FABRIC_UPDATED}
The \ac{PMIx} server library has completed updating the entries of all affected \refstruct{pmix_fabric_t} objects registered with the library. Access to the entries of those objects may now resume.

%
\declareconstitemNEW{PMIX_FABRIC_COORDS_UPDATED}
Fabric coordinates have been updated - the affected fabrics/planes are identified in the notification. Coordinates of processes and devices on those affected components should be refreshed prior to next use.

\end{constantdesc}

%%%%%%%%%%%
\section{Fabric Support Datatypes}

Several datatype definitions have been created to support fabric-related operations and information.

\subsection{Fabric Coordinate Structure}
\declarestruct{pmix_coord_t}

The \refstruct{pmix_coord_t} structure describes the fabric coordinates of a specified process in a given view

\versionMarker{4.0}
\cspecificstart
\begin{codepar}
typedef struct pmix_coord \{
    char *fabric;
    char *plane;
    pmix_coord_view_t view;
    uint32_t *coord;
    size_t dims;
\} pmix_coord_t;
\end{codepar}
\cspecificend

All coordinate values shall be expressed as unsigned integers due to their units being defined in fabric devices and not physical distances. The coordinate is therefore an indicator of connectivity and not relative communication distance.

The fabric and plane fields are assigned by the fabric provider to help the user identify the fabric to which the coordinates refer. Note that providers are not required to assign any particular value to the fields and may choose to leave the fields blank. Example entries include \{"Ethernet", "mgmt"\} or \{"infiniband", "data1"\}.

\adviceimplstart
Note that the \refstruct{pmix_coord_t} structure does not imply nor mandate any requirement on how the coordinate data is to be stored within the \ac{PMIx} library. Implementers are free to store the coordinate in whatever format they choose.
\adviceimplend

A fabric coordinate is usually associated with a given fabric device - e.g., a particular \ac{NIC} on a node. Thus, while the fabric coordinate of a device must be unique in a given view, the coordinate may be shared by multiple processes on a node. If the node contains multiple fabric devices, then either the device closest to the binding location of a process shall be used as its coordinate, or (if the process is unbound or its binding is not known) all devices on the node shall be reported as a \refstruct{pmix_data_array_t} of \refstruct{pmix_coord_t} structures.

Nodes with multiple fabric devices can also have those devices configured as multiple \refterm{fabric planes}. In such cases, a given process (even if bound to a specific location) may be associated with a coordinate on each plane. The resulting set of fabric coordinates shall be reported as a \refstruct{pmix_data_array_t} of \refstruct{pmix_coord_t} structures. The caller may request a coordinate from a specific fabric plane by passing the \refattr{PMIX_FABRIC_PLANE} attribute as a directive/qualifier to the \refapi{PMIx_Get} or \refapi{PMIx_Query_info_nb} call.

\subsection{Fabric Coordinate Support Macros}
\label{api:netcoord:macros}

The following macros are provided to support the \refstruct{pmix_coord_t} structure.

%%%%
\subsubsection{Initialize the \refstruct{pmix_coord_t} structure}
\declaremacro{PMIX_COORD_CONSTRUCT}

Initialize the \refstruct{pmix_coord_t} fields

\versionMarker{4.0}
\cspecificstart
\begin{codepar}
PMIX_COORD_CONSTRUCT(m)
\end{codepar}
\cspecificend

\begin{arglist}
\argin{m}{Pointer to the structure to be initialized (pointer to \refstruct{pmix_coord_t})}
\end{arglist}

%%%%
\subsubsection{Destruct the \refstruct{pmix_coord_t} structure}
\declaremacro{PMIX_COORD_DESTRUCT}

Destruct the \refstruct{pmix_coord_t} fields

\versionMarker{4.0}
\cspecificstart
\begin{codepar}
PMIX_COORD_DESTRUCT(m)
\end{codepar}
\cspecificend

\begin{arglist}
\argin{m}{Pointer to the structure to be destructed (pointer to \refstruct{pmix_coord_t})}
\end{arglist}

%%%%
\subsubsection{Create a \refstruct{pmix_coord_t} array}
\declaremacro{PMIX_COORD_CREATE}

Allocate and initialize a \refstruct{pmix_coord_t} array

\versionMarker{4.0}
\cspecificstart
\begin{codepar}
PMIX_COORD_CREATE(m, n)
\end{codepar}
\cspecificend

\begin{arglist}
\arginout{m}{Address where the pointer to the array of \refstruct{pmix_coord_t} structures shall be stored (handle)}
\argin{n}{Number of structures to be allocated (\code{size_t})}
\end{arglist}

%%%%
\subsubsection{Release a \refstruct{pmix_coord_t} array}
\declaremacro{PMIX_COORD_FREE}

Release an array of \refstruct{pmix_coord_t} structures

\versionMarker{4.0}
\cspecificstart
\begin{codepar}
PMIX_COORD_FREE(m, n)
\end{codepar}
\cspecificend

\begin{arglist}
\argin{m}{Pointer to the array of \refstruct{pmix_coord_t} structures (handle)}
\argin{n}{Number of structures in the array (\code{size_t})}
\end{arglist}


%%%%%%%%%%%%
\subsection{Fabric Coordinate Views}
\declarestruct{pmix_coord_view_t}

\versionMarker{4.0}
\cspecificstart
\begin{codepar}
typedef uint8_t pmix_coord_view_t;
#define PMIX_COORD_VIEW_UNDEF       0x00
#define PMIX_COORD_LOGICAL_VIEW     0x01
#define PMIX_COORD_PHYSICAL_VIEW    0x02
\end{codepar}
\cspecificend

Fabric coordinates can be reported based on different \emph{views} according to user preference at the time of request. The following views have been defined:

\begin{constantdesc}
%
\declareconstitemNEW{PMIX_COORD_VIEW_UNDEF}
The coordinate view has not been defined.
%
\declareconstitemNEW{PMIX_COORD_LOGICAL_VIEW}
The coordinates are provided in a \emph{logical} view, typically given in Cartesian (x,y,z) dimensions, that describes the data flow in the fabric as defined by the arrangement of the hierarchical addressing scheme, fabric segmentation, routing domains, and other similar factors employed by that fabric.
%
\declareconstitemNEW{PMIX_COORD_PHYSICAL_VIEW}
The coordinates are provided in a \emph{physical} view based on the actual wiring diagram of the fabric - i.e., values along each axis reflect the relative position of that interface on the specific fabric cabling.
%
\end{constantdesc}

\adviceimplstart
\ac{PMIx} library implementers are advised to avoid declaring the above constants as actual \code{enum} values in order to allow host environments to add support for possibly proprietary coordinate views.
\adviceimplend

If the requester does not specify a view, coordinates shall default to the \emph{logical} view.


\subsection{Fabric Link State}
\declarestruct{pmix_link_state_t}

The \refstruct{pmix_link_state_t} is a \code{uint32_t} type for fabric link states.

\versionMarker{4.0}
\cspecificstart
\begin{codepar}
typedef uint8_t pmix_link_state_t;
\end{codepar}
\cspecificend

The following constants can be used to set a variable of the type \refstruct{pmix_link_state_t}. All definitions were introduced in version 4 of the standard unless otherwise marked. Valid link state values start at zero.

\begin{constantdesc}
%
\declareconstitemNEW{PMIX_LINK_STATE_UNKNOWN}
The port state is unknown or not applicable.

\declareconstitemNEW{PMIX_LINK_DOWN}
The port is inactive.

\declareconstitemNEW{PMIX_LINK_UP}
The port is active.

\end{constantdesc}

\subsection{Fabric Operation Constants}
\declarestruct{pmix_fabric_operation_t}

\versionMarker{4.0}
The \refstruct{pmix_fabric_operation_t} structure is an enumerated type for specifying fabric operations used in the \ac{PMIx} server module's \refapi{pmix_server_fabric_fn_t} \ac{API}. All values were originally defined in version 4 of the standard unless otherwise marked.

\begin{constantdesc}
%
\declareconstitemNEW{PMIX_FABRIC_REQUEST_INFO}
Request information on a specific fabric - if the fabric isn't specified as per \refapi{PMIx_Fabric_register}, then return information on the system default fabric. Information to be returned is described in \refstruct{pmix_fabric_t}.
%
\declareconstitemNEW{PMIX_FABRIC_UPDATE_INFO}
Update information on a specific fabric - the index of the fabric (\refattr{PMIX_FABRIC_INDEX}) to be updated must be provided.
%
\declareconstitemNEW{PMIX_FABRIC_GET_VERTEX_INFO}
Request information on a specific \ac{NIC} within the identified fabric - the index of the device (\refattr{PMIX_FABRIC_DEVICE_INDEX}) and of the fabric (\refattr{PMIX_FABRIC_INDEX}) must be provided. If the \ac{NIC} identifier is not specified, then return vertex info on all \acp{NIC} in the fabric. Information to be included on each vertex is described in \refstruct{pmix_fabric_t}.

\adviceuserstart
Requesting information on every \ac{NIC} in the fabric may be an expensive operation in terms of both memory footprint and time.
\adviceuserend
%
\declareconstitemNEW{PMIX_FABRIC_GET_DEVICE_INDEX}
Request the fabric-wide index (returned as \refattr{PMIX_FABRIC_DEVICE_INDEX}) for a specific \ac{NIC} within the identified fabric based on the provided vertex information. The index of the fabric must be provided.
%
\end{constantdesc}


\subsection{Fabric registration structure}
\declarestruct{pmix_fabric_t}

The \refstruct{pmix_fabric_t} structure is used by a \ac{WLM} to interact with fabric-related \ac{PMIx} interfaces, and to provide information about the fabric for use in scheduling algorithms or other purposes.

\versionMarker{4.0}
\cspecificstart
\begin{codepar}
typedef struct pmix_fabric_s \{
    char *name;
    size_t index;
    pmix_info_t *info;
    size_t ninfo;
    void *module;
\} pmix_fabric_t;;
\end{codepar}
\cspecificend

Note that in this structure:

\begin{itemize}
    \item \refarg{name} is an optional user-supplied string name identifying the fabric being referenced by this struct. If provided, the field must be a \code{NULL}-terminated string composed of standard alphanumeric values supported by common utilities such as \textit{strcmp}.;
    \item \refarg{index} is a \ac{PMIx}-provided number identifying this object;
    \item \refarg{info} is an array of \refstruct{pmix_info_t} containing information (provided by the \ac{PMIx} library) about the fabric;
    \item \refarg{ninfo} is the number of elements in the \refarg{info} array
    \item \refarg{module} points to an opaque object reserved for use by the \ac{PMIx} server library.
\end{itemize}

Note that only the \refarg{name} field is provided by the user - all other fields are provided by the \ac{PMIx} library and must not be modified by the user. The \refarg{info} array contains a varying amount of information depending upon both the \ac{PMIx} implementation and information available from the fabric vendor. At a minimum, it must contain (ordering is arbitrary):

\reqattrstart

\pastePRIAttributeItem{PMIX_FABRIC_VENDOR}
\pastePRIAttributeItem{PMIX_FABRIC_IDENTIFIER}
\pastePRIAttributeItem{PMIX_FABRIC_NUM_VERTICES}

\reqattrend

and may optionally contain one or more of the following:

\optattrstart
\pastePRIAttributeItem{PMIX_FABRIC_COST_MATRIX}
\pastePRIAttributeItem{PMIX_FABRIC_GROUPS}
\pastePRIAttributeItem{PMIX_FABRIC_DIMS}
\pastePRIAttributeItem{PMIX_FABRIC_PLANE}
\pastePRIAttributeItem{PMIX_FABRIC_SHAPE}
\pastePRIAttributeItem{PMIX_FABRIC_SHAPE_STRING}

While unusual due to scaling issues, implementations may include an array of \refattr{PMIX_FABRIC_DEVICE} elements describing the vertex information for each \ac{NIC} in the system. Each element shall contain a \refstruct{pmix_data_array_t} of \refstruct{pmix_info_t} values describing the device. Each array may contain one or more of the following (ordering is arbitrary):

\pastePRIAttributeItem{PMIX_FABRIC_DEVICE_NAME}
\pastePRIAttributeItem{PMIX_FABRIC_DEVICE_VENDOR}
\pastePRIAttributeItem{PMIX_FABRIC_DEVICE_ID}
\pastePRIAttributeItem{PMIX_HOSTNAME}
\pastePRIAttributeItem{PMIX_FABRIC_DEVICE_DRIVER}
\pastePRIAttributeItem{PMIX_FABRIC_DEVICE_FIRMWARE}
\pastePRIAttributeItem{PMIX_FABRIC_DEVICE_ADDRESS}
\pastePRIAttributeItem{PMIX_FABRIC_DEVICE_MTU}
\pastePRIAttributeItem{PMIX_FABRIC_DEVICE_SPEED}
\pastePRIAttributeItem{PMIX_FABRIC_DEVICE_STATE}
\pastePRIAttributeItem{PMIX_FABRIC_DEVICE_TYPE}
\pastePRIAttributeItem{PMIX_FABRIC_DEVICE_BUS_TYPE}
\pastePRIAttributeItem{PMIX_FABRIC_DEVICE_PCI_DEVID}

\optattrend


\section{Fabric Support Attributes}
\label{api:sched:attrs}

The following attributes are used by the library supporting the system's \ac{WLM} to either access or return fabric-related information (e.g., as part of the \refstruct{pmix_fabric_t} structure).

\declareNewAttribute{PMIX_SERVER_SCHEDULER}{"pmix.srv.sched"}{bool}{
Server requests access to \ac{WLM}-supporting features - passed solely to the \refapi{PMIx_server_init} \ac{API} to indicate that the library is to be initialized for scheduler support.
}

\declareNewAttribute{PMIX_FABRIC_COST_MATRIX}{"pmix.fab.cm"}{pointer}{
Pointer to a two-dimensional array of point-to-point relative communication costs expressed as \code{uint16_t} values
}

\declareNewAttribute{PMIX_FABRIC_GROUPS}{"pmix.fab.grps"}{string}{
A string delineating the group membership of nodes in the system, where each fabric group consists of the group number followed by a colon and a comma-delimited list of nodes in that group, with the groups delimited by semi-colons (e.g., 0:node000,node002,node004,node006;1:node001,node003,node005,node007)
}

The following attributes may be returned by calls to the scheduler-related \acp{API} or in response to queries (e.g., \refapi{PMIx_Get} or \refapi{PMIx_Query_info}) made by processes or tools.

\declareNewAttribute{PMIX_FABRIC_VENDOR}{"pmix.fab.vndr"}{string}{
Name of fabric vendor (e.g., Amazon, Mellanox, Cray, Intel)
}

\declareNewAttribute{PMIX_FABRIC_IDENTIFIER}{"pmix.fab.id"}{string}{
An identifier for the fabric (e.g., MgmtEthernet, Slingshot-11, OmniPath-1)
}

\declareNewAttribute{PMIX_FABRIC_INDEX}{"pmix.fab.idx"}{size_t}{
The index of the fabric as returned in \refstruct{pmix_fabric_t}
}

\declareNewAttribute{PMIX_FABRIC_NUM_VERTICES}{"pmix.fab.nverts"}{size_t}{
Total number of \acp{NIC} in the system - corresponds to the number of vertices (i.e., rows and columns) in the cost matrix
}

%
\declareNewAttribute{PMIX_FABRIC_COORDINATE}{"pmix.fab.coord"}{pmix_data_array_t}{
Fabric coordinate(s) of the specified process in the view and/or plane provided by the requester. If only one \ac{NIC} has been assigned to the specified process, then the array will contain only one address. Otherwise, the array will contain the coordinates of all \acp{NIC} available to the process in order of least to greatest distance from the process (\acp{NIC} equally distant from the process will be listed in arbitrary order).
}

%
\declareNewAttribute{PMIX_FABRIC_VIEW}{"pmix.fab.view"}{pmix_coord_view_t}{
Fabric coordinate view to be used for the requested coordinate - see \refstruct{pmix_coord_view_t} for the list of accepted values.
}

%
\declareNewAttribute{PMIX_FABRIC_DIMS}{"pmix.fab.dims"}{uint32_t}{
Number of dimensions in the specified fabric plane/view. If no plane is specified in a request, then the dimensions of all planes in the system will be returned as a \refstruct{pmix_data_array_t} containing an array of \code{uint32_t} values. Default is to provide dimensions in \emph{logical} view.
}

%
\declareNewAttribute{PMIX_FABRIC_PLANE}{"pmix.fab.plane"}{char*}{
ID string of a fabric plane (e.g., CIDR for Ethernet). When used as a modifier in a request for information, specifies the plane whose information is to be returned. When used directly in a request, returns a \refstruct{pmix_data_array_t} of string identifiers for all fabric planes in the system.
}

%
\declareNewAttribute{PMIX_FABRIC_ENDPT}{"pmix.fab.endpt"}{pmix_data_array_t}{
Fabric endpoints for a specified process. As multiple endpoints may be assigned to a given process (e.g., in the case where multiple \acp{NIC} are associated with a package to which the process is bound), the returned values will be provided in a \refstruct{pmix_data_array_t} - the returned data type of the individual values in the array varies by fabric provider.
}

%
\declareNewAttribute{PMIX_FABRIC_SHAPE}{"pmix.fab.shape"}{pmix_data_array_t*}{
The size of each dimension in the specified fabric plane/view, returned in a \refstruct{pmix_data_array_t} containing an array of \code{uint32_t} values. The size is defined as the number of elements present in that dimension - e.g., the number of \acp{NIC} in one dimension of a physical view of a fabric plane. If no plane is specified, then the shape of each plane in the system will be returned in an array of fabric shapes. Default is to provide the shape in \emph{logical} view.
}

\declareNewAttribute{PMIX_FABRIC_SHAPE_STRING}{"pmix.fab.shapestr"}{string}{
Network shape expressed as a string (e.g., "10x12x2").
}

The following attributes are used to describe devices (a.k.a., \acp{NIC}) attached to the fabric.

\declareNewAttribute{PMIX_FABRIC_DEVICE}{"pmix.fabdev"}{\refstruct{pmix_data_array_t}}{
An array of \refstruct{pmix_info_t} describing a particular fabric device (\ac{NIC}).
}

\declareNewAttribute{PMIX_FABRIC_DEVICE_INDEX}{"pmix.fabdev.idx"}{\code{uint32_t}}{
System-unique index of a particular fabric device (\ac{NIC}).
}

\declareNewAttribute{PMIX_FABRIC_DEVICE_NAME}{"pmix.fabdev.nm"}{string}{
The operating system name associated with the device. This may be a logical fabric interface name (e.g. eth0 or eno1) or an absolute filename.
}

\declareNewAttribute{PMIX_FABRIC_DEVICE_VENDOR}{"pmix.fabdev.vndr"}{string}{
Indicates the name of the vendor that distributes the NIC.
}

\declareNewAttribute{PMIX_FABRIC_DEVICE_BUS_TYPE}{"pmix.fabdev.btyp"}{string}{
The type of bus to which the device is attached (e.g., "PCI", "GEN-Z").
}

\declareNewAttribute{PMIX_FABRIC_DEVICE_ID}{"pmix.fabdev.devid"}{string}{
This is a vendor-provided identifier for the device or product.
}

\declareNewAttribute{PMIX_FABRIC_DEVICE_DRIVER}{"pmix.fabdev.driver"}{string}{
The name of the driver associated with the device
}

\declareNewAttribute{PMIX_FABRIC_DEVICE_FIRMWARE}{"pmix.fabdev.fmwr"}{string}{
The device’s firmware version
}

\declareNewAttribute{PMIX_FABRIC_DEVICE_ADDRESS}{"pmix.fabdev.addr"}{string}{
The primary link-level address associated with the \ac{NIC}, such as a \ac{MAC} address. If multiple addresses are available, only one will be reported.
}

\declareNewAttribute{PMIX_FABRIC_DEVICE_MTU}{"pmix.fabdev.mtu"}{size_t}{
The maximum transfer unit of link level frames or packets, in bytes.
}

\declareNewAttribute{PMIX_FABRIC_DEVICE_SPEED}{"pmix.fabdev.speed"}{size_t}{
The active link data rate, given in bits per second.
}

\declareNewAttribute{PMIX_FABRIC_DEVICE_STATE}{"pmix.fabdev.state"}{\refstruct{pmix_link_state_t}}{
The last available physical port state. Possible values are \refconst{PMIX_LINK_STATE_UNKNOWN}, \refconst{PMIX_LINK_DOWN}, and \refconst{PMIX_LINK_UP}, to indicate if the port state is unknown or not applicable (unknown), inactive (down), or active (up).
}

\declareNewAttribute{PMIX_FABRIC_DEVICE_TYPE}{"pmix.fabdev.type"}{string}{
Specifies the type of fabric interface currently active on the device, such as Ethernet or InfiniBand.
}

\declareNewAttribute{PMIX_FABRIC_DEVICE_PCI_DEVID}{"pmix.fabdev.pcidevid"}{string}{
A node-level unique identifier for a \ac{PCI} device. Provided only if the device is located on a \ac{PCI} bus. The identifier is constructed as a four-part tuple delimited by colons comprised of the \ac{PCI} 16-bit domain, 8-bit bus, 8-bit device, and 8-bit function IDs, each expressed in zero-extended hexadecimal form. Thus, an example identifier might be "abc1:0f:23:01". The combination of node identifier (\refattr{PMIX_HOSTNAME} or \refattr{PMIX_NODEID}) and \refattr{PMIX_FABRIC_DEVICE_PCI_DEVID} shall be unique within the system.
}


%%%%%%%%%%%
\section{Fabric Support Functions}

The following \acp{API} allow the \ac{WLM} to request specific services from the fabric subsystem via the \ac{PMIx} library.

\advicermstart
Due to their high cost in terms of execution, memory consumption, and interactions with other \ac{SMS} components (e.g., a fabric manager), it is strongly advised that the underlying implementation of these \acp{API} be restricted to a single \ac{PMIx} server in a system that is supporting the \ac{SMS} component responsible for the scheduling of allocations (i.e., the system \refterm{scheduler}). The \refattr{PMIX_SERVER_SCHEDULER} attribute can be used for this purpose to control the execution path. Clients, tools, and other servers utilizing these functions are advised to have their requests forwarded to the server supporting the scheduler using the \refapi{pmix_server_fabric_fn_t} server module function, as needed.
\advicermend

%%%%%%%%%%%
\subsection{\code{PMIx_Fabric_register}}
\declareapi{PMIx_Fabric_register}

%%%%
\summary

Register for access to fabric-related information.

%%%%
\format

\versionMarker{4.0}
\cspecificstart
\begin{codepar}
pmix_status_t
PMIx_Fabric_register(pmix_fabric_t *fabric,
                     const pmix_info_t directives[],
                     size_t ndirs)
\end{codepar}
\cspecificend

\begin{arglist}
\argin{fabric}{address of a \refstruct{pmix_fabric_t} (backed by storage). User may populate the "name" field at will - \ac{PMIx} does not utilize this field (handle)}
\argin{directives}{an optional array of values indicating desired behaviors and/or fabric to be accessed. If \code{NULL}, then the highest priority available fabric will be used (array of handles)}
\argin{ndirs}{Number of elements in the \refarg{directives} array (integer)}
\end{arglist}

Returns \refconst{PMIX_SUCCESS} or a negative value corresponding to a \ac{PMIx} error constant.

\reqattrstart
The following directives are required to be supported by all \ac{PMIx} libraries to aid users in identifying the fabric whose data is being sought:

\pastePRIAttributeItem{PMIX_FABRIC_PLANE}
\pastePRIAttributeItem{PMIX_FABRIC_IDENTIFIER}
\pastePRIAttributeItem{PMIX_FABRIC_VENDOR}

\reqattrend

%%%%
\descr

Register for access to fabric-related information, including the communication cost matrix. This call must be made prior to requesting information from a fabric. The caller may request access to a particular fabric using the vendor, type, or identifier, or to a specific \refterm{fabric plane} via the \refattr{PMIX_FABRIC_PLANE} attribute - otherwise, the default fabric will be returned.

For performance reasons, the \ac{PMIx} library does not provide thread protection for accessing the information in the \refstruct{pmix_fabric_t} structure. Instead, the \ac{PMIx} implementation shall provide two methods for coordinating updates to the provided fabric information:

\begin{itemize}

    \item Users may periodically poll for updates using the \refapi{PMIx_Fabric_update} \ac{API}

    \item Users may register for \refconst{PMIX_FABRIC_UPDATE_PENDING} events indicating that an update to the cost matrix is pending. When received, users are required to terminate or pause any actions involving access to the cost matrix before returning from the event. Completion of the \refconst{PMIX_FABRIC_UPDATE_PENDING} event handler indicates to the \ac{PMIx} library that the fabric object's entries are available for updating. This may include releasing and re-allocating memory as the number of vertices may have changed (e.g., due to addition or removal of one or more \acp{NIC}). When the update has been completed, the \ac{PMIx} library will generate a \refconst{PMIX_FABRIC_UPDATED} event indicating that it is safe to begin using the updated fabric object(s).

\end{itemize}

There is no requirement that the caller exclusively use either one of these options. For example, the user may choose to both register for fabric update events, but poll for an update prior to some critical operation.

%%%%%%%%%%%
\subsection{\code{PMIx_Fabric_update}}
\declareapi{PMIx_Fabric_update}

%%%%
\summary

Update fabric-related information.

%%%%
\format

\versionMarker{4.0}
\cspecificstart
\begin{codepar}
pmix_status_t
PMIx_Fabric_update(pmix_fabric_t *fabric)
\end{codepar}
\cspecificend

\begin{arglist}
\argin{fabric}{address of a \refstruct{pmix_fabric_t} (backed by storage) (handle)}
\end{arglist}

Returns \refconst{PMIX_SUCCESS} or a negative value corresponding to a \ac{PMIx} error constant.

%%%%
\descr

Update fabric-related information. This call can be made at any time to request an update of the fabric information contained in the provided \refstruct{pmix_fabric_t} object. The caller is not allowed to access the provided \refstruct{pmix_fabric_t} until the call has returned.


%%%%%%%%%%%
\subsection{\code{PMIx_Fabric_deregister}}
\declareapi{PMIx_Fabric_deregister}

%%%%
\summary

Deregister a fabric object.

%%%%
\format

\versionMarker{4.0}
\cspecificstart
\begin{codepar}
pmix_status_t PMIx_Fabric_deregister(pmix_fabric_t *fabric)
\end{codepar}
\cspecificend

\begin{arglist}
\argin{input}{address of a \refstruct{pmix_fabric_t} (handle)}
\end{arglist}

Returns \refconst{PMIX_SUCCESS} or a negative value corresponding to a \ac{PMIx} error constant.

%%%%
\descr

Deregister a fabric object, providing an opportunity for the \ac{PMIx} library to cleanup any information (e.g., cost matrix) associated with it. Contents of the provided \refstruct{pmix_fabric_t} will be invalidated upon function return.


%%%%%%%%%%%
\subsection{\code{PMIx_Fabric_get_vertex_info}}
\declareapi{PMIx_Fabric_get_vertex_info}

%%%%
\summary

Given a communication cost matrix index for a specified fabric, return the corresponding vertex info.

%%%%
\format

\versionMarker{4.0}
\cspecificstart
\begin{codepar}
pmix_status_t
PMIx_Fabric_get_vertex_info(pmix_fabric_t *fabric, uint32_t index,
                            pmix_info_t **info, size_t *ninfo)
\end{codepar}
\cspecificend

\begin{arglist}
\argin{fabric}{address of a \refstruct{pmix_fabric_t} (handle)}
\argin{index}{vertex index (i.e., communication cost matrix row or column number) (integer)}
\arginout{info}{Address where a pointer to an array of \refstruct{pmix_info_t} containing the results of the query can be returned (memory reference)}
\arginout{ninfo}{Address where the number of elements in \refarg{info} can be returned (handle)}
\end{arglist}

Returns one of the following:

\begin{itemize}
    \item \refconst{PMIX_SUCCESS}, indicating return of a valid value.
    \item \refconst{PMIX_ERR_BAD_PARAM}, indicating that the provided index is out of bounds.
    \item a \ac{PMIx} error constant indicating either an error in the input or that the request failed.
\end{itemize}

%%%%
\descr

Query information about a specified vertex (fabric device, or \ac{NIC}) in the system. The returned \refarg{status} indicates if requested data was found or not. The returned array of \refstruct{pmix_info_t} will contain information on the specified vertex - the exact contents will depend on the \ac{PMIx} implementation and the fabric vendor. At a minimum, it must contain sufficient information to uniquely identify the device within the system (ordering is arbitrary):

\reqattrstart
\pastePRIAttributeItemBegin{PMIX_HOSTNAME} The \refattr{PMIX_NODEID} may be returned in its place, or in addition to the hostname.
\pastePRIAttributeItemEnd
\pastePRIAttributeItem{PMIX_FABRIC_DEVICE_NAME}
\pastePRIAttributeItem{PMIX_FABRIC_DEVICE_VENDOR}
\pastePRIAttributeItem{PMIX_FABRIC_DEVICE_BUS_TYPE}
\pastePRIAttributeItemBegin{PMIX_FABRIC_DEVICE_PCI_DEVID} This item should be included if the device bus type is \ac{PCI} - the equivalent should be provided for any other bus type.
\pastePRIAttributeItemEnd

\reqattrend

The returned array may optionally contain one or more of the following:

\optattrstart
\pastePRIAttributeItem{PMIX_FABRIC_DEVICE_ID}
\pastePRIAttributeItem{PMIX_FABRIC_DEVICE_DRIVER}
\pastePRIAttributeItem{PMIX_FABRIC_DEVICE_FIRMWARE}
\pastePRIAttributeItem{PMIX_FABRIC_DEVICE_ADDRESS}
\pastePRIAttributeItem{PMIX_FABRIC_DEVICE_MTU}
\pastePRIAttributeItem{PMIX_FABRIC_DEVICE_SPEED}
\pastePRIAttributeItem{PMIX_FABRIC_DEVICE_STATE}
\pastePRIAttributeItem{PMIX_FABRIC_DEVICE_TYPE}
\optattrend

The caller is responsible for releasing the returned array.



%%%%%%%%%%%
\subsection{\code{PMIx_Fabric_get_device_index}}
\declareapi{PMIx_Fabric_get_device_index}

%%%%
\summary

Given vertex info, return the corresponding communication cost matrix index.

%%%%
\format

\versionMarker{4.0}
\cspecificstart
\begin{codepar}
pmix_status_t
PMIx_Fabric_get_device_index(pmix_fabric_t *fabric,
                      const pmix_info_t vertex[], size_t ninfo,
                      uint32_t *index)
\end{codepar}
\cspecificend

\begin{arglist}
\argin{fabric}{address of a \refstruct{pmix_fabric_t} (handle)}
\argin{vertex}{array of \refstruct{pmix_info_t} containing info describing the vertex whose index is being queried (handle)}
\argin{ninfo} number of elements in \refarg{vertex}
\argout{index}{pointer to the location where the index is to be returned (memory reference (handle))}
\end{arglist}

Returns one of the following:

\begin{itemize}
    \item \refconst{PMIX_SUCCESS}, indicating return of a valid value.
    \item a \ac{PMIx} error constant indicating either an error in the input or that the request failed.
\end{itemize}


%%%%
\descr

Query the index number of a vertex corresponding to the provided description. The description must provide adequate information to uniquely identify the target vertex. At a minimum, this must include identification of the node hosting the device using either the \refattr{PMIX_HOSTNAME} or \refattr{PMIX_NODEID}, plus a node-level unique identifier for the device (e.g., the \refattr{PMIX_FABRIC_DEVICE_PCI_DEVID} for a \ac{PCI} device).


%%%%%%%%%%%%%%%%%%%%%%%%%%%%%%%%%%%%%%%%%%%%%%%%%


    % Security credentials
    %%%%%%%%%%%%%%%%%%%%%%%%%%%%%%%%%%%%%%%%%%%%%%%%%
% Chapter: Security
%%%%%%%%%%%%%%%%%%%%%%%%%%%%%%%%%%%%%%%%%%%%%%%%%
\chapter{Security}
\label{chap:api_security}

Applications and tools often interact with each other in ways that require verification of the identity of the user making the request, and access to that user's relevant authorizations. This is particularly important when tools connect directly to a system-level \ac{PMIx} server that may be operating at a privileged level. A variety of system management software packages provide this service, but the lack of standardized interfaces makes portability problematic.

This section defines two \ac{PMIx} client-side \acp{API} for this purpose. These are most likely to be used by user-space applications/tools, but are not restricted to that realm.


%%%%%%%%%%%%%%%%%%%%%%%%%%%%%%%%%%%%%%%%%%%%%%
%%%%%%%%%%%%%%%%%%%%%%%%%%%%%%%%%%%%%%%%%%%%%%
\section{Obtaining Credentials}
\label{chap:api_security:obtain}

The \ac{API} for obtaining a credential is a non-blocking operation since the host environment may have to contact a remote credential service. The definition takes into account the potential that the returned credential could be sent via some mechanism to another application that resides in an environment using a different security mechanism. Thus, provision is made for the system to return additional information (e.g., the identity of the issuing agent) outside of the credential itself and visible to the application.

%%%%%%%%%%%
\subsection{\code{PMIx_Get_credential}}
\declareapi{PMIx_Get_credential}

%%%%
\summary

Request a credential from the \ac{PMIx} server library or the host environment

%%%%
\format

\versionMarker{3.0}
\cspecificstart
\begin{codepar}
pmix_status_t
PMIx_Get_credential(const pmix_info_t info[], size_t ninfo,
                    pmix_credential_cbfunc_t cbfunc, void *cbdata)
\end{codepar}
\cspecificend

\begin{arglist}
\argin{info}{Array of \refattr{pmix_info_t} structures (array of handles)}
\argin{ninfo}{Number of elements in the \refarg{info} array (\code{size_t})}
\argin{cbfunc}{Callback function to return credential (\refapi{pmix_credential_cbfunc_t} function reference)}
\argin{cbdata}{Data to be passed to the callback function (memory reference)}
\end{arglist}

Returns one of the following:

\begin{itemize}
    \item \refconst{PMIX_SUCCESS}, indicating that the request has been communicated to the local \ac{PMIx} server - result will be returned in the provided \refarg{cbfunc}
    \item a \ac{PMIx} error constant indicating either an error in the input or that the request is unsupported - the \refarg{cbfunc} will \textit{not} be called
\end{itemize}

\reqattrstart
\ac{PMIx} libraries that choose not to support this operation \textit{must} return \refconst{PMIX_ERR_NOT_SUPPORTED} when the function is called. Implementations that support the operation may choose to internally execute integration for some security environments (e.g., directly contacting a \textit{munge} server) - there are no identified required attributes for this \ac{API}.

However, if the \ac{PMIx} implementation provides support for this \ac{API} and the request cannot be processed by the library itself, then any attributes that are provided by the client must be passed to the host environment for processing. In addition, the following attributes are required to be included in the \refarg{info} array passed from the \ac{PMIx} library to the host environment:

\pastePRIAttributeItem{PMIX_USERID}
\pastePRIAttributeItem{PMIX_GRPID}

\reqattrend

\optattrstart
The following attributes are optional for host environments that support this operation:

\pasteAttributeItem{PMIX_TIMEOUT}

\optattrend

\adviceimplstart
We recommend that implementation of the \refattr{PMIX_TIMEOUT} attribute be left to the host environment due to race condition considerations between completion of the operation versus internal timeout in the \ac{PMIx} server library. Implementers that choose to support \refattr{PMIX_TIMEOUT} directly in the \ac{PMIx} server library must take care to resolve the race condition and should avoid passing \refattr{PMIX_TIMEOUT} to the host environment so that multiple competing timeouts are not created.
\adviceimplend

%%%%
\descr

Request a credential from the \ac{PMIx} server library or the host environment

%%%%%%%%%%%%%%%%%%%%%%%%%%%%%%%%%%%%%%%%%%%%%%
%%%%%%%%%%%%%%%%%%%%%%%%%%%%%%%%%%%%%%%%%%%%%%
\section{Validating Credentials}
\label{chap:api_security:validate}

The \ac{API} for validating a credential is a non-blocking operation since the host environment may have to contact a remote credential service. Provision is made for the system to return additional information regarding possible authorization limitations beyond simple authentication.

%%%%%%%%%%%
\subsection{\code{PMIx_Validate_credential}}
\declareapi{PMIx_Validate_credential}

%%%%
\summary

Request validation of a credential by the \ac{PMIx} server library or the host environment

%%%%
\format

\versionMarker{3.0}
\cspecificstart
\begin{codepar}
pmix_status_t
PMIx_Validate_credential(const pmix_byte_object_t *cred,
                         const pmix_info_t info[], size_t ninfo,
                         pmix_validation_cbfunc_t cbfunc,
                         void *cbdata)
\end{codepar}
\cspecificend

\begin{arglist}
\argin{cred}{Pointer to \refstruct{pmix_byte_object_t} containing the credential (handle)}
\argin{info}{Array of \refstruct{pmix_info_t} structures (array of handles)}
\argin{ninfo}{Number of elements in the \refarg{info} array (\code{size_t})}
\argin{cbfunc}{Callback function to return result (\refapi{pmix_validation_cbfunc_t} function reference)}
\argin{cbdata}{Data to be passed to the callback function (memory reference)}
\end{arglist}

Returns one of the following:

\begin{itemize}
    \item \refconst{PMIX_SUCCESS}, indicating that the request has been communicated to the local \ac{PMIx} server - result will be returned in the provided \refarg{cbfunc}
    \item a \ac{PMIx} error constant indicating either an error in the input or that the request is unsupported - the \refarg{cbfunc} will \textit{not} be called
\end{itemize}

\reqattrstart
\ac{PMIx} libraries that choose not to support this operation \textit{must} return \refconst{PMIX_ERR_NOT_SUPPORTED} when the function is called. Implementations that support the operation may choose to internally execute integration for some security environments (e.g., directly contacting a \textit{munge} server) - there are no identified required attributes for this \ac{API}.

However, if the \ac{PMIx} implementation provides support for this \ac{API} and the request cannot be processed by the library itself, then any attributes that are provided by the client must be passed to the host environment for processing. In addition, the following attributes are required to be included in the \refarg{info} array passed from the \ac{PMIx} library to the host environment:

\pastePRIAttributeItem{PMIX_USERID}
\pastePRIAttributeItem{PMIX_GRPID}

\reqattrend

\optattrstart
The following attributes are optional for host environments that support this operation:

\pasteAttributeItem{PMIX_TIMEOUT}

\optattrend

\adviceimplstart
We recommend that implementation of the \refattr{PMIX_TIMEOUT} attribute be left to the host environment due to race condition considerations between completion of the operation versus internal timeout in the \ac{PMIx} server library. Implementers that choose to support \refattr{PMIX_TIMEOUT} directly in the \ac{PMIx} server library must take care to resolve the race condition and should avoid passing \refattr{PMIX_TIMEOUT} to the host environment so that multiple competing timeouts are not created.
\adviceimplend


%%%%
\descr

Request validation of a credential by the \ac{PMIx} server library or the host environment.



%%%%%%%%%%%%%%%%%%%%%%%%%%%%%%%%%%%%%%%%%%%%%%%%%


    % PMIx Server Specific Interfaces
    %  - setup_fork, (de)register_nspace, pmix_server_module_t
    %%%%%%%%%%%%%%%%%%%%%%%%%%%%%%%%%%%%%%%%%%%%%%%%%
% Chapter: API Server
%%%%%%%%%%%%%%%%%%%%%%%%%%%%%%%%%%%%%%%%%%%%%%%%%
\chapter{Server Specific Interfaces}
\label{chap:api_server}

\ldots


%%%%%%%%%%%
\subsection{\code{PMIx_generate_regex}}
\declareapi{PMIx_generate_regex}

%%%%
\summary

Generate a regular expression representation of the input string.

%%%%
\format

\cspecificstart
\begin{codepar}
pmix_status_t PMIx_generate_regex(const char *input, char **regex)
\end{codepar}
\cspecificend

\begin{arglist}
\argin{input}{String to process (string)}
\argout{regex}{Regular expression representation of \refarg{input} (string)}
\end{arglist}

Returns \refconst{PMIX_SUCCESS} or a negative value corresponding to a PMIx error constant.

%%%%
\descr

Given a semicolon-separated list of \refarg{input} values, generate a regular expression that can be passed down to the \ac{PMIx} client for parsing.
The caller is responsible for free'ing the resulting string.

If values have leading zero's, then that is preserved.
You have to add back any prefix/suffix for node names.

% JJH Format this
% * If values have leading zero's, then that is preserved. You
% * have to add back any prefix/suffix for node names
% * odin[009-015,017-023,076-086]
% *
% *     "pmix:odin[009-015,017-023,076-086]"
% *
% * Note that the "pmix" at the beginning of each regex indicates
% * that the PMIx native parser is to be used by the client for
% * parsing the provided regex. Other parsers may be supported - see
% * the pmix_client.h header for a list.


%%%%%%%%%%%
\subsection{\code{PMIx_generate_ppn}}
\declareapi{PMIx_generate_ppn}

%%%%
\summary

Generate a regular expression representation of the input string.

%%%%
\format

\cspecificstart
\begin{codepar}
pmix_status_t PMIx_generate_ppn(const char *input, char **ppn)
\end{codepar}
\cspecificend

\begin{arglist}
\argin{input}{String to process (string)}
\argout{regex}{Regular expression representation of \refarg{input} (string)}
\end{arglist}

Returns \refconst{PMIX_SUCCESS} or a negative value corresponding to a PMIx error constant.

%%%%
\descr

The input is expected to consist of a comma-separated list of ranges.

% JJH Format this
% * of ranges. Thus, an input of:
% *     "1-4;2-5;8,10,11,12;6,7,9"
% * would generate a regex of
% *     "[pmix:2x(3);8,10-12;6-7,9]"
% *
% * Note that the "pmix" at the beginning of each regex indicates
% * that the PMIx native parser is to be used by the client for
% * parsing the provided regex. Other parsers may be supported - see
% * the pmix_client.h header for a list.
% */


%%%%%%%%%%%
\subsection{\code{PMIx_server_register_nspace}}
\declareapi{PMIx_server_register_nspace}

%%%%
\summary

Setup the data about a particular namespace so it can be passed to any child process upon startup.

%%%%
\format

\cspecificstart
\begin{codepar}
pmix_status_t PMIx_server_register_nspace(const char nspace[], int nlocalprocs,
                                          pmix_info_t info[], size_t ninfo,
                                          pmix_op_cbfunc_t cbfunc, void *cbdata)
\end{codepar}
\cspecificend

\begin{arglist}
\argin{nspace}{namespace (string)}
\argin{nlocalprocs}{number of local processes (integer)}
\argin{info}{Array of info structures (array of handles)}
\argin{ninfo}{Number of elements in the \refarg{info} array (integer)}
\argin{cbfunc}{Callback function \refapi{pmix_op_cbfunc_t} (function reference)}
\argin{cbdata}{Data to be passed to the callback function (memory reference)}
\end{arglist}

Returns \refconst{PMIX_SUCCESS} or a negative value corresponding to a PMIx error constant.

%%%%
\descr

The PMIx connection procedure provides an opportunity for the host PMIx server to pass job-related info down to a child process.
This might include the number of processes in the job, relative local ranks of the processes within the job, and other information of use to the process.
The server is free to determine which, if any, of the supported elements it will provide (See \refsection{chap:struct}{Data Structures and Types} for values).

The PMIx server must register \emph{all} namespaces that will participate in collective operations with local processes.
This means that the server must register a namespace even if it will not host any local procs from within that nspace \emph{if} any local process might at some point perform a collective operation involving one or more processes from that namespace.
This is necessary so that the collective operation can know when it is locally complete.

The caller must also provide the number of local processes that will be launched within this namespace.
This is required for the PMIx server library to correctly handle collectives as a collective operation call can occur before all the processes have been started.


%%%%%%%%%%%
\subsection{\code{PMIx_server_deregister_nspace}}
\declareapi{PMIx_server_deregister_nspace}

%%%%
\summary

Deregister a namespace.

%%%%
\format

\cspecificstart
\begin{codepar}
void PMIx_server_deregister_nspace(const char nspace[],
                                   pmix_op_cbfunc_t cbfunc, void *cbdata)
\end{codepar}
\cspecificend

\begin{arglist}
\argin{nspace}{Namespace (string)}
\argin{cbfunc}{Callback function \refapi{pmix_op_cbfunc_t} (function reference)}
\argin{cbdata}{Data to be passed to the callback function (memory reference)}
\end{arglist}

%%%%
\descr

Deregister the specified \refarg{nspace} and purge all objects relating to it, including any client information from that namespace.
This is intended to support persistent PMIx servers by providing an opportunity for the host \ac{RM} to tell the PMIx server library to release all memory for a completed job.



%%%%%%%%%%%
\subsection{\code{PMIx_server_register_client}}
\declareapi{PMIx_server_register_client}

%%%%
\summary

Register a client process with the PMIx server library.

%%%%
\format

\cspecificstart
\begin{codepar}
pmix_status_t PMIx_server_register_client(const pmix_proc_t *proc,
                                          uid_t uid, gid_t gid,
                                          void *server_object,
                                          pmix_op_cbfunc_t cbfunc, void *cbdata)
\end{codepar}
\cspecificend

\begin{arglist}
\argin{proc}{\refstruct{pmix_proc_t} structure (handle)}
\argin{uid}{user id (integer)}
\argin{gid}{group id (integer)}
\argin{server_object}{(memory reference)}
\argin{cbfunc}{Callback function \refapi{pmix_op_cbfunc_t} (function reference)}
\argin{cbdata}{Data to be passed to the callback function (memory reference)}
\end{arglist}

Returns \refconst{PMIX_SUCCESS} or a negative value corresponding to a PMIx error constant.

%%%%
\descr

Register a client process with the PMIx server library.
The expected user ID and group ID of the child process helps the server library to properly authenticate clients as they connect by requiring the two values to match.

The host server can also, if it desires, provide an object it wishes to be returned when a server function is called that relates to a specific process.
For example, the host server may have an object that tracks the specific client.
Passing the object to the library allows the library to provide that object to the host server during subsequent calls related to that client, such as a ``pmix_server_client_connected_fn'' function.  This allows the host server to access the object without performing a lookup based the client's namespace and rank.


%%%%%%%%%%%
\subsection{\code{PMIx_server_deregister_client}}
\declareapi{PMIx_server_deregister_client}

%%%%
\summary

Deregister a client and purge all data relating to it.

%%%%
\format

\cspecificstart
\begin{codepar}
void PMIx_server_deregister_client(const pmix_proc_t *proc,
                                   pmix_op_cbfunc_t cbfunc, void *cbdata)
\end{codepar}
\cspecificend

\begin{arglist}
\argin{proc}{\refstruct{pmix_proc_t} structure (handle)}
\argin{cbfunc}{Callback function \refapi{pmix_op_cbfunc_t} (function reference)}
\argin{cbdata}{Data to be passed to the callback function (memory reference)}
\end{arglist}


%%%%
\descr

The \refapi{PMIx_server_deregister_nspace} API will automatically delete all client information for that namespace.
This API is therefore intended solely for use in exception cases.


%%%%%%%%%%%
\subsection{\code{PMIx_server_setup_fork}}
\declareapi{PMIx_server_setup_fork}

%%%%
\summary

Setup the environment of a child process to be forked by the host.

%%%%
\format

\cspecificstart
\begin{codepar}
pmix_status_t PMIx_server_setup_fork(const pmix_proc_t *proc, char ***env)
\end{codepar}
\cspecificend

\begin{arglist}
\argin{proc}{\refstruct{pmix_proc_t} structure (handle)}
\argin{env}{Environment array (array of strings)}
\end{arglist}

Returns \refconst{PMIX_SUCCESS} or a negative value corresponding to a PMIx error constant.

%%%%
\descr

Setup the environment of a child process to be forked by the host so it can correctly interact with the PMIx server.
The PMIx client needs some setup information so it can properly connect back to the server.
This function will set appropriate environmental variables for this purpose.


%%%%%%%%%%%
\subsection{\code{PMIx_server_dmodex_request}}
\declareapi{PMIx_server_dmodex_request}
\declareapi{pmix_dmodex_response_fn_t}

%%%%
\summary

Define a function by which the host server can request modex data from the local PMIx server.

%%%%
\format

\cspecificstart
\begin{codepar}
typedef void (*pmix_dmodex_response_fn_t)(pmix_status_t status,
                                          char *data, size_t sz,
                                          void *cbdata);

pmix_status_t PMIx_server_dmodex_request(const pmix_proc_t *proc,
                                         pmix_dmodex_response_fn_t cbfunc,
                                         void *cbdata)
\end{codepar}
\cspecificend

\begin{arglist}
\argin{proc}{\refstruct{pmix_proc_t} structure (handle)}
\argin{cbfunc}{Callback function \refapi{pmix_dmodex_response_fn_t} (function reference)}
\argin{cbdata}{Data to be passed to the callback function (memory reference)}
\end{arglist}

Returns \refconst{PMIX_SUCCESS} or a negative value corresponding to a PMIx error constant.

%%%%
\descr

Define a function by which the host server can request modex data from the local PMIx server.
This is used to support the direct modex operation (i.e., where data is cached locally on each PMIx server for its own local clients, and is obtained on-demand for remote requests.
Upon receiving a request from a remote server, the host server will call this function to pass the request into the PMIx server.
The PMIx server will return a blob (once it becomes available) via the \refarg{cbfunc} - the host server shall send the blob back to the original requestor.

The callback function used by the PMIx server to return direct modex requests to the host server.
The PMIx server will free the data blob upon return from the response function.


%%%%%%%%%%%
\subsection{\code{PMIx_server_setup_application}}
\declareapi{PMIx_server_setup_application}
\declareapi{pmix_setup_application_cbfunc_t}

%%%%
\summary

Provide a function by which the resource manager can request any application-specific environmental variables prior to launch of an application.
 
%%%%
\format

\cspecificstart
\begin{codepar}
typedef void (*pmix_setup_application_cbfunc_t)(pmix_status_t status,
                                                pmix_info_t info[], size_t ninfo,
                                                void *provided_cbdata,
                                                pmix_op_cbfunc_t cbfunc, void *cbdata)

pmix_status_t PMIx_server_setup_application(const char nspace[],
                                            pmix_info_t info[], size_t ninfo,
                                            pmix_setup_application_cbfunc_t cbfunc,
                                            void *cbdata)
\end{codepar}
\cspecificend

\begin{arglist}
\argin{nspace}{namespace (string)}
\argin{info}{Array of info structures (array of handles)}
\argin{ninfo}{Number of elements in the \refarg{info} array (integer)}
\argin{cbfunc}{Callback function \refapi{pmix_setup_application_cbfunc_t} (function reference)}
\argin{cbdata}{Data to be passed to the callback function (memory reference)}
\end{arglist}

Returns \refconst{PMIX_SUCCESS} or a negative value corresponding to a PMIx error constant.

%%%%
\descr

Provide a function by which the resource manager can request any application-specific environmental variables prior to launch of an application.
For example, network libraries may opt to provide security credentials for the application.
This is defined as a non-blocking operation in case network libraries need to perform some action before responding.
The returned env will be distributed along with the application

In the callback function, the returned \refarg{info} array is owned by the PMIx server library and will be free'd when the provided \refarg{cbfunc} is called.


%%%%%%%%%%%
\subsection{\code{PMIx_server_setup_local_support}}
\declareapi{PMIx_server_setup_local_support}

%%%%
\summary

Provide a function by which the local PMIx server can perform any application-specific operations prior to spawning local clients of a given application.

%%%%
\format

\cspecificstart
\begin{codepar}
pmix_status_t PMIx_server_setup_local_support(const char nspace[],
                                              pmix_info_t info[], size_t ninfo,
                                              pmix_op_cbfunc_t cbfunc, void *cbdata);
\end{codepar}
\cspecificend

\begin{arglist}
\argin{nspace}{Namespace (string)}
\argin{info}{Array of info structures (array of handles)}
\argin{ninfo}{Number of elements in the \refarg{info} array (integer)}
\argin{cbfunc}{Callback function \refapi{pmix_op_cbfunc_t} (function reference)}
\argin{cbdata}{Data to be passed to the callback function (memory reference)}
\end{arglist}

Returns \refconst{PMIX_SUCCESS} or a negative value corresponding to a PMIx error constant.

%%%%
\descr

Provide a function by which the local PMIx server can perform any application-specific operations prior to spawning local clients of a given application.
For example, a network library might need to setup the local driver for ``instant on'' addressing.


%%%%%%%%%%%
\section{Server Function Pointers}

The PMIx Server will set the function pointers in the \refapi{pmix_server_module_t} structure that they then pass to \refapi{PMIx_server_init}.
That module structure and associated function references is defined in this section.

%%%%%%%%%%%
\subsection{\code{pmix_server_module_t} Module}
\declareapi{pmix_server_module_t}

%%%%
\summary

List of function pointers that a PMIx server passes to \refapi{PMIx_server_init} during startup.

%%%%
\format

\cspecificstart
\begin{codepar}
typedef struct pmix_server_module_2_0_0_t {
    /* v1x interfaces */
    pmix_server_client_connected_fn_t   client_connected;
    pmix_server_client_finalized_fn_t   client_finalized;
    pmix_server_abort_fn_t              abort;
    pmix_server_fencenb_fn_t            fence_nb;
    pmix_server_dmodex_req_fn_t         direct_modex;
    pmix_server_publish_fn_t            publish;
    pmix_server_lookup_fn_t             lookup;
    pmix_server_unpublish_fn_t          unpublish;
    pmix_server_spawn_fn_t              spawn;
    pmix_server_connect_fn_t            connect;
    pmix_server_disconnect_fn_t         disconnect;
    pmix_server_register_events_fn_t    register_events;
    pmix_server_deregister_events_fn_t  deregister_events;
    pmix_server_listener_fn_t           listener;
    /* v2x interfaces */
    pmix_server_notify_event_fn_t       notify_event;
    pmix_server_query_fn_t              query;
    pmix_server_tool_connection_fn_t    tool_connected;
    pmix_server_log_fn_t                log;
    pmix_server_alloc_fn_t              allocate;
    pmix_server_job_control_fn_t        job_control;
    pmix_server_monitor_fn_t            monitor;
} pmix_server_module_t;
\end{codepar}
\cspecificend

%%%%
\descr

NOTE: for performance purposes, the host server is required to return as quickly as possible from all functions.
Execution of the function is thus to be done asynchronously so as to allow the PMIx server support library to handle multiple client requests as quickly and scalably as possible.

All data passed to the host server functions is ``owned'' by the PMIX server support library and MUST NOT be free'd.
Data returned by the host server via callback function is owned by the host server, which is free to release it upon return from the callback.



%%%%%%%%%%%
\subsection{\code{pmix_server_client_connected_fn_t}}
\declareapi{pmix_server_client_connected_fn_t}

%%%%
\summary

Notify the host server that a client connected to this server.

%%%%
\format

\cspecificstart
\begin{codepar}
typedef pmix_status_t (*pmix_server_client_connected_fn_t)(
                             const pmix_proc_t *proc, void* server_object,
                             pmix_op_cbfunc_t cbfunc, void *cbdata)
\end{codepar}
\cspecificend

\begin{arglist}
\argin{proc}{\refstruct{pmix_proc_t} structure (handle)}
\argin{server_object}{object reference (memory reference)}
\argin{cbfunc}{Callback function \refapi{pmix_op_cbfunc_t} (function reference)}
\argin{cbdata}{Data to be passed to the callback function (memory reference)}
\end{arglist}

Returns \refconst{PMIX_SUCCESS} or a negative value corresponding to a PMIx error constant.

%%%%
\descr

Notify the host server that a client has called PMIx_Init or PMIx_Tool_init.
\rcomment{I am guessing a bit on whether PMIx_Tool_init causes a call to pmix_server_client_connected_fn_t}
Note that the client will be in a blocked state until the host server executes the callback function, thus allowing the PMIx server support library to release 
the client.  
The server_object parameter will be the value of the server_object parameter passed to   
\refapi{PMIx_server_register_client} previously by the host server.  If provided, an implementation of \refapi{pmix_server_client_connected_fn_t} 
is only required to
call the callback function designated.  A host server can choose to not be notified when clients connect by setting \refapi{client_connected} to \code{NULL}. 

It is possible that only a subset of the clients in a namespace call PMIx_init.   The server's \refapi{pmix_server_client_connected_fn_t} implemenation 
should not depend on being called once per rank in a namespace or delaying calling the callback function until all ranks have connected.  
However, if a rank makes any PMIx calls, it must first call \refapi{PMIx_Init} and 
therefore the server's \refapi{mpix_server_client_connected_fn_t} will be called before any other server functions specific to the rank.

\adviceimplstart
 The \refapi{PMIx_server_client_connected_fn_t} implementation provided in the \refapi{pmix_server_module_2_0_0_t} is an opportunity for a host server 
 to update the status of the ranks it manages.  It is also a convenient and well defined time to perform initialization necessary to 
 support further calls into the server related to that rank. 
 \adviceimplend

%%%%%%%%%%%
\subsection{\code{pmix_server_client_finalized_fn_t}}
\declareapi{pmix_server_client_finalized_fn_t}

%%%%
\summary

Notify the host server that a client called \refapi{PMIx_Finalize}.

%%%%
\format

\cspecificstart
\begin{codepar}
typedef pmix_status_t (*pmix_server_client_finalized_fn_t)(
                             const pmix_proc_t *proc, void* server_object,
                             pmix_op_cbfunc_t cbfunc, void *cbdata)
\end{codepar}
\cspecificend

\begin{arglist}
\argin{proc}{\refstruct{pmix_proc_t} structure (handle)}
\argin{server_object}{object reference (memory reference)}
\argin{cbfunc}{Callback function \refapi{pmix_op_cbfunc_t} (function reference)}
\argin{cbdata}{Data to be passed to the callback function (memory reference)}
\end{arglist}

Returns \refconst{PMIX_SUCCESS} or a negative value corresponding to a PMIx error constant.

%%%%
\descr

Notify the host server that a client called \refapi{PMIx_Finalize}.
Note that the client will be in a blocked state until the host server executes the callback function, thus allowing the PMIx server support library to release the client.
The server_object parameter will be the value of the server_object parameter passed to   
\refapi{PMIx_server_register_client} previously by the host server.  If provided, an implementation of \refapi{pmix_server_client_finalized_fn_t} 
is only required to
call the callback function designated.  A host server can choose to not be notified when clients finalize by setting \refapi{client_finalized} to \code{NULL}. 

Note that the host server is only being informed that the client has called \refapi{PMIx_Finalize}.  The client might not have exited.  If a client 
exits without calling \reefapi{PMIx_Finalize}, the server support library will not call the \refapi{PMIx_server_client_finalized_fn_t} implementation.

\adviceimplstart
 The \refapi{PMIx_server_client_finalized_fn_t} implementation provided in the \refapi{pmix_server_module_2_0_0_t} is an opportunity for a host server
 to update the status of the tasks it manages.  It is also a convenient and well defined time to release resources used to support that client.   
 \adviceimplend


%%%%%%%%%%%
\subsection{\code{pmix_server_abort_fn_t}}
\declareapi{pmix_server_abort_fn_t}

%%%%
\summary

Notify PMIx Server that a local client called \refapi{PMIx_Abort}.

%%%%
\format

\cspecificstart
\begin{codepar}
typedef pmix_status_t (*pmix_server_abort_fn_t)(
                             const pmix_proc_t *proc, void *server_object,
                             int status, const char msg[],
                             pmix_proc_t procs[], size_t nprocs,
                             pmix_op_cbfunc_t cbfunc, void *cbdata)
\end{codepar}
\cspecificend


\begin{arglist}
\argin{proc}{\refstruct{pmix_proc_t} structure (handle)}
\argin{server_object}{object reference (memory reference)}
\argin{status}{exit status (integer)}
\argin{msg}{exit status message (string)}
\argin{procs}{Array of \refstruct{pmix_proc_t} structures (array of handles)}
\argin{nprocs}{Number of elements in the \refarg{procs} array (integer)}
\argin{cbfunc}{Callback function \refapi{pmix_op_cbfunc_t} (function reference)}
\argin{cbdata}{Data to be passed to the callback function (memory reference)}
\end{arglist}

Returns \refconst{PMIX_SUCCESS} or a negative value corresponding to a PMIx error constant.

%%%%
\descr

A local client called \refapi{PMIx_Abort}.
Note that the client will be in a blocked state until the host server executes the callback function, thus allowing the PMIx server support library to release the client.
The array of \refarg{procs} indicates which processes are to be terminated.
A \code{NULL} indicates that all processes in the client's namespace are to be terminated.


%%%%%%%%%%%
\subsection{\code{pmix_server_fencenb_fn_t}}
\declareapi{pmix_server_fencenb_fn_t}

%%%%
\summary

At least one client called either \refapi{PMIx_Fence} or \refapi{PMIx_Fence_nb}.

%%%%
\format

\cspecificstart
\begin{codepar}
typedef pmix_status_t (*pmix_server_fencenb_fn_t)(
                             const pmix_proc_t procs[], size_t nprocs,
                             const pmix_info_t info[], size_t ninfo,
                             char *data, size_t ndata,
                             pmix_modex_cbfunc_t cbfunc, void *cbdata)
\end{codepar}
\cspecificend

\begin{arglist}
\argin{procs}{Array of \refstruct{pmix_proc_t} structures (array of handles)}
\argin{nprocs}{Number of elements in the \refarg{procs} array (integer)}
\argin{info}{Array of info structures (array of handles)}
\argin{ninfo}{Number of elements in the \refarg{info} array (integer)}
\argin{data}{(string)}
\argin{ndata}{(integer)}
\argin{cbfunc}{Callback function \refapi{pmix_modex_cbfunc_t} (function reference)}
\argin{cbdata}{Data to be passed to the callback function (memory reference)}
\end{arglist}

Returns \refconst{PMIX_SUCCESS} or a negative value corresponding to a PMIx error constant.

%%%%
\descr

At least one client called either \refapi{PMIx_Fence} or \refapi{PMIx_Fence_nb}.
In either case, the host server will be called via a non-blocking function to execute the specified operation once all participating local processes have contributed.
All processes in the specified \refarg{procs} array are required to participate in the \refapi{PMIx_Fence}/\refapi{PMIx_Fence_nb} operation.
The callback is to be executed once each daemon hosting at least one participant has called the host server's \refapi{pmix_server_fencenb_fn_t} function.

The provided data is to be collectively shared with all PMIx servers involved in the fence operation, and returned in the modex \refarg{cbfunc}.
A \code{NULL} data value indicates that the local processes had no data to contribute.

The array of \refarg{info} structs is used to pass user-requested options to the server.
This can include directives as to the algorithm to be used to execute the fence operation.
The directives are optional \emph{unless} the \emph{mandatory} flag has been set - in such cases, the host \ac{RM} is required to return an error if the directive cannot be met.


%%%%%%%%%%%
\subsection{\code{pmix_server_dmodex_req_fn_t}}
\declareapi{pmix_server_dmodex_req_fn_t}

%%%%
\summary

Used by the PMIx server to request its local host contact the PMIx server on the remote node that hosts the specified proc to obtain and return a direct modex blob for that proc.

%%%%
\format

\cspecificstart
\begin{codepar}
typedef pmix_status_t (*pmix_server_dmodex_req_fn_t)(
                             const pmix_proc_t *proc,
                             const pmix_info_t info[], size_t ninfo,
                             pmix_modex_cbfunc_t cbfunc, void *cbdata)
\end{codepar}
\cspecificend

\begin{arglist}
\argin{proc}{\refstruct{pmix_proc_t} structure (handle)}
\argin{info}{Array of info structures (array of handles)}
\argin{ninfo}{Number of elements in the \refarg{info} array (integer)}
\argin{cbfunc}{Callback function \refapi{pmix_modex_cbfunc_t} (function reference)}
\argin{cbdata}{Data to be passed to the callback function (memory reference)}
\end{arglist}

Returns \refconst{PMIX_SUCCESS} or a negative value corresponding to a PMIx error constant.

%%%%
\descr

Used by the PMIx server to request its local host contact the PMIx server on the remote node that hosts the specified proc to obtain and return a direct modex blob for that proc.

The array of \refarg{info} structs is used to pass user-requested options to the server.
This can include a timeout to preclude an indefinite wait for data that may never become available.
The directives are optional \emph{unless} the \emph{mandatory} flag has been set - in such cases, the host \ac{RM} is required to return an error if the directive cannot be met.


%%%%%%%%%%%
\subsection{\code{pmix_server_publish_fn_t}}
\declareapi{pmix_server_publish_fn_t}

%%%%
\summary

Publish data per the PMIx API specification.

%%%%
\format

\cspecificstart
\begin{codepar}
typedef pmix_status_t (*pmix_server_publish_fn_t)(
                             const pmix_proc_t *proc,
                             const pmix_info_t info[], size_t ninfo,
                             pmix_op_cbfunc_t cbfunc, void *cbdata)
\end{codepar}
\cspecificend

\begin{arglist}
\argin{proc}{\refstruct{pmix_proc_t} structure (handle)}
\argin{info}{Array of info structures (array of handles)}
\argin{ninfo}{Number of elements in the \refarg{info} array (integer)}
\argin{cbfunc}{Callback function \refapi{pmix_op_cbfunc_t} (function reference)}
\argin{cbdata}{Data to be passed to the callback function (memory reference)}
\end{arglist}

Returns \refconst{PMIX_SUCCESS} or a negative value corresponding to a PMIx error constant.

%%%%
\descr

Publish data per the PMIx API specification.
The callback is to be executed upon completion of the operation.
The default data range is expected to be \refconst{PMIX_SESSION}, and the default persistence \refconst{PMIX_PERSIST_SESSION}.
These values can be modified by including the respective \refstruct{pmix_info_t} struct in the \refarg{info} array.

Note that the host server is not required to guarantee support for any specific range - i.e., the server does not need to return an error if the data store doesn't support range-based isolation.
However, the server must return an error (a) if the key is duplicative within the storage range, and (b) if the server does not allow overwriting of published info by the original publisher - it is left to the discretion of the host server to allow info-key-based flags to modify this behavior.

The persistence indicates how long the server should retain the data.

The identifier of the publishing process is also provided and is expected to be returned on any subsequent lookup request.


%%%%%%%%%%%
\subsection{\code{pmix_server_lookup_fn_t}}
\declareapi{pmix_server_lookup_fn_t}

%%%%
\summary

Lookup published data.

%%%%
\format

\cspecificstart
\begin{codepar}
typedef pmix_status_t (*pmix_server_lookup_fn_t)(
                             const pmix_proc_t *proc, char **keys,
                             const pmix_info_t info[], size_t ninfo,
                             pmix_lookup_cbfunc_t cbfunc, void *cbdata)
\end{codepar}
\cspecificend

\begin{arglist}
\argin{proc}{\refstruct{pmix_proc_t} structure (handle)}
\argin{keys}{(array of strings)}
\argin{info}{Array of info structures (array of handles)}
\argin{ninfo}{Number of elements in the \refarg{info} array (integer)}
\argin{cbfunc}{Callback function \refapi{pmix_lookup_cbfunc_t} (function reference)}
\argin{cbdata}{Data to be passed to the callback function (memory reference)}
\end{arglist}

Returns \refconst{PMIX_SUCCESS} or a negative value corresponding to a PMIx error constant.

%%%%
\descr

Lookup published data.
The host server will be passed a NULL-terminated array of string keys.

The array of \refarg{info} structs is used to pass user-requested options to the server.
This can include a wait flag to indicate that the server should wait for all data to become available before executing the callback function, or should immediately callback with whatever data is available.
In addition, a timeout can be specified on the wait to preclude an indefinite wait for data that may never be published.


%%%%%%%%%%%
\subsection{\code{pmix_server_unpublish_fn_t}}
\declareapi{pmix_server_unpublish_fn_t}

%%%%
\summary

Delete data from the data store.

%%%%
\format

\cspecificstart
\begin{codepar}
typedef pmix_status_t (*pmix_server_unpublish_fn_t)(
                             const pmix_proc_t *proc, char **keys,
                             const pmix_info_t info[], size_t ninfo,
                             pmix_op_cbfunc_t cbfunc, void *cbdata)
\end{codepar}
\cspecificend

\begin{arglist}
\argin{proc}{\refstruct{pmix_proc_t} structure (handle)}
\argin{keys}{(array of strings)}
\argin{info}{Array of info structures (array of handles)}
\argin{ninfo}{Number of elements in the \refarg{info} array (integer)}
\argin{cbfunc}{Callback function \refapi{pmix_op_cbfunc_t} (function reference)}
\argin{cbdata}{Data to be passed to the callback function (memory reference)}
\end{arglist}

Returns \refconst{PMIX_SUCCESS} or a negative value corresponding to a PMIx error constant.

%%%%
\descr

Delete data from the data store.
The host server will be passed a NULL-terminated array of string keys, plus potential directives such as the data range within which the keys should be deleted.
The callback is to be executed upon completion of the delete procedure.


%%%%%%%%%%%
\subsection{\code{pmix_server_spawn_fn_t}}
\declareapi{pmix_server_spawn_fn_t}

%%%%
\summary

Spawn a set of applications/processes as per the PMIx API.

%%%%
\format

\cspecificstart
\begin{codepar}
typedef pmix_status_t (*pmix_server_spawn_fn_t)(
                             const pmix_proc_t *proc,
                             const pmix_info_t job_info[], size_t ninfo,
                             const pmix_app_t apps[], size_t napps,
                             pmix_spawn_cbfunc_t cbfunc, void *cbdata)
\end{codepar}
\cspecificend

\begin{arglist}
\argin{proc}{\refstruct{pmix_proc_t} structure (handle)}
\argin{job_info}{Array of info structures (array of handles)}
\argin{ninfo}{Number of elements in the \refarg{jobinfo} array (integer)}
\argin{apps}{Array of \refstruct{pmix_app_t} structures (array of handles)}
\argin{napps}{Number of elements in the \refarg{apps} array (integer)}
\argin{cbfunc}{Callback function \refapi{pmix_spawn_cbfunc_t} (function reference)}
\argin{cbdata}{Data to be passed to the callback function (memory reference)}
\end{arglist}

Returns \refconst{PMIX_SUCCESS} or a negative value corresponding to a PMIx error constant.

%%%%
\descr

Spawn a set of applications/processes as per the PMIx API.
Note that applications are not required to be MPI or any other programming model.
Thus, the host server cannot make any assumptions as to their required support.
The callback function is to be executed once all processes have been started.
An error in starting any application or process in this request shall cause all applications and processes in the request to be terminated, and an error returned to the originating caller.

Note that a timeout can be specified in the job_info array to indicate that failure to start the requested job within the given time should result in termination to avoid hangs.


%%%%%%%%%%%
\subsection{\code{pmix_server_connect_fn_t}}
\declareapi{pmix_server_connect_fn_t}

%%%%
\summary

Record the specified processes as ``connected''.

%%%%
\format

\cspecificstart
\begin{codepar}
typedef pmix_status_t (*pmix_server_connect_fn_t)(
                             const pmix_proc_t procs[], size_t nprocs,
                             const pmix_info_t info[], size_t ninfo,
                             pmix_op_cbfunc_t cbfunc, void *cbdata)
\end{codepar}
\cspecificend

\begin{arglist}
\argin{procs}{Array of \refstruct{pmix_proc_t} structures (array of handles)}
\argin{nprocs}{Number of elements in the \refarg{procs} array (integer)}
\argin{info}{Array of info structures (array of handles)}
\argin{ninfo}{Number of elements in the \refarg{info} array (integer)}
\argin{cbfunc}{Callback function \refapi{pmix_op_cbfunc_t} (function reference)}
\argin{cbdata}{Data to be passed to the callback function (memory reference)}
\end{arglist}

Returns \refconst{PMIX_SUCCESS} or a negative value corresponding to a PMIx error constant.

%%%%
\descr

Record the specified processes as ``connected''.
This means that the resource manager should treat the failure of any process in the specified group as a reportable event, and take appropriate action.
The callback function is to be called once all participating processes have called connect.
Note that a process can only engage in \textbf{one} connect operation involving the identical set of processes at a time.
However, a process \emph{can} be simultaneously engaged in multiple connect operations, each involving a different set of processes.

Note also that this is a collective operation within the client library, and thus the client will be blocked until all processes participate.
Thus, the \refarg{info} array can be used to pass user directives, including a timeout.
The directives are optional \emph{unless} the \emph{mandatory} flag has been set - in such cases, the host RM is required to return an error if the directive cannot be met.


%%%%%%%%%%%
\subsection{\code{pmix_server_disconnect_fn_t}}
\declareapi{pmix_server_disconnect_fn_t}

%%%%
\summary

Disconnect a previously connected set of processes.

%%%%
\format

\cspecificstart
\begin{codepar}
typedef pmix_status_t (*pmix_server_disconnect_fn_t)(
                             const pmix_proc_t procs[], size_t nprocs,
                             const pmix_info_t info[], size_t ninfo,
                             pmix_op_cbfunc_t cbfunc, void *cbdata)
\end{codepar}
\cspecificend

\begin{arglist}
\argin{procs}{Array of \refstruct{pmix_proc_t} structures (array of handles)}
\argin{nprocs}{Number of elements in the \refarg{procs} array (integer)}
\argin{info}{Array of info structures (array of handles)}
\argin{ninfo}{Number of elements in the \refarg{info} array (integer)}
\argin{cbfunc}{Callback function \refapi{pmix_op_cbfunc_t} (function reference)}
\argin{cbdata}{Data to be passed to the callback function (memory reference)}
\end{arglist}

Returns \refconst{PMIX_SUCCESS} or a negative value corresponding to a PMIx error constant.

%%%%
\descr

Disconnect a previously connected set of processes.
An error should be returned if the specified set of processes was not previously ``connected''.
As above, a process may be involved in multiple simultaneous disconnect operations.
However, a process is not allowed to reconnect to a set of ranges that has not fully completed disconnect (i.e., you have to fully disconnect before you can reconnect to the same group of processes).

Note also that this is a collective operation within the client library, and thus the client will be blocked until all processes participate.
Thus, the \refarg{info} array can be used to pass user directives, including a timeout.
The directives are optional \emph{unless} the \emph{mandatory} flag has been set - in such cases, the host RM is required to return an error if the directive cannot be met.


%%%%%%%%%%%
\subsection{\code{pmix_server_register_events_fn_t}}
\declareapi{pmix_server_register_events_fn_t}

%%%%
\summary

Register to receive notifications for the specified events.

%%%%
\format

\cspecificstart
\begin{codepar}
 typedef pmix_status_t (*pmix_server_register_events_fn_t)(
                              pmix_status_t *codes, size_t ncodes,
                              const pmix_info_t info[], size_t ninfo,
                              pmix_op_cbfunc_t cbfunc, void *cbdata)
\end{codepar}
\cspecificend

\begin{arglist}
\argin{codes}{Array of \refstruct{pmix_status_t} structures (array of handles)}
\argin{ncodes}{Number of elements in the \refarg{codes} array (integer)}
\argin{info}{Array of info structures (array of handles)}
\argin{ninfo}{Number of elements in the \refarg{info} array (integer)}
\argin{cbfunc}{Callback function \refapi{pmix_op_cbfunc_t} (function reference)}
\argin{cbdata}{Data to be passed to the callback function (memory reference)}
\end{arglist}

Returns \refconst{PMIX_SUCCESS} or a negative value corresponding to a PMIx error constant.

%%%%
\descr

Register to receive notifications for the specified events.
The resource manager is \emph{required} to pass along to the local PMIx server all events that directly relate to a registered namespace.
However, the RM may have access to events beyond those (e.g., environmental events).
The PMIx server will register to receive environmental events that match specific PMIx event codes.
If the host RM supports such notifications, it will need to translate its own internal event codes to fit into a corresponding PMIx event code - any specific info beyond that can be passed in via the \refstruct{pmix_info_t} upon notification.

The \refarg{info} array included in this API is reserved for possible future directives to further steer notification.



%%%%%%%%%%%
\subsection{\code{pmix_server_deregister_events_fn_t}}
\declareapi{pmix_server_deregister_events_fn_t}

%%%%
\summary

Deregister to receive notifications for the specified events.

%%%%
\format

\cspecificstart
\begin{codepar}
 typedef pmix_status_t (*pmix_server_deregister_events_fn_t)(
                              pmix_status_t *codes, size_t ncodes,
                              pmix_op_cbfunc_t cbfunc, void *cbdata)
\end{codepar}
\cspecificend

\begin{arglist}
\argin{codes}{Array of \refstruct{pmix_status_t} structures (array of handles)}
\argin{ncodes}{Number of elements in the \refarg{codes} array (integer)}
\argin{cbfunc}{Callback function \refapi{pmix_op_cbfunc_t} (function reference)}
\argin{cbdata}{Data to be passed to the callback function (memory reference)}
\end{arglist}

Returns \refconst{PMIX_SUCCESS} or a negative value corresponding to a PMIx error constant.

%%%%
\descr

Deregister to receive notifications for the specified environmental events for which the PMIx server has previously registered.
The host RM remains required to notify of any job-related events.


%%%%%%%%%%%
\subsection{\code{pmix_server_notify_event_fn_t}}
\declareapi{pmix_server_notify_event_fn_t}

%%%%
\summary

Notify the specified processes of an event.

%%%%
\format

\cspecificstart
\begin{codepar}
typedef pmix_status_t (*pmix_server_notify_event_fn_t)(pmix_status_t code,
                                                       const pmix_proc_t *source,
                                                       pmix_data_range_t range,
                                                       pmix_info_t info[], size_t ninfo,
                                                       pmix_op_cbfunc_t cbfunc, void *cbdata);
\end{codepar}
\cspecificend

\begin{arglist}
\argin{code}{\refstruct{pmix_status_t} structure (handle)}
\argin{source}{\refstruct{pmix_proc_t} (handle)}
\argin{range}{\refstruct{pmix_data_range_t} (handle)}
\argin{info}{Array of info structures (array of handles)}
\argin{ninfo}{Number of elements in the \refarg{info} array (integer)}
\argin{cbfunc}{Callback function \refapi{pmix_op_cbfunc_t} (function reference)}
\argin{cbdata}{Data to be passed to the callback function (memory reference)}
\end{arglist}

Returns \refconst{PMIX_SUCCESS} or a negative value corresponding to a PMIx error constant.

%%%%
\descr

Notify the specified processes of an event generated either by the PMIx server itself, or by one of its local clients.
The process generating the event is provided in the source parameter.


%%%%%%%%%%%
\subsection{\code{pmix_connection_cbfunc_t}}
\declareapi{pmix_connection_cbfunc_t}

%%%%
\summary

Callback function for incoming connection requests from local clients.

%%%%
\format

\cspecificstart
\begin{codepar}
typedef void (*pmix_connection_cbfunc_t)(
                    int incoming_sd, void *cbdata)
\end{codepar}
\cspecificend

\begin{arglist}
\argin{incoming_sd}{(integer)}
\argin{cbdata}{ (memory reference)}
\end{arglist}

Returns \refconst{PMIX_SUCCESS} or a negative value corresponding to a PMIx error constant.

%%%%
\descr

Callback function for incoming connection requests from local clients.


%%%%%%%%%%%
\subsection{\code{pmix_server_listener_fn_t}}
\declareapi{pmix_server_listener_fn_t}

%%%%
\summary

Register a socket the host server can monitor for connection requests.

%%%%
\format

\cspecificstart
\begin{codepar}
typedef pmix_status_t (*pmix_server_listener_fn_t)(
                             int listening_sd,
                             pmix_connection_cbfunc_t cbfunc,
                             void *cbdata)
\end{codepar}
\cspecificend

\begin{arglist}
\argin{incoming_sd}{(integer)}
\argin{cbfunc}{Callback function \refapi{pmix_connection_cbfunc_t} (function reference)}
\argin{cbdata}{ (memory reference)}
\end{arglist}

Returns \refconst{PMIX_SUCCESS} or a negative value corresponding to a PMIx error constant.

%%%%
\descr

Register a socket the host server can monitor for connection requests, harvest them, and then call our internal callback function for further processing.
A listener thread is essential to efficiently harvesting connection requests from large numbers of local clients such as occur when running on large SMPs.
The host server listener is required to call accept on the incoming connection request, and then passing the resulting soct to the provided cbfunc.
A NULL for this function will cause the internal PMIx server to spawn its own listener thread.


%%%%%%%%%%%
\subsection{\code{pmix_server_query_fn_t}}
\declareapi{pmix_server_query_fn_t}

%%%%
\summary

Query information from the resource manager.

%%%%
\format

\cspecificstart
\begin{codepar}
typedef pmix_status_t (*pmix_server_query_fn_t)(
                             pmix_proc_t *proct,
                             pmix_query_t *queries, size_t nqueries,
                             pmix_info_cbfunc_t cbfunc,
                             void *cbdata)
\end{codepar}
\cspecificend

\begin{arglist}
\argin{proct}{\refstruct{pmix_proc_t} structure (handle)}
\argin{queries}{Array of \refstruct{pmix_query_t} structures (array of handles)}
\argin{nqueries}{Number of elements in the \refarg{queries} array (integer)}
\argin{cbfunc}{Callback function \refapi{pmix_info_cbfunc_t} (function reference)}
\argin{cbdata}{Data to be passed to the callback function (memory reference)}
\end{arglist}

Returns \refconst{PMIX_SUCCESS} or a negative value corresponding to a PMIx error constant.

%%%%
\descr

Query information from the resource manager.
The query will include the nspace/rank of the process that is requesting the info, an array of \refstruct{pmix_query_t} describing the request, and a callback function/data for the return.


%%%%%%%%%%%
\subsection{\code{pmix_tool_connection_cbfunc_t}}
\declareapi{pmix_tool_connection_cbfunc_t}

%%%%
\summary

Callback function for incoming tool connections.

%%%%
\format

\cspecificstart
\begin{codepar}
typedef void (*pmix_tool_connection_cbfunc_t)(
                    pmix_status_t status,
                    pmix_proc_t *proc, void *cbdata)
\end{codepar}
\cspecificend

\begin{arglist}
\argin{status}{\refstruct{pmix_status_t} structure (handle)}
\argin{proc}{\refstruct{pmix_proc_t} structure (handle)}
\argin{cbdata}{Data to be passed (memory reference)}
\end{arglist}

%%%%
\descr

Callback function for incoming tool connections.
The host RM shall provide an nspace/rank for the connecting tool.
We assume that a \code{rank=0} will be the normal assignment, but allow for the future possibility of a parallel set of tools connecting, and thus each proc requiring a rank.


%%%%%%%%%%%
\subsection{\code{pmix_server_tool_connection_fn_t}}
\declareapi{pmix_server_tool_connection_fn_t}

%%%%
\summary

Register that a tool has connected to the server.

%%%%
\format

\cspecificstart
\begin{codepar}
typedef void (*pmix_server_tool_connection_fn_t)(
                    pmix_info_t *info, size_t ninfo,
                    pmix_tool_connection_cbfunc_t cbfunc,
                    void *cbdata)
\end{codepar}
\cspecificend

\begin{arglist}
\argin{info}{Array of info structures (array of handles)}
\argin{ninfo}{Number of elements in the \refarg{info} array (integer)}
\argin{cbfunc}{Callback function \refapi{pmix_tool_connection_cbfunc_t} (function reference)}
\argin{cbdata}{Data to be passed to the callback function (memory reference)}
\end{arglist}


%%%%
\descr

Register that a tool has connected to the server, and request that the tool be assigned an nspace/rank for further interactions.
The optional \refstruct{pmix_info_t} array can be used to pass qualifiers for the connection request:

\begin{constantdesc}
%
\declareconstitem{PMIX_USERID} effective userid of the tool
%
\declareconstitem{PMIX_GRPID} effective groupid of the tool
%
\declareconstitem{PMIX_FWD_STDOUT} forward any stdout to this tool
%
\declareconstitem{PMIX_FWD_STDERR} forward any stderr to this tool
%
\declareconstitem{PMIX_FWD_STDIN} forward stdin from this tool to any processes spawned on its behalf
%
\end{constantdesc}


%%%%%%%%%%%
\subsection{\code{pmix_server_log_fn_t}}
\declareapi{pmix_server_log_fn_t}

%%%%
\summary

Log data on behalf of a client.

%%%%
\format

\cspecificstart
\begin{codepar}
typedef void (*pmix_server_log_fn_t)(
                    const pmix_proc_t *client,
                    const pmix_info_t data[], size_t ndata,
                    const pmix_info_t directives[], size_t ndirs,
                    pmix_op_cbfunc_t cbfunc, void *cbdata)
\end{codepar}
\cspecificend

\begin{arglist}
\argin{client}{\refstruct{pmix_proc_t} structure (handle)}
\argin{data}{Array of info structures (array of handles)}
\argin{ndata}{Number of elements in the \refarg{data} array (integer)}
\argin{directives}{Array of info structures (array of handles)}
\argin{ndirs}{Number of elements in the \refarg{directives} array (integer)}
\argin{cbfunc}{Callback function \refapi{pmix_op_cbfunc_t} (function reference)}
\argin{cbdata}{Data to be passed to the callback function (memory reference)}
\end{arglist}


%%%%
\descr

Log data on behalf of a client.


%%%%%%%%%%%
\subsection{\code{pmix_server_alloc_fn_t}}
\declareapi{pmix_server_alloc_fn_t}

%%%%
\summary

Request allocation modifications on behalf of a client.

%%%%
\format

\cspecificstart
\begin{codepar}
typedef pmix_status_t (*pmix_server_alloc_fn_t)(
                             const pmix_proc_t *client,
                             pmix_alloc_directive_t directive,
                             const pmix_info_t data[], size_t ndata,
                             pmix_info_cbfunc_t cbfunc, void *cbdata)
\end{codepar}
\cspecificend

\begin{arglist}
\argin{client}{\refstruct{pmix_proc_t} structure (handle)}
\argin{directive}{(handle)}
\argin{data}{Array of info structures (array of handles)}
\argin{ndata}{Number of elements in the \refarg{data} array (integer)}
\argin{cbfunc}{Callback function \refapi{pmix_info_cbfunc_t} (function reference)}
\argin{cbdata}{Data to be passed to the callback function (memory reference)}
\end{arglist}

Returns \refconst{PMIX_SUCCESS} or a negative value corresponding to a PMIx error constant.

%%%%
\descr

Request allocation modifications on behalf of a client.


%%%%%%%%%%%
\subsection{\code{pmix_server_job_control_fn_t}}
\declareapi{pmix_server_job_control_fn_t}

%%%%
\summary

Execute a job control action on behalf of a client.

%%%%
\format

\cspecificstart
\begin{codepar}
typedef pmix_status_t (*pmix_server_job_control_fn_t)(
                             const pmix_proc_t *requestor,
                             const pmix_proc_t targets[], size_t ntargets,
                             const pmix_info_t directives[], size_t ndirs,
                             pmix_info_cbfunc_t cbfunc, void *cbdata)
\end{codepar}
\cspecificend

\begin{arglist}
\argin{requestor}{\refstruct{pmix_proc_t} structure (handle)}
\argin{targets}{Array of proc structures (array of handles)}
\argin{ntargets}{Number of elements in the \refarg{targets} array (integer)}
\argin{directives}{Array of info structures (array of handles)}
\argin{ndirs}{Number of elements in the \refarg{info} array (integer)}
\argin{cbfunc}{Callback function \refapi{pmix_op_cbfunc_t} (function reference)}
\argin{cbdata}{Data to be passed to the callback function (memory reference)}
\end{arglist}

Returns \refconst{PMIX_SUCCESS} or a negative value corresponding to a PMIx error constant.

%%%%
\descr

Execute a job control action on behalf of a client.


%%%%%%%%%%%
\subsection{\code{pmix_server_monitor_fn_t}}
\declareapi{pmix_server_monitor_fn_t}

%%%%
\summary

Request that a client be monitored for activity.

%%%%
\format

\cspecificstart
\begin{codepar}
/* Request that a client be monitored for activity */
typedef pmix_status_t (*pmix_server_monitor_fn_t)(
                             const pmix_proc_t *requestor,
                             const pmix_info_t *monitor, pmix_status_t error,
                             const pmix_info_t directives[], size_t ndirs,
                             pmix_info_cbfunc_t cbfunc, void *cbdata);
\end{codepar}
\cspecificend

\begin{arglist}
\argin{requestor}{\refstruct{pmix_proc_t} structure (handle)}
\argin{monitor}{\refstruct{pmix_proc_t} structure (handle)}
\argin{error}{(integer)}
\argin{directives}{Array of info structures (array of handles)}
\argin{ndirs}{Number of elements in the \refarg{info} array (integer)}
\argin{cbfunc}{Callback function \refapi{pmix_op_cbfunc_t} (function reference)}
\argin{cbdata}{Data to be passed to the callback function (memory reference)}
\end{arglist}

Returns \refconst{PMIX_SUCCESS} or a negative value corresponding to a PMIx error constant.

%%%%
\descr

Request that a client be monitored for activity.

%%%%%%%%%%%%%%%%%%%%%%%%%%%%%%%%%%%%%%%%%%%%%%%%%


    % PMIx Tools and Debugger Support
    %%%%%%%%%%%%%%%%%%%%%%%%%%%%%%%%%%%%%%%%%%%%%%%%%
% Chapter: Tools
%%%%%%%%%%%%%%%%%%%%%%%%%%%%%%%%%%%%%%%%%%%%%%%%%
\chapter{Tools and Debuggers}
\label{chap:api_tools}

The term \textit{tool} widely refers to programs executed by the user or system administrator on a command line. Tools frequently interact with either the \ac{SMS}, user applications, or both to perform administrative and support functions. For example, a debugger tool might be used to remotely control the processes of a parallel application, monitoring their behavior on a step-by-step basis. Historically, such tools were custom-written for each specific host environment due to the customized and/or proprietary nature of the environment's interfaces.

The advent of \ac{PMIx} offers the possibility for creating portable tools capable of interacting with multiple \acp{RM} without modification. Possible use-cases include:

\begin{itemize}
\item querying the status of scheduling queues and estimated allocation time for various resource options
\item job submission and allocation requests
\item querying job status for executing applications
\item launching, monitoring, and debugging applications
\end{itemize}

Enabling these capabilities requires some extensions to the \ac{PMIx} Standard (both in terms of \acp{API} and attributes), and utilization of client-side \acp{API} for more tool-oriented purposes.

This chapter defines specific \acp{API} related to tools, provides tool developers with an overview of the support provided by \ac{PMIx}, and serves to guide \ac{RM} vendors regarding roles and responsibilities of \acp{RM} to support tools. As the number of tool-specific \acp{API} and attributes is fairly small, the bulk of the chapter serves to provide a "theory of operation" for tools and debuggers. Description of the \acp{API} themselves is therefore deferred to the Section \ref{chap:api_tools:apis} later in the chapter.

%%%%%%%%%%%%%%%%%%%%%%%%%%%%%%%%%%%%%%%%%%%%%%%%%
%%%%%%%%%%%%%%%%%%%%%%%%%%%%%%%%%%%%%%%%%%%%%%%%%
\section{Connection Mechanisms}
\label{chap:api_tools:cnct}

The key to supporting tools lies in providing mechanisms by which a tool can connect to a \ac{PMIx} server. Application processes are able to connect because their local \ac{RM} daemon provides them with the necessary contact information upon execution. A command-line tool, however, isn't spawned by an \ac{RM} daemon, and therefore lacks the information required for rendezvous with a \ac{PMIx} server.

Once a tool has started, it initializes \ac{PMIx} as a tool (via \refapi{PMIx_tool_init}) if its access is restricted to \ac{PMIx}-based informational services such as \refapi{PMIx_Query_info}. However, if the tool intends to start jobs, then it must include the \refattr{PMIX_LAUNCHER} attribute to inform the library of that intent so that the library can initialize and provide access to the corresponding support.

Support for tools requires that the \ac{PMIx} server be initialized with an appropriate attribute indicating that tool connections are to be allowed. Separate attributes are provided to "fine-tune" this permission by allowing the environment to independently enable (or disable) connections from tools executing on nodes other than the one hosting the server itself. The \ac{PMIx} server library shall provide an opportunity for the host environment to authenticate and approve each connection request from a specific tool by calling the \refapi{pmix_server_tool_connection_fn_t} "hook" provided in the server module for that purpose. Servers in environments that do not provide this "hook" shall automatically reject all tool connection requests.

Tools can connect to any local or remote \ac{PMIx} server provided they are either explicitly given the required connection information, or are able to discover it via one of several defined rendezvous protocols. Connection discovery centers around the existence of \emph{rendezvous files} containing the necessary connection information, as illustrated in Fig. \ref{fig:rndvz}.

\begingroup
\begin{figure*}[ht!]
  \begin{center}
    \includegraphics[clip,width=0.9\textwidth]{figs/rndvz.pdf}
  \end{center}
  \caption{Tool rendezvous files}
  \label{fig:rndvz}
\end{figure*}
\endgroup

The contents of each rendezvous file are specific to a given \ac{PMIx} implementation, but should at least contain the namespace and rank of the server along with its connection \ac{URI}. Note that tools linked to one \ac{PMIx} implementation are therefore unlikely to successfully connect to \ac{PMIx} server libraries from another implementation.

The top of the directory tree is defined by either the \refattr{PMIX_SYSTEM_TMPDIR} attribute (if given) or the \code{TMPDIR} environmental variable. \ac{PMIx} servers that are designated as \emph{system servers} by including the \refattr{PMIX_SERVER_SYSTEM_SUPPORT} attribute when calling \refapi{PMIx_server_init} will create a rendezvous file in this top-level directory. The filename will be of the form \emph{pmix.sys.hostname}, where \emph{hostname} is the string returned by the \code{gethostname} system call. Note that only one \ac{PMIx} server on a node can be designated as the system server.

Non-system \ac{PMIx} servers will create a set of three rendezvous files in the directory defined by either the \refattr{PMIX_SERVER_TMPDIR} attribute or the \code{TMPDIR} environmental variable:

\begin{itemize}
    \item \emph{pmix.host.tool.nspace} where \emph{host} is the string returned by the \code{gethostname} system call and \emph{nspace} is the namespace of the server.
    \item \emph{pmix.host.tool.pid} where \emph{host} is the string returned by the \code{gethostname} system call and \emph{pid} is the \ac{PID} of the server.
    \item \emph{pmix.host.tool}  where \emph{host} is the string returned by the \code{gethostname} system call. Note that servers which are not given a namespace-specific \refattr{PMIX_SERVER_TMPDIR} attribute may not generate this file due to conflicts should multiple servers be present on the node.
\end{itemize}

The files are identical and may be implemented as symlinks to a single instance. The individual file names are composed so as to aid the search process should a tool wish to connect to a server identified by its namespace or \ac{PID}.

Servers will additionally provide a rendezvous file in any given location if the path (either absolute or relative) and filename is specified either during \refapi{PMIx_server_init} using the \refattr{PMIX_LAUNCHER_RENDEZVOUS_FILE} attribute, or by the \refenvar{PMIX_LAUNCHER_RNDZ_FILE} environmental variable prior to executing the process containing the server. This latter mechanism may be the preferred mechanism for tools such as debuggers that need to fork/exec a launcher (e.g., "mpiexec") and then rendezvous with it. This is described in more detail in Section \ref{chap:api_tools:indirect}.

Rendezvous file ownerships are set to the \ac{UID} and \ac{GID} of the server that created them, with permissions set according to the desires of the implementation and/or system administrator policy. All connection attempts are first governed by read access privileges to the target rendezvous file - thus, the combination of permissions, \ac{UID}, and \ac{GID} of the rendezvous files act as a first-level of security for tool access.

A tool may connect to as many servers at one time as the implementation supports, but is limited to designating only one such connection as its \emph{primary} server. This is done to avoid confusion when the tool calls an \ac{API} as to which server should service the request. The first server the tool connects to is automatically designated as the \emph{primary} server.

Tools are allowed to change their primary server at any time via the \refapi{PMIx_tool_set_server} \ac{API}, and to connect/disconnect from a server as many times as desired. Note that standing requests (e.g., event registrations) with the current primary server may be lost and/or may not be transferred when transitioning to another primary server - \ac{PMIx} implementors are not required to maintain or transfer state across tool-server connections.

Tool process identifiers are assigned by one of the following methods:

\begin{itemize}
    \item If \refattr{PMIX_TOOL_NSPACE} is given, then the namespace of the tool will be assigned that value.
    \begin{itemize}
        \item If \refattr{PMIX_TOOL_RANK} is also given, then the rank of the tool will be assigned that value.
        \item If \refattr{PMIX_TOOL_RANK} is not given, then the rank will be set to a default value of zero.
    \end{itemize}
    \item If a process ID is not provided and the tool connects to a server, then one will be assigned by the host environment upon connection to that server.
    \item If a process ID is not provided and the tool does not connect to a server (e.g., if \refattr{PMIX_TOOL_DO_NOT_CONNECT} is given), then the tool shall self-assign a unique identifier. This is often done using some combination involving hostname and \ac{PID}.
\end{itemize}

Tool process identifiers remain constant across servers. Thus, it is critical that a system-wide unique namespace be provided if the tool itself sets the identifier, and that host environments provide a system-wide unique identifier in the case where the identifier is set by the server upon connection. The host environment is required to reject any connection request that fails to meet this criterion.

For simplicity, the following descriptions will refer to the:

\begin{itemize}
    \item \code{PMIX_SYSTEM_TMPDIR} as the directory specified by either the \refattr{PMIX_SYSTEM_TMPDIR} attribute (if given) or the \code{TMPDIR} environmental variable.
    \item \code{PMIX_SERVER_TMPDIR} as the directory specified by either the \refattr{PMIX_SERVER_TMPDIR} attribute or the \code{TMPDIR} environmental variable.
\end{itemize}

The rendezvous methods are automatically employed for the initial tool connection during \refapi{PMIx_tool_init} unless the \refattr{PMIX_TOOL_DO_NOT_CONNECT} attribute is specified, and on all subsequent calls to \refapi{PMIx_tool_attach_to_server}.

%%%%%%%%%%%%%%%%%%%%%%%%%%%%%%%%%%%%%%%%%%%%%%%%%
\subsection{Rendezvousing with a local server}

Connection to a local \ac{PMIx} server is pursued according to the following precedence chain based on attributes contained in the call to the \refapi{PMIx_tool_init} or \refapi{PMIx_tool_attach_to_server} \acp{API}. Servers to which the tool already holds a connection will be ignored. Except where noted, the \ac{PMIx} library will return an error if the specified file cannot be found, the caller lacks permissions to read it, or the server specified within the file does not respond to or accept the connection — the library will not proceed to check for other connection options as the user specified a particular one to use.

Note that the \ac{PMIx} implementation may choose to introduce a "delayed connection" protocol between steps in the precedence chain - i.e., the library may cycle several times, checking for creation of the rendezvous file each time after a delay of some period of time, thereby allowing the tool to wait for the server to create the rendezvous file before either returning an error or continuing to the next step in the chain.

\begin{itemize}
%
\item If \refattr{PMIX_TOOL_ATTACHMENT_FILE} is given, then the tool will attempt to read the specified file and connect to the server based on the information contained within it. The format of the attachment file is identical to the rendezvous files described in earlier in this section. An error will be returned if the specified file cannot be found.
%
\item If \refattr{PMIX_SERVER_URI} or \refattr{PMIX_TCP_URI} is given, then connection will be attempted to the server at the specified \ac{URI}. Note that it is an error for both of these attributes to be specified. \refattr{PMIX_SERVER_URI} is the preferred method as it is more generalized — \refattr{PMIX_TCP_URI} is provided for those cases where the user specifically wants to use a \ac{TCP} transport for the connection and wants to error out if one isn’t available or cannot be used.
%
\item If \refattr{PMIX_SERVER_PIDINFO} was provided, then the tool will search for a rendezvous file created by a \ac{PMIx} server of the given \ac{PID} in the \code{PMIX_SERVER_TMPDIR} directory. An error will be returned if a matching rendezvous file cannot be found.
%
\item If \refattr{PMIX_SERVER_NSPACE} is given, then the tool will search for a rendezvous file created by a \ac{PMIx} server of the given namespace in the \code{PMIX_SERVER_TMPDIR} directory. An error will be returned if a matching rendezvous file cannot be found.
%
\item If \refattr{PMIX_CONNECT_TO_SYSTEM} is given, then the tool will search for a system-level rendezvous file created by a \ac{PMIx} server in the \code{PMIX_SYSTEM_TMPDIR} directory. An error will be returned if a matching rendezvous file cannot be found.
%
\item If \refattr{PMIX_CONNECT_SYSTEM_FIRST} is given, then the tool will look for a system-level rendezvous file created by a \ac{PMIx} server in the \code{PMIX_SYSTEM_TMPDIR} directory. If found, then the tool will attempt to connect to it. In this case, no error will be returned if the rendezvous file is not found or connection is refused — the \ac{PMIx} library will silently continue to the next option.
%
\item By default, the tool will search the directory tree under the \code{PMIX_SERVER_TMPDIR} directory for rendezvous files of \ac{PMIx} servers, attempting to connect to each it finds until one accepts the connection. If no rendezvous files are found, or all contacted servers refuse connection, then the \ac{PMIx} library will return an error. No "delayed connection" protocols may be utilized at this point.
%
\end{itemize}

Note that there can be multiple local servers - one from the system plus others from launchers and active jobs. The \ac{PMIx} tool connection search method is not guaranteed to pick a particular server unless directed to do so. Tools can obtain a list of servers available on their local node using the \refapi{PMIx_Query_info} \acp{API} with the \refattr{PMIX_QUERY_AVAIL_SERVERS} key.

%%%%%%%%%%%%%%%%%%%%%%%%%%%%%%%%%%%%%%%%%%%%%%%%%
\subsection{Connecting to a remote server}

Connecting to remote servers is complicated due to the lack of access to the previously-described rendezvous files. Two methods are required to be supported, both based on the caller having explicit knowledge of either connection information or a path to a local file that contains such information:

\begin{itemize}
%
\item If \refattr{PMIX_TOOL_ATTACHMENT_FILE} is given, then the tool will attempt to read the specified file and connect to the server based on the information contained within it. The format of the attachment file is identical to the rendezvous files described in earlier in this section.
%
\item If \refattr{PMIX_SERVER_URI} or \refattr{PMIX_TCP_URI} is given, then connection will be attempted to the server at the specified \ac{URI}. Note that it is an error for both of these attributes to be specified. \refattr{PMIX_SERVER_URI} is the preferred method as it is more generalized — \refattr{PMIX_TCP_URI} is provided for those cases where the user specifically wants to use the \ac{TCP} transport for the connection and wants to error out if it isn’t available or cannot be used.
%
\end{itemize}

Additional methods may be provided by particular \ac{PMIx} implementations. For example, the tool may use \emph{ssh} to launch a \emph{probe} process onto the remote node so that the probe can search the \code{PMIX_SYSTEM_TMPDIR} and \code{PMIX_SERVER_TMPDIR} directories for rendezvous files, relaying the discovered information back to the requesting tool. If sufficient information is found to allow for remote connection, then the tool can use it to establish the connection. Note that this method is not required to be supported - it is provided here as an example and left to the discretion of \ac{PMIx} implementors.

%%%%%%%%%%%%%%%%%%%%%%%%%%%%%%%%%%%%%%%%%%%%%%%%%
\subsection{Attaching to running jobs}

When attaching to a running job, the tool must connect to a \ac{PMIx} server that is associated with that job - e.g., a server residing in the host environment's local daemon that spawned one or more of the job's processes, or the server residing in the launcher that is overseeing the job. Identifying an appropriate server can sometimes prove challenging, particularly in an environment where multiple job launchers may be in operation, possibly under control of the same user.

In cases where the user has only the one job of interest in operation on the local node (e.g., when engaged in an interactive session on the node from which the launcher was executed), the normal rendezvous file discovery method can often be used to successfully connect to the target job, even in the presence of jobs executed by other users. The permissions and security authorizations can, in many cases, reliably ensure that only the one connection can be made. However, this is not guaranteed in all cases.

The most common method, therefore, for attaching to a running job is to specify either the \ac{PID} of the job's launcher or the namespace of the launcher's job (note that the launcher's namespace frequently differs from the namespace of the job it has launched). Unless the application processes themselves act as \ac{PMIx} servers, connection must be to the servers in the daemons that oversee the application. This is typically either daemons specifically started by the job's launcher process, or daemons belonging to the host environment, that are responsible for starting the application's processes and oversee their execution.

Identifying the correct \ac{PID} or namespace can be accomplished in a variety of ways, including:

\begin{itemize}
    \item Using typical \ac{OS} or host environment tools to obtain a listing of active jobs and perusing those to find the target launcher.
    \item Using a \ac{PMIx}-based tool attached to a system-level server to query the active jobs and their command lines, thereby identifying the application of interest and its associated launcher.
    \item Manually recording the \ac{PID} of the launcher upon starting the job.
\end{itemize}

Once the namespace and/or \ac{PID} of the target server has been identified, either of the previous methods can be used to connect to it.


%%%%%%%%%%%%%%%%%%%%%%%%%%%%%%%%%%%%%%%%%%%%%%%%%
\subsection{Tool initialization attributes}
\label{api:tools:attributes:tool}

The following attributes are passed to the \refapi{PMIx_tool_init} \ac{API} for use when initializing the \ac{PMIx} library.

%
\declareAttribute{PMIX_TOOL_NSPACE}{"pmix.tool.nspace"}{char*}{
Name of the namespace to use for this tool.
}
%
\declareAttribute{PMIX_TOOL_RANK}{"pmix.tool.rank"}{uint32_t}{
Rank of this tool.
}
%
\declareAttribute{PMIX_LAUNCHER}{"pmix.tool.launcher"}{bool}{
Tool is a launcher and needs to create rendezvous files.
}

%%%%%%%%%%%%%%%%%%%%%%%%%%%%%%%%%%%%%%%%%%%%%%%%%
\subsection{Tool initialization environmental variables}
\label{api:tools:envars:tool}

The following environmental variables are used during \refapi{PMIx_tool_init} and \refapi{PMIx_server_init} to control various rendezvous-related operations when the process is started manually (e.g., on a command line) or by a fork/exec-like operation.

%
\declareEnvarNEW{PMIX_LAUNCHER_RNDZ_URI}{
The spawned tool is to be connected back to the spawning tool using the given \ac{URI} so that the spawning tool can provide directives (e.g., a \refapi{PMIx_Spawn} command) to it.
}
%
\declareEnvarNEW{PMIX_LAUNCHER_RNDZ_FILE}{
If the specified file does not exist, this variable contains the absolute path of the file where the spawned tool is to store its connection information so that the spawning tool can connect to it. If the file does exist, it contains the information specifying the server to which the spawned tool is to connect.
}
%
\declareEnvarNEW{PMIX_KEEPALIVE_PIPE}{
An integer \code{read}-end of a POSIX pipe that the tool should monitor for closure, thereby indicating that the parent tool has terminated. Used. for example, when a tool fork/exec's an intermediate launcher that should self-terminate if the originating tool exits.
}
%
Note that these environmental variables should be cleared from the environment after use and prior to forking child processes to avoid potentially unexpected behavior by the child processes.
%
%%%%%%%%%%%%%%%%%%%%%%%%%%%%%%%%%%%%%%%%%%%%%%%%%
\subsection{Tool connection attributes}
\label{api:struct:attributes:connection}


These attributes are defined to assist \ac{PMIx}-enabled tools to connect with a \ac{PMIx} server by passing them into either the \refapi{PMIx_tool_init} or the \refapi{PMIx_tool_attach_to_server} \acp{API} - thus, they are not typically accessed via the \refapi{PMIx_Get} \ac{API}.

%
\declareAttribute{PMIX_SERVER_PIDINFO}{"pmix.srvr.pidinfo"}{pid_t}{
\ac{PID} of the target \ac{PMIx} server for a tool.
}
%
\declareAttribute{PMIX_CONNECT_TO_SYSTEM}{"pmix.cnct.sys"}{bool}{
The requester requires that a connection be made only to a local, system-level \ac{PMIx} server.
}
%
\declareAttribute{PMIX_CONNECT_SYSTEM_FIRST}{"pmix.cnct.sys.first"}{bool}{
Preferentially, look for a system-level \ac{PMIx} server first.
}
%
\declareAttribute{PMIX_SERVER_URI}{"pmix.srvr.uri"}{char*}{
\ac{URI} of the \ac{PMIx} server to be contacted.
}
%
\declareAttribute{PMIX_SERVER_HOSTNAME}{"pmix.srvr.host"}{char*}{
Host where target \ac{PMIx} server is located.
}
%
\declareAttribute{PMIX_CONNECT_MAX_RETRIES}{"pmix.tool.mretries"}{uint32_t}{
Maximum number of times to try to connect to \ac{PMIx} server - the default value is implementation specific.
}
%
\declareAttribute{PMIX_CONNECT_RETRY_DELAY}{"pmix.tool.retry"}{uint32_t}{
Time in seconds between connection attempts to a \ac{PMIx} server - the default value is implementation specific.
}
%
\declareAttribute{PMIX_TOOL_DO_NOT_CONNECT}{"pmix.tool.nocon"}{bool}{
The tool wants to use internal \ac{PMIx} support, but does not want to connect to a \ac{PMIx} server.
}
%
\declareAttributeNEW{PMIX_TOOL_CONNECT_OPTIONAL}{"pmix.tool.conopt"}{bool}{
The tool shall connect to a server if available, but otherwise continue to operate unconnected.
}
%
\declareAttributeNEW{PMIX_TOOL_ATTACHMENT_FILE}{"pmix.tool.attach"}{char*}{
Pathname of file containing connection information to be used for attaching to a specific server.
}
%
\declareAttributeNEW{PMIX_LAUNCHER_RENDEZVOUS_FILE}{"pmix.tool.lncrnd"}{char*}{
Pathname of file where the launcher is to store its connection information so that the spawning tool can connect to it.
}
%
\declareAttributeNEW{PMIX_PRIMARY_SERVER}{"pmix.pri.srvr"}{bool}{
The server to which the tool is connecting shall be designated the \emph{primary} server once connection has been accomplished.
}
%
\declareAttributeNEW{PMIX_WAIT_FOR_CONNECTION}{"pmix.wait.conn"}{bool}{
Wait until the specified process has connected to the requesting tool or server, or the operation times out (if the \refattr{PMIX_TIMEOUT} directive is included in the request).
}


%%%%%%%%%%%%%%%%%%%%%%%%%%%%%%%%%%%%%%%%%%%%%%%%%
%%%%%%%%%%%%%%%%%%%%%%%%%%%%%%%%%%%%%%%%%%%%%%%%%
\section{Launching Applications with Tools}
\label{chap:api_tools:launch}

Tool-directed launches require that the tool include the \refattr{PMIX_LAUNCHER} attribute when calling \refapi{PMIx_tool_init}. Two launch modes are supported:

\begin{itemize}
    \item \emph{Direct launch} where the tool itself is directly responsible for launching all processes, including debugger daemons, using either the \ac{RM} or daemons launched by the tool – i.e., there is no \emph{intermediate launcher} (IL) such as \emph{mpiexec}. The case where the tool is self-contained (i.e., uses its own daemons without interacting with an external entity such as the \ac{RM}) lies outside the scope of this Standard; and
    \item \emph{Indirect launch} where all processes are started via an \ac{IL} such as \emph{mpiexec} and the tool itself is not directly involved in launching application processes or debugger daemons. Note that the \ac{IL} may utilize the \ac{RM} to launch processes and/or daemons under the tool's direction.
\end{itemize}

Either of these methods can be executed interactively or by a batch script. Note that not all host environments may support the direct launch method.

%%%%%%%%%%%%%%%%%%%%%%%%%%%%%%%%%%%%%%%%%%%%%%%%%
\subsection{Direct launch}
\label{chap:api_tools:direct}

In the direct-launch use-case (Fig. \ref{fig:dlaunch}), the tool itself performs the role of the launcher. Once invoked, the tool connects to an appropriate \ac{PMIx} server - e.g., a system-level server hosted by the \ac{RM}. The tool is responsible for assembling the description of the application to be launched (e.g., by parsing its command line) into a spawn request containing an array of \refstruct{pmix_app_t} applications and \refstruct{pmix_info_t} job-level information. An allocation of resources may or may not have been made in advance – if not, then the spawn request must include allocation request information.

\begingroup
\begin{figure*}[ht!]
  \begin{center}
    \includegraphics[clip,width=0.8\textwidth]{figs/directlaunch.pdf}
  \end{center}
  \caption{Direct Launch}
  \label{fig:dlaunch}
\end{figure*}
\endgroup


In addition to the attributes described in \refapi{PMIx_Spawn}, the tool may optionally wish to include the following tool-specific attributes in the \emph{job_info} argument to that \ac{API} (the debugger-related attributes are discussed in more detail in Section \ref{chap:api_tools:debuggers}):

\begin{itemize}
    \item \pasteAttributeItem{PMIX_FWD_STDIN}
    \item \pasteAttributeItem{PMIX_FWD_STDOUT}
    \item \pasteAttributeItem{PMIX_FWD_STDERR}
    \item \pasteAttributeItem{PMIX_FWD_STDDIAG}
    \item \pasteAttributeItem{PMIX_IOF_CACHE_SIZE}
    \item \pasteAttributeItem{PMIX_IOF_DROP_OLDEST}
    \item \pasteAttributeItem{PMIX_IOF_DROP_NEWEST}
    \item \pasteAttributeItem{PMIX_IOF_BUFFERING_SIZE}
    \item \pasteAttributeItem{PMIX_IOF_BUFFERING_TIME}
    \item \pasteAttributeItem{PMIX_IOF_OUTPUT_RAW}
    \item \pasteAttributeItem{PMIX_IOF_TAG_OUTPUT}
    \item \pasteAttributeItem{PMIX_IOF_TIMESTAMP_OUTPUT}
    \item \pasteAttributeItem{PMIX_IOF_XML_OUTPUT}
    \item \pasteAttributeItem{PMIX_IOF_RANK_OUTPUT}
    \item \pasteAttributeItem{PMIX_IOF_OUTPUT_TO_FILE}
    \item \pasteAttributeItem{PMIX_IOF_OUTPUT_TO_DIRECTORY}
    \item \pasteAttributeItem{PMIX_IOF_FILE_PATTERN}
    \item \pasteAttributeItem{PMIX_IOF_FILE_ONLY}
    \item \pasteAttributeItem{PMIX_IOF_MERGE_STDERR_STDOUT}
    \item \pasteAttributeItem{PMIX_NOHUP}
    \item \pasteAttributeItem{PMIX_NOTIFY_JOB_EVENTS}
    \item \pasteAttributeItem{PMIX_NOTIFY_COMPLETION}
    \item \pasteAttributeItem{PMIX_LOG_JOB_EVENTS}
    \item \pasteAttributeItem{PMIX_LOG_COMPLETION}
    \item \pasteAttributeItem{PMIX_DEBUG_STOP_ON_EXEC}
    \item \pasteAttributeItem{PMIX_DEBUG_STOP_IN_INIT}
    \item \pasteAttributeItem{PMIX_DEBUG_WAIT_FOR_NOTIFY}
\end{itemize}


\adviceuserstart
The \refattr{PMIX_IOF_FILE_ONLY} indicates output is directed to files and
no copy is sent back to the application.  For example, this can be combined with
\refattr{PMIX_IOF_OUTPUT_TO_FILE} or \refattr{PMIX_IOF_OUTPUT_TO_DIRECTORY} to
only output to files.
\adviceuserend

The tool then calls the \refapi{PMIx_Spawn} \ac{API} so that the \ac{PMIx} library can communicate the spawn request to the server.

Upon receipt, the \ac{PMIx} server library passes the spawn request to its host \ac{RM} daemon for processing via the \refapi{pmix_server_spawn_fn_t} server module function. If this callback was not provided, then the \ac{PMIx} server library will return the \refconst{PMIX_ERR_NOT_SUPPORTED} error status.

If an allocation must be made, then the host environment is responsible for
communicating the request to its associated scheduler. Once resources are
available, the host environment initiates the launch process to start the job.
The host environment must parse the spawn request for relevant directives,
returning an error if any required directive cannot be supported. Optional
directives may be ignored if they cannot be supported.

Any error while executing the spawn request must be returned by
\refapi{PMIx_Spawn} to the requester. Once the spawn request has succeeded in
starting the specified processes, the request will return
\refconst{PMIX_SUCCESS} back to the requester along with the namespace of the
started job. Upon termination of the spawned job, the host environment must
generate a \refconst{PMIX_EVENT_JOB_END} event for normal or abnormal
termination if requested to do so. The event shall include:

\begin{itemize}
    \item the returned status code (\refattr{PMIX_JOB_TERM_STATUS}) for the
    corresponding job;
    \item the identity (\refattr{PMIX_PROCID}) and exit status
    (\refattr{PMIX_EXIT_CODE}) of the first failed process, if applicable;
    \item a \refattr{PMIX_EVENT_TIMESTAMP} indicating the time the termination
    occurred; plus
    \item any other info provided by the host environment.
\end{itemize}

%%%%%%%%%%%%%%%%%%%%%%%%%%%%%%%%%%%%%%%%%%%%%%%%%
\subsection{Indirect launch}
\label{chap:api_tools:indirect}

In the indirect launch use-case, the application processes are started via an intermediate launcher (e.g., \emph{mpiexec}) that is itself started by the tool (see Fig \ref{fig:indirlnch}). Thus, at a high level, this is a two-stage launch procedure to start the application: the tool (henceforth referred to as the \emph{initiator}) starts the \ac{IL}, which then starts the applications. In practice, additional steps may be involved if, for example, the \ac{IL} starts its own daemons to shepherd the application processes.

A key aspect of this operational mode is the avoidance of any requirement that the initiator parse and/or understand the command line of the \ac{IL}. Instead, the indirect launch procedure supports either of two methods: one where the initiator assumes responsibility for parsing its command line to obtain the application as well as the \ac{IL} and its options, and another where the initiator defers the command line parsing to the \ac{IL}. Both of these methods are described in the following sections.

\subsubsection{Initiator-based command line parsing}
\label{chap:api_tools:indirect:tool}

This method utilizes a first call to the \refapi{PMIx_Spawn} \ac{API} to start the \ac{IL} itself, and then uses a second call to \refapi{PMIx_Spawn} to request that the \ac{IL} spawn the actual job. The burden of analyzing the initial command line to separately identify the \ac{IL}'s command line from the application itself falls upon the initiator. An example is provided below:

\begin{verbatim}
$ initiator --launcher "mpiexec --verbose" -n 3 ./app <appoptions>
\end{verbatim}

The initiator spawns the \ac{IL} using the same procedure for launching an application - it begins by assembling the description of the \ac{IL} into a spawn request containing an array of \refstruct{pmix_app_t} and \refstruct{pmix_info_t} job-level information. Note that this step does not include any information regarding the application itself - only the launcher is included. In addition, the initiator must include the rendezvous \ac{URI} in the environment so the \ac{IL} knows how to connect back to it.

An allocation of resources for the \ac{IL} itself may or may not be required – if it is, then the allocation must be made in advance or the spawn request must include allocation request information.

\begin{figure*}[ht!]
\centering
\begin{subfigure}{.5\textwidth}
  \centering
  \includegraphics[width=\textwidth]{figs/indirlnch-start.pdf}
  \caption{Indirect Launch - Start}
  \label{fig:indirlnch-start}
\end{subfigure}%
\begin{subfigure}{.5\textwidth}
  \centering
  \includegraphics[width=\textwidth]{figs/indirlnch-end.pdf}
  \caption{Indirect Launch - End}
  \label{fig:indirlnch-end}
\end{subfigure}
\caption{Indirect launch procedure}
\label{fig:indirlnch}
\end{figure*}

The initiator may optionally wish to include the following tool-specific attributes in the \emph{job_info} argument to \refapi{PMIx_Spawn} - note that these attributes refer only to the behavior of the \ac{IL} itself and not the eventual job to be launched:

\begin{itemize}
    \item \pasteAttributeItem{PMIX_FWD_STDIN}
    \item \pasteAttributeItem{PMIX_FWD_STDOUT}
    \item \pasteAttributeItem{PMIX_FWD_STDERR}
    \item \pasteAttributeItem{PMIX_FWD_STDDIAG}
    \item \pasteAttributeItem{PMIX_IOF_CACHE_SIZE}
    \item \pasteAttributeItem{PMIX_IOF_DROP_OLDEST}
    \item \pasteAttributeItem{PMIX_IOF_DROP_NEWEST}
    \item \pasteAttributeItem{PMIX_IOF_BUFFERING_SIZE}
    \item \pasteAttributeItem{PMIX_IOF_BUFFERING_TIME}
    \item \pasteAttributeItem{PMIX_IOF_TAG_OUTPUT}
    \item \pasteAttributeItem{PMIX_IOF_TIMESTAMP_OUTPUT}
    \item \pasteAttributeItem{PMIX_IOF_XML_OUTPUT}
    \item \pasteAttributeItem{PMIX_NOHUP}
    \item \pasteAttributeItem{PMIX_LAUNCHER_DAEMON}
    \item \pasteAttributeItem{PMIX_FORKEXEC_AGENT}
    \item \pasteAttributeItem{PMIX_EXEC_AGENT}
    \item \pasteAttributeItemBegin{PMIX_DEBUG_STOP_IN_INIT}In this context, the initiator is directing the \ac{IL} to stop in \refapi{PMIx_tool_init}. This gives the initiator a chance to connect to the \ac{IL} and register for events prior to the \ac{IL} launching the application job.
    \pasteAttributeItemEnd
\end{itemize}

and the following optional variables in the environment of the \ac{IL}:

\begin{itemize}
    \item \refenvar{PMIX_KEEPALIVE_PIPE} - an integer \code{read}-end of a POSIX pipe that the \ac{IL} should monitor for closure, thereby indicating that the initiator has terminated.
\end{itemize}

The initiator then calls the \refapi{PMIx_Spawn} \ac{API} so that the \ac{PMIx} library can either communicate the spawn request to a server (if connected to one), or locally spawn the \ac{IL} itself if not connected to a server and the \ac{PMIx} implementation includes self-spawn support. \refapi{PMIx_Spawn} shall return an error if neither of these conditions is met.

When initialized by the \ac{IL}, the \refapi{PMIx_tool_init} function must perform two operations:

\begin{itemize}
    \item check for the presence of the \refenvar{PMIX_KEEPALIVE_PIPE} environmental variable - if provided, then the library shall monitor the pipe for closure, providing a \refconst{PMIX_EVENT_JOB_END} event when the pipe closes (thereby indicating the termination of the initiator). The \ac{IL} should register for this event after completing \refapi{PMIx_tool_init} - the initiator's namespace can be obtained via a call to \refapi{PMIx_Get} with the \refattr{PMIX_PARENT_ID} key. Note that this feature will only be available if the spawned \ac{IL} is local to the initiator.
    \item check for the \refenvar{PMIX_LAUNCHER_RNDZ_URI} environmental parameter - if found, the library shall connect back to the initiator using the \refapi{PMIx_tool_attach_to_server} \ac{API}, retaining its current server as its primary server.
\end{itemize}

Once the \ac{IL} completes \refapi{PMIx_tool_init}, it must register for the \refconst{PMIX_EVENT_JOB_END} termination event and then idle until receiving that event - either directly from the initiator, or from the \ac{PMIx} library upon detecting closure of the keepalive pipe. The \ac{IL} idles in the intervening time as it is solely acting as a relay (if connected to a server that is performing the actual application launch) or as a \ac{PMIx} server responding to spawn requests.

Upon return from the \refapi{PMIx_Spawn} \ac{API}, the initiator should set the spawned \ac{IL} as its primary server using the \refapi{PMIx_tool_set_server} \ac{API} with the nspace returned by \refapi{PMIx_Spawn} and any valid rank (a rank of zero would ordinarily be used as only one \ac{IL} process is typically started). It is advisable to set a connection timeout value when calling this function. The initiator can then proceed to spawn the actual application according to the procedure described in Section \ref{chap:api_tools:direct}.

\subsubsection{\ac{IL}-based command line parsing}
\label{chap:api_tools:indirect:tool}

In the case where the initiator cannot parse its command line, it must defer that parsing to the \ac{IL}. A common example is provided below:

\begin{verbatim}
$ initiator mpiexec --verbose -n 3 ./app <appoptions>
\end{verbatim}

For this situation, the initiator proceeds as above with only one notable exception: instead of calling \refapi{PMIx_Spawn} twice (once to start the \ac{IL} and again to start the actual application), the initiator only calls that \ac{API} one time:

\begin{itemize}
    \item The \refarg{app} parameter passed to the spawn request contains only one \refstruct{pmix_app_t} that contains the entire command line, including both launcher and application(s).
    \item The launcher executable must be in the \refarg{app.cmd} field and in \refarg{app.argv[0]}, with the rest of the command line appended to the \refarg{app.argv} array.
    \item Any job-level directives for the \ac{IL} itself (e.g., \refattr{PMIX_FORKEXEC_AGENT} or \refattr{PMIX_FWD_STDOUT}) are included in the \refarg{job_info} parameter of the call to \refapi{PMIx_Spawn}.
    \item The job-level directives must include both the \refattr{PMIX_SPAWN_TOOL} attribute indicating that the initiator is spawning a tool, and the \refattr{PMIX_DEBUG_STOP_IN_INIT} attribute directing the \ac{IL} to stop during the call to \refapi{PMIx_tool_init}. The latter directive allows the initiator to connect to the \ac{IL} prior to launch of the application.
    \item The \refenvar{PMIX_LAUNCHER_RNDZ_URI} and \refenvar{PMIX_KEEPALIVE_PIPE} environmental variables are provided to the launcher in its environment via the \refarg{app.env} field.
    \item The \ac{IL} must use \refapi{PMIx_Get} with the \refattr{PMIX_LAUNCH_DIRECTIVES} key to obtain any initiator-provided directives (e.g., \refattr{PMIX_DEBUG_STOP_IN_INIT} or \refattr{PMIX_DEBUG_STOP_ON_EXEC}) aimed at the application(s) it will spawn.
\end{itemize}

Upon return from \refapi{PMIx_Spawn}, the initiator must:

\begin{itemize}
    \item use the \refapi{PMIx_tool_set_server} \ac{API} to set the spawned \ac{IL} as its primary server
    \item register with that server to receive the \refconst{PMIX_LAUNCH_COMPLETE} event. This allows the initiator to know when the \ac{IL} has completed launch of the application
    \item release the \ac{IL} from its "hold" in \refapi{PMIx_tool_init} by issuing the \refconst{PMIX_DEBUGGER_RELEASE} event, specifying the \ac{IL} as the custom range. Upon receipt of the event, the \ac{IL} is free to parse its command line, apply any provided directives, and execute the application.
\end{itemize}

Upon receipt of the \refconst{PMIX_LAUNCH_COMPLETE} event, the initiator should register to receive notification of completion of the returned namespace of the application. Receipt of the \refconst{PMIX_EVENT_JOB_END} event provides a signal that the initiator may itself terminate.


%%%%%%%%%%%%%%%%%%%%%%%%%%%%%%%%%%%%%%%%%%%%%%%%%
\subsection{Tool spawn-related attributes}
\label{api:tools:attributes:spawn}

Tools are free to utilize the spawn attributes available to applications (see \ref{api:struct:attributes:spawn}) when constructing a spawn request, but can also utilize the following attributes that are specific to tool-based spawn operations:

%
\declareAttribute{PMIX_FWD_STDIN}{"pmix.fwd.stdin"}{pmix_rank_t}{
The requester intends to push information from its \code{stdin} to the
indicated process. The local spawn agent should, therefore, ensure that the
\code{stdin} channel to that process remains available. A rank of
\refconst{PMIX_RANK_WILDCARD} indicates that all processes in the spawned job
are potential recipients. The requester will issue a call to
\refapi{PMIx_IOF_push} to initiate the actual forwarding of information to
specified targets - this attribute simply requests that the \ac{IL} retain the
ability to forward the information to the designated targets.
}
%
\declareAttribute{PMIX_FWD_STDOUT}{"pmix.fwd.stdout"}{bool}{
Requests that the ability to forward the \code{stdout} of the spawned
processes be
maintained. The requester will issue a call to \refapi{PMIx_IOF_pull} to
specify the callback function and other options for delivery of the forwarded
output.
}
%
\declareAttribute{PMIX_FWD_STDERR}{"pmix.fwd.stderr"}{bool}{
Requests that the ability to forward the \code{stderr} of the spawned
processes be
maintained. The requester will issue a call to \refapi{PMIx_IOF_pull} to
specify the callback function and other options for delivery of the forwarded
output.
}
%
\declareAttribute{PMIX_FWD_STDDIAG}{"pmix.fwd.stddiag"}{bool}{
Requests that the ability to forward the diagnostic channel (if it exists) of
the spawned processes be
maintained. The requester will issue a call to \refapi{PMIx_IOF_pull} to
specify the callback function and other options for delivery of the forwarded
output.
}
%
\declareAttributeNEW{PMIX_NOHUP}{"pmix.nohup"}{bool}{
Any processes started on behalf of the calling tool (or the specified namespace, if such specification is included in the list of attributes) should continue after the tool disconnects from its server.
}
%
\declareAttributeNEW{PMIX_LAUNCHER_DAEMON}{"pmix.lnch.dmn"}{char*}{
Path to executable that is to be used as the backend daemon for the launcher. This replaces the launcher's own daemon with the specified executable. Note that the user is therefore responsible for ensuring compatibility of the specified executable and the host launcher.
}
%
\declareAttributeNEW{PMIX_FORKEXEC_AGENT}{"pmix.frkex.agnt"}{char*}{
Path to executable that the launcher's backend daemons are to fork/exec in place of the actual application processes. The fork/exec agent shall connect back (as a \ac{PMIx} tool) to the launcher's daemon to receive its spawn instructions, and is responsible for starting the actual application process it replaced. See Section \ref{api:tools:debugger:agent} for details.
}
%
\declareAttributeNEW{PMIX_EXEC_AGENT}{"pmix.exec.agnt"}{char*}{
Path to executable that the launcher's backend daemons are to fork/exec in place of the actual application processes. The launcher's daemon shall pass the full command line of the application on the command line of the exec agent, which shall not connect back to the launcher's daemon. The exec agent is responsible for exec'ing the specified application process in its own place. See Section \ref{api:tools:debugger:agent} for details.
}
%
\declareAttributeNEW{PMIX_LAUNCH_DIRECTIVES}{"pmix.lnch.dirs"}{pmix_data_array_t*}{
Array of \refstruct{pmix_info_t} containing directives for the launcher - a convenience attribute for retrieving all directives with a single call to \refapi{PMIx_Get}.
}

%%%%%%%%%%%%%%%%%%%%%%%%%%%%%%%%%%%%%%%%%%%%%%%%%
\subsection{Tool rendezvous-related events}
\label{api:tools:attributes:spawnconst}

The following constants refer to events relating to rendezvous of a tool and launcher during spawn of the \ac{IL}.

\begin{constantdesc}
%
\declareconstitemNEW{PMIX_LAUNCHER_READY}
An application launcher (e.g., \emph{mpiexec}) shall generate this event to signal a tool that started it that the launcher is ready to receive directives/commands (e.g., \refapi{PMIx_Spawn}). This is only used when the initiator is able to parse the command line itself, or the launcher is started as a persistent \ac{DVM}.
%
\end{constantdesc}

%%%%%%%%%%%%%%%%%%%%%%%%%%%%%%%%%%%%%%%%%%%%%%%%%
%%%%%%%%%%%%%%%%%%%%%%%%%%%%%%%%%%%%%%%%%%%%%%%%%
\section{IO Forwarding}
\label{chap:api_tools:iof}

Underlying the operation of many tools is a common need to forward \code{stdin} from the tool to targeted processes, and to return \code{stdout}/\code{stderr} from those processes to the tool (e.g., for display on the user’s console). Historically, each tool developer was responsible for creating their own \ac{IO} forwarding subsystem. However, the introduction of \ac{PMIx} as a standard mechanism for interacting between applications and the host environment has made it possible to relieve tool developers of this burden.

This section defines functions by which tools can request forwarding of input/output to/from other processes and serves as a design guide to:

\begin{itemize}
    \item provide tool developers with an overview of the expected behavior of the \ac{PMIx} \ac{IO} forwarding support;
    \item guide \ac{RM} vendors regarding roles and responsibilities expected of the \ac{RM} to support \ac{IO} forwarding; and
    \item provide insight into the thinking of the \ac{PMIx} community behind the definition of the \ac{PMIx} \ac{IO} forwarding \acp{API}.
\end{itemize}

Note that the forwarding of \ac{IO} via \ac{PMIx} requires that both the host environment and the tool support \ac{PMIx}, but does not impose any similar requirements on the application itself.

The responsibility of the host environment in forwarding of \ac{IO} falls into the following areas:

\begin{itemize}
    \item Capturing output from specified processes.
    \item Forwarding that output to the host of the \ac{PMIx} server library that requested it.
    \item Delivering that payload to the \ac{PMIx} server library via the \refapi{PMIx_server_IOF_deliver} \ac{API} for final dispatch to the requesting tool.
\end{itemize}

It is the responsibility of the \ac{PMIx} library to buffer, format, and deliver the payload to the requesting client. This may require caching of output until a forwarding registration is received, as governed by the corresponding \ac{IO} forwarding attributes of Section \ref{api:tools:attributes:iof} that are supported by the implementation.


%%%%%%%%%%%%%%%%%%%%%%%%%%%%%%%%%%%%%%%%%%%%%%%%%
\subsection{Forwarding stdout/stderr}

At an appropriate point in its operation (usually during startup), a tool will utilize the \refapi{PMIx_tool_init} function to connect to a \ac{PMIx} server. The \ac{PMIx} server can be hosted by an \ac{RM} daemon or could be embedded in a library-provided starter program such as \textit{mpiexec} - in terms of \ac{IO} forwarding, the operations remain the same either way. For purposes of this discussion, we will assume the server is in an \ac{RM} daemon and that the application processes are directly launched by the \ac{RM}, as shown in Fig \ref{fig:stdouterr}.

\begingroup
\begin{figure*}[ht!]
  \begin{center}
    \includegraphics[clip,width=0.8\textwidth]{figs/output.pdf}
  \end{center}
  \caption{Forwarding stdout/stderr}
  \label{fig:stdouterr}
\end{figure*}
\endgroup

Once the tool has connected to the target server, it can request that
processes be spawned on its behalf or that output from a specified set of
existing processes in a given executing application be forwarded to it.
Requests to spawn processes should include the \refattr{PMIX_FWD_STDIN},
\refattr{PMIX_FWD_STDOUT}, and/or \refattr{PMIX_FWD_STDERR} attributes if the
tool intends to request that the corresponding streams be forwarded at some
point during execution.

Note that requests to capture output from existing processes via the
\refapi{PMIx_IOF_pull} \ac{API}, and/or to forward input to specified
processes via the \refapi{PMIx_IOF_push} \ac{API}, can only succeed if the
required attributes to retain that ability were passed when the corresponding
job was spawned. The host is required to return an error for all such requests
in cases where this condition is not met.

Two modes are supported when requesting that the host forward standard output/error via the \refapi{PMIx_IOF_pull} \ac{API} - these can be controlled by including one of the following attributes in the \refarg{info} array passed to that function:

\begin{itemize}
    \item \pasteAttributeItem{PMIX_IOF_COPY}
    \item \pasteAttributeItemBegin{PMIX_IOF_REDIRECT}This is the default mode of operation.
    \pasteAttributeItemEnd{}
\end{itemize}

When requesting to forward \code{stdout}/\code{stderr}, the tool can specify several formatting options to be used on the resulting output stream. These include:

\begin{itemize}
    \item \pasteAttributeItem{PMIX_IOF_TAG_OUTPUT}
    \item \pasteAttributeItem{PMIX_IOF_TIMESTAMP_OUTPUT}
    \item \pasteAttributeItem{PMIX_IOF_XML_OUTPUT}
    \item \pasteAttributeItem{PMIX_IOF_RANK_OUTPUT}
    \item \pasteAttributeItem{PMIX_IOF_OUTPUT_TO_FILE}
    \item \pasteAttributeItem{PMIX_IOF_OUTPUT_TO_DIRECTORY}
    \item \pasteAttributeItem{PMIX_IOF_FILE_PATTERN}
    \item \pasteAttributeItem{PMIX_IOF_FILE_ONLY}
    \item \pasteAttributeItem{PMIX_IOF_MERGE_STDERR_STDOUT}

\end{itemize}

The \ac{PMIx} client in the tool is responsible for formatting the output stream. Note that output from multiple processes will often be interleaved due to variations in arrival time - ordering of output is not guaranteed across processes and/or nodes.

%%%%%%%%%%%%%%%%%%%%%%%%%%%%%%%%%%%%%%%%%%%%%%%%%
\subsection{Forwarding stdin}

A tool is not necessarily a child of the \ac{RM} as it may have been started directly from the command line. Thus, provision must be made for the tool to collect its \code{stdin} and pass it to the host \ac{RM} (via the \ac{PMIx} server) for forwarding. Two methods of support for forwarding of \code{stdin} are defined:

\begingroup
\begin{figure*}[ht!]
  \begin{center}
    \includegraphics[clip,width=0.8\textwidth]{figs/stdin.pdf}
  \end{center}
  \caption{Forwarding stdin}
  \label{fig:stdin}
\end{figure*}
\endgroup

\begin{itemize}
    \item internal collection by the \ac{PMIx} tool library itself. This is requested via the \refattr{PMIX_IOF_PUSH_STDIN} attribute in the \refapi{PMIx_IOF_push} call. When this mode is selected, the tool library begins collecting all \code{stdin} data and internally passing it to the local server for distribution to the specified target processes. All collected data is sent to the same targets until \code{stdin} is closed, or a subsequent call to \refapi{PMIx_IOF_push} is made that includes the \refattr{PMIX_IOF_COMPLETE} attribute indicating that forwarding of \code{stdin} is to be terminated.
    \item external collection directly by the tool. It is assumed that the tool will provide its own code/mechanism for collecting its \code{stdin} as the tool developers may choose to insert some filtering and/or editing of the stream prior to forwarding it. In addition, the tool can directly control the targets for the data on a per-call basis – i.e., each call to \refapi{PMIx_IOF_push} can specify its own set of target recipients for that particular \emph{blob} of data. Thus, this method provides maximum flexibility, but requires that the tool developer provide their own code to capture \code{stdin}.
\end{itemize}

Note that it is the responsibility of the \ac{RM} to forward data to the host where the target process(es) are executing, and for the host daemon on that node to deliver the data to the \code{stdin} of target process(es). The \ac{PMIx} server on the remote node is not involved in this process. Systems that do not support forwarding of \code{stdin} shall return \refconst{PMIX_ERR_NOT_SUPPORTED} in response to a forwarding request.

\adviceuserstart
Scalable forwarding of \code{stdin} represents a significant challenge. Most environments will at least handle a \emph{send-to-1} model whereby \code{stdin} is forwarded to a single identified process, and occasionally an additional \emph{send-to-all} model where \code{stdin} is forwarded to all processes in the application. Users are advised to check their host environment for available support as the distribution method lies outside the scope of \ac{PMIx}.

\code{Stdin} buffering by the \ac{RM} and/or \ac{PMIx} library can be problematic. If any targeted recipient is slow reading data (or decides never to read data), then the data must be buffered in some intermediate daemon or the \ac{PMIx} tool library itself. Thus, piping a large amount of data into \code{stdin} can result in a very large memory footprint in the system management stack or the tool. Best practices, therefore, typically focus on reading of input files by application processes as opposed to forwarding of \code{stdin}.
\adviceuserend


%%%%%%%%%%%%%%%%%%%%%%%%%%%%%%%%%%%%%%%%%%%%%%%%%
\subsection{IO Forwarding Channels}
\declarestruct{pmix_iof_channel_t}
\label{api:tool:iofchannels}

\versionMarker{3.0}
The \refstruct{pmix_iof_channel_t} structure is a \code{uint16_t} type that defines a set of bit-mask flags for specifying IO forwarding channels. These can be bitwise OR'd together to reference multiple channels.

\begin{constantdesc}
%
\declareconstitem{PMIX_FWD_NO_CHANNELS}
Forward no channels.
%
\declareconstitem{PMIX_FWD_STDIN_CHANNEL}
Forward \code{stdin}.
%
\declareconstitem{PMIX_FWD_STDOUT_CHANNEL}
Forward \code{stdout}.
%
\declareconstitem{PMIX_FWD_STDERR_CHANNEL}
Forward \code{stderr}.
%
\declareconstitem{PMIX_FWD_STDDIAG_CHANNEL}
Forward \code{stddiag}, if available.
%
\declareconstitem{PMIX_FWD_ALL_CHANNELS}
Forward all available channels.
%
\end{constantdesc}

%%%%%%%%%%%%%%%%%%%%%%%%%%%%%%%%%%%%%%%%%%%%%%%%%
\subsection{IO Forwarding constants}

\begin{constantdesc}
%
\declareconstitemNEW{PMIX_ERR_IOF_FAILURE}
An \ac{IO} forwarding operation failed - the affected channel will be included in the notification.
%
\declareconstitemNEW{PMIX_ERR_IOF_COMPLETE}
\ac{IO} forwarding of the standard input for this process has completed - i.e., the stdin file descriptor has closed.
%
\end{constantdesc}

%%%%%%%%%%%%%%%%%%%%%%%%%%%%%%%%%%%%%%%%%%%%%%%%%
\subsection{IO Forwarding attributes}
\label{api:tools:attributes:iof}

The following attributes are used to control \ac{IO} forwarding behavior at the request of tools. Use of the attributes is optional - any option not provided will revert to some implementation-specific value.

%
\declareAttributeNEW{PMIX_IOF_LOCAL_OUTPUT}{"pmix.iof.local"}{bool}{
Write output streams to local stdout/err
}
%
\declareAttributeNEW{PMIX_IOF_MERGE_STDERR_STDOUT}{"pmix.iof.mrg"}{bool}{
Merge stdout and stderr streams from application procs
}
%
\declareAttribute{PMIX_IOF_CACHE_SIZE}{"pmix.iof.csize"}{uint32_t}{
The requested size of the \ac{PMIx} server cache in bytes for each specified channel. By default, the server is allowed (but not required) to drop all bytes received beyond the max size.
}
%
\declareAttribute{PMIX_IOF_DROP_OLDEST}{"pmix.iof.old"}{bool}{
In an overflow situation, the \ac{PMIx} server is to drop the oldest bytes to make room in the cache.
}
%
\declareAttribute{PMIX_IOF_DROP_NEWEST}{"pmix.iof.new"}{bool}{
In an overflow situation, the \ac{PMIx} server is to drop any new bytes received until room becomes available in the cache (default).
}
%
\declareAttribute{PMIX_IOF_BUFFERING_SIZE}{"pmix.iof.bsize"}{uint32_t}{
Requests that \ac{IO} on the specified channel(s) be aggregated in the \ac{PMIx} tool library until the specified number of bytes is collected to avoid being called every time a block of \ac{IO} arrives. The \ac{PMIx} tool library will execute the callback and reset the collection counter whenever the specified number of bytes becomes available. Any remaining buffered data will be \emph{flushed} to the callback upon a call to deregister the respective channel.
}
%
\declareAttribute{PMIX_IOF_BUFFERING_TIME}{"pmix.iof.btime"}{uint32_t}{
Max time in seconds to buffer \ac{IO} before delivering it. Used in conjunction with buffering size, this prevents \ac{IO} from being held indefinitely while waiting for another payload to arrive.
}
%
\declareAttributeNEW{PMIX_IOF_OUTPUT_RAW}{"pmix.iof.raw"}{bool}{
Do not buffer output to be written as complete lines - output characters as the stream delivers them
}
%
\declareAttribute{PMIX_IOF_COMPLETE}{"pmix.iof.cmp"}{bool}{
Indicates that the specified \ac{IO} channel has been closed by the source.
}
%
\declareAttribute{PMIX_IOF_TAG_OUTPUT}{"pmix.iof.tag"}{bool}{
Requests that output be prefixed with the nspace,rank of the source and a string identifying the channel (\code{stdout}, \code{stderr}, etc.).
}
%
\declareAttribute{PMIX_IOF_TIMESTAMP_OUTPUT}{"pmix.iof.ts"}{bool}{
Requests that output be marked with the time at which the data was received by the tool - note that this will differ from the time at which the data was collected from the source.
}
%
\declareAttributeNEW{PMIX_IOF_RANK_OUTPUT}{"pmix.iof.rank"}{bool}{
Tag output with the rank it came from
}
%
\declareAttribute{PMIX_IOF_XML_OUTPUT}{"pmix.iof.xml"}{bool}{
Requests that output be formatted in \ac{XML}.
}
%
\declareAttributeNEW{PMIX_IOF_PUSH_STDIN}{"pmix.iof.stdin"}{bool}{
Requests that the \ac{PMIx} library collect the \code{stdin} of the requester and forward it to the processes specified in the \refapi{PMIx_IOF_push} call. All collected data is sent to the same targets until \code{stdin} is closed, or a subsequent call to \refapi{PMIx_IOF_push} is made that includes the \refattr{PMIX_IOF_COMPLETE} attribute indicating that forwarding of \code{stdin} is to be terminated.
}
%
\declareAttributeNEW{PMIX_IOF_COPY}{"pmix.iof.cpy"}{bool}{
Requests that the host environment deliver a copy of the specified output stream(s) to the tool, letting the stream(s) continue to also be delivered to the default location. This allows the tool to tap into the output stream(s) without redirecting it from its current final destination.
}
%
\declareAttributeNEW{PMIX_IOF_REDIRECT}{"pmix.iof.redir"}{bool}{
Requests that the host environment intercept the specified output stream(s) and deliver it to the requesting tool instead of its current final destination. This might be used, for example, during a debugging procedure to avoid injection of debugger-related output into the application’s results file. The original output stream(s) destination is restored upon termination of the tool.
}
%
\declareAttributeNEW{PMIX_IOF_OUTPUT_TO_FILE}{"pmix.iof.file"}{char*}{
Direct application output into files of form "<filename>.<nspace>.<rank>.out" (for \code{stdout}) and "<filename>.<nspace>.<rank>.err" (for \code{stderr}). If \refattr{PMIX_IOF_MERGE_STDERR_STDOUT} was given, then only the \code{stdout} file will be created and both streams will be written into it.
}
%
\declareAttributeNEW{PMIX_IOF_OUTPUT_TO_DIRECTORY}{"pmix.iof.dir"}{char*}{
direct application output into files of form "<directory>/<nspace>/rank.<rank>/stdout" (for \code{stdout}) and "<directory>/<nspace>/rank.<rank>/stderr" (for \code{stderr}). If \refattr{PMIX_IOF_MERGE_STDERR_STDOUT} was given, then only the \code{stdout} file will be created and both streams will be written into it.
}
%
\declareAttributeNEW{PMIX_IOF_FILE_PATTERN}{"pmix.iof.fpt"}{bool}{
Specified output file is to be treated as a pattern and not automatically annotated by nspace, rank, or other parameters. The pattern can use \code{\%n} for the namespace, and \code{\%r} for the rank wherever those quantities are to be placed. The resulting filename will be appended with ".out" for the \code{stdout} stream and ".err" for the \code{stderr} stream. If \refattr{PMIX_IOF_MERGE_STDERR_STDOUT} was given, then only the \code{stdout} file will be created and both streams will be written into it.
}
%
\declareAttributeNEW{PMIX_IOF_FILE_ONLY}{"pmix.iof.fonly"}{bool}{
Output only into designated files - do not also output a copy to the console's stdout/stderr
}
%

%%%%%%%%%%%%%%%%%%%%%%%%%%%%%%%%%%%%%%%%%%%%%%%%%
%%%%%%%%%%%%%%%%%%%%%%%%%%%%%%%%%%%%%%%%%%%%%%%%%
\section{Debugger Support}
\label{chap:api_tools:debuggers}

Debuggers are a class of tool that merits special consideration due to their particular requirements for access to job-related information and control over process execution. The primary advantage of using \ac{PMIx} for these purposes lies in the resulting portability of the debugger as it can be used with any system and/or programming model that supports \ac{PMIx}. In addition to the general tool support described above, debugger support includes:

\begin{itemize}
    \item Co-location, co-spawn, and communication wireup of debugger daemons for scalable launch. This includes providing debugger daemons with endpoint connection information across the daemons themselves.
    \item Identification of the job that is to be debugged. This includes automatically providing debugger daemons with the job-level information for their target job.
\end{itemize}

Debuggers can also utilize the options in the \refapi{PMIx_Spawn} \ac{API} to exercise a degree of control over spawned jobs for debugging purposes. For example, a debugger can utilize the environmental parameter attributes of Section \ref{api:struct:attributes:spawn} to request \code{LD_PRELOAD} of a memory interceptor library prior to spawning an application process, or interject a custom fork/exec agent to shepherd the application process.

A key element of the debugging process is the ability of the debugger to require that processes \emph{pause} at some well-defined point, thereby providing the debugger with an opportunity to attach and control execution. The actual implementation of the \emph{pause} lies outside the scope of \ac{PMIx} - it typically requires either the launcher or the application itself to implement the necessary operations. However, \ac{PMIx} does provide several standard attributes by which the debugger can specify the desired attach point:

\begin{itemize}
    \item \pasteAttributeItemBegin{PMIX_DEBUG_STOP_ON_EXEC}Launchers that cannot support this operation shall return an error from the \refapi{PMIx_Spawn} \ac{API} if this behavior is requested.
    \pasteAttributeItemEnd{}
    \item \pasteAttributeItemBegin{PMIX_DEBUG_STOP_IN_INIT}\ac{PMIx} implementations that do not support this operation shall return an error from \refapi{PMIx_Init} if this behavior is requested. Launchers that cannot support this operation shall return an error from the \refapi{PMIx_Spawn} \ac{API} if this behavior is requested.
    \pasteAttributeItemEnd{}
    \item \pasteAttributeItemBegin{PMIX_DEBUG_WAIT_FOR_NOTIFY}Launchers that cannot support this operation shall return an error from the \refapi{PMIx_Spawn} \ac{API} if this behavior is requested.

    Note that there is no mechanism by which the \ac{PMIx} library or the launcher can verify that an application will recognize and support the \refattr{PMIX_DEBUG_WAIT_FOR_NOTIFY} request. Debuggers utilizing this attachment method must, therefore, be prepared to deal with the case where the application fails to recognize and/or honor the request.
    \pasteAttributeItemEnd{}
\end{itemize}

If the \ac{PMIx} implementation and/or the host environment support it, debuggers can utilize the \refapi{PMIx_Query_info} \ac{API} to determine which features are available via the \refattr{PMIX_QUERY_ATTRIBUTE_SUPPORT} attribute.

\begin{itemize}
    \item \refattr{PMIX_DEBUG_STOP_IN_INIT} by checking \refattr{PMIX_CLIENT_ATTRIBUTES} for the \refapi{PMIx_Init} \ac{API}.
    \item \refattr{PMIX_DEBUG_STOP_ON_EXEC} by checking \refattr{PMIX_HOST_ATTRIBUTES} for the \refapi{PMIx_Spawn} \ac{API}.
\end{itemize}

The target namespace or process (as given by the debugger in the spawn request) shall be provided to each daemon in its job-level information via the \refattr{PMIX_DEBUG_TARGET} attribute. Debugger daemons are responsible for self-determining their specific target process(es), and can then utilize the \refapi{PMIx_Query_info} \ac{API} to obtain information about them (see Fig \ref{fig:dbgptable}) - e.g., to obtain the \acp{PID} of the local processes to which they need to attach. \ac{PMIx} provides the \refstruct{pmix_proc_info_t} structure for organizing information about a process' \ac{PID}, location, and state. Debuggers may request information on a given job at two levels:

\begin{itemize}
    \item \pasteAttributeItem{PMIX_QUERY_PROC_TABLE}
    \item \pasteAttributeItem{PMIX_QUERY_LOCAL_PROC_TABLE}
\end{itemize}

Note that the information provided in the returned proctable represents a snapshot in time. Any process, regardless of role (tool, client, debugger, etc.) can obtain the proctable of a given namespace so long as it has the system-determined authorizations to do so. The list of namespaces available via a given server can be obtained using the \refapi{PMIx_Query_info} \ac{API} with the \refattr{PMIX_QUERY_NAMESPACES} key.

\begingroup
\begin{figure*}[ht!]
  \begin{center}
    \includegraphics[clip,width=0.8\textwidth]{figs/dbgptable.pdf}
  \end{center}
  \caption{Obtaining proctables}
  \label{fig:dbgptable}
\end{figure*}
\endgroup

Debugger daemons can be started in two ways - either at the same time the application is spawned, or separately at a later time.

%%%%%%%%%%%%%%%%%%%%%%%%%%%%%%%%%%%%%%%%%%%%%%%%%
\subsection{Co-Location of Debugger Daemons}
\label{chap:api_tools:colocate}

Debugging operations typically require the use of daemons that are located on
the same node as the processes they are attempting to debug. The debugger can,
of course, specify its own mapping method when issuing its spawn request or
utilize its own internal launcher to place the daemons. However, when attaching
to a running job, \ac{PMIx} provides debuggers with a simplified method for
requesting that the launcher associated with the job \emph{co-locate} the
required daemons. Debuggers can request \emph{co-location} of their daemons by
adding the following attributes to the \refapi{PMIx_Spawn} used to spawn them:

\begin{itemize}
    \item \refattr{PMIX_DEBUGGER_DAEMONS} - indicating that the launcher is
    being asked to spawn debugger daemons.
    \item \refattr{PMIX_DEBUG_TARGET} - indicating the job or process that is
    to be debugged. This allows the launcher to identify the processes to be
    debugged and their location. Note that the debugger job shall be assigned
    its own namespace (different from that of the job it is being spawned
    to debug) and each daemon will be assigned a unique rank within that
    namespace.
    \item \refattr{PMIX_DEBUG_DAEMONS_PER_PROC} - specifies the number of
    debugger daemons to be co-located per target process.
    \item \refattr{PMIX_DEBUG_DAEMONS_PER_NODE} - specifies the number of
    debugger daemons to be co-located per node where at least one target
    process is executing.
\end{itemize}

Debugger daemons spawned in this manner shall be provided with the typical
\ac{PMIx} information for their own job plus the target they are to debug via
the \refattr{PMIX_DEBUG_TARGET} attribute. The debugger daemons spawned on a
given node are responsible for self-determining their specific target
process(es) - e.g., by referencing their own \refattr{PMIX_LOCAL_RANK} in the
daemon debugger job versus the corresponding \refattr{PMIX_LOCAL_RANK} of the
target processes on the node. Note that the debugger will be attaching to the application processes
at some arbitrary point in the application's execution unless some method for pausing the application
(e.g., by providing a \ac{PMIx} directive at time of launch, or via a tool using the
\refapi{PMIx_Job_control} \ac{API} to direct that the process be paused) has been employed.

\adviceuserstart
Note that the tool calling \refapi{PMIx_Spawn} to request the launch of the debugger daemons is \emph{not} included in the resulting job - i.e., the debugger daemons do not inherit the namespace of the tool. Thus, collective operations and notifications that target the debugger daemon job will not include the tool unless the namespace/rank of the tool is explicitly included.
\adviceuserend

%%%%%%%%%%%%%%%%%%%%%%%%%%%%%%%%%%%%%%%%%%%%%%%%%
\subsection{Co-Spawn of Debugger Daemons}
\label{chap:api_tools:cospawn}

In the case where a job is being spawned under the control of a debugger, \ac{PMIx} provides a shortcut method for spawning the debugger's daemons in parallel with the job. This requires that the debugger be specified as one of the \refstruct{pmix_app_t} in the same spawn command used to start the job. The debugger application must include at least the \refattr{PMIX_DEBUGGER_DAEMONS} attribute identifying itself as a debugger, and may utilize either a mapping option to direct daemon placement, or one of the \refattr{PMIX_DEBUG_DAEMONS_PER_PROC} or \refattr{PMIX_DEBUG_DAEMONS_PER_NODE} directives.

The launcher must not include information regarding the debugger daemons in
the job-level info
provided to the rest of the \refstruct{pmix_app_t}s, nor in any calculated rank
values (e.g., \refattr{PMIX_NODE_RANK} or \refattr{PMIX_LOCAL_RANK}) in those applications. The
debugger job is to be assigned its own namespace and each debugger daemon shall
receive a unique rank - i.e., the debugger application is to be treated as a
completely separate \ac{PMIx} job that is simply being started in parallel with
the user's applications. The launcher is free to implement the launch as a
single operation for both the applications and debugger daemons (preferred), or
may stage the launches as required. The launcher shall not return from the
\refapi{PMIx_Spawn} command until all included applications and the debugger
daemons have been started.

Attributes that apply to both the debugger daemons and the application processes can
be specified in the \refarg{job_info} array passed into the
\refapi{PMIx_Spawn} \ac{API}. Attributes that either (a) apply solely to the
debugger daemons or to one of the applications included in the spawn request,
or (b) have values that differ from those provided in the \refarg{job_info}
array, should be specified in the \refarg{info} array in the corresponding
\refstruct{pmix_app_t}.
Note that \ac{PMIx} job \emph{pause} attributes (e.g., \refattr{PMIX_DEBUG_STOP_IN_INIT}) do not apply to applications (defined in \refstruct{pmix_app_t}) where the \refattr{PMIX_DEBUGGER_DAEMONS} attribute is set to \code{true}.

Debugger daemons spawned in this manner shall be provided with the typical
\ac{PMIx} information for their own job plus the target they are to debug via
the \refattr{PMIX_DEBUG_TARGET} attribute. The debugger daemons spawned on a
given node are responsible for self-determining their specific target
process(es) - e.g., by referencing their own \refattr{PMIX_LOCAL_RANK} in the
daemon debugger job versus the corresponding \refattr{PMIX_LOCAL_RANK} of the
target processes on the node.

\adviceuserstart
Note that the tool calling \refapi{PMIx_Spawn} to request the launch of the debugger daemons is \emph{not} included in the resulting job - i.e., the debugger daemons do not inherit the namespace of the tool. Thus, collective operations and notifications that target the debugger daemon job will not include the tool unless the namespace/rank of the tool is explicitly included.

The \refapi{PMIx_Spawn} \ac{API} only supports the return of a single namespace resulting from the spawn request. In the case where the debugger job is co-spawned with the application, the spawn function shall return the namespace of the application and not the debugger job. Tools requiring access to the namespace of the debugger job must query the launcher for the spawned namespaces to find the one belonging to the debugger job.
\adviceuserend

%%%%%%%%%%%%%%%%%%%%%%%%%%%%%%%%%%%%%%%%%%%%%%%%%
\subsection{Debugger Agents}
\label{api:tools:debugger:agent}

Individual debuggers may, depending upon implementation, require varying degrees of control over each application process when it is started beyond those available via directives to \refapi{PMIx_Spawn}. \ac{PMIx} offers two mechanisms to help provide a means of meeting these needs.

The \refattr{PMIX_FORKEXEC_AGENT} attribute allows the debugger to specify an intermediate process (the \ac{FEA}) for spawning the actual application process (see Fig. \ref{fig:dbgfea}), thereby interposing the debugger daemon between the application process and the launcher's daemon. Instead of spawning the application process, the launcher will spawn the \ac{FEA}, which will connect back to the \ac{PMIx} server as a tool to obtain the spawn description of the application process it is to spawn. The \ac{PMIx} server in the launcher's daemon shall not register the fork/exec agent as a local client process, nor shall the launcher include the agent in any of the job-level values (e.g., \refattr{PMIX_RANK} within the job or \refattr{PMIX_LOCAL_RANK} on the node) provided to the application process. The launcher shall treat the collection of \acp{FEA} as a debugger job equivalent to the co-spawn use-case described in Section \ref{chap:api_tools:cospawn}.

\begin{figure*}[ht!]
\centering
\begin{subfigure}{.5\textwidth}
  \centering
  \includegraphics[width=\textwidth]{figs/dbgfea.pdf}
  \caption{Fork/exec agent}
  \label{fig:dbgfea}
\end{subfigure}%
\begin{subfigure}{.5\textwidth}
  \centering
  \includegraphics[width=\textwidth]{figs/dbgea.pdf}
  \caption{Exec agent}
  \label{fig:dbgea}
\end{subfigure}
\caption{Intermediate agents}
\label{fig:dbginta}
\end{figure*}

In contrast, the \refattr{PMIX_EXEC_AGENT} attribute (Fig. \ref{fig:dbgea}) allows the debugger to specify an agent that will perform some preparatory actions and then exec the eventual application process to replace itself. In this scenario, the exec agent is provided with the application process' command line as arguments on its command line (e.g., \code{"./agent appargv[0] appargv[1]"}) and does not connect back to the host's \ac{PMIx} server. It is the responsibility of the exec agent to properly separate its own command line arguments (if any) from the application description.

%%%%%%%%%%%%%%%%%%%%%%%%%%%%%%%%%%%%%%%%%%%%%%%%%
\subsection{Tracking the job lifecycle}
\label{api:tools:trkjob}

There are a wide range of events a debugger can register to receive, but three
are specifically defined for tracking a job's progress:

\begin{itemize}
    \item \refconst{PMIX_EVENT_JOB_START} indicates when the first process in
    the job has been spawned.
    \item \refconst{PMIX_LAUNCH_COMPLETE} indicates when the last process in
    the job has been spawned.
    \item \refconst{PMIX_EVENT_JOB_END} indicates that all processes have
    terminated.
\end{itemize}

Each event is required to contain at least the namespace of the corresponding
job and a \refattr{PMIX_EVENT_TIMESTAMP} indicating the time the event
occurred. In addition, the \refconst{PMIX_EVENT_JOB_END} event shall contain
the returned status code (\refattr{PMIX_JOB_TERM_STATUS}) for the
corresponding job, plus the identity (\refattr{PMIX_PROCID}) and exit status
(\refattr{PMIX_EXIT_CODE}) of the first failed process, if applicable.
Generation of these events by the launcher can be requested by including the
\refattr{PMIX_NOTIFY_JOB_EVENTS} attributes in the spawn request. Note that
these events can be logged via the \refapi{PMIx_Log} \ac{API} by
including the \refattr{PMIX_LOG_JOB_EVENTS} attribute - this can be done either
in conjunction with generated events, or in place of them.

Alternatively, if the debugger or tool solely wants to be alerted to job
termination, then including the \refattr{PMIX_NOTIFY_COMPLETION} attribute in
the spawn request would suffice. This attribute directs the launcher to provide
just the \refconst{PMIX_EVENT_JOB_END} event. Note that this event can be
logged via the \refapi{PMIx_Log} \ac{API} by including the
\refattr{PMIX_LOG_COMPLETION} attribute - this can be done either in
conjunction with the generated event, or in place of it.

\adviceuserstart
The \ac{PMIx} server is required to cache events in order to avoid race
conditions - e.g., when a tool is trying to register for the
\refconst{PMIX_EVENT_JOB_END} event from a very short-lived job. Accordingly,
registering for job-related events can result in receiving events relating to
jobs other than the one of interest.

Users are therefore advised to specify the job whose events are of interest by
including the \refattr{PMIX_EVENT_AFFECTED_PROC} or
\refattr{PMIX_EVENT_AFFECTED_PROCS} attribute in the \refarg{info} array passed
to the \refapi{PMIx_Register_event_handler} \ac{API}.

\adviceuserend

%%%%%%%%%%%%%%%%%%%%%%%%%%%%%%%%%%%%%%%%%%%%%%%%%
\subsubsection{Job lifecycle events}

\begin{constantdesc}
%
\declareconstitemNEW{PMIX_EVENT_JOB_START}
The first process in the job has been spawned - includes \refattr{PMIX_EVENT_TIMESTAMP} as well as the \refattr{PMIX_JOBID} and/or \refattr{PMIX_NSPACE} of the job.
%
\declareconstitemNEW{PMIX_LAUNCH_COMPLETE}
All processes in the job have been spawned - includes \refattr{PMIX_EVENT_TIMESTAMP} as well as the \refattr{PMIX_JOBID} and/or \refattr{PMIX_NSPACE} of the job.
%
\declareconstitemNEW{PMIX_EVENT_JOB_END}
All processes in the job have terminated - includes \refattr{PMIX_EVENT_TIMESTAMP} when the last process terminated as well as the \refattr{PMIX_JOBID} and/or \refattr{PMIX_NSPACE} of the job.
%
\declareconstitemNEW{PMIX_EVENT_SESSION_START}
The allocation has been instantiated and is ready for use - includes \refattr{PMIX_EVENT_TIMESTAMP} as well as the \refattr{PMIX_SESSION_ID} of the allocation. This event is issued after any system-controlled prologue has completed, but before any user-specified actions are taken.
%
\declareconstitemNEW{PMIX_EVENT_SESSION_END}
The allocation has terminated - includes \refattr{PMIX_EVENT_TIMESTAMP} as well as the \refattr{PMIX_SESSION_ID} of the allocation. This event is issued after any user-specified actions have completed, but before any system-controlled epilogue is performed.
%
\end{constantdesc}

The following events relate to processes within a job:

\begin{constantdesc}
%
\declareconstitem{PMIX_EVENT_PROC_TERMINATED}
The specified process(es) terminated - normal or abnormal
termination will be indicated by the \refattr{PMIX_PROC_TERM_STATUS} in the
\refarg{info} array of the notification. Note that a request for individual
process events can generate a significant event volume from large-scale jobs.
%
\declareconstitemNEW{PMIX_ERR_PROC_TERM_WO_SYNC}
Process terminated without calling \refapi{PMIx_Finalize}, or was a member of an assemblage formed via \refapi{PMIx_Connect} and terminated or called \refapi{PMIx_Finalize} without first calling \refapi{PMIx_Disconnect} (or its non-blocking form) from that assemblage.
%
\end{constantdesc}

The following constants may be included via the
\refattr{PMIX_JOB_TERM_STATUS} attributed in the \refarg{info} array in the
\refconst{PMIX_EVENT_JOB_END} event notification to provide more detailed
information regarding the reason for job abnormal termination:

\begin{constantdesc}
%
\declareconstitemNEW{PMIX_ERR_JOB_CANCELED}
The job was canceled by the host environment.
%
\declareconstitemNEW{PMIX_ERR_JOB_ABORTED}
One or more processes in the job called abort, causing the job to be terminated.
%
\declareconstitemNEW{PMIX_ERR_JOB_KILLED_BY_CMD}
The job was killed by user command.
%
\declareconstitemNEW{PMIX_ERR_JOB_ABORTED_BY_SIG}
The job was aborted due to receipt of an error signal (e.g., SIGKILL).
%
\declareconstitemNEW{PMIX_ERR_JOB_TERM_WO_SYNC}
The job was terminated due to at least one process terminating without calling \refapi{PMIx_Finalize}, or was a member of an assemblage formed via \refapi{PMIx_Connect} and terminated or called \refapi{PMIx_Finalize} without first calling \refapi{PMIx_Disconnect} (or its non-blocking form) from that assemblage.
%
\declareconstitemNEW{PMIX_ERR_JOB_SENSOR_BOUND_EXCEEDED}
The job was terminated due to one or more processes exceeding a specified sensor limit.
%
\declareconstitemNEW{PMIX_ERR_JOB_NON_ZERO_TERM}
The job was terminated due to one or more processes exiting with a non-zero status.
%
\declareconstitemNEW{PMIX_ERR_JOB_ABORTED_BY_SYS_EVENT}
The job was aborted due to receipt of a system event.
%
\end{constantdesc}


%%%%%%%%%%%%%%%%%%%%%%%%%%%%%%%%%%%%%%%%%%%%%%%%%
\subsubsection{Job lifecycle attributes}

\declareAttribute{PMIX_JOB_TERM_STATUS}{"pmix.job.term.status"}{pmix_status_t}{
Status returned by job upon its termination. The status will be communicated as part of a \ac{PMIx} event payload provided by the host environment upon termination of a job. Note that generation of the \refconst{PMIX_EVENT_JOB_END} event is optional and host environments may choose to provide it only upon request.
}
%
\declareAttribute{PMIX_PROC_STATE_STATUS}{"pmix.proc.state"}{pmix_proc_state_t}{
State of the specified process as of the last report - may not be the actual current state based on update rate.
}
%
\declareAttribute{PMIX_PROC_TERM_STATUS}{"pmix.proc.term.status"}{pmix_status_t}{
Status returned by a process upon its termination. The status will be communicated as part of a \ac{PMIx} event payload provided by the host environment upon termination of a process. Note that generation of the \refconst{PMIX_EVENT_PROC_TERMINATED} event is optional and host environments may choose to provide it only upon request.
}

%%%%%%%%%%%%%%%%%%%%%%%%%%%%%%%%%%%%%%%%%%%%%%%%%
\subsection{Debugger-related constants}
\label{api:tools:attributes:dbgconst}

The following constants are used in events used to coordinate applications and the debuggers attaching to them.

\begin{constantdesc}
%
\declareconstitemNEW{PMIX_DEBUG_WAITING_FOR_NOTIFY}
All processes in the job to be debugged are paused waiting for a release at some point within the application. The application shall remain in a paused
state awaiting release until receipt of the \refconst{PMIX_DEBUGGER_RELEASE}.
%
\declareconstitemNEW{PMIX_DEBUGGER_RELEASE}
Release processes that are paused at the \refattr{PMIX_DEBUG_WAIT_FOR_NOTIFY}
point in the target application.
%
\end{constantdesc}

%%%%%%%%%%%%%%%%%%%%%%%%%%%%%%%%%%%%%%%%%%%%%%%%%
\subsection{Debugger attributes}
\label{api:struct:attributes:debugger}

Attributes used to assist debuggers - these are values that can either be passed to the \refapi{PMIx_Spawn} \acp{API} or accessed by a debugger itself using the \refapi{PMIx_Get} \ac{API} with the \refconst{PMIX_RANK_WILDCARD} rank.

%
\declareAttribute{PMIX_DEBUG_STOP_ON_EXEC}{"pmix.dbg.exec"}{bool}{
Included in either the \refstruct{pmix_info_t} array in a \refstruct{pmix_app_t} description (if the directive applies only to that application) or in the \emph{job_info} array if it applies to all applications in the given spawn request. Indicates that the application is being spawned under a debugger, and that the local launch agent is to pause the resulting application processes on first instruction for debugger attach. The launcher (\ac{RM} or \ac{IL}) is to generate the \refconst{PMIX_LAUNCH_COMPLETE} event when all processes are stopped at the exec point.
}
%
\declareAttribute{PMIX_DEBUG_STOP_IN_INIT}{"pmix.dbg.init"}{bool}{
Included in either the \refstruct{pmix_info_t} array in a \refstruct{pmix_app_t} description (if the directive applies only to that application) or in the \emph{job_info} array if it applies to all applications in the given spawn request. Indicates that the specified application is being spawned under a debugger. The \ac{PMIx} client library in each resulting application process shall notify its \ac{PMIx} server that it is pausing and then pause during \refapi{PMIx_Init} of the spawned processes until either released by debugger modification of an appropriate variable or receipt of the \refconst{PMIX_DEBUGGER_RELEASE} event. The launcher (\ac{RM} or \ac{IL}) is responsible for generating the \refconst{PMIX_DEBUG_WAITING_FOR_NOTIFY} event when all processes have reached the pause point.
}
%
\declareAttribute{PMIX_DEBUG_WAIT_FOR_NOTIFY}{"pmix.dbg.notify"}{bool}{
Included in either the \refstruct{pmix_info_t} array in a \refstruct{pmix_app_t} description (if the directive applies only to that application) or in the \emph{job_info} array if it applies to all applications in the given spawn request. Indicates that the specified application is being spawned under a debugger. The resulting application processes are to notify their server (by generating
the \refconst{PMIX_DEBUG_WAITING_FOR_NOTIFY} event) when they reach some application-determined location and pause at that point until either released by debugger modification of an appropriate variable or receipt of the \refconst{PMIX_DEBUGGER_RELEASE} event. The launcher (\ac{RM} or \ac{IL}) is responsible for generating the \refconst{PMIX_DEBUG_WAITING_FOR_NOTIFY} event when all processes have indicated they are at the pause point.
}
%
\declareAttributeNEW{PMIX_DEBUG_TARGET}{"pmix.dbg.tgt"}{pmix_proc_t*}{
Identifier of process(es) to be debugged - a rank of \refconst{PMIX_RANK_WILDCARD} indicates that all processes in the specified namespace are to be included.
}
%
\declareAttribute{PMIX_DEBUGGER_DAEMONS}{"pmix.debugger"}{bool}{
Included in the \refstruct{pmix_info_t} array of a \refstruct{pmix_app_t}, this attribute declares that the application consists of debugger daemons and shall be governed accordingly. If used as the sole \refstruct{pmix_app_t} in a \refapi{PMIx_Spawn} request, then the \refattr{PMIX_DEBUG_TARGET} attribute must also be provided (in either the \emph{job_info} or in the \emph{info} array of the \refstruct{pmix_app_t}) to identify the namespace to be debugged so that the launcher can determine where to place the spawned daemons. If neither \refattr{PMIX_DEBUG_DAEMONS_PER_PROC} nor \refattr{PMIX_DEBUG_DAEMONS_PER_NODE} is specified, then the launcher shall default to a placement policy of one daemon per process in the target job.
}
%
\declareAttribute{PMIX_COSPAWN_APP}{"pmix.cospawn"}{bool}{
Designated application is to be spawned as a disconnected job - i.e., the launcher shall not include the application in any of the job-level values (e.g., \refattr{PMIX_RANK} within the job) provided to any other application process generated by the same spawn request. Typically used to cospawn debugger daemons alongside an application.
}
%
\declareAttributeNEW{PMIX_DEBUG_DAEMONS_PER_PROC}{"pmix.dbg.dpproc"}{uint16_t}{
Number of debugger daemons to be spawned per application process. The launcher
is to pass the identifier of the namespace to be debugged by including the
\refattr{PMIX_DEBUG_TARGET} attribute in the daemon's job-level information. The debugger daemons spawned on a given node are responsible for
self-determining their specific target process(es) - e.g., by referencing
their own \refattr{PMIX_LOCAL_RANK} in the daemon debugger job versus the
corresponding \refattr{PMIX_LOCAL_RANK} of the target processes on the node.
}
%
\declareAttributeNEW{PMIX_DEBUG_DAEMONS_PER_NODE}{"pmix.dbg.dpnd"}{uint16_t}{
Number of debugger daemons to be spawned on each node where the target job is
executing. The launcher is to pass the identifier of the namespace to be
debugged by including the \refattr{PMIX_DEBUG_TARGET} attribute in the daemon's
job-level information. The debugger daemons spawned on a given node are
responsible for self-determining their specific target process(es) - e.g., by
referencing their own \refattr{PMIX_LOCAL_RANK} in the daemon debugger job
versus the corresponding \refattr{PMIX_LOCAL_RANK} of the target processes on
the node.
}
%
\declareAttribute{PMIX_QUERY_PROC_TABLE}{"pmix.qry.ptable"}{char*}{
Returns a (\refstruct{pmix_data_array_t}) array of \refstruct{pmix_proc_info_t}, one entry for each process in the specified namespace, ordered by process job rank. REQUIRED QUALIFIER: \refattr{PMIX_NSPACE} indicating the namespace whose process table is being queried.
}
%
\declareAttribute{PMIX_QUERY_LOCAL_PROC_TABLE}{"pmix.qry.lptable"}{char*}{
Returns a (\refstruct{pmix_data_array_t}) array of \refstruct{pmix_proc_info_t}, one entry for each process in the specified namespace executing on the same node as the requester, ordered by process job rank. REQUIRED QUALIFIER: \refattr{PMIX_NSPACE} indicating the namespace whose local process table is being queried. OPTIONAL QUALIFIER: \refattr{PMIX_HOSTNAME} indicating the host whose local process table is being queried. By default, the query assumes that the host upon which the request was made is to be used.
}


%%%%%%%%%%%%%%%%%%%%%%%%%%%%%%%%%%%%%%%%%%%%%%%%%
%%%%%%%%%%%%%%%%%%%%%%%%%%%%%%%%%%%%%%%%%%%%%%%%%
\section{Tool-Specific APIs}
\label{chap:api_tools:apis}

\ac{PMIx}-based tools automatically have access to all \ac{PMIx} client functions. Tools designated as a \emph{launcher} or a \emph{server} will also have access to all \ac{PMIx} server functions. There are, however, an additional set of functions (described in this section) that are specific to a \ac{PMIx} tool. Access to those functions require use of the tool initialization routine.

%%%%%%%%%%%%%%%%%%%%%%%%%%%%%%%%%%%%%%%%%%%%%%%%%
\subsection{\code{PMIx_tool_init}}
\declareapi{PMIx_tool_init}

%%%%
\summary

Initialize the \ac{PMIx} library for operating as a tool, optionally connecting to a specified \ac{PMIx} server.

%%%%
\format

\versionMarker{2.0}
\cspecificstart
\begin{codepar}
pmix_status_t
PMIx_tool_init(pmix_proc_t *proc,
               pmix_info_t info[], size_t ninfo);
\end{codepar}
\cspecificend

\begin{arglist}
\arginout{proc}{\refstruct{pmix_proc_t} structure (handle)}
\argin{info}{Array of \refstruct{pmix_info_t} structures (array of handles)}
\argin{ninfo}{Number of elements in the \refarg{info} array (\code{size_t})}
\end{arglist}

Returns \refconst{PMIX_SUCCESS} or a negative value corresponding to a PMIx error constant.

\reqattrstart
The following attributes are required to be supported by all \ac{PMIx} libraries:

\pasteAttributeItem{PMIX_TOOL_NSPACE}
\pasteAttributeItem{PMIX_TOOL_RANK}
\pasteAttributeItem{PMIX_TOOL_DO_NOT_CONNECT}
\pasteAttributeItem{PMIX_TOOL_ATTACHMENT_FILE}
\pasteAttributeItem{PMIX_SERVER_URI}
\pasteAttributeItem{PMIX_TCP_URI}
\pasteAttributeItem{PMIX_SERVER_PIDINFO}
\pasteAttributeItem{PMIX_SERVER_NSPACE}
\pasteAttributeItem{PMIX_CONNECT_TO_SYSTEM}
\pasteAttributeItem{PMIX_CONNECT_SYSTEM_FIRST}

\reqattrend

\optattrstart
The following attributes are optional for implementers of \ac{PMIx} libraries:

\pasteAttributeItem{PMIX_CONNECT_RETRY_DELAY}
\pasteAttributeItem{PMIX_CONNECT_MAX_RETRIES}
\pasteAttributeItemBegin{PMIX_SOCKET_MODE} If the library supports socket connections, this attribute may be supported for setting the socket mode.
\pasteAttributeItemEnd{}
\pasteAttributeItemBegin{PMIX_TCP_REPORT_URI} If the library supports TCP socket connections, this attribute may be supported for reporting the URI.
\pasteAttributeItemEnd{}
\pasteAttributeItemBegin{PMIX_TCP_IF_INCLUDE} If the library supports TCP socket connections, this attribute may be supported for specifying the interfaces to be used.
\pasteAttributeItemEnd{}
\pasteAttributeItemBegin{PMIX_TCP_IF_EXCLUDE} If the library supports TCP socket connections, this attribute may be supported for specifying the interfaces that are \textit{not} to be used.
\pasteAttributeItemEnd{}
\pasteAttributeItemBegin{PMIX_TCP_IPV4_PORT} If the library supports IPV4 connections, this attribute may be supported for specifying the port to be used.
\pasteAttributeItemEnd{}
\pasteAttributeItemBegin{PMIX_TCP_IPV6_PORT} If the library supports IPV6 connections, this attribute may be supported for specifying the port to be used.
\pasteAttributeItemEnd{}
\pasteAttributeItemBegin{PMIX_TCP_DISABLE_IPV4} If the library supports IPV4 connections, this attribute may be supported for disabling it.
\pasteAttributeItemEnd{}
\pasteAttributeItemBegin{PMIX_TCP_DISABLE_IPV6} If the library supports IPV6 connections, this attribute may be supported for disabling it.
\pasteAttributeItemEnd{}
\pasteAttributeItem{PMIX_EXTERNAL_PROGRESS}
\pasteAttributeItem{PMIX_EVENT_BASE}
\pasteAttributeItem{PMIX_IOF_LOCAL_OUTPUT}

\optattrend

%%%%
\descr

Initialize the \ac{PMIx} tool, returning the process identifier assigned to this tool in the provided \refstruct{pmix_proc_t} struct. The \refarg{info} array is used to pass user requests pertaining to the initialization and subsequent operations. Passing a \code{NULL} value for the array pointer is supported if no directives are desired.

If called with the \refattr{PMIX_TOOL_DO_NOT_CONNECT} attribute, the \ac{PMIx} tool library will fully initialize but not attempt to connect to a \ac{PMIx} server. The tool can connect to a server at a later point in time, if desired, by calling the \refapi{PMIx_tool_attach_to_server} function. If provided, the \refarg{proc} structure will be set to a zero-length namespace and a rank of \refconst{PMIX_RANK_UNDEF} unless the \refattr{PMIX_TOOL_NSPACE} and \refattr{PMIX_TOOL_RANK} attributes are included in the \refarg{info} array.

In all other cases, the \ac{PMIx} tool library will automatically attempt to connect to a \ac{PMIx} server according to the precedence chain described in Section \ref{chap:api_tools:cnct}. If successful, the function will return \refconst{PMIX_SUCCESS} and will fill the process structure (if provided) with the assigned namespace and rank of the tool. The server to which the tool connects will be designated its \emph{primary} server. Note that each connection attempt in the above precedence chain will retry (with delay between each retry) a number of times according to the values of the corresponding attributes.

Note that the \ac{PMIx} tool library is referenced counted, and so multiple calls to \refapi{PMIx_tool_init} are allowed. If the tool is not connected to any server when this \ac{API} is called, then the tool will attempt to connect to a server unless the \refattr{PMIX_TOOL_DO_NOT_CONNECT} is included in the call to \ac{API}.


%%%%%%%%%%%%%%%%%%%%%%%%%%%%%%%%%%%%%%%%%%%%%%%%%
\subsection{\code{PMIx_tool_finalize}}
\declareapi{PMIx_tool_finalize}

%%%%
\summary

Finalize the \ac{PMIx} tool library.

%%%%
\format

\versionMarker{2.0}
\cspecificstart
\begin{codepar}
pmix_status_t
PMIx_tool_finalize(void);
\end{codepar}
\cspecificend

Returns \refconst{PMIX_SUCCESS} or a negative value corresponding to a \ac{PMIx} error constant.

%%%%
\descr

Finalize the \ac{PMIx} tool library, closing all existing connections to
servers.
An error code will be returned if, for some reason, a connection cannot be
cleanly terminated --- in such cases, the connection is dropped. Upon
detecting loss of the connection, the \ac{PMIx} server shall cleanup all
associated records of the tool.


%%%%%%%%%%%%%%%%%%%%%%%%%%%%%%%%%%%%%%%%%%%%%%%%%
\subsection{\code{PMIx_tool_disconnect}}
\declareapi{PMIx_tool_disconnect}

%%%%
\summary

Disconnect the \ac{PMIx} tool from the specified server connection while leaving the tool library initialized.

%%%%
\format

\versionMarker{4.0}
\cspecificstart
\begin{codepar}
pmix_status_t
PMIx_tool_disconnect(const pmix_proc_t *server);
\end{codepar}
\cspecificend

\begin{arglist}
\argin{server}{\refstruct{pmix_proc_t} structure (handle)}
\end{arglist}

Returns \refconst{PMIX_SUCCESS} or a negative value corresponding to a PMIx error constant.

%%%%
\descr

Close the current connection to the specified server, if one has been made, while leaving the \ac{PMIx} library initialized. An error code will be returned if, for some reason, the connection cannot be cleanly terminated - in this case, the connection is dropped. In either case, the library will remain initialized.  Upon
detecting loss of the connection, the \ac{PMIx} server shall cleanup all
associated records of the tool.


Note that if the server being disconnected is the current \emph{primary} server, then all operations requiring support from a server will return the \refconst{PMIX_ERR_UNREACH} error until the tool either designates an existing connection to be the \emph{primary} server or, if no other connections exist, the tool establishes a connection to a \ac{PMIx} server.


%%%%%%%%%%%%%%%%%%%%%%%%%%%%%%%%%%%%%%%%%%%%%%%%%
\subsection{\code{PMIx_tool_attach_to_server}}
\declareapi{PMIx_tool_attach_to_server}

%%%%
\summary

Establish a connection to a \ac{PMIx} server.

%%%%
\format

\versionMarker{4.0}
\cspecificstart
\begin{codepar}
pmix_status_t
PMIx_tool_attach_to_server(pmix_proc_t *proc,
                           pmix_proc_t *server,
                           pmix_info_t info[],
                           size_t ninfo);
\end{codepar}
\cspecificend

\begin{arglist}
\arginout{proc}{Pointer to \refstruct{pmix_proc_t} structure (handle)}
\arginout{server}{Pointer to \refstruct{pmix_proc_t} structure (handle)}
\argin{info}{Array of \refstruct{pmix_info_t} structures (array of handles)}
\argin{ninfo}{Number of elements in the \refarg{info} array (\code{size_t})}
\end{arglist}

Returns \refconst{PMIX_SUCCESS} or a negative value corresponding to a PMIx error constant.

\reqattrstart
The following attributes are required to be supported by all \ac{PMIx} libraries:

\pasteAttributeItem{PMIX_TOOL_ATTACHMENT_FILE}
\pasteAttributeItem{PMIX_SERVER_URI}
\pasteAttributeItem{PMIX_TCP_URI}
\pasteAttributeItem{PMIX_SERVER_PIDINFO}
\pasteAttributeItem{PMIX_SERVER_NSPACE}
\pasteAttributeItem{PMIX_CONNECT_TO_SYSTEM}
\pasteAttributeItem{PMIX_CONNECT_SYSTEM_FIRST}
\pasteAttributeItem{PMIX_PRIMARY_SERVER}

\reqattrend

%%%%
\descr

Establish a connection to a server. This function can be called at any time by a \ac{PMIx} tool to create a new connection to a server. If a specific server is given and the tool is already attached to it, then the \ac{API} shall return \refconst{PMIX_SUCCESS} without taking any further action. In all other cases, the tool will attempt to discover a server using the method described in Section \ref{chap:api_tools:cnct}, ignoring all candidates to which it is already connected. The \refconst{PMIX_ERR_UNREACH} error shall be returned if no new connection is made.

The process identifier assigned to this tool is returned in the provided \refarg{proc} structure. Passing a value of \code{NULL} for the \refarg{proc} parameter is allowed if the user wishes solely to connect to a \ac{PMIx} server and does not require return of the identifier at that time.

The process identifier of the server to which the tool attached is returned in the \refarg{server} structure. Passing a value of \code{NULL} for the \refarg{proc} parameter is allowed if the user wishes solely to connect to a \ac{PMIx} server and does not require return of the identifier at that time.

Note that the \refattr{PMIX_PRIMARY_SERVER} attribute must be included in the
\refarg{info} array if the server being connected to is to become the primary
server, or a call to \refapi{PMIx_tool_set_server} must be provided immediately
after the call to this function.

\adviceimplstart
When a tool connects to a server that is under a different namespace manager (e.g., host \ac{RM}) from the prior server, the namespace in the identifier of the tool must remain unique in the new universe. If the namespace of the tool fails to meet this criteria in the new universe, then the new namespace manager is required to return an error and the connection attempt must fail.
\adviceimplend

\adviceuserstart
Some \ac{PMIx} implementations may not support connecting to a server that is not under the same namespace manager (e.g., host \ac{RM}) as the server to which the tool is currently connected.
\adviceuserend


%%%%%%%%%%%%%%%%%%%%%%%%%%%%%%%%%%%%%%%%%%%%%%%%%
\subsection{\code{PMIx_tool_get_servers}}
\declareapi{PMIx_tool_get_servers}

%%%%
\summary

Get an array containing the \refstruct{pmix_proc_t} process identifiers of all servers to which the tool is currently connected.

%%%%
\format

\versionMarker{4.0}
\cspecificstart
\begin{codepar}
pmix_status_t
PMIx_tool_get_servers(pmix_proc_t *servers[], size_t *nservers);
\end{codepar}
\cspecificend

\begin{arglist}
\argout{servers}{Address where the pointer to an array of \refstruct{pmix_proc_t} structures shall be returned (handle)}
\arginout{nservers}{Address where the number of elements in \refarg{servers} shall be returned (handle)}
\end{arglist}

Returns \refconst{PMIX_SUCCESS} or a negative value corresponding to a PMIx error constant.

%%%%
\descr

Return an array containing the \refstruct{pmix_proc_t} process identifiers of all servers to which the tool is currently connected. The process identifier of the current primary server shall be the first entry in the array, with the remaining entries in order of attachment from earliest to most recent.


%%%%%%%%%%%%%%%%%%%%%%%%%%%%%%%%%%%%%%%%%%%%%%%%%
\subsection{\code{PMIx_tool_set_server}}
\declareapi{PMIx_tool_set_server}

%%%%
\summary

Designate a server as the tool's \emph{primary} server.

%%%%
\format

\versionMarker{4.0}
\cspecificstart
\begin{codepar}
pmix_status_t
PMIx_tool_set_server(const pmix_proc_t *server,
                     pmix_info_t info[], size_t ninfo);
\end{codepar}
\cspecificend

\begin{arglist}
\argin{server}{\refstruct{pmix_proc_t} structure (handle)}
\argin{info}{Array of \refstruct{pmix_info_t} structures (array of handles)}
\argin{ninfo}{Number of elements in the \refarg{info} array (\code{size_t})}
\end{arglist}

Returns \refconst{PMIX_SUCCESS} or a negative value corresponding to a PMIx error constant.

\reqattrstart
The following attributes are required to be supported by all \ac{PMIx} libraries:

\pasteAttributeItem{PMIX_WAIT_FOR_CONNECTION}
\pasteAttributeItem{PMIX_TIMEOUT}

\reqattrend

%%%%
\descr

Designate the specified server to be the tool's \emph{primary} server for all subsequent \ac{API} calls.


%%%%%%%%%%%%%%%%%%%%%%%%%%%%%%%%%%%%%%%%%%%%%%%%%
\subsection{\code{PMIx_IOF_pull}}
\declareapi{PMIx_IOF_pull}

%%%%
\summary

Register to receive output forwarded from a set of remote processes.

%%%%
\format

\versionMarker{3.0}
\cspecificstart
\begin{codepar}
pmix_status_t
PMIx_IOF_pull(const pmix_proc_t procs[], size_t nprocs,
              const pmix_info_t directives[], size_t ndirs,
              pmix_iof_channel_t channel,
              pmix_iof_cbfunc_t cbfunc,
              pmix_hdlr_reg_cbfunc_t regcbfunc,
              void *regcbdata);
\end{codepar}
\cspecificend

\begin{arglist}
\argin{procs}{Array of proc structures identifying desired source processes (array of handles)}
\argin{nprocs}{Number of elements in the \refarg{procs} array (integer)}
\argin{directives}{Array of \refstruct{pmix_info_t} structures (array of handles)}
\argin{ndirs}{Number of elements in the \refarg{directives} array (integer)}
\argin{channel}{Bitmask of IO channels included in the request (\refstruct{pmix_iof_channel_t})}
\argin{cbfunc}{Callback function for delivering relevant output (\refapi{pmix_iof_cbfunc_t} function reference)}
\argin{regcbfunc}{Function to be called when registration is completed (\refapi{pmix_hdlr_reg_cbfunc_t} function reference)}
\argin{regcbdata}{Data to be passed to the \refarg{regcbfunc} callback function (memory reference)}
\end{arglist}

Returns \refconst{PMIX_SUCCESS} or a negative value corresponding to a PMIx error constant. In the event the function returns an error, the \refarg{regcbfunc} will \textit{not} be called.

\reqattrstart
The following attributes are required for \ac{PMIx} libraries that support \ac{IO} forwarding:

\pasteAttributeItem{PMIX_IOF_CACHE_SIZE}
\pasteAttributeItem{PMIX_IOF_DROP_OLDEST}
\pasteAttributeItem{PMIX_IOF_DROP_NEWEST}

\reqattrend

\optattrstart
The following attributes are optional for \ac{PMIx} libraries that support \ac{IO} forwarding:

\pasteAttributeItem{PMIX_IOF_BUFFERING_SIZE}
\pasteAttributeItem{PMIX_IOF_BUFFERING_TIME}
\pasteAttributeItem{PMIX_IOF_TAG_OUTPUT}
\pasteAttributeItem{PMIX_IOF_TIMESTAMP_OUTPUT}
\pasteAttributeItem{PMIX_IOF_XML_OUTPUT}

\optattrend

%%%%
\descr

Register to receive output forwarded from a set of remote processes.

\adviceuserstart
Providing a \code{NULL} function pointer for the \refarg{cbfunc} parameter will cause output for the indicated channels to be written to their corresponding \code{stdout}/\code{stderr} file descriptors. Use of \refconst{PMIX_RANK_WILDCARD} to specify all processes in a given namespace is supported but should be used carefully due to bandwidth and memory footprint considerations.
\adviceuserend


%%%%%%%%%%%%%%%%%%%%%%%%%%%%%%%%%%%%%%%%%%%%%%%%%
\subsection{\code{PMIx_IOF_deregister}}
\declareapi{PMIx_IOF_deregister}

%%%%
\summary

Deregister from output forwarded from a set of remote processes.

%%%%
\format

\versionMarker{3.0}
\cspecificstart
\begin{codepar}
pmix_status_t
PMIx_IOF_deregister(size_t iofhdlr,
                    const pmix_info_t directives[], size_t ndirs,
                    pmix_op_cbfunc_t cbfunc, void *cbdata);
\end{codepar}
\cspecificend

\begin{arglist}
\argin{iofhdlr}{Registration number returned from the \refapi{pmix_hdlr_reg_cbfunc_t} callback from the call to \refapi{PMIx_IOF_pull} (\code{size_t})}
\argin{directives}{Array of \refstruct{pmix_info_t} structures (array of handles)}
\argin{ndirs}{Number of elements in the \refarg{directives} array (integer)}
\argin{cbfunc}{Callback function to be called when deregistration has been completed. (function reference)}
\argin{cbdata}{Data to be passed to the \refarg{cbfunc} callback function (memory reference)}
\end{arglist}

Returns one of the following:

\begin{itemize}
    \item \refconst{PMIX_SUCCESS}, indicating that the request is being processed by the host environment - result will be returned in the provided \refarg{cbfunc}. Note that the library \emph{must not} invoke the callback function prior to returning from the \ac{API}.
    \item \refconst{PMIX_OPERATION_SUCCEEDED}, indicating that the request was immediately processed and returned \textit{success} - the \refarg{cbfunc} will \textit{not} be called
    \item a PMIx error constant indicating either an error in the input or that the request was immediately processed and failed - the \refarg{cbfunc} will \textit{not} be called
\end{itemize}

%%%%
\descr

Deregister from output forwarded from a set of remote processes.

\adviceimplstart
Any currently buffered \ac{IO} should be flushed upon receipt of a deregistration request. All received \ac{IO} after receipt of the request shall be discarded.
\adviceimplend


%%%%%%%%%%%%%%%%%%%%%%%%%%%%%%%%%%%%%%%%%%%%%%%%%
\subsection{\code{PMIx_IOF_push}}
\declareapi{PMIx_IOF_push}

%%%%
\summary

Push data collected locally (typically from \code{stdin} or a file) to \code{stdin} of the target recipients.

%%%%
\format

\versionMarker{3.0}
\cspecificstart
\begin{codepar}
pmix_status_t
PMIx_IOF_push(const pmix_proc_t targets[], size_t ntargets,
              pmix_byte_object_t *bo,
              const pmix_info_t directives[], size_t ndirs,
              pmix_op_cbfunc_t cbfunc, void *cbdata);
\end{codepar}
\cspecificend

\begin{arglist}
\argin{targets}{Array of proc structures identifying desired target processes (array of handles)}
\argin{ntargets}{Number of elements in the \refarg{targets} array (integer)}
\argin{bo}{Pointer to \refstruct{pmix_byte_object_t} containing the payload to be delivered (handle)}
\argin{directives}{Array of \refstruct{pmix_info_t} structures (array of handles)}
\argin{ndirs}{Number of elements in the \refarg{directives} array (integer)}
\argin{directives}{Array of \refstruct{pmix_info_t} structures (array of handles)}
\argin{cbfunc}{Callback function to be called when operation has been completed. (\refapi{pmix_op_cbfunc_t} function reference)}
\argin{cbdata}{Data to be passed to the \refarg{cbfunc} callback function (memory reference)}
\end{arglist}

Returns one of the following:

\begin{itemize}
    \item \refconst{PMIX_SUCCESS}, indicating that the request is being processed by the host environment - result will be returned in the provided \refarg{cbfunc}. Note that the library \emph{must not} invoke the callback function prior to returning from the \ac{API}.
    \item \refconst{PMIX_OPERATION_SUCCEEDED}, indicating that the request was immediately processed and returned \textit{success} - the \refarg{cbfunc} will \textit{not} be called.
    \item a PMIx error constant indicating either an error in the input or that the request was immediately processed and failed - the \refarg{cbfunc} will \textit{not} be called.
\end{itemize}

\reqattrstart
The following attributes are required for \ac{PMIx} libraries that support \ac{IO} forwarding:

\pasteAttributeItem{PMIX_IOF_CACHE_SIZE}
\pasteAttributeItem{PMIX_IOF_DROP_OLDEST}
\pasteAttributeItem{PMIX_IOF_DROP_NEWEST}

\reqattrend

\optattrstart
The following attributes are optional for \ac{PMIx} libraries that support \ac{IO} forwarding:

\pasteAttributeItem{PMIX_IOF_BUFFERING_SIZE}
\pasteAttributeItem{PMIX_IOF_BUFFERING_TIME}
\pasteAttributeItem{PMIX_IOF_PUSH_STDIN}

\optattrend

%%%%
\descr

Called either to:

\begin{itemize}
    \item push data collected by the caller themselves (typically from \code{stdin} or a file) to \code{stdin} of the target recipients;
    \item request that the \ac{PMIx} library automatically collect and push the \code{stdin} of the caller to the target recipients; or
    \item indicate that automatic collection and transmittal of \code{stdin} is to stop
\end{itemize}

\adviceuserstart
Execution of the \refarg{cbfunc} callback function serves as notice that the \ac{PMIx} library no longer requires the caller to maintain the \refarg{bo} data object - it does \textit{not} indicate delivery of the payload to the targets. Use of \refconst{PMIX_RANK_WILDCARD} to specify all processes in a given namespace is supported but should be used carefully due to bandwidth and memory footprint considerations.
\adviceuserend

%%%%%%%%%%%%%%%%%%%%%%%%%%%%%%%%%%%%%%%%%%%%%%%%%


%
% Appendix
%
    \setcounter{chapter}{0}  % restart chapter numbering with "letter A"
    \renewcommand{\thechapter}{\Alph{chapter}}%
    \appendix

    % Python bindings
    %%%%%%%%%%%%%%%%%%%%%%%%%%%%%%%%%%%%%%%%%%%%%%%%%
% Appendix: Python bindings
%%%%%%%%%%%%%%%%%%%%%%%%%%%%%%%%%%%%%%%%%%%%%%%%%
\chapter{Python Bindings}
\label{app:python}

While the \ac{PMIx} Standard is defined in terms of C-based \acp{API}, there is no intent to limit the use of \ac{PMIx} to that specific language. Support for other languages is captured in the Standard by describing their equivalent syntax for the \ac{PMIx} \acp{API} and native forms for the \ac{PMIx} datatypes. This Appendix specifically deals with Python interfaces, beginning with a review of the \ac{PMIx} datatypes. Support is restricted to Python 3 and above - i.e., the Python bindings do not support Python 2.

Note: the \ac{PMIx} \acp{API} have been loosely collected into three Python classes based on their \ac{PMIx} “class” (i.e., client, server, and tool). All processes have access to a basic set of the \acp{API}, and therefore those have been included in the “client” class. Servers can utilize any of those functions plus a set focused on operations not commonly executed by an application process. Finally, tools can also act as servers but have their own initialization function.


%%%%%%%%%%%%%%%%%%%%%%%%%%%%%%%%%%%%%%%%%%%%%%%%%
\section{Design Considerations}
\label{app:python:design}

Several issues arose during design of the Python bindings:

%%%%%%%%%%%%%%%%%%%%%%%%%%%%%%%%%%%%%%%%%%%%%%%%%
\subsection{Error Codes vs Python Exceptions}
\label{app:python:exceptions}

The C programming language reports errors through the return of the corresponding integer status codes. \ac{PMIx} has defined a range of negative values for this purpose. However, Python has the option of raising \emph{exceptions} that effectively operate as interrupts that can be trapped if the program appropriately tests for them. The \ac{PMIx} Python bindings opted to follow the C-based standard and return \ac{PMIx} status codes in lieu of raising exceptions as this method was considered more consistent for those working in both domains.

%%%%%%%%%%%%%%%%%%%%%%%%%%%%%%%%%%%%%%%%%%%%%%%%%
\subsection{Representation of Structured Data}
\label{app:python:rep}

\ac{PMIx} utilizes a number of C-language structures to efficiently bundle related information. For example, the \ac{PMIx} process identifier is represented as a struct containing a character array for the namespace and a 32-bit unsigned integer for the process rank. There are several options for translating such objects to Python – e.g., the \ac{PMIx} process identifier could be represented as a two-element tuple (nspace, rank) or as a dictionary {‘nspace’: name, ‘rank’: 0}. Exploration found no discernible benefit to either representation, nor was any clearly identifiable rationale developed that would lead a user to expect one versus the other for a given \ac{PMIx} data type. Consistency in the translation (i.e., exclusively using tuple or dictionary) appeared to be the most important criterion. Hence, the decision was made to express all complex datatypes as Python dictionaries.

%%%%%%%%%%%%%%%%%%%%%%%%%%%%%%%%%%%%%%%%%%%%
\section{Datatype Definitions}
\label{app:python:types}

\ac{PMIx} defines a number of datatypes comprised of fixed-size character arrays, restricted range integers (e.g., uint32_t), and structures. Each datatype is represented by a named unsigned 16-bit integer (\code{uint16_t}) constant. Users are advised to use the named \ac{PMIx} constants for indicating datatypes instead of integer values to ensure compatibility with future PMIx versions.

With only a few exceptions, the C-based \ac{PMIx} datatypes defined in \chapterref{chap:struct} directly translate to Python. However, Python lacks the size-specific value definitions of C (e.g., \code{uint8_t}) and thus some care must be taken to protect against overflow/underflow situations when moving between the languages. Python bindings that accept values including \ac{PMIx} datatypes shall therefore have the datatype and associated value checked for compatibility with their \ac{PMIx}-defined equivalents, returning an error if:

\begin{itemize}
    \item datatypes not defined by \ac{PMIx} are encountered
    \item provided values fall outside the range of the C-equivalent definition - e.g., if a value identified as \refconst{PMIX_UINT8} lies outside the \code{uint8_t}range
\end{itemize}

Note that explicit labeling of \ac{PMIx} datatype, even when Python itself doesn’t care, is often required for the Python bindings to know how to properly interpret and label the provided value when passing it to the \ac{PMIx} library.

Table~\ref{app:python:ctopy} lists the correspondence between datatypes in the two languages.

\begin{landscape}
\begin{small}
    \begin{longtable}{ | p{4.5cm} | p{4cm} | p{3cm} | p{5.5cm} |}
        \caption{C-to-Python Datatype Correspondence} \label{app:python:ctopy} \\
        \hline
        C-Definition & PMIx Name & Python Definition & Notes \\ \hline
        \endhead
        \code{bool} & PMIX_BOOL & boolean & \\ \hline
        \code{byte} & PMIX_BYTE & A single element byte array (i.e., a byte array of length one) & \\ \hline
        \code{char*} & PMIX_STRING & string & \\ \hline
        \code{size_t} & PMIX_SIZE & integer & \\ \hline
        \code{pid_t} & PMIX_PID & integer & value shall be limited to the \code{uint32_t} range \\ \hline
        \code{int, int8_t, int16_t, int32_t, int64_t} & PMIX_INT, PMIX_INT8, PMIX_INT16, PMIX_INT32, PMIX_INT64 & integer & value shall be limited to its corresponding range \\ \hline
        \code{uint, uint8_t, uint16_t, uint32_t, uint64_t} & PMIX_UINT, PMIX_UINT8, PMIX_UINT16, PMIX_UINT32, PMIX_UINT64 & integer & value shall be limited to its corresponding range \\ \hline
        \code{float, double} & PMIX_FLOAT, PMIX_DOUBLE & float & value shall be limited to its corresponding range \\ \hline
        \code{struct timeval} & PMIX_TIMEVAL & \{'sec': sec, 'usec': microsec\} & each field is an integer value \\ \hline
        \code{time_t} & PMIX_TIME & integer & limited to positive values \\ \hline
        \refstruct{pmix_data_type_t} & PMIX_DATA_TYPE & integer & value shall be limited to the \code{uint16_t} range \\ \hline
        \refstruct{pmix_status_t} & PMIX_STATUS & integer & \\ \hline
        \refstruct{pmix_key_t} & N/A & \pylabel{key}string & The string's length shall be limited to one less than the size of the \refstruct{pmix_key_t} array (to reserve space for the terminating \code{NULL})  \\ \hline
        \refstruct{pmix_nspace_t} & N/A & \pylabel{nspace}string & The string's length shall be limited to one less than the size of the \refstruct{pmix_nspace_t} array (to reserve space for the terminating \code{NULL})  \\ \hline
        \refstruct{pmix_rank_t} & PMIX_PROC_RANK & \pylabel{rank}integer & value shall be limited to the \code{uint32_t} range excepting the reserved values near \code{UINT32_MAX} \\ \hline
        \refstruct{pmix_proc_t} & PMIX_PROC & \pylabel{proc}\{'nspace': nspace, 'rank': rank\} & \refarg{nspace} is a Python string and \refarg{rank} is an integer value. The \refarg{nspace} string's length shall be limited to one less than the size of the \refstruct{pmix_nspace_t} array (to reserve space for the terminating \code{NULL}), and the \refarg{rank} value shall conform to the constraints associated with \refstruct{pmix_rank_t} \\ \hline
        \refstruct{pmix_byte_object_t} & PMIX_BYTE_OBJECT & \pylabel{byteobject}\{'bytes': bytes, 'size': size\} & \refarg{bytes} is a Python byte array and \refarg{size} is the integer number of bytes in that array. \\ \hline
        \refstruct{pmix_persistence_t} & PMIX_PERSISTENCE & integer & value shall be limited to the \code{uint8_t} range \\ \hline
        \refstruct{pmix_scope_t} & PMIX_SCOPE & integer & value shall be limited to the \code{uint8_t} range \\ \hline
        \refstruct{pmix_data_range_t} & PMIX_RANGE & \pylabel{range}integer & value shall be limited to the \code{uint8_t} range \\ \hline
        \refstruct{pmix_proc_state_t} & PMIX_PROC_STATE & integer & value shall be limited to the \code{uint8_t} range \\ \hline
        \refstruct{pmix_proc_info_t} & PMIX_PROC_INFO & \{'proc': \{'nspace': nspace, 'rank': rank\}, 'hostname': hostname, 'executable': executable, 'pid': pid, 'exitcode': exitcode, 'state': state\} & \refarg{proc} is a Python \refpy{proc} dictionary; \refarg{hostname} and \refarg{executable} are Python strings; and \refarg{pid}, \refarg{exitcode}, and \refarg{state} are Python integers \\ \hline
        \refstruct{pmix_data_array_t} & PMIX_DATA_ARRAY & \pylabel{array}\{'type': type, 'array': array\} & \refarg{type} is the \ac{PMIx} type of object in the array and \refarg{array} is a Python \emph{list} containing the individual array elements. Note that \refarg{array} can consist of \emph{any} \ac{PMIx} types, including (for example) a Python \refpy{info} object that itself contains an \refpy{array} value \\ \hline
        \refstruct{pmix_info_directives_t}  & PMIX_INFO_DIRECTIVES & \pylabel{info directives}bitarray & 32-bit array \\ \hline
        \refstruct{pmix_alloc_directive_t} & PMIX_ALLOC_DIRECTIVE & \pylabel{allocdir}integer & value shall be limited to the \code{uint8_t} range \\ \hline
        \refstruct{pmix_iof_channel_t} & PMIX_IOF_CHANNEL & \pylabel{channel}bitarray & 16-bit array \\ \hline
        \refstruct{pmix_envar_t} & PMIX_ENVAR & \{'envar': envar, 'value': value, 'separator': separator\} & \refarg{envar} and \refarg{value} are Python strings, and \refarg{separator} a single-character Python string \\ \hline
        \refstruct{pmix_value_t} & PMIX_VALUE & \pylabel{value}\{'value': value, 'val_type': type\} & \refarg{type} is the \ac{PMIx} datatype of \refarg{value}, and \refarg{value} is the associated value expressed in the appropriate Python form for the specified datatype  \\ \hline
        \refstruct{pmix_info_t} & PMIX_INFO & \pylabel{info}\{'key': key, 'flags': flags, value': value, 'val_type': type\} & \refarg{key} is a Python string \refpy{key}, \refarg{flags} is an \refpy{info directives} value, \refarg{type} is the \ac{PMIx} datatype of \refarg{value}, and \refarg{value} is the associated value expressed in the appropriate Python form for the specified datatype \\ \hline
        \refstruct{pmix_pdata_t} & PMIX_PDATA & \pylabel{pdata}\{'proc': \{'nspace': nspace, 'rank': rank\}, 'key': key, 'value': value, 'val_type': type\} & \refarg{proc} is a Python \refpy{proc} dictionary; \refarg{key} is a Python string \refpy{key}; \refarg{type} is the \ac{PMIx} datatype of \refarg{value}; and \refarg{value} is the associated value expressed in the appropriate Python form for the specified datatype  \\ \hline
        \refstruct{pmix_app_t} & PMIX_APP & \pylabel{app}\{'cmd': cmd, 'argv': [argv], 'env': [env], 'maxprocs': maxprocs, 'info': [info]\} & \refarg{cmd} is a Python string; \refarg{argv} and \refarg{env} are Python \emph{lists} containing Python strings; \refarg{maxprocs} is an integer; and \refarg{info} is a Python \emph{list} of \refpy{info} values   \\ \hline
        \refstruct{pmix_query_t} & PMIX_QUERY & \pylabel{query}\{'keys': [keys], 'qualifiers': [info]\} & \refarg{keys} is a Python \emph{list} of Python strings, and \refarg{qualifiers} is a Python \emph{list} of \refpy{info} values \\ \hline
        \refstruct{pmix_regattr_t} & PMIX_REGATTR & \pylabel{regattr}\{'name': name, 'key': key, 'type': type, 'info': [info], 'description': [desc]\} & \refarg{name} and \refarg{string} are Python strings; \refarg{type} is the \ac{PMIx} datatype for the attribute's value; \refarg{info} is a Python \emph{list} of \refpy{info} values; and \refarg{description} is a list of Python strings describing the attribute  \\ \hline
        \refstruct{pmix_coord_t} & PMIX_COORD & \pylabel{coord}\{'fabric': fabric, 'plane': plane, 'view': view, 'coord': [coords]\} & \refarg{fabric} and \refarg{plane} are Python strings; \refarg{view} is the \refstruct{pmix_coord_view_t} of the coordinate; and \refarg{coord} is a list of integer coordinates, one for each dimension of the fabric \\ \hline
        \refstruct{pmix_job_state_t} & PMIX_JOB_STATE & integer & value shall be limited to the \code{uint8_t} range \\ \hline
        \refstruct{pmix_link_state_t} & PMIX_LINK_STATE & integer & value shall be limited to the \code{uint8_t} range \\ \hline
        \refstruct{pmix_cpuset_t} & N/A & \pylabel{cpuset}\{'source': source, 'cpus': bitmap\} & \refarg{source} is a string name of the library that created the cpuset; and \refarg{cpus} is a bitarray containing the cpuset \\ \hline
        \refstruct{pmix_locality_t} & N/A & \pylabel{locality}bitarray & 16-bit array containing the relative locality of the specified local process \\ \hline
        \refstruct{pmix_fabric_t} & N/A & \pylabel{fabric}\{'name': name, 'index': idx, 'info': [info]\} & \refarg{name} is the string name assigned to the fabric; \refarg{index} is the integer ID assigned to the fabric; \refarg{info} is a list of \refpy{info} describing the fabric \\ \hline
    \end{longtable}
\end{small}
\end{landscape}

%%%%%%%%%%%%%%%%%%%%%%%%%%%%%%%%%%%%%%%%%%%%%%%%%
\subsection{Example}
Converting a C-based program to its Python equivalent requires translation of the relevant datatypes as well as use of the appropriate \ac{API} form. An example small program may help illustrate the changes. Consider the following C-based program snippet:

\cspecificstart
\begin{codepar}

#include <pmix.h>
...

pmix_info_t info[2];

PMIX_INFO_LOAD(&info[0], PMIX_PROGRAMMING_MODEL, "TEST", PMIX_STRING)
PMIX_INFO_LOAD(&info[1], PMIX_MODEL_LIBRARY_NAME, "PMIX", PMIX_STRING)

rc = PMIx_Init(&myproc, info, 2);

PMIX_INFO_DESTRUCT(&info[0]);  // free the copied string
PMIX_INFO_DESTRUCT(&info[1]);  // free the copied string
\end{codepar}
\cspecificend

Moving to the Python version requires that the \refstruct{pmix_info_t} be translated to the Python \refpy{info} equivalent, and that the returned information be captured in the return parameters as opposed to a pointer parameter in the function call, as shown below:

\pyspecificstart
\begin{codepar}
import pmix
...

myclient = PMIxClient()
info = [\{'key':PMIX_PROGRAMMING_MODEL,
          'value':'TEST', 'val_type':PMIX_STRING\},
        \{'key':PMIX_MODEL_LIBRARY_NAME,
          'value':'PMIX', 'val_type':PMIX_STRING\}]
(rc,myproc) = myclient.init(info)
\end{codepar}
\pyspecificend

Note the use of the \refconst{PMIX_STRING} identifier to ensure the Python bindings interpret the provided string value as a \ac{PMIx} "string" and not an array of bytes.


%%%%%%%%%%%%%%%%%%%%%%%%%%%%%%%%%%%%%%%%%%%%%%%%%
%%%%%%%%%%%%%%%%%%%%%%%%%%%%%%%%%%%%%%%%%%%%%%%%%
\section{Callback Function Definitions}
\label{app:python:fns}

%%%%%%%%%%%%%%%%%%%%%%%%%%%%%%%%%%%%%%%%%%%%%%%%%
\subsection{IOF Delivery Function}
\pylabel{iofcbfunc}

%%%%
\summary

Callback function for delivering forwarded \ac{IO} to a process

%%%%
\format

\versionMarker{4.0}
\pyspecificstart
\begin{codepar}
def iofcbfunc(iofhdlr:integer, channel:bitarray,
              source:dict, payload:dict, info:list)
\end{codepar}
\pyspecificend

\begin{arglist}
\argin{iofhdlr}{Registration number of the handler being invoked (integer)}
\argin{channel}{Python \refpy{channel} 16-bit bitarray identifying the channel the data arrived on (bitarray)}
\argin{source}{Python \refpy{proc} identifying the namespace/rank of the process that generated the data (dict)}
\argin{payload}{Python \refpy{byteobject} containing the data (dict)}
\argin{info}{List of Python \refpy{info} provided by the source containing metadata about the payload. This could include \refattr{PMIX_IOF_COMPLETE} (list)}
\end{arglist}

Returns: nothing

See \refapi{pmix_iof_cbfunc_t} for details


%%%%%%%%%%%%%%%%%%%%%%%%%%%%%%%%%%%%%%%%%%%%%%%%%
\subsection{Event Handler}
\pylabel{evhandler}

%%%%
\summary

Callback function for event handlers

%%%%
\format

\versionMarker{4.0}
\pyspecificstart
\begin{codepar}
def evhandler(evhdlr:integer, status:integer,
              source:dict, info:list, results:list)
\end{codepar}
\pyspecificend

\begin{arglist}
\argin{iofhdlr}{Registration number of the handler being invoked (integer)}
\argin{status}{Status associated with the operation (integer)}
\argin{source}{Python \refpy{proc} identifying the namespace/rank of the process that generated the event (dict)}
\argin{info}{List of Python \refpy{info} provided by the source containing metadata about the event (list)}
\argin{results}{List of Python \refpy{info} containing the aggregated results of all prior evhandlers (list)}
\end{arglist}

Returns:
\begin{itemize}
    \item \refarg{rc} - Status returned by the event handler's operation (integer)
    \item \refarg{results} - List of Python \refpy{info} containing results from this event handler's operation on the event (list)
\end{itemize}

See \refapi{pmix_notification_fn_t} for details


%%%%%%%%%%%%%%%%%%%%%%%%%%%%%%%%%%%%%%%%%%%%%%%%%
\subsection{Server Module Functions}
\pylabel{server module}

The following definitions represent functions that may be provided to the \ac{PMIx} server library at time of initialization for servicing of client requests. Module functions that are not provided default to returning "not supported" to the caller.


%%%%%%%%%%%%%%%%%%%%%%%%%%%%%%%%%%%%%%%%%%%%%%%%%
\subsubsection{Client Connected}

%%%%
\summary

Notify the host server that a client connected to this server.

%%%%
\format

\versionMarker{4.0}
\pyspecificstart
\begin{codepar}
def clientconnected2(proc:dict is not None, info:list)
\end{codepar}
\pyspecificend

\begin{arglist}
\argin{proc}{Python \refpy{proc} identifying the namespace/rank of the process that connected (dict)}
\argin{info}{list of Python \refpy{info} containing information about the process (list)}
\end{arglist}

Returns:
\begin{itemize}
    \item \refarg{rc} - \refconst{PMIX_SUCCESS} or a \ac{PMIx} error code indicating the connection should be rejected (integer)
\end{itemize}

See \refapi{pmix_server_client_connected2_fn_t} for details


%%%%%%%%%%%%%%%%%%%%%%%%%%%%%%%%%%%%%%%%%%%%%%%%%
\subsubsection{Client Finalized}

%%%%
\summary

Notify the host environment that a client called \refapi{PMIx_Finalize}.

%%%%
\format

\versionMarker{4.0}
\pyspecificstart
\begin{codepar}
def clientfinalized(proc:dict is not None):
\end{codepar}
\pyspecificend

\begin{arglist}
\argin{proc}{Python \refpy{proc} identifying the namespace/rank of the process that finalized (dict)}
\end{arglist}

Returns: nothing

See \refapi{pmix_server_client_finalized_fn_t} for details


%%%%%%%%%%%%%%%%%%%%%%%%%%%%%%%%%%%%%%%%%%%%%%%%%
\subsubsection{Client Aborted}

%%%%
\summary

Notify the host environment that a local client called \refapi{PMIx_Abort}.

%%%%
\format

\versionMarker{4.0}
\pyspecificstart
\begin{codepar}
def clientaborted(args:dict is not None)
\end{codepar}
\pyspecificend

\begin{arglist}
\argin{args}{Python dictionary containing:
    \begin{itemize}
        \item 'caller': Python \refpy{proc} identifying the namespace/rank of the process calling abort (dict)
        \item 'status': PMIx status  to be returned on exit (integer)
        \item 'msg': Optional string message to be printed (string)
        \item 'targets': Optional list of Python \refpy{proc} identifying the namespace/rank of the processes to be aborted (list)
    \end{itemize}}
\end{arglist}

Returns:
\begin{itemize}
    \item \refarg{rc} - \refconst{PMIX_SUCCESS} or a \ac{PMIx} error code indicating the operation failed (integer)
\end{itemize}

See \refapi{pmix_server_abort_fn_t} for details


%%%%%%%%%%%%%%%%%%%%%%%%%%%%%%%%%%%%%%%%%%%%%%%%%
\subsubsection{Fence}

%%%%
\summary

At least one client called either \refapi{PMIx_Fence} or \refapi{PMIx_Fence_nb}

%%%%
\format

\versionMarker{4.0}
\pyspecificstart
\begin{codepar}
def fence(args:dict is not None)
\end{codepar}
\pyspecificend

\begin{arglist}
\argin{args}{Python dictionary containing:
    \begin{itemize}
        \item 'procs': List of Python \refpy{proc} identifying the namespace/rank of the participating processes (list)
        \item 'directives': Optional list of Python \refpy{info} containing directives controlling the operation (list)
        \item 'data': Optional Python bytearray of data to be circulated during fence operation (bytearray)
    \end{itemize}}
\end{arglist}

Returns:
\begin{itemize}
    \item \refarg{rc} - \refconst{PMIX_SUCCESS} or a \ac{PMIx} error code indicating the operation failed (integer)
    \item \refarg{data} - Python bytearray containing the aggregated data from all participants (bytearray)
\end{itemize}

See \refapi{pmix_server_fencenb_fn_t} for details


%%%%%%%%%%%%%%%%%%%%%%%%%%%%%%%%%%%%%%%%%%%%%%%%%
\subsubsection{Direct Modex}

%%%%
\summary

Used by the PMIx server to request its local host contact the \ac{PMIx} server on the remote node that hosts the specified proc to obtain and return a direct modex blob for that proc.

%%%%
\format

\versionMarker{4.0}
\pyspecificstart
\begin{codepar}
def dmodex(args:dict is not None)
\end{codepar}
\pyspecificend

\begin{arglist}
\argin{args}{Python dictionary containing:
    \begin{itemize}
        \item 'proc': Python \refpy{proc} of process whose data is being requested (dict)
        \item 'directives': Optional list of Python \refpy{info} containing directives controlling the operation (list)
    \end{itemize}}
\end{arglist}

Returns:
\begin{itemize}
    \item \refarg{rc} - \refconst{PMIX_SUCCESS} or a \ac{PMIx} error code indicating the operation failed (integer)
    \item \refarg{data} - Python bytearray containing the data for the specified process (bytearray)
\end{itemize}

See \refapi{pmix_server_dmodex_req_fn_t} for details


%%%%%%%%%%%%%%%%%%%%%%%%%%%%%%%%%%%%%%%%%%%%%%%%%
\subsubsection{Publish}

%%%%
\summary

Publish data per the PMIx API specification.

%%%%
\format

\versionMarker{4.0}
\pyspecificstart
\begin{codepar}
def publish(args:dict is not None)
\end{codepar}
\pyspecificend

\begin{arglist}
\argin{args}{Python dictionary containing:
    \begin{itemize}
        \item 'proc': Python \refpy{proc} dictionary of process publishing the data (dict)
        \item 'directives': List of Python \refpy{info} containing data and directives (list)
    \end{itemize}}
\end{arglist}

Returns:
\begin{itemize}
    \item \refarg{rc} - \refconst{PMIX_SUCCESS} or a \ac{PMIx} error code indicating the operation failed (integer)
\end{itemize}

See \refapi{pmix_server_publish_fn_t} for details


%%%%%%%%%%%%%%%%%%%%%%%%%%%%%%%%%%%%%%%%%%%%%%%%%
\subsubsection{Lookup}

%%%%
\summary

Lookup published data.

%%%%
\format

\versionMarker{4.0}
\pyspecificstart
\begin{codepar}
def lookup(args:dict is not None)
\end{codepar}
\pyspecificend

\begin{arglist}
\argin{args}{Python dictionary containing:
    \begin{itemize}
        \item 'proc': Python \refpy{proc} of process seeking the data (dict)
        \item 'keys': List of Python strings (list)
        \item 'directives': Optional list of Python \refpy{info} containing directives (list)
    \end{itemize}}
\end{arglist}

Returns:
\begin{itemize}
    \item \refarg{rc} - \refconst{PMIX_SUCCESS} or a \ac{PMIx} error code indicating the operation failed (integer)
    \item \refarg{pdata} - List of \refpy{pdata} containing the returned results (list)
\end{itemize}

See \refapi{pmix_server_lookup_fn_t} for details


%%%%%%%%%%%%%%%%%%%%%%%%%%%%%%%%%%%%%%%%%%%%%%%%%
\subsubsection{Unpublish}

%%%%
\summary

Delete data from the data store.

%%%%
\format

\versionMarker{4.0}
\pyspecificstart
\begin{codepar}
def unpublish(args:dict is not None)
\end{codepar}
\pyspecificend

\begin{arglist}
\argin{args}{Python dictionary containing:
    \begin{itemize}
        \item 'proc': Python \refpy{proc} of process unpublishing data (dict)
        \item 'keys': List of Python strings (list)
        \item 'directives': Optional list of Python \refpy{info} containing directives (list)
    \end{itemize}}
\end{arglist}

Returns:
\begin{itemize}
    \item \refarg{rc} - \refconst{PMIX_SUCCESS} or a \ac{PMIx} error code indicating the operation failed (integer)
\end{itemize}

See \refapi{pmix_server_unpublish_fn_t} for details


%%%%%%%%%%%%%%%%%%%%%%%%%%%%%%%%%%%%%%%%%%%%%%%%%
\subsubsection{Spawn}

%%%%
\summary

Spawn a set of applications/processes as per the \refapi{PMIx_Spawn} API.

%%%%
\format

\versionMarker{4.0}
\pyspecificstart
\begin{codepar}
def spawn(args:dict is not None)
\end{codepar}
\pyspecificend

\begin{arglist}
\argin{args}{Python dictionary containing:
    \begin{itemize}
        \item 'proc': Python \refpy{proc} of process making the request (dict)
        \item 'jobinfo': Optional list of Python \refpy{info} job-level directives and information (list)
        \item 'apps': List of Python \refpy{app} describing applications to be spawned (list)
    \end{itemize}}
\end{arglist}

Returns:
\begin{itemize}
    \item \refarg{rc} - \refconst{PMIX_SUCCESS} or a \ac{PMIx} error code indicating the operation failed (integer)
    \item \refarg{nspace} - Python string containing namespace of the spawned job (str)
\end{itemize}

See \refapi{pmix_server_spawn_fn_t} for details


%%%%%%%%%%%%%%%%%%%%%%%%%%%%%%%%%%%%%%%%%%%%%%%%%
\subsubsection{Connect}

%%%%
\summary

Record the specified processes as \textit{connected}.

%%%%
\format

\versionMarker{4.0}
\pyspecificstart
\begin{codepar}
def connect(args:dict is not None)
\end{codepar}
\pyspecificend

\begin{arglist}
\argin{args}{Python dictionary containing:
    \begin{itemize}
        \item 'procs': List of Python \refpy{proc} identifying the namespace/rank of the participating processes (list)
        \item 'directives': Optional list of Python \refpy{info} containing directives controlling the operation (list)
    \end{itemize}}
\end{arglist}

Returns:
\begin{itemize}
    \item \refarg{rc} - \refconst{PMIX_SUCCESS} or a \ac{PMIx} error code indicating the operation failed (integer)
\end{itemize}

See \refapi{pmix_server_connect_fn_t} for details


%%%%%%%%%%%%%%%%%%%%%%%%%%%%%%%%%%%%%%%%%%%%%%%%%
\subsubsection{Disconnect}

%%%%
\summary

Disconnect a previously connected set of processes.

%%%%
\format

\versionMarker{4.0}
\pyspecificstart
\begin{codepar}
def disconnect(args:dict is not None)
\end{codepar}
\pyspecificend

\begin{arglist}
\argin{args}{Python dictionary containing:
    \begin{itemize}
        \item 'procs': List of Python \refpy{proc} identifying the namespace/rank of the participating processes (list)
        \item 'directives': Optional list of Python \refpy{info} containing directives controlling the operation (list)
    \end{itemize}}
\end{arglist}

Returns:
\begin{itemize}
    \item \refarg{rc} - \refconst{PMIX_SUCCESS} or a \ac{PMIx} error code indicating the operation failed (integer)
\end{itemize}

See \refapi{pmix_server_disconnect_fn_t} for details


%%%%%%%%%%%%%%%%%%%%%%%%%%%%%%%%%%%%%%%%%%%%%%%%%
\subsubsection{Register Events}

%%%%
\summary

Register to receive notifications for the specified events.

%%%%
\format

\versionMarker{4.0}
\pyspecificstart
\begin{codepar}
def register_events(args:dict is not None)
\end{codepar}
\pyspecificend

\begin{arglist}
\argin{args}{Python dictionary containing:
    \begin{itemize}
        \item 'codes': List of Python integers (list)
        \item 'directives': Optional list of Python \refpy{info} containing directives controlling the operation (list)
    \end{itemize}}
\end{arglist}

Returns:
\begin{itemize}
    \item \refarg{rc} - \refconst{PMIX_SUCCESS} or a \ac{PMIx} error code indicating the operation failed (integer)
\end{itemize}

See \refapi{pmix_server_register_events_fn_t} for details


%%%%%%%%%%%%%%%%%%%%%%%%%%%%%%%%%%%%%%%%%%%%%%%%%
\subsubsection{Deregister Events}

%%%%
\summary

Deregister to receive notifications for the specified events.

%%%%
\format

\versionMarker{4.0}
\pyspecificstart
\begin{codepar}
def deregister_events(args:dict is not None)
\end{codepar}
\pyspecificend

\begin{arglist}
\argin{args}{Python dictionary containing:
    \begin{itemize}
        \item 'codes': List of Python integers (list)
    \end{itemize}}
\end{arglist}

Returns:
\begin{itemize}
    \item \refarg{rc} - \refconst{PMIX_SUCCESS} or a \ac{PMIx} error code indicating the operation failed (integer)
\end{itemize}

See \refapi{pmix_server_deregister_events_fn_t} for details


%%%%%%%%%%%%%%%%%%%%%%%%%%%%%%%%%%%%%%%%%%%%%%%%%
\subsubsection{Notify Event}

%%%%
\summary

Notify the specified range of processes of an event.

%%%%
\format

\versionMarker{4.0}
\pyspecificstart
\begin{codepar}
def notify_event(args:dict is not None)
\end{codepar}
\pyspecificend

\begin{arglist}
\argin{args}{Python dictionary containing:
    \begin{itemize}
        \item 'code': Python integer \refstruct{pmix_status_t} (integer)
        \item 'source': Python \refpy{proc} of process that generated the event (dict)
        \item 'range': Python \refpy{range} in which the event is to be reported (integer)
        \item 'directives': Optional list of Python \refpy{info} directives (list)
    \end{itemize}}
\end{arglist}

Returns:
\begin{itemize}
    \item \refarg{rc} - \refconst{PMIX_SUCCESS} or a \ac{PMIx} error code indicating the operation failed (integer)
\end{itemize}

See \refapi{pmix_server_notify_event_fn_t} for details


%%%%%%%%%%%%%%%%%%%%%%%%%%%%%%%%%%%%%%%%%%%%%%%%%
\subsubsection{Query}

%%%%
\summary

Query information from the resource manager.

%%%%
\format

\versionMarker{4.0}
\pyspecificstart
\begin{codepar}
def query(args:dict is not None)
\end{codepar}
\pyspecificend

\begin{arglist}
\argin{args}{Python dictionary containing:
    \begin{itemize}
        \item 'source': Python \refpy{proc} of requesting process (dict)
        \item 'queries': List of Python \refpy{query} directives (list)
    \end{itemize}}
\end{arglist}

Returns:
\begin{itemize}
    \item \refarg{rc} - \refconst{PMIX_SUCCESS} or a \ac{PMIx} error code indicating the operation failed (integer)
    \item \refarg{info} - List of Python \refpy{info} containing the returned results (list)
\end{itemize}

See \refapi{pmix_server_query_fn_t} for details


%%%%%%%%%%%%%%%%%%%%%%%%%%%%%%%%%%%%%%%%%%%%%%%%%
\subsubsection{Tool Connected}

%%%%
\summary

Register that a tool has connected to the server.

%%%%
\format

\versionMarker{4.0}
\pyspecificstart
\begin{codepar}
def tool_connected(args:dict is not None)
\end{codepar}
\pyspecificend

\begin{arglist}
\argin{args}{Python dictionary containing:
    \begin{itemize}
        \item 'directives': Optional list of Python \refpy{info} info on the connecting tool (list)
    \end{itemize}}
\end{arglist}

Returns:
\begin{itemize}
    \item \refarg{rc} - \refconst{PMIX_SUCCESS} or a \ac{PMIx} error code indicating the operation failed (integer)
    \item \refarg{proc} - Python \refpy{proc} containing the assigned namespace:rank for the tool (dict)
\end{itemize}

See \refapi{pmix_server_tool_connection_fn_t} for details


%%%%%%%%%%%%%%%%%%%%%%%%%%%%%%%%%%%%%%%%%%%%%%%%%
\subsubsection{Log}

%%%%
\summary

Log data on behalf of a client.

%%%%
\format

\versionMarker{4.0}
\pyspecificstart
\begin{codepar}
def log(args:dict is not None)
\end{codepar}
\pyspecificend

\begin{arglist}
\argin{args}{Python dictionary containing:
    \begin{itemize}
        \item 'source': Python \refpy{proc} of requesting process (dict)
        \item 'data': Optional list of Python \refpy{info} containing data to be logged (list)
        \item 'directives': Optional list of Python \refpy{info} containing directives (list)
    \end{itemize}}
\end{arglist}

Returns:
\begin{itemize}
    \item \refarg{rc} - \refconst{PMIX_SUCCESS} or a \ac{PMIx} error code indicating the operation failed (integer)
\end{itemize}

See \refapi{pmix_server_log_fn_t} for details.


%%%%%%%%%%%%%%%%%%%%%%%%%%%%%%%%%%%%%%%%%%%%%%%%%
\subsubsection{Allocate Resources}

%%%%
\summary

Request allocation operations on behalf of a client.

%%%%
\format

\versionMarker{4.0}
\pyspecificstart
\begin{codepar}
def allocate(args:dict is not None)
\end{codepar}
\pyspecificend

\begin{arglist}
\argin{args}{Python dictionary containing:
    \begin{itemize}
        \item 'source': Python \refpy{proc} of requesting process (dict)
        \item 'action': Python \refpy{allocdir} specifying requested action (integer)
        \item 'directives': Optional list of Python \refpy{info} containing directives (list)
    \end{itemize}}
\end{arglist}

Returns:
\begin{itemize}
    \item \refarg{rc} - \refconst{PMIX_SUCCESS} or a \ac{PMIx} error code indicating the operation failed (integer)
    \item refarg{info} - List of Python \refpy{info} containing results of requested operation (list)
\end{itemize}

See \refapi{pmix_server_alloc_fn_t} for details.


%%%%%%%%%%%%%%%%%%%%%%%%%%%%%%%%%%%%%%%%%%%%%%%%%
\subsubsection{Job Control}

%%%%
\summary

Execute a job control action on behalf of a client.

%%%%
\format

\versionMarker{4.0}
\pyspecificstart
\begin{codepar}
def job_control(args:dict is not None)
\end{codepar}
\pyspecificend

\begin{arglist}
\argin{args}{Python dictionary containing:
    \begin{itemize}
        \item 'source': Python \refpy{proc} of requesting process (dict)
        \item 'targets': List of Python \refpy{proc} specifying target processes (list)
        \item 'directives': Optional list of Python \refpy{info} containing directives (list)
    \end{itemize}}
\end{arglist}

Returns:
\begin{itemize}
    \item \refarg{rc} - \refconst{PMIX_SUCCESS} or a \ac{PMIx} error code indicating the operation failed (integer)
\end{itemize}

See \refapi{pmix_server_job_control_fn_t} for details.


%%%%%%%%%%%%%%%%%%%%%%%%%%%%%%%%%%%%%%%%%%%%%%%%%
\subsubsection{Monitor}

%%%%
\summary

Request that a client be monitored for activity.

%%%%
\format

\versionMarker{4.0}
\pyspecificstart
\begin{codepar}
def monitor(args:dict is not None)
\end{codepar}
\pyspecificend

\begin{arglist}
\argin{args}{Python dictionary containing:
    \begin{itemize}
        \item 'source': Python \refpy{proc} of requesting process (dict)
        \item 'monitor': Python \refpy{info} attribute indicating the type of monitor being requested (dict)
        \item 'error': Status code to be used when generating an event notification (integer) alerting that the monitor has been triggered.
        \item 'directives': Optional list of Python \refpy{info} containing directives (list)
    \end{itemize}}
\end{arglist}

Returns:
\begin{itemize}
    \item \refarg{rc} - \refconst{PMIX_SUCCESS} or a \ac{PMIx} error code indicating the operation failed (integer)
\end{itemize}

See \refapi{pmix_server_monitor_fn_t} for details.


%%%%%%%%%%%%%%%%%%%%%%%%%%%%%%%%%%%%%%%%%%%%%%%%%
\subsubsection{Get Credential}

%%%%
\summary

Request a credential from the host environment.

%%%%
\format

\versionMarker{4.0}
\pyspecificstart
\begin{codepar}
def get_credential(args:dict is not None)
\end{codepar}
\pyspecificend

\begin{arglist}
\argin{args}{Python dictionary containing:
    \begin{itemize}
        \item 'source': Python \refpy{proc} of requesting process (dict)
        \item 'directives': Optional list of Python \refpy{info} containing directives (list)
    \end{itemize}}
\end{arglist}

Returns:
\begin{itemize}
    \item \refarg{rc} - \refconst{PMIX_SUCCESS} or a \ac{PMIx} error code indicating the operation failed (integer)
    \item \refarg{cred} - Python \refpy{byteobject} containing returned credential (dict)
    \item \refarg{info} - List of Python \refpy{info} containing any additional info about the credential (list)
\end{itemize}

See \refapi{pmix_server_get_cred_fn_t} for details.


%%%%%%%%%%%%%%%%%%%%%%%%%%%%%%%%%%%%%%%%%%%%%%%%%
\subsubsection{Validate Credential}

%%%%
\summary

Request validation of a credential

%%%%
\format

\versionMarker{4.0}
\pyspecificstart
\begin{codepar}
def validate_credential(args:dict is not None)
\end{codepar}
\pyspecificend

\begin{arglist}
\argin{args}{Python dictionary containing:
    \begin{itemize}
        \item 'source': Python \refpy{proc} of requesting process (dict)
        \item 'credential': Python \refpy{byteobject} containing credential (dict)
        \item 'directives': Optional list of Python \refpy{info} containing directives (list)
    \end{itemize}}
\end{arglist}

Returns:
\begin{itemize}
    \item \refarg{rc} - \refconst{PMIX_SUCCESS} or a \ac{PMIx} error code indicating the operation failed (integer)
    \item \refarg{info} - List of Python \refpy{info} containing any additional info from the credential (list)
\end{itemize}

See \refapi{pmix_server_validate_cred_fn_t} for details.


%%%%%%%%%%%%%%%%%%%%%%%%%%%%%%%%%%%%%%%%%%%%%%%%%
\subsubsection{IO Forward}

%%%%
\summary

Request the specified IO channels be forwarded from the given array of processes.

%%%%
\format

\versionMarker{4.0}
\pyspecificstart
\begin{codepar}
def iof_pull(args:dict is not None)
\end{codepar}
\pyspecificend

\begin{arglist}
\argin{args}{Python dictionary containing:
    \begin{itemize}
        \item 'sources': List of Python \refpy{proc} of processes  whose IO is being requested (list)
        \item 'channels': Bitmask of Python \refpy{channel} identifying IO channels to be forwarded (integer)
        \item 'directives': Optional list of Python \refpy{info} containing directives (list)
    \end{itemize}}
\end{arglist}

Returns:
\begin{itemize}
    \item \refarg{rc} - \refconst{PMIX_SUCCESS} or a \ac{PMIx} error code indicating the operation failed (integer)
\end{itemize}

See \refapi{pmix_server_iof_fn_t} for details.


%%%%%%%%%%%%%%%%%%%%%%%%%%%%%%%%%%%%%%%%%%%%%%%%%
\subsubsection{IO Push}

%%%%
\summary

Pass standard input data to the host environment for transmission to specified recipients.

%%%%
\format

\versionMarker{4.0}
\pyspecificstart
\begin{codepar}
def iof_push(args:dict is not None)
\end{codepar}
\pyspecificend

\begin{arglist}
\argin{args}{Python dictionary containing:
   \begin{itemize}
        \item 'source': Python \refpy{proc} of process whose input is being forwarded (dict)
        \item 'payload': Python \refpy{byteobject} containing input bytes (dict)
        \item 'targets': List of \refpy{proc} of processes that are to receive the payload (list)
        \item 'directives': Optional list of Python \refpy{info} containing directives (list)
    \end{itemize}}
\end{arglist}

Returns:
\begin{itemize}
    \item \refarg{rc} - \refconst{PMIX_SUCCESS} or a \ac{PMIx} error code indicating the operation failed (integer)
\end{itemize}

See \refapi{pmix_server_stdin_fn_t} for details.


%%%%%%%%%%%%%%%%%%%%%%%%%%%%%%%%%%%%%%%%%%%%%%%%%
\subsubsection{Group Operations}

%%%%
\summary

Request group operations (construct, destruct, etc.) on behalf of a set of processes.

%%%%
\format

\versionMarker{4.0}
\pyspecificstart
\begin{codepar}
def group(args:dict is not None)
\end{codepar}
\pyspecificend

\begin{arglist}
\argin{args}{Python dictionary containing:
   \begin{itemize}
        \item 'op': Operation host is to perform on the specified group (integer)
        \item 'group': String identifier of target group (str)
        \item 'procs': List of Python \refpy{proc} of participating processes (dict)
        \item 'directives': Optional list of Python \refpy{info} containing directives (list)
    \end{itemize}}
\end{arglist}

Returns:
\begin{itemize}
    \item \refarg{rc} - \refconst{PMIX_SUCCESS} or a \ac{PMIx} error code indicating the operation failed (integer)
    \item refarg{info} - List of Python \refpy{info} containing results of requested operation (list)
\end{itemize}

See \refapi{pmix_server_grp_fn_t} for details.


%%%%%%%%%%%%%%%%%%%%%%%%%%%%%%%%%%%%%%%%%%%%%%%%%
\subsubsection{Fabric Operations}

%%%%
\summary

Request fabric-related operations (e.g., information on a fabric) on behalf of a tool or other process.

%%%%
\format

\versionMarker{4.0}
\pyspecificstart
\begin{codepar}
def fabric(args:dict is not None)
\end{codepar}
\pyspecificend

\begin{arglist}
\argin{args}{Python dictionary containing:
   \begin{itemize}
        \item 'source': Python \refpy{proc} of requesting process (dict)
        \item 'index': Identifier of the fabric being operated upon (integer)
        \item 'op': Operation host is to perform on the specified fabric (integer)
        \item 'directives': Optional list of Python \refpy{info} containing directives (list)
    \end{itemize}}
\end{arglist}

Returns:
\begin{itemize}
    \item \refarg{rc} - \refconst{PMIX_SUCCESS} or a \ac{PMIx} error code indicating the operation failed (integer)
    \item refarg{info} - List of Python \refpy{info} containing results of requested operation (list)
\end{itemize}

See \refapi{pmix_server_fabric_fn_t} for details.


%%%%%%%%%%%%%%%%%%%%%%%%%%%%%%%%%%%%%%%%%%%%%%%%%
%%%%%%%%%%%%%%%%%%%%%%%%%%%%%%%%%%%%%%%%%%%%%%%%%
\section{PMIxClient}
\label{app:python:client}

The client Python class is by far the richest in terms of \acp{API} as it houses all the \acp{API} that an application might utilize. Due to the datatype translation requirements of the C-Python interface, only the blocking form of each \ac{API} is supported – providing a Python callback function directly to the C interface underlying the bindings was not a supportable option.


%%%%%%%%%%%%%%%%%%%%%%%%%%%%%%%%%%%%%%%%%%%%%%%%%
\subsection{Client.init}
\declareapibinding{PMIxClient.init}{PMIx_Init}{Python}

\summary Initialize the \ac{PMIx} client library after obtaining a new PMIxClient object.

\format

\versionMarker{4.0}
\pyspecificstart
\begin{codepar}
rc, proc = myclient.init(info:list)
\end{codepar}
\pyspecificend


\begin{arglist}
\argin{info}{List of Python \refpy{info} dictionaries (list)}
\end{arglist}

Returns:

\begin{itemize}
    \item \refarg{rc} - \refconst{PMIX_SUCCESS} or a negative value corresponding to a PMIx error constant (integer)
    \item \refarg{proc} - a Python \refpy{proc} dictionary (dict)
\end{itemize}


See \refapi{PMIx_Init} for description of all relevant attributes and behaviors.


%%%%%%%%%%%%%%%%%%%%%%%%%%%%%%%%%%%%%%%%%%%%%%%%%
\subsection{Client.initialized}
\declareapibinding{PMIxClient.initialized}{PMIx_Initialized}{Python}

\format

\versionMarker{4.0}
\pyspecificstart
\begin{codepar}
rc = myclient.initialized()
\end{codepar}
\pyspecificend

Returns:

\begin{itemize}
    \item \refarg{rc} - a value of \code{1} (true) will be returned if the \ac{PMIx} library has been initialized, and \code{0} (false) otherwise (integer)

\end{itemize}

See \refapi{PMIx_Initialized} for description of all relevant attributes and behaviors.


%%%%%%%%%%%%%%%%%%%%%%%%%%%%%%%%%%%%%%%%%%%%%%%%%
\subsection{Client.get_version}
\declareapibinding{PMIxClient.get_version}{PMIx_Get_version}{Python}

\format

\versionMarker{4.0}
\pyspecificstart
\begin{codepar}
vers = myclient.get_version()
\end{codepar}
\pyspecificend

Returns:

\begin{itemize}
    \item \refarg{vers} - Python string containing the version of the \ac{PMIx} library (e.g., "3.1.4") (integer)

\end{itemize}

See \refapi{PMIx_Get_version} for description of all relevant attributes and behaviors.


%%%%%%%%%%%%%%%%%%%%%%%%%%%%%%%%%%%%%%%%%%%%%%%%%
\subsection{Client.finalize}
\declareapibinding{PMIxClient.finalize}{PMIx_Finalize}{Python}

%%%%
\summary

Finalize the PMIx client library.

%%%%
\format

\versionMarker{4.0}
\pyspecificstart
\begin{codepar}
rc = myclient.finalize(info:list)
\end{codepar}
\pyspecificend

\begin{arglist}
\argin{info}{List of Python \refpy{info} dictionaries (list)}
\end{arglist}

Returns:

\begin{itemize}
    \item \refarg{rc} - \refconst{PMIX_SUCCESS} or a negative value corresponding to a PMIx error constant (integer)
\end{itemize}

See \refapi{PMIx_Finalize} for description of all relevant attributes and behaviors.


%%%%%%%%%%%%%%%%%%%%%%%%%%%%%%%%%%%%%%%%%%%%%%%%%
\subsection{Client.abort}
\declareapibinding{PMIxClient.abort}{PMIx_Abort}{Python}

%%%%
\summary

Request that the provided list of processes be aborted.

%%%%
\format

\versionMarker{4.0}
\pyspecificstart
\begin{codepar}
rc = myclient.abort(status:integer, msg:str, targets:list)
\end{codepar}
\pyspecificend

\begin{arglist}
\argin{status}{PMIx status to be returned on exit (integer)}
\argin{msg}{String message to be printed (string)}
\argin{targets}{List of Python \refpy{proc} dictionaries (list)}
\end{arglist}

Returns:

\begin{itemize}
    \item \refarg{rc} - \refconst{PMIX_SUCCESS} or a negative value corresponding to a PMIx error constant (integer)
\end{itemize}

See \refapi{PMIx_Abort} for description of all relevant attributes and behaviors.


%%%%%%%%%%%%%%%%%%%%%%%%%%%%%%%%%%%%%%%%%%%%%%%%%
\subsection{Client.store_internal}
\declareapibinding{PMIxClient.store_internal}{PMIx_Store_internal}{Python}

%%%%
\summary

Store some data locally for retrieval by other areas of the process

%%%%
\format

\versionMarker{4.0}
\pyspecificstart
\begin{codepar}
rc = myclient.store_internal(proc:dict, key:str, value:dict)
\end{codepar}
\pyspecificend

\begin{arglist}
\argin{proc}{Python \refpy{proc} dictionary of the process being referenced (dict)}
\argin{key}{String key of the data (string)}
\argin{value}{Python \refpy{value} dictionary (dict)}
\end{arglist}

Returns:

\begin{itemize}
    \item \refarg{rc} - \refconst{PMIX_SUCCESS} or a negative value corresponding to a PMIx error constant (integer)
\end{itemize}

See \refapi{PMIx_Store_internal} for details.


%%%%%%%%%%%%%%%%%%%%%%%%%%%%%%%%%%%%%%%%%%%%%%%%%
\subsection{Client.put}
\declareapibinding{PMIxClient.put}{PMIx_Put}{Python}

%%%%
\summary

Push a key/value pair into the client's namespace.

%%%%
\format

\versionMarker{4.0}
\pyspecificstart
\begin{codepar}
rc = myclient.put(scope:integer, key:str, value:dict)
\end{codepar}
\pyspecificend

\begin{arglist}
\argin{scope}{Scope of the data being posted (integer)}
\argin{key}{String key of the data (string)}
\argin{value}{Python \refpy{value} dictionary (dict)}
\end{arglist}

Returns:

\begin{itemize}
    \item \refarg{rc} - \refconst{PMIX_SUCCESS} or a negative value corresponding to a PMIx error constant (integer)
\end{itemize}

See \refapi{PMIx_Put} for description of all relevant attributes and behaviors.


%%%%%%%%%%%%%%%%%%%%%%%%%%%%%%%%%%%%%%%%%%%%%%%%%
\subsection{Client.commit}
\declareapibinding{PMIxClient.commit}{PMIx_Commit}{Python}

%%%%
\summary

Push all previously \refapibinding{PMIxClient.put} values to the local PMIx server.

%%%%
\format

\versionMarker{4.0}
\pyspecificstart
\begin{codepar}
rc = myclient.commit()
\end{codepar}
\pyspecificend

Returns:

\begin{itemize}
    \item \refarg{rc} - \refconst{PMIX_SUCCESS} or a negative value corresponding to a PMIx error constant (integer)
\end{itemize}

See \refapi{PMIx_Commit} for description of all relevant attributes and behaviors.


%%%%%%%%%%%%%%%%%%%%%%%%%%%%%%%%%%%%%%%%%%%%%%%%%
\subsection{Client.fence}
\declareapibinding{PMIxClient.fence}{PMIx_Fence}{Python}

%%%%
\summary

Execute a blocking barrier across the processes identified in the specified list.

%%%%
\format

\versionMarker{4.0}
\pyspecificstart
\begin{codepar}
rc = myclient.fence(peers:list, directives:list)
\end{codepar}
\pyspecificend

\begin{arglist}
\argin{peers}{List of Python \refpy{proc} dictionaries (list)}
\argin{directives}{List of Python \refpy{info} dictionaries (list)}
\end{arglist}

Returns:

\begin{itemize}
    \item \refarg{rc} - \refconst{PMIX_SUCCESS} or a negative value corresponding to a PMIx error constant (integer)
\end{itemize}

See \refapi{PMIx_Fence} for description of all relevant attributes and behaviors.


%%%%%%%%%%%%%%%%%%%%%%%%%%%%%%%%%%%%%%%%%%%%%%%%%
\subsection{Client.get}
\declareapibinding{PMIxClient.get}{PMIx_Get}{Python}

%%%%
\summary

Retrieve a key/value pair.

%%%%
\format

\versionMarker{4.0}
\pyspecificstart
\begin{codepar}
rc, val = myclient.get(proc:dict, key:str, directives:list)
\end{codepar}
\pyspecificend

\begin{arglist}
\argin{proc}{Python \refpy{proc} whose data is being requested (dict)}
\argin{key}{Python string key of the data to be returned (str)}
\argin{directives}{List of Python \refpy{info} dictionaries (list)}
\end{arglist}

Returns:

\begin{itemize}
    \item \refarg{rc} - \refconst{PMIX_SUCCESS} or a negative value corresponding to a PMIx error constant (integer)
    \item \refarg{val} - Python \refpy{value} containing the returned data (dict)
\end{itemize}

See \refapi{PMIx_Get} for description of all relevant attributes and behaviors.


%%%%%%%%%%%%%%%%%%%%%%%%%%%%%%%%%%%%%%%%%%%%%%%%%
\subsection{Client.publish}
\declareapibinding{PMIxClient.publish}{PMIx_Publish}{Python}

%%%%
\summary

Publish data for later access via \refapi{PMIx_Lookup}.

%%%%
\format

\versionMarker{4.0}
\pyspecificstart
\begin{codepar}
rc = myclient.publish(directives:list)
\end{codepar}
\pyspecificend

\begin{arglist}
\argin{directives}{List of Python \refpy{info} dictionaries containing data to be published and directives (list)}
\end{arglist}

Returns:

\begin{itemize}
    \item \refarg{rc} - \refconst{PMIX_SUCCESS} or a negative value corresponding to a PMIx error constant (integer)
\end{itemize}

See \refapi{PMIx_Publish} for description of all relevant attributes and behaviors.


%%%%%%%%%%%%%%%%%%%%%%%%%%%%%%%%%%%%%%%%%%%%%%%%%
\subsection{Client.lookup}
\declareapibinding{PMIxClient.lookup}{PMIx_Lookup}{Python}

%%%%
\summary

Lookup information published by this or another process with \refapi{PMIx_Publish}.

%%%%
\format

\versionMarker{4.0}
\pyspecificstart
\begin{codepar}
rc,info = myclient.lookup(pdata:list, directives:list)
\end{codepar}
\pyspecificend

\begin{arglist}
\argin{pdata}{List of Python \refpy{pdata} dictionaries identifying data to be retrieved (list)}
\argin{directives}{List of Python \refpy{info} dictionaries (list)}
\end{arglist}

Returns:

\begin{itemize}
    \item \refarg{rc} - \refconst{PMIX_SUCCESS} or a negative value corresponding to a PMIx error constant (integer)
    \item \refarg{info} - Python list of \refpy{info} containing the returned data (list)
\end{itemize}

See \refapi{PMIx_Lookup} for description of all relevant attributes and behaviors.


%%%%%%%%%%%%%%%%%%%%%%%%%%%%%%%%%%%%%%%%%%%%%%%%%
\subsection{Client.unpublish}
\declareapibinding{PMIxClient.unpublish}{PMIx_Unpublish}{Python}

%%%%
\summary

Delete data published by this process with \refapi{PMIx_Publish}.

%%%%
\format

\versionMarker{4.0}
\pyspecificstart
\begin{codepar}
rc = myclient.unpublish(keys:list, directives:list)
\end{codepar}
\pyspecificend

\begin{arglist}
\argin{keys}{List of Python string keys identifying data to be deleted (list)}
\argin{directives}{List of Python \refpy{info} dictionaries (list)}
\end{arglist}

Returns:

\begin{itemize}
    \item \refarg{rc} - \refconst{PMIX_SUCCESS} or a negative value corresponding to a PMIx error constant (integer)
\end{itemize}

See \refapi{PMIx_Unpublish} for description of all relevant attributes and behaviors.


%%%%%%%%%%%%%%%%%%%%%%%%%%%%%%%%%%%%%%%%%%%%%%%%%
\subsection{Client.spawn}
\declareapibinding{PMIxClient.spawn}{PMIx_Spawn}{Python}

%%%%
\summary

Spawn a new job.

%%%%
\format

\versionMarker{4.0}
\pyspecificstart
\begin{codepar}
rc,nspace = myclient.spawn(jobinfo:list, apps:list)
\end{codepar}
\pyspecificend

\begin{arglist}
\argin{jobinfo}{List of Python \refpy{info} dictionaries (list)}
\argin{apps}{List of Python \refpy{app} dictionaries (list)}
\end{arglist}

Returns:

\begin{itemize}
    \item \refarg{rc} - \refconst{PMIX_SUCCESS} or a negative value corresponding to a PMIx error constant (integer)
    \item \refarg{nspace} - Python \refpy{nspace} of the new job (dict)
\end{itemize}

See \refapi{PMIx_Spawn} for description of all relevant attributes and behaviors.


%%%%%%%%%%%%%%%%%%%%%%%%%%%%%%%%%%%%%%%%%%%%%%%%%
\subsection{Client.connect}
\declareapibinding{PMIxClient.connect}{PMIx_Connect}{Python}

%%%%
\summary

Connect namespaces.

%%%%
\format

\versionMarker{4.0}
\pyspecificstart
\begin{codepar}
rc = myclient.connect(peers:list, directives:list)
\end{codepar}
\pyspecificend

\begin{arglist}
\argin{peers}{List of Python \refpy{proc} dictionaries (list)}
\argin{directives}{List of Python \refpy{info} dictionaries (list)}
\end{arglist}

Returns:

\begin{itemize}
    \item \refarg{rc} - \refconst{PMIX_SUCCESS} or a negative value corresponding to a PMIx error constant (integer)
\end{itemize}

See \refapi{PMIx_Connect} for description of all relevant attributes and behaviors.


%%%%%%%%%%%%%%%%%%%%%%%%%%%%%%%%%%%%%%%%%%%%%%%%%
\subsection{Client.disconnect}
\declareapibinding{PMIxClient.disconnect}{PMIx_Disconnect}{Python}

%%%%
\summary

Disconnect namespaces.

%%%%
\format

\versionMarker{4.0}
\pyspecificstart
\begin{codepar}
rc = myclient.disconnect(peers:list, directives:list)
\end{codepar}
\pyspecificend

\begin{arglist}
\argin{peers}{List of Python \refpy{proc} dictionaries (list)}
\argin{directives}{List of Python \refpy{info} dictionaries (list)}
\end{arglist}

Returns:

\begin{itemize}
    \item \refarg{rc} - \refconst{PMIX_SUCCESS} or a negative value corresponding to a PMIx error constant (integer)
\end{itemize}

See \refapi{PMIx_Disconnect} for description of all relevant attributes and behaviors.


%%%%%%%%%%%%%%%%%%%%%%%%%%%%%%%%%%%%%%%%%%%%%%%%%
\subsection{Client.resolve_peers}
\declareapibinding{PMIxClient.resolve_peers}{PMIx_Resolve_peers}{Python}

%%%%
\summary

Return list of processes within the specified \refpy{nspace} on the given node.

%%%%
\format

\versionMarker{4.0}
\pyspecificstart
\begin{codepar}
rc,procs = myclient.resolve_peers(node:str, nspace:str)
\end{codepar}
\pyspecificend

\begin{arglist}
\argin{node}{Name of node whose processes are being requested (str)}
\argin{nspace}{Python \refpy{nspace} whose processes are to be returned (str)}
\end{arglist}

Returns:

\begin{itemize}
    \item \refarg{rc} - \refconst{PMIX_SUCCESS} or a negative value corresponding to a PMIx error constant (integer)
    \item \refarg{procs} - List of Python \refpy{proc} dictionaries (list)
\end{itemize}

See \refapi{PMIx_Resolve_peers} for description of all relevant attributes and behaviors.


%%%%%%%%%%%%%%%%%%%%%%%%%%%%%%%%%%%%%%%%%%%%%%%%%
\subsection{Client.resolve_nodes}
\declareapibinding{PMIxClient.resolve_nodes}{PMIx_Resolve_nodes}{Python}

%%%%
\summary

Return list of nodes hosting processes within the specified \refpy{nspace}.

%%%%
\format

\versionMarker{4.0}
\pyspecificstart
\begin{codepar}
rc,nodes = myclient.resolve_nodes(nspace:str)
\end{codepar}
\pyspecificend

\begin{arglist}
\argin{nspace}{Python \refpy{nspace} (str)}
\end{arglist}

Returns:

\begin{itemize}
    \item \refarg{rc} - \refconst{PMIX_SUCCESS} or a negative value corresponding to a PMIx error constant (integer)
    \item \refarg{nodes} - List of Python string node names (list)
\end{itemize}

See \refapi{PMIx_Resolve_nodes} for description of all relevant attributes and behaviors.


%%%%%%%%%%%%%%%%%%%%%%%%%%%%%%%%%%%%%%%%%%%%%%%%%
\subsection{Client.query}
\declareapibinding{PMIxClient.query}{PMIx_Query_info_nb}{Python}

%%%%
\summary

Query information about the system in general.

%%%%
\format

\versionMarker{4.0}
\pyspecificstart
\begin{codepar}
rc,info = myclient.query(queries:list)
\end{codepar}
\pyspecificend

\begin{arglist}
\argin{queries}{List of Python \refpy{query} dictionaries (list)}
\end{arglist}

Returns:

\begin{itemize}
    \item \refarg{rc} - \refconst{PMIX_SUCCESS} or a negative value corresponding to a PMIx error constant (integer)
    \item \refarg{info} - List of Python \refpy{info} containing results of the query (list)
\end{itemize}

See \refapi{PMIx_Query_info_nb} for description of all relevant attributes and behaviors.


%%%%%%%%%%%%%%%%%%%%%%%%%%%%%%%%%%%%%%%%%%%%%%%%%
\subsection{Client.log}
\declareapibinding{PMIxClient.log}{PMIx_Log}{Python}

%%%%
\summary

Log data to a central data service/store.

%%%%
\format

\versionMarker{4.0}
\pyspecificstart
\begin{codepar}
rc = myclient.log(data:list, directives:list)
\end{codepar}
\pyspecificend

\begin{arglist}
\argin{data}{List of Python \refpy{info} (list)}
\argin{directives}{Optional list of Python \refpy{info} (list)}
\end{arglist}

Returns:

\begin{itemize}
    \item \refarg{rc} - \refconst{PMIX_SUCCESS} or a negative value corresponding to a PMIx error constant (integer)
\end{itemize}

See \refapi{PMIx_Log} for description of all relevant attributes and behaviors.


%%%%%%%%%%%%%%%%%%%%%%%%%%%%%%%%%%%%%%%%%%%%%%%%%
\subsection{Client.allocate}
\declareapibinding{PMIxClient.allocate}{PMIx_Allocation_request_nb}{Python}

%%%%
\summary

Request an allocation operation from the host resource manager.

%%%%
\format

\versionMarker{4.0}
\pyspecificstart
\begin{codepar}
rc,info = myclient.allocate(request:integer, directives:list)
\end{codepar}
\pyspecificend

\begin{arglist}
\argin{request}{Python \refpy{allocdir} specifying requested operation (integer)}
\argin{directives}{List of Python \refpy{info} describing request (list)}
\end{arglist}

Returns:

\begin{itemize}
    \item \refarg{rc} - \refconst{PMIX_SUCCESS} or a negative value corresponding to a PMIx error constant (integer)
    \item \refarg{info} - List of Python \refpy{info} containing results of the request (list)
\end{itemize}

See \refapi{PMIx_Allocation_request_nb} for description of all relevant attributes and behaviors.


%%%%%%%%%%%%%%%%%%%%%%%%%%%%%%%%%%%%%%%%%%%%%%%%%
\subsection{Client.job_ctrl}
\declareapibinding{PMIxClient.job_ctrl}{PMIx_Job_control_nb}{Python}

%%%%
\summary

Request a job control action.

%%%%
\format

\versionMarker{4.0}
\pyspecificstart
\begin{codepar}
rc,info = myclient.job_ctrl(targets:list, directives:list)
\end{codepar}
\pyspecificend

\begin{arglist}
\argin{targets}{List of Python \refpy{proc} specifying targets of requested operation (integer)}
\argin{directives}{List of Python \refpy{info} describing operation to be performed (list)}
\end{arglist}

Returns:

\begin{itemize}
    \item \refarg{rc} - \refconst{PMIX_SUCCESS} or a negative value corresponding to a PMIx error constant (integer)
    \item \refarg{info} - List of Python \refpy{info} containing results of the request (list)
\end{itemize}

See \refapi{PMIx_Job_control_nb} for description of all relevant attributes and behaviors.


%%%%%%%%%%%%%%%%%%%%%%%%%%%%%%%%%%%%%%%%%%%%%%%%%
\subsection{Client.monitor}
\declareapibinding{PMIxClient.monitor}{PMIx_Process_monitor_nb}{Python}

%%%%
\summary

Request that something be monitored.

%%%%
\format

\versionMarker{4.0}
\pyspecificstart
\begin{codepar}
rc,info = myclient.monitor(monitor:dict, error_code:integer, directives:list)
\end{codepar}
\pyspecificend

\begin{arglist}
\argin{monitor}{Python \refpy{info} specifying specifying the type of monitor being requested (dict)}
\argin{error_code}{Status code to be used when generating an event notification alerting that the monitor has been triggered (integer)}
\argin{directives}{List of Python \refpy{info} describing request (list)}
\end{arglist}

Returns:

\begin{itemize}
    \item \refarg{rc} - \refconst{PMIX_SUCCESS} or a negative value corresponding to a PMIx error constant (integer)
    \item \refarg{info} - List of Python \refpy{info} containing results of the request (list)
\end{itemize}

See \refapi{PMIx_Process_monitor_nb} for description of all relevant attributes and behaviors.


%%%%%%%%%%%%%%%%%%%%%%%%%%%%%%%%%%%%%%%%%%%%%%%%%
\subsection{Client.get_credential}
\declareapibinding{PMIxClient.get_credential}{PMIx_Get_credential}{Python}

%%%%
\summary

Request a credential from the PMIx server/SMS.

%%%%
\format

\versionMarker{4.0}
\pyspecificstart
\begin{codepar}
rc,cred = myclient.get_credential(directives:list)
\end{codepar}
\pyspecificend

\begin{arglist}
\argin{directives}{Optional list of Python \refpy{info} describing request (list)}
\end{arglist}

Returns:

\begin{itemize}
    \item \refarg{rc} - \refconst{PMIX_SUCCESS} or a negative value corresponding to a PMIx error constant (integer)
    \item \refarg{cred} - Python \refpy{byteobject} containing returned credential (dict)
\end{itemize}

See \refapi{PMIx_Get_credential} for description of all relevant attributes and behaviors.


%%%%%%%%%%%%%%%%%%%%%%%%%%%%%%%%%%%%%%%%%%%%%%%%%
\subsection{Client.validate_credential}
\declareapibinding{PMIxClient.validate_credential}{PMIx_Validate_credential}{Python}

%%%%
\summary

Request validation of a credential by the PMIx server/SMS.

%%%%
\format

\versionMarker{4.0}
\pyspecificstart
\begin{codepar}
rc,info = myclient.validate_credential(cred:dict, directives:list)
\end{codepar}
\pyspecificend

\begin{arglist}
\argin{cred}{Python \refpy{byteobject} containing credential (dict)}
\argin{directives}{Optional list of Python \refpy{info} describing request (list)}
\end{arglist}

Returns:

\begin{itemize}
    \item \refarg{rc} - \refconst{PMIX_SUCCESS} or a negative value corresponding to a PMIx error constant (integer)
    \item \refarg{info} - List of Python \refpy{info} containing additional results of the request (list)
\end{itemize}

See \refapi{PMIx_Validate_credential} for description of all relevant attributes and behaviors.


%%%%%%%%%%%%%%%%%%%%%%%%%%%%%%%%%%%%%%%%%%%%%%%%%
\subsection{Client.group_construct}
\declareapibinding{PMIxClient.group_construct}{PMIx_Group_construct}{Python}

%%%%
\summary

Construct a new group composed of the specified processes and identified with
the provided group identifier.

%%%%
\format

\versionMarker{4.0}
\pyspecificstart
\begin{codepar}
rc,info = myclient.construct_group(grp:string,
                        members:list, directives:list)
\end{codepar}
\pyspecificend

\begin{arglist}
\argin{grp}{Python string identifier for the group (str)}
\argin{members}{List of Python \refpy{proc} dictionaries identifying group members (list)}
\argin{directives}{Optional list of Python \refpy{info} describing request (list)}
\end{arglist}

Returns:

\begin{itemize}
    \item \refarg{rc} - \refconst{PMIX_SUCCESS} or a negative value corresponding to a PMIx error constant (integer)
    \item \refarg{info} - List of Python \refpy{info} containing results of the request (list)
\end{itemize}

See \refapi{PMIx_Group_construct} for description of all relevant attributes and behaviors.


%%%%%%%%%%%%%%%%%%%%%%%%%%%%%%%%%%%%%%%%%%%%%%%%%
\subsection{Client.group_invite}
\declareapibinding{PMIxClient.group_invite}{PMIx_Group_invite}{Python}

%%%%
\summary

Explicitly invite specified processes to join a group.

%%%%
\format

\versionMarker{4.0}
\pyspecificstart
\begin{codepar}
rc,info = myclient.group_invite(grp:string,
                        members:list, directives:list)
\end{codepar}
\pyspecificend

\begin{arglist}
\argin{grp}{Python string identifier for the group (str)}
\argin{members}{List of Python \refpy{proc} dictionaries identifying processes to be invited (list)}
\argin{directives}{Optional list of Python \refpy{info} describing request (list)}
\end{arglist}

Returns:

\begin{itemize}
    \item \refarg{rc} - \refconst{PMIX_SUCCESS} or a negative value corresponding to a PMIx error constant (integer)
    \item \refarg{info} - List of Python \refpy{info} containing results of the request (list)
\end{itemize}

See \refapi{PMIx_Group_invite} for description of all relevant attributes and behaviors.


%%%%%%%%%%%%%%%%%%%%%%%%%%%%%%%%%%%%%%%%%%%%%%%%%
\subsection{Client.group_join}
\declareapibinding{PMIxClient.group_join}{PMIx_Group_join}{Python}

%%%%
\summary

Respond to an invitation to join a group that is being asynchronously constructed.

%%%%
\format

\versionMarker{4.0}
\pyspecificstart
\begin{codepar}
rc,info = myclient.group_join(grp:string,
                        leader:dict, opt:integer,
                        directives:list)
\end{codepar}
\pyspecificend

\begin{arglist}
\argin{grp}{Python string identifier for the group (str)}
\argin{leader}{Python \refpy{proc} dictionary identifying process leading the group (dict)}
\argin{opt}{One of the \refstruct{pmix_group_opt_t} values indicating decline/accept (integer)}
\argin{directives}{Optional list of Python \refpy{info} describing request (list)}
\end{arglist}

Returns:

\begin{itemize}
    \item \refarg{rc} - \refconst{PMIX_SUCCESS} or a negative value corresponding to a PMIx error constant (integer)
    \item \refarg{info} - List of Python \refpy{info} containing results of the request (list)
\end{itemize}

See \refapi{PMIx_Group_join} for description of all relevant attributes and behaviors.


%%%%%%%%%%%%%%%%%%%%%%%%%%%%%%%%%%%%%%%%%%%%%%%%%
\subsection{Client.group_leave}
\declareapibinding{PMIxClient.group_leave}{PMIx_Group_leave}{Python}

%%%%
\summary

Leave a PMIx Group.

%%%%
\format

\versionMarker{4.0}
\pyspecificstart
\begin{codepar}
rc = myclient.group_leave(grp:string, directives:list)
\end{codepar}
\pyspecificend

\begin{arglist}
\argin{grp}{Python string identifier for the group (str)}
\argin{directives}{Optional list of Python \refpy{info} describing request (list)}
\end{arglist}

Returns:

\begin{itemize}
    \item \refarg{rc} - \refconst{PMIX_SUCCESS} or a negative value corresponding to a PMIx error constant (integer)
\end{itemize}

See \refapi{PMIx_Group_leave} for description of all relevant attributes and behaviors.


%%%%%%%%%%%%%%%%%%%%%%%%%%%%%%%%%%%%%%%%%%%%%%%%%
\subsection{Client.group_destruct}
\declareapibinding{PMIxClient.group_destruct}{PMIx_Group_destruct}{Python}

%%%%
\summary

Destruct a PMIx Group.

%%%%
\format

\versionMarker{4.0}
\pyspecificstart
\begin{codepar}
rc = myclient.group_destruct(grp:string, directives:list)
\end{codepar}
\pyspecificend

\begin{arglist}
\argin{grp}{Python string identifier for the group (str)}
\argin{directives}{Optional list of Python \refpy{info} describing request (list)}
\end{arglist}

Returns:

\begin{itemize}
    \item \refarg{rc} - \refconst{PMIX_SUCCESS} or a negative value corresponding to a PMIx error constant (integer)
\end{itemize}

See \refapi{PMIx_Group_destruct} for description of all relevant attributes and behaviors.


%%%%%%%%%%%%%%%%%%%%%%%%%%%%%%%%%%%%%%%%%%%%%%%%%
\subsection{Client.register_event_handler}
\declareapibinding{PMIxClient.register_event_handler}{PMIx_Register_event_handler}{Python}

%%%%
\summary

Register an event handler to report events.

%%%%
\format

\versionMarker{4.0}
\pyspecificstart
\begin{codepar}
rc,id = myclient.register_event_handler(codes:list,
                        directives:list, cbfunc)
\end{codepar}
\pyspecificend

\begin{arglist}
\argin{codes}{List of Python integer status codes that should be reported to this handler (llist)}
\argin{directives}{Optional list of Python \refpy{info} describing request (list)}
\argin{cbfunc}{Python \refpy{evhandler} to be called when event is received (func)}
\end{arglist}

Returns:

\begin{itemize}
    \item \refarg{rc} - \refconst{PMIX_SUCCESS} or a negative value corresponding to a PMIx error constant (integer)
    \item \refarg{id} - \ac{PMIx} reference identifier for handler (integer)
\end{itemize}

See \refapi{PMIx_Register_event_handler} for description of all relevant attributes and behaviors.


%%%%%%%%%%%%%%%%%%%%%%%%%%%%%%%%%%%%%%%%%%%%%%%%%
\subsection{Client.deregister_event_handler}
\declareapibinding{PMIxClient.deregister_event_handler}{PMIx_Deregister_event_handler}{Python}

%%%%
\summary

Deregister an event handler.

%%%%
\format

\versionMarker{4.0}
\pyspecificstart
\begin{codepar}
myclient.deregister_event_handler(id:integer)
\end{codepar}
\pyspecificend

\begin{arglist}
\argin{id}{\ac{PMIx} reference identifier for handler (integer)}
\end{arglist}

Returns: None

See \refapi{PMIx_Deregister_event_handler} for description of all relevant attributes and behaviors.


%%%%%%%%%%%%%%%%%%%%%%%%%%%%%%%%%%%%%%%%%%%%%%%%%
\subsection{Client.notify_event}
\declareapibinding{PMIxClient.notify_event}{PMIx_Notify_event}{Python}

%%%%
\summary

Report an event for notification via any registered handler.

%%%%
\format

\versionMarker{4.0}
\pyspecificstart
\begin{codepar}
rc = myclient.notify_event(status:integer, source:dict,
                           range:integer, directives:list)
\end{codepar}
\pyspecificend

\begin{arglist}
\argin{status}{\ac{PMIx} status code indicating the event being reported (integer)}
\argin{source}{Python \refpy{proc} of the process that generated the event (dict)}
\argin{range}{Python \refpy{range} in which the event is to be reported (integer)}
\argin{directives}{Optional list of Python \refpy{info} dictionaries describing the event (list)}
\end{arglist}

Returns:
\begin{itemize}
    \item \refarg{rc} - \refconst{PMIX_SUCCESS} or a negative value corresponding to a PMIx error constant (integer)
\end{itemize}

See \refapi{PMIx_Notify_event} for description of all relevant attributes and behaviors.


%%%%%%%%%%%%%%%%%%%%%%%%%%%%%%%%%%%%%%%%%%%%%%%%%
\subsection{Client.fabric_register}
\declareapibinding{PMIxClient.fabric_register}{PMIx_Fabric_register}{Python}

\summary
Register for access to fabric-related information, including communication cost matrix.

\format

\versionMarker{4.0}
\pyspecificstart
\begin{codepar}
rc,idx,fabricinfo = myclient.fabric_register(directives:list)
\end{codepar}
\pyspecificend


\begin{arglist}
\argin{directives}{Optional list of Python \refpy{info} containing directives (list)}
\end{arglist}

Returns:

\begin{itemize}
    \item \refarg{rc} - \refconst{PMIX_SUCCESS} or a negative value corresponding to a PMIx error constant (integer)
    \item \refarg{idx} - Index of the registered fabric (integer)
    \item \refarg{fabricinfo} - List of Python \refpy{info} containing fabric info (list)
\end{itemize}

See \refapi{PMIx_Fabric_register} for details.


%%%%%%%%%%%%%%%%%%%%%%%%%%%%%%%%%%%%%%%%%%%%%%%%%
\subsection{Client.fabric_update}
\declareapibinding{PMIxClient.fabric_update}{PMIx_Fabric_update}{Python}

\summary
Update fabric-related information, including communication cost matrix.

\format

\versionMarker{4.0}
\pyspecificstart
\begin{codepar}
rc,fabricinfo = myclient.fabric_update(idx:integer)
\end{codepar}
\pyspecificend


\begin{arglist}
\argin{idx}{Index of the registered fabric (list)}
\end{arglist}

Returns:

\begin{itemize}
    \item \refarg{rc} - \refconst{PMIX_SUCCESS} or a negative value corresponding to a PMIx error constant (integer)
    \item \refarg{fabricinfo} - List of Python \refpy{info} containing updated fabric info (list)
\end{itemize}

See \refapi{PMIx_Fabric_update} for details.


%%%%%%%%%%%%%%%%%%%%%%%%%%%%%%%%%%%%%%%%%%%%%%%%%
\subsection{Client.fabric_deregister}
\declareapibinding{PMIxClient.fabric_deregister}{PMIx_Fabric_deregister}{Python}

\summary
Deregister fabric.

\format

\versionMarker{4.0}
\pyspecificstart
\begin{codepar}
rc = myclient.fabric_deregister(idx:integer)
\end{codepar}
\pyspecificend


\begin{arglist}
\argin{idx}{Index of the registered fabric (list)}
\end{arglist}

Returns:

\begin{itemize}
    \item \refarg{rc} - \refconst{PMIX_SUCCESS} or a negative value corresponding to a PMIx error constant (integer)
\end{itemize}

See \refapi{PMIx_Fabric_deregister} for details.


%%%%%%%%%%%%%%%%%%%%%%%%%%%%%%%%%%%%%%%%%%%%%%%%%
\subsection{Client.fabric_get_vertex_info}
\declareapibinding{PMIxClient.fabric_get_vertex_info}{PMIx_Fabric_get_vertex_info}{Python}

\summary
Given a communication cost matrix index for a specified fabric, return an array of information describing the corresponding \ac{NIC}.

\format

\versionMarker{4.0}
\pyspecificstart
\begin{codepar}
rc,nicinfo = myclient.fabric_get_vertex_info(fabric:integer,
                        vertex:integer)
\end{codepar}
\pyspecificend


\begin{arglist}
\argin{fabric}{Index of the registered fabric (list)}
\argin{fabric}{Index of the vertex within that fabric (list)}
\end{arglist}

Returns:

\begin{itemize}
    \item \refarg{rc} - \refconst{PMIX_SUCCESS} or a negative value corresponding to a PMIx error constant (integer)
    \item \refarg{nicinfo} - List of Python \refpy{info} describing the referenced \ac{NIC} (list)
\end{itemize}

See \refapi{PMIx_Fabric_get_vertex_info} for details.


%%%%%%%%%%%%%%%%%%%%%%%%%%%%%%%%%%%%%%%%%%%%%%%%%
\subsection{Client.fabric_get_device_index}
\declareapibinding{PMIxClient.fabric_get_device_index}{PMIx_Fabric_get_device_index}{Python}

\summary
Given info describing a given vertex, return the corresponding communication cost matrix index.

\format

\versionMarker{4.0}
\pyspecificstart
\begin{codepar}
rc,index = myclient.fabric_get_device_index(fabric:integer, info:list)
\end{codepar}
\pyspecificend

\begin{arglist}
\argin{fabric}{Index of the registered fabric (list)}
\argin{info}{List of Python \refpy{info} containing vertex description (list)}
\end{arglist}

Returns:

\begin{itemize}
    \item \refarg{rc} - \refconst{PMIX_SUCCESS} or a negative value corresponding to a PMIx error constant (integer)
    \item \refarg{index} - Index of corresponding \ac{NIC} (integer)
\end{itemize}

See \refapi{PMIx_Fabric_get_device_index} for details.


%%%%%%%%%%%%%%%%%%%%%%%%%%%%%%%%%%%%%%%%%%%%%%%%%
\subsection{Client.load_topology}
\declareapibinding{PMIxClient.load_topology}{PMIx_Load_topology}{Python}

\summary
Load the local hardware topology into the \ac{PMIx} library.

\format

\versionMarker{4.0}
\pyspecificstart
\begin{codepar}
rc = myclient.load_topology()
\end{codepar}
\pyspecificend

Returns:

\begin{itemize}
    \item \refarg{rc} - \refconst{PMIX_SUCCESS} or a negative value corresponding to a PMIx error constant (integer)
\end{itemize}

See \refapi{PMIx_Load_topology} for details - note that the topology loaded into the \ac{PMIx} library may be utilized by \ac{PMIx} and other libraries, but is not accessible by Python.


%%%%%%%%%%%%%%%%%%%%%%%%%%%%%%%%%%%%%%%%%%%%%%%%%
\subsection{Client.get_relative_locality}
\declareapibinding{PMIxClient.get_relative_locality}{PMIx_Get_relative_locality}{Python}

\summary
Get the relative locality of two local processes.

\format

\versionMarker{4.0}
\pyspecificstart
\begin{codepar}
rc,locality = myclient.get_relative_locality(loc1:str, loc2:str)
\end{codepar}
\pyspecificend

\begin{arglist}
\argin{loc1}{Locality string of a process (str)}
\argin{loc2}{Locality string of a process (str)}
\end{arglist}


Returns:

\begin{itemize}
    \item \refarg{rc} - \refconst{PMIX_SUCCESS} or a negative value corresponding to a PMIx error constant (integer)
    \item \refarg{locality} - \refpy{locality} bitarray containing the relative locality of the two processes (bitarray)
\end{itemize}

See \refapi{PMIx_Get_relative_locality} for details.


%%%%%%%%%%%%%%%%%%%%%%%%%%%%%%%%%%%%%%%%%%%%%%%%%
\subsection{Client.error_string}
\declareapibinding{PMIxClient.error_string}{PMIx_Error_string}{Python}

%%%%
\summary

Pretty-print string representation of \refstruct{pmix_status_t}.

%%%%
\format

\versionMarker{4.0}
\pyspecificstart
\begin{codepar}
rep = myclient.error_string(status:integer)
\end{codepar}
\pyspecificend

\begin{arglist}
\argin{status}{\ac{PMIx} status code (integer)}
\end{arglist}

Returns:
\begin{itemize}
    \item \refarg{rep} - String representation of the provided status code (str)
\end{itemize}

See \refapi{PMIx_Error_string} for further details.


%%%%%%%%%%%%%%%%%%%%%%%%%%%%%%%%%%%%%%%%%%%%%%%%%
\subsection{Client.proc_state_string}
\declareapibinding{PMIxClient.proc_state_string}{PMIx_Proc_state_string}{Python}

%%%%
\summary

Pretty-print string representation of \refstruct{pmix_proc_state_t}.

%%%%
\format

\versionMarker{4.0}
\pyspecificstart
\begin{codepar}
rep = myclient.proc_state_string(state:integer)
\end{codepar}
\pyspecificend

\begin{arglist}
\argin{state}{\ac{PMIx} process state code (integer)}
\end{arglist}

Returns:
\begin{itemize}
    \item \refarg{rep} - String representation of the provided process state (str)
\end{itemize}

See \refapi{PMIx_Proc_state_string} for further details.


%%%%%%%%%%%%%%%%%%%%%%%%%%%%%%%%%%%%%%%%%%%%%%%%%
\subsection{Client.scope_string}
\declareapibinding{PMIxClient.scope_string}{PMIx_Scope_string}{Python}

%%%%
\summary

Pretty-print string representation of \refstruct{pmix_scope_t}.

%%%%
\format

\versionMarker{4.0}
\pyspecificstart
\begin{codepar}
rep = myclient.scope_string(scope:integer)
\end{codepar}
\pyspecificend

\begin{arglist}
\argin{scope}{\ac{PMIx} scope value (integer)}
\end{arglist}

Returns:
\begin{itemize}
    \item \refarg{rep} - String representation of the provided scope (str)
\end{itemize}

See \refapi{PMIx_Scope_string} for further details


%%%%%%%%%%%%%%%%%%%%%%%%%%%%%%%%%%%%%%%%%%%%%%%%%
\subsection{Client.persistence_string}
\declareapibinding{PMIxClient.persistence_string}{PMIx_Persistence_string}{Python}

%%%%
\summary

Pretty-print string representation of \refstruct{pmix_persistence_t}.

%%%%
\format

\versionMarker{4.0}
\pyspecificstart
\begin{codepar}
rep = myclient.persistence_string(persistence:integer)
\end{codepar}
\pyspecificend

\begin{arglist}
\argin{persistence}{\ac{PMIx} persistence value (integer)}
\end{arglist}

Returns:
\begin{itemize}
    \item \refarg{rep} - String representation of the provided persistence (str)
\end{itemize}

See \refapi{PMIx_Persistence_string} for further details.


%%%%%%%%%%%%%%%%%%%%%%%%%%%%%%%%%%%%%%%%%%%%%%%%%
\subsection{Client.data_range_string}
\declareapibinding{PMIxClient.data_range_string}{PMIx_Data_range_string}{Python}

%%%%
\summary

Pretty-print string representation of \refstruct{pmix_data_range_t}.

%%%%
\format

\versionMarker{4.0}
\pyspecificstart
\begin{codepar}
rep = myclient.data_range_string(range:integer)
\end{codepar}
\pyspecificend

\begin{arglist}
\argin{range}{\ac{PMIx} data range value (integer)}
\end{arglist}

Returns:
\begin{itemize}
    \item \refarg{rep} - String representation of the provided data range (str)
\end{itemize}

See \refapi{PMIx_Data_range_string} for further details.


%%%%%%%%%%%%%%%%%%%%%%%%%%%%%%%%%%%%%%%%%%%%%%%%%
\subsection{Client.info_directives_string}
\declareapibinding{PMIxClient.info_directives_string}{PMIx_Info_directives_string}{Python}

%%%%
\summary

Pretty-print string representation of \refstruct{pmix_info_directives_t}.

%%%%
\format

\versionMarker{4.0}
\pyspecificstart
\begin{codepar}
rep = myclient.info_directives_string(directives:bitarray)
\end{codepar}
\pyspecificend

\begin{arglist}
\argin{directives}{\ac{PMIx} \refpy{info directives} value (bitarray)}
\end{arglist}

Returns:
\begin{itemize}
    \item \refarg{rep} - String representation of the provided info directives (str)
\end{itemize}

See \refapi{PMIx_Info_directives_string} for further details.


%%%%%%%%%%%%%%%%%%%%%%%%%%%%%%%%%%%%%%%%%%%%%%%%%
\subsection{Client.data_type_string}
\declareapibinding{PMIxClient.data_type_string}{PMIx_Data_type_string}{Python}

%%%%
\summary

Pretty-print string representation of \refstruct{pmix_data_type_t}.

%%%%
\format

\versionMarker{4.0}
\pyspecificstart
\begin{codepar}
rep = myclient.data_type_string(dtype:integer)
\end{codepar}
\pyspecificend

\begin{arglist}
\argin{dtype}{\ac{PMIx} datatype value (integer)}
\end{arglist}

Returns:
\begin{itemize}
    \item \refarg{rep} - String representation of the provided datatype (str)
\end{itemize}

See \refapi{PMIx_Data_type_string} for further details.


%%%%%%%%%%%%%%%%%%%%%%%%%%%%%%%%%%%%%%%%%%%%%%%%%
\subsection{Client.alloc_directive_string}
\declareapibinding{PMIxClient.alloc_directive_string}{PMIx_Alloc_directive_string}{Python}

%%%%
\summary

Pretty-print string representation of \refstruct{pmix_alloc_directive_t}.

%%%%
\format

\versionMarker{4.0}
\pyspecificstart
\begin{codepar}
rep = myclient.alloc_directive_string(adir:integer)
\end{codepar}
\pyspecificend

\begin{arglist}
\argin{adir}{\ac{PMIx} allocation directive value (integer)}
\end{arglist}

Returns:
\begin{itemize}
    \item \refarg{rep} - String representation of the provided allocation directive (str)
\end{itemize}

See \refapi{PMIx_Alloc_directive_string} for further details.


%%%%%%%%%%%%%%%%%%%%%%%%%%%%%%%%%%%%%%%%%%%%%%%%%
\subsection{Client.iof_channel_string}
\declareapibinding{PMIxClient.iof_channel_string}{PMIx_IOF_channel_string}{Python}

%%%%
\summary

Pretty-print string representation of \refstruct{pmix_iof_channel_t}.

%%%%
\format

\versionMarker{4.0}
\pyspecificstart
\begin{codepar}
rep = myclient.iof_channel_string(channel:bitarray)
\end{codepar}
\pyspecificend

\begin{arglist}
\argin{channel}{\ac{PMIx} IOF \refpy{channel} value (bitarray)}
\end{arglist}

Returns:
\begin{itemize}
    \item \refarg{rep} - String representation of the provided IOF channel (str)
\end{itemize}

See \refapi{PMIx_IOF_channel_string} for further details.


%%%%%%%%%%%%%%%%%%%%%%%%%%%%%%%%%%%%%%%%%%%%%%%%%
\subsection{Client.job_state_string}
\declareapibinding{PMIxClient.job_state_string}{PMIx_Job_state_string}{Python}

%%%%
\summary

Pretty-print string representation of \refstruct{pmix_job_state_t}.

%%%%
\format

\versionMarker{4.0}
\pyspecificstart
\begin{codepar}
rep = myclient.job_state_string(state:integer)
\end{codepar}
\pyspecificend

\begin{arglist}
\argin{state}{\ac{PMIx} job state value (integer)}
\end{arglist}

Returns:
\begin{itemize}
    \item \refarg{rep} - String representation of the provided job state (str)
\end{itemize}

See \refapi{PMIx_Job_state_string} for further details.


%%%%%%%%%%%%%%%%%%%%%%%%%%%%%%%%%%%%%%%%%%%%%%%%%
\subsection{Client.get_attribute_string}
\declareapibinding{PMIxClient.get_attribute_string}{PMIx_Get_attribute_string}{Python}

%%%%
\summary

Pretty-print string representation of a \ac{PMIx} attribute.

%%%%
\format

\versionMarker{4.0}
\pyspecificstart
\begin{codepar}
rep = myclient.get_attribute_string(attribute:str)
\end{codepar}
\pyspecificend

\begin{arglist}
\argin{attribute}{\ac{PMIx} attribute name (string)}
\end{arglist}

Returns:
\begin{itemize}
    \item \refarg{rep} - String representation of the provided attribute (str)
\end{itemize}

See \refapi{PMIx_Get_attribute_string} for further details.


%%%%%%%%%%%%%%%%%%%%%%%%%%%%%%%%%%%%%%%%%%%%%%%%%
\subsection{Client.get_attribute_name}
\declareapibinding{PMIxClient.get_attribute_name}{PMIx_Get_attribute_name}{Python}

%%%%
\summary

Pretty-print name of a \ac{PMIx} attribute corresponding to the provided string.

%%%%
\format

\versionMarker{4.0}
\pyspecificstart
\begin{codepar}
rep = myclient.get_attribute_name(attribute:str)
\end{codepar}
\pyspecificend

\begin{arglist}
\argin{attributestring}{Attribute string (string)}
\end{arglist}

Returns:
\begin{itemize}
    \item \refarg{rep} - Attribute name corresponding to the provided string (str)
\end{itemize}

See \refapi{PMIx_Get_attribute_name} for further details.


%%%%%%%%%%%%%%%%%%%%%%%%%%%%%%%%%%%%%%%%%%%%%%%%%
\subsection{Client.link_state_string}
\declareapibinding{PMIxClient.link_state_string}{PMIx_Link_state_string}{Python}

%%%%
\summary

Pretty-print string representation of \refstruct{pmix_link_state_t}.

%%%%
\format

\versionMarker{4.0}
\pyspecificstart
\begin{codepar}
rep = myclient.link_state_string(state:integer)
\end{codepar}
\pyspecificend

\begin{arglist}
\argin{state}{\ac{PMIx} link state value (integer)}
\end{arglist}

Returns:
\begin{itemize}
    \item \refarg{rep} - String representation of the provided link state (str)
\end{itemize}

See \refapi{PMIx_Link_state_string} for further details.



%%%%%%%%%%%%%%%%%%%%%%%%%%%%%%%%%%%%%%%%%%%%%%%%%
%%%%%%%%%%%%%%%%%%%%%%%%%%%%%%%%%%%%%%%%%%%%%%%%%
\section{PMIxServer}
\label{app:python:server}

The server Python class inherits the Python "client" class as its parent. Thus, it includes all client functions in addition to the ones defined in this section.

%%%%%%%%%%%%%%%%%%%%%%%%%%%%%%%%%%%%%%%%%%%%%%%%%
\subsection{Server.init}
\declareapibinding{PMIxServer.init}{PMIx_server_init}{Python}

\summary Initialize the \ac{PMIx} server library after obtaining a new PMIxServer object.

\format

\versionMarker{4.0}
\pyspecificstart
\begin{codepar}
rc = myserver.init(directives:list, map:dict)
\end{codepar}
\pyspecificend


\begin{arglist}
\argin{directives}{List of Python \refpy{info} dictionaries (list)}
\argin{map}{Python dictionary key-function pairs that map \refpy{server module} callback functions to provided implementations (dict)}
\end{arglist}

Returns:

\begin{itemize}
    \item \refarg{rc} - \refconst{PMIX_SUCCESS} or a negative value corresponding to a PMIx error constant (integer)
\end{itemize}

See \refapi{PMIx_server_init} for description of all relevant attributes and behaviors.


%%%%%%%%%%%%%%%%%%%%%%%%%%%%%%%%%%%%%%%%%%%%%%%%%
\subsection{Server.finalize}
\declareapibinding{PMIxServer.finalize}{PMIx_server_finalize}{Python}

\summary Finalize the \ac{PMIx} server library.

\format

\versionMarker{4.0}
\pyspecificstart
\begin{codepar}
rc = myserver.finalize()
\end{codepar}
\pyspecificend


Returns:

\begin{itemize}
    \item \refarg{rc} - \refconst{PMIX_SUCCESS} or a negative value corresponding to a PMIx error constant (integer)
\end{itemize}

See \refapi{PMIx_server_finalize} for details.


%%%%%%%%%%%%%%%%%%%%%%%%%%%%%%%%%%%%%%%%%%%%%%%%%
\subsection{Server.generate_regex}
\declareapibinding{PMIxServer.generate_regex}{PMIx_generate_regex}{Python}

\summary
Generate a regular expression representation of the input strings.

\format

\versionMarker{4.0}
\pyspecificstart
\begin{codepar}
rc,regex = myserver.generate_regex(input:list)
\end{codepar}
\pyspecificend


\begin{arglist}
\argin{input}{List of Python strings (e.g., node names)  (list)}
\end{arglist}

Returns:

\begin{itemize}
    \item \refarg{rc} - \refconst{PMIX_SUCCESS} or a negative value corresponding to a PMIx error constant (integer)
    \item \refarg{regex} - Python \code{bytearray} containing regular expression representation of the input list (\code{bytearray})
\end{itemize}

See \refapi{PMIx_generate_regex} for details.


%%%%%%%%%%%%%%%%%%%%%%%%%%%%%%%%%%%%%%%%%%%%%%%%%
\subsection{Server.generate_ppn}
\declareapibinding{PMIxServer.generate_ppn}{PMIx_generate_ppn}{Python}

\summary
Generate a regular expression representation of the input strings.

\format

\versionMarker{4.0}
\pyspecificstart
\begin{codepar}
rc,regex = myserver.generate_ppn(input:list)
\end{codepar}
\pyspecificend


\begin{arglist}
\argin{input}{List of Python strings, each string consisting of a comma-delimited list of ranks on each node, with the strings being in the same order as the node names provided to "generate_regex" (list)}
\end{arglist}

Returns:

\begin{itemize}
    \item \refarg{rc} - \refconst{PMIX_SUCCESS} or a negative value corresponding to a PMIx error constant (integer)
    \item \refarg{regex} - Python \code{bytearray} containing regular expression representation of the input list (\code{bytearray})
\end{itemize}

See \refapi{PMIx_generate_ppn} for details.


%%%%%%%%%%%%%%%%%%%%%%%%%%%%%%%%%%%%%%%%%%%%%%%%%
\subsection{Server.register_nspace}
\declareapibinding{PMIxServer.register_nspace}{PMIx_server_register_nspace}{Python}

\summary Setup the data about a particular namespace.

\format

\versionMarker{4.0}
\pyspecificstart
\begin{codepar}
rc = myserver.register_nspace(nspace:str,
                              nlocalprocs:integer,
                              directives:list)
\end{codepar}
\pyspecificend


\begin{arglist}
\argin{nspace}{Python string containing the namespace (str)}
\argin{nlocalprocs}{Number of local processes (integer)}
\argin{directives}{List of Python \refpy{info} dictionaries (list)}
\end{arglist}

Returns:

\begin{itemize}
    \item \refarg{rc} - \refconst{PMIX_SUCCESS} or a negative value corresponding to a PMIx error constant (integer)
\end{itemize}

See \refapi{PMIx_server_register_nspace} for description of all relevant attributes and behaviors.


%%%%%%%%%%%%%%%%%%%%%%%%%%%%%%%%%%%%%%%%%%%%%%%%%
\subsection{Server.deregister_nspace}
\declareapibinding{PMIxServer.deregister_nspace}{PMIx_server_deregister_nspace}{Python}

\summary

Deregister a namespace.

\format

\versionMarker{4.0}
\pyspecificstart
\begin{codepar}
myserver.deregister_nspace(nspace:str)
\end{codepar}
\pyspecificend


\begin{arglist}
\argin{nspace}{Python string containing the namespace (str)}
\end{arglist}

Returns: None

See \refapi{PMIx_server_deregister_nspace} for details.


%%%%%%%%%%%%%%%%%%%%%%%%%%%%%%%%%%%%%%%%%%%%%%%%%
\subsection{Server.register_client}
\declareapibinding{PMIxServer.register_client}{PMIx_server_register_client}{Python}

\summary
Register a client process with the PMIx server library.

\format

\versionMarker{4.0}
\pyspecificstart
\begin{codepar}
rc = myserver.register_client(proc:dict, uid:integer, gid:integer)
\end{codepar}
\pyspecificend


\begin{arglist}
\argin{proc}{Python \refpy{proc} dictionary identifying the client process (dict)}
\argin{uid}{Linux uid value for user executing client process (integer)}
\argin{gid}{Linux gid value for user executing client process (integer)}
\end{arglist}

Returns:

\begin{itemize}
    \item \refarg{rc} - \refconst{PMIX_SUCCESS} or a negative value corresponding to a PMIx error constant (integer)
\end{itemize}

See \refapi{PMIx_server_register_client} for details.


%%%%%%%%%%%%%%%%%%%%%%%%%%%%%%%%%%%%%%%%%%%%%%%%%
\subsection{Server.deregister_client}
\declareapibinding{PMIxServer.deregister_client}{PMIx_server_deregister_client}{Python}

\summary
Deregister a client process and purge all data relating to it.


\format

\versionMarker{4.0}
\pyspecificstart
\begin{codepar}
myserver.deregister_client(proc:dict)
\end{codepar}
\pyspecificend


\begin{arglist}
\argin{proc}{Python \refpy{proc} dictionary identifying the client process (dict)}
\end{arglist}

Returns: None

See \refapi{PMIx_server_deregister_client} for details.


%%%%%%%%%%%%%%%%%%%%%%%%%%%%%%%%%%%%%%%%%%%%%%%%%
\subsection{Server.setup_fork}
\declareapibinding{PMIxServer.setup_fork}{PMIx_server_setup_fork}{Python}

\summary
Setup the environment of a child process that is to be forked
by the host.

\format

\versionMarker{4.0}
\pyspecificstart
\begin{codepar}
rc = myserver.setup_fork(proc:dict, envin:dict)
\end{codepar}
\pyspecificend


\begin{arglist}
\argin{proc}{Python \refpy{proc} dictionary identifying the client process (dict)}
\arginout{envin}{Python dictionary containing the environment to be passed to the client (dict)}
\end{arglist}

Returns:

\begin{itemize}
    \item \refarg{rc} - \refconst{PMIX_SUCCESS} or a negative value corresponding to a PMIx error constant (integer)
\end{itemize}

See \refapi{PMIx_server_setup_fork} for details.


%%%%%%%%%%%%%%%%%%%%%%%%%%%%%%%%%%%%%%%%%%%%%%%%%
\subsection{Server.dmodex_request}
\declareapibinding{PMIxServer.dmodex_request}{PMIx_server_dmodex_request}{Python}

\summary
Function by which the host server can request modex data from the local PMIx server.

\format

\versionMarker{4.0}
\pyspecificstart
\begin{codepar}
rc,data = myserver.dmodex_request(proc:dict)
\end{codepar}
\pyspecificend


\begin{arglist}
\argin{proc}{Python \refpy{proc} dictionary identifying the process whose data is requested (dict)}
\end{arglist}

Returns:

\begin{itemize}
    \item \refarg{rc} - \refconst{PMIX_SUCCESS} or a negative value corresponding to a PMIx error constant (integer)
    \item \refarg{data} - Python \refpy{byteobject} containing the returned data (dict)
\end{itemize}

See \refapi{PMIx_server_dmodex_request} for details.


%%%%%%%%%%%%%%%%%%%%%%%%%%%%%%%%%%%%%%%%%%%%%%%%%
\subsection{Server.setup_application}
\declareapibinding{PMIxServer.setup_application}{PMIx_server_setup_application}{Python}

\summary
Function by which the resource manager can request application-specific setup data prior to launch of a \refterm{job}.

\format

\versionMarker{4.0}
\pyspecificstart
\begin{codepar}
rc,info = myserver.setup_application(nspace:str, directives:list)
\end{codepar}
\pyspecificend


\begin{arglist}
\argin{nspace}{Namespace whose setup information is being requested (str)}
\argin{directives}{Python list of \refpy{info} directives}
\end{arglist}

Returns:

\begin{itemize}
    \item \refarg{rc} - \refconst{PMIX_SUCCESS} or a negative value corresponding to a PMIx error constant (integer)
    \item \refarg{info} - Python list of \refpy{info} dictionaries containing the returned data (list)
\end{itemize}

See \refapi{PMIx_server_setup_application} for details.


%%%%%%%%%%%%%%%%%%%%%%%%%%%%%%%%%%%%%%%%%%%%%%%%%
\subsection{Server.register_attributes}
\declareapibinding{PMIxServer.register_attributes}{PMIx_Register_attributes}{Python}

\summary
Register host environment attribute support for a function.

\format

\versionMarker{4.0}
\pyspecificstart
\begin{codepar}
rc = myserver.register_attributes(function:str, attrs:list)
\end{codepar}
\pyspecificend


\begin{arglist}
\argin{function}{Name of the function (str)}
\argin{attrs}{Python list of \refpy{regattr} describing the supported attributes}
\end{arglist}

Returns:

\begin{itemize}
    \item \refarg{rc} - \refconst{PMIX_SUCCESS} or a negative value corresponding to a PMIx error constant (integer)
\end{itemize}

See \refapi{PMIx_Register_attributes} for details.


%%%%%%%%%%%%%%%%%%%%%%%%%%%%%%%%%%%%%%%%%%%%%%%%%
\subsection{Server.setup_local_support}
\declareapibinding{PMIxServer.setup_local_support}{PMIx_server_setup_local_support}{Python}

\summary
Function by which the local \ac{PMIx} server can perform any application-specific operations prior to spawning local clients of a given application.

\format

\versionMarker{4.0}
\pyspecificstart
\begin{codepar}
rc = myserver.setup_local_support(nspace:str, info:list)
\end{codepar}
\pyspecificend


\begin{arglist}
\argin{nspace}{Namespace whose setup information is being requested (str)}
\argin{info}{Python list of \refpy{info} containing the setup data (list)}
\end{arglist}

Returns:

\begin{itemize}
    \item \refarg{rc} - \refconst{PMIX_SUCCESS} or a negative value corresponding to a PMIx error constant (integer)
\end{itemize}

See \refapi{PMIx_server_setup_local_support} for details.


%%%%%%%%%%%%%%%%%%%%%%%%%%%%%%%%%%%%%%%%%%%%%%%%%
\subsection{Server.iof_deliver}
\declareapibinding{PMIxServer.iof_deliver}{PMIx_server_IOF_deliver}{Python}

\summary
Function by which the host environment can pass forwarded \ac{IO} to the \ac{PMIx} server library for distribution to its clients.

\format

\versionMarker{4.0}
\pyspecificstart
\begin{codepar}
rc = myserver.iof_deliver(source:dict, channel:integer,
                          data:dict, directives:list)
\end{codepar}
\pyspecificend


\begin{arglist}
\argin{source}{Python \refpy{proc} dictionary identifying the process who generated the data (dict)}
\argin{channel}{Python \refpy{channel} bitmask identifying IO channel of the provided data (integer)}
\argin{data}{Python \refpy{byteobject} containing the data (dict)}
\argin{directives}{Python list of \refpy{info} containing directives (list)}
\end{arglist}

Returns:

\begin{itemize}
    \item \refarg{rc} - \refconst{PMIX_SUCCESS} or a negative value corresponding to a PMIx error constant (integer)
\end{itemize}

See \refapi{PMIx_server_IOF_deliver} for details.


%%%%%%%%%%%%%%%%%%%%%%%%%%%%%%%%%%%%%%%%%%%%%%%%%
\subsection{Server.collect_inventory}
\declareapibinding{PMIxServer.collect_inventory}{PMIx_server_collect_inventory}{Python}

\summary
Collect inventory of resources on a node.

\format

\versionMarker{4.0}
\pyspecificstart
\begin{codepar}
rc,info = myserver.collect_inventory(directives:list)
\end{codepar}
\pyspecificend


\begin{arglist}
\argin{directives}{Optional Python list of \refpy{info} containing directives (list)}
\end{arglist}

Returns:

\begin{itemize}
    \item \refarg{rc} - \refconst{PMIX_SUCCESS} or a negative value corresponding to a PMIx error constant (integer)
    \item \refarg{info} - Python list of \refpy{info} containing the returned data (list)
\end{itemize}

See \refapi{PMIx_server_collect_inventory} for details.


%%%%%%%%%%%%%%%%%%%%%%%%%%%%%%%%%%%%%%%%%%%%%%%%%
\subsection{Server.deliver_inventory}
\declareapibinding{PMIxServer.deliver_inventory}{PMIx_server_deliver_inventory}{Python}

\summary
Pass collected inventory to the \ac{PMIx} server library for storage.

\format

\versionMarker{4.0}
\pyspecificstart
\begin{codepar}
rc = myserver.deliver_inventory(info:list, directives:list)
\end{codepar}
\pyspecificend


\begin{arglist}
\argin{info} - Python list of \refpy{info} dictionaries containing the inventory data (list)
\argin{directives}{Python list of \refpy{info} dictionaries containing directives (list)}
\end{arglist}

Returns:

\begin{itemize}
    \item \refarg{rc} - \refconst{PMIX_SUCCESS} or a negative value corresponding to a PMIx error constant (integer)
\end{itemize}

See \refapi{PMIx_server_deliver_inventory} for details.


%%%%%%%%%%%%%%%%%%%%%%%%%%%%%%%%%%%%%%%%%%%%%%%%%
\subsection{Server.generate_locality_string}
\declareapibinding{PMIxServer.generate_locality_string}{PMIx_server_generate_locality_string}{Python}

\summary
Generate a \ac{PMIx} locality string from a given cpuset.

\format

\versionMarker{4.0}
\pyspecificstart
\begin{codepar}
rc,locality = myserver.generate_locality_string(cpuset:\refpy{cpuset})
\end{codepar}
\pyspecificend


\begin{arglist}
\argin{cpuset} - Python \refpy{cpuset} identifying the \acp{PU} (\refpy{cpuset})
\end{arglist}

Returns:

\begin{itemize}
    \item \refarg{rc} - \refconst{PMIX_SUCCESS} or a negative value corresponding to a PMIx error constant (integer)
    \item \refarg{locality} - string representation of the locality corresponding to the provided \refarg{cpuset} (str)
\end{itemize}

See \refapi{PMIx_server_generate_locality_string} for details.


%%%%%%%%%%%%%%%%%%%%%%%%%%%%%%%%%%%%%%%%%%%%%%%%%
\subsection{Server.define_process_set}
\declareapibinding{PMIxServer.define_process_set}{PMIx_server_define_process_set}{Python}

\summary
Add members to a \ac{PMIx} process set.

\format

\versionMarker{4.0}
\pyspecificstart
\begin{codepar}
rc = myserver.define_process_set(members:list, name:str)
\end{codepar}
\pyspecificend


\begin{arglist}
\argin{members} - List of Python \refpy{proc} dictionaries identifying the processes to be added to the process set (list)
\argin{name} - Name of the process set (str)
\end{arglist}

Returns:

\begin{itemize}
    \item \refarg{rc} - \refconst{PMIX_SUCCESS} or a negative value corresponding to a PMIx error constant (integer)
\end{itemize}

See \refapi{PMIx_server_define_process_set} for details.


%%%%%%%%%%%%%%%%%%%%%%%%%%%%%%%%%%%%%%%%%%%%%%%%%
\subsection{Server.delete_process_set}
\declareapibinding{PMIxServer.delete_process_set}{PMIx_server_delete_process_set}{Python}

\summary
Delete members from a \ac{PMIx} process set.

\format

\versionMarker{4.0}
\pyspecificstart
\begin{codepar}
rc = myserver.delete_process_set(members:list, name:str)
\end{codepar}
\pyspecificend


\begin{arglist}
\argin{members} - List of Python \refpy{proc} dictionaries identifying the processes to be removed from the process set (list)
\argin{name} - Name of the process set (str)
\end{arglist}

Returns:

\begin{itemize}
    \item \refarg{rc} - \refconst{PMIX_SUCCESS} or a negative value corresponding to a PMIx error constant (integer)
\end{itemize}

See \refapi{PMIx_server_delete_process_set} for details.


%%%%%%%%%%%%%%%%%%%%%%%%%%%%%%%%%%%%%%%%%%%%%%%%%
%%%%%%%%%%%%%%%%%%%%%%%%%%%%%%%%%%%%%%%%%%%%%%%%%
\section{PMIxTool}
\label{app:python:tool}

The tool Python class inherits the Python "server" class as its parent. Thus, it includes all client and server functions in addition to the ones defined in this section.


%%%%%%%%%%%%%%%%%%%%%%%%%%%%%%%%%%%%%%%%%%%%%%%%%
\subsection{Tool.init}
\declareapibinding{PMIxTool.init}{PMIx_tool_init}{Python}

\summary

Initialize the \ac{PMIx} tool library after obtaining a new PMIxTool object.

\format

\versionMarker{4.0}
\pyspecificstart
\begin{codepar}
rc,proc = mytool.init(info:list)
\end{codepar}
\pyspecificend


\begin{arglist}
\argin{info}{List of Python \refpy{info} directives (list)}
\end{arglist}

Returns:

\begin{itemize}
    \item \refarg{rc} - \refconst{PMIX_SUCCESS} or a negative value corresponding to a PMIx error constant (integer)
    \item \refarg{proc} - a Python \refpy{proc} (dict)
\end{itemize}

See \refapi{PMIx_tool_init} for description of all relevant attributes and behaviors.


%%%%%%%%%%%%%%%%%%%%%%%%%%%%%%%%%%%%%%%%%%%%%%%%%
\subsection{Tool.finalize}
\declareapibinding{PMIxTool.finalize}{PMIx_tool_finalize}{Python}

\summary

Finalize the PMIx tool library, closing the connection to the server.

\format

\versionMarker{4.0}
\pyspecificstart
\begin{codepar}
rc = mytool.finalize()
\end{codepar}
\pyspecificend


Returns:

\begin{itemize}
    \item \refarg{rc} - \refconst{PMIX_SUCCESS} or a negative value corresponding to a PMIx error constant (integer)
\end{itemize}


See \refapi{PMIx_tool_finalize} for description of all relevant attributes and behaviors.


%%%%%%%%%%%%%%%%%%%%%%%%%%%%%%%%%%%%%%%%%%%%%%%%%
\subsection{Tool.disconnect}
\declareapibinding{PMIxTool.disconnect}{PMIx_tool_disconnect}{Python}

\summary

Disconnect the \ac{PMIx} tool from the specified server connection while leaving the tool library initialized.

\format

\versionMarker{4.0}
\pyspecificstart
\begin{codepar}
rc = mytool.disconnect(server:dict)
\end{codepar}
\pyspecificend

\begin{arglist}
\argin{server}{Process identifier of server from which the tool is to be disconnected (\refpy{proc})}
\end{arglist}

Returns:

\begin{itemize}
    \item \refarg{rc} - \refconst{PMIX_SUCCESS} or a negative value corresponding to a PMIx error constant (integer)
\end{itemize}

See \refapi{PMIx_tool_disconnect} for details.


%%%%%%%%%%%%%%%%%%%%%%%%%%%%%%%%%%%%%%%%%%%%%%%%%
\subsection{Tool.attach_to_server}
\declareapibinding{PMIxTool.attach_to_server}{PMIx_tool_attach_to_server}{Python}

\summary
Switch connection from the current \ac{PMIx} server to another one, or initialize a connection to a specified server.


\format

\versionMarker{4.0}
\pyspecificstart
\begin{codepar}
rc,proc,server = mytool.connect_to_server(info:list)
\end{codepar}
\pyspecificend


\begin{arglist}
\argin{info}{List of Python \refpy{info} dictionaries (list)}
\end{arglist}

Returns:

\begin{itemize}
    \item \refarg{rc} - \refconst{PMIX_SUCCESS} or a negative value corresponding to a PMIx error constant (integer)
    \item \refarg{proc} - a Python \refpy{proc} containing the tool's identifier (dict)
    \item \refarg{server} - a Python \refpy{proc} containing the identifier of the server to which the tool attached (dict)
\end{itemize}

See \refapi{PMIx_tool_attach_to_server} for details.


%%%%%%%%%%%%%%%%%%%%%%%%%%%%%%%%%%%%%%%%%%%%%%%%%
\subsection{Tool.get_servers}
\declareapibinding{PMIxTool.get_servers}{PMIx_tool_get_servers}{Python}

\summary
Get a list containing the \refpy{proc} process identifiers of all servers to which the tool is currently connected.


\format

\versionMarker{4.0}
\pyspecificstart
\begin{codepar}
rc,servers = mytool.get_servers()
\end{codepar}
\pyspecificend

Returns:

\begin{itemize}
    \item \refarg{rc} - \refconst{PMIX_SUCCESS} or a negative value corresponding to a PMIx error constant (integer)
    \item \refarg{servers} - a list of Python \refpy{proc} containing the identifiers of the servers to which the tool is currently attached (dict)
\end{itemize}

See \refapi{PMIx_tool_get_servers} for details.


%%%%%%%%%%%%%%%%%%%%%%%%%%%%%%%%%%%%%%%%%%%%%%%%%
\subsection{Tool.iof_pull}
\declareapibinding{PMIxTool.iof_pull}{PMIx_IOF_pull}{Python}

%%%%
\summary

Register to receive output forwarded from a remote process.

%%%%
\format

\versionMarker{4.0}
\pyspecificstart
\begin{codepar}
rc,id = mytool.iof_pull(sources:list, channel:integer,
                        directives:list, cbfunc)
\end{codepar}
\pyspecificend

\begin{arglist}
\argin{sources}{List of Python \refpy{proc} dictionaries of processes whose IO is being requested (list)}
\argin{channel}{Python \refpy{channel} bitmask identifying IO channels to be forwarded (integer)}
\argin{directives}{List of Python \refpy{info} dictionaries describing request (list)}
\argin{cbfunc}{Python \refpy{iofcbfunc} to receive IO payloads (func)}
\end{arglist}

Returns:

\begin{itemize}
    \item \refarg{rc} - \refconst{PMIX_SUCCESS} or a negative value corresponding to a PMIx error constant (integer)
    \item \refarg{id} - \ac{PMIx} reference identifier for request (integer)
\end{itemize}

See \refapi{PMIx_IOF_pull} for description of all relevant attributes and behaviors.


%%%%%%%%%%%%%%%%%%%%%%%%%%%%%%%%%%%%%%%%%%%%%%%%%
\subsection{Tool.iof_deregister}
\declareapibinding{PMIxTool.iof_deregister}{PMIx_IOF_deregister}{Python}

%%%%
\summary

Deregister from output forwarded from a remote process.

%%%%
\format

\versionMarker{4.0}
\pyspecificstart
\begin{codepar}
rc = mytool.iof_deregister(id:integer, directives:list)
\end{codepar}
\pyspecificend

\begin{arglist}
\argin{id}{\ac{PMIx} reference identifier returned by pull request (list)}
\argin{directives}{List of Python \refpy{info} dictionaries describing request (list)}
\end{arglist}

Returns:

\begin{itemize}
    \item \refarg{rc} - \refconst{PMIX_SUCCESS} or a negative value corresponding to a PMIx error constant (integer)
\end{itemize}

See \refapi{PMIx_IOF_deregister} for description of all relevant attributes and behaviors.


%%%%%%%%%%%%%%%%%%%%%%%%%%%%%%%%%%%%%%%%%%%%%%%%%
\subsection{Tool.iof_push}
\declareapibinding{PMIxTool.iof_push}{PMIx_IOF_push}{Python}

%%%%
\summary

Push data collected locally (typically from stdin) to
stdin of target recipients.

%%%%
\format

\versionMarker{4.0}
\pyspecificstart
\begin{codepar}
rc = mytool.iof_push(targets:list, data:dict, directives:list)
\end{codepar}
\pyspecificend

\begin{arglist}
\argin{sources}{List of Python \refpy{proc} of target processes (list)}
\argin{data}{Python \refpy{byteobject} containing data to be delivered (dict)}
\argin{directives}{Optional list of Python \refpy{info} describing request (list)}
\end{arglist}

Returns:

\begin{itemize}
    \item \refarg{rc} - \refconst{PMIX_SUCCESS} or a negative value corresponding to a PMIx error constant (integer)
\end{itemize}

See \refapi{PMIx_IOF_push} for description of all relevant attributes and behaviors.


%%%%%%%%%%%%%%%%%%%%%%%%%%%%%%%%%%%%%%%%%%%%%%%%%
%%%%%%%%%%%%%%%%%%%%%%%%%%%%%%%%%%%%%%%%%%%%%%%%%
\section{Example Usage}
\label{app:python:examples}

The following examples are provided to illustrate the use of the Python bindings.

\subsection{Python Client}

The following example contains a client program that illustrates a fairly common usage pattern. The program instantiates and initializes the PMIxClient class, posts some data that is to be shared across all processes in the job, executes a “fence” that circulates the data, and then retrieves a value posted by one of its peers. Note that the example has been formatted to fit the document layout.


\pyspecificstart
\begin{codepar}
from pmix import *

def main():
    # Instantiate a client object
    myclient = PMIxClient()
    print("Testing PMIx ", myclient.get_version())

    # Initialize the PMIx client library, declaring the programming model
    # as “TEST” and the library name as “PMIX”, just for the example
    info = [{'key':PMIX_PROGRAMMING_MODEL,
             'value':'TEST', 'val_type':PMIX_STRING},
            {'key':PMIX_MODEL_LIBRARY_NAME,
             'value':'PMIX', 'val_type':PMIX_STRING}]
    rc,myname = myclient.init(info)
    if PMIX_SUCCESS != rc:
        print("FAILED TO INIT WITH ERROR", myclient.error_string(rc))
        exit(1)

    # try posting a value
    rc = myclient.put(PMIX_GLOBAL, "mykey",
                      {'value':1, 'val_type':PMIX_INT32})
    if PMIX_SUCCESS != rc:
        print("PMIx_Put FAILED WITH ERROR", myclient.error_string(rc))
        # cleanly finalize
        myclient.finalize()
        exit(1)

    # commit it
    rc = myclient.commit()
    if PMIX_SUCCESS != rc:
        print("PMIx_Commit FAILED WITH ERROR",
              myclient.error_string(rc))
        # cleanly finalize
        myclient.finalize()
        exit(1)

    # execute fence across all processes in my job
    procs = []
    info = []
    rc = myclient.fence(procs, info)
    if PMIX_SUCCESS != rc:
        print("PMIx_Fence FAILED WITH ERROR", myclient.error_string(rc))
        # cleanly finalize
        myclient.finalize()
        exit(1)

    # Get a value from a peer
    if 0 != myname['rank']:
        info = []
        rc, get_val = myclient.get({'nspace':"testnspace", 'rank': 0},
                                   "mykey", info)
        if PMIX_SUCCESS != rc:
            print("PMIx_Commit FAILED WITH ERROR",
                  myclient.error_string(rc))
            # cleanly finalize
            myclient.finalize()
            exit(1)
        print("Get value returned: ", get_val)

    # test a fence that should return not_supported because
    # we pass a required attribute that the server is known
    # not to support
    procs = []
    info = [{'key': 'ARBIT', 'flags': PMIX_INFO_REQD,
             'value':10, 'val_type':PMIX_INT}]
    rc = myclient.fence(procs, info)
    if PMIX_SUCCESS == rc:
        print("PMIx_Fence SUCCEEDED BUT SHOULD HAVE FAILED")
        # cleanly finalize
        myclient.finalize()
        exit(1)

    # Publish something
    info = [{'key': 'ARBITRARY', 'value':10, 'val_type':PMIX_INT}]
    rc = myclient.publish(info)
    if PMIX_SUCCESS != rc:
        print("PMIx_Publish FAILED WITH ERROR",
              myclient.error_string(rc))
        # cleanly finalize
        myclient.finalize()
        exit(1)

    # finalize
    info = []
    myclient.finalize(info)
    print("Client finalize complete")

# Python main program entry point
if __name__ == '__main__':
    main()
\end{codepar}
\pyspecificend


%%%%%%%%%%%%%%%%%%%%%%%%%%%%%%%%%%%%%%%%%%%%%%%%%
\subsection{Python Server}

The following example contains a minimum-level server host program that instantiates and initializes the PMIxServer class. The program illustrates passing several server module functions to the bindings and includes code to setup and spawn a simple client application, waiting until the spawned client terminates before finalizing and exiting itself. Note that the example has been formatted to fit the document layout.

\pyspecificstart
\begin{codepar}
from pmix import *
import signal, time
import os
import select
import subprocess

def clientconnected(proc:tuple is not None):
    print("CLIENT CONNECTED", proc)
    return PMIX_OPERATION_SUCCEEDED

def clientfinalized(proc:tuple is not None):
    print("CLIENT FINALIZED", proc)
    return PMIX_OPERATION_SUCCEEDED

def clientfence(procs:list, directives:list, data:bytearray):
    # check directives
    if directives is not None:
        for d in directives:
            # these are each an info dict
            if "pmix" not in d['key']:
                # we do not support such directives - see if
                # it is required
                try:
                    if d['flags'] & PMIX_INFO_REQD:
                        # return an error
                        return PMIX_ERR_NOT_SUPPORTED
                except:
                    #it can be ignored
                    pass
    return PMIX_OPERATION_SUCCEEDED

def main():
    try:
        myserver = PMIxServer()
    except:
        print("FAILED TO CREATE SERVER")
        exit(1)
    print("Testing server version ", myserver.get_version())

    args = [{'key':PMIX_SERVER_SCHEDULER,
             'value':'T', 'val_type':PMIX_BOOL}]
    map = {'clientconnected': clientconnected,
           'clientfinalized': clientfinalized,
           'fencenb': clientfence}
    my_result = myserver.init(args, map)

    # get our environment as a base
    env = os.environ.copy()

    # register an nspace for the client app
    (rc, regex) = myserver.generate_regex("test000,test001,test002")
    (rc, ppn) = myserver.generate_ppn("0")
    kvals = [{'key':PMIX_NODE_MAP,
              'value':regex, 'val_type':PMIX_STRING},
             {'key':PMIX_PROC_MAP,
              'value':ppn, 'val_type':PMIX_STRING},
             {'key':PMIX_UNIV_SIZE,
              'value':1, 'val_type':PMIX_UINT32},
             {'key':PMIX_JOB_SIZE,
              'value':1, 'val_type':PMIX_UINT32}]
    rc = foo.register_nspace("testnspace", 1, kvals)
    print("RegNspace ", rc)

    # register a client
    uid = os.getuid()
    gid = os.getgid()
    rc = myserver.register_client({'nspace':"testnspace", 'rank':0},
                                  uid, gid)
    print("RegClient ", rc)
    # setup the fork
    rc = myserver.setup_fork({'nspace':"testnspace", 'rank':0}, env)
    print("SetupFrk", rc)

    # setup the client argv
    args = ["./client.py"]
    # open a subprocess with stdout and stderr
    # as distinct pipes so we can capture their
    # output as the process runs
    p = subprocess.Popen(args, env=env,
        stdout=subprocess.PIPE, stderr=subprocess.PIPE)
    # define storage to catch the output
    stdout = []
    stderr = []
    # loop until the pipes close
    while True:
        reads = [p.stdout.fileno(), p.stderr.fileno()]
        ret = select.select(reads, [], [])

        stdout_done = True
        stderr_done = True

        for fd in ret[0]:
            # if the data
            if fd == p.stdout.fileno():
                read = p.stdout.readline()
                if read:
                    read = read.decode('utf-8').rstrip()
                    print('stdout: ' + read)
                    stdout_done = False
            elif fd == p.stderr.fileno():
                read = p.stderr.readline()
                if read:
                    read = read.decode('utf-8').rstrip()
                    print('stderr: ' + read)
                    stderr_done = False

        if stdout_done and stderr_done:
            break
    print("FINALIZING")
    myserver.finalize()


if __name__ == '__main__':
    main()
\end{codepar}
\pyspecificend

%%%%%%%%%%%%%%%%%%%%%%%%%%%%%%%%%%%%%%%%%%%%%%%%%


    % Use-Cases
    %%%%%%%%%%%%%%%%%%%%%%%%%%%%%%%%%%%%%%%%%%%%%%%%%
% Appendix: Use Cases
%%%%%%%%%%%%%%%%%%%%%%%%%%%%%%%%%%%%%%%%%%%%%%%%%
\chapter{Use-Cases}
\label{app:use-cases}

The \ac{PMIx} standard provides many generic interfaces that can be composed into higher-level use cases in a variety of ways. While the specific interfaces and attributes are standardized, the use cases themselves are not (and should not) be standardized. Common use cases are included here as examples of how PMIx's generic interfaces \textit{might} be composed together for a higher-level purpose. The use cases are intended for both \ac{PMIx} interface users and library implementors. Whereby a better understanding of the general usage model within the community can help users picking up PMIx for the first and help implementors optimize their implementation for the common cases.

Each use case is structured to provide background information about the high-level use case as well as specific details about how the PMIx interfaces are used within the use case. Some use cases even provide code snippets. These code snippets are apart of larger code examples located within the standard's source code repository, and each complete code example is fully compilable and runnable. The related interfaces and attributes collected at the bottom of each use case are mainly for conveinence and link to the full standardized definitions.

%%%%%%%%%%%%%%%%%%%%%%%%%%%%%%%%%%%%%%%%%%%%%%
%%%%%%%%%%%%%%%%%%%%%%%%%%%%%%%%%%%%%%%%%%%%%%
\section {Business Card Exchange for Process-to-Process Wire-up}
\label{app:uc-business-card-exchange}

\subsection{Use Case Summary}

Multi-process communication libraries, such as MPI, need to establish communication channels between a set of those processes. In this scenario, each process needs to share connectivity information (a.k.a. Business Cards) with all other processes before communication channels can be established. This connectivity information may take the form of one or more unique strings that allow a different process to establish a communication channel with the originator. The runtime environment must provide a mechanism for the efficient exchange of this connectivity information. Additional information about the current state of the job (e.g., number of processes globally and locally) and of how the process was started (e.g., process binding) is also helpful.

Note: The Instant-On wire-up mechanism is a separate, related use case.

\subsection{Use Case Details}

Each process provides their business card to PMIx via one or more \refapi{PMIx_Put} operations to store the tuple of \code{\{UID, key, value\}}. The \code{UID} is the unique name for this process in the \ac{PMIx} universe (i.e., \code{namespace} and \code{rank}). The \code{key} is a unique key that other processes can reference generically (note that since the \code{UID} is also associated with the \code{key} there is no need to make the \code{key} uniquely named per process). The \code{value} is the string representation of the connectivity information.

Some business card information is meant for remote processes (e.g., TCP or InfiniBand addresses) while others are meant only for local processes (e.g., shared memory information). As such a \code{scope} should be associated with the \refapi{PMIx_Put} operation to differentiate this intention.

The \refapi{PMIx_Put} operations may be cached local to the process. Once all \refapi{PMIx_Put} operations have been called each process should call \refapi{PMIx_Commit} to push those values to the local PMIx server. Note that in a multi-library configuration each library may \refapi{PMIx_Put} then \refapi{PMIx_Commit} values - so there may be multiple \refapi{PMIx_Commit} calls before a Business Card Exchange is activated.

After calling \refapi{PMIx_Commit} a process can activate the Business Card Exchange collective operation by calling \refapi{PMIx_Fence}. The \refapi{PMIx_Fence} operation is collective over the set of processes specified in the argument set. That allows for the collective to span a subset of a namespace or multiple namespaces. After the completion of the \refapi{PMIx_Fence} operation, the data stored by other processes via \refapi{PMIx_Put} is available to the local process through a call to \refapi{PMIx_Get} which returns the key/value pairs necessary to establish the connection(s) with the other processes.

The \refapi{PMIx_Fence} operation has a "Synchronize Only" mode that works as a barrier operation. This is helpful if the communication library requires a synchronization before leaving initialization or starting finalization, for example.

The \refapi{PMIx_Fence} operation has a "Sparse" mode in addition to a "Full" mode for the data exchange. The "Full" mode will fully exchange all Business Card information with all other processes. This is helpful for tightly communicating applications. The "Sparse" mode will dynamically pull the connectivity information on-demand from inside of \refapi{PMIx_Get} (if it is not already available locally). This is helpful for sparsely communicating applications. Since which mode is best for an application cannot be inferred by the PMIx library the caller must specify which mode works best for their application. The \refapi{PMIx_Fence} operation has an option for the end user to specify which mode they desire for this operation.

Additional information about the current state of the job (e.g., number of processes globally and locally) and of how the process was started (e.g., process binding) is also helpful. This "job level" information is available immediately after \refapi{PMIx_Init} without the need for any explicit synchronization.

The number of processes globally in the namespace and this process's rank within that namespace is important to know before establishing the Business Card information to best allocate resources.

The number of processes local to the node and this process's local rank is important to know before establishing the Business Card information to help the caller determine the scope of the put operation. For example, to designate a leader to set up a shared memory segment of the proper size before putting that information into the locally scoped Business Card information.

The number of processes local to a remote node is also helpful to know before establishing the Business Card information. This information is useful to pre-establish local resources before that remote node starts to initiate a connection or to determine the number of connections that need to be advertised in the Business Card when it is sent out.

Note that some of the job level information may change over the course of the job in a dynamic application.

\littleheader{Related Interfaces}

{\large \refapi{PMIx_Put}}
\pasteSignature{PMIx_Put}

{\large \refapi{PMIx_Get}}
\pasteSignature{PMIx_Get}

{\large \refapi{PMIx_Commit}}
\pasteSignature{PMIx_Commit}

{\large \refapi{PMIx_Fence}}
\pasteSignature{PMIx_Fence}

{\large \refapi{PMIx_Init}}
\pasteSignature{PMIx_Init}

\littleheader{Related Attributes}

The following job level information is useful to have before establishing Business Card information:

\pasteAttributeItem{PMIX_NODE_LIST}
\pasteAttributeItem{PMIX_NUM_NODES}
\pasteAttributeItem{PMIX_NODEID}
\pasteAttributeItem{PMIX_JOB_SIZE}
\pasteAttributeItem{PMIX_PROC_MAP}
\pasteAttributeItem{PMIX_LOCAL_PEERS}
\pasteAttributeItem{PMIX_LOCAL_SIZE}

For each process this information is also useful (note that any one process may want to access this list of information about any other process in the system):

\pasteAttributeItem{PMIX_RANK}
\pasteAttributeItem{PMIX_LOCAL_RANK}
\pasteAttributeItem{PMIX_GLOBAL_RANK}
\pasteAttributeItem{PMIX_LOCALITY_STRING}
\pasteAttributeItem{PMIX_HOSTNAME}

There are other keys that are helpful to have before a synchronization point. This is not meant to be a comprehensive list.

\section{Debugging}
\label{app:uc-debugging}

\subsection{Terminology}

\subsubsection{Tools vs Debuggers}

A \textit{tool} is a process designed to monitor, record, analyze, or control the execution of another process.  Typically used for the purposes of profiling and debugging. A \textit{first-party tool} runs within the address space of the application process while a \textit{third-party tool} run within its own process. A \textit{debugger} is a third-party tool that inspects and controls an application process's execution using system-level debug APIs (e.g., \code{ptrace}).

\subsubsection{Parallel Launching Methods}

A \textit{starter} program is a program responsible for launching a parallel runtime, such as \ac{MPI}. \ac{PMIx} supports two primary methods for launching parallel applications under tools and debuggers: indirect and direct. In the indirect launching method (Section~\ref{chap:api_tools:indirect}, the tool is attached to the starter. In the direct launching method (Section~\ref{chap:api_tools:direct}, the tool takes the place of the starter.
\ac{PMIx} also supports attaching to already running programs via the \textit{Process Acquisition} interfaces (Section~\ref{subsubsec:process-acq}).

\subsubsection{Process Synchronization}

Process Synchronization is a technique tools use to start the processes of a parallel application such that the tools can still attach to the process early in its lifetime.  Said another away, the tool must be able to start the application processes without them ``running away'' from the tool.  In the case of \ac{MPI} (Version 3.1~\cite{mpi-3.1} or the MPI World Process in future versions), this means stopping the applications processes before they return from \code{MPI_Init} or \code{MPI_Init_thread}.

\subsubsection{Process Acquisition}\label{subsubsec:process-acq}

Process Acquisition is a technique tools use to locate all of the processes, local and remote, of a given parallel application.  This typically boils down to collecting the following information for every process in the parallel application: the hostname or IP of the machine running the process, the executable name, and the process ID.

\subsection{Use Case Details}
\subsubsection{Direct-Launch Debugger Tool}

PMIx can support the tool itself using the PMIx spawn options to control the app’s startup, including directing the RM/application as to when to block and wait for tool attachment, or stipulating that an interceptor library be preloaded. However, this means that the user is restricted to whatever command line options the tool vendor has provided for operations such as process placement and binding, which places a significant burden on the tool vendor. An example might look like the following: \code{dbgr -n 3 ./myapp}.

Assuming it is supported, co-launch of debugger daemons in this use-case is supported by adding a \code{pmix_app_t} to the \refapi{PMIx_Spawn} command, indicating that the resulting processes are debugger daemons by setting the \refattr{PMIX_DEBUGGER_DAEMONS} attribute.

\begingroup
\begin{figure*}
  \begin{center}
    \includegraphics[width=\textwidth,height=\textheight,keepaspectratio]{figs/direct-launch}
  \end{center}
  \caption{Interaction diagram showing an example of the Direct Launch mechanism}
  \label{fig:direct_launch}
\end{figure*}
\endgroup


\littleheader{Related Interfaces}

{\large \refapi{PMIx_tool_init}}
\pasteSignature{PMIx_tool_init}

{\large \refapi{PMIx_Register_event_handler}}
\pasteSignature{PMIx_Register_event_handler}

{\large \refapi{PMIx_Query_info}}
\pasteSignature{PMIx_Query_info}

{\large \refapi{PMIx_Spawn}}
\pasteSignature{PMIx_Spawn}

{\large \refapi{PMIx_Get}}
\pasteSignature{PMIx_Get}

{\large \refapi{PMIx_Notify_event}}
\pasteSignature{PMIx_Notify_event}

\littleheader{Related Attributes}

\pasteAttributeItem{PMIX_QUERY_SPAWN_SUPPORT}
\pasteAttributeItem{PMIX_QUERY_DEBUG_SUPPORT}
\pasteAttributeItem{PMIX_DEBUG_STOP_IN_INIT}
\pasteAttributeItem{PMIX_DEBUG_STOP_ON_EXEC}
\pasteAttributeItem{PMIX_DEBUG_DAEMONS_PER_PROC}
\pasteAttributeItem{PMIX_DEBUG_DAEMONS_PER_NODE}
\pasteAttributeItem{PMIX_COSPAWN_APP}
\pasteAttributeItem{PMIX_MAPBY}
\pasteAttributeItem{PMIX_FWD_STDOUT}
\pasteAttributeItem{PMIX_FWD_STDERR}
\pasteAttributeItem{PMIX_NOTIFY_COMPLETION}
\pasteAttributeItem{PMIX_SETUP_APP_ENVARS}
\pasteAttributeItem{PMIX_EVENT_AFFECTED_PROC}
\pasteAttributeItem{PMIX_DEBUGGER_DAEMONS}
\pasteAttributeItem{PMIX_DEBUG_TARGET}
\pasteAttributeItem{PMIX_DEBUG_WAIT_FOR_NOTIFY}
\pasteAttributeItem{PMIX_QUERY_LOCAL_PROC_TABLE}

\littleheader{Related Constants}

\refconst{PMIX_DEBUG_WAITING_FOR_NOTIFY} \\
\refconst{PMIX_DEBUGGER_RELEASE}

\subsubsection{Indirect-Launch Debugger Tool}

Executing a program under a tool using an intermediate launcher such as \code{mpiexec} can also be made possible. This requires some degree of coordination between the tool and the launcher. Ultimately, it is the launcher that is going to launch the application, and the tool must somehow inform the launcher (and the application) that this is being done in a debug session so that the application knows to ``block'' until the tool attaches to it.

In this operational mode, the user invokes a tool (typically on a non-compute, or ``head'', node) that in turn uses \code{mpiexec} to launch their application – a typical command line might look like the following: \code{dbgr -dbgoption mpiexec -n 32 ./myapp}.

\begingroup
\begin{figure*}
  \begin{center}
    \includegraphics[width=\textwidth,height=\textheight,keepaspectratio]{figs/indirect-launch}
  \end{center}
  \caption{Interaction diagram showing an example of the Indirect Launch mechanism}
  \label{fig:indirect_launch}
\end{figure*}
\endgroup


\littleheader{Related Interfaces}

{\large \refapi{PMIx_tool_init}}
\pasteSignature{PMIx_tool_init}

{\large \refapi{PMIx_Register_event_handler}}
\pasteSignature{PMIx_Register_event_handler}

{\large \refapi{PMIx_Spawn}}
\pasteSignature{PMIx_Spawn}

{\large \refapi{PMIx_Notify_event}}
\pasteSignature{PMIx_Notify_event}

{\large \refapi{PMIx_tool_attach_to_server}}
\pasteSignature{PMIx_tool_attach_to_server}

{\large \refapi{PMIx_Query_info}}
\pasteSignature{PMIx_Query_info}

{\large \refapi{PMIx_Get}}
\pasteSignature{PMIx_Get}

\littleheader{Related Attributes}

\pasteAttributeItem{PMIX_LAUNCH_DIRECTIVES}
\pasteAttributeItem{PMIX_SPAWN_TOOL}
\pasteAttributeItem{PMIX_COSPAWN_APP}
\pasteAttributeItem{PMIX_FWD_STDOUT}
\pasteAttributeItem{PMIX_FWD_STDERR}
\pasteAttributeItem{PMIX_SETUP_APP_ENVARS}
\pasteAttributeItem{PMIX_DEBUG_STOP_IN_INIT}
\pasteAttributeItem{PMIX_DEBUG_STOP_ON_EXEC}
\pasteAttributeItem{PMIX_DEBUG_DAEMONS_PER_PROC}
\pasteAttributeItem{PMIX_DEBUG_DAEMONS_PER_NODE}
\pasteAttributeItem{PMIX_MAPBY}
\pasteAttributeItem{PMIX_QUERY_PROC_TABLE}
\pasteAttributeItem{PMIX_QUERY_LOCAL_PROC_TABLE}
\pasteAttributeItem{PMIX_DEBUGGER_DAEMONS}
\pasteAttributeItem{PMIX_NOTIFY_COMPLETION}
\pasteAttributeItem{PMIX_DEBUG_TARGET}
\pasteAttributeItem{PMIX_WAIT_FOR_CONNECTION}

\littleheader{Related Constants}

\refconst{PMIX_LAUNCHER_READY} \\
\refconst{PMIX_LAUNCH_COMPLETE} \\
\refconst{PMIX_DEBUG_WAITING_FOR_NOTIFY} \\
\refconst{PMIX_DEBUGGER_RELEASE} \\
\refenvar{PMIX_LAUNCHER_RNDZ_URI}

\subsubsection{Attaching to a Running Job}

PMIx supports attaching to an already running parallel job in two ways. In the first way, the main process of a tool calls \refapi{PMIx_Query_info} with the \refattr{PMIX_QUERY_PROC_TABLE} attribute. This returns an array of structs containing the information required for \hyperref[subsubsec:process-acq]{process acquisition}. This includes remote hostnames, executable names, and process IDs. In the second way, every tool daemon calls \refapi{PMIx_Query_info} with the \refattr{PMIX_QUERY_LOCAL_PROC_TABLE} attribute. This returns a similar array of structs but only for processes on the same node.

An example of this use-case may look like the following: \code{mpiexec -n~32~./myApp \&\& dbgr attach \$!}.

\begingroup
\begin{figure*}
  \begin{center}
    \includegraphics[width=\textwidth,height=\textheight,keepaspectratio]{figs/process-acquisition}
  \end{center}
  \caption{Interaction diagram showing an example of the attaching to a running job}
  \label{fig:proc_acq}
\end{figure*}
\endgroup

{\large \refapi{PMIx_tool_init}}
\pasteSignature{PMIx_tool_init}

{\large \refapi{PMIx_Register_event_handler}}
\pasteSignature{PMIx_Register_event_handler}

{\large \refapi{PMIx_Query_info}}
\pasteSignature{PMIx_Query_info}

{\large \refapi{PMIx_Spawn}}
\pasteSignature{PMIx_Spawn}

\littleheader{Related Attributes}

\pasteAttributeItem{PMIX_QUERY_PROC_TABLE}
\pasteAttributeItem{PMIX_DEBUGGER_DAEMONS}
\pasteAttributeItem{PMIX_DEBUG_TARGET}
\pasteAttributeItem{PMIX_DEBUG_DAEMONS_PER_PROC}
\pasteAttributeItem{PMIX_DEBUG_DAEMONS_PER_NODE}
\pasteAttributeItem{PMIX_MAPBY}
\pasteAttributeItem{PMIX_FWD_STDOUT}
\pasteAttributeItem{PMIX_FWD_STDERR}
\pasteAttributeItem{PMIX_NOTIFY_COMPLETION}
\pasteAttributeItem{PMIX_REQUESTOR_IS_TOOL}
\pasteAttributeItem{PMIX_QUERY_NAMESPACES}

\subsubsection{Tool Interaction with RM}

Tools can benefit from a mechanism by which they may interact with a local PMIx server that has opted to accept such connections along with support for tool connections to system-level PMIx servers, and a logging feature. To add support for tool connections to a specified system-level, PMIx server environments could choose to launch a set of PMIx servers to support a given allocation - these servers will (if so instructed) provide a tool rendezvous point that is tagged with their pid and typically placed in an allocation-specific temporary directory to allow for possible multi-tenancy scenarios. Supporting such operations requires that a system-level PMIx connection be provided which is not associated with a specific user or allocation. A new key has been added to direct the PMIx server to expose a rendezvous point specifically for this purpose.

{\large \refapi{PMIx_Query_info_nb}}
\pasteSignature{PMIx_Query_info_nb}

{\large \refapi{PMIx_Register_event_handler}}
\pasteSignature{PMIx_Register_event_handler}

{\large \refapi{PMIx_Deregister_event_handler}}
\pasteSignature{PMIx_Deregister_event_handler}

{\large \refapi{PMIx_Notify_event}}
\pasteSignature{PMIx_Notify_event}

{\large \refapi{PMIx_server_init}}
\pasteSignature{PMIx_server_init}

\subsubsection{Environmental Parameter Directives for Applications and Launchers}

It is sometimes desirable or required that standard environmental variables (e.g., \code{PATH}, \code{LD_LIBRARY_PATH}, \code{LD_PRELOAD}) be modified prior to executing an application binary or a starter such as \code{mpiexec} - this is particularly true when tools/debuggers are used to start the application.

\littleheader{Related Interfaces}

{\large \refapi{PMIx_Spawn}}
\pasteSignature{PMIx_Spawn}

\littleheader{Related Structs}

\refstruct{pmix_envar_t}

\littleheader{Related Attributes}

\pasteAttributeItem{PMIX_SET_ENVAR}
\pasteAttributeItem{PMIX_ADD_ENVAR}
\pasteAttributeItem{PMIX_UNSET_ENVAR}
\pasteAttributeItem{PMIX_PREPEND_ENVAR}
\pasteAttributeItem{PMIX_APPEND_ENVAR}

Resource managers and launchers must scan for relevant directives, modifying environmental parameters as directed. Directives are to be processed in the order in which they were given, starting with job-level directives (applied to each app) followed by app-level directives.

\section{Hybrid Applications}
\label{app:uc-hybrid-applications}

\subsection{Use Case Summary}

Hybrid applications (i.e., applications that utilize more than one programming model or runtime system, such as an application using MPI that also uses OpenMP or UPS) are growing in popularity, especially as processors with increasingly large numbers of cores and/or hardware threads proliferate. Unfortunately, the various corresponding runtime systems currently operate under the assumption that they alone control execution. This leads to conflicts in hybrid applications. Deadlock of parallel applications can occur when one runtime system prevents the other from making progress due to lack of coordination between them~\cite{2016:Hamidouche}. Sub-optimal performance can also occur due to uncoordinated division of hardware resources between the runtime systems implementing the different programming models or systems~\cite{ompix-moc,2018:Vallee}. This use-case offers potential solutions to this
problem by providing a pathway for parallel runtime systems to coordinate their actions.

\subsection{Use Case Details}

\subsubsection{Identifying Active Parallel Runtime Systems}

The current state-of-the-practice for concurrently used runtime systems in a single application to detect one another is via set environment variables. For example, some OpenMP implementations look for environment variables to indicate that an MPI library is active.  Unfortunately, this technique is not completely reliable as environment variables change over time and with new software versions, and this detection is implementation specific. Also, the fact that an environment variable is present doesn't guarantee that a particular runtime system is in active use since Resource Managers routinely set environment variables "just in case" the application needs them. PMIx provides a reliable mechanism by which each library can determine that another runtime library is in operation.

When initializing PMIx, runtime libraries implementing a parallel programming model can register themselves, including their name, the library version, the version of the API they implement, and the threading model.  This information is then cached locally and can then be read asynchronously by other runtime systems using PMIx's Event Notification system.

This initialization mechanism also allows runtime libraries to share knowledge of each other's resources and intended resource utilization. For example, if an OpenMP implementation knows which hardware threads an MPI library is using it could potentially avoid core and cache contention.

\littleheader{Code Example}

\pmixCodeImportC[]{sources/_autogen_/hybrid-prog-model.c_declare_model}

\littleheader{Related Interfaces}

{\large \refapi{PMIx_Init}}
\pasteSignature{PMIx_Init}

\littleheader{Related Attributes}

\pasteAttributeItem{PMIX_PROGRAMMING_MODEL}
\pasteAttributeItem{PMIX_MODEL_LIBRARY_NAME}
\pasteAttributeItem{PMIX_MODEL_LIBRARY_VERSION}
\pasteAttributeItem{PMIX_THREADING_MODEL}
\pasteAttributeItem{PMIX_MODEL_NUM_THREADS}
\pasteAttributeItem{PMIX_MODEL_NUM_CPUS}
\pasteAttributeItem{PMIX_MODEL_CPU_TYPE}
\pasteAttributeItem{PMIX_MODEL_PHASE_NAME}
\pasteAttributeItem{PMIX_MODEL_PHASE_TYPE}
\pasteAttributeItem{PMIX_MODEL_AFFINITY_POLICY}

\subsubsection{Coordinating at Runtime}

The PMIx Event Notification system provides a mechanism by which the resource manager can communicate system events to applications, thus providing applications with an opportunity to generate an appropriate response. Hybrid applications can leverage these events for cross-library coordination.

Runtime libraries can access the information provided by other runtime libraries during their initialization using the event notification system.  In this case, runtime libraries should register a callback for the \refconst{PMIX_MODEL_DECLARED} event.

Applications, runtime libraries, and resource managers can also use the PMIx event notification system to communicate dynamic information, such as entering a new application phase (\refattr{PMIX_MODEL_PHASE_NAME}) or a change in resources used (\refconst{PMIX_MODEL_RESOURCES}). This dynamic information can be broadcast using the \refapi{PMIx_Notify_event} function. Runtime libraries can register callback functions to run when these events occur using \refapi{PMIx_Register_event_handler}.

\littleheader{Code Example}

Registering a callback to run when another runtime library initializes:
\pmixCodeImportC[]{sources/_autogen_/hybrid-prog-model.c_declare_model_cb}


Notifying an event:
\pmixCodeImportC[]{sources/_autogen_/hybrid-prog-model.c_notify_event}

\littleheader{Related Interfaces}

{\large \refapi{PMIx_Notify_event}}
\pasteSignature{PMIx_Notify_event}

{\large \refapi{PMIx_Register_event_handler}}
\pasteSignature{PMIx_Register_event_handler}

{\large \refapi{pmix_event_notification_cbfunc_fn_t}}
\pasteSignature{pmix_event_notification_cbfunc_fn_t}

\littleheader{Related Constants}

\refconst{PMIX_MODEL_DECLARED} \\
\refconst{PMIX_MODEL_RESOURCES} \\
\refconst{PMIX_OPENMP_PARALLEL_ENTERED} \\
\refconst{PMIX_OPENMP_PARALLEL_EXITED} \\
\refconst{PMIX_EVENT_ACTION_COMPLETE}


\subsubsection{Coordinating at Runtime with Multiple Event Handlers}

Coordinating with a threading library such as an OpenMP runtime library creates the need for separate event handlers for threads of the same process. For example in an MPI+OpenMP hybrid application, the MPI main thread and the OpenMP primary thread may both want to be notified anytime an OpenMP thread starts executing in a parallel region. This requires support for multiple threads to potentially register different event handlers against the same status code.

Multiple event handlers registered against the same event are processed in a chain-like manner based on the order in which they were registered, as modified by any directives. Registrations against specific event codes are processed first, followed by registrations against multiple event codes and then any default registrations. At each point in the chain, an event handler is called by the PMIx progress thread and given a function to call when that handler has completed its operation. The handler callback notifies PMIx that the handler is done, returning a status code to indicate the result of its work. The results are appended to the array of prior results, with the returned values combined into an array within a single \refstruct{pmix_info_t} as follows:
\begin{itemize}
\item \texttt{array[0]}: the event handler name provided at registration (may be an empty field if a string name was not given) will be in the key, with the \refstruct{pmix_status_t} value returned by the handler
\item \texttt{array[*]}: the array of results returned by the handler, if any.
\end{itemize}

The current PMIx standard does not actually specify a default ordering for event handlers as they are being registered. However, it does include an inherent ordering for invocation. Specifically, PMIx stipulates that handlers be called in the following categorical order:

\begin{itemize}
\item single status event handlers - handlers that were registered against a single specific status.
\item multi status event handlers - those registered against more than one specific status.
\item default event handlers - those registered against no specific status.
\end{itemize}

\littleheader{Code Example}

From the OpenMP primary thread:

\pmixCodeImportC[]{sources/_autogen_/hybrid-prog-model.c_omp_thread}

From the MPI process:

\pmixCodeImportC[]{sources/_autogen_/hybrid-prog-model.c_mpi_process}

\littleheader{Related Interfaces}

{\large \refapi{PMIx_Register_event_handler}}
\pasteSignature{PMIx_Register_event_handler}

{\large \refapi{pmix_event_notification_cbfunc_fn_t}}
\pasteSignature{pmix_event_notification_cbfunc_fn_t}

\littleheader{Related Attributes}

\pasteAttributeItem{PMIX_EVENT_HDLR_NAME}
\pasteAttributeItem{PMIX_EVENT_HDLR_FIRST}
\pasteAttributeItem{PMIX_EVENT_HDLR_LAST}
\pasteAttributeItem{PMIX_EVENT_HDLR_FIRST_IN_CATEGORY}
\pasteAttributeItem{PMIX_EVENT_HDLR_LAST_IN_CATEGORY}
\pasteAttributeItem{PMIX_EVENT_HDLR_BEFORE}
\pasteAttributeItem{PMIX_EVENT_HDLR_AFTER}
\pasteAttributeItem{PMIX_EVENT_HDLR_APPEND}

\littleheader{Related Constants}
\refconst{PMIX_EVENT_NO_ACTION_TAKEN} \\
\refconst{PMIX_EVENT_PARTIAL_ACTION_TAKEN} \\
\refconst{PMIX_EVENT_ACTION_DEFERRED} \\

\section{MPI Sessions}
\label{app:uc-MPI-sessions}

\subsection{Use Case Summary}
MPI Sessions addresses a number of the limitations of the current MPI programming model. Among the immediate problems MPI Sessions is intended to address are the following:

\begin{itemize}
\item MPI cannot be initialized within an MPI process from different application components without a priori knowledge or coordination,
\item MPI cannot be initialized more than once, and MPI cannot be reinitialized after MPI finalize has been called.
\item With MPI Sessions, an application no longer needs to explicitly call \code{MPI_Init} to make use of MPI, but rather can use a Session to only initialize MPI resources for specific communication needs.
\item Unless the MPI process explicitly calls MPI_Init, there is also no explicit \code{MPI_COMM_WORLD} communicator. Sessions can be created and destroyed multiple times in an MPI process.
\end{itemize}

\subsection{Use Case Details}

\begingroup
\begin{figure*}
  \begin{center}
    \includegraphics[width=.5\textwidth,keepaspectratio]{figs/mpi-sessions1}
  \end{center}
  \caption{MPI Communicator from MPI Session Handle using PMIx}
  \label{fig:mpi_s1}
\end{figure*}
\endgroup

A PMIx Process Set (PSET) is a user-provided or host environment assigned
label associated with a given set of application processes. Processes can
belong to multiple process sets at a time. Definition of a PMIx
process set typically occurs at time of application execution - e.g., on a
command line: \code{prun -n 4 --pset ocean myoceanapp : -n 3 --pset ice myiceapp}

PMIx PSETs are used for query functions (\code{MPI_SESSION_GET_NUM_PSETS}, \code{MPI_SESSION_GET_NTH_PSET)} and to create \code{MPI_GROUP} from a process set name.

In OpenMPI's MPI Sessions prototype, PMIx groups are used during creation of \code{MPI_COMM} from an \code{MPI_GROUP}. The PMIx group constructor returns a 64-bit PMIx Group Context Identifier (PGCID) that is guaranteed to be unique for the duration of an allocation (in the case of a batch managed environment). This PGCID could be used as a direct replacement for the existing unique identifiers for communicators in MPI (E.g. Communicator Identifiers (CIDs) in Open MPI), but may have performance implications.

There is an important distinction between process sets and process groups. The process set identifiers are set by the host environment and currently there are no PMIx APIs provided by which an application can change a process set membership. In contrast, PMIx process groups can only be defined dynamically by the application.

\littleheader{Related Interfaces}

{\large \refapi{PMIx_Get}}
\pasteSignature{PMIx_Get}

{\large \refapi{PMIx_Group_construct}}
\pasteSignature{PMIx_Group_construct}

\littleheader{Related Attributes}

\pasteAttributeItem{PMIX_PSET_NAMES}
\pasteAttributeItem{PMIX_QUERY_NUM_GROUPS}
\pasteAttributeItem{PMIX_QUERY_GROUP_NAMES}
\pasteAttributeItem{PMIX_QUERY_GROUP_MEMBERSHIP}

\littleheader{Related Constants}

\refconst{PMIX_SUCCESS}
\refconst{PMIX_ERR_NOT_SUPPORTED}

%%%%%%%%%%%%%%%%%%%%%%%%%%%%%%%%%%%%%%%%%%%%%%%%%


% Revisions, Acknowledgements
    % Revisions
    %%%%%%%%%%%%%%%%%%%%%%%%%%%%%%%%%%%%%%%%%%%%%%%%%
\chapter{Revision History}
\label{chap:revisions}

%%%%%%%%%%%%%%%%%%%%%%%%%%%%%%%%%%%%%%%%%%%%%%%%%
%%%%%%%%%% History: Version 1.0
\section{Version 1.0: June 12, 2015}

\par
The \ac{PMIx} version 1.0 \textit{ad hoc} standard was defined in a set of header files as part of the v1.0.0 release of the OpenPMIx library prior to the creation of the formal \ac{PMIx} 2.0 standard.
Below are a summary listing of the interfaces defined in the 1.0 headers.

\begin{itemize}
\item Client APIs
\begin{compactitemize}
\item PMIx\_Init, \refapi{PMIx_Initialized}, \refapi{PMIx_Abort}, \refapi{PMIx_Finalize}
\item \refapi{PMIx_Put}, \refapi{PMIx_Commit},
\item \refapi{PMIx_Fence}, \refapi{PMIx_Fence_nb}
\item \refapi{PMIx_Get}, \refapi{PMIx_Get_nb}
\item \refapi{PMIx_Publish}, \refapi{PMIx_Publish_nb}
\item \refapi{PMIx_Lookup}, \refapi{PMIx_Lookup_nb}
\item \refapi{PMIx_Unpublish}, \refapi{PMIx_Unpublish_nb}
\item \refapi{PMIx_Spawn}, \refapi{PMIx_Spawn_nb}
\item \refapi{PMIx_Connect}, \refapi{PMIx_Connect_nb}
\item \refapi{PMIx_Disconnect}, \refapi{PMIx_Disconnect_nb}
\item \refapi{PMIx_Resolve_nodes}, \refapi{PMIx_Resolve_peers}
\end{compactitemize}
\item Server \acp{API}
\begin{compactitemize}
\item \refapi{PMIx_server_init}, \refapi{PMIx_server_finalize}
\item \refapi{PMIx_generate_regex}, \refapi{PMIx_generate_ppn}
\item \refapi{PMIx_server_register_nspace}, \refapi{PMIx_server_deregister_nspace}
\item \refapi{PMIx_server_register_client}, \refapi{PMIx_server_deregister_client}
\item \refapi{PMIx_server_setup_fork}, \refapi{PMIx_server_dmodex_request}
\end{compactitemize}
\item Common \acp{API}
\begin{compactitemize}
\item \refapi{PMIx_Get_version}, \refapi{PMIx_Store_internal}, \refapi{PMIx_Error_string}
\item PMIx_Register_errhandler, PMIx_Deregister_errhandler, PMIx_Notify_error
\end{compactitemize}
\end{itemize}

The \code{PMIx_Init} \ac{API} was subsequently modified in the v1.1.0 release of that library.

%%%%%%%%%%%%%%%%%%%%%%%%%%%%%%%%%%%%%%%%%%%%%%%%%
%%%%%%%%%% History: Version 2.0
\section{Version 2.0: Sept. 2018}

The following \acp{API} were introduced in v2.0 of the PMIx Standard:

\begin{itemize}
\item Client APIs
\begin{compactitemize}
\item \refapi{PMIx_Query_info_nb}, \refapi{PMIx_Log_nb}
\item \refapi{PMIx_Allocation_request_nb}, \refapi{PMIx_Job_control_nb}, \refapi{PMIx_Process_monitor_nb}, \refmacro{PMIx_Heartbeat}
\end{compactitemize}
\item Server \acp{API}
\begin{compactitemize}
\item \refapi{PMIx_server_setup_application}, \refapi{PMIx_server_setup_local_support}
\end{compactitemize}
\item Tool \acp{API}
\begin{compactitemize}
\item \refapi{PMIx_tool_init}, \refapi{PMIx_tool_finalize}
\end{compactitemize}
\item Common \acp{API}
\begin{compactitemize}
\item \refapi{PMIx_Register_event_handler}, \refapi{PMIx_Deregister_event_handler}
\item \refapi{PMIx_Notify_event}
\item \refapi{PMIx_Proc_state_string}, \refapi{PMIx_Scope_string}
\item \refapi{PMIx_Persistence_string}, \refapi{PMIx_Data_range_string}
\item \refapi{PMIx_Info_directives_string}, \refapi{PMIx_Data_type_string}
\item \refapi{PMIx_Alloc_directive_string}
\item \refapi{PMIx_Data_pack}, \refapi{PMIx_Data_unpack}, \refapi{PMIx_Data_copy}
\item \refapi{PMIx_Data_print}, \refapi{PMIx_Data_copy_payload}
\end{compactitemize}
\end{itemize}

\subsection{Removed/Modified \acp{API}}

The \refapi{PMIx_Init} \ac{API} was modified in v2.0 of the standard from its \textit{ad hoc} v1.0 signature to include passing of a \refstruct{pmix_info_t} array for flexibility and ``future-proofing'' of the \ac{API}.
In addition, the \code{PMIx_Notify_error}, \code{PMIx_Register_errhandler}, and \code{PMIx_Deregister_errhandler} \acp{API} were replaced. This pre-dated official adoption of \ac{PMIx} as a Standard.

\subsection{Deprecated constants}

The following constants were deprecated in v2.0:

\begin{constantdesc}

\declareconstitemDEP{PMIX_MODEX}
\declareconstitemDEP{PMIX_INFO_ARRAY}

\end{constantdesc}

\subsection{Deprecated attributes}

The following attributes were deprecated in v2.0:

%
\declareAttributeDEP{PMIX_ERROR_NAME}{"pmix.errname"}{pmix_status_t}{
Specific error to be notified
}
%
\declareAttributeDEP{PMIX_ERROR_GROUP_COMM}{"pmix.errgroup.comm"}{bool}{
Set true to get comm errors notification
}
%
\declareAttributeDEP{PMIX_ERROR_GROUP_ABORT}{"pmix.errgroup.abort"}{bool}{
Set true to get abort errors notification
}
%
\declareAttributeDEP{PMIX_ERROR_GROUP_MIGRATE}{"pmix.errgroup.migrate"}{bool}{
Set true to get migrate errors notification
}
%
\declareAttributeDEP{PMIX_ERROR_GROUP_RESOURCE}{"pmix.errgroup.resource"}{bool}{
Set true to get resource errors notification
}
%
\declareAttributeDEP{PMIX_ERROR_GROUP_SPAWN}{"pmix.errgroup.spawn"}{bool}{
Set true to get spawn errors notification
}
%
\declareAttributeDEP{PMIX_ERROR_GROUP_NODE}{"pmix.errgroup.node"}{bool}{
Set true to get node status notification
}
%
\declareAttributeDEP{PMIX_ERROR_GROUP_LOCAL}{"pmix.errgroup.local"}{bool}{
Set true to get local errors notification
}
%
\declareAttributeDEP{PMIX_ERROR_GROUP_GENERAL}{"pmix.errgroup.gen"}{bool}{
Set true to get notified of generic errors
}
%
\declareAttributeDEP{PMIX_ERROR_HANDLER_ID}{"pmix.errhandler.id"}{int}{
Errhandler reference id of notification being reported
}

%%%%%%%%%%%%%%%%%%%%%%%%%%%%%%%%%%%%%%%%%%%%%%%%%
%%%%%%%%%% History: Version 2.1
\section{Version 2.1: Dec. 2018}

The v2.1 update includes clarifications and corrections from the v2.0 document, plus addition of examples:

\begin{compactitemize}
    \item Clarify description of \refapi{PMIx_Connect} and \refapi{PMIx_Disconnect} \acp{API}.
    \item Explain that values for the \refattr{PMIX_COLLECTIVE_ALGO} are environment-dependent
    \item Identify the namespace/rank values required for retrieving attribute-associated information using the \refapi{PMIx_Get} \ac{API}
    \item Provide definitions for \refterm{session}, \refterm{job}, \refterm{application}, and other terms used throughout the document
    \item Clarify definitions of \refattr{PMIX_UNIV_SIZE} versus \refattr{PMIX_JOB_SIZE}
    \item Clarify server module function return values
    \item Provide examples of the use of \refapi{PMIx_Get} for retrieval of information
    \item Clarify the use of \refapi{PMIx_Get} versus \refapi{PMIx_Query_info_nb}
    \item Clarify return values for non-blocking \acp{API} and emphasize that callback functions must not be invoked prior to return from the \ac{API}
    \item Provide detailed example for construction of the \refapi{PMIx_server_register_nspace} input information array
    \item Define information levels (e.g., \refterm{session} vs \refterm{job}) and associated attributes for both storing and retrieving values
    \item Clarify roles of \ac{PMIx} server library and host environment for collective operations
    \item Clarify definition of \refattr{PMIX_UNIV_SIZE}
\end{compactitemize}


%%%%%%%%%%%%%%%%%%%%%%%%%%%%%%%%%%%%%%%%%%%%%%%%%
%%%%%%%%%% History: Version 2.2
\section{Version 2.2: Jan 2019}

The v2.2 update includes the following clarifications and corrections from the v2.1 document:

\begin{compactitemize}
    \item Direct modex upcall function (\refapi{pmix_server_dmodex_req_fn_t}) cannot complete atomically as the \ac{API} cannot return the requested information except via the provided callback function
    \item Add missing \refstruct{pmix_data_array_t} definition and support macros
    \item Add a rule divider between implementer and host environment required attributes for clarity
    \item Add \refmacro{PMIX_QUERY_QUALIFIERS_CREATE} macro to simplify creation of \refstruct{pmix_query_t} qualifiers
    \item Add \refmacro{PMIX_APP_INFO_CREATE} macro to simplify creation of \refstruct{pmix_app_t} directives
    \item Add flag and \refmacro{PMIX_INFO_IS_END} macro for marking and detecting the end of a \refstruct{pmix_info_t} array
    \item Clarify the allowed hierarchical nesting of the \refattr{PMIX_SESSION_INFO_ARRAY}, \refattr{PMIX_JOB_INFO_ARRAY}, and associated attributes
\end{compactitemize}

%%%%%%%%%%%%%%%%%%%%%%%%%%%%%%%%%%%%%%%%%%%%%%%%%
%%%%%%%%%% History: Version 3.0
\section{Version 3.0: Dec. 2018}

The following \acp{API} were introduced in v3.0 of the PMIx Standard:

\begin{itemize}
\item Client APIs
\begin{compactitemize}
\item \refapi{PMIx_Log}, \refapi{PMIx_Job_control}
\item \refapi{PMIx_Allocation_request}, \refapi{PMIx_Process_monitor}
\item \refapi{PMIx_Get_credential}, \refapi{PMIx_Validate_credential}
\end{compactitemize}
\item Server \acp{API}
\begin{compactitemize}
\item \refapi{PMIx_server_IOF_deliver}
\item \refapi{PMIx_server_collect_inventory}, \refapi{PMIx_server_deliver_inventory}
\end{compactitemize}
\item Tool \acp{API}
\begin{compactitemize}
\item \refapi{PMIx_IOF_pull}, \refapi{PMIx_IOF_push}, \refapi{PMIx_IOF_deregister}
\item \refapi{PMIx_tool_connect_to_server}
\end{compactitemize}
\item Common \acp{API}
\begin{compactitemize}
\item \refapi{PMIx_IOF_channel_string}
\end{compactitemize}
\end{itemize}

The document added a chapter on security credentials, a new section for \ac{IO} forwarding to the Process Management chapter, and a few blocking forms of previously-existing non-blocking \acp{API}. Attributes supporting the new \acp{API} were introduced, as well as additional attributes for a few existing functions.

\subsection{Removed constants}

The following constants were removed in v3.0:

\refconst{PMIX_MODEX}\\
\refconst{PMIX_INFO_ARRAY}

\subsection{Deprecated attributes}

The following attributes were deprecated in v3.0:

\declareAttributeDEP{PMIX_COLLECTIVE_ALGO_REQD}{"pmix.calreqd"}{bool}{
If \code{true}, indicates that the requested choice of algorithm is mandatory.
}

\subsection{Removed attributes}

The following attributes were removed in v3.0:

%
\declareAttributeRM{PMIX_ERROR_NAME}{"pmix.errname"}{pmix_status_t}{
Specific error to be notified
}
%
\declareAttributeRM{PMIX_ERROR_GROUP_COMM}{"pmix.errgroup.comm"}{bool}{
Set true to get comm errors notification
}
%
\declareAttributeRM{PMIX_ERROR_GROUP_ABORT}{"pmix.errgroup.abort"}{bool}{
Set true to get abort errors notification
}
%
\declareAttributeRM{PMIX_ERROR_GROUP_MIGRATE}{"pmix.errgroup.migrate"}{bool}{
Set true to get migrate errors notification
}
%
\declareAttributeRM{PMIX_ERROR_GROUP_RESOURCE}{"pmix.errgroup.resource"}{bool}{
Set true to get resource errors notification
}
%
\declareAttributeRM{PMIX_ERROR_GROUP_SPAWN}{"pmix.errgroup.spawn"}{bool}{
Set true to get spawn errors notification
}
%
\declareAttributeRM{PMIX_ERROR_GROUP_NODE}{"pmix.errgroup.node"}{bool}{
Set true to get node status notification
}
%
\declareAttributeRM{PMIX_ERROR_GROUP_LOCAL}{"pmix.errgroup.local"}{bool}{
Set true to get local errors notification
}
%
\declareAttributeRM{PMIX_ERROR_GROUP_GENERAL}{"pmix.errgroup.gen"}{bool}{
Set true to get notified of generic errors
}
%
\declareAttributeRM{PMIX_ERROR_HANDLER_ID}{"pmix.errhandler.id"}{int}{
Errhandler reference id of notification being reported
}

%%%%%%%%%%%%%%%%%%%%%%%%%%%%%%%%%%%%%%%%%%%%%%%%%
%%%%%%%%%% History: Version 3.1
\section{Version 3.1: Jan. 2019}

The v3.1 update includes clarifications and corrections from the v3.0 document:

\begin{compactitemize}
    \item Direct modex upcall function (\refapi{pmix_server_dmodex_req_fn_t}) cannot complete atomically as the \ac{API} cannot return the requested information except via the provided callback function
    \item Fix typo in name of \refattr{PMIX_FWD_STDDIAG} attribute
    \item Correctly identify the information retrieval and storage attributes as ``new'' to v3 of the standard
    \item Add missing \refstruct{pmix_data_array_t} definition and support macros
    \item Add a rule divider between implementer and host environment required attributes for clarity
    \item Add \refmacro{PMIX_QUERY_QUALIFIERS_CREATE} macro to simplify creation of \refstruct{pmix_query_t} qualifiers
    \item Add \refmacro{PMIX_APP_INFO_CREATE} macro to simplify creation of \refstruct{pmix_app_t} directives
    \item Add new attributes to specify the level of information being requested where ambiguity may exist (see \ref{api:struct:attributes:retrieval})
    \item Add new attributes to assemble information by its level for storage where ambiguity may exist (see \ref{api:struct:attributes:storage})
    \item Add flag and \refmacro{PMIX_INFO_IS_END} macro for marking and detecting the end of a \refstruct{pmix_info_t} array
    \item Clarify that \code{PMIX_NUM_SLOTS} is duplicative of (a) \refattr{PMIX_UNIV_SIZE} when used at the \refterm{session} level and (b) \refattr{PMIX_MAX_PROCS} when used at the \refterm{job} and \refterm{application} levels, but leave it in for backward compatibility.
    \item Clarify difference between \refattr{PMIX_JOB_SIZE} and \refattr{PMIX_MAX_PROCS}
    \item Clarify that \refapi{PMIx_server_setup_application} must be called per-\refterm{job} instead of per-\refterm{application} as the name implies. Unfortunately, this is a historical artifact. Note that both \refattr{PMIX_NODE_MAP} and \refattr{PMIX_PROC_MAP} must be included as input in the \refarg{info} array provided to that function. Further descriptive explanation of the ``instant on'' procedure will be provided in the next version of the \ac{PMIx} Standard.
    \item Clarify how the \ac{PMIx} server expects data passed to the host by \refapi{pmix_server_fencenb_fn_t} should be aggregated across nodes, and provide a code snippet example
\end{compactitemize}

%%%%%%%%%%%%%%%%%%%%%%%%%%%%%%%%%%%%%%%%%%%%%%%%%
%%%%%%%%%% History: Version 3.2
\section{Version 3.2: Oct. 2020}

The v3.2 update includes clarifications and corrections from the v3.1 document:

\begin{compactitemize}
    \item Correct an error in the \refapi{PMIx_Allocation_request} function signature, and clarify the allocation ID attributes
    \item Rename the \refattr{PMIX_ALLOC_ID} attribute to \refattr{PMIX_ALLOC_REQ_ID} to clarify that this is a string the user provides as a means to identify their request to query status
    \item Add a new \refattr{PMIX_ALLOC_ID} attribute that contains the identifier (provided by the host environment) for the resulting allocation which can later be used to reference the allocated resources in, for example, a call to \refapi{PMIx_Spawn}
    \item Update the \refapi{PMIx_generate_regex} and \refapi{PMIx_generate_ppn} descriptions to clarify that the output from these generator functions may not be a NULL-terminated string, but instead could be a byte array of arbitrary binary content.
    \item Add a new \refconst{PMIX_REGEX} constant that represents a regular expression data type.
\end{compactitemize}


\subsection{Deprecated constants}

The following constants were deprecated in v3.2:

\begin{constantdesc}
%
\declareconstitemDEP{PMIX_ERR_DATA_VALUE_NOT_FOUND}
Data value not found
%
\declareconstitemDEP{PMIX_ERR_HANDSHAKE_FAILED}
Connection handshake failed
%
\declareconstitemDEP{PMIX_ERR_IN_ERRNO}
Error defined in \code{errno}
%
\declareconstitemDEP{PMIX_ERR_INVALID_ARG}
Invalid argument
%
\declareconstitemDEP{PMIX_ERR_INVALID_ARGS}
Invalid arguments
%
\declareconstitemDEP{PMIX_ERR_INVALID_KEY}
Invalid key
%
\declareconstitemDEP{PMIX_ERR_INVALID_KEY_LENGTH}
Invalid key length
%
\declareconstitemDEP{PMIX_ERR_INVALID_KEYVALP}
Invalid key/value pair
%
\declareconstitemDEP{PMIX_ERR_INVALID_LENGTH}
Invalid argument length
%
\declareconstitemDEP{PMIX_ERR_INVALID_NAMESPACE}
Invalid namespace
%
\declareconstitemDEP{PMIX_ERR_INVALID_NUM_ARGS}
Invalid number of arguments
%
\declareconstitemDEP{PMIX_ERR_INVALID_NUM_PARSED}
Invalid number parsed
%
\declareconstitemDEP{PMIX_ERR_INVALID_SIZE}
Invalid size
%
\declareconstitemDEP{PMIX_ERR_INVALID_VAL}
Invalid value
%
\declareconstitemDEP{PMIX_ERR_INVALID_VAL_LENGTH}
Invalid value length
%
\declareconstitemDEP{PMIX_ERR_NOT_IMPLEMENTED}
Not implemented
%
\declareconstitemDEP{PMIX_ERR_PACK_MISMATCH}
Pack mismatch
%
\declareconstitemDEP{PMIX_ERR_PROC_ENTRY_NOT_FOUND}
Process not found
%
\declareconstitemDEP{PMIX_ERR_PROC_REQUESTED_ABORT}
Process is already requested to abort
%
\declareconstitemDEP{PMIX_ERR_READY_FOR_HANDSHAKE}
Ready for handshake
%
\declareconstitemDEP{PMIX_ERR_SERVER_FAILED_REQUEST}
Failed to connect to the server
%
\declareconstitemDEP{PMIX_ERR_SERVER_NOT_AVAIL}
Server is not available
%
\declareconstitemDEP{PMIX_ERR_SILENT}
Silent error
%
\declareconstitemDEP{PMIX_GDS_ACTION_COMPLETE}
The \ac{GDS} action has completed
%
\declareconstitemDEP{PMIX_NOTIFY_ALLOC_COMPLETE}
Notify that a requested allocation operation is complete - the result of
the request will be included in the \refarg{info} array
%
\end{constantdesc}


\subsection{Deprecated attributes}

The following attributes were deprecated in v3.2:

\declareAttributeDEP{PMIX_ARCH}{"pmix.arch"}{uint32_t}{
Architecture flag.
}
%
\declareAttributeDEP{PMIX_COLLECTIVE_ALGO}{"pmix.calgo"}{char*}{
Comma-delimited list of algorithms to use for the collective operation. \ac{PMIx} does not impose any requirements on a host environment's collective algorithms. Thus, the acceptable values for this attribute will be environment-dependent - users are encouraged to check their host environment for supported values.
}
%
\declareAttributeDEP{PMIX_DSTPATH}{"pmix.dstpath"}{char*}{
Path to shared memory data storage (dstore) files. Deprecated from Standard as being implementation specific.
}
%
\declareAttributeDEP{PMIX_HWLOC_HOLE_KIND}{"pmix.hwlocholek"}{char*}{
Kind of VM ``hole'' HWLOC should use for shared memory
}
%
\declareAttributeDEP{PMIX_HWLOC_SHARE_TOPO}{"pmix.hwlocsh"}{bool}{
Share the HWLOC topology via shared memory
}
%
\declareAttributeDEP{PMIX_HWLOC_SHMEM_ADDR}{"pmix.hwlocaddr"}{size_t}{
Address of the HWLOC shared memory segment.
}
%
\declareAttributeDEP{PMIX_HWLOC_SHMEM_FILE}{"pmix.hwlocfile"}{char*}{
Path to the HWLOC shared memory file.
}
%
\declareAttributeDEP{PMIX_HWLOC_SHMEM_SIZE}{"pmix.hwlocsize"}{size_t}{
Size of the HWLOC shared memory segment.
}
%
\declareAttributeDEP{PMIX_HWLOC_XML_V1}{"pmix.hwlocxml1"}{char*}{
\ac{XML} representation of local topology using HWLOC's v1.x format.
}
%
\declareAttributeDEP{PMIX_HWLOC_XML_V2}{"pmix.hwlocxml2"}{char*}{
\ac{XML} representation of local topology using HWLOC's v2.x format.
}
%
\declareAttributeDEP{PMIX_LOCAL_TOPO}{"pmix.ltopo"}{char*}{
\ac{XML} representation of local node topology.
}
%
\declareAttributeDEP{PMIX_MAPPER}{"pmix.mapper"}{char*}{
Mapping mechanism to use for placing spawned processes - when accessed using \refapi{PMIx_Get}, use the \refconst{PMIX_RANK_WILDCARD} value for the rank to discover the mapping mechanism used for the provided namespace.
}
%
\declareAttributeDEP{PMIX_MAP_BLOB}{"pmix.mblob"}{pmix_byte_object_t}{
Packed blob of process location.
}
%
\declareAttributeDEP{PMIX_NON_PMI}{"pmix.nonpmi"}{bool}{
Spawned processes will not call \refapi{PMIx_Init}.
}
%
\declareAttributeDEP{PMIX_PROC_BLOB}{"pmix.pblob"}{pmix_byte_object_t}{
Packed blob of process data.
}
%
\declareAttributeDEP{PMIX_PROC_URI}{"pmix.puri"}{char*}{
\ac{URI} containing contact information for the specified process.
}
%
\declareAttributeDEP{PMIX_TOPOLOGY_FILE}{"pmix.topo.file"}{char*}{
Full path to file containing \ac{XML} topology description
}
%
\declareAttributeDEP{PMIX_TOPOLOGY_SIGNATURE}{"pmix.toposig"}{char*}{
Topology signature string.
}
%
\declareAttributeDEP{PMIX_TOPOLOGY_XML}{"pmix.topo.xml"}{char*}{
\ac{XML}-based description of topology
}
%



%%%%%%%%%%%%%%%%%%%%%%%%%%%%%%%%%%%%%%%%%%%%%%%%%
%%%%%%%%%% History: Version 4.0
\section{Version 4.0: Dec. 2020}

NOTE: The PMIx Standard document has undergone significant reorganization in an
effort to become more user-friendly. Highlights include:

\begin{compactitemize}
    \item Moving all added, deprecated, and removed items to this revision log
    section to make them more visible
    \item Co-locating constants and attribute definitions with the primary
    API that uses them - citations and hyperlinks are retained elsewhere
    \item Splitting the Key-Value Management chapter into separate chapters on
    the use of reserved keys, non-reserved keys, and non-process-related
    key-value data exchange
    \item Creating a new chapter on synchronization and data access methods
    \item Removing references to specific implementations of \ac{PMIx} and to implementation-specific features
    and/or behaviors
\end{compactitemize}

In addition to the reorganization, the following changes were introduced in v4.0 of the PMIx Standard:

\begin{compactitemize}
    \item Clarified that the \refapi{PMIx_Fence_nb} operation can immediately return \refconst{PMIX_OPERATION_SUCCEEDED} in lieu of passing the request to a \ac{PMIx} server if only the calling process is involved in the operation
    \item Added the \refapi{PMIx_Register_attributes} \ac{API} by which a host environment can register the attributes it supports for each server-to-host operation
    \item Added the ability to query supported attributes from the \ac{PMIx} tool, client and server libraries, as well as the host environment via the new \refstruct{pmix_regattr_t} structure. Both human-readable and machine-parsable output is supported. New attributes to support this operation include:
    \begin{compactitemize}
        \item \refattr{PMIX_CLIENT_ATTRIBUTES}, \refattr{PMIX_SERVER_ATTRIBUTES}, \refattr{PMIX_TOOL_ATTRIBUTES}, and \refattr{PMIX_HOST_ATTRIBUTES} to identify which library supports the attribute; and
        \item \refattr{PMIX_MAX_VALUE}, \refattr{PMIX_MIN_VALUE}, and \refattr{PMIX_ENUM_VALUE} to provide machine-parsable description of accepted values
    \end{compactitemize}
    \item Add \refconst{PMIX_APP_WILDCARD} to reference all applications within a given job
    \item Fix signature of blocking APIs \refapi{PMIx_Allocation_request}, \refapi{PMIx_Job_control}, \refapi{PMIx_Process_monitor}, \refapi{PMIx_Get_credential}, and \refapi{PMIx_Validate_credential} to allow return of results
    \item Update description to provide an option for blocking behavior of the \refapi{PMIx_Register_event_handler}, \refapi{PMIx_Deregister_event_handler}, \refapi{PMIx_Notify_event}, \refapi{PMIx_IOF_pull}, \refapi{PMIx_IOF_deregister}, and \refapi{PMIx_IOF_push} APIs. The need for blocking forms of these functions was not initially anticipated but has emerged over time. For these functions, the return value is sufficient to provide the caller with information otherwise returned via callback. Thus, use of a \code{NULL} value as the callback function parameter was deemed a minimal disruption method for providing the desired capability
    \item Added a chapter on fabric support that includes new \acp{API}, datatypes, and attributes
    \item Added a chapter on process sets and groups that includes new \acp{API} and attributes
    \item Added \acp{API} and a new datatypes to support generation and parsing of \ac{PMIx} locality and cpuset strings
    \item Added a new chapter on tools that provides deeper explanation on their operation and collecting all tool-relevant definitions into one location. Also introduced two new \acp{API} and removed restriction that limited tools to being connected to only one server at a time.
    \item Extended behavior of \refapi{PMIx_server_init} to scalably expose the topology description to the local clients. This includes creating any required shared memory backing stores and/or \ac{XML} representations, plus ensuring that all necessary key-value pairs for clients to access the description are included in the job-level information provided to each client.
    \item Added a new \ac{API} by which the host can manually progress the \ac{PMIx} library in lieu of the library's own progress thread.
s\end{compactitemize}

The above changes included introduction of the following \acp{API} and data types:

\begin{itemize}
    \item Client APIs
    \begin{compactitemize}
        \item \refapi{PMIx_Group_construct}, \refapi{PMIx_Group_construct_nb}
        \item \refapi{PMIx_Group_destruct}, \refapi{PMIx_Group_destruct_nb}
        \item \refapi{PMIx_Group_invite}, \refapi{PMIx_Group_invite_nb}
        \item \refapi{PMIx_Group_join}, \refapi{PMIx_Group_join_nb}
        \item \refapi{PMIx_Group_leave}, \refapi{PMIx_Group_leave_nb}
        \item \refapi{PMIx_Get_relative_locality}, \refapi{PMIx_Load_topology}
        \item \refapi{PMIx_Parse_cpuset_string}, \refapi{PMIx_Get_cpuset}
        \item \refapi{PMIx_Link_state_string}, \refapi{PMIx_Job_state_string}
        \item \refapi{PMIx_Device_type_string}
        \item \refapi{PMIx_Fabric_register}, \refapi{PMIx_Fabric_register_nb}
        \item \refapi{PMIx_Fabric_update}, \refapi{PMIx_Fabric_update_nb}
        \item \refapi{PMIx_Fabric_deregister}, \refapi{PMIx_Fabric_deregister_nb}
        \item \refapi{PMIx_Compute_distances}, \refapi{PMIx_Compute_distances_nb}
        \item \refapi{PMIx_Get_attribute_string}, \refapi{PMIx_Get_attribute_name}
        \item \refapi{PMIx_Progress}
    \end{compactitemize}

    \item Server \acp{API}
    \begin{compactitemize}
        \item \refapi{PMIx_server_generate_locality_string}
        \item \refapi{PMIx_Register_attributes}
        \item \refapi{PMIx_server_define_process_set}, \refapi{PMIx_server_delete_process_set}
        \item \refapi{pmix_server_grp_fn_t}, \refapi{pmix_server_fabric_fn_t}
        \item \refapi{pmix_server_client_connected2_fn_t}
        \item \refapi{PMIx_server_generate_cpuset_string}
        \item \refapi{PMIx_server_register_resources}, \refapi{PMIx_server_deregister_resources}
    \end{compactitemize}

    \item Tool \acp{API}
    \begin{compactitemize}
        \item \refapi{PMIx_tool_disconnect}
        \item \refapi{PMIx_tool_set_server}
        \item \refapi{PMIx_tool_attach_to_server}
        \item \refapi{PMIx_tool_get_servers}
    \end{compactitemize}

    \item Data types
    \begin{compactitemize}
        \item \refstruct{pmix_regattr_t}
        \item \refstruct{pmix_cpuset_t}
        \item \refstruct{pmix_topology_t}
        \item \refstruct{pmix_locality_t}
        \item \refstruct{pmix_bind_envelope_t}
        \item \refstruct{pmix_group_opt_t}
        \item \refstruct{pmix_group_operation_t}
        \item \refstruct{pmix_fabric_t}
        \item \refstruct{pmix_device_distance_t}
        \item \refstruct{pmix_coord_t}
        \item \refstruct{pmix_coord_view_t}
        \item \refstruct{pmix_geometry_t}
        \item \refstruct{pmix_link_state_t}
        \item \refstruct{pmix_job_state_t}
        \item \refstruct{pmix_device_type_t}
    \end{compactitemize}

    \item Callback functions
    \begin{compactitemize}
        \item \refapi{pmix_device_dist_cbfunc_t}
    \end{compactitemize}

\end{itemize}

\subsection{Added Constants}

%
\littleheader{General error constants}
\refconst{PMIX_ERR_EXISTS_OUTSIDE_SCOPE} \\
\refconst{PMIX_ERR_PARAM_VALUE_NOT_SUPPORTED} \\
\refconst{PMIX_ERR_EMPTY} \\
%
\littleheader{Data type constants}
\refconst{PMIX_COORD} \\
\refconst{PMIX_REGATTR} \\
\refconst{PMIX_REGEX} \\
\refconst{PMIX_JOB_STATE} \\
\refconst{PMIX_LINK_STATE} \\
\refconst{PMIX_PROC_CPUSET} \\
\refconst{PMIX_GEOMETRY} \\
\refconst{PMIX_DEVICE_DIST} \\
\refconst{PMIX_ENDPOINT} \\
\refconst{PMIX_TOPO} \\
\refconst{PMIX_DEVTYPE} \\
\refconst{PMIX_LOCTYPE} \\
\refconst{PMIX_DATA_TYPE_MAX} \\
\refconst{PMIX_COMPRESSED_BYTE_OBJECT} \\

%
\littleheader{Info directives}
\refconst{PMIX_INFO_REQD_PROCESSED} \\

%
\littleheader{Server constants}
\refconst{PMIX_ERR_REPEAT_ATTR_REGISTRATION} \\
%
%
\littleheader{Job-Mgmt constants}
\refconst{PMIX_ERR_CONFLICTING_CLEANUP_DIRECTIVES} \\
%
%
\littleheader{Publish constants}
\refconst{PMIX_ERR_DUPLICATE_KEY} \\
%
%
\littleheader{Tool constants}
%
\refconst{PMIX_LAUNCHER_READY} \\
\refconst{PMIX_ERR_IOF_FAILURE} \\
\refconst{PMIX_ERR_IOF_COMPLETE} \\
\refconst{PMIX_EVENT_JOB_START} \\
\refconst{PMIX_LAUNCH_COMPLETE} \\
\refconst{PMIX_EVENT_JOB_END} \\
\refconst{PMIX_EVENT_SESSION_START} \\
\refconst{PMIX_EVENT_SESSION_END} \\
\refconst{PMIX_ERR_PROC_TERM_WO_SYNC} \\
\refconst{PMIX_ERR_JOB_CANCELED} \\
\refconst{PMIX_ERR_JOB_ABORTED} \\
\refconst{PMIX_ERR_JOB_KILLED_BY_CMD} \\
\refconst{PMIX_ERR_JOB_ABORTED_BY_SIG} \\
\refconst{PMIX_ERR_JOB_TERM_WO_SYNC} \\
\refconst{PMIX_ERR_JOB_SENSOR_BOUND_EXCEEDED} \\
\refconst{PMIX_ERR_JOB_NON_ZERO_TERM} \\
\refconst{PMIX_ERR_JOB_ABORTED_BY_SYS_EVENT} \\
\refconst{PMIX_DEBUG_WAITING_FOR_NOTIFY} \\
\refconst{PMIX_DEBUGGER_RELEASE} \\
%
%
\littleheader{Fabric constants}
\refconst{PMIX_FABRIC_UPDATE_PENDING} \\
\refconst{PMIX_FABRIC_UPDATED} \\
\refconst{PMIX_FABRIC_UPDATE_ENDPOINTS} \\
\refconst{PMIX_COORD_VIEW_UNDEF} \\
\refconst{PMIX_COORD_LOGICAL_VIEW} \\
\refconst{PMIX_COORD_PHYSICAL_VIEW} \\
\refconst{PMIX_LINK_STATE_UNKNOWN} \\
\refconst{PMIX_LINK_DOWN} \\
\refconst{PMIX_LINK_UP} \\
\refconst{PMIX_FABRIC_REQUEST_INFO} \\
\refconst{PMIX_FABRIC_UPDATE_INFO} \\
%
%
\littleheader{Sets-Groups constants}
\refconst{PMIX_PROCESS_SET_DEFINE} \\
\refconst{PMIX_PROCESS_SET_DELETE} \\
\refconst{PMIX_GROUP_INVITED} \\
\refconst{PMIX_GROUP_LEFT} \\
\refconst{PMIX_GROUP_MEMBER_FAILED} \\
\refconst{PMIX_GROUP_INVITE_ACCEPTED} \\
\refconst{PMIX_GROUP_INVITE_DECLINED} \\
\refconst{PMIX_GROUP_INVITE_FAILED} \\
\refconst{PMIX_GROUP_MEMBERSHIP_UPDATE} \\
\refconst{PMIX_GROUP_CONSTRUCT_ABORT} \\
\refconst{PMIX_GROUP_CONSTRUCT_COMPLETE} \\
\refconst{PMIX_GROUP_LEADER_FAILED} \\
\refconst{PMIX_GROUP_LEADER_SELECTED} \\
\refconst{PMIX_GROUP_CONTEXT_ID_ASSIGNED} \\
%
%
\littleheader{Process-Mgmt constants}
\refconst{PMIX_ERR_JOB_ALLOC_FAILED} \\
\refconst{PMIX_ERR_JOB_APP_NOT_EXECUTABLE} \\
\refconst{PMIX_ERR_JOB_NO_EXE_SPECIFIED} \\
\refconst{PMIX_ERR_JOB_FAILED_TO_MAP} \\
\refconst{PMIX_ERR_JOB_FAILED_TO_LAUNCH} \\
\refconst{PMIX_LOCALITY_UNKNOWN} \\
\refconst{PMIX_LOCALITY_NONLOCAL} \\
\refconst{PMIX_LOCALITY_SHARE_HWTHREAD} \\
\refconst{PMIX_LOCALITY_SHARE_CORE} \\
\refconst{PMIX_LOCALITY_SHARE_L1CACHE} \\
\refconst{PMIX_LOCALITY_SHARE_L2CACHE} \\
\refconst{PMIX_LOCALITY_SHARE_L3CACHE} \\
\refconst{PMIX_LOCALITY_SHARE_PACKAGE} \\
\refconst{PMIX_LOCALITY_SHARE_NUMA} \\
\refconst{PMIX_LOCALITY_SHARE_NODE} \\
%
%
\littleheader{Events}
\refconst{PMIX_EVENT_SYS_BASE} \\
\refconst{PMIX_EVENT_NODE_DOWN} \\
\refconst{PMIX_EVENT_NODE_OFFLINE} \\
\refconst{PMIX_EVENT_SYS_OTHER} \\
%
%
\subsection{Added Attributes}
%
\littleheader{Sync-Access attributes}
%
\pasteAttributeItem{PMIX_COLLECT_GENERATED_JOB_INFO}
\pasteAttributeItem{PMIX_ALL_CLONES_PARTICIPATE}
\pasteAttributeItem{PMIX_GET_POINTER_VALUES}
\pasteAttributeItem{PMIX_GET_STATIC_VALUES}
\pasteAttributeItem{PMIX_GET_REFRESH_CACHE}
\pasteAttributeItem{PMIX_QUERY_RESULTS}
\pasteAttributeItem{PMIX_QUERY_QUALIFIERS}
\pasteAttributeItem{PMIX_QUERY_SUPPORTED_KEYS}
\pasteAttributeItem{PMIX_QUERY_SUPPORTED_QUALIFIERS}
\pasteAttributeItem{PMIX_QUERY_NAMESPACE_INFO}
\pasteAttributeItem{PMIX_QUERY_ATTRIBUTE_SUPPORT}
\pasteAttributeItem{PMIX_QUERY_AVAIL_SERVERS}
\pasteAttributeItem{PMIX_SERVER_INFO_ARRAY}
\pasteAttributeItem{PMIX_CLIENT_FUNCTIONS}
\pasteAttributeItem{PMIX_CLIENT_ATTRIBUTES}
\pasteAttributeItem{PMIX_SERVER_FUNCTIONS}
\pasteAttributeItem{PMIX_SERVER_ATTRIBUTES}
\pasteAttributeItem{PMIX_HOST_FUNCTIONS}
\pasteAttributeItem{PMIX_HOST_ATTRIBUTES}
\pasteAttributeItem{PMIX_TOOL_FUNCTIONS}
\pasteAttributeItem{PMIX_TOOL_ATTRIBUTES}
%
%
\littleheader{Server attributes}
%
\pasteAttributeItem{PMIX_TOPOLOGY2}
\pasteAttributeItem{PMIX_SERVER_SHARE_TOPOLOGY}
\pasteAttributeItem{PMIX_SERVER_SESSION_SUPPORT}
\pasteAttributeItem{PMIX_SERVER_START_TIME}
\pasteAttributeItem{PMIX_SERVER_SCHEDULER}
\pasteAttributeItem{PMIX_JOB_INFO_ARRAY}
\pasteAttributeItem{PMIX_APP_INFO_ARRAY}
\pasteAttributeItem{PMIX_PROC_INFO_ARRAY}
\pasteAttributeItem{PMIX_NODE_INFO_ARRAY}
\pasteAttributeItem{PMIX_MAX_VALUE}
\pasteAttributeItem{PMIX_MIN_VALUE}
\pasteAttributeItem{PMIX_ENUM_VALUE}
\pasteAttributeItem{PMIX_HOMOGENEOUS_SYSTEM}
\pasteAttributeItem{PMIX_REQUIRED_KEY}
%
%
\littleheader{Job-Mgmt attributes}
%
\pasteAttributeItem{PMIX_ALLOC_ID}
\pasteAttributeItem{PMIX_ALLOC_QUEUE}
%
%
\littleheader{Publish attributes}
%
\pasteAttributeItem{PMIX_ACCESS_PERMISSIONS}
\pasteAttributeItem{PMIX_ACCESS_USERIDS}
\pasteAttributeItem{PMIX_ACCESS_GRPIDS}
%
%
\littleheader{Reserved keys}
%
\pasteAttributeItem{PMIX_NUM_ALLOCATED_NODES}
\pasteAttributeItem{PMIX_NUM_NODES}
\pasteAttributeItem{PMIX_CMD_LINE}
\pasteAttributeItem{PMIX_APP_ARGV}
\pasteAttributeItem{PMIX_PACKAGE_RANK}
\pasteAttributeItem{PMIX_REINCARNATION}
\pasteAttributeItem{PMIX_HOSTNAME_ALIASES}
\pasteAttributeItem{PMIX_HOSTNAME_KEEP_FQDN}
\pasteAttributeItem{PMIX_CPUSET_BITMAP}
\pasteAttributeItem{PMIX_EXTERNAL_PROGRESS}
\pasteAttributeItem{PMIX_NODE_MAP_RAW}
\pasteAttributeItem{PMIX_PROC_MAP_RAW}
%
%
\littleheader{Tool attributes}
%
\pasteAttributeItem{PMIX_TOOL_CONNECT_OPTIONAL}
\pasteAttributeItem{PMIX_TOOL_ATTACHMENT_FILE}
\pasteAttributeItem{PMIX_LAUNCHER_RENDEZVOUS_FILE}
\pasteAttributeItem{PMIX_PRIMARY_SERVER}
\pasteAttributeItem{PMIX_NOHUP}
\pasteAttributeItem{PMIX_LAUNCHER_DAEMON}
\pasteAttributeItem{PMIX_FORKEXEC_AGENT}
\pasteAttributeItem{PMIX_EXEC_AGENT}
\pasteAttributeItem{PMIX_IOF_PUSH_STDIN}
\pasteAttributeItem{PMIX_IOF_COPY}
\pasteAttributeItem{PMIX_IOF_REDIRECT}
\pasteAttributeItem{PMIX_DEBUG_TARGET}
\pasteAttributeItem{PMIX_DEBUG_DAEMONS_PER_PROC}
\pasteAttributeItem{PMIX_DEBUG_DAEMONS_PER_NODE}
\pasteAttributeItem{PMIX_WAIT_FOR_CONNECTION}
\pasteAttributeItem{PMIX_LAUNCH_DIRECTIVES}
%
%
\littleheader{Fabric attributes}
\pasteAttributeItem{PMIX_SERVER_SCHEDULER}
\pasteAttributeItem{PMIX_FABRIC_COST_MATRIX}
\pasteAttributeItem{PMIX_FABRIC_GROUPS}
\pasteAttributeItem{PMIX_FABRIC_VENDOR}
\pasteAttributeItem{PMIX_FABRIC_IDENTIFIER}
\pasteAttributeItem{PMIX_FABRIC_INDEX}
\pasteAttributeItem{PMIX_FABRIC_NUM_DEVICES}
\pasteAttributeItem{PMIX_FABRIC_COORDINATES}
\pasteAttributeItem{PMIX_FABRIC_DIMS}
\pasteAttributeItem{PMIX_FABRIC_ENDPT}
\pasteAttributeItem{PMIX_FABRIC_SHAPE}
\pasteAttributeItem{PMIX_FABRIC_SHAPE_STRING}
\pasteAttributeItem{PMIX_SWITCH_PEERS}
\pasteAttributeItem{PMIX_FABRIC_PLANE}
\pasteAttributeItem{PMIX_FABRIC_SWITCH}
\pasteAttributeItem{PMIX_FABRIC_DEVICE}
\pasteAttributeItem{PMIX_FABRIC_DEVICE_INDEX}
\pasteAttributeItem{PMIX_FABRIC_DEVICE_NAME}
\pasteAttributeItem{PMIX_FABRIC_DEVICE_VENDOR}
\pasteAttributeItem{PMIX_FABRIC_DEVICE_BUS_TYPE}
\pasteAttributeItem{PMIX_FABRIC_DEVICE_VENDORID}
\pasteAttributeItem{PMIX_FABRIC_DEVICE_DRIVER}
\pasteAttributeItem{PMIX_FABRIC_DEVICE_FIRMWARE}
\pasteAttributeItem{PMIX_FABRIC_DEVICE_ADDRESS}
\pasteAttributeItem{PMIX_FABRIC_DEVICE_COORDINATES}
\pasteAttributeItem{PMIX_FABRIC_DEVICE_MTU}
\pasteAttributeItem{PMIX_FABRIC_DEVICE_SPEED}
\pasteAttributeItem{PMIX_FABRIC_DEVICE_STATE}
\pasteAttributeItem{PMIX_FABRIC_DEVICE_TYPE}
\pasteAttributeItem{PMIX_FABRIC_DEVICE_PCI_DEVID}
%
%
\littleheader{Device attributes}
\pasteAttributeItem{PMIX_DEVICE_DISTANCES}
\pasteAttributeItem{PMIX_DEVICE_TYPE}
\pasteAttributeItem{PMIX_DEVICE_ID}
%
%
\littleheader{Sets-Groups attributes}
\pasteAttributeItem{PMIX_QUERY_NUM_PSETS}
\pasteAttributeItem{PMIX_QUERY_PSET_NAMES}
\pasteAttributeItem{PMIX_QUERY_PSET_MEMBERSHIP}
\pasteAttributeItem{PMIX_PSET_NAME}
\pasteAttributeItem{PMIX_PSET_MEMBERS}
\pasteAttributeItem{PMIX_PSET_NAMES}
\pasteAttributeItem{PMIX_QUERY_NUM_GROUPS}
\pasteAttributeItem{PMIX_QUERY_GROUP_NAMES}
\pasteAttributeItem{PMIX_QUERY_GROUP_MEMBERSHIP}
\pasteAttributeItem{PMIX_GROUP_ID}
\pasteAttributeItem{PMIX_GROUP_LEADER}
\pasteAttributeItem{PMIX_GROUP_OPTIONAL}
\pasteAttributeItem{PMIX_GROUP_NOTIFY_TERMINATION}
\pasteAttributeItem{PMIX_GROUP_FT_COLLECTIVE}
\pasteAttributeItem{PMIX_GROUP_ASSIGN_CONTEXT_ID}
\pasteAttributeItem{PMIX_GROUP_LOCAL_ONLY}
\pasteAttributeItem{PMIX_GROUP_CONTEXT_ID}
\pasteAttributeItem{PMIX_GROUP_ENDPT_DATA}
\pasteAttributeItem{PMIX_GROUP_NAMES}
%
%
\littleheader{Process Mgmt attributes}
\pasteAttributeItem{PMIX_OUTPUT_TO_DIRECTORY}
\pasteAttributeItem{PMIX_TIMEOUT_STACKTRACES}
\pasteAttributeItem{PMIX_TIMEOUT_REPORT_STATE}
\pasteAttributeItem{PMIX_NOTIFY_JOB_EVENTS}
\pasteAttributeItem{PMIX_NOTIFY_PROC_TERMINATION}
\pasteAttributeItem{PMIX_NOTIFY_PROC_ABNORMAL_TERMINATION}
\pasteAttributeItem{PMIX_LOG_PROC_TERMINATION}
\pasteAttributeItem{PMIX_LOG_PROC_ABNORMAL_TERMINATION}
\pasteAttributeItem{PMIX_LOG_JOB_EVENTS}
\pasteAttributeItem{PMIX_LOG_COMPLETION}
\pasteAttributeItem{PMIX_FIRST_ENVAR}
%
%
\littleheader{Event attributes}
\pasteAttributeItem{PMIX_EVENT_TIMESTAMP}
%
%
\subsection{Added Environmental Variables}
%
\littleheader{Tool environmental variables}
\refenvar{PMIX_LAUNCHER_RNDZ_URI} \\
\refenvar{PMIX_LAUNCHER_RNDZ_FILE} \\
\refenvar{PMIX_KEEPALIVE_PIPE} \\
%
%
\subsection{Added Macros}
%
\refmacro{PMIX_CHECK_RESERVED_KEY}
\refmacro{PMIX_INFO_WAS_PROCESSED}
\refmacro{PMIX_INFO_PROCESSED}
\refmacro{PMIX_INFO_LIST_START}
\refmacro{PMIX_INFO_LIST_ADD}
\refmacro{PMIX_INFO_LIST_XFER}
\refmacro{PMIX_INFO_LIST_CONVERT}
\refmacro{PMIX_INFO_LIST_RELEASE}
%
%
\subsection{Deprecated \acp{API}}

\declareapiDEP{pmix_evhdlr_reg_cbfunc_t}
Renamed to \refapi{pmix_hdlr_reg_cbfunc_t}

The \refapiDEP{pmix_server_client_connected_fn_t} server module entry point has
been \emph{deprecated} in favor of
\refapi{pmix_server_client_connected2_fn_t}

\declareapiDEP{PMIx_tool_connect_to_server}
Replaced by \refapi{PMIx_tool_attach_to_server} to allow return of the process identifier of the server to which the tool has attached.

\subsection{Deprecated constants}

The following constants were deprecated in v4.0:

\begin{constantdesc}
%
\declareconstitemDEP{PMIX_ERR_DEBUGGER_RELEASE}
Renamed to \refconst{PMIX_DEBUGGER_RELEASE}
%
\declareconstitemDEP{PMIX_ERR_JOB_TERMINATED}
Renamed to \refconst{PMIX_EVENT_JOB_END}
%
\declareconstitemDEP{PMIX_EXISTS}
Renamed to \refconst{PMIX_ERR_EXISTS}
%
\declareconstitemDEP{PMIX_ERR_PROC_ABORTED}
Consolidated with \refconst{PMIX_EVENT_PROC_TERMINATED}
%
\declareconstitemDEP{PMIX_ERR_PROC_ABORTING}
Consolidated with \refconst{PMIX_EVENT_PROC_TERMINATED}
%
\declareconstitemDEP{PMIX_ERR_LOST_CONNECTION_TO_SERVER}
Consolidated into \refconst{PMIX_ERR_LOST_CONNECTION}
%
\declareconstitemDEP{PMIX_ERR_LOST_PEER_CONNECTION}
Consolidated into \refconst{PMIX_ERR_LOST_CONNECTION}
%
\declareconstitemDEP{PMIX_ERR_LOST_CONNECTION_TO_CLIENT}
Consolidated into \refconst{PMIX_ERR_LOST_CONNECTION}
%
\declareconstitemDEP{PMIX_ERR_INVALID_TERMINATION}
Renamed to \refconst{PMIX_ERR_JOB_TERM_WO_SYNC}
%
\declareconstitemDEP{PMIX_PROC_TERMINATED}
Renamed to \refconst{PMIX_EVENT_PROC_TERMINATED}
%
\declareconstitemDEP{PMIX_ERR_NODE_DOWN}
Renamed to \refconst{PMIX_EVENT_NODE_DOWN}
%
\declareconstitemDEP{PMIX_ERR_NODE_OFFLINE}
Renamed to \refconst{PMIX_EVENT_NODE_OFFLINE}
%
\declareconstitemDEP{PMIX_ERR_SYS_OTHER}
Renamed to \refconst{PMIX_EVENT_SYS_OTHER}
%
\declareconstitemDEP{PMIX_CONNECT_REQUESTED}
Connection has been requested by a PMIx-based tool - deprecated as
not required.
%
\declareconstitemDEP{PMIX_PROC_HAS_CONNECTED}
A tool or client has connected to the \ac{PMIx} server - deprecated in
favor of the new \refapi{pmix_server_client_connected2_fn_t} server
module \ac{API}
%
\end{constantdesc}

\subsection{Removed constants}

The following constants were removed from the \ac{PMIx} Standard in v4.0
as they are internal to a particular \ac{PMIx} implementation.

\begin{constantdesc}
%!TEX encoding = UTF-8 Unicode
\declareconstitemRM{PMIX_ERR_HANDSHAKE_FAILED}
Connection handshake failed
%
\declareconstitemRM{PMIX_ERR_READY_FOR_HANDSHAKE}
Ready for handshake
%
\declareconstitemRM{PMIX_ERR_IN_ERRNO}
Error defined in \code{errno}
%
\declareconstitemRM{PMIX_ERR_INVALID_VAL_LENGTH}
Invalid value length
%
\declareconstitemRM{PMIX_ERR_INVALID_LENGTH}
Invalid argument length
%
\declareconstitemRM{PMIX_ERR_INVALID_NUM_ARGS}
Invalid number of arguments
%
\declareconstitemRM{PMIX_ERR_INVALID_ARGS}
Invalid arguments
%
\declareconstitemRM{PMIX_ERR_INVALID_NUM_PARSED}
Invalid number parsed
%
\declareconstitemRM{PMIX_ERR_INVALID_KEYVALP}
Invalid key/value pair
%
\declareconstitemRM{PMIX_ERR_INVALID_SIZE}
Invalid size
%
\declareconstitemRM{PMIX_ERR_PROC_REQUESTED_ABORT}
Process is already requested to abort
%
\declareconstitemRM{PMIX_ERR_SERVER_FAILED_REQUEST}
Failed to connect to the server
%
\declareconstitemRM{PMIX_ERR_PROC_ENTRY_NOT_FOUND}
Process not found
%
\declareconstitemRM{PMIX_ERR_INVALID_ARG}
Invalid argument
%
\declareconstitemRM{PMIX_ERR_INVALID_KEY}
Invalid key
%
\declareconstitemRM{PMIX_ERR_INVALID_KEY_LENGTH}
Invalid key length
%
\declareconstitemRM{PMIX_ERR_INVALID_VAL}
Invalid value
%
\declareconstitemRM{PMIX_ERR_INVALID_NAMESPACE}
Invalid namespace
%
\declareconstitemRM{PMIX_ERR_SERVER_NOT_AVAIL}
Server is not available
%
\declareconstitemRM{PMIX_ERR_SILENT}
Silent error
%
\declareconstitemRM{PMIX_ERR_PACK_MISMATCH}
Pack mismatch
%
\declareconstitemRM{PMIX_ERR_DATA_VALUE_NOT_FOUND}
Data value not found
%
\declareconstitemRM{PMIX_ERR_NOT_IMPLEMENTED}
Not implemented
%
\declareconstitemRM{PMIX_GDS_ACTION_COMPLETE}
The \ac{GDS} action has completed
%
\declareconstitemRM{PMIX_NOTIFY_ALLOC_COMPLETE}
Notify that a requested allocation operation is complete - the result of
the request will be included in the \refarg{info} array
%
\end{constantdesc}

\subsection{Deprecated attributes}

The following attributes were deprecated in v4.0:

%
\declareAttributeDEP{PMIX_TOPOLOGY}{"pmix.topo"}{hwloc_topology_t}{
Renamed to \refattr{PMIX_TOPOLOGY2}.
}
%
\declareAttributeDEP{PMIX_DEBUG_JOB}{"pmix.dbg.job"}{char*}{
Renamed to \refattr{PMIX_DEBUG_TARGET})
}
%
\declareAttributeDEP{PMIX_RECONNECT_SERVER}{"pmix.tool.recon"}{bool}{
Renamed to the \refapi{PMIx_tool_connect_to_server} \ac{API}
}
%
\declareAttributeDEP{PMIX_ALLOC_NETWORK}{"pmix.alloc.net"}{array}{
Renamed to \refattr{PMIX_ALLOC_FABRIC}
}
%
\declareAttributeDEP{PMIX_ALLOC_NETWORK_ID}{"pmix.alloc.netid"}{char*}{
Renamed to \refattr{PMIX_ALLOC_FABRIC_ID}
}
%
\declareAttributeDEP{PMIX_ALLOC_NETWORK_QOS}{"pmix.alloc.netqos"}{char*}{
Renamed to \refattr{PMIX_ALLOC_FABRIC_QOS}
}
%
\declareAttributeDEP{PMIX_ALLOC_NETWORK_TYPE}{"pmix.alloc.nettype"}{char*}{
Renamed to \refattr{PMIX_ALLOC_FABRIC_TYPE}
}
%
\declareAttributeDEP{PMIX_ALLOC_NETWORK_PLANE}{"pmix.alloc.netplane"}{char*}{
Renamed to \refattr{PMIX_ALLOC_FABRIC_PLANE}
}
%
\declareAttributeDEP{PMIX_ALLOC_NETWORK_ENDPTS}{"pmix.alloc.endpts"}{size_t}{
Renamed to \refattr{PMIX_ALLOC_FABRIC_ENDPTS}
}
%
\declareAttributeDEP{PMIX_ALLOC_NETWORK_ENDPTS_NODE}{"pmix.alloc.endpts.nd"}{size_t}{
Renamed to \refattr{PMIX_ALLOC_FABRIC_ENDPTS_NODE}
}
%
\declareAttributeDEP{PMIX_ALLOC_NETWORK_SEC_KEY}{"pmix.alloc.nsec"}{pmix_byte_object_t}{
Renamed to \refattr{PMIX_ALLOC_FABRIC_SEC_KEY}
}
%
\declareAttributeDEP{PMIX_PROC_DATA}{"pmix.pdata"}{pmix_data_array_t}{
Renamed to \refattr{PMIX_PROC_INFO_ARRAY}
}
%
\declareAttributeDEP{PMIX_LOCALITY}{"pmix.loc"}{\refstruct{pmix_locality_t}}{
Relative locality of the specified process to the requester, expressed as a bitmask as per the description in the \refstruct{pmix_locality_t} section. This value is unique to the requesting process and thus cannot be communicated by the server as part of the job-level information. Its use has been replaced by the \refapi{PMIx_Get_relative_locality} function.
}

\subsection{Removed attributes}

The following attributes were removed from the \ac{PMIx} Standard in v4.0
as they
are internal to a particular \ac{PMIx} implementation. Users are referred to the
\refapi{PMIx_Load_topology} \ac{API} for obtaining the local topology
description.

%
\declareAttributeRM{PMIX_LOCAL_TOPO}{"pmix.ltopo"}{char*}{
\ac{XML} representation of local node topology.
}
%
\declareAttributeRM{PMIX_TOPOLOGY_XML}{"pmix.topo.xml"}{char*}{
\ac{XML}-based description of topology
}
%
\declareAttributeRM{PMIX_TOPOLOGY_FILE}{"pmix.topo.file"}{char*}{
Full path to file containing \ac{XML} topology description
}
%
\declareAttributeRM{PMIX_TOPOLOGY_SIGNATURE}{"pmix.toposig"}{char*}{
Topology signature string.
}
%
\declareAttributeRM{PMIX_HWLOC_SHMEM_ADDR}{"pmix.hwlocaddr"}{size_t}{
Address of the HWLOC shared memory segment.
}
%
\declareAttributeRM{PMIX_HWLOC_SHMEM_SIZE}{"pmix.hwlocsize"}{size_t}{
Size of the HWLOC shared memory segment.
}
%
\declareAttributeRM{PMIX_HWLOC_SHMEM_FILE}{"pmix.hwlocfile"}{char*}{
Path to the HWLOC shared memory file.
}
%
\declareAttributeRM{PMIX_HWLOC_XML_V1}{"pmix.hwlocxml1"}{char*}{
\ac{XML} representation of local topology using HWLOC's v1.x format.
}
%
\declareAttributeRM{PMIX_HWLOC_XML_V2}{"pmix.hwlocxml2"}{char*}{
\ac{XML} representation of local topology using HWLOC's v2.x format.
}
%
\declareAttributeRM{PMIX_HWLOC_SHARE_TOPO}{"pmix.hwlocsh"}{bool}{
Share the HWLOC topology via shared memory
}
%
\declareAttributeRM{PMIX_HWLOC_HOLE_KIND}{"pmix.hwlocholek"}{char*}{
Kind of VM ``hole'' HWLOC should use for shared memory
}
%
\declareAttributeRM{PMIX_DSTPATH}{"pmix.dstpath"}{char*}{
Path to shared memory data storage (dstore) files. Deprecated from Standard as being implementation specific.
}
%
\declareAttributeRM{PMIX_COLLECTIVE_ALGO}{"pmix.calgo"}{char*}{
Comma-delimited list of algorithms to use for the collective operation. \ac{PMIx} does not impose any requirements on a host environment's collective algorithms. Thus, the acceptable values for this attribute will be environment-dependent - users are encouraged to check their host environment for supported values.
}
%
\declareAttributeRM{PMIX_COLLECTIVE_ALGO_REQD}{"pmix.calreqd"}{bool}{
If \code{true}, indicates that the requested choice of algorithm is mandatory.
}
%
\declareAttributeRM{PMIX_PROC_BLOB}{"pmix.pblob"}{pmix_byte_object_t}{
Packed blob of process data.
}
%
\declareAttributeRM{PMIX_MAP_BLOB}{"pmix.mblob"}{pmix_byte_object_t}{
Packed blob of process location.
}
%
\declareAttributeRM{PMIX_MAPPER}{"pmix.mapper"}{char*}{
Mapping mechanism to use for placing spawned processes - when accessed using \refapi{PMIx_Get}, use the \refconst{PMIX_RANK_WILDCARD} value for the rank to discover the mapping mechanism used for the provided namespace.
}
%
\declareAttributeRM{PMIX_NON_PMI}{"pmix.nonpmi"}{bool}{
Spawned processes will not call \refapi{PMIx_Init}.
}
%
\declareAttributeRM{PMIX_PROC_URI}{"pmix.puri"}{char*}{
\ac{URI} containing contact information for the specified process.
}
%
\declareAttributeRM{PMIX_ARCH}{"pmix.arch"}{uint32_t}{
Architecture flag.
}

%%%%%%%%%%%%%%%%%%%%%%%%%%%%%%%%%%%%%%%%%%%%%%%%%
%%%%%%%%%% History: Version 4.1
\section{Version 4.1: Oct. 2021}

The v4.1 update includes clarifications and corrections from the v4.0 document:

\begin{compactitemize}
    \item Remove some stale language in \refsection{chap:api_event:notify}{Chapter 9.1}.
    \item Provisional Items:
    \begin{compactitemize}
        \item Storage \chapterref{chap:api_storage}
    \end{compactitemize}
\end{compactitemize}

\subsection{Added Functions (Provisional)}

\begin{compactitemize}
  \item \refapi{PMIx_Data_load}
  \item \refapi{PMIx_Data_unload}
  \item \refapi{PMIx_Data_compress}
  \item \refapi{PMIx_Data_decompress}
\end{compactitemize}

\subsection{Added Data Structures (Provisional)}

\begin{compactitemize}
    \item \refstruct{pmix_storage_medium_t}
    \item \refstruct{pmix_storage_accessibility_t}
    \item \refstruct{pmix_storage_persistence_t}
    \item \refstruct{pmix_storage_access_type_t}
\end{compactitemize}

\subsection{Added Macros (Provisional)}

\begin{compactitemize}
  \item \refmacro{PMIX_NSPACE_INVALID}
  \item \refmacro{PMIX_RANK_IS_VALID}
  \item \refmacro{PMIX_PROCID_INVALID}
  \item \refmacro{PMIX_PROCID_XFER}
\end{compactitemize}

\subsection{Added Constants (Provisional)}

\begin{compactitemize}
  \item \refconst{PMIX_PROC_NSPACE}
\end{compactitemize}

\littleheader{Storage constants}

\begin{compactitemize}
  \item \refconst{PMIX_STORAGE_MEDIUM_UNKNOWN}
  \item \refconst{PMIX_STORAGE_MEDIUM_TAPE}
  \item \refconst{PMIX_STORAGE_MEDIUM_HDD}
  \item \refconst{PMIX_STORAGE_MEDIUM_SSD}
  \item \refconst{PMIX_STORAGE_MEDIUM_NVME}
  \item \refconst{PMIX_STORAGE_MEDIUM_PMEM}
  \item \refconst{PMIX_STORAGE_MEDIUM_RAM}
  \item \refconst{PMIX_STORAGE_ACCESSIBILITY_NODE}
  \item \refconst{PMIX_STORAGE_ACCESSIBILITY_SESSION}
  \item \refconst{PMIX_STORAGE_ACCESSIBILITY_JOB}
  \item \refconst{PMIX_STORAGE_ACCESSIBILITY_RACK}
  \item \refconst{PMIX_STORAGE_ACCESSIBILITY_CLUSTER}
  \item \refconst{PMIX_STORAGE_ACCESSIBILITY_REMOTE}
  \item \refconst{PMIX_STORAGE_PERSISTENCE_TEMPORARY}
  \item \refconst{PMIX_STORAGE_PERSISTENCE_NODE}
  \item \refconst{PMIX_STORAGE_PERSISTENCE_SESSION}
  \item \refconst{PMIX_STORAGE_PERSISTENCE_JOB}
  \item \refconst{PMIX_STORAGE_PERSISTENCE_SCRATCH}
  \item \refconst{PMIX_STORAGE_PERSISTENCE_PROJECT}
  \item \refconst{PMIX_STORAGE_PERSISTENCE_ARCHIVE}
  \item \refconst{PMIX_STORAGE_ACCESS_RD}
  \item \refconst{PMIX_STORAGE_ACCESS_WR}
  \item \refconst{PMIX_STORAGE_ACCESS_RDWR}
\end{compactitemize}

\subsection{Added Attributes (Provisional)}

\littleheader{Storage attributes}
\pasteAttributeItem{PMIX_STORAGE_ID}
\pasteAttributeItem{PMIX_STORAGE_PATH}
\pasteAttributeItem{PMIX_STORAGE_TYPE}
\pasteAttributeItem{PMIX_STORAGE_VERSION}
\pasteAttributeItem{PMIX_STORAGE_MEDIUM}
\pasteAttributeItem{PMIX_STORAGE_ACCESSIBILITY}
\pasteAttributeItem{PMIX_STORAGE_PERSISTENCE}
\pasteAttributeItem{PMIX_QUERY_STORAGE_LIST}
\pasteAttributeItem{PMIX_STORAGE_CAPACITY_LIMIT}
\pasteAttributeItem{PMIX_STORAGE_CAPACITY_USED}
\pasteAttributeItem{PMIX_STORAGE_OBJECT_LIMIT}
\pasteAttributeItem{PMIX_STORAGE_OBJECTS_USED}
\pasteAttributeItem{PMIX_STORAGE_MINIMAL_XFER_SIZE}
\pasteAttributeItem{PMIX_STORAGE_SUGGESTED_XFER_SIZE}
\pasteAttributeItem{PMIX_STORAGE_BW_MAX}
\pasteAttributeItem{PMIX_STORAGE_BW_CUR}
\pasteAttributeItem{PMIX_STORAGE_IOPS_MAX}
\pasteAttributeItem{PMIX_STORAGE_IOPS_CUR}
\pasteAttributeItem{PMIX_STORAGE_ACCESS_TYPE}


%%%%%%%%%%%%%%%%%%%%%%%%%%%%%%%%%%%%%%%%%%%%%%%%%
%%%%%%%%%% History: Version 4.2
\section{Version 4.2: TBD}

The v4.2 update includes the following changes from the v4.1 document:

\subsection{Added Attributes (Provisional)}

\littleheader{Tool attributes}
\pasteAttributeItem{PMIX_IOF_LOCAL_OUTPUT}
\pasteAttributeItem{PMIX_IOF_MERGE_STDERR_STDOUT}
\pasteAttributeItem{PMIX_IOF_OUTPUT_RAW}
\pasteAttributeItem{PMIX_IOF_RANK_OUTPUT}
\pasteAttributeItem{PMIX_IOF_OUTPUT_TO_FILE}
\pasteAttributeItem{PMIX_IOF_OUTPUT_TO_DIRECTORY}
\pasteAttributeItem{PMIX_IOF_FILE_PATTERN}
\pasteAttributeItem{PMIX_IOF_FILE_ONLY}


    %%%%%%%%%%%%%%%%%%%%%%%%%%%%%%%%%%%%%%%%%%%%%%%%%
% Chapter: Acknowledgements
%%%%%%%%%%%%%%%%%%%%%%%%%%%%%%%%%%%%%%%%%%%%%%%%%
\chapter{Acknowledgements}
\label{chap:acknowledgements}

This document represents the work of many people who have contributed to the PMIx community.
Without the hard work and dedication of these people this document would not have been possible.
The sections below list some of the active participants and organizations in the various PMIx standard iterations.

%%%%%%%%%% Version 2.0
\section{Version 2.0}

The following list includes some of the active participants in the PMIx v2 standardization process.

\begin{itemize}
\item Ralph H. Castain, Annapurna Dasari, Christopher A. Holguin, Andrew Friedley, Michael Klemm and Terry Wilmarth
\item Joshua Hursey, David Solt, Alexander Eichenberger, Geoff Paulsen, and Sameh Sharkawi
\item Aurelien Bouteiller and George Bosilca
\item Artem Polyakov, Igor Ivanov and Boris Karasev
\item Gilles Gouaillardet
\item Michael A Raymond and Jim Stoffel
\item Dirk Schubert
\item Moe Jette
\item Takahiro Kawashima and Shinji Sumimoto
\item Howard Pritchard
\item David Beer
\item Brice Goglin
\item Geoffroy Vallee, Swen Boehm, Thomas Naughton and David Bernholdt
\item Adam Moody and Martin Schulz
\item Ryan Grant and Stephen Olivier
\item Michael Karo
\end{itemize}

The following institutions supported this effort through time and travel support for the people listed above.

\begin{itemize}
\item Intel Corporation
\item IBM, Inc.
\item University of Tennessee, Knoxville
\item The Exascale Computing Project, an initiative of the US Department of Energy
\item National Science Foundation
\item Mellanox, Inc.
\item Research Organization for Information Science and Technology
\item HPE Co.
\item Allinea (ARM)
\item SchedMD, Inc.
\item Fujitsu Limited
\item Los Alamos National Laboratory
\item Adaptive Solutions, Inc.
\item INRIA
\item Oak Ridge National Laboratory
\item Lawrence Livermore National Laboratory
\item Sandia National Laboratory
\item Altair
\end{itemize}


%%%%%%%%%% Version 1.0
\section{Version 1.0}

The following list includes some of the active participants in the PMIx v1 standardization process.

\begin{itemize}
\item Ralph H. Castain, Annapurna Dasari and Christopher A. Holguin
\item Joshua Hursey and David Solt
\item Aurelien Bouteiller and George Bosilca
\item Artem Polyakov, Elena Shipunova, Igor Ivanov, and Joshua Ladd
\item Gilles Gouaillardet
\item Gary Brown
\item Moe Jette
\end{itemize}

The following institutions supported this effort through time and travel support for the people listed above.

\begin{itemize}
\item Intel Corporation
\item IBM, Inc.
\item University of Tennessee, Knoxville
\item Mellanox, Inc.
\item Research Organization for Information Science and Technology
\item Adaptive Solutions, Inc.
\item SchedMD, Inc.
\end{itemize}


%
% Bibliography
%
	\nolinenumbers
	\bibliography{pmix-standard}{}
	\addcontentsline{toc}{chapter}{Bibliography}
	\bibliographystyle{plain}

%
% Index
%
	\nolinenumbers

	\indexprologue{General terms and other items not induced in the other indices.}
	\printindex

	%\indexprologue{Functions}
	\printindex[index_api]

	%\indexprologue{Macros}
	\printindex[index_macro]

	%\indexprologue{Data structures}
	\printindex[index_struct]

	%\indexprologue{Constants}
	\printindex[index_const]

    %\indexprologue{Envars}
    \printindex[index_envars]

	%\indexprologue{Attributes}
	\printindex[index_attribute]


\end{document}

%%%%%%%%%%%%%%%%%%%%%%%%%%%%%%%%%%%%%%%%%%%%%%%%%
