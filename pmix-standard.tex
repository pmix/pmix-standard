% Welcome to pmix-stanard.tex.
% This is the master LaTex file for the PMIx Standard document.
%
% The files in this set include:
%
%    pmix-standard.tex                - this file, the master file
%    Makefile                         - makes the document
%    pmix.sty                         - the main style file
%    Title_Page.tex                   - the title page
%    Chap_Introduction.tex            - unnumbered introductory chapter
%    figs/*.png                       - Figures
%    sources/*.c, *.f                 - C/C++/Fortran example source files
%
% When editing this file:
%
%    1. To change formatting, appearance, or style, please edit pmix.sty.
%
%    2. Custom commands and macros are defined in pmix.sty.
%
%    3. Be kind to other editors -- keep a consistent style by copying-and-pasting to
%       create new content.
%
%    4. We use semantic markup, e.g. (see pmix.sty for a full list):
%         \code{}     % for bold monospace keywords, code, operators, etc.
%         \plc{}      % for italic placeholder names, grammar, etc.
%
%    5. Other recommendations:
%         Use the convenience macros defined in pmix.sty for the minor headers
%         such as Comments, Syntax, etc.
%
%         To keep items together on the same page, prefer the use of 
%         \begin{samepage}.... Avoid \parbox for text blocks as it interrupts line numbering.
%         When possible, avoid \filbreak, \pagebreak, \newpage, \clearpage unless that's
%         what you mean. Use \needspace{} cautiously for troublesome paragraphs.
%
%         Avoid absolute lengths and measures in this file; use relative units when possible.
%         Vertical space can be relative to \baselineskip or ex units. Horizontal space
%         can be relative to \linewidth or em units.
%
%         Prefer \emph{} to italicize terminology, e.g.:
%             This is a \emph{definition}, not a placeholder.
%             This is a \plc{var-name}.
%

% The following says letter size, but the style sheet may change the size
\documentclass[10pt,letterpaper,twoside,makeidx,hidelinks]{scrreprt}

% Text to appear in the footer on even-numbered pages:
\newcommand{\VER}{2.0 (draft)}
\newcommand{\VERDATE}{November 2017}
\newcommand{\footerText}{PMIx Standard -- Version \VER{} -- \VERDATE}

% Unified style sheet for PMIx documents:
% This is pmix.sty, the preamble and style definitions for the PMIx specification.
%
% This specification file, and latex structure was derived from/inspired by the OpenMP specification. So some similarity between the two latex files is expected.
%
%%%%%%%%%%%%%%%%%%%%%%%%%%%%%%%%%%%%%%%%%%%%%%%%%%%%%%%%%%%%%%%%%%%%%%%%%%%%%%%%%%%%%%%%%%%%%
% Quick list of the environments, commands and macros supported.
% Search below for more details.
%
% Formatting Text:
%   -----------------------
%   \notestart            - "Note:" Callout section
%   \noteheader           - \noteheader is optional "Note:" prefix for text
%     ...
%   \noteend
%   -----------------------
%   \rationalestart       - "Rationale" Callout section
%     ...
%   \rationaleend
%   -----------------------
%   \adviceuserstart      - "Advice to users" Callout section
%     ...
%   \adviceuserend
%   -----------------------
%   \adviceimplstart      - "Advice to PMIx library implementers" Callout section
%     ...
%   \adviceimplend
%   -----------------------
%   \advicermstart      - "Advice to PMIx server hosts" Callout section
%     ...
%   \advicermend
%   -----------------------
%
% Formatting Code:
%   \code{}               - Code text
%   \var{}                - Variable
%   -----------------------
%   \begin{codepar}       - Section of generic code
%     ...                 - use language specific macro if language specific code
%   \end[codepar}
%   -----------------------
%   \cspecificstart       - C specific code block
%     ...
%   \cspecificend
%   -----------------------
%
% Attributes:
%   \refAttributeItem{}   - Cross reference
%   \refattr{}            - Same as above
%   \pasteAttributeItem{} - Paste full description
%
% Structures:
%   \refstruct{}          - Reference a structure
%   \structref{}          - Same as above
%   \specrefstruct{}      - Reference a structure by section number and page
%
% APIs:
%   \refapi{}             - Reference an API function
%   \refconst{}           - Constant reference
%   \refarg{} / \argref{} - Reference an argument to an API function
%
% Cross referencing:
%   \chapterref{}         - Reference a Chapter by number and page
%   \specref{}            - Reference a Section by number and page
%
%%%%%%%%%%%%%%%%%%%%%%%%%%%%%%%%%%%%%%%%%%%%%%%%%%%%%%%%%%%%%%%%%%%%%%%%%%%%%%%%%%%%%%%%%%%%%
\usepackage{comment}            % allow use of \begin{comment}
\usepackage{ifpdf,ifthen}       % allow conditional tests in LaTeX definitions
\usepackage{makecell}           % Allows common formatting in cells with \thread & \makecell

\usepackage[T1]{fontenc}        % Allow us to use underscore freely in the document
\catcode`\_=12                  % Use \sb for subscripts
\usepackage{verbatim}


%%%%%%%%%%%%%%%%%%%%%%%%%%%%%%%%%%%%%%%%%%%%%%%%%%%%%%%%%%%%%%%%%%%%%%%%%%%%%%%%%%%%%%%%%%%%%
% Document data
%
\author{}


%%%%%%%%%%%%%%%%%%%%%%%%%%%%%%%%%%%%%%%%%%%%%%%%%%%%%%%%%%%%%%%%%%%%%%%%%%%%%%%%%%%%%%%%%%%%%
% Fonts

\usepackage{amsmath}
\usepackage{amsfonts}
\usepackage{amssymb}
\usepackage{courier}
\usepackage{helvet}
\usepackage[utf8]{inputenc}
\usepackage{textgreek}

% Main body serif font:
\usepackage{tgtermes}
\usepackage[T1]{fontenc}


%%%%%%%%%%%%%%%%%%%%%%%%%%%%%%%%%%%%%%%%%%%%%%%%%%%%%%%%%%%%%%%%%%%%%%%%%%%%%%%%%%%%%%%%%%%%%
% Graphic elements

\usepackage{graphicx}
\usepackage{framed}    % for making boxes with \begin{framed}
\usepackage{tikz}      % for flow charts, diagrams, arrows


%%%%%%%%%%%%%%%%%%%%%%%%%%%%%%%%%%%%%%%%%%%%%%%%%%%%%%%%%%%%%%%%%%%%%%%%%%%%%%%%%%%%%%%%%%%%%
% Page formatting

\usepackage[paperwidth=7.5in, paperheight=9in,
            top=0.75in, bottom=1.0in, left=1.4in, right=0.6in]{geometry}

\usepackage{changepage}   % allows left/right-page margin readjustments

\setlength{\oddsidemargin}{0.185in}
\setlength{\evensidemargin}{0.185in}
\raggedbottom


%%%%%%%%%%%%%%%%%%%%%%%%%%%%%%%%%%%%%%%%%%%%%%%%%%%%%%%%%%%%%%%%%%%%%%%%%%%%%%%%%%%%%%%%%%%%%
% Paragraph formatting

\usepackage{setspace}     % allows use of \singlespacing, \onehalfspacing
\usepackage{needspace}    % allows use of \needspace to keep lines together
\usepackage{parskip}      % removes paragraph indenting

\raggedright
\usepackage[raggedrightboxes]{ragged2e}  % is this needed?

\lefthyphenmin=60         % only hyphenate if the left part is >= this many chars
\righthyphenmin=60        % only hyphenate if the right part is >= this many chars


%%%%%%%%%%%%%%%%%%%%%%%%%%%%%%%%%%%%%%%%%%%%%%%%%%%%%%%%%%%%%%%%%%%%%%%%%%%%%%%%%%%%%%%%%%%%%%
% Bulleted (itemized) lists
%    Align bullets with section header
%    Align text left
%    Small bullets
%    \compactitem for single-spaced lists (used in the Examples doc)

\usepackage{enumitem}     % for setting margins on lists
\setlist{leftmargin=*}    % don't indent bullet items
\renewcommand{\labelitemi}{{\normalsize$\bullet$}} % bullet size

% There is a \compactitem defined in package parlist (and perhaps others), however,
% we'll define our own version of compactitem in terms of package enumitem that
% we already use:
\newenvironment{compactitem}
{\begin{itemize}[itemsep=-1.2ex]}
{\end{itemize}}

%%%%%%%%%%%%%%%%%%%%%%%%%%%%%%%%%%%%%%%%%%%%%%%%%%%%%%%%%%%%%%%%%%%%%%%%%%%%%%%%%%%%%%%%%%%%%
% Floating version
%\usepackage[showboxes]{textpos}
\usepackage{textpos}

\setlength{\TPHorizModule}{1pt}%
\setlength{\TPVertModule}{\TPHorizModule}%
\TPMargin{1pt}%

\newcommand{\versionMarker}[1]{%
 % y is 8 = \parskip
 \begin{textblock}{50}(-55,8)%
   \textit{PMIx v#1}%
   \raggedright
 \end{textblock}%
}
% Alternative is to make a box inline, but that gets tricky when positioning close
% to codepar's
% \makebox[-7pt][r]{\textit{PMIx #4}\raggedright}

%%%%%%%%%%%%%%%%%%%%%%%%%%%%%%%%%%%%%%%%%%%%%%%%%%%%%%%%%%%%%%%%%%%%%%%%%%%%%%%%%%%%%%%%%%%%%%
% Enumerated list with lowercase alphabet lettering
%    \alphaenum for default-spaced lists
%    \compactalphaenum for single-spaced lists

% There is a \compactitem defined in package parlist (and perhaps others), however,
% we'll define our own version of compactitem in terms of package enumitem that
% we already use:
\newenvironment{alphaenum}
{\begin{enumerate}[label=\alph*)]}
{\end{enumerate}}

\newenvironment{compactalphaenum}
{\begin{enumerate}[label=\alph*),itemsep=-1.2ex]}
{\end{enumerate}}

% Argument list for an interface, for use in a \begin{arglist} section
% \argin      Input argument
% \argout     Output argument
% \arginout   Input/Output argument
% \argreturn  Value returned
%%% Old Method using tables.... line numbers didn't work if a cell wrapped...
%\newlength\argdesclen
%\setlength\argdesclen{\dimexpr \linewidth -13em -4\tabcolsep}
%\newenvironment{arglist}{%
%    \begin{edtable}{tabular}{p{3em}p{10em}p{\argdesclen}}}
%    {\end{edtable}\vspace{.25em}}
%
%\newcommand{\argin}[2]{\textbf{IN} & \code{#1} & #2\\}
%\newcommand{\argout}[2]{\textbf{OUT} & \code{#1} & #2\\}
%\newcommand{\arginout}[2]{\textbf{INOUT} & \code{#1} & #2\\}

\newenvironment{arglist}
{\begin{description}[style=nextline,labelindent=\parindent,leftmargin=*,itemindent=\dimexpr-17pt-\labelsep\relax,itemsep=-1.3ex]}
{\end{description}}

\newcommand{\argin}[2]{\item[IN ~~~~\code{#1}] #2}
\newcommand{\argout}[2]{\item[OUT ~~~\code{#1}] #2}
\newcommand{\arginout}[2]{\item[INOUT ~\code{#1}] #2}

% Constant list
%   \declareconstitem  Declare constant with description
\newenvironment{constantdesc}
{\begin{description}[itemsep=-1.3ex,itemindent=\dimexpr-17pt-\labelsep\relax]}
{\end{description}}

\newcommand{\declareconstitem}[1]{\item[\code{#1}] \index{#1} \label{const:#1} \hspace{1em}}
\newcommand{\declareconstitemvalue}[2]{\item[\code{#1}] \index{#1} \hspace{0.25em} \code{#2}  \hspace{1em}}
\newcommand{\declareconstitemDEP}[2]{\item[\code{#1} (Deprecated in PMIx #2)] \index{#1} \label{const:#1} \hspace{1em}}
\newcommand{\declareconstitemNEW}[1]{\item[\color{magenta}\code{#1}] \index{#1} \label{const:#1} \hspace{1em}}


%%%%%%%%%%%%%%%%%%%%%%%%%%%%%%%%%%%%%%%%%%%%%%%%%%%%%%%%%%%%%%%%%%%%%%%%%%%%%%%%%%%%%%%%%%%%%%
% Tables

% This allows tables to flow across page breaks, headers on each new page, etc.
\usepackage{supertabular}
\usepackage{caption}
\usepackage{longtable}
\usepackage{pdflscape} % for 'landscape' environment

%%%%%%%%%%%%%%%%%%%%%%%%%%%%%%%%%%%%%%%%%%%%%%%%%%%%%%%%%%%%%%%%%%%%%%%%%%%%%%%%%%%%%%%%%%%%%
% Line numbering

\usepackage[pagewise,edtable]{lineno}       % for line numbers on left side of the page
\pagewiselinenumbers
\setlength\linenumbersep{6em}
\renewcommand\linenumberfont{\normalfont\small\sffamily}
\nolinenumbers            % start with line numbers off


%%%%%%%%%%%%%%%%%%%%%%%%%%%%%%%%%%%%%%%%%%%%%%%%%%%%%%%%%%%%%%%%%%%%%%%%%%%%%%%%%%%%%%%%%%%%%
% Footers

\usepackage{fancyhdr}     % makes right/left footers
\pagestyle{fancy}
\fancyhead{} % clear all header fields
\cfoot{}
\renewcommand{\headrulewidth}{0pt}

% Left side on even pages:
% This requires that \footerText be defined in the master document:
\fancyfoot[LE]{\bfseries \thepage \mdseries \hspace{2em} \footerText}
\fancyhfoffset[E]{4em}

% Right side on odd pages:
\fancyfoot[RO]{\mdseries  \leftmark \hspace{2em} \bfseries \thepage}


%%%%%%%%%%%%%%%%%%%%%%%%%%%%%%%%%%%%%%%%%%%%%%%%%%%%%%%%%%%%%%%%%%%%%%%%%%%%%%%%%%%%%%%%%%%%%
% Section header format - we use five levels: \chapter \section \subsection \subsubsection

\usepackage{titlesec}     % format headers with \titleformat{}

% Format and spacing for chapter, section, subsection, and subsubsection headers:

\setcounter{secnumdepth}{5}          % show numbers down to subsubsection level

\titleformat{\chapter}[display]%
{\normalfont\sffamily\upshape\Huge\bfseries\nolinenumbers\fontsize{20}{20}\selectfont}%
{\normalfont\sffamily\scshape\large\bfseries\nolinenumbers \hspace{-0.7in} \MakeUppercase%
    {\chaptertitlename} \thechapter}%
{0em}{}[\vspace{1.0em}\hrule]
% {<left>}{<before-sep>}{<after-sep>}
\titlespacing{\chapter}{0ex}{0em plus 1em minus 1em}{1em plus 1em minus 1em}[10em]

\titleformat{\section}[hang]{\huge\bfseries\sffamily\fontsize{16}{16}\selectfont}{\thesection}{1.0em}{}
% {<left>}{<before-sep>}{<after-sep>}
\titlespacing{\section}{-5em}{2em plus 1em minus 1em}{1em plus 0.5em minus 0em}[10em]

\titleformat{\subsection}[hang]{\LARGE\bfseries\sffamily\fontsize{14}{14}\selectfont}{\thesubsection}{1.0em}{}
\titlespacing{\subsection}{-5em}{2em plus 1em minus 2.0em}{0.75em plus 0.5em minus 0em}[10em]

\titleformat{\subsubsection}[hang]{\needspace{1\baselineskip}%
\Large\bfseries\sffamily\fontsize{12}{12}\selectfont}{\thesubsubsection}{1.0em}{}
\titlespacing{\subsubsection}{-5em}{0.5em plus 1em minus 1em}{0.5em plus 0.5em minus 0em}[10em]


%%%%%%%%%%%%%%%%%%%%%%%%%%%%%%%%%%%%%%%%%%%%%%%%%%%%%%%%%%%%%%%%%%%%%%%%%%%%%%%%%%%%%%%%%%%%%%
% Macros for minor headers: Summary, Syntax, Description, etc.
% These headers are defined in terms of \paragraph

\titleformat{\paragraph}[block]{\large\bfseries\sffamily\fontsize{11}{11}\selectfont}{}{}{}
\titlespacing{\paragraph}{0em}{1.0em plus 0.55em minus 0.5em}{0.0em plus 0.55em minus 0.0em}

% Use one of the convenience macros below, or \littleheader{} for an arbitrary header
\newcommand{\littleheader}[1] {\paragraph*{#1}}

\newcommand{\comments} {\littleheader{Comments}}
\newcommand{\descr} {\littleheader{Description}}
\newcommand{\format} {\littleheader{Format}}
\newcommand{\summary} {\littleheader{Summary}}
\newcommand{\history} {\littleheader{History}}
\newcommand{\priattr} {\littleheader{PRI Attributes}}
\newcommand{\reqattr} {\littleheader{\ac{RM} Required Attributes}}
\newcommand{\optattr} {\littleheader{\ac{RM} Optional Attributes}}

%%%%%%%%%%%%%%%%%%%%%%%%%%%%%%%%%%%%%%%%%%%%%%%%%%%%%%%%%%%%%%%%%%%%%%%%%%%%%%%%%%%%%%%%%%%%%
% Clipboard
%
% \StdCopy{TAG}{BODY}
% \StdPaste{TAG}
%
% Inspired by this thread:
%   https://tex.stackexchange.com/questions/150790/how-to-make-text-be-copied-to-another-part-of-a-document
\makeatletter
\newcommand\StdCopy              [2] {
  \immediate\write\@auxout{\unexpanded{\global\long\@namedef{clipbrd@#1}{#2}}}
}
\newcommand\StdCopyEcho          [2] {
  \StdCopy{#1}{#2}%
  #2
}
\newcommand\StdPaste             [1] {%
  \ifcsname clipbrd@#1\endcsname
    \@nameuse{clipbrd@#1}%
  \else
    ??unknown??
  \fi
}
\makeatother


% Attributes
%   \declareAttribute       Declare an attribute with a description
%   \pasteAttributeItem     Paste the attribute description here
%   \refAttributeItem       Reference the original definition of the attribute
%
\newcommand{\declareAttribute}[4]{%
    \code{#1} ~~\code{#2}~~(\code{#3})%
    \index{#1!Definition|textbf} \label{attr:#1}%
    \StdCopy{str:#1}{\code{#2}}%
    \StdCopy{attr:#1}{\code{#3}}%
    \vspace{-1.3ex}%
      \expandafter\begin{adjustwidth}{.95cm}{}%
      \StdCopyEcho{#1}{#4}%
    \end{adjustwidth}%
  \vspace{-1.3ex}%
}

\newcommand{\declareNewAttribute}[4]{%
   {\color{magenta}\code{#1}} ~~\code{#2}~~(\code{#3})%
    \index{#1!Definition|textbf} \label{attr:#1}%
    \StdCopy{str:#1}{\code{#2}}%
    \StdCopy{attr:#1}{\code{#3}}%
    \vspace{-1.3ex}%
      \expandafter\begin{adjustwidth}{.95cm}{}%
      \StdCopyEcho{#1}{#4}%
    \end{adjustwidth}%
  \vspace{-1.3ex}%
}

\newcommand{\declareDepAttribute}[4]{%
   {\color{green!80!black}\code{#1}} ~~\code{#2}~~(\code{#3})%
    \index{#1!Definition|textbf} \label{attr:#1}%
    \StdCopy{str:#1}{\code{#2}}%
    \StdCopy{attr:#1}{\code{#3}}%
    \vspace{-1.3ex}%
      \expandafter\begin{adjustwidth}{.95cm}{}%
      \StdCopyEcho{#1}{#4}%
    \end{adjustwidth}%
  \vspace{-1.3ex}%
}

\newcommand{\pasteAttributeItemBegin}[1]{
  \refAttributeItem{#1} ~~\StdPaste{str:#1}~~(\StdPaste{attr:#1})
  \vspace{-1.3ex}
   \expandafter
   \begin{adjustwidth}{.95cm}{}
    \StdPaste{#1}
}
\newcommand{\pasteAttributeItemEnd}{
   \end{adjustwidth}
}
\newcommand{\pasteAttributeItem}[1]{
	\pasteAttributeItemBegin{#1}
	\pasteAttributeItemEnd{}
}
\newcommand{\refAttributeItem}[1]{\index{#1} \hyperref[attr:#1]{\code{#1}} }
\newcommand{\refattr}[1]{\refAttributeItem{#1}}

\newcommand{\refPRIAttributeItem}[1]{\index{#1} \hyperref[attr:#1]{\color{red}\code{#1}} }

\newcommand{\pastePRIAttributeItemBegin}[1]{
  \refPRIAttributeItem{#1} ~~\StdPaste{str:#1}~~(\StdPaste{attr:#1})
  \vspace{-1.3ex}
   \expandafter
   \begin{adjustwidth}{.95cm}{}
    \StdPaste{#1}
}
\newcommand{\pastePRIAttributeItemEnd}{
   \end{adjustwidth}
}

\newcommand{\pastePRIAttributeItem}[1]{
    \pastePRIAttributeItemBegin{#1}
    \pastePRIAttributeItemEnd{}
}

\newcommand{\refPRRTEAttributeItem}[1]{\index{#1} \hyperref[attr:#1]{\color{green!60!black}\code{#1}} }

\newcommand{\pastePRRTEAttributeItemBegin}[1]{
  \refPRRTEAttributeItem{#1} ~~\StdPaste{str:#1}~~(\StdPaste{attr:#1})
  \vspace{-1.3ex}
   \expandafter
   \begin{adjustwidth}{.95cm}{}
    \StdPaste{#1}
}
\newcommand{\pastePRRTEAttributeItemEnd}{
   \end{adjustwidth}
}

\newcommand{\pastePRRTEAttributeItem}[1]{
    \pastePRRTEAttributeItemBegin{#1}
    \pastePRRTEAttributeItemEnd{}
}

%%%%%%%%%%%%%%%%%%%%%%%%%%%%%%%%%%%%%%%%%%%%%%%%%%%%%%%%%%%%%%%%%%%%%%%%%%%%%%%%%%%%%%%%%%%%%%
% Code and placeholder semantic tagging.
%
% When possible, prefer semantic tags instead of typographic tags. The
% following semantics tags are defined here:
%
%     \code{}     % for bold monospace keywords, code, operators, etc.
%     \plc{}      % for italic placeholder names, grammar, etc.
%
% For function prototypes or other code snippets, you can use \code{} as
% the outer wrapper, and use \plc{{} inside. Example:
%
%     \code{\#pragma omp directive ( \plc{some-placeholder-identifier} :}
%
% To format text in italics for emphasis (rather than text as a placeholder),
% use the generic \emph{} command. Example:
%
%     This sentence \emph{emphasizes some non-placeholder words}.

% Enable \alltt{} for formatting blocks of code:
\usepackage{alltt}

% This sets the default \code{} font to tt (monospace) and bold:
\newcommand{\code}[1]{{\texttt{\textbf{#1}}}}
\newcommand{\var}[1] {{\textrm{\textmd{\itshape{#1}}}}}


% Environment for a paragraph of literal code, single-spaced, no outline, no indenting:
\newenvironment{codepar}[1]
{\begin{alltt}\bfseries #1}
{\end{alltt}}

\usepackage{setspace}

%%%%%%%%%%%%%%%%%%%%%%%%%%%%%%%%%%%%%%%%%%%%%%%%%%%%%%%%%%%%%%%%%%%%%%%%%%%%%%%%%%%%%%%%%%%%%%
% Macros for the black and blue lines and arrows delineating language-specific
% and notes sections. Example:
%
%   \fortranspecificstart
%   This is text that applies to Fortran.
%   \fortranspecificend

% local parameters for use \linewitharrows and \notelinewitharrows:
\newlength{\sbsz}\setlength{\sbsz}{0.05in}  % size of arrows
\newlength{\sblw}\setlength{\sblw}{1.35pt}  % line width (thickness)
\newlength{\sbtw}                           % text width
\newlength{\sblen}                          % total width of horizontal rule
\newlength{\sbht}                           % height of arrows
\newlength{\sbhadj}                         % vertical adjustment for aligning arrows with the line
\newlength{\sbns}\setlength{\sbns}{7\baselineskip}       % arg for \needspace for downward arrows

% \notelinewitharrows is a helper command that makes a black Note marker:
%     arg 1 = 1 or -1 for up or down arrows
%     arg 2 = solid or dashed or loosely dashed, etc.
\newcommand{\notelinewitharrows}[2]{%
    \needspace{0.1\baselineskip}%
    \vbox{\begin{tikzpicture}%
        \setlength{\sblen}{\linewidth}%
        \setlength{\sbht}{#1\sbsz}\setlength{\sbht}{1.4\sbht}%
        \setlength{\sbhadj}{#1\sblw}\setlength{\sbhadj}{0.25\sbhadj}%
        \filldraw (\sblen, 0) -- (\sblen - \sbsz, \sbht) -- (\sblen - 2\sbsz, 0) -- (\sblen, 0);
        \draw[line width=\sblw, #2] (2\sbsz - \sblw, \sbhadj) -- (\sblen - 2\sbsz + \sblw, \sbhadj);
        \filldraw (0, 0) -- (\sbsz, \sbht) -- (0 + 2\sbsz, 0) -- (0, 0);
    \end{tikzpicture}}}

% \adviceuserline is a helper command that makes a red horizontal line, up or down arrows, and some text:
% arg 1 = 1 or -1 for up or down arrows
% arg 2 = solid or dashed or loosely dashed, etc.
% arg 3 = text
% arg 4 = text width
\newcommand{\adviceuserline}[4]{%
    \needspace{0.1\baselineskip}%
    \vbox to 1\baselineskip {\begin{tikzpicture}%
        \setlength{\sbtw}{#4}%
        \setlength{\sblen}{\linewidth}%
        \setlength{\sbht}{#1\sbsz}\setlength{\sbht}{1.4\sbht}%
        \setlength{\sbhadj}{#1\sblw}\setlength{\sbhadj}{0.25\sbhadj}%
        \filldraw[color=red!80!black] (\sblen, 0) -- (\sblen - \sbsz, \sbht) -- (\sblen - 2\sbsz, 0) -- (\sblen, 0);
        \draw[line width=\sblw, color=red!80!black, #2] (2\sbsz - \sblw, \sbhadj) -- (0.5\sblen - 0.5\sbtw, \sbhadj);
        \draw[line width=\sblw, color=red!80!black, #2] (0.5\sblen + 0.5\sbtw, \sbhadj) -- (\sblen - 2\sbsz + \sblw, \sbhadj);
        \filldraw[color=red!80!black] (0, 0) -- (\sbsz, \sbht) -- (0 + 2\sbsz, 0) -- (0, 0);
        \node[color=red!80!black] at (0.5\sblen, 0) {\large  \textsf{\textup{#3}}};
    \end{tikzpicture}}}

% \adviceimpline is a helper command that makes a green horizontal line, up or down arrows, and some text:
% arg 1 = 1 or -1 for up or down arrows
% arg 2 = solid or dashed or loosely dashed, etc.
% arg 3 = text
% arg 4 = text width
\newcommand{\adviceimpline}[4]{%
    \needspace{0.1\baselineskip}%
    \vbox to 1\baselineskip {\begin{tikzpicture}%
        \setlength{\sbtw}{#4}%
        \setlength{\sblen}{\linewidth}%
        \setlength{\sbht}{#1\sbsz}\setlength{\sbht}{1.4\sbht}%
        \setlength{\sbhadj}{#1\sblw}\setlength{\sbhadj}{0.25\sbhadj}%
        \filldraw[color=green!60!black] (\sblen, 0) -- (\sblen - \sbsz, \sbht) -- (\sblen - 2\sbsz, 0) -- (\sblen, 0);
        \draw[line width=\sblw, color=green!60!black, #2] (2\sbsz - \sblw, \sbhadj) -- (0.5\sblen - 0.5\sbtw, \sbhadj);
        \draw[line width=\sblw, color=green!60!black, #2] (0.5\sblen + 0.5\sbtw, \sbhadj) -- (\sblen - 2\sbsz + \sblw, \sbhadj);
        \filldraw[color=green!60!black] (0, 0) -- (\sbsz, \sbht) -- (0 + 2\sbsz, 0) -- (0, 0);
        \node[color=green!60!black] at (0.5\sblen, 0) {\large  \textsf{\textup{#3}}};
    \end{tikzpicture}}}

% \advicermline is a helper command that makes an orange horizontal line, up or down arrows, and some text:
% arg 1 = 1 or -1 for up or down arrows
% arg 2 = solid or dashed or loosely dashed, etc.
% arg 3 = text
% arg 4 = text width
\newcommand{\advicermline}[4]{%
    \needspace{0.1\baselineskip}%
    \vbox to 1\baselineskip {\begin{tikzpicture}%
        \setlength{\sbtw}{#4}%
        \setlength{\sblen}{\linewidth}%
        \setlength{\sbht}{#1\sbsz}\setlength{\sbht}{1.4\sbht}%
        \setlength{\sbhadj}{#1\sblw}\setlength{\sbhadj}{0.25\sbhadj}%
        \filldraw[color=orange!60!black] (\sblen, 0) -- (\sblen - \sbsz, \sbht) -- (\sblen - 2\sbsz, 0) -- (\sblen, 0);
        \draw[line width=\sblw, color=orange!60!black, #2] (2\sbsz - \sblw, \sbhadj) -- (0.5\sblen - 0.5\sbtw, \sbhadj);
        \draw[line width=\sblw, color=orange!60!black, #2] (0.5\sblen + 0.5\sbtw, \sbhadj) -- (\sblen - 2\sbsz + \sblw, \sbhadj);
        \filldraw[color=orange!60!black] (0, 0) -- (\sbsz, \sbht) -- (0 + 2\sbsz, 0) -- (0, 0);
        \node[color=orange!60!black] at (0.5\sblen, 0) {\large  \textsf{\textup{#3}}};
    \end{tikzpicture}}}

% \ratline is a helper command that makes a purple horizontal line, up or down arrows, and some text:
% arg 1 = 1 or -1 for up or down arrows
% arg 2 = solid or dashed or loosely dashed, etc.
% arg 3 = text
% arg 4 = text width
\newcommand{\ratline}[4]{%
    \needspace{0.1\baselineskip}%
    \vbox to 1\baselineskip {\begin{tikzpicture}%
        \setlength{\sbtw}{#4}%
        \setlength{\sblen}{\linewidth}%
        \setlength{\sbht}{#1\sbsz}\setlength{\sbht}{1.4\sbht}%
        \setlength{\sbhadj}{#1\sblw}\setlength{\sbhadj}{0.25\sbhadj}%
        \filldraw[color=purple!40] (\sblen, 0) -- (\sblen - \sbsz, \sbht) -- (\sblen - 2\sbsz, 0) -- (\sblen, 0);
        \draw[line width=\sblw, color=purple!40, #2] (2\sbsz - \sblw, \sbhadj) -- (0.5\sblen - 0.5\sbtw, \sbhadj);
        \draw[line width=\sblw, color=purple!40, #2] (0.5\sblen + 0.5\sbtw, \sbhadj) -- (\sblen - 2\sbsz + \sblw, \sbhadj);
        \filldraw[color=purple!40] (0, 0) -- (\sbsz, \sbht) -- (0 + 2\sbsz, 0) -- (0, 0);
        \node[color=purple!90] at (0.5\sblen, 0) {\large  \textsf{\textup{#3}}};
    \end{tikzpicture}}}

% \linewitharrows is a helper command that makes a blue horizontal line, up or down arrows, and some text:
% arg 1 = 1 or -1 for up or down arrows
% arg 2 = solid or dashed or loosely dashed, etc.
% arg 3 = text
% arg 4 = text width
\newcommand{\linewitharrows}[4]{%
    \needspace{0.1\baselineskip}%
    \vbox to 1\baselineskip {\begin{tikzpicture}%
        \setlength{\sbtw}{#4}%
        \setlength{\sblen}{\linewidth}%
        \setlength{\sbht}{#1\sbsz}\setlength{\sbht}{1.4\sbht}%
        \setlength{\sbhadj}{#1\sblw}\setlength{\sbhadj}{0.25\sbhadj}%
        \filldraw[color=blue!40] (\sblen, 0) -- (\sblen - \sbsz, \sbht) -- (\sblen - 2\sbsz, 0) -- (\sblen, 0);
        \draw[line width=\sblw, color=blue!40, #2] (2\sbsz - \sblw, \sbhadj) -- (0.5\sblen - 0.5\sbtw, \sbhadj);
        \draw[line width=\sblw, color=blue!40, #2] (0.5\sblen + 0.5\sbtw, \sbhadj) -- (\sblen - 2\sbsz + \sblw, \sbhadj);
        \filldraw[color=blue!40] (0, 0) -- (\sbsz, \sbht) -- (0 + 2\sbsz, 0) -- (0, 0);
        \node[color=blue!90] at (0.5\sblen, 0) {\large  \textsf{\textup{#3}}};
    \end{tikzpicture}}}

\newcommand{\VSPb}{\vspace{0.5ex plus 5ex minus 0.25ex}}
\newcommand{\VSPa}{\vspace{0.25ex plus 5ex minus 0.25ex}}

% C
\newcommand{\cspecificstart}{\needspace{\sbns}\linewitharrows{-1}{solid}{C}{3em}}
\newcommand{\cspecificend}{\linewitharrows{1}{solid}{C}{3em}\VSPa}

% Fortran
\newcommand{\fortranspecificstart}{\VSPb\linewitharrows{-1}{solid}{Fortran}{6em}\VSPa}
\newcommand{\fortranspecificend}{\VSPb\linewitharrows{1}{solid}{Fortran}{6em}\VSPa}

% Python
\newcommand{\pyspecificstart}{\needspace{\sbns}\linewitharrows{-1}{solid}{Python}{6em}}
\newcommand{\pyspecificend}{\linewitharrows{1}{solid}{Python}{6em}\VSPa}

% Note
\newcommand{\notestart}{\VSPb\notelinewitharrows{-1}{solid}\VSPa}
\newcommand{\noteend}{\VSPb\notelinewitharrows{1}{solid}\VSPa}
% convenience macro for formatting the word "Note:" at the beginning of note blocks:
\newcommand{\noteheader}{{\textrm{\textsf{\textbf\textup\normalsize{{{{Note: }}}}}}}}

% Rationale
\newcommand{\rationalestart}{\VSPb\ratline{-1}{dashed}{Rationale}{7em}\VSPa}
\newcommand{\rationaleend}{\VSPb\ratline{1}{dashed}{}{0em}\VSPa}

% Advice to users
\newcommand{\adviceuserstart}{\VSPb\adviceuserline{-1}{solid}{Advice to users}{10em}\VSPa}
\newcommand{\adviceuserend}{\VSPb\adviceuserline{1}{solid}{}{0em}\VSPa}

% Advice to implementers
\newcommand{\adviceimplstart}{\VSPb\adviceimpline{-1}{solid}{Advice to PMIx library implementers}{20em}\VSPa}
\newcommand{\adviceimplend}{\VSPb\adviceimpline{1}{solid}{}{0em}\VSPa}

% Advice to hosts
\newcommand{\advicermstart}{\VSPb\advicermline{-1}{solid}{Advice to PMIx server hosts}{16em}\VSPa}
\newcommand{\advicermend}{\VSPb\advicermline{1}{solid}{}{0em}\VSPa}

% Required attributes
\newcommand{\reqattrstart}{\VSPb\adviceuserline{-1}{dashed}{Required Attributes}{16em}\VSPa}
\newcommand{\reqattrend}{\VSPb\adviceuserline{1}{dashed}{}{0em}\VSPa}

% Optional attributes
\newcommand{\optattrstart}{\VSPb\adviceimpline{-1}{dashed}{Optional Attributes}{16em}\VSPa}
\newcommand{\optattrend}{\VSPb\adviceimpline{1}{dashed}{}{0em}\VSPa}


%%%%%%%%%%%%%%%%%%%%%%%%%%%%%%%%%%%%%%%%%%%%%%%%%%%%%%%%%%%%%%%%%%%%%%%%%%%%%%%%%%%%%%%%%%%%%%
% Glossary formatting

\newcommand{\glossaryterm}[1]{\needspace{1ex}
\begin{adjustwidth}{-0.75in}{0.0in}
\nolinenumbers\parbox[b][-0.95\baselineskip][t]{1.4in}{\flushright \textbf{#1}}
\end{adjustwidth}\linenumbers}

\newcommand{\glossarydefstart}{
\begin{adjustwidth}{0.79in}{0.0in}}

\newcommand{\glossarydefend}{
\end{adjustwidth}\vspace{-1.5\baselineskip}}


%%%%%%%%%%%%%%%%%%%%%%%%%%%%%%%%%%%%%%%%%%%%%%%%%%%%%%%%%%%%%%%%%%%%%%%%%%%%%%%%%%%%%%%%%%%%%
% Indexing and Table of Contents

\usepackage{imakeidx}
\usepackage[nodotinlabels]{titletoc}   % required for its [nodotinlabels] option

% Clickable links in TOC and index:
\usepackage[hyperindex=true,linktocpage=true]{hyperref}
\hypersetup{
  bookmarksnumbered = true,
  bookmarksopen     = false,
  colorlinks  = true, % Colors links instead of red boxes
  urlcolor    = blue, % Color for external links
  linkcolor   = blue  % Color for internal links
}

% \url styled in Roman font.
\urlstyle{rm}

%%%%%%%%%%%%%%%%%%%%%%%%%%%%%%%%%%%%%%%%%%%%%%%%%%%%%%%%%%%%%%%%%%%%%%%%%%%%%%%%%%%%%%%%%%%%%
% Cross reference macros
% This defines:
%     \specref          cross reference label as "Section X on page Y"
%     \refsection       Link this label to a specific section label in the document
%
%     \declarstruct     Mark the declaration of a structure
%     \refstruct        Reference the structure declaration
%
%     \declareapi       Mark the declaration of an API function
%     \refapi           Reference the API declaration
%
%     \declaremacro     Mark the declaration of a user-level macro
%     \refmacro         Reference the macro declaration
%

\newcommand{\chapterref}[1]{Chapter~\ref{#1} on page~\pageref{#1}}
\newcommand{\specref}[1]{Section~\ref{#1} on page~\pageref{#1}}

\newcommand{\refsection}[2]{\hyperref[#1]{#2}}

\newcommand{\declarestruct}[1]{\index{#1!Definition|textbf} \label{struct:#1}}
\newcommand{\refstruct}[1]{\index{#1} \hyperref[struct:#1]{\code{#1} }}
\newcommand{\structref}[1] {\refstruct{#1}}
\newcommand{\specrefstruct}[1]{Section~\ref{struct:#1} on page~\pageref{struct:#1}}

\newcommand{\declareapi}[1]{\index{#1!Definition|textbf} \label{api:#1}}
\newcommand{\refapi}[1]{\index{#1} \hyperref[api:#1]{\code{#1} }}
\newcommand{\argapi}[1] {\refapi{#1}}

\newcommand{\refconst}[1]{\hyperref[const:#1]{\code{#1} }}

\newcommand{\declareattr}[1]{\index{#1!Definition|textbf} \label{attr:#1}}

\newcommand{\refarg}[1] {{\textrm{\textmd{\itshape{#1}}}}}
\newcommand{\argref}[1] {\refarg{#1}}

\newcommand{\declaremacro}[1]{\index{#1!Definition|textbf} \label{macro:#1}}
\newcommand{\refmacro}[1]{\index{#1} \hyperref[macro:#1]{\code{#1} }}

\newcommand{\declareterm}[1]{\index{#1!Definition|textbf} \label{macro:#1}}
\newcommand{\refterm}[1]{\index{#1} \hyperref[macro:#1]{\code{#1} }}

% Place in text for in-text questions during review
\newcommand{\rcomment}[1]{(REVIEW COMMENT: \textbf{#1})}

%%%%%%%%%%%%%%%%%%%%%%%%%%%%%%%%%%%%%%%%%%%%%%%%%%%%%%%%%%%%%%%%%%%%%%%%%%%%%%%%%%%%%%%%%%%%%
% Set default fonts:
\rmfamily\mdseries\upshape\normalsize

%%%%%%%%%%%%%%%%%%%%%%%%%%%%%%%%%%%%%%%%%%%%%%%%%
% Define a divider for splitting implementer vs host attribute requirements/options
\newcommand{\divider}{\noindent\makebox[\linewidth]{\rule{\linewidth}{0.8pt}}}


\newcounter{pycounter}
\newcommand{\pylabel}[1]{\refstepcounter{pycounter} \label{appB:#1}}
\newcommand{\refpy}[1]{\hyperref[appB:#1]{\code{#1} }}

\makeindex[intoc,columns=2]

%%%%%%%%%%%%%%%%%%%
\usepackage{acronym}
\acrodef{PMI}[PMI]{Process Management Interface}
\acrodef{PMIx}[PMIx]{Process Management Interface - Exascale}
\acrodef{HPC}[HPC]{High Performance Computing}
\acrodef{MPI}[MPI]{Message Passing Interface}
\acrodef{RM}[RM]{resource manager}
\acrodef{SMS}[SMS]{system management software stack}
%%%%%%%%%%%%%%%%%%%


\begin{document}
%
% Title page
%
    \pagenumbering{roman}
    %%%%%%%%%%%%%%%%%%%%%%%%%%%%%%%%%%%%%%%%%%%%%%%%%
% Title page
%%%%%%%%%%%%%%%%%%%%%%%%%%%%%%%%%%%%%%%%%%%%%%%%%

  \begin{titlepage}
    \begin{flushleft}
     \hspace{-6em} \includegraphics[width=0.4\textwidth]{figs/pmix-logo.png}
    \end{flushleft}

    \begin{adjustwidth}{-0.75in}{0in}
    \begin{center}
      \Huge
      \textsf{Process Management Interface\\for Exascale (PMIx) Standard}

      \vspace{1.0in}
	  \huge
      \textbf{Version \VER{}}

      \vspace{0.15in}
	  \Large
      \textbf{\VERDATE}

    \end{center}
    \end{adjustwidth}

    \vspace{1.2in}

\par
This document describes the Process Management Interface for Exascale (PMIx) Standard, version \VER{}.

\par
\textbf{Comments:}
Please provide comments on the PMIx Standard by filing issues on the document repository \url{https://github.com/pmix/pmix-standard/issues} or by sending them to the PMIx Community mailing list at \url{https://groups.google.com/forum/#!forum/pmix}.
Comments should include the version of the PMIx standard you are commenting about, and the page, section, and line numbers that you are referencing.
Please note that messages sent to the mailing list from an unsubscribed e-mail address will be ignored.

\vfill

\begin{adjustwidth}{0pt}{1em}\setlength{\parskip}{0.25\baselineskip}%
Copyright \textsuperscript{\textcopyright} 2018-2020 PMIx \acf{ASC}.\\
Permission to copy without fee all or part of this material is granted,
provided the PMIx \ac{ASC} copyright notice and
the title of this document appear, and notice is given that copying is by
permission of PMIx \ac{ASC}.
\end{adjustwidth}

  \end{titlepage}

%%%%%%%%%%%%%%%%%%%%%%%%%%%%%%%%%%%%%%%%%%%%%%%%%
% Blank page
%%%%%%%%%%%%%%%%%%%%%%%%%%%%%%%%%%%%%%%%%%%%%%%%%
\clearpage
\thispagestyle{empty}
\phantom{a}
\begin{center}
\emph{This page intentionally left blank}
\end{center}

\vfill



%
% Table of contents
%
    \setcounter{page}{0}
    \setcounter{tocdepth}{2}

    \begin{spacing}{1.3}
        \tableofcontents
    \end{spacing}

%
% Introductory materials
%
    % Uncomment the next line to enable line numbering on the main body text:
    \linenumbers\pagewiselinenumbers
    \newpage\pagenumbering{arabic}
    \setcounter{chapter}{0}  % start chapter numbering here

%
% Chapters
%
    %%%%%%%%%%%%%%%%%%%%%%%%%%%%%%%%%%%%%%%%%%%%%%%%%
% Chapter: Introduction
%%%%%%%%%%%%%%%%%%%%%%%%%%%%%%%%%%%%%%%%%%%%%%%%%
\chapter{Introduction}
\label{chap:intro}

The \ac{PMI} has been used for quite some time as a means of exchanging wireup information needed for inter-process communication.
Two versions (PMI-1 and PMI-2) have been released as part of the MPICH effort, with PMI-2 demonstrating better scaling properties than its PMI-1 predecessor. However, two significant challenges face the \ac{HPC} community as it continues to move towards machines capable of exaflop and higher performance levels:

\begin{itemize}
\item the physical scale of the machines, and the corresponding number of total processes they support, is expected to reach levels approaching  1 million processes executing across 100 thousand nodes. Prior methods for initiating applications relied on exchanging communication endpoint information between the processes, either directly or in some form of hierarchical collective operation. Regardless of the specific mechanism employed, the exchange across such large applications would consume considerable time, with estimates running in excess of 5-10 minutes; and
\item whether it be hybrid applications that combine OpenMP threading operations with MPI, or application-steered workflow computations, the HPC community is experiencing an unprecedented wave of new approaches for computing at exascale levels. One common thread across the proposed methods is an increasing need for orchestration between the application and the \ac{SMS} comprising the scheduler (a.k.a. the \ac{WLM}), the \ac{RM}, global file system, fabric, and other subsystems. The lack of available support for application-to-SMS integration has forced researchers to develop "virtual" environments that hide the SMS behind a customized abstraction layer, but this results in considerable duplication of effort and a lack of portability.
\end{itemize}

\ac{PMIx} represents an attempt to resolve these questions by providing an extended version of the \ac{PMI} definitions specifically designed to support clusters up to exascale and larger sizes.
The overall objective of the project is not to branch the existing definitions -- in fact, PMIx fully supports both of the existing PMI-1 and PMI-2 APIs -- but rather to:

\begin{compactalphaenum}
\item augment those APIs to eliminate some current restrictions that impact scalability,
\item extend the breadth of the \ac{PMI} definitions to providing an abstraction layer for \ac{SMS} interactions,
\item establish a standards-like body for maintaining the definitions, and
\item provide a reference implementation of the PMIx standard that demonstrates the desired level of scalability and features.
\end{compactalphaenum}

Complete information about the \ac{PMIx} standard and affiliated projects can be found at the \ac{PMIx} web site: \url{https://pmix.org}


%%%%%%%%%%%%%%%%%%%%%%%%%%%%%%%%%%%%%%%%%%%%%%%%%
%%%%%%%%%%%%%%%%%%%%%%%%%%%%%%%%%%%%%%%%%%%%%%%%%
\section{Charter}
\label{chap:intro:charter}

The charter of the PMIx community is to:
\begin{itemize}
\item Define a set of agnostic APIs (not affiliated with any specific programming model or code base) to support interactions between application processes and the \ac{SMS}.
\item Develop an open source (non-copy-left licensed) standalone ``reference'' library to facilitate adoption of the \ac{PMIx} standard.
\item Retain transparent backward compatibility with the existing PMI-1 and PMI-2 definitions, any future \ac{PMI} releases, and across all \ac{PMIx} versions.
\item Support the ``Instant On'' initiative for rapid startup of applications at exascale and beyond.
\item Work with the \ac{HPC} community to define and implement new APIs that support evolving programming model requirements for application interactions with the \ac{SMS}.
\end{itemize}

Participation in the \ac{PMIx} community is open to anyone, and not restricted to only code contributors to the reference implementation.


%%%%%%%%%%%%%%%%%%%%%%%%%%%%%%%%%%%%%%%%%%%%%%%%%
%%%%%%%%%%%%%%%%%%%%%%%%%%%%%%%%%%%%%%%%%%%%%%%%%
\section{PMIx Standard Overview}
\label{chap:intro:std_overview}

\ldots

%%%%%%%%%%%
\subsection{Who should use the standard?}

\ldots

%%%%%%%%%%%
\subsection{What is defined in the standard?}

\ldots

%%%%%%%%%%%
\subsection{What is \emph{not} defined in the standard?}

The \ac{PMIx} Standard does not include anything, either stated or implied, regarding implementation.
It instead focuses exclusively on defining APIs and associated attribute key strings, and describing the expected behavior of those entities.
How that behavior is realized is entirely at the discretion of the implementer.

As previously noted, system environments and \ac{PMIx} library implementers are free to return ``not supported'' for any request. Thus, users should design their applications accordingly.


%%%%%%%%%%%%%%%%%%%%%%%%%%%%%%%%%%%%%%%%%%%%%%%%%
%%%%%%%%%%%%%%%%%%%%%%%%%%%%%%%%%%%%%%%%%%%%%%%%%
\section{PMIx Architecture Overview}
\label{chap:intro:arch_overview}

This section presents a brief overview the \ac{PMIx} Architecture~\cite{2017-Castain-EuroMPI}.

\ldots

%%%%%%%%%%%
\subsection{The PMIx Reference Implementation}

Note that the definition of the \ac{PMIx} Standard is not contingent upon use of the \ac{PMIx} Reference Implementation.
Any implementation that supports the defined APIs is a \ac{PMIx} Standard compliant implementation, and some environments have chosen to pursue their own custom implementation.
The \ac{PMIx} Reference Implementation is provided solely for the following purposes:
\begin{itemize}
\item Validation of the standard.\\
No proposed change and/or extension to the \ac{PMIx} standard is accepted without an accompanying prototype implementation in the \ac{PMIx} Reference Implementation.
This ensures that the proposal has undergone at least some minimal level of scrutiny and testing before being considered.
\item Ease of adoption.\\
The \ac{PMIx} Reference Implementation is designed to be particularly easy for resource managers (and the \ac{SMS} in general) to adopt, thus facilitating a rapid uptake into that community for application portability.
Both client and server \ac{PMIx} libraries are included, along with examples of client usage and server-side integration.
A list of supported environments and versions is provided on the \ac{PMIx} web site \url{www.pmix.org}
\end{itemize}

The \ac{PMIx} Reference Implementation targets support for the Linux operating system.
A reasonable effort is made to support all major, modern Linux distributions; however, validation is limited to the most recent 2-3 releases of RedHat Enterprise Linux (RHEL), Fedora, CentOS, and SUSE Linux Enterprise Server (SLES).
In addition, development support is maintained for Mac OSX.
Production support for vendor-specific operating systems is included as provided by the vendor.

%%%%%%%%%%%
\subsection{The PMIx Reference Server}

\ldots


%%%%%%%%%%%%%%%%%%%%%%%%%%%%%%%%%%%%%%%%%%%%%%%%%
\section{Organization of this document}

The remainder of this document is structured as follows:

\begin{itemize}
\item Introduction and Overview in \chapterref{chap:intro}
\item Terms and Conventions in \chapterref{chap:terms}
\item Data Structures and Types in \chapterref{chap:struct}
\item \ac{PMIx} Initialization and Finalization in \chapterref{chap:api_init}
\item Key/Value Management in \chapterref{chap:api_kv_mgmt}
\item Process Management in \chapterref{chap:api_proc_mgmt}
\item Job Management in \chapterref{chap:api_job_mgmt}
\item Event Notification in \chapterref{chap:api_event}
\item Data Packing and Unpacking in \chapterref{chap:api_data_mgmt}
\item \ac{PMIx} Server Specific Interfaces in \chapterref{chap:api_server}
\end{itemize}

%%%%%%%%%%%%%%%%%%%%%%%%%%%%%%%%%%%%%%%%%%%%%%%%%

    %%%%%%%%%%%%%%%%%%%%%%%%%%%%%%%%%%%%%%%%%%%%%%%%%
% Chapter: Terms and Conventions
%%%%%%%%%%%%%%%%%%%%%%%%%%%%%%%%%%%%%%%%%%%%%%%%%
\chapter{PMIx Terms and Conventions}
\label{chap:terms}

Define ``attributes'' and how they are used, intent is to allow for definition of flexible APIs that can change behavior based on attributes instead of modifying function signature.
Include description of data types.

This document borrows freely from other standards (most notably from the \ac{MPI} and OpenMP standards) in its use of notation and conventions in an attempt to reduce confusion.

%%%%%%%%%%%
\section{Notional Conventions}

Some sections of this document describe programming language specific examples or APIs.
Text that applies only to programs for which the base language is C is show as follows:

\cspecificstart
C specific text...
\begin{codepar}
int foo = 42;
\end{codepar}
\cspecificend

Some text is for information only, and is not part of the normative specification.
These take three forms, described in their examples below:

\notestart
\noteheader
General text...
\noteend

\rationalestart
Throughout this document, the rationale for the design choices made in the interface specification is set off in this section.
Some readers may wish to skip these sections, while readers interested in interface design may want to read them carefully.
\rationaleend

\adviceuserstart
Throughout this document, material aimed at users and that illustrates usage is set off in this section.
Some readers may wish to skip these sections, while readers interested in programming in \ac{MPI} may want to read them carefully.
\adviceuserend

\adviceimplstart
Throughout this document, material that is primarily commentary to implementers is set off in this section.
Some readers may wish to skip these sections, while readers interested in \ac{PMIx} implementations may want to read them carefully. 
\adviceimplend

%%%%%%%%%%%
\section{Semantics}

The following terms will be taken to mean:

\begin{itemize}
\item \emph{shall} and \emph{will} indicate that the specified behavior is \emph{required} of all conforming implementations
\item \emph{should} and \emph{may} indicate behaviors that a quality implementation would include, but are not required of all conforming implementations
\end{itemize}

%%%%%%%%%%%
\section{Naming Conventions}

\ldots

%%%%%%%%%%%
\section{Procedure Conventions}

While current \ac{PMIx} Reference Implementation is solely based on the C programming language, it is not the intent of the \ac{PMIx} Standard to preclude the use of other languages.
Accordingly, the procedure specifications in the \ac{PMIx} Standard are written in a language-independent syntax with the arguments marked as IN, OUT, or INOUT.
The meanings of these are:
\begin{itemize}
\item IN:
The call may use the input value but does not update the argument from the perspective of the caller at any time during the call?s execution, 
\item OUT:
The call may update the argument but does not use its input value
\item INOUT:
The call may both use and update the argument. 
\end{itemize}

%%%%%%%%%%%%%%%%%%%%%%%%%%%%%%%%%%%%%%%%%%%%%%%%%

    %%%%%%%%%%%%%%%%%%%%%%%%%%%%%%%%%%%%%%%%%%%%%%%%%
% Chapter: Overview
%%%%%%%%%%%%%%%%%%%%%%%%%%%%%%%%%%%%%%%%%%%%%%%%%
\chapter{PMIx Architecture Overview}
\label{chap:overview}

\ldots

%%%%%%%%%%%%%%%%%%%%%%%%%%%%%%%%%%%%%%%%%%%%%%%%%

    %%%%%%%%%%%%%%%%%%%%%%%%%%%%%%%%%%%%%%%%%%%%%%%%%
% Chapter: Data Structures
%%%%%%%%%%%%%%%%%%%%%%%%%%%%%%%%%%%%%%%%%%%%%%%%%
\chapter{Data Structures and Types}
\label{chap:struct}

This chapter defines \ac{PMIx} standard data structures (along with macros for convenient use), types, and constants.
These apply to all consumers of the \ac{PMIx} interface.
Where necessary for clarification, the description of, for example, an attribute may be copied from this chapter into a section where it is used.

A PMIx implementation may define additional attributes beyond those specified in this document.

\adviceimplstart
Structures, types, and macros in the \ac{PMIx} Standard are defined in terms of the C-programming language. Implementers wishing to support other languages should provide the equivalent definitions in a language-appropriate manner.

If a PMIx implementation chooses to define additional attributes they should avoid using the \code{"PMIX"} prefix in their name or starting the attribute string with a \code{"pmix"} prefix.
This helps the end user distinguish between what is defined by the PMIx standard and what is specific to that PMIx implementation, and avoids potential conflicts with attributes defined by the Standard.
\adviceimplend

\adviceuserstart
Use of increment/decrement operations on indices inside \ac{PMIx} macros is discouraged due to unpredictable behavior. For example, the following sequence:

\begin{codepar}
PMIX_INFO_LOAD(&array[n++], "mykey", &mystring, PMIX_STRING);
PMIX_INFO_LOAD(&array[n++], "mykey2", &myint, PMIX_INT);
\end{codepar}

will load the given key-values into incorrect locations if the macro is implemented as:

\begin{codepar}
define PMIX_INFO_LOAD(m, k, v, t)                      \textbackslash
  do \{                                                 \textbackslash
    if (NULL != (k)) \{                                 \textbackslash
      pmix_strncpy((m)->key, (k), PMIX_MAX_KEYLEN);    \textbackslash
    \}                                                  \textbackslash
    (m)->flags = 0;                                    \textbackslash
    pmix_value_load(&((m)->value), (v), (t));          \textbackslash
  \} while (0)
\end{codepar}

since the index is cited more than once in the macro. The \ac{PMIx} standard only governs the existence and syntax of macros - it does not specify their implementation. Given the freedom of implementation, a safer call sequence might be as follows:

\begin{codepar}
PMIX_INFO_LOAD(&array[n], "mykey", &mystring, PMIX_STRING);
++n;
PMIX_INFO_LOAD(&array[n], "mykey2", &myint, PMIX_INT);
++n;
\end{codepar}

Users are also advised to use the macros for creating, loading, and releasing
\ac{PMIx} structures to avoid potential issues with release of memory. For
example, pointing a \refstruct{pmix_envar_t} element at a static string
variable and then using \refmacro{PMIX_ENVAR_DESTRUCT} to clear it would
generate an error as the static string had not been allocated.

\adviceuserend

%%%%%%%%%%%%%%%%%%%%%%%%%%%%%%%%%%%%%%%%%%%%%%%%%
%%%%%%%%%%%%%%%%%%%%%%%%%%%%%%%%%%%%%%%%%%%%%%%%%
\section{Constants}
\label{chap:struct:const}

\ac{PMIx} defines a few values that are used throughout the standard to set the size of fixed arrays or as a means of identifying values with special meaning.
The community makes every attempt to minimize the number of such definitions.
The constants defined in this section may be used before calling any \ac{PMIx} library initialization routine.
Additional constants associated with specific data structures or types are defined in the section describing that data structure or type.

\begin{constantdesc}
%
\declareconstitem{PMIX_MAX_NSLEN}
Maximum namespace string length as an integer.
\end{constantdesc}

\adviceimplstart
\refconst{PMIX_MAX_NSLEN} should have a minimum value of 63 characters. Namespace arrays in \ac{PMIx} defined structures must reserve
a space of size \refconst{PMIX_MAX_NSLEN}+1 to allow room for the \code{NULL} terminator
\adviceimplend

\begin{constantdesc}
%
\declareconstitem{PMIX_MAX_KEYLEN}
Maximum key string length as an integer.
\end{constantdesc}

\adviceimplstart
\refconst{PMIX_MAX_KEYLEN} should have a minimum value of 63 characters. Key arrays in \ac{PMIx} defined structures must reserve
a space of size \refconst{PMIX_MAX_KEYLEN}+1 to allow room for the \code{NULL} terminator
\adviceimplend

\begin{constantdesc}
%
\declareconstitemNEW{PMIX_APP_WILDCARD}
A value to indicate that the user wants the data for the given key from every application that posted that key, or that the given value applies to all applications within the given namespace.
\end{constantdesc}


%%%%%%%%%%%%%%%%%%%%%%%%%%%%%%%%%%%%%%%%%%%%%%%%%
\subsection{PMIx Return Status Constants}
\label{api:struct:errors}
\declarestruct{pmix_status_t}

The \refstruct{pmix_status_t} structure is an \code{int} type for return status. The tables shown in this section define the possible values for \refstruct{pmix_status_t}.
PMIx errors are required to always be negative, with \code{0} reserved for \refconst{PMIX_SUCCESS}. Values in the list that were deprecated in later standards are denoted as such. Values added to the list in this version of the standard are shown in \textbf{\color{magenta}magenta}.

\adviceimplstart
A PMIx implementation must define all of the constants defined in this section, even if they will never return the specific value to the caller.
\adviceimplend

\adviceuserstart
Other than \refconst{PMIX_SUCCESS} (which is required to be zero), the actual value of any \ac{PMIx} error constant is left to the \ac{PMIx} library implementer. Thus, users are advised to always refer to constant by name, and not a specific implementation's value, for portability between implementations and compatibility across library versions.
\adviceuserend

The following values are general constants used in a variety of places.

\begin{constantdesc}
%
\declareconstitem{PMIX_SUCCESS}
Success.
%
\declareconstitem{PMIX_ERROR}
General Error.
%
\declareconstitemNEW{PMIX_ERR_EXISTS}
Requested operation would overwrite an existing value - typically returned
when an operation would overwrite an existing file or directory.
%
\declareconstitemNEW{PMIX_ERR_EXISTS_OUTSIDE_SCOPE}
The requested key exists, but was posted in a \emph{scope} (see Section \ref{api:nres:scope}) that does not include the requester
%
\declareconstitem{PMIX_ERR_INVALID_CRED}
Invalid security credentials.
%
\declareconstitem{PMIX_ERR_WOULD_BLOCK}
Operation would block.
%
\declareconstitem{PMIX_ERR_UNKNOWN_DATA_TYPE}
The data type specified in an input to the \ac{PMIx} library is not recognized
by the implementation.
%
\declareconstitem{PMIX_ERR_TYPE_MISMATCH}
The data type found in an object does not match the expected data type
as specified in the \ac{API} call - e.g., a request to unpack a
\refconst{PMIX_BOOL} value from a buffer that does not contain a value of
that type in the current unpack location.
%
\declareconstitem{PMIX_ERR_UNPACK_INADEQUATE_SPACE}
Inadequate space to unpack data - the number of values in the buffer exceeds
the specified number to unpack.
%
\declareconstitem{PMIX_ERR_UNPACK_READ_PAST_END_OF_BUFFER}
Unpacking past the end of the provided buffer - the number of values in the
buffer is less than the specified number to unpack, or a request was made to
unpack a buffer beyond the buffer's end.
%
\declareconstitem{PMIX_ERR_UNPACK_FAILURE}
The unpack operation failed for an unspecified reason.
%
\declareconstitem{PMIX_ERR_PACK_FAILURE}
The pack operation failed for an unspecified reason.
%
\declareconstitem{PMIX_ERR_NO_PERMISSIONS}
The user lacks permissions to execute the specified operation.
%
\declareconstitem{PMIX_ERR_TIMEOUT}
Either a user-specified or system-internal timeout expired.
%
\declareconstitem{PMIX_ERR_UNREACH}
The specified target server or client process is not reachable - i.e., a
suitable connection either has not been or can not be made.
%
\declareconstitem{PMIX_ERR_BAD_PARAM}
One or more incorrect parameters (e.g., passing an attribute with a value of the wrong type), or multiple parameters containing conflicting directives (e.g., multiple instances of the same attribute with different values, or different attributes specifying conflicting behaviors), were passed to a \ac{PMIx} \ac{API}.
%
\declareconstitemNEW{PMIX_ERR_EMPTY}
An array or list was given that has no members in it - i.e., the object is empty.
%
\declareconstitem{PMIX_ERR_RESOURCE_BUSY}
Resource busy - typically seen when an attempt to establish a connection
to another process (e.g., a \ac{PMIx} server) cannot be made due to a
communication failure.
%
\declareconstitem{PMIX_ERR_OUT_OF_RESOURCE}
Resource exhausted.
%
\declareconstitem{PMIX_ERR_INIT}
Error during initialization.
%
\declareconstitem{PMIX_ERR_NOMEM}
Out of memory.
%
\declareconstitem{PMIX_ERR_NOT_FOUND}
The requested information was not found.
%
\declareconstitem{PMIX_ERR_NOT_SUPPORTED}
The requested operation is not supported by either the \ac{PMIx} implementation
or the host environment.
%
\declareconstitemNEW{PMIX_ERR_PARAM_VALUE_NOT_SUPPORTED}
The requested operation is supported by the \ac{PMIx} implementation and (if applicable) the host environment. However, at least one supplied parameter was given an unsupported value, and the operation cannot therefore be executed as requested.
%
\declareconstitem{PMIX_ERR_COMM_FAILURE}
Communication failure - a message failed to be sent or received, but the
connection remains intact.
%
\declareconstitemNEW{PMIX_ERR_LOST_CONNECTION}
Lost connection between server and client or tool.
%
\declareconstitem{PMIX_ERR_INVALID_OPERATION}
The requested operation is supported by the implementation and host environment, but fails to meet a requirement (e.g., requesting to \textit{disconnect} from processes without first \textit{connecting} to them, inclusion of conflicting directives, or a request to perform an operation that conflicts with an ongoing one).
%
\declareconstitem{PMIX_OPERATION_IN_PROGRESS}
A requested operation is already in progress - the duplicate request
shall therefore be ignored.
%
\declareconstitem{PMIX_OPERATION_SUCCEEDED}
The requested operation was performed atomically - no callback function will be executed.
%
\declareconstitemNEW{PMIX_ERR_PARTIAL_SUCCESS}
The operation is considered successful but not all elements of the operation were concluded (e.g., some members of a group construct operation chose not to participate).
%
\end{constantdesc}


%%%%%%%%%%%%%%%%%%%%%%%%%%%%%%%%%%%%%%%%%%%%%%%%%
\subsubsection{User-Defined Error and Event Constants}
\label{api:struct:usererrors}

\ac{PMIx} establishes a boundary for constants defined in the \ac{PMIx} standard. Negative values larger (i.e., more negative) than this (and any positive values greater than zero) are guaranteed not to conflict with \ac{PMIx} values.

\begin{constantdesc}
%
\declareconstitem{PMIX_EXTERNAL_ERR_BASE}
A starting point for user-level defined error and event constants.
Negative values that are more negative than the defined constant are guaranteed not to conflict with \ac{PMIx} values.
Definitions should always be based on the \refconst{PMIX_EXTERNAL_ERR_BASE} constant and not a specific value as the value of the constant may change.
%
\end{constantdesc}



%%%%%%%%%%%%%%%%%%%%%%%%%%%%%%%%%%%%%%%%%%%%%%%%%
%%%%%%%%%%%%%%%%%%%%%%%%%%%%%%%%%%%%%%%%%%%%%%%%%
\section{Data Types}

This section defines various data types used by the \ac{PMIx} APIs. The version of the standard in which a particular data type was introduced is shown in the margin.

%%%%%%%%%%%%%%%%%%%%%%%%%%%%%%%%%%%%%%%%%%%%%%%%%
\subsection{Key Structure}
\declarestruct{pmix_key_t}

The \refstruct{pmix_key_t} structure is a statically defined character array of length \refconst{PMIX_MAX_KEYLEN}+1, thus supporting keys of maximum length \refconst{PMIX_MAX_KEYLEN} while preserving space for a mandatory \code{NULL} terminator.

\versionMarker{2.0}
\cspecificstart
\begin{codepar}
typedef char pmix_key_t[PMIX_MAX_KEYLEN+1];
\end{codepar}
\cspecificend

Characters in the key must be standard alphanumeric values supported by common utilities such as \textit{strcmp}.

\adviceuserstart
References to keys in \ac{PMIx} v1 were defined simply as an array of characters of size \code{PMIX_MAX_KEYLEN+1}. The \refstruct{pmix_key_t} type definition was introduced in version 2 of the standard. The two definitions are code-compatible and thus do not represent a break in backward compatibility.

Passing a \refstruct{pmix_key_t} value to the standard \textit{sizeof} utility can result in compiler warnings of incorrect returned value. Users are advised to avoid using \textit{sizeof(pmix_key_t)} and instead rely on the \refconst{PMIX_MAX_KEYLEN} constant.
\adviceuserend

%%%%%%%%%%%%%%%%%%%%%%%%%%%%%%%%%%%%%%%%%%%%%%%%%
\subsubsection{Key support macros}

The following macros are provided for convenience when working with \ac{PMIx} keys.

\littleheader{Check key macro}
\declaremacro{PMIX_CHECK_KEY}

Compare the key in a \refstruct{pmix_info_t} to a given value.

\versionMarker{3.0}
\cspecificstart
\begin{codepar}
PMIX_CHECK_KEY(a, b)
\end{codepar}
\cspecificend

\begin{arglist}
\argin{a}{Pointer to the structure whose key is to be checked (pointer to \refstruct{pmix_info_t})}
\argin{b}{String value to be compared against (\code{char*})}
\end{arglist}

Returns \code{true} if the key matches the given value

\littleheader{Check reserved key macro}
\declaremacro{PMIX_CHECK_RESERVED_KEY}

Check if the given key is a \ac{PMIx} \emph{reserved} key as described in Chapter \ref{chap:api_rsvd_keys}.

\versionMarker{4.0}
\cspecificstart
\begin{codepar}
PMIX_CHECK_RESERVED_KEY(a)
\end{codepar}
\cspecificend

\begin{arglist}
\argin{a}{String value to be checked (\code{char*})}
\end{arglist}

Returns \code{true} if the key is reserved by the Standard.

\littleheader{Load key macro}
\declaremacro{PMIX_LOAD_KEY}

Load a key into a \refstruct{pmix_info_t}.

\versionMarker{4.0}
\cspecificstart
\begin{codepar}
PMIX_LOAD_KEY(a, b)
\end{codepar}
\cspecificend

\begin{arglist}
\argin{a}{Pointer to the structure whose key is to be loaded (pointer to \refstruct{pmix_info_t})}
\argin{b}{String value to be loaded (\code{char*})}
\end{arglist}

No return value.

%%%%%%%%%%%%%%%%%%%%%%%%%%%%%%%%%%%%%%%%%%%%%%%%%
\subsection{Namespace Structure}
\declarestruct{pmix_nspace_t}

The \refstruct{pmix_nspace_t} structure is a statically defined character array of length \refconst{PMIX_MAX_NSLEN}+1, thus supporting namespaces of maximum length \refconst{PMIX_MAX_NSLEN} while preserving space for a mandatory \code{NULL} terminator.

\versionMarker{2.0}
\cspecificstart
\begin{codepar}
typedef char pmix_nspace_t[PMIX_MAX_NSLEN+1];
\end{codepar}
\cspecificend

Characters in the namespace must be standard alphanumeric values supported by common utilities such as \textit{strcmp}.

\adviceuserstart
References to namespace values in \ac{PMIx} v1 were defined simply as an array of characters of size \code{PMIX_MAX_NSLEN+1}. The \refstruct{pmix_nspace_t} type definition was introduced in version 2 of the standard. The two definitions are code-compatible and thus do not represent a break in backward compatibility.

Passing a \refstruct{pmix_nspace_t} value to the standard \textit{sizeof} utility can result in compiler warnings of incorrect returned value. Users are advised to avoid using \textit{sizeof(pmix_nspace_t)} and instead rely on the \refconst{PMIX_MAX_NSLEN} constant.
\adviceuserend

%%%%%%%%%%%%%%%%%%%%%%%%%%%%%%%%%%%%%%%%%%%%%%%%%
\subsubsection{Namespace support macros}

The following macros are provided for convenience when working with \ac{PMIx} namespace structures.

\littleheader{Check namespace macro}
\declaremacro{PMIX_CHECK_NSPACE}

Compare the string in a \refstruct{pmix_nspace_t} to a given value.

\versionMarker{3.0}
\cspecificstart
\begin{codepar}
PMIX_CHECK_NSPACE(a, b)
\end{codepar}
\cspecificend

\begin{arglist}
\argin{a}{Pointer to the structure whose value is to be checked (pointer to \refstruct{pmix_nspace_t})}
\argin{b}{String value to be compared against (\code{char*})}
\end{arglist}

Returns \code{true} if the namespace matches the given value

\littleheader{Check invalid namespace macro}
\declaremacro{PMIX_NSPACE_INVALID}

Check the string in a \refstruct{pmix_nspace_t}

\versionMarker{4.1}
\cspecificstart
\begin{codepar}
PMIX_NSPACE_INVALID(a)
\end{codepar}
\cspecificend

\begin{arglist}
\argin{a}{Pointer to the structure whose value is to be checked (pointer to \refstruct{pmix_nspace_t})}
\end{arglist}

Returns \code{true} if the namespace is invalid (i.e., starts with a \code{NULL} resulting in a zero-length string value)

\littleheader{Load namespace macro}
\declaremacro{PMIX_LOAD_NSPACE}

Load a namespace into a \refstruct{pmix_nspace_t}.

\versionMarker{4.0}
\cspecificstart
\begin{codepar}
PMIX_LOAD_NSPACE(a, b)
\end{codepar}
\cspecificend

\begin{arglist}
\argin{a}{Pointer to the target structure (pointer to \refstruct{pmix_nspace_t})}
\argin{b}{String value to be loaded - if \code{NULL} is given, then the target structure will be initialized to zero's (\code{char*})}
\end{arglist}

No return value.


%%%%%%%%%%%%%%%%%%%%%%%%%%%%%%%%%%%%%%%%%%%%%%%%%
\subsection{Rank Structure}
\declarestruct{pmix_rank_t}

The \refstruct{pmix_rank_t} structure is a \code{uint32_t} type for rank values.

\versionMarker{1.0}
\cspecificstart
\begin{codepar}
typedef uint32_t pmix_rank_t;
\end{codepar}
\cspecificend

The following constants can be used to set a variable of the type \refstruct{pmix_rank_t}. All definitions were introduced in version 1 of the standard unless otherwise marked. Valid rank values start at zero.

\begin{constantdesc}
%
\declareconstitem{PMIX_RANK_UNDEF}
A value to request job-level data where the information itself is not associated with any specific rank, or when passing a \refstruct{pmix_proc_t} identifier to an operation that only references the namespace field of that structure.
%
\declareconstitem{PMIX_RANK_WILDCARD}
A value to indicate that the user wants the data for the given key from every rank that posted that key.
%
\declareconstitem{PMIX_RANK_LOCAL_NODE}
Special rank value used to define groups of ranks.
This constant defines the group of all ranks on a local node.
%
\declareconstitem{PMIX_RANK_LOCAL_PEERS}
Special rank value used to define groups of ranks.
This constant defines the group of all ranks on a local node within the same namespace as the current process.
%
\declareconstitem{PMIX_RANK_INVALID}
An invalid rank value.
%
\declareconstitem{PMIX_RANK_VALID}
Define an upper boundary for valid rank values.
%
\end{constantdesc}


%%%%%%%%%%%%%%%%%%%%%%%%%%%%%%%%%%%%%%%%%%%%%%%%%
\subsubsection{Rank support macros}

The following macros are provided for convenience when working with \ac{PMIx} ranks.

\littleheader{Check rank macro}
\declaremacro{PMIX_CHECK_RANK}

Check two ranks for equality, taking into account wildcard values

\versionMarker{4.0}
\cspecificstart
\begin{codepar}
PMIX_CHECK_RANK(a, b)
\end{codepar}
\cspecificend

\begin{arglist}
\argin{a}{Rank to be checked (\refstruct{pmix_rank_t})}
\argin{b}{Rank to be checked (\refstruct{pmix_rank_t})}
\end{arglist}

Returns \code{true} if the ranks are equal, or at least one of the ranks is \refconst{PMIX_RANK_WILDCARD}

\littleheader{Check rank is valid macro}
\declaremacro{PMIX_RANK_IS_VALID}

Check is the given rank is a valid value

\versionMarker{4.1}
\cspecificstart
\begin{codepar}
PMIX_RANK_IS_VALID(a)
\end{codepar}
\cspecificend

\begin{arglist}
\argin{a}{Rank to be checked (\refstruct{pmix_rank_t})}
\end{arglist}

Returns \code{true} if the given rank is valid (i.e., less than \refconst{PMIX_RANK_VALID})

%%%%%%%%%%%%%%%%%%%%%%%%%%%%%%%%%%%%%%%%%%%%%%%%%
\subsection{Process Structure}
\declarestruct{pmix_proc_t}

The \refstruct{pmix_proc_t} structure is used to identify a single process in the PMIx universe.
It contains a reference to the namespace and the \refstruct{pmix_rank_t} within that namespace.

\versionMarker{1.0}
\cspecificstart
\begin{codepar}
typedef struct pmix_proc \{
    pmix_nspace_t nspace;
    pmix_rank_t rank;
\} pmix_proc_t;
\end{codepar}
\cspecificend

%%%%%%%%%%%%%%%%%%%%%%%%%%%%%%%%%%%%%%%%%%%%%%%%%
\subsubsection{Process structure support macros}
The following macros are provided to support the \refstruct{pmix_proc_t} structure.

\littleheader{Initialize the proc structure}
\declaremacro{PMIX_PROC_CONSTRUCT}

Initialize the \refstruct{pmix_proc_t} fields.

\versionMarker{1.0}
\cspecificstart
\begin{codepar}
PMIX_PROC_CONSTRUCT(m)
\end{codepar}
\cspecificend

\begin{arglist}
\argin{m}{Pointer to the structure to be initialized (pointer to \refstruct{pmix_proc_t})}
\end{arglist}

\littleheader{Destruct the proc structure}
\declaremacro{PMIX_PROC_DESTRUCT}

Destruct the \refstruct{pmix_proc_t} fields.

\cspecificstart
\begin{codepar}
PMIX_PROC_DESTRUCT(m)
\end{codepar}
\cspecificend

\begin{arglist}
\argin{m}{Pointer to the structure to be destructed (pointer to \refstruct{pmix_proc_t})}
\end{arglist}

There is nothing to release here as the fields in \refstruct{pmix_proc_t} are either a statically-declared array (the namespace) or a single value (the rank). However, the macro is provided for symmetry in the code and for future-proofing should some allocated field be included some day.

\littleheader{Create a proc array}
\declaremacro{PMIX_PROC_CREATE}

Allocate and initialize an array of \refstruct{pmix_proc_t} structures.

\versionMarker{1.0}
\cspecificstart
\begin{codepar}
PMIX_PROC_CREATE(m, n)
\end{codepar}
\cspecificend

\begin{arglist}
\arginout{m}{Address where the pointer to the array of \refstruct{pmix_proc_t} structures shall be stored (handle)}
\argin{n}{Number of structures to be allocated (\code{size_t})}
\end{arglist}


\littleheader{Free a proc structure}
\declaremacro{PMIX_PROC_RELEASE}

Release a \refstruct{pmix_proc_t} structure.

\versionMarker{4.0}
\cspecificstart
\begin{codepar}
PMIX_PROC_RELEASE(m)
\end{codepar}
\cspecificend

\begin{arglist}
\argin{m}{Pointer to a \refstruct{pmix_proc_t} structure (handle)}
\end{arglist}

\littleheader{Free a proc array}
\declaremacro{PMIX_PROC_FREE}

Release an array of \refstruct{pmix_proc_t} structures.

\versionMarker{1.0}
\cspecificstart
\begin{codepar}
PMIX_PROC_FREE(m, n)
\end{codepar}
\cspecificend

\begin{arglist}
\argin{m}{Pointer to the array of \refstruct{pmix_proc_t} structures (handle)}
\argin{n}{Number of structures in the array (\code{size_t})}
\end{arglist}

\littleheader{Load a proc structure}
\declaremacro{PMIX_PROC_LOAD}

Load values into a \refstruct{pmix_proc_t}.

\versionMarker{2.0}
\cspecificstart
\begin{codepar}
PMIX_PROC_LOAD(m, n, r)
\end{codepar}
\cspecificend

\begin{arglist}
\argin{m}{Pointer to the structure to be loaded (pointer to \refstruct{pmix_proc_t})}
\argin{n}{Namespace to be loaded (\refstruct{pmix_nspace_t})}
\argin{r}{Rank to be assigned (\refstruct{pmix_rank_t})}
\end{arglist}

No return value. Deprecated in favor of \refmacro{PMIX_LOAD_PROCID}

\littleheader{Compare identifiers}
\declaremacro{PMIX_CHECK_PROCID}

Compare two \refstruct{pmix_proc_t} identifiers.

\versionMarker{3.0}
\cspecificstart
\begin{codepar}
PMIX_CHECK_PROCID(a, b)
\end{codepar}
\cspecificend

\begin{arglist}
\argin{a}{Pointer to a structure whose ID is to be compared (pointer to \refstruct{pmix_proc_t})}
\argin{b}{Pointer to a structure whose ID is to be compared (pointer to \refstruct{pmix_proc_t})}
\end{arglist}

Returns \code{true} if the two structures contain matching namespaces and:

\begin{itemize}
    \item the ranks are the same value
    \item one of the ranks is \refconst{PMIX_RANK_WILDCARD}
\end{itemize}

\littleheader{Check if a process identifier is valid}
\declaremacro{PMIX_PROCID_INVALID}

Check for invalid namespace or rank value

\versionMarker{4.1}
\cspecificstart
\begin{codepar}
PMIX_PROCID_INVALID(a)
\end{codepar}
\cspecificend

\begin{arglist}
\argin{a}{Pointer to a structure whose ID is to be checked (pointer to \refstruct{pmix_proc_t})}
\end{arglist}

Returns \code{true} if the process identifier contains either an empty (i.e., invalid) \refarg{nspace} field or a \refarg{rank} field of \refconst{PMIX_RANK_INVALID}

\littleheader{Load a procID structure}
\declaremacro{PMIX_LOAD_PROCID}

Load values into a \refstruct{pmix_proc_t}.

\versionMarker{4.0}
\cspecificstart
\begin{codepar}
PMIX_LOAD_PROCID(m, n, r)
\end{codepar}
\cspecificend

\begin{arglist}
\argin{m}{Pointer to the structure to be loaded (pointer to \refstruct{pmix_proc_t})}
\argin{n}{Namespace to be loaded (\refstruct{pmix_nspace_t})}
\argin{r}{Rank to be assigned (\refstruct{pmix_rank_t})}
\end{arglist}

\littleheader{Transfer a procID structure}
\declaremacro{PMIX_XFER_PROCID}

Transfer contents of one \refstruct{pmix_proc_t} value to another \refstruct{pmix_proc_t}.

\versionMarker{4.1}
\cspecificstart
\begin{codepar}
PMIX_XFER_PROCID(m, n)
\end{codepar}
\cspecificend

\begin{arglist}
\argin{m}{Pointer to the target structure (pointer to \refstruct{pmix_proc_t})}
\argin{n}{Pointer to the source structure (pointer to \refstruct{pmix_proc_t})}
\end{arglist}

\littleheader{Construct a multi-cluster namespace}
\declaremacro{PMIX_MULTICLUSTER_NSPACE_CONSTRUCT}

Construct a multi-cluster identifier containing a cluster ID and a namespace.

\versionMarker{4.0}
\cspecificstart
\begin{codepar}
PMIX_MULTICLUSTER_NSPACE_CONSTRUCT(m, n, r)
\end{codepar}
\cspecificend

\begin{arglist}
\argin{m}{\refstruct{pmix_nspace_t} structure that will contain the multi-cluster identifier (\refstruct{pmix_nspace_t})}
\argin{n}{Cluster identifier (\code{char*})}
\argin{n}{Namespace to be loaded (\refstruct{pmix_nspace_t})}
\end{arglist}

Combined length of the cluster identifier and namespace must be less than \refconst{PMIX_MAX_NSLEN}-2.

\littleheader{Parse a multi-cluster namespace}
\declaremacro{PMIX_MULTICLUSTER_NSPACE_PARSE}

Parse a multi-cluster identifier into its cluster ID and namespace parts.

\versionMarker{4.0}
\cspecificstart
\begin{codepar}
PMIX_MULTICLUSTER_NSPACE_PARSE(m, n, r)
\end{codepar}
\cspecificend

\begin{arglist}
\argin{m}{\refstruct{pmix_nspace_t} structure containing the multi-cluster identifier (pointer to \refstruct{pmix_nspace_t})}
\argin{n}{Location where the cluster ID is to be stored (\refstruct{pmix_nspace_t})}
\argin{n}{Location where the namespace is to be stored (\refstruct{pmix_nspace_t})}
\end{arglist}


%%%%%%%%%%%%%%%%%%%%%%%%%%%%%%%%%%%%%%%%%%%%%%%%%
\subsection{Process State Structure}
\label{api:struct:processstate}
\declarestruct{pmix_proc_state_t}

\versionMarker{2.0}
The \refstruct{pmix_proc_state_t} structure is a \code{uint8_t} type for process state values. The following constants can be used to set a variable of the type \refstruct{pmix_proc_state_t}.

\adviceuserstart
The fine-grained nature of the following constants may exceed the ability of an \ac{RM} to provide updated process state values during the process lifetime. This is particularly true of states for short-lived processes.
\adviceuserend

\begin{constantdesc}
%
\declareconstitem{PMIX_PROC_STATE_UNDEF}
Undefined process state.
%
\declareconstitem{PMIX_PROC_STATE_PREPPED}
Process is ready to be launched.
%
\declareconstitem{PMIX_PROC_STATE_LAUNCH_UNDERWAY}
Process launch is underway.
%
\declareconstitem{PMIX_PROC_STATE_RESTART}
Process is ready for restart.
%
\declareconstitem{PMIX_PROC_STATE_TERMINATE}
Process is marked for termination.
%
\declareconstitem{PMIX_PROC_STATE_RUNNING}
Process has been locally \code{fork}'ed by the \ac{RM}.
%
\declareconstitem{PMIX_PROC_STATE_CONNECTED}
Process has connected to PMIx server.
%
\declareconstitem{PMIX_PROC_STATE_UNTERMINATED}
Define a ``boundary'' between the terminated states and \refconst{PMIX_PROC_STATE_CONNECTED} so users can easily and quickly determine if a process is still running or not.
Any value less than this constant means that the process has not terminated.
%
\declareconstitem{PMIX_PROC_STATE_TERMINATED}
Process has terminated and is no longer running.
%
\declareconstitem{PMIX_PROC_STATE_ERROR}
Define a boundary so users can easily and quickly determine if a process abnormally terminated.
Any value above this constant means that the process has terminated abnormally.
%
\declareconstitem{PMIX_PROC_STATE_KILLED_BY_CMD}
Process was killed by a command.
%
\declareconstitem{PMIX_PROC_STATE_ABORTED}
Process was aborted by a call to \refapi{PMIx_Abort}.
%
\declareconstitem{PMIX_PROC_STATE_FAILED_TO_START}
Process failed to start.
%
\declareconstitem{PMIX_PROC_STATE_ABORTED_BY_SIG}
Process aborted by a signal.
%
\declareconstitem{PMIX_PROC_STATE_TERM_WO_SYNC}
Process exited without calling \refapi{PMIx_Finalize}.
%
\declareconstitem{PMIX_PROC_STATE_COMM_FAILED}
Process communication has failed.
%
\declareconstitemNEW{PMIX_PROC_STATE_SENSOR_BOUND_EXCEEDED}
Process exceeded a specified sensor limit.
%
\declareconstitem{PMIX_PROC_STATE_CALLED_ABORT}
Process called \refapi{PMIx_Abort}.
%
\declareconstitemNEW{PMIX_PROC_STATE_HEARTBEAT_FAILED}
Frocess failed to send heartbeat within specified time limit.
%
\declareconstitem{PMIX_PROC_STATE_MIGRATING}
Process failed and is waiting for resources before restarting.
%
\declareconstitem{PMIX_PROC_STATE_CANNOT_RESTART}
Process failed and cannot be restarted.
%
\declareconstitem{PMIX_PROC_STATE_TERM_NON_ZERO}
Process exited with a non-zero status.
%
\declareconstitem{PMIX_PROC_STATE_FAILED_TO_LAUNCH}
Unable to launch process.
%
\end{constantdesc}


%%%%%%%%%%%%%%%%%%%%%%%%%%%%%%%%%%%%%%%%%%%%%%%%%
\subsection{Process Information Structure}
\declarestruct{pmix_proc_info_t}

The \refstruct{pmix_proc_info_t} structure defines a set of information about a specific process including it's name, location, and state.

\versionMarker{2.0}
\cspecificstart
\begin{codepar}
typedef struct pmix_proc_info \{
    /** Process structure */
    pmix_proc_t proc;
    /** Hostname where process resides */
    char *hostname;
    /** Name of the executable */
    char *executable_name;
    /** Process ID on the host */
    pid_t pid;
    /** Exit code of the process. Default: 0 */
    int exit_code;
    /** Current state of the process */
    pmix_proc_state_t state;
\} pmix_proc_info_t;
\end{codepar}
\cspecificend


%%%%%%%%%%%%%%%%%%%%%%%%%%%%%%%%%%%%%%%%%%%%%%%%%
\subsubsection{Process information structure support macros}

The following macros are provided to support the \refstruct{pmix_proc_info_t} structure.

%%%%
\littleheader{Initialize the process information structure}
\declaremacro{PMIX_PROC_INFO_CONSTRUCT}

Initialize the \refstruct{pmix_proc_info_t} fields.

\versionMarker{2.0}
\cspecificstart
\begin{codepar}
PMIX_PROC_INFO_CONSTRUCT(m)
\end{codepar}
\cspecificend

\begin{arglist}
\argin{m}{Pointer to the structure to be initialized (pointer to \refstruct{pmix_proc_info_t})}
\end{arglist}

%%%%
\littleheader{Destruct the process information structure}
\declaremacro{PMIX_PROC_INFO_DESTRUCT}

Destruct the \refstruct{pmix_proc_info_t} fields.

\versionMarker{2.0}
\cspecificstart
\begin{codepar}
PMIX_PROC_INFO_DESTRUCT(m)
\end{codepar}
\cspecificend

\begin{arglist}
\argin{m}{Pointer to the structure to be destructed (pointer to \refstruct{pmix_proc_info_t})}
\end{arglist}

%%%%
\littleheader{Create a process information array}
\declaremacro{PMIX_PROC_INFO_CREATE}

Allocate and initialize a \refstruct{pmix_proc_info_t} array.

\versionMarker{2.0}
\cspecificstart
\begin{codepar}
PMIX_PROC_INFO_CREATE(m, n)
\end{codepar}
\cspecificend

\begin{arglist}
\arginout{m}{Address where the pointer to the array of \refstruct{pmix_proc_info_t} structures shall be stored (handle)}
\argin{n}{Number of structures to be allocated (\code{size_t})}
\end{arglist}

%%%%
\littleheader{Free a process information structure}
\declaremacro{PMIX_PROC_INFO_RELEASE}

Release a \refstruct{pmix_proc_info_t} structure.

\versionMarker{2.0}
\cspecificstart
\begin{codepar}
PMIX_PROC_INFO_RELEASE(m)
\end{codepar}
\cspecificend

\begin{arglist}
\argin{m}{Pointer to a \refstruct{pmix_proc_info_t} structure (handle)}
\end{arglist}

%%%%
\littleheader{Free a process information array}
\declaremacro{PMIX_PROC_INFO_FREE}

Release an array of \refstruct{pmix_proc_info_t} structures.

\versionMarker{2.0}
\cspecificstart
\begin{codepar}
PMIX_PROC_INFO_FREE(m, n)
\end{codepar}
\cspecificend

\begin{arglist}
\argin{m}{Pointer to the array of \refstruct{pmix_proc_info_t} structures (handle)}
\argin{n}{Number of structures in the array (\code{size_t})}
\end{arglist}


%%%%%%%%%%%%%%%%%%%%%%%%%%%%%%%%%%%%%%%%%%%%%%%%%
\subsection{Job State Structure}
\label{api:struct:jobstate}
\declarestruct{pmix_job_state_t}

\versionMarker{4.0}
The \refstruct{pmix_job_state_t} structure is a \code{uint8_t} type for job state values. The following constants can be used to set a variable of the type \refstruct{pmix_job_state_t}.

\adviceuserstart
The fine-grained nature of the following constants may exceed the ability of an \ac{RM} to provide updated job state values during the job lifetime. This is particularly true for short-lived jobs.
\adviceuserend

\begin{constantdesc}
%
\declareconstitemNEW{PMIX_JOB_STATE_UNDEF}
Undefined job state.
%
\declareconstitemNEW{PMIX_JOB_STATE_AWAITING_ALLOC}
Job is waiting for resources to be allocated to it.
%
\declareconstitemNEW{PMIX_JOB_STATE_LAUNCH_UNDERWAY}
Job launch is underway.
%
\declareconstitemNEW{PMIX_JOB_STATE_RUNNING}
All processes in the job have been spawned and are executing.
%
\declareconstitemNEW{PMIX_JOB_STATE_SUSPENDED}
All processes in the job have been suspended.
%
\declareconstitemNEW{PMIX_JOB_STATE_CONNECTED}
All processes in the job have connected to their \ac{PMIx} server.
%
\declareconstitemNEW{PMIX_JOB_STATE_UNTERMINATED}
Define a ``boundary'' between the terminated states and \refconst{PMIX_JOB_STATE_TERMINATED} so users can easily and quickly determine if a job is still running or not.
Any value less than this constant means that the job has not terminated.
%
\declareconstitemNEW{PMIX_JOB_STATE_TERMINATED}
All processes in the job have terminated and are no longer running - typically will be accompanied by the job exit status in response to a query.
%
\declareconstitemNEW{PMIX_JOB_STATE_TERMINATED_WITH_ERROR}
Define a boundary so users can easily and quickly determine if a job abnormally terminated - typically will be accompanied by a job-related error code in response to a query
Any value above this constant means that the job terminated abnormally.
%
\end{constantdesc}


%%%%%%%%%%%%%%%%%%%%%%%%%%%%%%%%%%%%%%%%%%%%%%%%%
\subsection{Value Structure}
\declarestruct{pmix_value_t}

The \refstruct{pmix_value_t} structure is used to represent the value passed to \refapi{PMIx_Put} and retrieved by \refapi{PMIx_Get}, as well as many of the other \ac{PMIx} functions.

A collection of values may be specified under a single key by passing a \refstruct{pmix_value_t} containing an array of type \refstruct{pmix_data_array_t}, with each array element containing its own object. All members shown below were introduced in version 1 of the standard unless otherwise marked.

\versionMarker{1.0}
\cspecificstart
\begin{codepar}
typedef struct pmix_value \{
    pmix_data_type_t type;
    union \{
        bool flag;
        uint8_t byte;
        char *string;
        size_t size;
        pid_t pid;
        int integer;
        int8_t int8;
        int16_t int16;
        int32_t int32;
        int64_t int64;
        unsigned int uint;
        uint8_t uint8;
        uint16_t uint16;
        uint32_t uint32;
        uint64_t uint64;
        float fval;
        double dval;
        struct timeval tv;
        time_t time;                    // version 2.0
        pmix_status_t status;           // version 2.0
        pmix_rank_t rank;               // version 2.0
        pmix_proc_t *proc;              // version 2.0
        pmix_byte_object_t bo;
        pmix_persistence_t persist;     // version 2.0
        pmix_scope_t scope;             // version 2.0
        pmix_data_range_t range;        // version 2.0
        pmix_proc_state_t state;        // version 2.0
        pmix_proc_info_t *pinfo;        // version 2.0
        pmix_data_array_t *darray;      // version 2.0
        void *ptr;                      // version 2.0
        pmix_alloc_directive_t adir;    // version 2.0
    \} data;
\} pmix_value_t;
\end{codepar}
\cspecificend

%%%%%%%%%%%%%%%%%%%%%%%%%%%%%%%%%%%%%%%%%%%%%%%%%
\subsubsection{Value structure support macros}
The following macros are provided to support the \refstruct{pmix_value_t} structure.

\littleheader{Initialize the value structure}
\declaremacro{PMIX_VALUE_CONSTRUCT}

Initialize the \refstruct{pmix_value_t} fields.

\versionMarker{1.0}
\cspecificstart
\begin{codepar}
PMIX_VALUE_CONSTRUCT(m)
\end{codepar}
\cspecificend

\begin{arglist}
\argin{m}{Pointer to the structure to be initialized (pointer to \refstruct{pmix_value_t})}
\end{arglist}

\littleheader{Destruct the value structure}
\declaremacro{PMIX_VALUE_DESTRUCT}

Destruct the \refstruct{pmix_value_t} fields.

\versionMarker{1.0}
\cspecificstart
\begin{codepar}
PMIX_VALUE_DESTRUCT(m)
\end{codepar}
\cspecificend

\begin{arglist}
\argin{m}{Pointer to the structure to be destructed (pointer to \refstruct{pmix_value_t})}
\end{arglist}

%%%%%%%%%%%
\littleheader{Create a value array}
\declaremacro{PMIX_VALUE_CREATE}

Allocate and initialize an array of \refstruct{pmix_value_t} structures.

\versionMarker{1.0}
\cspecificstart
\begin{codepar}
PMIX_VALUE_CREATE(m, n)
\end{codepar}
\cspecificend

\begin{arglist}
\arginout{m}{Address where the pointer to the array of \refstruct{pmix_value_t} structures shall be stored (handle)}
\argin{n}{Number of structures to be allocated (\code{size_t})}
\end{arglist}


%%%%%%%%%%%
\littleheader{Free a value structure}
\declaremacro{PMIX_VALUE_RELEASE}

Release a \refstruct{pmix_value_t} structure.

\versionMarker{4.0}
\cspecificstart
\begin{codepar}
PMIX_VALUE_RELEASE(m)
\end{codepar}
\cspecificend

\begin{arglist}
\argin{m}{Pointer to a \refstruct{pmix_value_t} structure (handle)}
\end{arglist}

%%%%%%%%%%%
\littleheader{Free a value array}
\declaremacro{PMIX_VALUE_FREE}

Release an array of \refstruct{pmix_value_t} structures.

\versionMarker{1.0}
\cspecificstart
\begin{codepar}
PMIX_VALUE_FREE(m, n)
\end{codepar}
\cspecificend

\begin{arglist}
\argin{m}{Pointer to the array of \refstruct{pmix_value_t} structures (handle)}
\argin{n}{Number of structures in the array (\code{size_t})}
\end{arglist}

%%%%%%%%%%%
\littleheader{Load a value structure}
\declaremacro{PMIX_VALUE_LOAD}

Load data into a \refstruct{pmix_value_t} structure.

\versionMarker{2.0}
\cspecificstart
\begin{codepar}
PMIX_VALUE_LOAD(v, d, t);
\end{codepar}
\cspecificend

\begin{arglist}
\argin{v}{The \refstruct{pmix_value_t} into which the data is to be loaded (pointer to \refstruct{pmix_value_t})}
\argin{d}{Pointer to the data value to be loaded (handle)}
\argin{t}{Type of the provided data value (\refstruct{pmix_data_type_t})}
\end{arglist}

This macro simplifies the loading of data into a \refstruct{pmix_value_t} by correctly assigning values to the structure's fields.

\adviceuserstart
The data will be copied into the \refstruct{pmix_value_t} - thus, any data stored in the source value can be modified or free'd without affecting the copied data once the macro has completed.
\adviceuserend

%%%%%%%%%%%
\littleheader{Unload a value structure}
\declaremacro{PMIX_VALUE_UNLOAD}

Unload data from a \refstruct{pmix_value_t} structure.

\versionMarker{2.2}
\cspecificstart
\begin{codepar}
PMIX_VALUE_UNLOAD(r, v, d, t);
\end{codepar}
\cspecificend

\begin{arglist}
\argout{r}{Status code indicating result of the operation {\refstruct{pmix_status_t}}}
\argin{v}{The \refstruct{pmix_value_t} from which the data is to be unloaded (pointer to \refstruct{pmix_value_t})}
\arginout{d}{Pointer to the location where the data value is to be returned (handle)}
\arginout{t}{Pointer to return the data type of the unloaded value (handle)}
\end{arglist}

This macro simplifies the unloading of data from a \refstruct{pmix_value_t}.

\adviceuserstart
Memory will be allocated and the data will be in the \refstruct{pmix_value_t} returned - the source \refstruct{pmix_value_t} will not be altered.
\adviceuserend

%%%%%%%%%%%
\littleheader{Transfer data between value structures}
\declaremacro{PMIX_VALUE_XFER}

Transfer the data value between two \refstruct{pmix_value_t} structures.

\versionMarker{2.0}
\cspecificstart
\begin{codepar}
PMIX_VALUE_XFER(r, d, s);
\end{codepar}
\cspecificend

\begin{arglist}
\argout{r}{Status code indicating success or failure of the transfer (\refstruct{pmix_status_t})}
\argin{d}{Pointer to the \refstruct{pmix_value_t} destination (handle)}
\argin{s}{Pointer to the \refstruct{pmix_value_t} source (handle)}
\end{arglist}

This macro simplifies the transfer of data between two \refstruct{pmix_value_t} structures, ensuring that all fields are properly copied.

\adviceuserstart
The data will be copied into the destination \refstruct{pmix_value_t} - thus, any data stored in the source value can be modified or free'd without affecting the copied data once the macro has completed.
\adviceuserend

%%%%%%%%%%%
\littleheader{Retrieve a numerical value from a value struct}
\declaremacro{PMIX_VALUE_GET_NUMBER}

Retrieve a numerical value from a \refstruct{pmix_value_t} structure.

\versionMarker{3.0}
\cspecificstart
\begin{codepar}
PMIX_VALUE_GET_NUMBER(s, m, n, t)
\end{codepar}
\cspecificend

\begin{arglist}
\argout{s}{Status code for the request (\refstruct{pmix_status_t})}
\argin{m}{Pointer to the\refstruct{pmix_value_t} structure (handle)}
\argout{n}{Variable to be set to the value (match expected type)}
\argin{t}{Type of number expected in \refarg{m} (\refstruct{pmix_data_type_t})}
\end{arglist}

Sets the provided variable equal to the numerical value contained in the given \refstruct{pmix_value_t}, returning success if the data type of the value matches the expected type and \refconst{PMIX_ERR_BAD_PARAM} if it doesn't

%%%%%%%%%%%%%%%%%%%%%%%%%%%%%%%%%%%%%%%%%%%%%%%%%
\subsection{Info Structure}
\label{chap:struct:info}
\declarestruct{pmix_info_t}

The \refstruct{pmix_info_t} structure defines a key/value pair with associated directive. All fields were defined in version 1.0 unless otherwise marked.

\versionMarker{1.0}
\cspecificstart
\begin{codepar}
typedef struct pmix_info_t \{
    pmix_key_t key;
    pmix_info_directives_t flags;    // version 2.0
    pmix_value_t value;
\} pmix_info_t;
\end{codepar}
\cspecificend

%%%%%%%%%%%
\subsubsection{Info structure support macros}
The following macros are provided to support the \refstruct{pmix_info_t} structure.

\littleheader{Initialize the info structure}
\declaremacro{PMIX_INFO_CONSTRUCT}

Initialize the \refstruct{pmix_info_t} fields.

\versionMarker{1.0}
\cspecificstart
\begin{codepar}
PMIX_INFO_CONSTRUCT(m)
\end{codepar}
\cspecificend

\begin{arglist}
\argin{m}{Pointer to the structure to be initialized (pointer to \refstruct{pmix_info_t})}
\end{arglist}

\littleheader{Destruct the info structure}
\declaremacro{PMIX_INFO_DESTRUCT}

Destruct the \refstruct{pmix_info_t} fields.

\versionMarker{1.0}
\cspecificstart
\begin{codepar}
PMIX_INFO_DESTRUCT(m)
\end{codepar}
\cspecificend

\begin{arglist}
\argin{m}{Pointer to the structure to be destructed (pointer to \refstruct{pmix_info_t})}
\end{arglist}

%%%%%%%%%%%
\littleheader{Create an info array}
\declaremacro{PMIX_INFO_CREATE}

Allocate and initialize an array of info structures.

\versionMarker{1.0}
\cspecificstart
\begin{codepar}
PMIX_INFO_CREATE(m, n)
\end{codepar}
\cspecificend

\begin{arglist}
\arginout{m}{Address where the pointer to the array of \refstruct{pmix_info_t} structures shall be stored (handle)}
\argin{n}{Number of structures to be allocated (\code{size_t})}
\end{arglist}


%%%%%%%%%%%
\littleheader{Free an info array}
\declaremacro{PMIX_INFO_FREE}

Release an array of \refstruct{pmix_info_t} structures.

\versionMarker{1.0}
\cspecificstart
\begin{codepar}
PMIX_INFO_FREE(m, n)
\end{codepar}
\cspecificend

\begin{arglist}
\argin{m}{Pointer to the array of \refstruct{pmix_info_t} structures (handle)}
\argin{n}{Number of structures in the array (\code{size_t})}
\end{arglist}

%%%%%%%%%%%
\littleheader{Load key and value data into a info struct}
\declaremacro{PMIX_INFO_LOAD}

\versionMarker{1.0}
\cspecificstart
\begin{codepar}
PMIX_INFO_LOAD(v, k, d, t);
\end{codepar}
\cspecificend

\begin{arglist}
\argin{v}{Pointer to the \refstruct{pmix_info_t} into which the key and data are to be loaded (pointer to \refstruct{pmix_info_t})}
\argin{k}{String key to be loaded - must be less than or equal to \refconst{PMIX_MAX_KEYLEN} in length (handle)}
\argin{d}{Pointer to the data value to be loaded (handle)}
\argin{t}{Type of the provided data value (\refstruct{pmix_data_type_t})}
\end{arglist}

This macro simplifies the loading of key and data into a \refstruct{pmix_info_t} by correctly assigning values to the structure's fields.

\adviceuserstart
Both key and data will be copied into the \refstruct{pmix_info_t} - thus, the key and any data stored in the source value can be modified or free'd without affecting the copied data once the macro has completed.
\adviceuserend

%%%%%%%%%%%
\littleheader{Copy data between info structures}
\declaremacro{PMIX_INFO_XFER}

Copy all data (including key, value, and directives) between two \refstruct{pmix_info_t} structures.

\versionMarker{2.0}
\cspecificstart
\begin{codepar}
PMIX_INFO_XFER(d, s);
\end{codepar}
\cspecificend

\begin{arglist}
\argin{d}{Pointer to the destination \refstruct{pmix_info_t} (pointer to \refstruct{pmix_info_t})}
\argin{s}{Pointer to the source \refstruct{pmix_info_t} (pointer to \refstruct{pmix_info_t})}
\end{arglist}

This macro simplifies the transfer of data between two\refstruct{pmix_info_t} structures.

\adviceuserstart
All data (including key, value, and directives) will be copied into the destination \refstruct{pmix_info_t} - thus, the source \refstruct{pmix_info_t} may be free'd without affecting the copied data once the macro has completed.
\adviceuserend


%%%%%%%%%%%
\littleheader{Test a boolean info struct}
\declaremacro{PMIX_INFO_TRUE}

A special macro for checking if a boolean \refstruct{pmix_info_t} is \code{true}.

\versionMarker{2.0}
\cspecificstart
\begin{codepar}
PMIX_INFO_TRUE(m)
\end{codepar}
\cspecificend

\begin{arglist}
\argin{m}{Pointer to a \refstruct{pmix_info_t} structure (handle)}
\end{arglist}

A \refstruct{pmix_info_t} structure is considered to be of type \refconst{PMIX_BOOL} and value \code{true} if:

\begin{compactitemize}
    \item the structure reports a type of \refconst{PMIX_UNDEF}, or
    \item the structure reports a type of \refconst{PMIX_BOOL} and the data flag is \code{true}
\end{compactitemize}

%%%%%%%%%%%
\subsubsection{Info structure list macros}
Constructing an array of \refstruct{pmix_info_t} is a fairly common operation. The following macros are provided to simplify this construction.

%%%%%%%%%%%
\littleheader{Start a list of \refstruct{pmix_info_t} structures}
\declaremacro{PMIX_INFO_LIST_START}

Initialize a list of \refstruct{pmix_info_t} structures. The actual list is opaque to the caller and is implementation-dependent.

\versionMarker{4.0}
\cspecificstart
\begin{codepar}
PMIX_INFO_LIST_START(m)
\end{codepar}
\cspecificend

\begin{arglist}
\argin{m}{A \code{void*} pointer (handle)}
\end{arglist}

Note that the pointer will be initialized to an opaque structure whose elements are implementation-dependent. The caller must not modify or dereference the object.

%%%%%%%%%%%
\littleheader{Add a \refstruct{pmix_info_t} structure to a list}
\declaremacro{PMIX_INFO_LIST_ADD}

Add a \refstruct{pmix_info_t} structure containing the specified value to the provided list.

\versionMarker{4.0}
\cspecificstart
\begin{codepar}
PMIX_INFO_LIST_ADD(rc, m, k, d, t)
\end{codepar}
\cspecificend

\begin{arglist}
\arginout{rc}{Return status for the operation (\refstruct{pmix_status_t})}
\argin{m}{A \code{void*} pointer initialized via \refmacro{PMIX_INFO_LIST_START} (handle)}
\argin{k}{String key to be loaded - must be less than or equal to \refconst{PMIX_MAX_KEYLEN} in length (handle)}
\argin{d}{Pointer to the data value to be loaded (handle)}
\argin{t}{Type of the provided data value (\refstruct{pmix_data_type_t})}
\end{arglist}

\adviceuserstart
Both key and data will be copied into the \refstruct{pmix_info_t} on the list - thus, the key and any data stored in the source value can be modified or free'd without affecting the copied data once the macro has completed.
\adviceuserend

%%%%%%%%%%%
\littleheader{Transfer a \refstruct{pmix_info_t} structure to a list}
\declaremacro{PMIX_INFO_LIST_XFER}

Transfer the information in a \refstruct{pmix_info_t} structure to the provided list.

\versionMarker{4.0}
\cspecificstart
\begin{codepar}
PMIX_INFO_LIST_XFER(rc, m, s)
\end{codepar}
\cspecificend

\begin{arglist}
\arginout{rc}{Return status for the operation (\refstruct{pmix_status_t})}
\argin{m}{A \code{void*} pointer initialized via \refmacro{PMIX_INFO_LIST_START} (handle)}
\argin{s}{Pointer to the source \refstruct{pmix_info_t} (pointer to \refstruct{pmix_info_t})}
\end{arglist}

\adviceuserstart
All data (including key, value, and directives) will be copied into the destination \refstruct{pmix_info_t} on the list - thus, the source \refstruct{pmix_info_t} may be free'd without affecting the copied data once the macro has completed.
\adviceuserend

%%%%%%%%%%%
\littleheader{Convert a \refstruct{pmix_info_t} list to an array}
\declaremacro{PMIX_INFO_LIST_CONVERT}

Transfer the information in the provided \refstruct{pmix_info_t} list to a \refstruct{pmix_data_array_t} array

\versionMarker{4.0}
\cspecificstart
\begin{codepar}
PMIX_INFO_LIST_CONVERT(rc, m, d)
\end{codepar}
\cspecificend

\begin{arglist}
\arginout{rc}{Return status for the operation (\refstruct{pmix_status_t})}
\argin{m}{A \code{void*} pointer initialized via \refmacro{PMIX_INFO_LIST_START} (handle)}
\argin{d}{Pointer to an instantiated \refstruct{pmix_data_array_t} structure where the \refstruct{pmix_info_t} array is to be stored (pointer to \refstruct{pmix_data_array_t})}
\end{arglist}

%%%%%%%%%%%
\littleheader{Release a \refstruct{pmix_info_t} list}
\declaremacro{PMIX_INFO_LIST_RELEASE}

Release the provided \refstruct{pmix_info_t} list

\versionMarker{4.0}
\cspecificstart
\begin{codepar}
PMIX_INFO_LIST_RELEASE(m)
\end{codepar}
\cspecificend

\begin{arglist}
\argin{m}{A \code{void*} pointer initialized via \refmacro{PMIX_INFO_LIST_START} (handle)}
\end{arglist}

Information contained in the \refstruct{pmix_info_t} on the list shall be released in addition to whatever backing storage the implementation may have allocated to support construction of the list.


%%%%%%%%%%%%%%%%%%%%%%%%%%%%%%%%%%%%%%%%%%%%%%%%%
\subsection{Info Type Directives}
\declarestruct{pmix_info_directives_t}
\label{api:struct:infodirs}

\versionMarker{2.0}
The \refstruct{pmix_info_directives_t} structure is a \code{uint32_t} type that defines the behavior of command directives via \refstruct{pmix_info_t} arrays.
By default, the values in the \refstruct{pmix_info_t} array passed to a PMIx are \emph{optional}.

\adviceuserstart
A PMIx implementation or PMIx-enabled \ac{RM} may ignore any \refstruct{pmix_info_t} value passed to a \ac{PMIx} \ac{API} that it does not support or does not recognize if it is not explicitly marked as \refconst{PMIX_INFO_REQD}.
This is because the values specified default to optional, meaning they can be ignored in such circumstances.
This may lead to unexpected behavior when porting between environments or \ac{PMIx} implementations if the user is relying on the behavior specified by the \refstruct{pmix_info_t} value.
Users relying on the behavior defined by the \refstruct{pmix_info_t} are advised to set the \refconst{PMIX_INFO_REQD} flag using the \refmacro{PMIX_INFO_REQUIRED} macro.
\adviceuserend

\adviceimplstart
The top 16-bits of the \refstruct{pmix_info_directives_t} are reserved for internal use by \ac{PMIx} library implementers - the \ac{PMIx} standard will \textit{not} specify their intent, leaving them for customized use by implementers. Implementers are advised to use the provided \refmacro{PMIX_INFO_IS_REQUIRED} macro for testing this flag, and must return \refconst{PMIX_ERR_NOT_SUPPORTED} as soon as possible to the caller if the required behavior is not supported.
\adviceimplend

The following constants were introduced in version 2.0 (unless otherwise marked) and can be used to set a variable of the type \refstruct{pmix_info_directives_t}.

\begin{constantdesc}
%
\declareconstitem{PMIX_INFO_REQD}
The behavior defined in the \refstruct{pmix_info_t} array is required, and not optional. This is a bit-mask value.
%
\declareconstitemNEW{PMIX_INFO_REQD_PROCESSED}
Mark that this required attribute has been processed. A required attribute can be handled at any level - the \ac{PMIx} client library might take care of it, or it may be resolved by the \ac{PMIx} server library, or it may pass up to the host environment for handling. If a level does not recognize or support the required attribute, it is required to pass it upwards to give the next level an opportunity to process it. Thus, the host environment (or the server library if the host does not support the given operation) must know if a lower level has handled the requirement so it can return a \refconst{PMIX_ERR_NOT_SUPPORTED} error status if the host itself cannot meet the request. Upon processing the request, the level must therefore mark the attribute with this directive to alert any subsequent levels that the requirement has been met.
%
\declareconstitem{PMIX_INFO_ARRAY_END}
Mark that this \refstruct{pmix_info_t} struct is at the end of an array created by the \refmacro{PMIX_INFO_CREATE} macro. This is a bit-mask value.
%
\declareconstitemNEW{PMIX_INFO_DIR_RESERVED}
A bit-mask identifying the bits reserved for internal use by implementers - these currently are set as \code{0xffff0000}.
%
\end{constantdesc}

\advicermstart
Host environments are advised to use the provided \refmacro{PMIX_INFO_IS_REQUIRED} macro for testing this flag and must return \refconst{PMIX_ERR_NOT_SUPPORTED} as soon as possible to the caller if the required behavior is not supported.
\advicermend


\subsubsection{Info Directive support macros}

The following macros are provided to support the setting and testing of \refstruct{pmix_info_t} directives.

%%%%
\littleheader{Mark an info structure as required}
\declaremacro{PMIX_INFO_REQUIRED}

Set the \refconst{PMIX_INFO_REQD} flag in a \refstruct{pmix_info_t} structure.

\versionMarker{2.0}
\cspecificstart
\begin{codepar}
PMIX_INFO_REQUIRED(info);
\end{codepar}
\cspecificend

\begin{arglist}
\argin{info}{Pointer to the \refstruct{pmix_info_t} (pointer to \refstruct{pmix_info_t})}
\end{arglist}

This macro simplifies the setting of the \refconst{PMIX_INFO_REQD} flag in \refstruct{pmix_info_t} structures.

%%%%
\littleheader{Mark an info structure as optional}
\declaremacro{PMIX_INFO_OPTIONAL}

Unsets the \refconst{PMIX_INFO_REQD} flag in a \refstruct{pmix_info_t} structure.

\versionMarker{2.0}
\cspecificstart
\begin{codepar}
PMIX_INFO_OPTIONAL(info);
\end{codepar}
\cspecificend

\begin{arglist}
\argin{info}{Pointer to the \refstruct{pmix_info_t} (pointer to \refstruct{pmix_info_t})}
\end{arglist}

This macro simplifies marking a \refstruct{pmix_info_t} structure as \textit{optional}.

%%%%%%%%%%%
\littleheader{Test an info structure for \textit{required} directive}
\declaremacro{PMIX_INFO_IS_REQUIRED}

Test the \refconst{PMIX_INFO_REQD} flag in a \refstruct{pmix_info_t} structure, returning \code{true} if the flag is set.

\versionMarker{2.0}
\cspecificstart
\begin{codepar}
PMIX_INFO_IS_REQUIRED(info);
\end{codepar}
\cspecificend

\begin{arglist}
\argin{info}{Pointer to the \refstruct{pmix_info_t} (pointer to \refstruct{pmix_info_t})}
\end{arglist}

This macro simplifies the testing of the required flag in \refstruct{pmix_info_t} structures.

%%%%%%%%%%%
\littleheader{Test an info structure for \textit{optional} directive}
\declaremacro{PMIX_INFO_IS_OPTIONAL}

Test a \refstruct{pmix_info_t} structure, returning \code{true} if the structure is \textit{optional}.

\versionMarker{2.0}
\cspecificstart
\begin{codepar}
PMIX_INFO_IS_OPTIONAL(info);
\end{codepar}
\cspecificend

\begin{arglist}
\argin{info}{Pointer to the \refstruct{pmix_info_t} (pointer to \refstruct{pmix_info_t})}
\end{arglist}

Test the \refconst{PMIX_INFO_REQD} flag in a \refstruct{pmix_info_t} structure, returning \code{true} if the flag is \textit{not} set.

%%%%%%%%%%%
\littleheader{Mark a required attribute as processed}
\declaremacro{PMIX_INFO_PROCESSED}

Mark that a required \refstruct{pmix_info_t} structure has been processed.

\versionMarker{4.0}
\cspecificstart
\begin{codepar}
PMIX_INFO_PROCESSED(info);
\end{codepar}
\cspecificend

\begin{arglist}
\argin{info}{Pointer to the \refstruct{pmix_info_t} (pointer to \refstruct{pmix_info_t})}
\end{arglist}

Set the \refconst{PMIX_INFO_REQD_PROCESSED} flag in a \refstruct{pmix_info_t} structure indicating that is has been processed.

%%%%%%%%%%%
\littleheader{Test if a required attribute has been processed}
\declaremacro{PMIX_INFO_WAS_PROCESSED}

Test that a required \refstruct{pmix_info_t} structure has been processed.

\versionMarker{4.0}
\cspecificstart
\begin{codepar}
PMIX_INFO_WAS_PROCESSED(info);
\end{codepar}
\cspecificend

\begin{arglist}
\argin{info}{Pointer to the \refstruct{pmix_info_t} (pointer to \refstruct{pmix_info_t})}
\end{arglist}

Test the \refconst{PMIX_INFO_REQD_PROCESSED} flag in a \refstruct{pmix_info_t} structure.

%%%%%%%%%%%
\littleheader{Test an info structure for \textit{end of array} directive}
\declaremacro{PMIX_INFO_IS_END}

Test a \refstruct{pmix_info_t} structure, returning \code{true} if the structure is at the end of an array created by the \refmacro{PMIX_INFO_CREATE} macro.

\versionMarker{2.2}
\cspecificstart
\begin{codepar}
PMIX_INFO_IS_END(info);
\end{codepar}
\cspecificend

\begin{arglist}
\argin{info}{Pointer to the \refstruct{pmix_info_t} (pointer to \refstruct{pmix_info_t})}
\end{arglist}

This macro simplifies the testing of the end-of-array flag in \refstruct{pmix_info_t} structures.

%%%%%%%%%%%%%%%%%%%%%%%%%%%%%%%%%%%%%%%%%%%%%%%%%
\subsection{Environmental Variable Structure}
\declarestruct{pmix_envar_t}

\versionMarker{3.0}
Define a structure for specifying environment variable modifications.
Standard environment variables (e.g., \code{PATH}, \code{LD_LIBRARY_PATH}, and \code{LD_PRELOAD})
take multiple arguments separated by delimiters. Unfortunately, the delimiters
depend upon the variable itself - some use semi-colons, some colons, etc. Thus,
the operation requires not only the name of the variable to be modified and
the value to be inserted, but also the separator to be used when composing
the aggregate value.

\cspecificstart
\begin{codepar}
typedef struct \{
    char *envar;
    char *value;
    char separator;
\} pmix_envar_t;
\end{codepar}
\cspecificend

%%%%%%%%%%%%%%%%%%%%%%%%%%%%%%%%%%%%%%%%%%%%%%%%%
\subsubsection{Environmental variable support macros}

The following macros are provided to support the \refstruct{pmix_envar_t} structure.

\littleheader{Initialize the envar structure}
\declaremacro{PMIX_ENVAR_CONSTRUCT}

Initialize the \refstruct{pmix_envar_t} fields.

\versionMarker{3.0}
\cspecificstart
\begin{codepar}
PMIX_ENVAR_CONSTRUCT(m)
\end{codepar}
\cspecificend

\begin{arglist}
\argin{m}{Pointer to the structure to be initialized (pointer to \refstruct{pmix_envar_t})}
\end{arglist}

\littleheader{Destruct the envar structure}
\declaremacro{PMIX_ENVAR_DESTRUCT}

Clear the \refstruct{pmix_envar_t} fields.

\versionMarker{3.0}
\cspecificstart
\begin{codepar}
PMIX_ENVAR_DESTRUCT(m)
\end{codepar}
\cspecificend

\begin{arglist}
\argin{m}{Pointer to the structure to be destructed (pointer to \refstruct{pmix_envar_t})}
\end{arglist}


\littleheader{Create an envar array}
\declaremacro{PMIX_ENVAR_CREATE}

Allocate and initialize an array of \refstruct{pmix_envar_t} structures.

\versionMarker{3.0}
\cspecificstart
\begin{codepar}
PMIX_ENVAR_CREATE(m, n)
\end{codepar}
\cspecificend

\begin{arglist}
\arginout{m}{Address where the pointer to the array of \refstruct{pmix_envar_t} structures shall be stored (handle)}
\argin{n}{Number of structures to be allocated (\code{size_t})}
\end{arglist}


\littleheader{Free an envar array}
\declaremacro{PMIX_ENVAR_FREE}

Release an array of \refstruct{pmix_envar_t} structures.

\versionMarker{3.0}
\cspecificstart
\begin{codepar}
PMIX_ENVAR_FREE(m, n)
\end{codepar}
\cspecificend

\begin{arglist}
\argin{m}{Pointer to the array of \refstruct{pmix_envar_t} structures (handle)}
\argin{n}{Number of structures in the array (\code{size_t})}
\end{arglist}

\littleheader{Load an envar structure}
\declaremacro{PMIX_ENVAR_LOAD}

Load values into a \refstruct{pmix_envar_t}.

\versionMarker{2.0}
\cspecificstart
\begin{codepar}
PMIX_ENVAR_LOAD(m, e, v, s)
\end{codepar}
\cspecificend

\begin{arglist}
\argin{m}{Pointer to the structure to be loaded (pointer to \refstruct{pmix_envar_t})}
\argin{e}{Environmental variable name (\code{char*})}
\argin{v}{Value of variable (\code{char*})}
\argin{v}{Separator character (\code{char})}
\end{arglist}


%%%%%%%%%%%%%%%%%%%%%%%%%%%%%%%%%%%%%%%%%%%%%%%%%
\subsection{Byte Object Type}
\declarestruct{pmix_byte_object_t}

The \refstruct{pmix_byte_object_t} structure describes a raw byte sequence.

\versionMarker{1.0}
\cspecificstart
\begin{codepar}
typedef struct pmix_byte_object \{
    char *bytes;
    size_t size;
\} pmix_byte_object_t;
\end{codepar}
\cspecificend

%%%%%%%%%%%%%%%%%%%%%%%%%%%%%%%%%%%%%%%%%%%%%%%%%
\subsubsection{Byte object support macros}
The following macros support the \refstruct{pmix_byte_object_t} structure.

\littleheader{Initialize the byte object structure}
\declaremacro{PMIX_BYTE_OBJECT_CONSTRUCT}

Initialize the \refstruct{pmix_byte_object_t} fields.

\versionMarker{2.0}
\cspecificstart
\begin{codepar}
PMIX_BYTE_OBJECT_CONSTRUCT(m)
\end{codepar}
\cspecificend

\begin{arglist}
\argin{m}{Pointer to the structure to be initialized (pointer to \refstruct{pmix_byte_object_t})}
\end{arglist}

\littleheader{Destruct the byte object structure}
\declaremacro{PMIX_BYTE_OBJECT_DESTRUCT}

Clear the \refstruct{pmix_byte_object_t} fields.

\versionMarker{2.0}
\cspecificstart
\begin{codepar}
PMIX_BYTE_OBJECT_DESTRUCT(m)
\end{codepar}
\cspecificend

\begin{arglist}
\argin{m}{Pointer to the structure to be destructed (pointer to \refstruct{pmix_byte_object_t})}
\end{arglist}

\littleheader{Create a byte object structure}
\declaremacro{PMIX_BYTE_OBJECT_CREATE}

Allocate and intitialize an array of \refstruct{pmix_byte_object_t} structures.

\versionMarker{2.0}
\cspecificstart
\begin{codepar}
PMIX_BYTE_OBJECT_CREATE(m, n)
\end{codepar}
\cspecificend

\begin{arglist}
\arginout{m}{Address where the pointer to the array of \refstruct{pmix_byte_object_t} structures shall be stored (handle)}
\argin{n}{Number of structures to be allocated (\code{size_t})}
\end{arglist}

\littleheader{Free a byte object array}
\declaremacro{PMIX_BYTE_OBJECT_FREE}

Release an array of \refstruct{pmix_byte_object_t} structures.

\versionMarker{2.0}
\cspecificstart
\begin{codepar}
PMIX_BYTE_OBJECT_FREE(m, n)
\end{codepar}
\cspecificend

\begin{arglist}
\argin{m}{Pointer to the array of \refstruct{pmix_byte_object_t} structures (handle)}
\argin{n}{Number of structures in the array (\code{size_t})}
\end{arglist}

\littleheader{Load a byte object structure}
\declaremacro{PMIX_BYTE_OBJECT_LOAD}

Load values into a \refstruct{pmix_byte_object_t}.

\versionMarker{2.0}
\cspecificstart
\begin{codepar}
PMIX_BYTE_OBJECT_LOAD(b, d, s)
\end{codepar}
\cspecificend

\begin{arglist}
\argin{b}{Pointer to the structure to be loaded (pointer to \refstruct{pmix_byte_object_t})}
\argin{d}{Pointer to the data to be loaded (\code{char*})}
\argin{s}{Number of bytes in the data array (\code{size_t})}
\end{arglist}


%%%%%%%%%%%%%%%%%%%%%%%%%%%%%%%%%%%%%%%%%%%%%%%%%
\subsection{Data Array Structure}
\declarestruct{pmix_data_array_t}

The \refstruct{pmix_data_array_t} structure defines an array data structure.

\versionMarker{2.0}
\cspecificstart
\begin{codepar}
typedef struct pmix_data_array \{
    pmix_data_type_t type;
    size_t size;
    void *array;
\} pmix_data_array_t;
\end{codepar}
\cspecificend

%%%%%%%%%%%%%%%%%%%%%%%%%%%%%%%%%%%%%%%%%%%%%%%%%
\subsubsection{Data array support macros}
The following macros support the \refstruct{pmix_data_array_t} structure.

\littleheader{Initialize a data array structure}
\declaremacro{PMIX_DATA_ARRAY_CONSTRUCT}

Initialize the \refstruct{pmix_data_array_t} fields, allocating memory for the array of the indicated type.

\versionMarker{2.2}
\cspecificstart
\begin{codepar}
PMIX_DATA_ARRAY_CONSTRUCT(m, n, t)
\end{codepar}
\cspecificend

\begin{arglist}
\argin{m}{Pointer to the structure to be initialized (pointer to \refstruct{pmix_data_array_t})}
\argin{n}{Number of elements in the array (\code{size_t})}
\argin{t}{\ac{PMIx} data type of the array elements (\refstruct{pmix_data_type_t})}
\end{arglist}


\littleheader{Destruct a data array structure}
\declaremacro{PMIX_DATA_ARRAY_DESTRUCT}

Destruct the \refstruct{pmix_data_array_t}, releasing the memory in the array.

\versionMarker{2.2}
\cspecificstart
\begin{codepar}
PMIX_DATA_ARRAY_CONSTRUCT(m)
\end{codepar}
\cspecificend

\begin{arglist}
\argin{m}{Pointer to the structure to be destructed (pointer to \refstruct{pmix_data_array_t})}
\end{arglist}


\littleheader{Create a data array structure}
\declaremacro{PMIX_DATA_ARRAY_CREATE}

Allocate memory for the \refstruct{pmix_data_array_t} object itself, and then allocate memory for the array of the indicated type.

\versionMarker{2.2}
\cspecificstart
\begin{codepar}
PMIX_DATA_ARRAY_CREATE(m, n, t)
\end{codepar}
\cspecificend

\begin{arglist}
\arginout{m}{Variable to be set to the address of the structure (pointer to \refstruct{pmix_data_array_t})}
\argin{n}{Number of elements in the array (\code{size_t})}
\argin{t}{\ac{PMIx} data type of the array elements (\refstruct{pmix_data_type_t})}
\end{arglist}


\littleheader{Free a data array structure}
\declaremacro{PMIX_DATA_ARRAY_FREE}

Release the memory in the array, and then release the \refstruct{pmix_data_array_t} object itself.

\versionMarker{2.2}
\cspecificstart
\begin{codepar}
PMIX_DATA_ARRAY_FREE(m)
\end{codepar}
\cspecificend

\begin{arglist}
\argin{m}{Pointer to the structure to be released (pointer to \refstruct{pmix_data_array_t})}
\end{arglist}

%%%%%%%%%%%%%%%%%%%%%%%%%%%%%%%%%%%%%%%%%%%%%%%%%
\subsection{Argument Array Macros}

The following macros support the construction and release of \code{NULL}-terminated argv arrays of strings.

%%%%
\littleheader{Argument array extension}
\declaremacro{PMIX_ARGV_APPEND}

Append a string to a NULL-terminated, argv-style array of strings.

\cspecificstart
\begin{codepar}
PMIX_ARGV_APPEND(r, a, b);
\end{codepar}
\cspecificend

\begin{arglist}
\argout{r}{Status code indicating success or failure of the operation (\refstruct{pmix_status_t})}
\arginout{a}{Argument list (pointer to NULL-terminated array of strings)}
\argin{b}{Argument to append to the list (string)}
\end{arglist}

This function helps the caller build the \code{argv} portion of \refstruct{pmix_app_t} structure, arrays of keys for querying, or other places where argv-style string arrays are required.

\adviceuserstart
The provided argument is copied into the destination array - thus, the source string can be free'd without affecting the array once the macro has completed.
\adviceuserend

%%%%
\littleheader{Argument array prepend}
\declaremacro{PMIX_ARGV_PREPEND}

Prepend a string to a NULL-terminated, argv-style array of strings.

\cspecificstart
\begin{codepar}
PMIX_ARGV_PREPEND(r, a, b);
\end{codepar}
\cspecificend

\begin{arglist}
\argout{r}{Status code indicating success or failure of the operation (\refstruct{pmix_status_t})}
\arginout{a}{Argument list (pointer to NULL-terminated array of strings)}
\argin{b}{Argument to append to the list (string)}
\end{arglist}

This function helps the caller build the \code{argv} portion of \refstruct{pmix_app_t} structure, arrays of keys for querying, or other places where argv-style string arrays are required.

\adviceuserstart
The provided argument is copied into the destination array - thus, the source string can be free'd without affecting the array once the macro has completed.
\adviceuserend

%%%%%%%%%%%
\littleheader{Argument array extension - unique}
\declaremacro{PMIX_ARGV_APPEND_UNIQUE}

Append a string to a NULL-terminated, argv-style array of strings, but only if the provided argument doesn't already exist somewhere in the array.

\cspecificstart
\begin{codepar}
PMIX_ARGV_APPEND_UNIQUE(r, a, b);
\end{codepar}
\cspecificend

\begin{arglist}
\argout{r}{Status code indicating success or failure of the operation (\refstruct{pmix_status_t})}
\arginout{a}{Argument list (pointer to NULL-terminated array of strings)}
\argin{b}{Argument to append to the list (string)}
\end{arglist}

This function helps the caller build the \code{argv} portion of \refstruct{pmix_app_t} structure, arrays of keys for querying, or other places where argv-style string arrays are required.

\adviceuserstart
The provided argument is copied into the destination array - thus, the source string can be free'd without affecting the array once the macro has completed.
\adviceuserend

%%%%%%%%%%%
\littleheader{Argument array release}
\declaremacro{PMIX_ARGV_FREE}

Free an argv-style array and all of the strings that it contains.

\cspecificstart
\begin{codepar}
PMIX_ARGV_FREE(a);
\end{codepar}
\cspecificend

\begin{arglist}
\argin{a}{Argument list (pointer to NULL-terminated array of strings)}
\end{arglist}

This function releases the array and all of the strings it contains.

%%%%%%%%%%%
\littleheader{Argument array split}
\declaremacro{PMIX_ARGV_SPLIT}

Split a string into a NULL-terminated argv array.

\cspecificstart
\begin{codepar}
PMIX_ARGV_SPLIT(a, b, c);
\end{codepar}
\cspecificend

\begin{arglist}
\argout{a}{Resulting argv-style array (\code{char**})}
\argin{b}{String to be split (\code{char*})}
\argin{c}{Delimiter character (\code{char})}
\end{arglist}

Split an input string into a NULL-terminated argv array. Do not include empty strings in the resulting array.

\adviceuserstart
All strings are inserted into the argv array by value; the newly-allocated array makes no references to the src_string argument (i.e., it can be freed after calling this function without invalidating the output argv array)
\adviceuserend

%%%%%%%%%%%
\littleheader{Argument array join}
\declaremacro{PMIX_ARGV_JOIN}

Join all the elements of an argv array into a single newly-allocated string.

\cspecificstart
\begin{codepar}
PMIX_ARGV_JOIN(a, b, c);
\end{codepar}
\cspecificend

\begin{arglist}
\argout{a}{Resulting string (\code{char*})}
\argin{b}{Argv-style array to be joined (\code{char**})}
\argin{c}{Delimiter character (\code{char})}
\end{arglist}

Join all the elements of an argv array into a single newly-allocated string.

%%%%%%%%%%%
\littleheader{Argument array count}
\declaremacro{PMIX_ARGV_COUNT}

Return the length of a NULL-terminated argv array.

\cspecificstart
\begin{codepar}
PMIX_ARGV_COUNT(r, a);
\end{codepar}
\cspecificend

\begin{arglist}
\argout{r}{Number of strings in the array (integer)}
\argin{a}{Argv-style array (\code{char**})}
\end{arglist}

Count the number of elements in an argv array

%%%%%%%%%%%
\littleheader{Argument array copy}
\declaremacro{PMIX_ARGV_COPY}

Copy an argv array, including copying all of its strings.

\cspecificstart
\begin{codepar}
PMIX_ARGV_COPY(a, b);
\end{codepar}
\cspecificend

\begin{arglist}
\argout{a}{New argv-style array (\code{char**})}
\argin{b}{Argv-style array (\code{char**})}
\end{arglist}

Copy an argv array, including copying all of its strings.


%%%%%%%%%%%%%%%%%%%%%%%%%%%%%%%%%%%%%%%%%%%%%%%%%
\subsection{Set Environment Variable}
\declaremacro{PMIX_SETENV}

%%%%
\summary

Set an environment variable in a \code{NULL}-terminated, env-style array.

\cspecificstart
\begin{codepar}
PMIX_SETENV(r, name, value, env);
\end{codepar}
\cspecificend


\begin{arglist}
\argout{r}{Status code indicating success or failure of the operation (\refstruct{pmix_status_t})}
\argin{name}{Argument name (string)}
\argin{value}{Argument value (string)}
\arginout{env}{Environment array to update (pointer to array of strings)}
\end{arglist}

%%%%
\descr

Similar to \code{setenv} from the C API, this allows the caller to set an environment variable in the specified \code{env} array, which could then be passed to the \refstruct{pmix_app_t} structure or any other destination.

\adviceuserstart
The provided name and value are copied into the destination environment array - thus, the source strings can be free'd without affecting the array once the macro has completed.
\adviceuserend


%%%%%%%%%%%%%%%%%%%%%%%%%%%%%%%%%%%%%%%%%%%%%%%%%
%%%%%%%%%%%%%%%%%%%%%%%%%%%%%%%%%%%%%%%%%%%%%%%%%
\section{Generalized Data Types Used for Packing/Unpacking}
\declarestruct{pmix_data_type_t}

The \refstruct{pmix_data_type_t} structure is a \code{uint16_t} type for identifying the data type for packing/unpacking purposes. New data type values introduced in this version of the Standard are shown in \textbf{\color{magenta}magenta}.

\adviceimplstart
The following constants can be used to set a variable of the type \refstruct{pmix_data_type_t}. Data types in the \ac{PMIx} Standard are defined in terms of the C-programming language. Implementers wishing to support other languages should provide the equivalent definitions in a language-appropriate manner. Additionally, a PMIx implementation may choose to add additional types.
\adviceimplend

\begin{constantdesc}
%
\declareconstitem{PMIX_UNDEF}
Undefined.
%
\declareconstitem{PMIX_BOOL}
Boolean (converted to/from native \code{true}/\code{false}) (\code{bool}).
%
\declareconstitem{PMIX_BYTE}
A byte of data (\code{uint8_t}).
%
\declareconstitem{PMIX_STRING}
\code{NULL} terminated string (\code{char*}).
%
\declareconstitem{PMIX_SIZE}
Size \code{size_t}.
%
\declareconstitem{PMIX_PID}
Operating \ac{PID} (\code{pid_t}).
%
\declareconstitem{PMIX_INT}
Integer (\code{int}).
%
\declareconstitem{PMIX_INT8}
8-byte integer (\code{int8_t}).
%
\declareconstitem{PMIX_INT16}
16-byte integer (\code{int16_t}).
%
\declareconstitem{PMIX_INT32}
32-byte integer (\code{int32_t}).
%
\declareconstitem{PMIX_INT64}
64-byte integer (\code{int64_t}).
%
\declareconstitem{PMIX_UINT}
Unsigned integer (\code{unsigned int}).
%
\declareconstitem{PMIX_UINT8}
Unsigned 8-byte integer (\code{uint8_t}).
%
\declareconstitem{PMIX_UINT16}
Unsigned 16-byte integer (\code{uint16_t}).
%
\declareconstitem{PMIX_UINT32}
Unsigned 32-byte integer (\code{uint32_t}).
%
\declareconstitem{PMIX_UINT64}
Unsigned 64-byte integer (\code{uint64_t}).
%
\declareconstitem{PMIX_FLOAT}
Float (\code{float}).
%
\declareconstitem{PMIX_DOUBLE}
Double (\code{double}).
%
\declareconstitem{PMIX_TIMEVAL}
Time value (\code{struct timeval}).
%
\declareconstitem{PMIX_TIME}
Time (\code{time_t}).
%
\declareconstitem{PMIX_STATUS}
Status code {\refstruct{pmix_status_t}}.
%
\declareconstitem{PMIX_VALUE}
Value (\refstruct{pmix_value_t}).
%
\declareconstitem{PMIX_PROC}
Process (\refstruct{pmix_proc_t}).
%
\declareconstitem{PMIX_APP}
Application context.
%
\declareconstitem{PMIX_INFO}
Info object.
%
\declareconstitem{PMIX_PDATA}
Pointer to data.
%
\declareconstitem{PMIX_BUFFER}
Buffer.
%
\declareconstitem{PMIX_BYTE_OBJECT}
Byte object (\refstruct{pmix_byte_object_t}).
%
\declareconstitem{PMIX_KVAL}
Key/value pair.
%
\declareconstitem{PMIX_PERSIST}
Persistance (\refstruct{pmix_persistence_t}).
%
\declareconstitem{PMIX_POINTER}
Pointer to an object (\code{void*}).
%
\declareconstitem{PMIX_SCOPE}
Scope (\refstruct{pmix_scope_t}).
%
\declareconstitem{PMIX_DATA_RANGE}
Range for data (\refstruct{pmix_data_range_t}).
%
\declareconstitem{PMIX_COMMAND}
PMIx command code (used internally).
%
\declareconstitem{PMIX_INFO_DIRECTIVES}
Directives flag for \refstruct{pmix_info_t} (\refstruct{pmix_info_directives_t}).
%
\declareconstitem{PMIX_DATA_TYPE}
Data type code (\refstruct{pmix_data_type_t}).
%
\declareconstitem{PMIX_PROC_STATE}
Process state (\refstruct{pmix_proc_state_t}).
%
\declareconstitem{PMIX_PROC_INFO}
Process information (\refstruct{pmix_proc_info_t}).
%
\declareconstitem{PMIX_DATA_ARRAY}
Data array (\refstruct{pmix_data_array_t}).
%
\declareconstitem{PMIX_PROC_RANK}
Process rank (\refstruct{pmix_rank_t}).
%
\declareconstitem{PMIX_QUERY}
Query structure (\refstruct{pmix_query_t}).
%
\declareconstitem{PMIX_COMPRESSED_STRING}
String compressed with zlib (\code{char*}).
%
\declareconstitemNEW{PMIX_COMPRESSED_BYTE_OBJECT}
Byte object whose bytes have been compressed with zlib (\code{pmix_byte_object_t}).
%
\declareconstitem{PMIX_ALLOC_DIRECTIVE}
Allocation directive (\refstruct{pmix_alloc_directive_t}).
%
\declareconstitem{PMIX_IOF_CHANNEL}
Input/output forwarding channel (\refstruct{pmix_iof_channel_t}).
%
\declareconstitem{PMIX_ENVAR}
Environmental variable structure (\refstruct{pmix_envar_t}).
%
\declareconstitemNEW{PMIX_COORD}
Structure containing fabric coordinates (\refstruct{pmix_coord_t}).
%
\declareconstitemNEW{PMIX_REGATTR}
Structure supporting attribute registrations (\refstruct{pmix_regattr_t}).
%
\declareconstitemNEW{PMIX_REGEX}
Regular expressions - can be a valid NULL-terminated string or an arbitrary array of bytes.
%
\declareconstitemNEW{PMIX_JOB_STATE}
Job state (\refstruct{pmix_job_state_t}).
%
\declareconstitemNEW{PMIX_LINK_STATE}
Link state (\refstruct{pmix_link_state_t}).
%
\declareconstitemNEW{PMIX_PROC_CPUSET}
Structure containing the binding bitmap of a process (\refstruct{pmix_cpuset_t}).
%
\declareconstitemNEW{PMIX_GEOMETRY}
Geometry structure containing the fabric coordinates of a specified device.(\refstruct{pmix_geometry_t}).
%
\declareconstitemNEW{PMIX_DEVICE_DIST}
Structure containing the minimum and maximum relative distance from the caller to a given fabric device. (\refstruct{pmix_device_distance_t}).
%
\declareconstitemNEW{PMIX_ENDPOINT}
Structure containing an assigned endpoint for a given fabric device. (\refstruct{pmix_endpoint_t}).
%
\declareconstitemNEW{PMIX_TOPO}
Structure containing the topology for a given node. (\refstruct{pmix_topology_t}).
%
\declareconstitemNEW{PMIX_DEVTYPE}
Bitmask containing the types of devices being referenced. (\refstruct{pmix_device_type_t}).
%
\declareconstitemNEW{PMIX_LOCTYPE}
Bitmask describing the relative location of another process. (\refstruct{pmix_locality_t}).
%
\declareconstitemNEW{PMIX_DATA_TYPE_MAX}
A starting point for implementer-specific data types.
Values above this are guaranteed not to conflict with \ac{PMIx} values.
Definitions should always be based on the \refconst{PMIX_DATA_TYPE_MAX} constant and not a specific value as the value of the constant may change.
%
\end{constantdesc}


%%%%%%%%%%%%%%%%%%%%%%%%%%%%%%%%%%%%%%%%%%%%%%%%%
%%%%%%%%%%%%%%%%%%%%%%%%%%%%%%%%%%%%%%%%%%%%%%%%%
\section{General Callback Functions}

PMIx provides blocking and nonblocking versions of most APIs.
In the nonblocking versions, a callback is activated upon completion of the the operation.
This section describes many of those callbacks.

%%%%%%%%%%%%%%%%%%%%%%%%%%%%%%%%%%%%%%%%%%%%%%%%%
\subsection{Release Callback Function}
\declareapi{pmix_release_cbfunc_t}

%%%%
\summary

The \refapi{pmix_release_cbfunc_t} is used by the \refapi{pmix_modex_cbfunc_t} and \refapi{pmix_info_cbfunc_t} operations to indicate that the callback data may be reclaimed/freed by the caller.

%%%%
\format

\versionMarker{1.0}
\cspecificstart
\begin{codepar}
typedef void (*pmix_release_cbfunc_t)
    (void *cbdata);
\end{codepar}
\cspecificend

\begin{arglist}
\arginout{cbdata}{Callback data passed to original API call (memory reference)}
\end{arglist}

%%%%
\descr

Since the data is ``owned'' by the host server, provide a callback function to notify the host server that we are done with the data so it can be released.


%%%%%%%%%%%%%%%%%%%%%%%%%%%%%%%%%%%%%%%%%%%%%%%%%
\subsection{Op Callback Function}
\declareapi{pmix_op_cbfunc_t}

%%%%
\summary

The \refapi{pmix_op_cbfunc_t} is used by operations that simply return a status.

\versionMarker{1.0}
\cspecificstart
\begin{codepar}
typedef void (*pmix_op_cbfunc_t)
    (pmix_status_t status, void *cbdata);
\end{codepar}
\cspecificend

\begin{arglist}
\argin{status}{Status associated with the operation (handle)}
\argin{cbdata}{Callback data passed to original API call (memory reference)}
\end{arglist}

%%%%
\descr

Used by a wide range of \ac{PMIx} API's including \refapi{PMIx_Fence_nb}, \refapi{pmix_server_client_connected2_fn_t}, \refapi{PMIx_server_register_nspace}.
This callback function is used to return a status to an often nonblocking operation.


%%%%%%%%%%%%%%%%%%%%%%%%%%%%%%%%%%%%%%%%%%%%%%%%%
\subsection{Value Callback Function}
\declareapi{pmix_value_cbfunc_t}

%%%%
\summary

The \refapi{pmix_value_cbfunc_t} is used by \refapi{PMIx_Get_nb} to return data.

\versionMarker{1.0}
\cspecificstart
\begin{codepar}
typedef void (*pmix_value_cbfunc_t)
    (pmix_status_t status,
     pmix_value_t *kv, void *cbdata);
\end{codepar}
\cspecificend

\begin{arglist}
\argin{status}{Status associated with the operation (handle)}
\argin{kv}{Key/value pair representing the data (\refstruct{pmix_value_t})}
\argin{cbdata}{Callback data passed to original API call (memory reference)}
\end{arglist}


%%%%
\descr

A callback function for calls to \refapi{PMIx_Get_nb}.
The \refarg{status} indicates if the requested data was found or not.
A pointer to the \refstruct{pmix_value_t} structure containing the found data is returned.
The pointer will be \code{NULL} if the requested data was not found.


%%%%%%%%%%%%%%%%%%%%%%%%%%%%%%%%%%%%%%%%%%%%%%%%%
\subsection{Info Callback Function}
\declareapi{pmix_info_cbfunc_t}

%%%%
\summary

The \refapi{pmix_info_cbfunc_t} is a general information callback used by various APIs.

\versionMarker{2.0}
\cspecificstart
\begin{codepar}
typedef void (*pmix_info_cbfunc_t)
    (pmix_status_t status,
     pmix_info_t info[], size_t ninfo,
     void *cbdata,
     pmix_release_cbfunc_t release_fn,
     void *release_cbdata);
\end{codepar}
\cspecificend

\begin{arglist}
\argin{status}{Status associated with the operation (\refstruct{pmix_status_t})}
\argin{info}{Array of \refstruct{pmix_info_t} returned by the operation (pointer)}
\argin{ninfo}{Number of elements in the \argref{info} array (\code{size_t})}
\argin{cbdata}{Callback data passed to original API call (memory reference)}
\argin{release_fn}{Function to be called when done with the \argref{info} data (function pointer)}
\argin{release_cbdata}{Callback data to be passed to \argref{release_fn} (memory reference)}
\end{arglist}


%%%%
\descr

The \refarg{status} indicates if requested data was found or not.
An array of \refstruct{pmix_info_t} will contain the key/value pairs.

%%%%%%%%%%%
\subsection{Handler registration callback function}
\declareapi{pmix_hdlr_reg_cbfunc_t}

%%%%
\summary

Callback function for calls to register handlers, e.g., event notification and IOF requests.

%%%%
\format

\versionMarker{3.0}
\cspecificstart
\begin{codepar}
typedef void (*pmix_hdlr_reg_cbfunc_t)
    (pmix_status_t status,
     size_t refid,
     void *cbdata);
\end{codepar}
\cspecificend

\begin{arglist}
\argin{status}{\refconst{PMIX_SUCCESS} or an appropriate error constant (\refstruct{pmix_status_t})}
\argin{refid}{reference identifier assigned to the handler by PMIx, used to deregister the handler (\code{size_t})}
\argin{cbdata}{object provided to the registration call (pointer)}
\end{arglist}

%%%%
\descr

Callback function for calls to register handlers, e.g., event notification and IOF requests.


%%%%%%%%%%%%%%%%%%%%%%%%%%%%%%%%%%%%%%%%%%%%%%%%%
%%%%%%%%%%%%%%%%%%%%%%%%%%%%%%%%%%%%%%%%%%%%%%%%%
\section{PMIx Datatype Value String Representations}

Provide a string representation for several types of values.
Note that the provided string is statically defined and must NOT be \code{free}'d.

%%%%
\summary
\declareapi{PMIx_Error_string}

String representation of a \refstruct{pmix_status_t}.

\versionMarker{1.0}
\cspecificstart
\begin{codepar}
const char*
PMIx_Error_string(pmix_status_t status);
\end{codepar}
\cspecificend

%%%%
\summary
\declareapi{PMIx_Proc_state_string}

String representation of a \refstruct{pmix_proc_state_t}.

\versionMarker{2.0}
\cspecificstart
\begin{codepar}
const char*
PMIx_Proc_state_string(pmix_proc_state_t state);
\end{codepar}
\cspecificend

%%%%
\summary
\declareapi{PMIx_Scope_string}

String representation of a \refstruct{pmix_scope_t}.

\versionMarker{2.0}
\cspecificstart
\begin{codepar}
const char*
PMIx_Scope_string(pmix_scope_t scope);
\end{codepar}
\cspecificend

%%%%
\summary
\declareapi{PMIx_Persistence_string}

String representation of a \refstruct{pmix_persistence_t}.

\versionMarker{2.0}
\cspecificstart
\begin{codepar}
const char*
PMIx_Persistence_string(pmix_persistence_t persist);
\end{codepar}
\cspecificend

%%%%
\summary
\declareapi{PMIx_Data_range_string}

String representation of a \refstruct{pmix_data_range_t}.

\versionMarker{2.0}
\cspecificstart
\begin{codepar}
const char*
PMIx_Data_range_string(pmix_data_range_t range);
\end{codepar}
\cspecificend

%%%%
\summary
\declareapi{PMIx_Info_directives_string}

String representation of a \refstruct{pmix_info_directives_t}.

\versionMarker{2.0}
\cspecificstart
\begin{codepar}
const char*
PMIx_Info_directives_string(pmix_info_directives_t directives);
\end{codepar}
\cspecificend

%%%%
\summary
\declareapi{PMIx_Data_type_string}

String representation of a \refstruct{pmix_data_type_t}.

\versionMarker{2.0}
\cspecificstart
\begin{codepar}
const char*
PMIx_Data_type_string(pmix_data_type_t type);
\end{codepar}
\cspecificend

%%%%
\summary
\declareapi{PMIx_Alloc_directive_string}

String representation of a \refstruct{pmix_alloc_directive_t}.

\versionMarker{2.0}
\cspecificstart
\begin{codepar}
const char*
PMIx_Alloc_directive_string(pmix_alloc_directive_t directive);
\end{codepar}
\cspecificend

%%%%
\summary
\declareapi{PMIx_IOF_channel_string}

String representation of a \refstruct{pmix_iof_channel_t}.

\versionMarker{3.0}
\cspecificstart
\begin{codepar}
const char*
PMIx_IOF_channel_string(pmix_iof_channel_t channel);
\end{codepar}
\cspecificend

%%%%
\summary
\declareapi{PMIx_Job_state_string}

String representation of a \refstruct{pmix_job_state_t}.

\versionMarker{4.0}
\cspecificstart
\begin{codepar}
const char*
PMIx_Job_state_string(pmix_job_state_t state);
\end{codepar}
\cspecificend

%%%%
\summary
\declareapi{PMIx_Get_attribute_string}

String representation of a \ac{PMIx} attribute.

\versionMarker{4.0}
\cspecificstart
\begin{codepar}
const char*
PMIx_Get_attribute_string(char *attributename);
\end{codepar}
\cspecificend

%%%%
\summary
\declareapi{PMIx_Get_attribute_name}

Return the \ac{PMIx} attribute name corresponding to the given attribute string.

\versionMarker{4.0}
\cspecificstart
\begin{codepar}
const char*
PMIx_Get_attribute_name(char *attributestring);
\end{codepar}
\cspecificend

%%%%
\summary
\declareapi{PMIx_Link_state_string}

String representation of a \refstruct{pmix_link_state_t}.

\versionMarker{4.0}
\cspecificstart
\begin{codepar}
const char*
PMIx_Link_state_string(pmix_link_state_t state);
\end{codepar}
\cspecificend

%%%%
\summary
\declareapi{PMIx_Device_type_string}

String representation of a \refstruct{pmix_device_type_t}.

\versionMarker{4.0}
\cspecificstart
\begin{codepar}
const char*
PMIx_Device_type_string(pmix_device_type_t type);
\end{codepar}
\cspecificend


%%%%%%%%%%%%%%%%%%%%%%%%%%%%%%%%%%%%%%%%%%%%%%%%%

    %%%%%%%%%%%%%%%%%%%%%%%%%%%%%%%%%%%%%%%%%%%%%%%%%
% Chapter: API Client
%%%%%%%%%%%%%%%%%%%%%%%%%%%%%%%%%%%%%%%%%%%%%%%%%
\chapter{Client-Side API}
\label{chap:api_client}

\ldots

%%%%%%%%%%%
\section{Startup and Shutdown}
\label{chap:api_client:startup}

Initialization and finalization routines for \ac{PMIx} clients.

%%%%%%%%%%%
\subsection{\code{PMIx_Init}}
\declareapi{PMIx_Init}

%%%%
\summary

Initialize the PMIx client.

%%%%
\format

\cspecificstart
\begin{codepar}
pmix_status_t PMIx_Init(pmix_proc_t *proc,
                        pmix_info_t info[], size_t ninfo)
\end{codepar}
\cspecificend

\begin{arglist}
\arginout{proc}{proc structure (handle)}
\argin{info}{Array of info structures (array of handles)}
\argin{ninfo}{Number of element in the \refarg{info} array (integer)}
\end{arglist}

Returns \refconst{PMIX_SUCCESS} or a negative value corresponding to a PMIx error constant.

%%%%
\descr

Initialize the PMIx client, returning the process identifier assigned to this client's application in the provided \refstruct{pmix_proc_t} struct.
Passing a value of \code{NULL} for this parameter is allowed if the user wishes solely to initialize the PMIx system and does not require return of the identifier at that time.

When called, the PMIx client shall check for the required connection information of the local PMIx server and establish the connection.
If the information is not found, or the server connection fails, then an appropriate error constant shall be returned.

If successful, the function shall return \refconst{PMIX_SUCCESS} and fill the \refarg{proc} structure with the server-assigned namespace and rank of the process within the application.
In addition, all startup information provided by the resource manager shall be made available to the client process via subsequent calls to \refapi{PMIx_Get}.

The PMIx client library shall be reference counted, and so multiple calls to \refapi{PMIx_Init} are allowed by the standard.
Thus, one way for an application process to obtain its namespace and rank is to simply call \refapi{PMIx_Init} with a non-NULL \refarg{proc} parameter.
Note that each call to \refapi{PMIx_Init} must be balanced with a call to \refapi{PMIx_Finalize} to maintain the reference count.

Each call to \refapi{PMIx_Init} may contain an array of \refstruct{pmix_info_t} structures passing directives to the PMIx client library.
This might include information about the location of temporary directories set up for the application, or constraints on communication protocols for connecting to the local PMIx server.
Multiple calls to \refapi{PMIx_Init} shall not include conflicting directives (e.g., a directive indicating that one particular communication method be used to connect to the server, followed by a subsequent call that includes a directive that a different method be used).
The \refapi{PMIx_Init} function will return an error when directives that conflict with prior directives are encountered.


%%%%%%%%%%%
\subsection{\code{PMIx_Finalize}}
\declareapi{PMIx_Finalize}

%%%%
\summary

Finalize the PMIx client library.

%%%%
\format

\cspecificstart
\begin{codepar}
pmix_status_t PMIx_Finalize(const pmix_info_t info[], size_t ninfo)
\end{codepar}
\cspecificend

\begin{arglist}
\argin{info}{Array of info structures (array of handles)}
\argin{ninfo}{Number of element in the \refarg{info} array (integer)}
\end{arglist}

Returns \refconst{PMIX_SUCCESS} or a negative value corresponding to a PMIx error constant.

%%%%
\descr

Decrement the PMIx client library reference count.
When the reference count reaches zero, the library will finalize the PMIx client, closing the connection with the local PMIx server and releasing all internally allocated memory.

By default, \refapi{PMIx_Finalize} will not include an internal barrier operation.
Users desiring a barrier as part of the finalize operation can request it by including the \refattr{PMIX_EMBED_BARRIER} attribute in the provided \refstruct{pmix_info_t} array.


%%%%%%%%%%%
\subsection{\code{PMIx_Initialized}}
\declareapi{PMIx_Initialized}

%%%%
\summary

Determines if the PMIx library has been initialized.

%%%%
\format

\cspecificstart
\begin{codepar}
int PMIx_Initialized(void)
\end{codepar}
\cspecificend

A value of \code{1} (true) will be returned if the PMIx library has been initialized, and \code{0} (false) otherwise.

\rationalestart
The return value is an integer for historical reasons as that was the signature of prior PMI libraries.
\rationaleend

%%%%
\descr

Check to see if the PMIx library has been initialized using any of the init functions:
\refapi{PMIx_Init}, \refapi{PMIx_server_init}, or \refapi{PMIx_tool_init}.


%%%%%%%%%%%
\subsection{\code{PMIx_Abort}}
\declareapi{PMIx_Abort}

%%%%
\summary

Abort the specified process.

%%%%
\format

\cspecificstart
\begin{codepar}
pmix_status_t PMIx_Abort(int status, const char msg[],
                         pmix_proc_t procs[], size_t nprocs)
\end{codepar}
\cspecificend

\begin{arglist}
\argin{status}{Error code to return to invoking environment (integer)}
\argin{msg}{String message to be returned to user (string)}
\argin{procs}{Array of \refstruct{pmix_proc_t} structures (array of handles)}
\argin{nprocs}{Number of elements in the \refarg{procs} array (integer)}
\end{arglist}

Returns \refconst{PMIX_SUCCESS} or a negative value corresponding to a PMIx error constant.

%%%%
\descr

Request that the host resource manager print the provided message and abort the provided array of \refarg{procs}.
A Unix or POSIX environment should handle the provided status as a return error code from the main program that launched the application.
A \code{NULL} for the \refarg{procs} array indicates that all processes in the caller's namespace are to be aborted, including itself.
Passing a \code{NULL} \refarg{msg} parameter is allowed.

\adviceuserstart
The response to this request is somewhat dependent on the specific \acl{RM} and its configuration (e.g., some resource managers will not abort the application if the provided status is zero unless specifically configured to do so, and some cannot abort subsets of processes in an application), and thus lies outside the control of PMIx itself.
However, the PMIx client library shall inform the \ac{RM} of the request that the specified \refarg{procs} be aborted, regardless of the value of the provided status.

Note that race conditions caused by multiple processes calling \refapi{PMIx_Abort} are left to the server implementation to resolve with regard to which status is returned and what messages (if any) are printed.
\adviceuserend


%%%%%%%%%%%
\section{Key/Value Management}
\label{chap:api_client:keyvalue}

\ldots

%%%%%%%%%%%
\subsection{\code{PMIx_Put}}
\declareapi{PMIx_Put}

%%%%
\summary

Push a key/value pair into the client's namespace.

%%%%
\format

\cspecificstart
\begin{codepar}
/* Push a value into the client's namespace. The client library will cache
 * the information locally until _PMIx_Commit_ is called. The provided scope
 * value is passed to the local PMIx server, which will distribute the data
 * as directed. */
pmix_status_t PMIx_Put(pmix_scope_t scope,
                       const char key[], pmix_value_t *val)
\end{codepar}
\cspecificend

\begin{arglist}
\argin{scope}{Distribution scope of the provided value (handle)}
\argin{key}{key (string)}
\argin{value}{Reference to a \refstruct{pmix_value_t} structure (handle)}
\end{arglist}

Returns \refconst{PMIX_SUCCESS} or a negative value corresponding to a PMIx error constant.

%%%%
\descr

Push a value into the client's namespace.
The client library will cache the information locally until \refapi{PMIx_Commit} is called.

The provided \refarg{scope} is passed to the local PMIx server, which will distribute the data to other processes according to the provided scope.
The \refstruct{pmix_scope_t} values are defined in \specrefstruct{pmix_scope_t}.
Specific implementations may support different scope values, but all implementations must support at least \code{PMIX\_GLOBAL}.

The \refstruct{pmix_value_t} structure supports both string and binary values.
Implementations will support heterogeneous environments by properly converting binary values between host architectures, and will copy the provided \refarg{value} into internal memory.

\adviceimplstart
The \refapi{PMIx_Data_pack}/\refapi{PMIx_Data_unpack} routines are provided to assist in meeting the heterogeneity requirement.
\adviceimplend

\adviceuserstart
The value is copied by the PMIx client library.
Thus, the application is free to release and/or modify the value once the call to \refapi{PMIx_Put} has completed.
\adviceuserend


%%%%%%%%%%%
\subsection{\code{PMIx_Get}}
\declareapi{PMIx_Get}

%%%%
\summary

Retrieve a key/value pair from the client's namespace.

%%%%
\format

\cspecificstart
\begin{codepar}
pmix_status_t PMIx_Get(const pmix_proc_t *proc, const char key[],
                       const pmix_info_t info[], size_t ninfo,
                       pmix_value_t **val)
\end{codepar}
\cspecificend

\begin{arglist}
\argin{proc}{process reference (handle)}
\argin{key}{key to retrieve (string)}
\argin{info}{Array of info structures (array of handles)}
\argin{ninfo}{Number of element in the \refarg{info} array (integer)}
\argout{val}{value (handle)}
\end{arglist}

Returns \refconst{PMIX_SUCCESS} or a negative value corresponding to a PMIx error constant.

%%%%
\descr

Retrieve information for the specified \refarg{key} as published by the process identified in the given \refstruct{pmix_proc_t}, returning a pointer to the value in the given address.

This is a blocking operation - the caller will block until the specified data has been \refapi{PMIx_Put} by the specified rank in the \refarg{proc} structure.
The caller is responsible for freeing all memory associated with the returned \refarg{value} when no longer required.

The \refarg{info} array is used to pass user requests regarding the get operation.
This can include the \refattr{PMIX_TIMEOUT} attribute.


%%%%%%%%%%%
\subsection{\code{PMIx_Get_nb}}
\declareapi{PMIx_Get_nb}

%%%%
\summary

Nonblocking \refapi{PMIx_Get} operation.

%%%%
\format

\cspecificstart
\begin{codepar}
pmix_status_t PMIx_Get_nb(const pmix_proc_t *proc, const char key[],
                          const pmix_info_t info[], size_t ninfo,
                          pmix_value_cbfunc_t cbfunc, void *cbdata)
\end{codepar}
\cspecificend

\begin{arglist}
\argin{proc}{process reference (handle)}
\argin{key}{key to retrieve (string)}
\argin{info}{Array of info structures (array of handles)}
\argin{ninfo}{Number of elements in the \refarg{info} array (integer)}
\argin{cbfunc}{Callback function (function reference)}
\argin{cbdata}{Data to be passed to the callback function (memory reference)}
\end{arglist}

Returns \refconst{PMIX_SUCCESS} or a negative value corresponding to a PMIx error constant.

%%%%
\descr

The callback function will be executed once the specified data has been \refapi{PMIx_Put} by the identified process and retrieved by the local server.
The \argref{info} array is used as described by the \refapi{PMIx_Get} routine.


%%%%%%%%%%%
\subsection{\code{PMIx_Commit}}
\declareapi{PMIx_Commit}

%%%%
\summary

Push all previously \refapi{PMIx_Put} values to the local PMIx server.

%%%%
\format

\cspecificstart
\begin{codepar}
pmix_status_t PMIx_Commit(void)
\end{codepar}
\cspecificend

Returns \refconst{PMIX_SUCCESS} or a negative value corresponding to a PMIx error constant.

%%%%
\descr

This is an asynchronous operation.
The PMIx library will immediately return to the caller while the data is transmitted to the local server in the background.

\adviceuserstart
The local PMIx server will cache the information locally.
Meaning that the committed data will not be circulated during \refapi{PMIx_Commit}.
Availability of the data upon completion of \refapi{PMIx_Commit} is therefore implementation-dependent.
\adviceuserend


%%%%%%%%%%%
\subsection{\code{PMIx_Fence}}
\declareapi{PMIx_Fence}

%%%%
\summary

Execute a blocking barrier across the processes identified in the specified array.

%%%%
\format

\cspecificstart
\begin{codepar}
pmix_status_t PMIx_Fence(const pmix_proc_t procs[], size_t nprocs,
                         const pmix_info_t info[], size_t ninfo)
\end{codepar}
\cspecificend

\begin{arglist}
\argin{procs}{Array of \refstruct{pmix_proc_t} structures (array of handles)}
\argin{nprocs}{Number of element in the \refarg{procs} array (integer)}
\argin{info}{Array of info structures (array of handles)}
\argin{ninfo}{Number of element in the \refarg{info} array (integer)}
\end{arglist}

Returns \refconst{PMIX_SUCCESS} or a negative value corresponding to a PMIx error constant.

%%%%
\descr

Passing a \code{NULL} pointer as the \refarg{procs} parameter indicates that the fence is to span all processes in the client's namespace.
Each provided \refstruct{pmix_proc_t} struct can pass \refconst{PMIX_RANK_WILDCARD} to indicate that all processes in the given namespace are participating.

The \refarg{info} array is used to pass user requests regarding the fence operation.
This can include:

\begin{attributedesc}
%
\declareattritem{PMIX_COLLECT_DATA} (string)
A boolean indicating whether or not the barrier operation is to return the \emph{put} data from all participating processes.
A value of \emph{false} indicates that the callback is just used as a release and no data is to be returned at that time.
A value of \emph{true} indicates that all \emph{put} data is to be collected by the barrier.
Returned data is cached at the server to reduce memory footprint, and can be retrieved as needed by calls to \refapi{PMIx_Get}/\refapi{PMIx_Get_nb}.
%
\declareattritem{PMIX_COLLECTIVE_ALGO} (string)
A comma-delimited string indicating the algorithm to be used for executing the barrier, in priority order.
%
\declareattritem{PMIX_COLLECTIVE_ALGO_REQD} (string)
Instructs the host \ac{RM} that it should return an error if none of the specified algorithms are available.
Otherwise, the \ac{RM} is to use one of the algorithms if possible, but is otherwise free to use any of its available methods to execute the operation.
%
\declareattritem{PMIX_TIMEOUT} (string)
Maximum time for the fence to execute before declaring an error.
By default, the \ac{RM} shall terminate the operation and notify participants if one or more of the indicated \refarg{procs} fails during the fence.
However, the timeout parameter can help avoid ``hangs'' due to programming errors that prevent one or more processes from reaching the ``fence''.
%
\end{attributedesc}

Note that for scalability reasons, the default behavior for \refapi{PMIx_Fence} is to \emph{not} collect the data.


%%%%%%%%%%%
\subsection{\code{PMIx_Fence_nb}}
\declareapi{PMIx_Fence_nb}

%%%%
\summary

Execute a nonblocking \refapi{PMIx_Fence} across the processes identified in the specified array of processes.

%%%%
\format

\cspecificstart
\begin{codepar}
pmix_status_t PMIx_Fence_nb(const pmix_proc_t procs[], size_t nprocs,
                            const pmix_info_t info[], size_t ninfo,
                            pmix_op_cbfunc_t cbfunc, void *cbdata)
\end{codepar}
\cspecificend

\begin{arglist}
\argin{procs}{Array of \refstruct{pmix_proc_t} structures (array of handles)}
\argin{nprocs}{Number of element in the \refarg{procs} array (integer)}
\argin{info}{Array of info structures (array of handles)}
\argin{ninfo}{Number of element in the \refarg{info} array (integer)}
\argin{cbfunc}{Callback function (function reference)}
\argin{cbdata}{Data to be passed to the callback function (memory reference)}
\end{arglist}

Returns \refconst{PMIX_SUCCESS} or a negative value corresponding to a PMIx error constant.

%%%%
\descr

Nonblocking \refapi{PMIx_Fence} routine.
Note that the function will return an error if a \code{NULL} callback function is given.


%%%%%%%%%%%
\subsection{\code{PMIx_Publish}}
\declareapi{PMIx_Publish}

%%%%
\summary

Publish data for later access via \refapi{PMIx_Lookup}.

%%%%
\format

\cspecificstart
\begin{codepar}
pmix_status_t PMIx_Publish(const pmix_info_t info[], size_t ninfo)
\end{codepar}
\cspecificend

\begin{arglist}
\argin{info}{Array of info structures (array of handles)}
\argin{ninfo}{Number of element in the \refarg{info} array (integer)}
\end{arglist}

Returns \refconst{PMIX_SUCCESS} or a negative value corresponding to a PMIx error constant.

%%%%
\descr

Publish the data in the \refarg{info} array for lookup.
By default, the data will be published into the \refconst{PMIX_SESSION} range and with \refconst{PMIX_PERSIST_APP} persistence.
Changes to those values, and any additional directives, can be included in the \refstruct{pmix_info_t} array.

Note that the keys must be unique within the specified data range or else an error will be returned (first published wins).
Attempts to access the data by processes outside of the provided data range will be rejected.

The persistence parameter instructs the server as to how long the data is to be retained.

The blocking form will block until the server confirms that the data has been posted and is available.
The non-blocking form will return immediately, executing the callback when the server confirms availability of the data.


%%%%%%%%%%%
\subsection{\code{PMIx_Publish_nb}}
\declareapi{PMIx_Publish_nb}

%%%%
\summary

Nonblocking \refapi{PMIx_Publish} routine.

%%%%
\format

\cspecificstart
\begin{codepar}
pmix_status_t PMIx_Publish_nb(const pmix_info_t info[], size_t ninfo,
                              pmix_op_cbfunc_t cbfunc, void *cbdata)

\end{codepar}
\cspecificend

\begin{arglist}
\argin{info}{Array of info structures (array of handles)}
\argin{ninfo}{Number of element in the \refarg{info} array (integer)}
\argin{cbfunc}{Callback function \refapi{pmix_op_cbfunc_t} (function reference)}
\argin{cbdata}{Data to be passed to the callback function (memory reference)}
\end{arglist}

Returns \refconst{PMIX_SUCCESS} or a negative value corresponding to a PMIx error constant.

%%%%
\descr

Nonblocking \refapi{PMIx_Publish} routine.
Note that the function will return an error if a \code{NULL} callback function is given.


%%%%%%%%%%%
\subsection{\code{PMIx_Lookup}}
\declareapi{PMIx_Lookup}

%%%%
\summary

Lookup information published by this or another process with \refapi{PMIx_Publish} or \refapi{PMIx_Publish_nb}.

%%%%
\format

\cspecificstart
\begin{codepar}
pmix_status_t PMIx_Lookup(pmix_pdata_t data[], size_t ndata,
                          const pmix_info_t info[], size_t ninfo)
\end{codepar}
\cspecificend

\begin{arglist}
\argin{data}{Array of publishable data structures (array of handles)}
\argin{ndata}{Number of elements in the \refarg{data} array (integer)}
\argin{info}{Array of info structures (array of handles)}
\argin{ninfo}{Number of elements in the \refarg{info} array (integer)}
\end{arglist}

Returns \refconst{PMIX_SUCCESS} or a negative value corresponding to a PMIx error constant.

%%%%
\descr

Lookup information published by this or another process.
By default, the search will be conducted across the \refconst{PMIX_SESSION} range.
Changes to the range, and any additional directives, can be provided in the \refstruct{pmix_info_t} array.

Note that the search is also constrained to only data published by the current user (i.e., the search will not return data published by an application being executed by another user).
There currently is no option to override this behavior - such an option may become available later via an appropriate \refstruct{pmix_info_t} directive.

The \argref{data} parameter consists of an array of \refstruct{pmix_pdata_t} struct with the keys specifying the requested information.
Data will be returned for each key in the associated \refarg{info} struct.
Any key that cannot be found will return with a data type of \refconst{PMIX_UNDEF}.
The function will return \refconst{PMIX_SUCCESS} if \emph{any} values can be found, so the caller must check each data element to ensure it was returned.

The proc field in each \refstruct{pmix_pdata_t} struct will contain the namespace/rank of the process that published the data.

\adviceuserstart
Although this is a blocking function, it will \emph{not} wait by default for the requested data to be published.
Instead, it will block for the time required by the server to lookup its current data and return any found items.
Thus, the caller is responsible for ensuring that data is published prior to executing a lookup, or for retrying until the requested data is found.
\adviceuserend

Optionally, the \refarg{info} array can be used to modify this behavior by including:
%%%%%%%%%%%%%%%%%%5 JJH RETURN HERE
% *
% * (a) PMIX_WAIT - wait for the requested data to be published. The
% *     server is to wait until all data has become available.
% *
% * (b) PMIX_TIMEOUT - max time to wait for data to become available.
% *
% */


%%%%%%%%%%%
\subsection{\code{PMIx_Lookup_nb}}
\declareapi{PMIx_Lookup_nb}

%%%%
\summary

Nonblocking version of \refapi{PMIx_Lookup}.

%%%%
\format

\cspecificstart
\begin{codepar}
pmix_status_t PMIx_Lookup_nb(char **keys,
                             const pmix_info_t info[], size_t ninfo,
                             pmix_lookup_cbfunc_t cbfunc, void *cbdata)
\end{codepar}
\cspecificend

\begin{arglist}
\argin{keys}{Array to be provided to the callback (array of strings)}
\argin{info}{Array of info structures (array of handles)}
\argin{ninfo}{Number of element in the \refarg{info} array (integer)}
\argin{cbfunc}{Callback function (handle)}
\argin{cbdata}{Callback data to be provided to the callback function (pointer)}
\end{arglist}

Returns \refconst{PMIX_SUCCESS} or a negative value corresponding to a PMIx error constant.

%%%%
\descr

Non-blocking form of the \refapi{PMIx_Lookup} function.
Data for the provided NULL-terminated \refarg{keys} array will be returned in the provided callback function.
As with \refapi{PMIx_Lookup}, the default behavior is to \emph{not} wait for data to be published.
The \refarg{info} keys can be used to modify the behavior as previously described by \refapi{PMIx_Lookup}.


%%%%%%%%%%%
\subsection{\code{PMIx_Unpublish}}
\declareapi{PMIx_Unpublish}

%%%%
\summary

Unpublish data posted by this process using the given keys.

%%%%
\format

\cspecificstart
\begin{codepar}
pmix_status_t PMIx_Unpublish(char **keys,
                             const pmix_info_t info[], size_t ninfo)
\end{codepar}
\cspecificend

\begin{arglist}
\argin{info}{Array of info structures (array of handles)}
\argin{ninfo}{Number of element in the \refarg{info} array (integer)}
\end{arglist}

Returns \refconst{PMIX_SUCCESS} or a negative value corresponding to a PMIx error constant.

%%%%
\descr

Unpublish data posted by this process using the given \refarg{keys}.
The function will block until the data has been removed by the server.
A value of \code{NULL} for the \refarg{keys} parameter instructs the server to remove \emph{all} data published by this process.

By default, the range is assumed to be \refconst{PMIX_SESSION}.
Changes to the range, and any additional directives, can be provided in the \refarg{info} array.


%%%%%%%%%%%
\subsection{\code{PMIx_Unpublish_nb}}
\declareapi{PMIx_Unpublish_nb}

%%%%
\summary

Nonblocking version of \refapi{PMIx_Unpublish}.

%%%%
\format

\cspecificstart
\begin{codepar}
pmix_status_t PMIx_Unpublish_nb(char **keys,
                                const pmix_info_t info[], size_t ninfo,
                                pmix_op_cbfunc_t cbfunc, void *cbdata)
\end{codepar}
\cspecificend

\begin{arglist}
\argin{keys}{(array of strings)}
\argin{info}{Array of info structures (array of handles)}
\argin{ninfo}{Number of element in the \refarg{info} array (integer)}
\argin{cbfunc}{Callback function \refapi{pmix_op_cbfunc_t} (function reference)}
\argin{cbdata}{Data to be passed to the callback function (memory reference)}
\end{arglist}

Returns \refconst{PMIX_SUCCESS} or a negative value corresponding to a PMIx error constant.

%%%%
\descr

Non-blocking form of the \refapi{PMIx_Unpublish} function.
The callback function will be executed once the server confirms removal of the specified data.



%%%%%%%%%%%
\section{Process Management}
\label{chap:api_client:processmgmt}

\ldots

%%%%%%%%%%%
\subsection{\code{PMIx_Spawn}}
\declareapi{PMIx_Spawn}

%%%%
\summary

Spawn a new job.

%%%%
\format

\cspecificstart
\begin{codepar}
pmix_status_t PMIx_Spawn(const pmix_info_t job_info[], size_t ninfo,
                         const pmix_app_t apps[], size_t napps,
                         char nspace[])
\end{codepar}
\cspecificend

\begin{arglist}
\argin{job_info}{Array of info structures (array of handles)}
\argin{ninfo}{Number of elements in the \refarg{job_info} array (integer)}
\argin{apps}{Array of \refstruct{pmix_app_t} structures (array of handles)}
\argin{napps}{Number of elements in the \refarg{apps} array (integer)}
\argout{nspace}{Namespace of the new job (string)}
\end{arglist}

Returns \refconst{PMIX_SUCCESS} or a negative value corresponding to a PMIx error constant.

%%%%
\descr

Spawn a new job.
The assigned namespace of the spawned applications is returned in the \refarg{nspace} parameter.
A \code{NULL} value in that location indicates that the caller doesn't wish to have the namespace returned.
The \refarg{nspace} array must be at least of size one more than \refconst{PMIX_MAX_NSLEN}.
Behavior of individual resource managers may differ, but it is expected that failure of any application process to start will result in termination/cleanup of \emph{all} processes in the newly spawned job and return of an error code to the caller.

By default, the spawned processes will be PMIx ``connected'' to the parent process upon successful launch (see \refapi{PMIx_Connect} description for details).
Note that this only means that the parent process (a) will be given a copy of the new job's
information so it can query job-level info without incurring any communication penalties, and (b) will receive notification of errors from process in the child job.

Job-level directives can be specified in the \refarg{job_info} array.
This can include:
\begin{attributedesc}
%
\declareattritem{PMIX_NON_PMI} (string)
Processes in the spawned job will not be calling \refapi{PMIx_Init}.
%
\declareattritem{PMIX_TIMEOUT} (string)
Declare the spawn as having failed if the launched processes do not call \refapi{PMIx_Init} within the specified time.
%
\declareattritem{PMIX_NOTIFY_COMPLETION} (string)
Notify the parent process when the child job terminates, either normally or with error.
%
\end{attributedesc}


%%%%%%%%%%%
\subsection{\code{PMIx_Spawn_nb}}
\declareapi{PMIx_Spawn_nb}

%%%%
\summary

Nonblocking version of the \refapi{PMIx_Spawn} routine.

%%%%
\format

\cspecificstart
\begin{codepar}
pmix_status_t PMIx_Spawn_nb(const pmix_info_t job_info[], size_t ninfo,
                            const pmix_app_t apps[], size_t napps,
                            pmix_spawn_cbfunc_t cbfunc, void *cbdata)
\end{codepar}
\cspecificend

\begin{arglist}
\argin{job_info}{Array of info structures (array of handles)}
\argin{ninfo}{Number of elements in the \refarg{job_info} array (integer)}
\argin{apps}{Array of \refstruct{pmix_app_t} structures (array of handles)}
\argin{cbfunc}{Callback function \refapi{pmix_spawn_cbfunc_t} (function reference)}
\argin{cbdata}{Data to be passed to the callback function (memory reference)}
\end{arglist}

Returns \refconst{PMIX_SUCCESS} or a negative value corresponding to a PMIx error constant.

%%%%
\descr

Nonblocking version of the \refapi{PMIx_Spawn} routine.


%%%%%%%%%%%
\subsection{\code{PMIx_Connect}}
\declareapi{PMIx_Connect}

%%%%
\summary

Connect namespaces.

%%%%
\format

\cspecificstart
\begin{codepar}
pmix_status_t PMIx_Connect(const pmix_proc_t procs[], size_t nprocs,
                           const pmix_info_t info[], size_t ninfo)
\end{codepar}
\cspecificend

\begin{arglist}
\argin{procs}{Array of proc structures (array of handles)}
\argin{nprocs}{Number of elements in the \refarg{procs} array (integer)}
\argin{info}{Array of info structures (array of handles)}
\argin{ninfo}{Number of elements in the \refarg{info} array (integer)}
\end{arglist}

Returns \refconst{PMIX_SUCCESS} or a negative value corresponding to a PMIx error constant.

%%%%
\descr

Record the specified processes as ``connected''.
This means that the resource manager should treat the failure of any process in the specified group as a reportable event, and take appropriate action.
Note that different resource managers may respond to failures in different manners.

The callback function is to be called once all participating processes have called connect.
The server is required to return any job-level info for the connecting processes that might not already have (i.e., if the connect request involves \refarg{procs} from different namespaces, then each \refarg{proc} shall receive the job-level info from those namespaces other than their own.

A process can only engage in \emph{one} connect operation involving the identical set of processes at a time.
However, a process \emph{can} be simultaneously engaged in multiple connect operations, each involving a different set of processes.

As in the case of the fence operation, the info array can be used to pass user-level directives regarding the algorithm to be used for the collective operation involved in the ``connect'', timeout constraints, and other options available from the host RM.


%%%%%%%%%%%
\subsection{\code{PMIx_Connect_nb}}
\declareapi{PMIx_Connect_nb}

%%%%
\summary

Nonblocking \refapi{PMIx_Connect_nb} routine.

%%%%
\format

\cspecificstart
\begin{codepar}
pmix_status_t PMIx_Connect_nb(const pmix_proc_t procs[], size_t nprocs,
                              const pmix_info_t info[], size_t ninfo,
                              pmix_op_cbfunc_t cbfunc, void *cbdata)
\end{codepar}
\cspecificend

\begin{arglist}
\argin{procs}{Array of proc structures (array of handles)}
\argin{nprocs}{Number of elements in the \refarg{procs} array (integer)}
\argin{info}{Array of info structures (array of handles)}
\argin{ninfo}{Number of element in the \refarg{info} array (integer)}
\argin{cbfunc}{Callback function \refapi{pmix_op_cbfunc_t} (function reference)}
\argin{cbdata}{Data to be passed to the callback function (memory reference)}
\end{arglist}

Returns \refconst{PMIX_SUCCESS} or a negative value corresponding to a PMIx error constant.

%%%%
\descr

Nonblocking \refapi{PMIx_Connect_nb} routine.


%%%%%%%%%%%
\subsection{\code{PMIx_Disconnect}}
\declareapi{PMIx_Disconnect}

%%%%
\summary

Disconnect a previously connected set of processes.

%%%%
\format

\cspecificstart
\begin{codepar}
pmix_status_t PMIx_Disconnect(const pmix_proc_t procs[], size_t nprocs,
                              const pmix_info_t info[], size_t ninfo);
\end{codepar}
\cspecificend

\begin{arglist}
\argin{procs}{Array of proc structures (array of handles)}
\argin{nprocs}{Number of elements in the \refarg{procs} array (integer)}
\argin{info}{Array of info structures (array of handles)}
\argin{ninfo}{Number of element in the \refarg{info} array (integer)}
\end{arglist}

Returns \refconst{PMIX_SUCCESS} or a negative value corresponding to a PMIx error constant.

%%%%
\descr

Disconnect a previously connected set of processes.
An error will be returned if the specified set of \refarg{procs} was not previously ``connected''.
As with \refapi{PMIx_Connect}, a process may be involved in multiple simultaneous disconnect operations.
However, a process is not allowed to reconnect to a set of \refarg{procs} that has not fully completed disconnect (i.e., you have to fully disconnect before you can reconnect to the \emph{same} group of processes.
The \refarg{info} array is used as in \refapi{PMIx_Connect}.


%%%%%%%%%%%
\subsection{\code{PMIx_Disconnect_nb}}
\declareapi{PMIx_Disconnect_nb}

%%%%
\summary

Nonblocking \refapi{PMIx_Disconnect} routine.

%%%%
\format

\cspecificstart
\begin{codepar}
pmix_status_t PMIx_Disconnect_nb(const pmix_proc_t ranges[], size_t nprocs,
                                 const pmix_info_t info[], size_t ninfo,
                                 pmix_op_cbfunc_t cbfunc, void *cbdata);
\end{codepar}
\cspecificend

\begin{arglist}
\argin{procs}{Array of proc structures (array of handles)}
\argin{nprocs}{Number of elements in the \refarg{procs} array (integer)}
\argin{info}{Array of info structures (array of handles)}
\argin{ninfo}{Number of element in the \refarg{info} array (integer)}
\argin{cbfunc}{Callback function \refapi{pmix_op_cbfunc_t} (function reference)}
\argin{cbdata}{Data to be passed to the callback function (memory reference)}
\end{arglist}

Returns \refconst{PMIX_SUCCESS} or a negative value corresponding to a PMIx error constant.

%%%%
\descr

Nonblocking \refapi{PMIx_Disconnect} routine.


%%%%%%%%%%%
\subsection{\code{PMIx_Resolve_peers}}
\declareapi{PMIx_Resolve_peers}

%%%%
\summary

Access an array of processes within the specified namespace on a node.

%%%%
\format

\cspecificstart
\begin{codepar}
pmix_status_t PMIx_Resolve_peers(const char *nodename, const char *nspace,
                                 pmix_proc_t **procs, size_t *nprocs)
\end{codepar}
\cspecificend

\begin{arglist}
\argin{nodename}{Name of the node to query (string)}
\argin{nspace}{namespace (string)}
\argout{procs}{Array of process structures (array of handles)}
\argout{nprocs}{Number of elements in the \refarg{procs} array (integer)}
\end{arglist}

Returns \refconst{PMIX_SUCCESS} or a negative value corresponding to a PMIx error constant.

%%%%
\descr

Given a \refarg{nodename}, return an array of processes within the specified \refarg{nspace}
on that node.
If the \refarg{nspace} is \code{NULL}, then all processes on the node will be returned.
If the specified node does not currently host any processes, then the returned array will be \code{NULL}, and \refarg{nprocs} will be \code{0}.
The caller is responsible for releasing the \refarg{procs} array when done with it.
The \refapi{PMIX_PROC_FREE} macro is provided for this purpose.



%%%%%%%%%%%
\subsection{\code{PMIx_Resolve_nodes}}
\declareapi{PMIx_Resolve_nodes}

%%%%
\summary

Return a list of nodes hosting processes.

%%%%
\format

\cspecificstart
\begin{codepar}
pmix_status_t PMIx_Resolve_nodes(const char *nspace, char **nodelist)
\end{codepar}
\cspecificend

\begin{arglist}
\argin{nspace}{Namespace (string)}
\argout{nodelist}{Comma-delimited list of nodenames (string)}
\end{arglist}

Returns \refconst{PMIX_SUCCESS} or a negative value corresponding to a PMIx error constant.

%%%%
\descr

Given a \refarg{nspace}, return the list of nodes hosting processes within that namespace.
The returned string will contain a comma-delimited list of nodenames.
The caller is responsible for releasing the string when done with it.


%%%%%%%%%%%
\subsection{\code{PMIx_Query_info_nb}}
\declareapi{PMIx_Query_info_nb}
\declareapi{pmix_info_cbfunc_t}

%%%%
\summary

Query information about the system in general.

%%%%
\format

\cspecificstart
\begin{codepar}
typedef void (*pmix_info_cbfunc_t)(pmix_status_t status,
                                   pmix_info_t *info, size_t ninfo,
                                   void *cbdata,
                                   pmix_release_cbfunc_t release_fn,
                                   void *release_cbdata);

pmix_status_t PMIx_Query_info_nb(pmix_query_t queries[], size_t nqueries,
                                 pmix_info_cbfunc_t cbfunc, void *cbdata)
\end{codepar}
\cspecificend

\begin{arglist}
\argin{queries}{Array of query structures (array of handles)}
\argin{nqueries}{Number of elements in the \refarg{queries} array (integer)}
\argin{cbfunc}{Callback function \refapi{pmix_info_cbfunc_t} (function reference)}
\argin{cbdata}{Data to be passed to the callback function (memory reference)}
\end{arglist}

\begin{constantdesc}
\item \refconst{PMIX_SUCCESS} All data has been returned
\item \refconst{PMIX_ERR_NOT_FOUND} None of the requested data was available
\item \refconst{PMIX_ERR_PARTIAL_SUCCESS} Some of the data has been returned
\item \refconst{PMIX_ERR_NOT_SUPPORTED} The host \ac{RM} does not support this function
\end{constantdesc}

%%%%
\descr

Query information about the system in general.
This can include a list of active namespaces, network topology, etc.
Also can be used to query node-specific info such as the list of peers executing on a given node.
We assume that the host \ac{RM} will exercise appropriate access control on the information.

NOTE: There is no blocking form of this API as the structures passed to query info differ from those for receiving the results.

The \refarg{status} argument to the callback function indicates if requested data was found or not.
An array of \refstruct{pmix_info_t} will contain the key/value pairs.


%%%%%%%%%%%
\subsection{\code{PMIx_Log_nb}}
\declareapi{PMIx_Log_nb}

%%%%
\summary

Log data to a data service.

%%%%
\format

\cspecificstart
\begin{codepar}
pmix_status_t PMIx_Log_nb(const pmix_info_t data[], size_t ndata,
                          const pmix_info_t directives[], size_t ndirs,
                          pmix_op_cbfunc_t cbfunc, void *cbdata)
\end{codepar}
\cspecificend

\begin{arglist}
\argin{data}{Array of info structures (array of handles)}
\argin{ndata}{Number of elements in the \refarg{data} array (integer)}
\argin{directives}{Array of info structures (array of handles)}
\argin{ndirs}{Number of elements in the \refarg{directives} array (integer)}
\argin{cbfunc}{Callback function \refapi{pmix_op_cbfunc_t} (function reference)}
\argin{cbdata}{Data to be passed to the callback function (memory reference)}
\end{arglist}

Returns \refconst{PMIX_SUCCESS} or a negative value corresponding to a PMIx error constant.

%%%%
\descr

Log data to a ``central'' data service/store, subject to the services offered by the host resource manager.
The data to be logged is provided in the \refarg{data} array.
The (optional) \refarg{directives} can be used to request specific storage options and direct the choice of storage option.

The callback function will be executed when the log operation has been completed.
The \refarg{data} array must be maintained until the callback is provided.


%%%%%%%%%%%
\section{Job Allocation Management}
\label{chap:api_client:allocationmgmt}

\ldots


%%%%%%%%%%%
\subsection{\code{PMIx_Allocation_request_nb}}
\declareapi{PMIx_Allocation_request_nb}

%%%%
\summary

Request an allocation operation from the host resource manager.

%%%%
\format

\cspecificstart
\begin{codepar}
pmix_status_t PMIx_Allocation_request_nb(pmix_alloc_directive_t directive,
                                         pmix_info_t *info, size_t ninfo,
                                         pmix_info_cbfunc_t cbfunc, void *cbdata);
\end{codepar}
\cspecificend

\begin{arglist}
\argin{directive}{Allocation directive (handle)}
\argin{info}{Array of info structures (array of handles)}
\argin{ninfo}{Number of elements in the \refarg{info} array (integer)}
\argin{cbfunc}{Callback function \refapi{pmix_info_cbfunc_t} (function reference)}
\argin{cbdata}{Data to be passed to the callback function (memory reference)}
\end{arglist}

Returns \refconst{PMIX_SUCCESS} or a negative value corresponding to a PMIx error constant.

%%%%
\descr

Request an allocation operation from the host resource manager.
Several broad categories are envisioned, including the ability to:

\begin{compactitem}
%
\item Request allocation of additional resources, including memory, bandwidth, and compute.
This should be accomplished in a non-blocking manner so that the application can continue to progress while waiting for resources to become available.
Note that the new allocation will be disjoint from (i.e., not affiliated with) the allocation of the requestor - thus the termination of one allocation will not impact the other.
%
\item Extend the reservation on currently allocated resources, subject to scheduling availability and priorities.
This includes extending the time limit on current resources, and/or requesting additional resources be allocated to the requesting job.
Any additional allocated resources will be considered as part of the current allocation, and thus will be released at the same time.
%
\item Release currently allocated resources that are no longer required.
This is intended to support partial release of resources since all resources are normally released upon termination of the job.
The identified use-cases include resource variations across discrete steps of a workflow, as well as applications that spawn sub-jobs and/or dynamically grow/shrink over time.
%
\item ``Lend'' resources back to the scheduler with an expectation of getting them back at some later time in the job.
This can be a proactive operation (e.g., to save on computing costs when resources are temporarily not required), or in response to scheduler requests in lieue of preemption.
A corresponding ability to ``reacquire'' resources previously released is included.
%
\end{compactitem}


%%%%%%%%%%%
\subsection{\code{PMIx_Job_control_nb}}
\declareapi{PMIx_Job_control_nb}

%%%%
\summary

Request a job control action.

%%%%
\format

\cspecificstart
\begin{codepar}
pmix_status_t PMIx_Job_control_nb(const pmix_proc_t targets[], size_t ntargets,
                                  const pmix_info_t directives[], size_t ndirs,
                                  pmix_info_cbfunc_t cbfunc, void *cbdata)
\end{codepar}
\cspecificend

\begin{arglist}
\argin{targets}{Array of proc structures (array of handles)}
\argin{ntargets}{Number of element in the \refarg{targets} array (integer)}
\argin{directives}{Array of info structures (array of handles)}
\argin{ndirs}{Number of element in the \refarg{directives} array (integer)}
\argin{cbfunc}{Callback function \refapi{pmix_info_cbfunc_t} (function reference)}
\argin{cbdata}{Data to be passed to the callback function (memory reference)}
\end{arglist}

Returns \refconst{PMIX_SUCCESS} or a negative value corresponding to a PMIx error constant.

%%%%
\descr

Request a job control action.
The \refarg{targets} array identifies the processes to which the requested job control action is to be applied.
A \code{NULL} value can be used to indicate all processes in the caller's namespace.
The use of \refconst{PMIX_RANK_WILDARD} can also be used to indicate that all processes in the given namespace are to be included.

The directives are provided as \refstruct{pmix_info_t} structures in the \refarg{directives} array.
The callback function provides a \refarg{status} to indicate whether or not the request was granted, and to provide some information as to the reason for any denial in the \refapi{pmix_info_cbfunc_t} array of \refstruct{pmix_info_t} structures.
If non-\code{NULL}, then the specified \refarg{release_fn} must be called when the callback function completes - this will be used to release any provided \refstruct{pmix_info_t} array.


%%%%%%%%%%%
\subsection{\code{PMIx_Process_monitor_nb}}
\declareapi{PMIx_Process_monitor_nb}

%%%%
\summary

Request that something be monitored.

%%%%
\format

\cspecificstart
\begin{codepar}
pmix_status_t PMIx_Process_monitor_nb(const pmix_info_t *monitor, pmix_status_t error,
                                      const pmix_info_t directives[], size_t ndirs,
                                      pmix_info_cbfunc_t cbfunc, void *cbdata)
\end{codepar}
\cspecificend

\begin{arglist}
\argin{monitor}{info (handle)}
\argin{error}{status (integer)}
\argin{directives}{Array of info structures (array of handles)}
\argin{ndirs}{Number of elements in the \refarg{directives} array (integer)}
\argin{cbfunc}{Callback function \refapi{pmix_info_cbfunc_t} (function reference)}
\argin{cbdata}{Data to be passed to the callback function (memory reference)}
\end{arglist}

Returns \refconst{PMIX_SUCCESS} or a negative value corresponding to a PMIx error constant.

%%%%
\descr

Request that something be monitored.
For example, that the server monitor this process for periodic heartbeats as an indication that the process has not become ``wedged''.
When a monitor detects the specified alarm condition, it will generate an event notification using the provided error code and passing along any available relevant information.
It is up to the caller to register a corresponding event handler.

The \refarg{monitor} argument is an attribute indicating the type of monitor being requested.
For example, \refattr{PMIX_MONITOR_FILE} to indicate that the requestor is asking that a file be monitored.

The \refarg{error} argument is the status code to be used when generating an event notification alerting that the monitor has been triggered.
The range of the notification defaults to \refconst{PMIX_RANGE_NAMESPACE}.
This can be changed by providing a \refconst{PMIX_RANGE} directive.

The \refarg{directives} argument characterizes the monitoring request (e.g., monitor file size) and frequency of checking to be done

The \refarg{cbfunc} function provides a \refarg{status} to indicate whether or not the request was granted, and to provide some information as to the reason for any denial in the \refapi{pmix_info_cbfunc_t} array of \refstruct{pmix_info_t} structures.


%%%%%%%%%%%
\subsection{\code{PMIx_Heartbeat}}
\declareapi{PMIx_Heartbeat}

%%%%
\summary

Send a heartbeat to the \ac{RM}

%%%%
\format

\cspecificstart
\begin{codepar}
void PMIx_Heartbeat(void)
\end{codepar}
\cspecificend


%%%%
\descr

A simplified version of \refapi{PMIx_Process_monitor_nb} that sends a heartbeat to the \ac{RM}.

%%%%%%%%%%%%%%%%%%%%%%%%%%%%%%%%%%%%%%%%%%%%%%%%%

    %%%%%%%%%%%%%%%%%%%%%%%%%%%%%%%%%%%%%%%%%%%%%%%%%
% Chapter: API Server
%%%%%%%%%%%%%%%%%%%%%%%%%%%%%%%%%%%%%%%%%%%%%%%%%
\chapter{Server Specific Interfaces}
\label{chap:api_server}

\ldots


%%%%%%%%%%%
\subsection{\code{PMIx_generate_regex}}
\declareapi{PMIx_generate_regex}

%%%%
\summary

Generate a regular expression representation of the input string.

%%%%
\format

\cspecificstart
\begin{codepar}
pmix_status_t PMIx_generate_regex(const char *input, char **regex)
\end{codepar}
\cspecificend

\begin{arglist}
\argin{input}{String to process (string)}
\argout{regex}{Regular expression representation of \refarg{input} (string)}
\end{arglist}

Returns \refconst{PMIX_SUCCESS} or a negative value corresponding to a PMIx error constant.

%%%%
\descr

Given a semicolon-separated list of \refarg{input} values, generate a regular expression that can be passed down to the \ac{PMIx} client for parsing.
The caller is responsible for free'ing the resulting string.

If values have leading zero's, then that is preserved.
You have to add back any prefix/suffix for node names.

% JJH Format this
% * If values have leading zero's, then that is preserved. You
% * have to add back any prefix/suffix for node names
% * odin[009-015,017-023,076-086]
% *
% *     "pmix:odin[009-015,017-023,076-086]"
% *
% * Note that the "pmix" at the beginning of each regex indicates
% * that the PMIx native parser is to be used by the client for
% * parsing the provided regex. Other parsers may be supported - see
% * the pmix_client.h header for a list.


%%%%%%%%%%%
\subsection{\code{PMIx_generate_ppn}}
\declareapi{PMIx_generate_ppn}

%%%%
\summary

Generate a regular expression representation of the input string.

%%%%
\format

\cspecificstart
\begin{codepar}
pmix_status_t PMIx_generate_ppn(const char *input, char **ppn)
\end{codepar}
\cspecificend

\begin{arglist}
\argin{input}{String to process (string)}
\argout{regex}{Regular expression representation of \refarg{input} (string)}
\end{arglist}

Returns \refconst{PMIX_SUCCESS} or a negative value corresponding to a PMIx error constant.

%%%%
\descr

The input is expected to consist of a comma-separated list of ranges.

% JJH Format this
% * of ranges. Thus, an input of:
% *     "1-4;2-5;8,10,11,12;6,7,9"
% * would generate a regex of
% *     "[pmix:2x(3);8,10-12;6-7,9]"
% *
% * Note that the "pmix" at the beginning of each regex indicates
% * that the PMIx native parser is to be used by the client for
% * parsing the provided regex. Other parsers may be supported - see
% * the pmix_client.h header for a list.
% */


%%%%%%%%%%%
\subsection{\code{PMIx_server_register_nspace}}
\declareapi{PMIx_server_register_nspace}

%%%%
\summary

Setup the data about a particular namespace so it can be passed to any child process upon startup.

%%%%
\format

\cspecificstart
\begin{codepar}
pmix_status_t PMIx_server_register_nspace(const char nspace[], int nlocalprocs,
                                          pmix_info_t info[], size_t ninfo,
                                          pmix_op_cbfunc_t cbfunc, void *cbdata)
\end{codepar}
\cspecificend

\begin{arglist}
\argin{nspace}{namespace (string)}
\argin{nlocalprocs}{number of local processes (integer)}
\argin{info}{Array of info structures (array of handles)}
\argin{ninfo}{Number of elements in the \refarg{info} array (integer)}
\argin{cbfunc}{Callback function \refapi{pmix_op_cbfunc_t} (function reference)}
\argin{cbdata}{Data to be passed to the callback function (memory reference)}
\end{arglist}

Returns \refconst{PMIX_SUCCESS} or a negative value corresponding to a PMIx error constant.

%%%%
\descr

The PMIx connection procedure provides an opportunity for the host PMIx server to pass job-related info down to a child process.
This might include the number of processes in the job, relative local ranks of the processes within the job, and other information of use to the process.
The server is free to determine which, if any, of the supported elements it will provide (See \refsection{chap:struct}{Data Structures and Types} for values).

The PMIx server must register \emph{all} namespaces that will participate in collective operations with local processes.
This means that the server must register a namespace even if it will not host any local procs from within that nspace \emph{if} any local process might at some point perform a collective operation involving one or more processes from that namespace.
This is necessary so that the collective operation can know when it is locally complete.

The caller must also provide the number of local processes that will be launched within this namespace.
This is required for the PMIx server library to correctly handle collectives as a collective operation call can occur before all the processes have been started.


%%%%%%%%%%%
\subsection{\code{PMIx_server_deregister_nspace}}
\declareapi{PMIx_server_deregister_nspace}

%%%%
\summary

Deregister a namespace.

%%%%
\format

\cspecificstart
\begin{codepar}
void PMIx_server_deregister_nspace(const char nspace[],
                                   pmix_op_cbfunc_t cbfunc, void *cbdata)
\end{codepar}
\cspecificend

\begin{arglist}
\argin{nspace}{Namespace (string)}
\argin{cbfunc}{Callback function \refapi{pmix_op_cbfunc_t} (function reference)}
\argin{cbdata}{Data to be passed to the callback function (memory reference)}
\end{arglist}

%%%%
\descr

Deregister the specified \refarg{nspace} and purge all objects relating to it, including any client information from that namespace.
This is intended to support persistent PMIx servers by providing an opportunity for the host \ac{RM} to tell the PMIx server library to release all memory for a completed job.



%%%%%%%%%%%
\subsection{\code{PMIx_server_register_client}}
\declareapi{PMIx_server_register_client}

%%%%
\summary

Register a client process with the PMIx server library.

%%%%
\format

\cspecificstart
\begin{codepar}
pmix_status_t PMIx_server_register_client(const pmix_proc_t *proc,
                                          uid_t uid, gid_t gid,
                                          void *server_object,
                                          pmix_op_cbfunc_t cbfunc, void *cbdata)
\end{codepar}
\cspecificend

\begin{arglist}
\argin{proc}{\refstruct{pmix_proc_t} structure (handle)}
\argin{uid}{user id (integer)}
\argin{gid}{group id (integer)}
\argin{server_object}{(memory reference)}
\argin{cbfunc}{Callback function \refapi{pmix_op_cbfunc_t} (function reference)}
\argin{cbdata}{Data to be passed to the callback function (memory reference)}
\end{arglist}

Returns \refconst{PMIX_SUCCESS} or a negative value corresponding to a PMIx error constant.

%%%%
\descr

Register a client process with the PMIx server library.
The expected user ID and group ID of the child process helps the server library to properly authenticate clients as they connect by requiring the two values to match.

The host server can also, if it desires, provide an object it wishes to be returned when a server function is called that relates to a specific process.
For example, the host server may have an object that tracks the specific client.
Passing the object to the library allows the library to provide that object to the host server during subsequent calls related to that client, such as a ``pmix_server_client_connected_fn'' function.  This allows the host server to access the object without performing a lookup based the client's namespace and rank.


%%%%%%%%%%%
\subsection{\code{PMIx_server_deregister_client}}
\declareapi{PMIx_server_deregister_client}

%%%%
\summary

Deregister a client and purge all data relating to it.

%%%%
\format

\cspecificstart
\begin{codepar}
void PMIx_server_deregister_client(const pmix_proc_t *proc,
                                   pmix_op_cbfunc_t cbfunc, void *cbdata)
\end{codepar}
\cspecificend

\begin{arglist}
\argin{proc}{\refstruct{pmix_proc_t} structure (handle)}
\argin{cbfunc}{Callback function \refapi{pmix_op_cbfunc_t} (function reference)}
\argin{cbdata}{Data to be passed to the callback function (memory reference)}
\end{arglist}


%%%%
\descr

The \refapi{PMIx_server_deregister_nspace} API will automatically delete all client information for that namespace.
This API is therefore intended solely for use in exception cases.


%%%%%%%%%%%
\subsection{\code{PMIx_server_setup_fork}}
\declareapi{PMIx_server_setup_fork}

%%%%
\summary

Setup the environment of a child process to be forked by the host.

%%%%
\format

\cspecificstart
\begin{codepar}
pmix_status_t PMIx_server_setup_fork(const pmix_proc_t *proc, char ***env)
\end{codepar}
\cspecificend

\begin{arglist}
\argin{proc}{\refstruct{pmix_proc_t} structure (handle)}
\argin{env}{Environment array (array of strings)}
\end{arglist}

Returns \refconst{PMIX_SUCCESS} or a negative value corresponding to a PMIx error constant.

%%%%
\descr

Setup the environment of a child process to be forked by the host so it can correctly interact with the PMIx server.
The PMIx client needs some setup information so it can properly connect back to the server.
This function will set appropriate environmental variables for this purpose.


%%%%%%%%%%%
\subsection{\code{PMIx_server_dmodex_request}}
\declareapi{PMIx_server_dmodex_request}
\declareapi{pmix_dmodex_response_fn_t}

%%%%
\summary

Define a function by which the host server can request modex data from the local PMIx server.

%%%%
\format

\cspecificstart
\begin{codepar}
typedef void (*pmix_dmodex_response_fn_t)(pmix_status_t status,
                                          char *data, size_t sz,
                                          void *cbdata);

pmix_status_t PMIx_server_dmodex_request(const pmix_proc_t *proc,
                                         pmix_dmodex_response_fn_t cbfunc,
                                         void *cbdata)
\end{codepar}
\cspecificend

\begin{arglist}
\argin{proc}{\refstruct{pmix_proc_t} structure (handle)}
\argin{cbfunc}{Callback function \refapi{pmix_dmodex_response_fn_t} (function reference)}
\argin{cbdata}{Data to be passed to the callback function (memory reference)}
\end{arglist}

Returns \refconst{PMIX_SUCCESS} or a negative value corresponding to a PMIx error constant.

%%%%
\descr

Define a function by which the host server can request modex data from the local PMIx server.
This is used to support the direct modex operation (i.e., where data is cached locally on each PMIx server for its own local clients, and is obtained on-demand for remote requests.
Upon receiving a request from a remote server, the host server will call this function to pass the request into the PMIx server.
The PMIx server will return a blob (once it becomes available) via the \refarg{cbfunc} - the host server shall send the blob back to the original requestor.

The callback function used by the PMIx server to return direct modex requests to the host server.
The PMIx server will free the data blob upon return from the response function.


%%%%%%%%%%%
\subsection{\code{PMIx_server_setup_application}}
\declareapi{PMIx_server_setup_application}
\declareapi{pmix_setup_application_cbfunc_t}

%%%%
\summary

Provide a function by which the resource manager can request any application-specific environmental variables prior to launch of an application.
 
%%%%
\format

\cspecificstart
\begin{codepar}
typedef void (*pmix_setup_application_cbfunc_t)(pmix_status_t status,
                                                pmix_info_t info[], size_t ninfo,
                                                void *provided_cbdata,
                                                pmix_op_cbfunc_t cbfunc, void *cbdata)

pmix_status_t PMIx_server_setup_application(const char nspace[],
                                            pmix_info_t info[], size_t ninfo,
                                            pmix_setup_application_cbfunc_t cbfunc,
                                            void *cbdata)
\end{codepar}
\cspecificend

\begin{arglist}
\argin{nspace}{namespace (string)}
\argin{info}{Array of info structures (array of handles)}
\argin{ninfo}{Number of elements in the \refarg{info} array (integer)}
\argin{cbfunc}{Callback function \refapi{pmix_setup_application_cbfunc_t} (function reference)}
\argin{cbdata}{Data to be passed to the callback function (memory reference)}
\end{arglist}

Returns \refconst{PMIX_SUCCESS} or a negative value corresponding to a PMIx error constant.

%%%%
\descr

Provide a function by which the resource manager can request any application-specific environmental variables prior to launch of an application.
For example, network libraries may opt to provide security credentials for the application.
This is defined as a non-blocking operation in case network libraries need to perform some action before responding.
The returned env will be distributed along with the application

In the callback function, the returned \refarg{info} array is owned by the PMIx server library and will be free'd when the provided \refarg{cbfunc} is called.


%%%%%%%%%%%
\subsection{\code{PMIx_server_setup_local_support}}
\declareapi{PMIx_server_setup_local_support}

%%%%
\summary

Provide a function by which the local PMIx server can perform any application-specific operations prior to spawning local clients of a given application.

%%%%
\format

\cspecificstart
\begin{codepar}
pmix_status_t PMIx_server_setup_local_support(const char nspace[],
                                              pmix_info_t info[], size_t ninfo,
                                              pmix_op_cbfunc_t cbfunc, void *cbdata);
\end{codepar}
\cspecificend

\begin{arglist}
\argin{nspace}{Namespace (string)}
\argin{info}{Array of info structures (array of handles)}
\argin{ninfo}{Number of elements in the \refarg{info} array (integer)}
\argin{cbfunc}{Callback function \refapi{pmix_op_cbfunc_t} (function reference)}
\argin{cbdata}{Data to be passed to the callback function (memory reference)}
\end{arglist}

Returns \refconst{PMIX_SUCCESS} or a negative value corresponding to a PMIx error constant.

%%%%
\descr

Provide a function by which the local PMIx server can perform any application-specific operations prior to spawning local clients of a given application.
For example, a network library might need to setup the local driver for ``instant on'' addressing.


%%%%%%%%%%%
\section{Server Function Pointers}

The PMIx Server will set the function pointers in the \refapi{pmix_server_module_t} structure that they then pass to \refapi{PMIx_server_init}.
That module structure and associated function references is defined in this section.

%%%%%%%%%%%
\subsection{\code{pmix_server_module_t} Module}
\declareapi{pmix_server_module_t}

%%%%
\summary

List of function pointers that a PMIx server passes to \refapi{PMIx_server_init} during startup.

%%%%
\format

\cspecificstart
\begin{codepar}
typedef struct pmix_server_module_2_0_0_t {
    /* v1x interfaces */
    pmix_server_client_connected_fn_t   client_connected;
    pmix_server_client_finalized_fn_t   client_finalized;
    pmix_server_abort_fn_t              abort;
    pmix_server_fencenb_fn_t            fence_nb;
    pmix_server_dmodex_req_fn_t         direct_modex;
    pmix_server_publish_fn_t            publish;
    pmix_server_lookup_fn_t             lookup;
    pmix_server_unpublish_fn_t          unpublish;
    pmix_server_spawn_fn_t              spawn;
    pmix_server_connect_fn_t            connect;
    pmix_server_disconnect_fn_t         disconnect;
    pmix_server_register_events_fn_t    register_events;
    pmix_server_deregister_events_fn_t  deregister_events;
    pmix_server_listener_fn_t           listener;
    /* v2x interfaces */
    pmix_server_notify_event_fn_t       notify_event;
    pmix_server_query_fn_t              query;
    pmix_server_tool_connection_fn_t    tool_connected;
    pmix_server_log_fn_t                log;
    pmix_server_alloc_fn_t              allocate;
    pmix_server_job_control_fn_t        job_control;
    pmix_server_monitor_fn_t            monitor;
} pmix_server_module_t;
\end{codepar}
\cspecificend

%%%%
\descr

NOTE: for performance purposes, the host server is required to return as quickly as possible from all functions.
Execution of the function is thus to be done asynchronously so as to allow the PMIx server support library to handle multiple client requests as quickly and scalably as possible.

All data passed to the host server functions is ``owned'' by the PMIX server support library and MUST NOT be free'd.
Data returned by the host server via callback function is owned by the host server, which is free to release it upon return from the callback.



%%%%%%%%%%%
\subsection{\code{pmix_server_client_connected_fn_t}}
\declareapi{pmix_server_client_connected_fn_t}

%%%%
\summary

Notify the host server that a client connected to this server.

%%%%
\format

\cspecificstart
\begin{codepar}
typedef pmix_status_t (*pmix_server_client_connected_fn_t)(
                             const pmix_proc_t *proc, void* server_object,
                             pmix_op_cbfunc_t cbfunc, void *cbdata)
\end{codepar}
\cspecificend

\begin{arglist}
\argin{proc}{\refstruct{pmix_proc_t} structure (handle)}
\argin{server_object}{object reference (memory reference)}
\argin{cbfunc}{Callback function \refapi{pmix_op_cbfunc_t} (function reference)}
\argin{cbdata}{Data to be passed to the callback function (memory reference)}
\end{arglist}

Returns \refconst{PMIX_SUCCESS} or a negative value corresponding to a PMIx error constant.

%%%%
\descr

Notify the host server that a client has called PMIx_Init or PMIx_Tool_init.
\rcomment{I am guessing a bit on whether PMIx_Tool_init causes a call to pmix_server_client_connected_fn_t}
Note that the client will be in a blocked state until the host server executes the callback function, thus allowing the PMIx server support library to release 
the client.  
The server_object parameter will be the value of the server_object parameter passed to   
\refapi{PMIx_server_register_client} previously by the host server.  If provided, an implementation of \refapi{pmix_server_client_connected_fn_t} 
is only required to
call the callback function designated.  A host server can choose to not be notified when clients connect by setting \refapi{client_connected} to \code{NULL}. 

It is possible that only a subset of the clients in a namespace call PMIx_init.   The server's \refapi{pmix_server_client_connected_fn_t} implemenation 
should not depend on being called once per rank in a namespace or delaying calling the callback function until all ranks have connected.  
However, if a rank makes any PMIx calls, it must first call \refapi{PMIx_Init} and 
therefore the server's \refapi{mpix_server_client_connected_fn_t} will be called before any other server functions specific to the rank.

\adviceimplstart
 The \refapi{PMIx_server_client_connected_fn_t} implementation provided in the \refapi{pmix_server_module_2_0_0_t} is an opportunity for a host server 
 to update the status of the ranks it manages.  It is also a convenient and well defined time to perform initialization necessary to 
 support further calls into the server related to that rank. 
 \adviceimplend

%%%%%%%%%%%
\subsection{\code{pmix_server_client_finalized_fn_t}}
\declareapi{pmix_server_client_finalized_fn_t}

%%%%
\summary

Notify the host server that a client called \refapi{PMIx_Finalize}.

%%%%
\format

\cspecificstart
\begin{codepar}
typedef pmix_status_t (*pmix_server_client_finalized_fn_t)(
                             const pmix_proc_t *proc, void* server_object,
                             pmix_op_cbfunc_t cbfunc, void *cbdata)
\end{codepar}
\cspecificend

\begin{arglist}
\argin{proc}{\refstruct{pmix_proc_t} structure (handle)}
\argin{server_object}{object reference (memory reference)}
\argin{cbfunc}{Callback function \refapi{pmix_op_cbfunc_t} (function reference)}
\argin{cbdata}{Data to be passed to the callback function (memory reference)}
\end{arglist}

Returns \refconst{PMIX_SUCCESS} or a negative value corresponding to a PMIx error constant.

%%%%
\descr

Notify the host server that a client called \refapi{PMIx_Finalize}.
Note that the client will be in a blocked state until the host server executes the callback function, thus allowing the PMIx server support library to release the client.
The server_object parameter will be the value of the server_object parameter passed to   
\refapi{PMIx_server_register_client} previously by the host server.  If provided, an implementation of \refapi{pmix_server_client_finalized_fn_t} 
is only required to
call the callback function designated.  A host server can choose to not be notified when clients finalize by setting \refapi{client_finalized} to \code{NULL}. 

Note that the host server is only being informed that the client has called \refapi{PMIx_Finalize}.  The client might not have exited.  If a client 
exits without calling \reefapi{PMIx_Finalize}, the server support library will not call the \refapi{PMIx_server_client_finalized_fn_t} implementation.

\adviceimplstart
 The \refapi{PMIx_server_client_finalized_fn_t} implementation provided in the \refapi{pmix_server_module_2_0_0_t} is an opportunity for a host server
 to update the status of the tasks it manages.  It is also a convenient and well defined time to release resources used to support that client.   
 \adviceimplend


%%%%%%%%%%%
\subsection{\code{pmix_server_abort_fn_t}}
\declareapi{pmix_server_abort_fn_t}

%%%%
\summary

Notify PMIx Server that a local client called \refapi{PMIx_Abort}.

%%%%
\format

\cspecificstart
\begin{codepar}
typedef pmix_status_t (*pmix_server_abort_fn_t)(
                             const pmix_proc_t *proc, void *server_object,
                             int status, const char msg[],
                             pmix_proc_t procs[], size_t nprocs,
                             pmix_op_cbfunc_t cbfunc, void *cbdata)
\end{codepar}
\cspecificend


\begin{arglist}
\argin{proc}{\refstruct{pmix_proc_t} structure (handle)}
\argin{server_object}{object reference (memory reference)}
\argin{status}{exit status (integer)}
\argin{msg}{exit status message (string)}
\argin{procs}{Array of \refstruct{pmix_proc_t} structures (array of handles)}
\argin{nprocs}{Number of elements in the \refarg{procs} array (integer)}
\argin{cbfunc}{Callback function \refapi{pmix_op_cbfunc_t} (function reference)}
\argin{cbdata}{Data to be passed to the callback function (memory reference)}
\end{arglist}

Returns \refconst{PMIX_SUCCESS} or a negative value corresponding to a PMIx error constant.

%%%%
\descr

A local client called \refapi{PMIx_Abort}.
Note that the client will be in a blocked state until the host server executes the callback function, thus allowing the PMIx server support library to release the client.
The array of \refarg{procs} indicates which processes are to be terminated.
A \code{NULL} indicates that all processes in the client's namespace are to be terminated.


%%%%%%%%%%%
\subsection{\code{pmix_server_fencenb_fn_t}}
\declareapi{pmix_server_fencenb_fn_t}

%%%%
\summary

At least one client called either \refapi{PMIx_Fence} or \refapi{PMIx_Fence_nb}.

%%%%
\format

\cspecificstart
\begin{codepar}
typedef pmix_status_t (*pmix_server_fencenb_fn_t)(
                             const pmix_proc_t procs[], size_t nprocs,
                             const pmix_info_t info[], size_t ninfo,
                             char *data, size_t ndata,
                             pmix_modex_cbfunc_t cbfunc, void *cbdata)
\end{codepar}
\cspecificend

\begin{arglist}
\argin{procs}{Array of \refstruct{pmix_proc_t} structures (array of handles)}
\argin{nprocs}{Number of elements in the \refarg{procs} array (integer)}
\argin{info}{Array of info structures (array of handles)}
\argin{ninfo}{Number of elements in the \refarg{info} array (integer)}
\argin{data}{(string)}
\argin{ndata}{(integer)}
\argin{cbfunc}{Callback function \refapi{pmix_modex_cbfunc_t} (function reference)}
\argin{cbdata}{Data to be passed to the callback function (memory reference)}
\end{arglist}

Returns \refconst{PMIX_SUCCESS} or a negative value corresponding to a PMIx error constant.

%%%%
\descr

At least one client called either \refapi{PMIx_Fence} or \refapi{PMIx_Fence_nb}.
In either case, the host server will be called via a non-blocking function to execute the specified operation once all participating local processes have contributed.
All processes in the specified \refarg{procs} array are required to participate in the \refapi{PMIx_Fence}/\refapi{PMIx_Fence_nb} operation.
The callback is to be executed once each daemon hosting at least one participant has called the host server's \refapi{pmix_server_fencenb_fn_t} function.

The provided data is to be collectively shared with all PMIx servers involved in the fence operation, and returned in the modex \refarg{cbfunc}.
A \code{NULL} data value indicates that the local processes had no data to contribute.

The array of \refarg{info} structs is used to pass user-requested options to the server.
This can include directives as to the algorithm to be used to execute the fence operation.
The directives are optional \emph{unless} the \emph{mandatory} flag has been set - in such cases, the host \ac{RM} is required to return an error if the directive cannot be met.


%%%%%%%%%%%
\subsection{\code{pmix_server_dmodex_req_fn_t}}
\declareapi{pmix_server_dmodex_req_fn_t}

%%%%
\summary

Used by the PMIx server to request its local host contact the PMIx server on the remote node that hosts the specified proc to obtain and return a direct modex blob for that proc.

%%%%
\format

\cspecificstart
\begin{codepar}
typedef pmix_status_t (*pmix_server_dmodex_req_fn_t)(
                             const pmix_proc_t *proc,
                             const pmix_info_t info[], size_t ninfo,
                             pmix_modex_cbfunc_t cbfunc, void *cbdata)
\end{codepar}
\cspecificend

\begin{arglist}
\argin{proc}{\refstruct{pmix_proc_t} structure (handle)}
\argin{info}{Array of info structures (array of handles)}
\argin{ninfo}{Number of elements in the \refarg{info} array (integer)}
\argin{cbfunc}{Callback function \refapi{pmix_modex_cbfunc_t} (function reference)}
\argin{cbdata}{Data to be passed to the callback function (memory reference)}
\end{arglist}

Returns \refconst{PMIX_SUCCESS} or a negative value corresponding to a PMIx error constant.

%%%%
\descr

Used by the PMIx server to request its local host contact the PMIx server on the remote node that hosts the specified proc to obtain and return a direct modex blob for that proc.

The array of \refarg{info} structs is used to pass user-requested options to the server.
This can include a timeout to preclude an indefinite wait for data that may never become available.
The directives are optional \emph{unless} the \emph{mandatory} flag has been set - in such cases, the host \ac{RM} is required to return an error if the directive cannot be met.


%%%%%%%%%%%
\subsection{\code{pmix_server_publish_fn_t}}
\declareapi{pmix_server_publish_fn_t}

%%%%
\summary

Publish data per the PMIx API specification.

%%%%
\format

\cspecificstart
\begin{codepar}
typedef pmix_status_t (*pmix_server_publish_fn_t)(
                             const pmix_proc_t *proc,
                             const pmix_info_t info[], size_t ninfo,
                             pmix_op_cbfunc_t cbfunc, void *cbdata)
\end{codepar}
\cspecificend

\begin{arglist}
\argin{proc}{\refstruct{pmix_proc_t} structure (handle)}
\argin{info}{Array of info structures (array of handles)}
\argin{ninfo}{Number of elements in the \refarg{info} array (integer)}
\argin{cbfunc}{Callback function \refapi{pmix_op_cbfunc_t} (function reference)}
\argin{cbdata}{Data to be passed to the callback function (memory reference)}
\end{arglist}

Returns \refconst{PMIX_SUCCESS} or a negative value corresponding to a PMIx error constant.

%%%%
\descr

Publish data per the PMIx API specification.
The callback is to be executed upon completion of the operation.
The default data range is expected to be \refconst{PMIX_SESSION}, and the default persistence \refconst{PMIX_PERSIST_SESSION}.
These values can be modified by including the respective \refstruct{pmix_info_t} struct in the \refarg{info} array.

Note that the host server is not required to guarantee support for any specific range - i.e., the server does not need to return an error if the data store doesn't support range-based isolation.
However, the server must return an error (a) if the key is duplicative within the storage range, and (b) if the server does not allow overwriting of published info by the original publisher - it is left to the discretion of the host server to allow info-key-based flags to modify this behavior.

The persistence indicates how long the server should retain the data.

The identifier of the publishing process is also provided and is expected to be returned on any subsequent lookup request.


%%%%%%%%%%%
\subsection{\code{pmix_server_lookup_fn_t}}
\declareapi{pmix_server_lookup_fn_t}

%%%%
\summary

Lookup published data.

%%%%
\format

\cspecificstart
\begin{codepar}
typedef pmix_status_t (*pmix_server_lookup_fn_t)(
                             const pmix_proc_t *proc, char **keys,
                             const pmix_info_t info[], size_t ninfo,
                             pmix_lookup_cbfunc_t cbfunc, void *cbdata)
\end{codepar}
\cspecificend

\begin{arglist}
\argin{proc}{\refstruct{pmix_proc_t} structure (handle)}
\argin{keys}{(array of strings)}
\argin{info}{Array of info structures (array of handles)}
\argin{ninfo}{Number of elements in the \refarg{info} array (integer)}
\argin{cbfunc}{Callback function \refapi{pmix_lookup_cbfunc_t} (function reference)}
\argin{cbdata}{Data to be passed to the callback function (memory reference)}
\end{arglist}

Returns \refconst{PMIX_SUCCESS} or a negative value corresponding to a PMIx error constant.

%%%%
\descr

Lookup published data.
The host server will be passed a NULL-terminated array of string keys.

The array of \refarg{info} structs is used to pass user-requested options to the server.
This can include a wait flag to indicate that the server should wait for all data to become available before executing the callback function, or should immediately callback with whatever data is available.
In addition, a timeout can be specified on the wait to preclude an indefinite wait for data that may never be published.


%%%%%%%%%%%
\subsection{\code{pmix_server_unpublish_fn_t}}
\declareapi{pmix_server_unpublish_fn_t}

%%%%
\summary

Delete data from the data store.

%%%%
\format

\cspecificstart
\begin{codepar}
typedef pmix_status_t (*pmix_server_unpublish_fn_t)(
                             const pmix_proc_t *proc, char **keys,
                             const pmix_info_t info[], size_t ninfo,
                             pmix_op_cbfunc_t cbfunc, void *cbdata)
\end{codepar}
\cspecificend

\begin{arglist}
\argin{proc}{\refstruct{pmix_proc_t} structure (handle)}
\argin{keys}{(array of strings)}
\argin{info}{Array of info structures (array of handles)}
\argin{ninfo}{Number of elements in the \refarg{info} array (integer)}
\argin{cbfunc}{Callback function \refapi{pmix_op_cbfunc_t} (function reference)}
\argin{cbdata}{Data to be passed to the callback function (memory reference)}
\end{arglist}

Returns \refconst{PMIX_SUCCESS} or a negative value corresponding to a PMIx error constant.

%%%%
\descr

Delete data from the data store.
The host server will be passed a NULL-terminated array of string keys, plus potential directives such as the data range within which the keys should be deleted.
The callback is to be executed upon completion of the delete procedure.


%%%%%%%%%%%
\subsection{\code{pmix_server_spawn_fn_t}}
\declareapi{pmix_server_spawn_fn_t}

%%%%
\summary

Spawn a set of applications/processes as per the PMIx API.

%%%%
\format

\cspecificstart
\begin{codepar}
typedef pmix_status_t (*pmix_server_spawn_fn_t)(
                             const pmix_proc_t *proc,
                             const pmix_info_t job_info[], size_t ninfo,
                             const pmix_app_t apps[], size_t napps,
                             pmix_spawn_cbfunc_t cbfunc, void *cbdata)
\end{codepar}
\cspecificend

\begin{arglist}
\argin{proc}{\refstruct{pmix_proc_t} structure (handle)}
\argin{job_info}{Array of info structures (array of handles)}
\argin{ninfo}{Number of elements in the \refarg{jobinfo} array (integer)}
\argin{apps}{Array of \refstruct{pmix_app_t} structures (array of handles)}
\argin{napps}{Number of elements in the \refarg{apps} array (integer)}
\argin{cbfunc}{Callback function \refapi{pmix_spawn_cbfunc_t} (function reference)}
\argin{cbdata}{Data to be passed to the callback function (memory reference)}
\end{arglist}

Returns \refconst{PMIX_SUCCESS} or a negative value corresponding to a PMIx error constant.

%%%%
\descr

Spawn a set of applications/processes as per the PMIx API.
Note that applications are not required to be MPI or any other programming model.
Thus, the host server cannot make any assumptions as to their required support.
The callback function is to be executed once all processes have been started.
An error in starting any application or process in this request shall cause all applications and processes in the request to be terminated, and an error returned to the originating caller.

Note that a timeout can be specified in the job_info array to indicate that failure to start the requested job within the given time should result in termination to avoid hangs.


%%%%%%%%%%%
\subsection{\code{pmix_server_connect_fn_t}}
\declareapi{pmix_server_connect_fn_t}

%%%%
\summary

Record the specified processes as ``connected''.

%%%%
\format

\cspecificstart
\begin{codepar}
typedef pmix_status_t (*pmix_server_connect_fn_t)(
                             const pmix_proc_t procs[], size_t nprocs,
                             const pmix_info_t info[], size_t ninfo,
                             pmix_op_cbfunc_t cbfunc, void *cbdata)
\end{codepar}
\cspecificend

\begin{arglist}
\argin{procs}{Array of \refstruct{pmix_proc_t} structures (array of handles)}
\argin{nprocs}{Number of elements in the \refarg{procs} array (integer)}
\argin{info}{Array of info structures (array of handles)}
\argin{ninfo}{Number of elements in the \refarg{info} array (integer)}
\argin{cbfunc}{Callback function \refapi{pmix_op_cbfunc_t} (function reference)}
\argin{cbdata}{Data to be passed to the callback function (memory reference)}
\end{arglist}

Returns \refconst{PMIX_SUCCESS} or a negative value corresponding to a PMIx error constant.

%%%%
\descr

Record the specified processes as ``connected''.
This means that the resource manager should treat the failure of any process in the specified group as a reportable event, and take appropriate action.
The callback function is to be called once all participating processes have called connect.
Note that a process can only engage in \textbf{one} connect operation involving the identical set of processes at a time.
However, a process \emph{can} be simultaneously engaged in multiple connect operations, each involving a different set of processes.

Note also that this is a collective operation within the client library, and thus the client will be blocked until all processes participate.
Thus, the \refarg{info} array can be used to pass user directives, including a timeout.
The directives are optional \emph{unless} the \emph{mandatory} flag has been set - in such cases, the host RM is required to return an error if the directive cannot be met.


%%%%%%%%%%%
\subsection{\code{pmix_server_disconnect_fn_t}}
\declareapi{pmix_server_disconnect_fn_t}

%%%%
\summary

Disconnect a previously connected set of processes.

%%%%
\format

\cspecificstart
\begin{codepar}
typedef pmix_status_t (*pmix_server_disconnect_fn_t)(
                             const pmix_proc_t procs[], size_t nprocs,
                             const pmix_info_t info[], size_t ninfo,
                             pmix_op_cbfunc_t cbfunc, void *cbdata)
\end{codepar}
\cspecificend

\begin{arglist}
\argin{procs}{Array of \refstruct{pmix_proc_t} structures (array of handles)}
\argin{nprocs}{Number of elements in the \refarg{procs} array (integer)}
\argin{info}{Array of info structures (array of handles)}
\argin{ninfo}{Number of elements in the \refarg{info} array (integer)}
\argin{cbfunc}{Callback function \refapi{pmix_op_cbfunc_t} (function reference)}
\argin{cbdata}{Data to be passed to the callback function (memory reference)}
\end{arglist}

Returns \refconst{PMIX_SUCCESS} or a negative value corresponding to a PMIx error constant.

%%%%
\descr

Disconnect a previously connected set of processes.
An error should be returned if the specified set of processes was not previously ``connected''.
As above, a process may be involved in multiple simultaneous disconnect operations.
However, a process is not allowed to reconnect to a set of ranges that has not fully completed disconnect (i.e., you have to fully disconnect before you can reconnect to the same group of processes).

Note also that this is a collective operation within the client library, and thus the client will be blocked until all processes participate.
Thus, the \refarg{info} array can be used to pass user directives, including a timeout.
The directives are optional \emph{unless} the \emph{mandatory} flag has been set - in such cases, the host RM is required to return an error if the directive cannot be met.


%%%%%%%%%%%
\subsection{\code{pmix_server_register_events_fn_t}}
\declareapi{pmix_server_register_events_fn_t}

%%%%
\summary

Register to receive notifications for the specified events.

%%%%
\format

\cspecificstart
\begin{codepar}
 typedef pmix_status_t (*pmix_server_register_events_fn_t)(
                              pmix_status_t *codes, size_t ncodes,
                              const pmix_info_t info[], size_t ninfo,
                              pmix_op_cbfunc_t cbfunc, void *cbdata)
\end{codepar}
\cspecificend

\begin{arglist}
\argin{codes}{Array of \refstruct{pmix_status_t} structures (array of handles)}
\argin{ncodes}{Number of elements in the \refarg{codes} array (integer)}
\argin{info}{Array of info structures (array of handles)}
\argin{ninfo}{Number of elements in the \refarg{info} array (integer)}
\argin{cbfunc}{Callback function \refapi{pmix_op_cbfunc_t} (function reference)}
\argin{cbdata}{Data to be passed to the callback function (memory reference)}
\end{arglist}

Returns \refconst{PMIX_SUCCESS} or a negative value corresponding to a PMIx error constant.

%%%%
\descr

Register to receive notifications for the specified events.
The resource manager is \emph{required} to pass along to the local PMIx server all events that directly relate to a registered namespace.
However, the RM may have access to events beyond those (e.g., environmental events).
The PMIx server will register to receive environmental events that match specific PMIx event codes.
If the host RM supports such notifications, it will need to translate its own internal event codes to fit into a corresponding PMIx event code - any specific info beyond that can be passed in via the \refstruct{pmix_info_t} upon notification.

The \refarg{info} array included in this API is reserved for possible future directives to further steer notification.



%%%%%%%%%%%
\subsection{\code{pmix_server_deregister_events_fn_t}}
\declareapi{pmix_server_deregister_events_fn_t}

%%%%
\summary

Deregister to receive notifications for the specified events.

%%%%
\format

\cspecificstart
\begin{codepar}
 typedef pmix_status_t (*pmix_server_deregister_events_fn_t)(
                              pmix_status_t *codes, size_t ncodes,
                              pmix_op_cbfunc_t cbfunc, void *cbdata)
\end{codepar}
\cspecificend

\begin{arglist}
\argin{codes}{Array of \refstruct{pmix_status_t} structures (array of handles)}
\argin{ncodes}{Number of elements in the \refarg{codes} array (integer)}
\argin{cbfunc}{Callback function \refapi{pmix_op_cbfunc_t} (function reference)}
\argin{cbdata}{Data to be passed to the callback function (memory reference)}
\end{arglist}

Returns \refconst{PMIX_SUCCESS} or a negative value corresponding to a PMIx error constant.

%%%%
\descr

Deregister to receive notifications for the specified environmental events for which the PMIx server has previously registered.
The host RM remains required to notify of any job-related events.


%%%%%%%%%%%
\subsection{\code{pmix_server_notify_event_fn_t}}
\declareapi{pmix_server_notify_event_fn_t}

%%%%
\summary

Notify the specified processes of an event.

%%%%
\format

\cspecificstart
\begin{codepar}
typedef pmix_status_t (*pmix_server_notify_event_fn_t)(pmix_status_t code,
                                                       const pmix_proc_t *source,
                                                       pmix_data_range_t range,
                                                       pmix_info_t info[], size_t ninfo,
                                                       pmix_op_cbfunc_t cbfunc, void *cbdata);
\end{codepar}
\cspecificend

\begin{arglist}
\argin{code}{\refstruct{pmix_status_t} structure (handle)}
\argin{source}{\refstruct{pmix_proc_t} (handle)}
\argin{range}{\refstruct{pmix_data_range_t} (handle)}
\argin{info}{Array of info structures (array of handles)}
\argin{ninfo}{Number of elements in the \refarg{info} array (integer)}
\argin{cbfunc}{Callback function \refapi{pmix_op_cbfunc_t} (function reference)}
\argin{cbdata}{Data to be passed to the callback function (memory reference)}
\end{arglist}

Returns \refconst{PMIX_SUCCESS} or a negative value corresponding to a PMIx error constant.

%%%%
\descr

Notify the specified processes of an event generated either by the PMIx server itself, or by one of its local clients.
The process generating the event is provided in the source parameter.


%%%%%%%%%%%
\subsection{\code{pmix_connection_cbfunc_t}}
\declareapi{pmix_connection_cbfunc_t}

%%%%
\summary

Callback function for incoming connection requests from local clients.

%%%%
\format

\cspecificstart
\begin{codepar}
typedef void (*pmix_connection_cbfunc_t)(
                    int incoming_sd, void *cbdata)
\end{codepar}
\cspecificend

\begin{arglist}
\argin{incoming_sd}{(integer)}
\argin{cbdata}{ (memory reference)}
\end{arglist}

Returns \refconst{PMIX_SUCCESS} or a negative value corresponding to a PMIx error constant.

%%%%
\descr

Callback function for incoming connection requests from local clients.


%%%%%%%%%%%
\subsection{\code{pmix_server_listener_fn_t}}
\declareapi{pmix_server_listener_fn_t}

%%%%
\summary

Register a socket the host server can monitor for connection requests.

%%%%
\format

\cspecificstart
\begin{codepar}
typedef pmix_status_t (*pmix_server_listener_fn_t)(
                             int listening_sd,
                             pmix_connection_cbfunc_t cbfunc,
                             void *cbdata)
\end{codepar}
\cspecificend

\begin{arglist}
\argin{incoming_sd}{(integer)}
\argin{cbfunc}{Callback function \refapi{pmix_connection_cbfunc_t} (function reference)}
\argin{cbdata}{ (memory reference)}
\end{arglist}

Returns \refconst{PMIX_SUCCESS} or a negative value corresponding to a PMIx error constant.

%%%%
\descr

Register a socket the host server can monitor for connection requests, harvest them, and then call our internal callback function for further processing.
A listener thread is essential to efficiently harvesting connection requests from large numbers of local clients such as occur when running on large SMPs.
The host server listener is required to call accept on the incoming connection request, and then passing the resulting soct to the provided cbfunc.
A NULL for this function will cause the internal PMIx server to spawn its own listener thread.


%%%%%%%%%%%
\subsection{\code{pmix_server_query_fn_t}}
\declareapi{pmix_server_query_fn_t}

%%%%
\summary

Query information from the resource manager.

%%%%
\format

\cspecificstart
\begin{codepar}
typedef pmix_status_t (*pmix_server_query_fn_t)(
                             pmix_proc_t *proct,
                             pmix_query_t *queries, size_t nqueries,
                             pmix_info_cbfunc_t cbfunc,
                             void *cbdata)
\end{codepar}
\cspecificend

\begin{arglist}
\argin{proct}{\refstruct{pmix_proc_t} structure (handle)}
\argin{queries}{Array of \refstruct{pmix_query_t} structures (array of handles)}
\argin{nqueries}{Number of elements in the \refarg{queries} array (integer)}
\argin{cbfunc}{Callback function \refapi{pmix_info_cbfunc_t} (function reference)}
\argin{cbdata}{Data to be passed to the callback function (memory reference)}
\end{arglist}

Returns \refconst{PMIX_SUCCESS} or a negative value corresponding to a PMIx error constant.

%%%%
\descr

Query information from the resource manager.
The query will include the nspace/rank of the process that is requesting the info, an array of \refstruct{pmix_query_t} describing the request, and a callback function/data for the return.


%%%%%%%%%%%
\subsection{\code{pmix_tool_connection_cbfunc_t}}
\declareapi{pmix_tool_connection_cbfunc_t}

%%%%
\summary

Callback function for incoming tool connections.

%%%%
\format

\cspecificstart
\begin{codepar}
typedef void (*pmix_tool_connection_cbfunc_t)(
                    pmix_status_t status,
                    pmix_proc_t *proc, void *cbdata)
\end{codepar}
\cspecificend

\begin{arglist}
\argin{status}{\refstruct{pmix_status_t} structure (handle)}
\argin{proc}{\refstruct{pmix_proc_t} structure (handle)}
\argin{cbdata}{Data to be passed (memory reference)}
\end{arglist}

%%%%
\descr

Callback function for incoming tool connections.
The host RM shall provide an nspace/rank for the connecting tool.
We assume that a \code{rank=0} will be the normal assignment, but allow for the future possibility of a parallel set of tools connecting, and thus each proc requiring a rank.


%%%%%%%%%%%
\subsection{\code{pmix_server_tool_connection_fn_t}}
\declareapi{pmix_server_tool_connection_fn_t}

%%%%
\summary

Register that a tool has connected to the server.

%%%%
\format

\cspecificstart
\begin{codepar}
typedef void (*pmix_server_tool_connection_fn_t)(
                    pmix_info_t *info, size_t ninfo,
                    pmix_tool_connection_cbfunc_t cbfunc,
                    void *cbdata)
\end{codepar}
\cspecificend

\begin{arglist}
\argin{info}{Array of info structures (array of handles)}
\argin{ninfo}{Number of elements in the \refarg{info} array (integer)}
\argin{cbfunc}{Callback function \refapi{pmix_tool_connection_cbfunc_t} (function reference)}
\argin{cbdata}{Data to be passed to the callback function (memory reference)}
\end{arglist}


%%%%
\descr

Register that a tool has connected to the server, and request that the tool be assigned an nspace/rank for further interactions.
The optional \refstruct{pmix_info_t} array can be used to pass qualifiers for the connection request:

\begin{constantdesc}
%
\declareconstitem{PMIX_USERID} effective userid of the tool
%
\declareconstitem{PMIX_GRPID} effective groupid of the tool
%
\declareconstitem{PMIX_FWD_STDOUT} forward any stdout to this tool
%
\declareconstitem{PMIX_FWD_STDERR} forward any stderr to this tool
%
\declareconstitem{PMIX_FWD_STDIN} forward stdin from this tool to any processes spawned on its behalf
%
\end{constantdesc}


%%%%%%%%%%%
\subsection{\code{pmix_server_log_fn_t}}
\declareapi{pmix_server_log_fn_t}

%%%%
\summary

Log data on behalf of a client.

%%%%
\format

\cspecificstart
\begin{codepar}
typedef void (*pmix_server_log_fn_t)(
                    const pmix_proc_t *client,
                    const pmix_info_t data[], size_t ndata,
                    const pmix_info_t directives[], size_t ndirs,
                    pmix_op_cbfunc_t cbfunc, void *cbdata)
\end{codepar}
\cspecificend

\begin{arglist}
\argin{client}{\refstruct{pmix_proc_t} structure (handle)}
\argin{data}{Array of info structures (array of handles)}
\argin{ndata}{Number of elements in the \refarg{data} array (integer)}
\argin{directives}{Array of info structures (array of handles)}
\argin{ndirs}{Number of elements in the \refarg{directives} array (integer)}
\argin{cbfunc}{Callback function \refapi{pmix_op_cbfunc_t} (function reference)}
\argin{cbdata}{Data to be passed to the callback function (memory reference)}
\end{arglist}


%%%%
\descr

Log data on behalf of a client.


%%%%%%%%%%%
\subsection{\code{pmix_server_alloc_fn_t}}
\declareapi{pmix_server_alloc_fn_t}

%%%%
\summary

Request allocation modifications on behalf of a client.

%%%%
\format

\cspecificstart
\begin{codepar}
typedef pmix_status_t (*pmix_server_alloc_fn_t)(
                             const pmix_proc_t *client,
                             pmix_alloc_directive_t directive,
                             const pmix_info_t data[], size_t ndata,
                             pmix_info_cbfunc_t cbfunc, void *cbdata)
\end{codepar}
\cspecificend

\begin{arglist}
\argin{client}{\refstruct{pmix_proc_t} structure (handle)}
\argin{directive}{(handle)}
\argin{data}{Array of info structures (array of handles)}
\argin{ndata}{Number of elements in the \refarg{data} array (integer)}
\argin{cbfunc}{Callback function \refapi{pmix_info_cbfunc_t} (function reference)}
\argin{cbdata}{Data to be passed to the callback function (memory reference)}
\end{arglist}

Returns \refconst{PMIX_SUCCESS} or a negative value corresponding to a PMIx error constant.

%%%%
\descr

Request allocation modifications on behalf of a client.


%%%%%%%%%%%
\subsection{\code{pmix_server_job_control_fn_t}}
\declareapi{pmix_server_job_control_fn_t}

%%%%
\summary

Execute a job control action on behalf of a client.

%%%%
\format

\cspecificstart
\begin{codepar}
typedef pmix_status_t (*pmix_server_job_control_fn_t)(
                             const pmix_proc_t *requestor,
                             const pmix_proc_t targets[], size_t ntargets,
                             const pmix_info_t directives[], size_t ndirs,
                             pmix_info_cbfunc_t cbfunc, void *cbdata)
\end{codepar}
\cspecificend

\begin{arglist}
\argin{requestor}{\refstruct{pmix_proc_t} structure (handle)}
\argin{targets}{Array of proc structures (array of handles)}
\argin{ntargets}{Number of elements in the \refarg{targets} array (integer)}
\argin{directives}{Array of info structures (array of handles)}
\argin{ndirs}{Number of elements in the \refarg{info} array (integer)}
\argin{cbfunc}{Callback function \refapi{pmix_op_cbfunc_t} (function reference)}
\argin{cbdata}{Data to be passed to the callback function (memory reference)}
\end{arglist}

Returns \refconst{PMIX_SUCCESS} or a negative value corresponding to a PMIx error constant.

%%%%
\descr

Execute a job control action on behalf of a client.


%%%%%%%%%%%
\subsection{\code{pmix_server_monitor_fn_t}}
\declareapi{pmix_server_monitor_fn_t}

%%%%
\summary

Request that a client be monitored for activity.

%%%%
\format

\cspecificstart
\begin{codepar}
/* Request that a client be monitored for activity */
typedef pmix_status_t (*pmix_server_monitor_fn_t)(
                             const pmix_proc_t *requestor,
                             const pmix_info_t *monitor, pmix_status_t error,
                             const pmix_info_t directives[], size_t ndirs,
                             pmix_info_cbfunc_t cbfunc, void *cbdata);
\end{codepar}
\cspecificend

\begin{arglist}
\argin{requestor}{\refstruct{pmix_proc_t} structure (handle)}
\argin{monitor}{\refstruct{pmix_proc_t} structure (handle)}
\argin{error}{(integer)}
\argin{directives}{Array of info structures (array of handles)}
\argin{ndirs}{Number of elements in the \refarg{info} array (integer)}
\argin{cbfunc}{Callback function \refapi{pmix_op_cbfunc_t} (function reference)}
\argin{cbdata}{Data to be passed to the callback function (memory reference)}
\end{arglist}

Returns \refconst{PMIX_SUCCESS} or a negative value corresponding to a PMIx error constant.

%%%%
\descr

Request that a client be monitored for activity.

%%%%%%%%%%%%%%%%%%%%%%%%%%%%%%%%%%%%%%%%%%%%%%%%%

    %%%%%%%%%%%%%%%%%%%%%%%%%%%%%%%%%%%%%%%%%%%%%%%%%
% Chapter: API Tool
%%%%%%%%%%%%%%%%%%%%%%%%%%%%%%%%%%%%%%%%%%%%%%%%%
\chapter{Tool API}
\label{chap:api_tool}

This interface extends the \refsection{chap:api_client}{Client-side API} for tools to connect to the \ac{PMIx} server and query information about the \ac{PMIx} environment including the application namespaces.

%%%%%%%%%%%
\section{Startup and Shutdown}

A separate set of initialization and finalization routines are defined for tools to help facilitate the differentiation of \emph{clients} for the \ac{PMIx} server.

%%%%%%%%%%%
\subsection{\code{PMIx_tool_init}}
\declareapi{PMIx_tool_init}

%%%%
\summary

Initialize the \ac{PMIx} library for a tool connection.

%%%%
\format

\cspecificstart
\begin{codepar}
pmix_status_t PMIx_tool_init(pmix_proc_t *proc,
                             pmix_info_t info[], size_t ninfo)
\end{codepar}
\cspecificend

\begin{arglist}
\arginout{proc}{\refstruct{pmix_proc_t} structure (handle)}
\argin{info}{Array of info structures (array of handles)}
\argin{ninfo}{Number of element in the \refarg{info} array (integer)}
\end{arglist}

Returns \refconst{PMIX_SUCCESS} or a negative value corresponding to a PMIx error constant.

%%%%
\descr

Initialize the PMIx tool, returning the process identifier assigned to this tool in the provided \refstruct{pmix_proc_t} struct.

When called the PMIx tool library will check for the required connection information of the local PMIx server and will establish the connection.
If the information is not found, or the server connection fails, then an appropriate error constant will be returned.

If successful, the function will return \refconst{PMIX_SUCCESS} and will fill the provided structure with the server-assigned namespace and rank of the tool.

Note that the PMIx tool library is referenced counted, and so multiple calls to \refapi{PMIx_tool_init} are allowed.
Thus, one way to obtain the namespace and rank of the process is to simply call \refapi{PMIx_tool_init} with a non-NULL parameter.

The \refarg{info} array is used to pass user requests pertaining to the init and subsequent operations.
Passing a \code{NULL} value for the array pointer is supported if no directives are desired.


%%%%%%%%%%%
\subsection{\code{PMIx_tool_finalize}}
\declareapi{PMIx_tool_finalize}

%%%%
\summary

Finalize the \ac{PMIx} library for a tool connection.

%%%%
\format

\cspecificstart
\begin{codepar}
pmix_status_t PMIx_tool_finalize(void)
\end{codepar}
\cspecificend

Returns \refconst{PMIX_SUCCESS} or a negative value corresponding to a PMIx error constant.

%%%%
\descr

Finalize the PMIx tool library, closing the connection to the local server.
An error code will be returned if, for some reason, the connection cannot be closed.

%%%%%%%%%%%%%%%%%%%%%%%%%%%%%%%%%%%%%%%%%%%%%%%%%


%
% Appendix
%
    \setcounter{chapter}{0}  % restart chapter numbering with "letter A"
    \renewcommand{\thechapter}{\Alph{chapter}}%
    \appendix

    %%%%%%%%%%%%%%%%%%%%%%%%%%%%%%%%%%%%%%%%%%%%%%%%%
% Chapter: History
%%%%%%%%%%%%%%%%%%%%%%%%%%%%%%%%%%%%%%%%%%%%%%%%%
\chapter{Document Revision History}
\label{chap:history}

%%%%%%%%%% Version 2.0
\section{Version 2.0: Aug 31, 2018}

The following \acp{API} were introduced in v2 of the PMIx Standard:

\begin{itemize}
\item Client APIs
\begin{itemize}
\item \refapi{PMIx_Query_info_nb}, \refapi{PMIx_Log_nb}
\item \refapi{PMIx_Allocation_request_nb}, \refapi{PMIx_Job_control_nb}, \refapi{PMIx_Process_monitor_nb}
\end{itemize}
\item Server APIs
\begin{itemize}
\item \refapi{PMIx_server_setup_application}, \refapi{PMIx_server_setup_local_support}
\end{itemize}
\item Tool APIs
\begin{itemize}
\item \refapi{PMIx_tool_init}, \refapi{PMIx_tool_finalize}
\end{itemize}
\item Common APIs
\begin{itemize}
\item \refapi{PMIx_Register_event_handler}, \refapi{PMIx_Deregister_event_handler}
\item \refapi{PMIx_Notify_event}
\item \refapi{PMIx_Proc_state_string}, \refapi{PMIx_Scope_string}
\item \refapi{PMIx_Persistence_string}, \refapi{PMIx_Data_range_string}
\item \refapi{PMIx_Info_directives_string}, \refapi{PMIx_Data_type_string}
\item \refapi{PMIx_Alloc_directive_string}
\item \refapi{PMIx_Data_pack}, \refapi{PMIx_Data_unpack}, \refapi{PMIx_Data_copy}
\item \refapi{PMIx_Data_print}, \refapi{PMIx_Data_copy_payload}
\end{itemize}
\end{itemize}

The \refapi{PMIx_Init} \ac{API} was modified in v2 of the standard from its \textit{ad hoc} v1 signature to include passing of a \refstruct{pmix_info_t} array for flexibility and ``future-proofing'' of the \ac{API}. In addition, the PMIx\_Notify\_error, PMIx\_Register\_errhandler, and PMIx\_Deregister\_errhandler \acp{API} were replaced.

%%%%%%%%%% Version 1.0
\section{Version 1.0: June 12, 2015}

\par
An \textit{ad hoc} standard was defined in the \acf{PRI} header files as part of the \ac{PMIx} v1.0.0 release prior to the creation of the formal 2.0 standard.
Below are a summary listing of the interfaces defined in the 1.0 headers.

\begin{itemize}
\item Client APIs
\begin{itemize}
\item PMIx\_Init, \refapi{PMIx_Initialized}, \refapi{PMIx_Abort}, \refapi{PMIx_Finalize}
\item \refapi{PMIx_Put}, \refapi{PMIx_Commit},
\item \refapi{PMIx_Fence}, \refapi{PMIx_Fence_nb}
\item \refapi{PMIx_Get}, \refapi{PMIx_Get_nb}
\item \refapi{PMIx_Publish}, \refapi{PMIx_Publish_nb}
\item \refapi{PMIx_Lookup}, \refapi{PMIx_Lookup}
\item \refapi{PMIx_Unpublish}, \refapi{PMIx_Unpublish_nb}
\item \refapi{PMIx_Spawn}, \refapi{PMIx_Spawn_nb}
\item \refapi{PMIx_Connect}, \refapi{PMIx_Connect_nb}
\item \refapi{PMIx_Disconnect}, \refapi{PMIx_Disconnect_nb}
\item \refapi{PMIx_Resolve_nodes}, \refapi{PMIx_Resolve_peers}
\end{itemize}
\item Server APIs
\begin{itemize}
\item \refapi{PMIx_server_init}, \refapi{PMIx_server_finalize}
\item \refapi{PMIx_generate_regex}, \refapi{PMIx_generate_ppn}
\item \refapi{PMIx_server_register_nspace}, \refapi{PMIx_server_deregister_nspace}
\item \refapi{PMIx_server_register_client}, \refapi{PMIx_server_deregister_client}
\item \refapi{PMIx_server_setup_fork}, \refapi{PMIx_server_dmodex_request}
\end{itemize}
\item Common APIs
\begin{itemize}
\item \refapi{PMIx_Get_version}, \refapi{PMIx_Store_internal}, \refapi{PMIx_Error_string}
\item \refapi{PMIx_Register_errhandler}, \refapi{PMIx_Deregister_errhandler}, \refapi{PMIx_Notify_error}
\end{itemize}
\end{itemize}

The PMIx\_Init \ac{API} was subsequently modified in release \ac{PMIx} v1.1.0.


%
% Index
%
	\nolinenumbers
	\printindex
	
\end{document}

%%%%%%%%%%%%%%%%%%%%%%%%%%%%%%%%%%%%%%%%%%%%%%%%%
