%%%%%%%%%%%%%%%%%%%%%%%%%%%%%%%%%%%%%%%%%%%%%%%%%
% Chapter: API Tool
%%%%%%%%%%%%%%%%%%%%%%%%%%%%%%%%%%%%%%%%%%%%%%%%%
\chapter{Tool API}
\label{chap:api_tool}

This interface extends the \refsection{chap:api_client}{Client-side API} for tools to connect to the \ac{PMIx} server and query information about the \ac{PMIx} environment including the application namespaces.

%%%%%%%%%%%
\section{Startup and Shutdown}

A separate set of initialization and finalization routines are defined for tools to help facilitate the differentiation of \emph{clients} for the \ac{PMIx} server.

%%%%%%%%%%%
\subsection{\code{PMIx_tool_init}}
\declareapi{PMIx_tool_init}

%%%%
\summary

Initialize the \ac{PMIx} library for a tool connection.

%%%%
\format

\cspecificstart
\begin{codepar}
pmix_status_t PMIx_tool_init(pmix_proc_t *proc,
                             pmix_info_t info[], size_t ninfo)
\end{codepar}
\cspecificend

\begin{arglist}
\arginout{proc}{\refstruct{pmix_proc_t} structure (handle)}
\argin{info}{Array of info structures (array of handles)}
\argin{ninfo}{Number of element in the \refarg{info} array (integer)}
\end{arglist}

Returns \refconst{PMIX_SUCCESS} or a negative value corresponding to a PMIx error constant.

%%%%
\descr

Initialize the PMIx tool, returning the process identifier assigned to this tool in the provided \refstruct{pmix_proc_t} struct.

When called the PMIx tool library will check for the required connection information of the local PMIx server and will establish the connection.
If the information is not found, or the server connection fails, then an appropriate error constant will be returned.

If successful, the function will return \refconst{PMIX_SUCCESS} and will fill the provided structure with the server-assigned namespace and rank of the tool.

Note that the PMIx tool library is referenced counted, and so multiple calls to \refapi{PMIx_tool_init} are allowed.
Thus, one way to obtain the namespace and rank of the process is to simply call \refapi{PMIx_tool_init} with a non-NULL parameter.

The \refarg{info} array is used to pass user requests pertaining to the init and subsequent operations.
Passing a \code{NULL} value for the array pointer is supported if no directives are desired.


%%%%%%%%%%%
\subsection{\code{PMIx_tool_finalize}}
\declareapi{PMIx_tool_finalize}

%%%%
\summary

Finalize the \ac{PMIx} library for a tool connection.

%%%%
\format

\cspecificstart
\begin{codepar}
pmix_status_t PMIx_tool_finalize(void)
\end{codepar}
\cspecificend

Returns \refconst{PMIX_SUCCESS} or a negative value corresponding to a PMIx error constant.

%%%%
\descr

Finalize the PMIx tool library, closing the connection to the local server.
An error code will be returned if, for some reason, the connection cannot be closed.

%%%%%%%%%%%%%%%%%%%%%%%%%%%%%%%%%%%%%%%%%%%%%%%%%
