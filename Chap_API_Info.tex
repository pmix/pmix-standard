%%%%%%%%%%%%%%%%%%%%%%%%%%%%%%%%%%%%%%%%%%%%%%%%%
% Chapter: Informational
%%%%%%%%%%%%%%%%%%%%%%%%%%%%%%%%%%%%%%%%%%%%%%%%%
\chapter{General Information Interfaces}
\label{chap:api_info}

The \acp{API} defined in this chapter can be used by any \ac{PMIx} process, regardless of their role in the \ac{PMIx} universe.


%%%%%%%%%%%%%%%%%%%%%%%%%%%%%%%%%%%%%%%%%%%%%%
%%%%%%%%%%%%%%%%%%%%%%%%%%%%%%%%%%%%%%%%%%%%%%
\section{Initialization Status}
\label{chap:api_info:init}

The \acp{API} defined in this section return information about the status of the \ac{PMIx} library.

%%%%%%%%%%%
\subsection{\code{PMIx_Initialized}}
\declareapi{PMIx_Initialized}

%%%%
\format

\versionMarker{1.0}
\cspecificstart
\begin{codepar}
int PMIx_Initialized(void)
\end{codepar}
\cspecificend

A value of \code{1} (true) will be returned if the \ac{PMIx} library has been initialized, and \code{0} (false) otherwise.

\rationalestart
The return value is an integer for historical reasons as that was the signature of prior PMI libraries.
\rationaleend

%%%%
\descr

Check to see if the \ac{PMIx} library has been initialized using any of the initialization functions:
\refapi{PMIx_Init}, \refapi{PMIx_server_init}, or \refapi{PMIx_tool_init}.
It is valid to call this \ac{API} outside of a region of initialization.

%%%%%%%%%%%%%%%%%%%%%%%%%%%%%%%%%%%%%%%%%%%%%%
%%%%%%%%%%%%%%%%%%%%%%%%%%%%%%%%%%%%%%%%%%%%%%
\section{Library Information}
\label{chap:api_info:lib}

The \acp{API} defined in this section return information about the \ac{PMIx} library.

%%%%%%%%%%%
\subsection{\code{PMIx_Get_version}}
\declareapi{PMIx_Get_version}

%%%%
\summary

Get the \ac{PMIx} version information.

%%%%
\format

\versionMarker{1.0}
\cspecificstart
\begin{codepar}
const char* PMIx_Get_version(void)
\end{codepar}
\cspecificend

%%%%
\descr

Get the \ac{PMIx} version string.
Note that the provided string is statically defined and must \textit{not} be free'd.


%%%%%%%%%%%%%%%%%%%%%%%%%%%%%%%%%%%%%%%%%%%%%%%%%
