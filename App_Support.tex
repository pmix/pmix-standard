%%%%%%%%%%%%%%%%%%%%%%%%%%%%%%%%%%%%%%%%%%%%%%%%%
% Appendix: Support functions
%%%%%%%%%%%%%%%%%%%%%%%%%%%%%%%%%%%%%%%%%%%%%%%%%
\chapter{PMIx Support Functions}
\label{app:support}

This chapter describes some additional support APIs that are provided in the \ac{PRI} headers, but are not part of the core \ac{PMIx} standard specification as the macros reference internal \ac{PRI} functions exposed in the \ac{PRI}'s external headers.


%%%%%%%%%%%
\section{Data Structure Support}

This section describes some additional support macros focused on the data structures defined in \chapterref{chap:struct}.


%%%%%%%%%%%
\subsection{Argument array extension}
\declaremacro{PMIX_ARGV_APPEND}

%%%%
\summary

Append a string to a NULL-terminated, argv-style array of strings.

\cspecificstart
\begin{codepar}
PMIX_ARGV_APPEND(r, a, b);
\end{codepar}
\cspecificend

\begin{arglist}
\argout{r}{Status code indicating success or failure of the operation (\refstruct{pmix_status_t})}
\arginout{a}{Argument list (pointer to NULL-terminated array of strings)}
\argin{b}{Argument to append to the list (string)}
\end{arglist}

%%%%
\descr

This function helps the caller build the \code{argv} portion of \refstruct{pmix_app_t} structure, arrays of keys for querying, or other places where argv-style string arrays are required in the way that the \ac{PRI} expects it to be constructed.

\adviceuserstart
The provided argument is copied into the destination array - thus, the source string can be free'd without affecting the array once the macro has completed.
\adviceuserend

%%%%%%%%%%%
\subsection{Argument array extension - unique}
\declaremacro{PMIX_ARGV_APPEND_UNIQUE}

%%%%
\summary

Append a string to a NULL-terminated, argv-style array of strings, but only if the provided argument doesn't already exist somewhere in the array.

\cspecificstart
\begin{codepar}
PMIX_ARGV_APPEND_UNIQUE(r, a, b);
\end{codepar}
\cspecificend

\begin{arglist}
\argout{r}{Status code indicating success or failure of the operation (\refstruct{pmix_status_t})}
\arginout{a}{Argument list (pointer to NULL-terminated array of strings)}
\argin{b}{Argument to append to the list (string)}
\end{arglist}

%%%%
\descr

This function helps the caller build the \code{argv} portion of \refstruct{pmix_app_t} structure, arrays of keys for querying, or other places where argv-style string arrays are required in the way that the \ac{PRI} expects it to be constructed.

\adviceuserstart
The provided argument is copied into the destination array - thus, the source string can be free'd without affecting the array once the macro has completed.
\adviceuserend

%%%%%%%%%%%
\subsection{Argument array release}
\declaremacro{PMIX_ARGV_FREE}

%%%%
\summary

Free an argv-style array and all of the strings that it contains

\cspecificstart
\begin{codepar}
PMIX_ARGV_FREE(a);
\end{codepar}
\cspecificend

\begin{arglist}
\argin{a}{Argument list (pointer to NULL-terminated array of strings)}
\end{arglist}

%%%%
\descr

This function releases the array and all of the strings it contains.

%%%%%%%%%%%
\subsection{Argument array split}
\declaremacro{PMIX_ARGV_SPLIT}

%%%%
\summary

Split a string into a NULL-terminated argv array.

\cspecificstart
\begin{codepar}
PMIX_ARGV_SPLIT(a, b, c);
\end{codepar}
\cspecificend

\begin{arglist}
\argout{a}{Resulting argv-style array (\code{char**})}
\argin{b}{String to be split (\code{char*})}
\argin{c}{Delimiter character (\code{char})}
\end{arglist}

%%%%
\descr

Split an input string into a NULL-terminated argv array. Do not include empty strings in the resulting array.

\adviceuserstart
All strings are inserted into the argv array by value; the newly-allocated array makes no references to the src_string argument (i.e., it can be freed after calling this function without invalidating the output argv array)
\adviceuserend

%%%%%%%%%%%
\subsection{Argument array join}
\declaremacro{PMIX_ARGV_JOIN}

%%%%
\summary

Join all the elements of an argv array into a single newly-allocated string.

\cspecificstart
\begin{codepar}
PMIX_ARGV_JOIN(a, b, c);
\end{codepar}
\cspecificend

\begin{arglist}
\argout{a}{Resulting string (\code{char*})}
\argin{b}{Argv-style array to be joined (\code{char**})}
\argin{c}{Delimiter character (\code{char})}
\end{arglist}

%%%%
\descr

Join all the elements of an argv array into a single newly-allocated string.

%%%%%%%%%%%
\subsection{Argument array count}
\declaremacro{PMIX_ARGV_COUNT}

%%%%
\summary

Return the length of a NULL-terminated argv array.

\cspecificstart
\begin{codepar}
PMIX_ARGV_COUNT(r, a);
\end{codepar}
\cspecificend

\begin{arglist}
\argout{r}{Number of strings in the array (integer)}
\argin{a}{Argv-style array (\code{char**})}
\end{arglist}

%%%%
\descr

Count the number of elements in an argv array


%%%%%%%%%%%
\subsection{Argument array copy}
\declaremacro{PMIX_ARGV_COPY}

%%%%
\summary

Copy an argv array, including copying all off its strings.

\cspecificstart
\begin{codepar}
PMIX_ARGV_COPY(a, b);
\end{codepar}
\cspecificend

\begin{arglist}
\argout{a}{New argv-style array (\code{char**})}
\argin{b}{Argv-style array (\code{char**})}
\end{arglist}

%%%%
\descr

Copy an argv array, including copying all off its strings.


%%%%%%%%%%%
\section{Environment Manipulation Support}

This section describes some additional support APIs focused on environment manipulation.

%%%%%%%%%%%
\subsection{Set an environment variable}
\declaremacro{PMIX_SETENV}

%%%%
\summary

Set an environment variable in a NULL-terminated, env-style array

\cspecificstart
\begin{codepar}
PMIX_SETENV(r, name, value, env);
\end{codepar}
\cspecificend


\begin{arglist}
\argout{r}{Status code indicating success or failure of the operation (\refstruct{pmix_status_t})}
\argin{name}{Argument name (string)}
\argin{value}{Argument value (string)}
\arginout{env}{Environment array to update (pointer to array of strings)}
\end{arglist}

%%%%
\descr

Similar to \code{setenv} from the C API, this allows the caller to set an environment variable in the specified \code{env} array, which could then be passed to the \refstruct{pmix_app_t} structure or any other destination.

\adviceuserstart
The provided name and value are copied into the destination environment array - thus, the source strings can be free'd without affecting the array once the macro has completed.
\adviceuserend


%%%%%%%%%%%%%%%%%%%%%%%%%%%%%%%%%%%%%%%%%%%%%%%%%
