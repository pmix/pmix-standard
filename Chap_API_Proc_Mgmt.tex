%%%%%%%%%%%%%%%%%%%%%%%%%%%%%%%%%%%%%%%%%%%%%%%%%
% Chapter: Process Management
%%%%%%%%%%%%%%%%%%%%%%%%%%%%%%%%%%%%%%%%%%%%%%%%%
\chapter{Process Management}
\label{chap:api_proc_mgmt}

This chapter defines functionality processes can use to create and manage processes. The management features presented in this chapter include aborting processes, connecting and disconnecting processes and determining the relative locality of local processes.

%%%%%%%%%%%%%%%%%%%%%%%%%%%%%%%%%%%%%%%%%%%%%%%%%
%%%%%%%%%%%%%%%%%%%%%%%%%%%%%%%%%%%%%%%%%%%%%%%%%
\section{Process Creation}
\label{chap:api_proc_mgmt:spawn}

The \refapi{PMIx_Spawn} commands spawn new processes and/or applications in the \ac{PMIx} universe. This may include requests to extend the existing resource allocation or obtain a new one, depending upon provided and supported attributes.

%%%%%%%%%%%%%%%%%%%%%%%%%%%%%%%%%%%%%%%%%%%%%%%%%
\subsection{\code{PMIx_Spawn}}
\declareapi{PMIx_Spawn}

%%%%
\summary

Spawn a new job.

%%%%
\format

\copySignature{PMIx_Spawn}{1.0}{
pmix_status_t \\
PMIx_Spawn(const pmix_info_t job_info[], size_t ninfo, \\
\hspace*{11\sigspace}const pmix_app_t apps[], size_t napps, \\
\hspace*{11\sigspace}char nspace[])
}

\begin{arglist}
\argin{job_info}{Array of info structures (array of handles)}
\argin{ninfo}{Number of elements in the \refarg{job_info} array (integer)}
\argin{apps}{Array of \refstruct{pmix_app_t} structures (array of handles)}
\argin{napps}{Number of elements in the \refarg{apps} array (integer)}
\argout{nspace}{Namespace of the new job (string)}
\end{arglist}

\returnstart
\begin{constantdesc}
\item \refconst{PMIX_ERR_JOB_ALLOC_FAILED} The job request could not be executed due to failure to obtain the specified allocation
\item \refconst{PMIX_ERR_JOB_APP_NOT_EXECUTABLE} The specified application executable either could not be found, or lacks execution privileges.
\item \refconst{PMIX_ERR_JOB_NO_EXE_SPECIFIED} The job request did not specify an executable.
\item \refconst{PMIX_ERR_JOB_FAILED_TO_MAP} The launcher was unable to map the processes for the specified job request.
\item \refconst{PMIX_ERR_JOB_FAILED_TO_LAUNCH} One or more processes in the job request failed to launch
\item \refconst{PMIX_ERR_JOB_EXE_NOT_FOUND} Specified executable not found
\item \refconst{PMIX_ERR_JOB_INSUFFICIENT_RESOURCES} Insufficient resources to spawn job
\item \refconst{PMIX_ERR_JOB_SYS_OP_FAILED} System library operation failed
\item \refconst{PMIX_ERR_JOB_WDIR_NOT_FOUND} Specified working directory not found
\end{constantdesc}
\returnend

\reqattrstart
\ac{PMIx} libraries are not required to directly support any attributes for this function. However, any provided attributes must be passed to the host environment for processing.

Host environments are required to support the following attributes when present in either the \refarg{job_info} or the \textit{info} array of an element of the \refarg{apps} array:

\pasteAttributeItem{PMIX_WDIR}
\pasteAttributeItem{PMIX_SET_SESSION_CWD}
\pasteAttributeItem{PMIX_PREFIX}
\pasteAttributeItem{PMIX_HOST}
\pasteAttributeItem{PMIX_HOSTFILE}

\reqattrend

\optattrstart
The following attributes are optional for host environments that support this operation:

\pasteAttributeItem{PMIX_ADD_HOSTFILE}
\pasteAttributeItem{PMIX_ADD_HOST}
\pasteAttributeItem{PMIX_PRELOAD_BIN}
\pasteAttributeItem{PMIX_PRELOAD_FILES}
\pasteAttributeItem{PMIX_PERSONALITY}
\pasteAttributeItem{PMIX_DISPLAY_MAP}
\pasteAttributeItem{PMIX_PPR}
\pasteAttributeItem{PMIX_MAPBY}
\pasteAttributeItem{PMIX_RANKBY}
\pasteAttributeItem{PMIX_BINDTO}
\pasteAttributeItem{PMIX_STDIN_TGT}
\pasteAttributeItem{PMIX_TAG_OUTPUT}
\pasteAttributeItem{PMIX_TIMESTAMP_OUTPUT}
\pasteAttributeItem{PMIX_MERGE_STDERR_STDOUT}
\pasteAttributeItem{PMIX_OUTPUT_TO_FILE}
\pasteAttributeItem{PMIX_INDEX_ARGV}
\pasteAttributeItem{PMIX_CPUS_PER_PROC}
\pasteAttributeItem{PMIX_NO_PROCS_ON_HEAD}
\pasteAttributeItem{PMIX_NO_OVERSUBSCRIBE}
\pasteAttributeItem{PMIX_REPORT_BINDINGS}
\pasteAttributeItem{PMIX_CPU_LIST}
\pasteAttributeItem{PMIX_JOB_RECOVERABLE}
\pasteAttributeItem{PMIX_JOB_CONTINUOUS}
\pasteAttributeItem{PMIX_MAX_RESTARTS}
\pasteAttributeItem{PMIX_SET_ENVAR}
\pasteAttributeItem{PMIX_UNSET_ENVAR}
\pasteAttributeItem{PMIX_ADD_ENVAR}
\pasteAttributeItem{PMIX_PREPEND_ENVAR}
\pasteAttributeItem{PMIX_APPEND_ENVAR}
\pasteAttributeItem{PMIX_FIRST_ENVAR}
\pasteAttributeItem{PMIX_ALLOC_QUEUE}
\pasteAttributeItem{PMIX_ALLOC_TIME}
\pasteAttributeItem{PMIX_ALLOC_NUM_NODES}
\pasteAttributeItem{PMIX_ALLOC_NODE_LIST}
\pasteAttributeItem{PMIX_ALLOC_NUM_CPUS}
\pasteAttributeItem{PMIX_ALLOC_NUM_CPU_LIST}
\pasteAttributeItem{PMIX_ALLOC_CPU_LIST}
\pasteAttributeItem{PMIX_ALLOC_MEM_SIZE}
\pasteAttributeItem{PMIX_ALLOC_BANDWIDTH}
\pasteAttributeItem{PMIX_ALLOC_FABRIC_QOS}
\pasteAttributeItem{PMIX_ALLOC_FABRIC_TYPE}
\pasteAttributeItem{PMIX_ALLOC_FABRIC_PLANE}
\pasteAttributeItem{PMIX_ALLOC_FABRIC_ENDPTS}
\pasteAttributeItem{PMIX_ALLOC_FABRIC_ENDPTS_NODE}
\pasteAttributeItem{PMIX_COSPAWN_APP}
\pasteAttributeItem{PMIX_SPAWN_TOOL}
\pasteAttributeItem{PMIX_EVENT_SILENT_TERMINATION}
\pasteAttributeItem{PMIX_ENVARS_HARVESTED}
\pasteAttributeItem{PMIX_JOB_TIMEOUT}
\pasteAttributeItem{PMIX_SPAWN_TIMEOUT}
\pasteAttributeItem{PMIX_NOTIFY_COMPLETION}
\pasteAttributeItem{PMIX_NOTIFY_PROC_TERMINATION}
\pasteAttributeItem{PMIX_NOTIFY_PROC_ABNORMAL_TERMINATION}
\pasteAttributeItem{PMIX_LOG_COMPLETION}
\pasteAttributeItem{PMIX_LOG_PROC_TERMINATION}
\pasteAttributeItem{PMIX_LOG_PROC_ABNORMAL_TERMINATION}
\pasteAttributeItem{PMIX_LOG_JOB_EVENTS}
\pasteAttributeItem{PMIX_LOG_COMPLETION}
\optattrend

%%%%
\descr

Spawn a new job.
The assigned namespace of the spawned applications is returned in the \refarg{nspace} parameter.
A \code{NULL} value in that location indicates that the caller does not wish to have the namespace returned.
The \refarg{nspace} array must be at least of size one more than \refconst{PMIX_MAX_NSLEN}.

By default, the spawned processes will be PMIx ``connected'' to the parent process upon successful launch (see Section \ref{chap:api_proc_mgmt:connect}
for details).  
Both the parent process and members of the child job will receive notification of errors from processes in their combined assemblage.

\adviceimplstart
It is recommended that an implementation will cause the parent process to be given a copy of the new job's
information so it can query job-level info without incurring any communication penalties.  Similarly, the newly 
spawned child processes should receive a copy of the parent processes' job-level info due to the high likelihood 
that the child will make subsequent queries about its parent.
\adviceimplend

\adviceuserstart
Behavior of individual resource managers may differ, but it is expected that failure of any application process to start will result in termination/cleanup of all processes in the newly spawned job and return of an error code to the caller.
\adviceuserend

\adviceimplstart
Tools may utilize \refapi{PMIx_Spawn} to start intermediate launchers as described in Section \ref{chap:api_tools:indirect}. For times where the tool is not attached to a \ac{PMIx} server, internal support for fork/exec of the specified applications would allow the tool to maintain a single code path for both the connected and disconnected cases. Inclusion of such support is recommended, but not required.
\adviceimplend


%%%%%%%%%%%%%%%%%%%%%%%%%%%%%%%%%%%%%%%%%%%%%%%%%
\subsection{\code{PMIx_Spawn_nb}}
\declareapi{PMIx_Spawn_nb}

%%%%
\summary

Nonblocking version of the \refapi{PMIx_Spawn} routine.

%%%%
\format

\copySignature{PMIx_Spawn_nb}{1.0}{
pmix_status_t \\
PMIx_Spawn_nb(const pmix_info_t job_info[], size_t ninfo, \\
\hspace*{14\sigspace}const pmix_app_t apps[], size_t napps, \\
\hspace*{14\sigspace}pmix_spawn_cbfunc_t cbfunc, void *cbdata)
}

\begin{arglist}
\argin{job_info}{Array of info structures (array of handles)}
\argin{ninfo}{Number of elements in the \refarg{job_info} array (integer)}
\argin{apps}{Array of \refstruct{pmix_app_t} structures (array of handles)}
\argin{cbfunc}{Callback function \refapi{pmix_spawn_cbfunc_t} (function reference)}
\argin{cbdata}{Data to be passed to the callback function (memory reference)}
\end{arglist}

\returnsimplenb

If executed, the status returned in the provided callback function will be one of the following constants:

\begin{itemize}
\item \refconst{PMIX_SUCCESS} All data was found and has been returned.
\item \refconst{PMIX_ERR_JOB_ALLOC_FAILED} The job request could not be executed due to failure to obtain the specified allocation
\item \refconst{PMIX_ERR_JOB_APP_NOT_EXECUTABLE} The specified application executable either could not be found, or lacks execution privileges.
\item \refconst{PMIX_ERR_JOB_NO_EXE_SPECIFIED} The job request did not specify an executable.
\item \refconst{PMIX_ERR_JOB_FAILED_TO_MAP} The launcher was unable to map the processes for the specified job request.
\item \refconst{PMIX_ERR_JOB_FAILED_TO_LAUNCH} One or more processes in the job request failed to launch
\item \refconst{PMIX_ERR_JOB_EXE_NOT_FOUND} Specified executable not found
\item \refconst{PMIX_ERR_JOB_INSUFFICIENT_RESOURCES} Insufficient resources to spawn job
\item \refconst{PMIX_ERR_JOB_SYS_OP_FAILED} System library operation failed
\item \refconst{PMIX_ERR_JOB_WDIR_NOT_FOUND} Specified working directory not found
\item a non-zero \ac{PMIx} error constant indicating a reason for the request's failure.
\end{itemize}

\reqattrstart
\ac{PMIx} libraries are not required to directly support any attributes for this function. However, any provided attributes must be passed to the host \ac{SMS} daemon for processing.

Host environments are required to support the following attributes when present in either the \refarg{job_info} or the \textit{info} array of an element of the \refarg{apps} array:

\pasteAttributeItem{PMIX_WDIR}
\pasteAttributeItem{PMIX_SET_SESSION_CWD}
\pasteAttributeItem{PMIX_PREFIX}
\pasteAttributeItem{PMIX_HOST}
\pasteAttributeItem{PMIX_HOSTFILE}

\reqattrend

\optattrstart
The following attributes are optional for \ac{PMIx} implementations:

\pasteAttributeItem{PMIX_ADD_HOSTFILE}
\pasteAttributeItem{PMIX_ADD_HOST}
\pasteAttributeItem{PMIX_PRELOAD_BIN}
\pasteAttributeItem{PMIX_PRELOAD_FILES}
\pasteAttributeItem{PMIX_PERSONALITY}
\pasteAttributeItem{PMIX_DISPLAY_MAP}
\pasteAttributeItem{PMIX_PPR}
\pasteAttributeItem{PMIX_MAPBY}
\pasteAttributeItem{PMIX_RANKBY}
\pasteAttributeItem{PMIX_BINDTO}
\pasteAttributeItem{PMIX_STDIN_TGT}
\pasteAttributeItem{PMIX_TAG_OUTPUT}
\pasteAttributeItem{PMIX_TIMESTAMP_OUTPUT}
\pasteAttributeItem{PMIX_MERGE_STDERR_STDOUT}
\pasteAttributeItem{PMIX_OUTPUT_TO_FILE}
\pasteAttributeItem{PMIX_INDEX_ARGV}
\pasteAttributeItem{PMIX_CPUS_PER_PROC}
\pasteAttributeItem{PMIX_NO_PROCS_ON_HEAD}
\pasteAttributeItem{PMIX_NO_OVERSUBSCRIBE}
\pasteAttributeItem{PMIX_REPORT_BINDINGS}
\pasteAttributeItem{PMIX_CPU_LIST}
\pasteAttributeItem{PMIX_JOB_RECOVERABLE}
\pasteAttributeItem{PMIX_JOB_CONTINUOUS}
\pasteAttributeItem{PMIX_MAX_RESTARTS}
\pasteAttributeItem{PMIX_SET_ENVAR}
\pasteAttributeItem{PMIX_UNSET_ENVAR}
\pasteAttributeItem{PMIX_ADD_ENVAR}
\pasteAttributeItem{PMIX_PREPEND_ENVAR}
\pasteAttributeItem{PMIX_APPEND_ENVAR}
\pasteAttributeItem{PMIX_FIRST_ENVAR}
\pasteAttributeItem{PMIX_ALLOC_QUEUE}
\pasteAttributeItem{PMIX_ALLOC_TIME}
\pasteAttributeItem{PMIX_ALLOC_NUM_NODES}
\pasteAttributeItem{PMIX_ALLOC_NODE_LIST}
\pasteAttributeItem{PMIX_ALLOC_NUM_CPUS}
\pasteAttributeItem{PMIX_ALLOC_NUM_CPU_LIST}
\pasteAttributeItem{PMIX_ALLOC_CPU_LIST}
\pasteAttributeItem{PMIX_ALLOC_MEM_SIZE}
\pasteAttributeItem{PMIX_ALLOC_BANDWIDTH}
\pasteAttributeItem{PMIX_ALLOC_FABRIC_QOS}
\pasteAttributeItem{PMIX_ALLOC_FABRIC_TYPE}
\pasteAttributeItem{PMIX_ALLOC_FABRIC_PLANE}
\pasteAttributeItem{PMIX_ALLOC_FABRIC_ENDPTS}
\pasteAttributeItem{PMIX_ALLOC_FABRIC_ENDPTS_NODE}
\pasteAttributeItem{PMIX_COSPAWN_APP}
\pasteAttributeItem{PMIX_SPAWN_TOOL}
\pasteAttributeItem{PMIX_EVENT_SILENT_TERMINATION}
\pasteAttributeItem{PMIX_ENVARS_HARVESTED}
\pasteAttributeItem{PMIX_JOB_TIMEOUT}
\pasteAttributeItem{PMIX_SPAWN_TIMEOUT}
\pasteAttributeItem{PMIX_NOTIFY_COMPLETION}
\pasteAttributeItem{PMIX_NOTIFY_PROC_TERMINATION}
\pasteAttributeItem{PMIX_NOTIFY_PROC_ABNORMAL_TERMINATION}
\pasteAttributeItem{PMIX_LOG_COMPLETION}
\pasteAttributeItem{PMIX_LOG_PROC_TERMINATION}
\pasteAttributeItem{PMIX_LOG_PROC_ABNORMAL_TERMINATION}
\pasteAttributeItem{PMIX_LOG_JOB_EVENTS}
\pasteAttributeItem{PMIX_LOG_COMPLETION}
\optattrend

%%%%
\descr

Nonblocking version of the \refapi{PMIx_Spawn} routine. The provided callback function will be executed upon successful start of \textit{all} specified application processes.

\adviceuserstart
Behavior of individual resource managers may differ, but it is expected that failure of any application process to start will result in termination/cleanup of all processes in the newly spawned job and return of an error code to the caller.
\adviceuserend

%%%%%%%%%%%%%%%%%%%%%%%%%%%%%%%%%%%%%%%%%%%%%%%%%
\subsubsection{Spawn Callback Function}
\declareapi{pmix_spawn_cbfunc_t}

%%%%
\summary

The \refapi{pmix_spawn_cbfunc_t} is used on the PMIx client side by \refapi{PMIx_Spawn_nb} and on the PMIx server side by \refapi{pmix_server_spawn_fn_t}.

\copySignature{pmix_spawn_cbfunc_t}{1.0}{
typedef void (*pmix_spawn_cbfunc_t) \\
\hspace*{4\sigspace}(pmix_status_t status, \\
\hspace*{5\sigspace}pmix_nspace_t nspace, void *cbdata);
}

\begin{arglist}
\argin{status}{Status associated with the operation (handle)}
\argin{nspace}{Namespace string (\refstruct{pmix_nspace_t})}
\argin{cbdata}{Callback data passed to original API call (memory reference)}
\end{arglist}


%%%%
\descr

The callback will be executed upon launch of the specified applications in \refapi{PMIx_Spawn_nb}, or upon failure to launch any of them.

The \refarg{status} of the callback will indicate whether or not the spawn succeeded.
The \refarg{nspace} of the spawned processes will be returned, along with any provided callback data.
Note that the returned \refarg{nspace} value will not be protected upon return from the callback function, so the receiver must copy it if it needs to be retained.

%%%%%%%%%%%%%%%%%%%%%%%%%%%%%%%%%%%%%%%%%%%%%%%%%
\subsection{Spawn-specific constants}
\label{api:struct:constants:spawn}

In addition to the generic error constants, the following spawn-specific error constants may be returned by the spawn \acp{API}:

\begin{constantdesc}
%
\declareconstitemvalue{PMIX_ERR_JOB_ALLOC_FAILED}{-188}
The job request could not be executed due to failure to obtain the specified allocation
%
\declareconstitemvalue{PMIX_ERR_JOB_APP_NOT_EXECUTABLE}{-177}
The specified application executable either could not be found, or lacks execution privileges.
%
\declareconstitemvalue{PMIX_ERR_JOB_NO_EXE_SPECIFIED}{-178}
The job request did not specify an executable.
%
\declareconstitemvalue{PMIX_ERR_JOB_FAILED_TO_MAP}{-179}
The launcher was unable to map the processes for the specified job request.
%
\declareconstitemvalue{PMIX_ERR_JOB_FAILED_TO_LAUNCH}{-181}
One or more processes in the job request failed to launch
%
\declareconstitemProvisional{PMIX_ERR_JOB_EXE_NOT_FOUND}
Specified executable not found
%
\declareconstitemProvisional{PMIX_ERR_JOB_INSUFFICIENT_RESOURCES}
Insufficient resources to spawn job
%
\declareconstitemProvisional{PMIX_ERR_JOB_SYS_OP_FAILED}
System library operation failed
%
\declareconstitemProvisional{PMIX_ERR_JOB_WDIR_NOT_FOUND}
Specified working directory not found
%
\end{constantdesc}

%%%%%%%%%%%%%%%%%%%%%%%%%%%%%%%%%%%%%%%%%%%%%%%%%
\subsection{Spawn attributes}
\label{api:struct:attributes:spawn}

Attributes used to describe \refapi{PMIx_Spawn} behavior - they are values passed to the \refapi{PMIx_Spawn} \ac{API} and therefore are not accessed using the \refapi{PMIx_Get} \acp{API} when used in that context. However, some of the attributes defined in this section can be provided by the host environment for other purposes - e.g., the host might provide the \refattr{PMIX_MAPBY} attribute in the job-related information so that an application can use \refapi{PMIx_Get} to discover the mapping used for determining process locations. Multi-use attributes and their respective access reference rank are denoted below.

%
\declareAttribute{PMIX_PERSONALITY}{"pmix.pers"}{char*}{
Name of personality corresponding to programming model used by application - supported values depend upon \ac{PMIx} implementation.
}
%
\declareAttribute{PMIX_HOST}{"pmix.host"}{char*}{
Comma-delimited list of hosts to use for spawned processes.
}
%
\declareAttribute{PMIX_HOSTFILE}{"pmix.hostfile"}{char*}{
Hostfile to use for spawned processes.  The format of this file is determined by the \ac{PMIx} implementation or host environment.  The file may not be portable across implementations.
}
%
\declareAttribute{PMIX_ADD_HOST}{"pmix.addhost"}{char*}{
Comma-delimited list of hosts to add to the allocation.
}
%
\declareAttribute{PMIX_ADD_HOSTFILE}{"pmix.addhostfile"}{char*}{
Hostfile containing hosts to add to existing allocation.  The format of this file is determined by the \ac{PMIx} implementation or host environment.  The file may not be portable across implementations.
}
%
\declareAttribute{PMIX_PREFIX}{"pmix.prefix"}{char*}{
Prefix to use for starting spawned processes - i.e., the directory where the executables can be found.
}
%
\declareAttribute{PMIX_WDIR}{"pmix.wdir"}{char*}{
Working directory for spawned processes.
}
%
\declareAttribute{PMIX_DISPLAY_MAP}{"pmix.dispmap"}{bool}{
Provide process mapping upon spawn.  Format of the provided map is implementation specific.
}
%
\declareAttribute{PMIX_PPR}{"pmix.ppr"}{char*}{
Number of processes to spawn on each identified resource.
}
%
\declareAttribute{PMIX_MAPBY}{"pmix.mapby"}{char*}{
Process mapping policy - when accessed using \refapi{PMIx_Get}, use the \refconst{PMIX_RANK_WILDCARD} value for the rank to discover the mapping policy used for the provided namespace. Supported values are launcher specific.
}
%
\declareAttribute{PMIX_RANKBY}{"pmix.rankby"}{char*}{
Process ranking policy - when accessed using \refapi{PMIx_Get}, use the \refconst{PMIX_RANK_WILDCARD} value for the rank to discover the ranking algorithm used for the provided namespace. Supported values are launcher specific.
}
%
\declareAttribute{PMIX_BINDTO}{"pmix.bindto"}{char*}{
Process binding policy - when accessed using \refapi{PMIx_Get}, use the \refconst{PMIX_RANK_WILDCARD} value for the rank to discover the binding policy used for the provided namespace. Supported values are launcher specific.
}
%
\declareAttribute{PMIX_PRELOAD_BIN}{"pmix.preloadbin"}{bool}{
Preload executables onto nodes prior to executing launch procedure.
}
%
\declareAttribute{PMIX_PRELOAD_FILES}{"pmix.preloadfiles"}{char*}{
Comma-delimited list of files to pre-position on nodes prior to executing launch procedure.
}
%
\declareAttribute{PMIX_STDIN_TGT}{"pmix.stdin"}{uint32_t}{
Spawned process rank that is to receive any forwarded \code{stdin}.
}
%
\declareAttribute{PMIX_SET_SESSION_CWD}{"pmix.ssncwd"}{bool}{
Set the current working directory to the session working directory assigned by the \ac{RM} - can be assigned to the entire job (by including attribute in the \refarg{job_info} array) or on a per-application basis in the \refarg{info} array for each \refstruct{pmix_app_t}.
}
%
\declareAttribute{PMIX_TAG_OUTPUT}{"pmix.tagout"}{bool}{
Tag \code{stdout}/\code{stderr} with the identity of the source process - can be assigned to the entire job (by including attribute in the \refarg{job_info} array) or on a per-application basis in the \refarg{info} array for each \refstruct{pmix_app_t}.  The format of how the text is tagged is implementation dependent.
}
%
\declareAttribute{PMIX_TIMESTAMP_OUTPUT}{"pmix.tsout"}{bool}{
Timestamp output - can be assigned to the entire job (by including attribute in the \refarg{job_info} array) or on a per-application basis in the \refarg{info} array for each \refstruct{pmix_app_t}.  The format of how the text is tagged is implementation dependent.
}
%
\declareAttribute{PMIX_MERGE_STDERR_STDOUT}{"pmix.mergeerrout"}{bool}{
Merge \code{stdout} and \code{stderr} streams - can be assigned to the entire job (by including attribute in the \refarg{job_info} array) or on a per-application basis in the \refarg{info} array for each \refstruct{pmix_app_t}.
}
%
\declareAttribute{PMIX_OUTPUT_TO_FILE}{"pmix.outfile"}{char*}{
Direct output (both stdout and stderr) into files of form \code{"<filename>.rank"} - can be assigned to the entire job (by including attribute in the \refarg{job_info} array) or on a per-application basis in the \refarg{info} array for each \refstruct{pmix_app_t}.
}
%
\declareAttribute{PMIX_OUTPUT_TO_DIRECTORY}{"pmix.outdir"}{char*}{
Direct output into files of form \code{"<directory>/\allowbreak <jobid>/\allowbreak rank.<rank>/\allowbreak stdout[err]"} - can be assigned to the entire job (by including attribute in the \refarg{job_info} array) or on a per-application basis in the \refarg{info} array for each \refstruct{pmix_app_t}.
}
%
\declareAttribute{PMIX_INDEX_ARGV}{"pmix.indxargv"}{bool}{
If set to true, will use the given name of the executable (\code{argv[0]}) as a base name and each rank will be invoked using the base name with the string "-<\emph{rank}>" appended to it, where \emph{rank} is the \ac{PMIx} rank of the process being invoked.
}
%
\declareAttribute{PMIX_CPUS_PER_PROC}{"pmix.cpuperproc"}{uint32_t}{
Number of \acp{PU} to assign to each rank - when accessed using \refapi{PMIx_Get}, use the \refconst{PMIX_RANK_WILDCARD} value for the rank to discover the \acp{PU}/process assigned to the provided namespace.
}
%
\declareAttribute{PMIX_NO_PROCS_ON_HEAD}{"pmix.nolocal"}{bool}{
Do not place processes on the head node.
}
%
\declareAttribute{PMIX_NO_OVERSUBSCRIBE}{"pmix.noover"}{bool}{
Do not oversubscribe the nodes - i.e., do not place more processes than allocated slots on a node.
}
%
\declareAttribute{PMIX_REPORT_BINDINGS}{"pmix.repbind"}{bool}{
Report bindings of the individual processes.  How and where this information is reported is implementation dependent as well as dependent on whether the processes are created through a launching tool or by a direct call to \refapi{PMIx_Spawn}.
}
%
\declareAttribute{PMIX_CPU_LIST}{"pmix.cpulist"}{char*}{
List of \acp{PU} to use for this job - when accessed using \refapi{PMIx_Get}, use the \refconst{PMIX_RANK_WILDCARD} value for the rank to discover the \ac{PU} list used for the provided namespace.
}
%
\declareAttribute{PMIX_JOB_RECOVERABLE}{"pmix.recover"}{bool}{
Application supports recoverable operations.
}
%
\declareAttribute{PMIX_JOB_CONTINUOUS}{"pmix.continuous"}{bool}{
Application is continuous, all failed processes should be immediately restarted.
}
%
\declareAttribute{PMIX_MAX_RESTARTS}{"pmix.maxrestarts"}{uint32_t}{
Maximum number of times to restart a process - when accessed using \refapi{PMIx_Get}, use the \refconst{PMIX_RANK_WILDCARD} value for the rank to discover the max restarts for the provided namespace.
}
%
\declareAttribute{PMIX_SPAWN_TOOL}{"pmix.spwn.tool"}{bool}{
Indicate that the job being spawned is a tool.  The repercussions of setting this attribute varies based on implementation.  For example, some implementations may not perform cpu-binding on process marked as a tool. 
}
%
\declareAttribute{PMIX_TIMEOUT_STACKTRACES}{"pmix.tim.stack"}{bool}{
Include process stacktraces in timeout report from a job.
}
%
\declareAttribute{PMIX_TIMEOUT_REPORT_STATE}{"pmix.tim.state"}{bool}{
Report process states in timeout report from a job.
}
%
\declareAttribute{PMIX_NOTIFY_JOB_EVENTS}{"pmix.note.jev"}{bool}{
Requests that the launcher generate the
\refconst{PMIX_EVENT_JOB_START}, \refconst{PMIX_LAUNCH_COMPLETE}, and
\refconst{PMIX_EVENT_JOB_END} events. Each event is to include at least the
namespace of the corresponding job and a \refattr{PMIX_EVENT_TIMESTAMP}
indicating the time the event occurred. Note that the requester must register
for these individual events, or capture
and process them by registering a default event handler instead of individual
handlers and then process the events based on the returned status code.
Another common method is to register one event handler for all job-related
events, with a separate handler for non-job events - see
\refapi{PMIx_Register_event_handler} for details.
}
%
\declareAttribute{PMIX_NOTIFY_COMPLETION}{"pmix.notecomp"}{bool}{
Requests that the launcher generate the \refconst{PMIX_EVENT_JOB_END} event
for normal or abnormal termination of the spawned job. The event shall include
the returned status code (\refattr{PMIX_JOB_TERM_STATUS}) for the
corresponding job; the identity (\refattr{PMIX_PROCID}) and exit status
(\refattr{PMIX_EXIT_CODE}) of the first failed process, if applicable; and a
\refattr{PMIX_EVENT_TIMESTAMP} indicating the time the termination occurred.
Note that the requester must register for the event or capture and process it
within a default event handler.
}
%
\declareAttribute{PMIX_NOTIFY_PROC_TERMINATION}{"pmix.noteproc"}{bool}{
Requests that the launcher generate the \refconst{PMIX_EVENT_PROC_TERMINATED}
event whenever a process either normally or abnormally terminates.
}
%
\declareAttribute{PMIX_NOTIFY_PROC_ABNORMAL_TERMINATION}{"pmix.noteabproc"}{bool}{
Requests that the launcher generate the \refconst{PMIX_EVENT_PROC_TERMINATED}
event only when a process abnormally terminates.
}
%
\declareAttribute{PMIX_LOG_PROC_TERMINATION}{"pmix.logproc"}{bool}{
Requests that the launcher log the \refconst{PMIX_EVENT_PROC_TERMINATED} event
whenever a process either normally or abnormally terminates.
}
%
\declareAttribute{PMIX_LOG_PROC_ABNORMAL_TERMINATION}{"pmix.logabproc"}{bool}{
Requests that the launcher log the \refconst{PMIX_EVENT_PROC_TERMINATED} event
only when a process abnormally terminates.
}
%
\declareAttribute{PMIX_LOG_JOB_EVENTS}{"pmix.log.jev"}{bool}{
Requests that the launcher log the \refconst{PMIX_EVENT_JOB_START},
\refconst{PMIX_LAUNCH_COMPLETE}, and \refconst{PMIX_EVENT_JOB_END} events using
\refapi{PMIx_Log}, subject to the logging attributes of Section
\ref{api:struct:attributes:log}.
}
%
\declareAttribute{PMIX_LOG_COMPLETION}{"pmix.logcomp"}{bool}{
Requests that the launcher log the \refconst{PMIX_EVENT_JOB_END} event
for normal or abnormal termination of the spawned job using
\refapi{PMIx_Log}, subject to the logging attributes of Section
\ref{api:struct:attributes:log}. The event shall include
the returned status code (\refattr{PMIX_JOB_TERM_STATUS}) for the
corresponding job; the identity (\refattr{PMIX_PROCID}) and exit status
(\refattr{PMIX_EXIT_CODE}) of the first failed process, if applicable; and a
\refattr{PMIX_EVENT_TIMESTAMP} indicating the time the termination occurred.
}
%
\declareAttribute{PMIX_EVENT_SILENT_TERMINATION}{"pmix.evsilentterm"}{bool}{
Do not generate a \refconst{PMIX_EVENT_JOB_END} event when this job normally terminates.
}
%
\declareAttributeProvisional{PMIX_ENVARS_HARVESTED}{"pmix.evar.hvstd"}{bool}{
Environmental parameters have been harvested by the spawn requestor - the server
does not need to harvest them.
}
%
\declareAttributeProvisional{PMIX_JOB_TIMEOUT}{"pmix.job.time"}{int}{
Time in seconds before the spawned job should time out and be terminated (0 => infinite), defined as the total runtime of the job (equivalent to the walltime limit of typical batch schedulers).
}
%
\declareAttributeProvisional{PMIX_SPAWN_TIMEOUT}{"pmix.sp.time"}{int}{
Time in seconds before spawn operation should time out (0 => infinite).
Logically equivalent to passing the \refattr{PMIX_TIMEOUT} attribute to the
\refapi{PMIx_Spawn} \ac{API}, it is provided as a separate attribute to distinguish
it from the \refattr{PMIX_JOB_TIMEOUT} attribute.
}

\vspace{\baselineskip}
Attributes used to adjust remote environment variables prior to spawning the specified application processes.

%
\declareAttribute{PMIX_SET_ENVAR}{"pmix.envar.set"}{pmix_envar_t*}{
Set the envar to the given value, overwriting any pre-existing one
}
%
\declareAttribute{PMIX_UNSET_ENVAR}{"pmix.envar.unset"}{char*}{
Unset the environment variable specified in the string.
}
%
\declareAttribute{PMIX_ADD_ENVAR}{"pmix.envar.add"}{pmix_envar_t*}{
Add the environment variable, but do not overwrite any pre-existing one
}
%
\declareAttribute{PMIX_PREPEND_ENVAR}{"pmix.envar.prepnd"}{pmix_envar_t*}{
Prepend the given value to the specified environmental value using the given separator character, creating the variable if it does not already exist
}
%
\declareAttribute{PMIX_APPEND_ENVAR}{"pmix.envar.appnd"}{pmix_envar_t*}{
Append the given value to the specified environmental value using the given separator character, creating the variable if it does not already exist
}
%
\declareAttribute{PMIX_FIRST_ENVAR}{"pmix.envar.first"}{pmix_envar_t*}{
Ensure the given value appears first in the specified envar using the separator character, creating the envar if it does not already exist
}

%%%%%%%%%%%%%%%%%%%%%%%%%%%%%%%%%%%%%%%%%%%%%%%%%
\subsection{Application Structure}
\declarestruct{pmix_app_t}

The \refstruct{pmix_app_t} structure describes the application context for the \refapi{PMIx_Spawn} and \refapi{PMIx_Spawn_nb} operations.

\copySignature{pmix_app_t}{1.0}{
typedef struct pmix_app \{ \\
\hspace*{4\sigspace}/** Executable */ \\
\hspace*{4\sigspace}char *cmd; \\
\hspace*{4\sigspace}/** Argument set, NULL terminated */ \\
\hspace*{4\sigspace}char **argv; \\
\hspace*{4\sigspace}/** Environment set, NULL terminated */ \\
\hspace*{4\sigspace}char **env; \\
\hspace*{4\sigspace}/** Current working directory */ \\
\hspace*{4\sigspace}char *cwd; \\
\hspace*{4\sigspace}/** Maximum processes with this profile */ \\
\hspace*{4\sigspace}int maxprocs; \\
\hspace*{4\sigspace}/** Array of info keys describing this application*/ \\
\hspace*{4\sigspace}pmix_info_t *info; \\
\hspace*{4\sigspace}/** Number of info keys in 'info' array */ \\
\hspace*{4\sigspace}size_t ninfo; \\
\} pmix_app_t;
}

%%%%%%%%%%%%%%%%%%%%%%%%%%%%%%%%%%%%%%%%%%%%%%%%%
\subsubsection{App structure support macros}
The following macros are provided to support the \refstruct{pmix_app_t} structure.

%%%%
\littleheader{Static initializer for the app structure}
\declaremacroProvisional{PMIX_APP_STATIC_INIT}

Provide a static initializer for the \refstruct{pmix_app_t} fields.

\versionMarker{5.0}
\cspecificstart
\begin{codepar}
PMIX_APP_STATIC_INIT
\end{codepar}
\cspecificend


%%%%%%%%%%%
\littleheader{Initialize the app structure}
\declaremacro{PMIX_APP_CONSTRUCT}

Initialize the \refstruct{pmix_app_t} fields

\copySignature{PMIX_APP_CONSTRUCT}{1.0}{
PMIX_APP_CONSTRUCT(m)
}

\begin{arglist}
\argin{m}{Pointer to the structure to be initialized (pointer to \refstruct{pmix_app_t})}
\end{arglist}

%%%%%%%%%%%
\littleheader{Destruct the app structure}
\declaremacro{PMIX_APP_DESTRUCT}

Destruct the \refstruct{pmix_app_t} fields

\copySignature{PMIX_APP_DESTRUCT}{1.0}{
PMIX_APP_DESTRUCT(m)
}

\begin{arglist}
\argin{m}{Pointer to the structure to be destructed (pointer to \refstruct{pmix_app_t})}
\end{arglist}

%%%%%%%%%%%
\littleheader{Create an app array}
\declaremacro{PMIX_APP_CREATE}

Allocate and initialize an array of \refstruct{pmix_app_t} structures

\copySignature{PMIX_APP_CREATE}{1.0}{
PMIX_APP_CREATE(m, n)
}

\begin{arglist}
\arginout{m}{Address where the pointer to the array of \refstruct{pmix_app_t} structures shall be stored (handle)}
\argin{n}{Number of structures to be allocated (\code{size_t})}
\end{arglist}

%%%%%%%%%%%
\littleheader{Free an app structure}
\declaremacro{PMIX_APP_RELEASE}

Release a \refstruct{pmix_app_t} structure

\copySignature{PMIX_APP_RELEASE}{4.0}{
PMIX_APP_RELEASE(m)
}

\begin{arglist}
\argin{m}{Pointer to a \refstruct{pmix_app_t} structure (handle)}
\end{arglist}

%%%%%%%%%%%
\littleheader{Free an app array}
\declaremacro{PMIX_APP_FREE}

Release an array of \refstruct{pmix_app_t} structures

\copySignature{PMIX_APP_FREE}{1.0}{
PMIX_APP_FREE(m, n)
}

\begin{arglist}
\argin{m}{Pointer to the array of \refstruct{pmix_app_t} structures (handle)}
\argin{n}{Number of structures in the array (\code{size_t})}
\end{arglist}

%%%%%%%%%%%
\littleheader{Create the info array of application directives}
\declaremacro{PMIX_APP_INFO_CREATE}

Create an array of \refstruct{pmix_info_t} structures for passing application-level directives, updating the \refarg{ninfo} field of the \refstruct{pmix_app_t} structure.

\copySignature{PMIX_APP_INFO_CREATE}{2.2}{
PMIX_APP_INFO_CREATE(m, n)
}

\begin{arglist}
\argin{m}{Pointer to the \refstruct{pmix_app_t} structure (handle)}
\argin{n}{Number of directives to be allocated (\code{size_t})}
\end{arglist}



%%%%%%%%%%%%%%%%%%%%%%%%%%%%%%%%%%%%%%%%%%%%%%%%%
%%%%%%%%%%%%%%%%%%%%%%%%%%%%%%%%%%%%%%%%%%%%%%%%%
\section{Abort}
\label{chap:api_proc_mgmt:abort}

\ac{PMIx} provides a dedicated API by which an application can request that specified processes be aborted by the system.

%%%%%%%%%%%%%%%%%%%%%%%%%%%%%%%%%%%%%%%%%%%%%%%%%
\subsection{\code{PMIx_Abort}}
\declareapi{PMIx_Abort}

%%%%
\summary

Abort the specified processes

%%%%
\format

\copySignature{PMIx_Abort}{1.0}{
pmix_status_t \\
PMIx_Abort(int status, const char msg[], \\
\hspace*{11\sigspace}pmix_proc_t procs[], size_t nprocs)
}

\begin{arglist}
\argin{status}{Error code to return to invoking environment (integer)}
\argin{msg}{String message to be returned to user (string)}
\argin{procs}{Array of \refstruct{pmix_proc_t} structures (array of handles)}
\argin{nprocs}{Number of elements in the \refarg{procs} array (integer)}
\end{arglist}

A successful return indicates that the requested processes are in a terminated state.  Note that the function shall not return in this situation if the caller's own process was included in the request.

\returnstart
\begin{itemize}
    \item \refconst{PMIX_ERR_PARAM_VALUE_NOT_SUPPORTED} if the \ac{PMIx} implementation and host environment support this \ac{API}, but the request includes processes that the host environment cannot abort - e.g., if the request is to abort subsets of processes from a namespace, or processes outside of the caller's own namespace, and the host environment does not permit such operations. In this case, none of the specified processes will be terminated.
\end{itemize}
\returnend

%%%%
\descr
Request that the host resource manager abort the provided array of procs.   If the design of the host resource manager allows, the provided message should be associated with any record it prints or logs of the operation.   
If the processes were launched by an application designed to launch the processes and which exists for the lifetime of the processes, than this application should terminate with the return code provided if the system allows.
A \code{NULL} for the \refarg{procs} array indicates that all processes in the caller's namespace are to be aborted, including itself - this is the equivalent of passing a \refstruct{pmix_proc_t} array element containing the caller's namespace and a rank value of \refconst{PMIX_RANK_WILDCARD}. While it is permitted for a caller to request abort of processes from namespaces other than its own, not all environments will support such requests.
Passing a \code{NULL} \refarg{msg} parameter is allowed.

The function shall not return until the host environment has carried out the operation on the specified processes. If the caller is included in the array of targets, then the function will not return unless the host is unable to execute the operation.

\adviceuserstart
The response to this request is somewhat dependent on the specific \ac{RM} and its configuration (e.g., some resource managers will not abort the application if the provided status is zero unless specifically configured to do so, some cannot abort subsets of processes in an application, and some may not permit termination of processes outside of the caller's own namespace), and thus lies outside the control of PMIx itself.
However, the PMIx client library shall inform the \ac{RM} of the request that the specified \refarg{procs} be aborted, regardless of the value of the provided status.

Note that race conditions caused by multiple processes calling \refapi{PMIx_Abort} are left to the server implementation to resolve with regard to which status is returned and what messages (if any) are printed.
\adviceuserend


%%%%%%%%%%%%%%%%%%%%%%%%%%%%%%%%%%%%%%%%%%%%%%%%%
%%%%%%%%%%%%%%%%%%%%%%%%%%%%%%%%%%%%%%%%%%%%%%%%%
\section{Connecting and Disconnecting Processes}
\label{chap:api_proc_mgmt:connect}

This section defines functions to connect and disconnect processes in two or more separate \ac{PMIx} namespaces. 
The \ac{PMIx} definition of \textit{connected} solely implies that the host environment should treat the failure of any process in the assemblage as a reportable event, taking action on the assemblage as if it were a single application. 
The call requests that \ac{PMIx}, together with the \ac{RM}, should treat connected processes as a single assemblage for the purposes of event notification and responses to abnormal process termination.
For example, if the environment defaults (in the absence of any application directives) to terminating an application upon failure of any process in that application, then the environment should terminate all processes in the connected assemblage upon failure of any member.

The host environment may choose to assign a new namespace to the connected assemblage and/or assign new ranks for its members for its own internal tracking purposes.  For implementations which use this approach, it is up to the implementation whether such namespaces are exposed to users or clients (e.g., in response to an appropriate call to \refapi{PMIx_Query_info_nb}). The host environment is required to generate a \refconst{PMIX_ERR_PROC_TERM_WO_SYNC} event should any process in the assemblage terminate or call \refapi{PMIx_Finalize} without first \textit{disconnecting} from the assemblage. If the job including the process is terminated as a result of that action, then the host environment is required to also generate the \refconst{PMIX_ERR_JOB_TERM_WO_SYNC} for all jobs that were terminated as a result.

\adviceuserstart
Attempting to \textit{connect} processes solely within the same namespace is essentially a \textit{no-op} operation. While not explicitly prohibited, users are advised that a \ac{PMIx} implementation or host environment may return an error in such cases.
\adviceuserend

\advicermstart
The \textit{connect} operation does not require the exchange of job-level information nor the inclusion of information posted by  participating processes via \refapi{PMIx_Put}. Indeed, the callback function utilized in \refapi{pmix_server_connect_fn_t} cannot pass information back into the \ac{PMIx} server library. However, host environments are advised that collecting such information at the participating daemons represents an optimization opportunity as participating processes are likely to request such information after the connect operation completes.
\advicermend


%%%%%%%%%%%%%%%%%%%%%%%%%%%%%%%%%%%%%%%%%%%%%%%%%
\subsection{\code{PMIx_Connect}}
\declareapi{PMIx_Connect}

%%%%
\summary

Connect namespaces.

%%%%
\format

\copySignature{PMIx_Connect}{1.0}{
pmix_status_t \\
PMIx_Connect(const pmix_proc_t procs[], size_t nprocs, \\
\hspace*{13\sigspace}const pmix_info_t info[], size_t ninfo)
}

\begin{arglist}
\argin{procs}{Array of proc structures (array of handles)}
\argin{nprocs}{Number of elements in the \refarg{procs} array (integer)}
\argin{info}{Array of info structures (array of handles)}
\argin{ninfo}{Number of elements in the \refarg{info} array (integer)}
\end{arglist}

\returnsimple

\reqattrstart
\ac{PMIx} libraries are not required to directly support any attributes for this function. However, any provided attributes must be passed to the host \ac{SMS} daemon for processing.

\reqattrend

\optattrstart
The following attributes are optional for \ac{PMIx} implementations:

\pasteAttributeItem{PMIX_ALL_CLONES_PARTICIPATE}
\pasteAttributeItem{PMIX_TIMEOUT}

\optattrend

%%%%
\descr

Record the processes specified by the \refarg{procs} array as \textit{connected}.  
The \ac{PMIx} definition of \textit{connected} solely implies that the host environment should treat the failure of any process in the assemblage as a reportable event, taking action on the assemblage as if it were a single application. 
The function will return once all processes identified in \refarg{procs} have called either \refapi{PMIx_Connect} or its non-blocking version, \textit{and} the host environment has completed any supporting operations required to meet the terms of the \ac{PMIx} definition of \textit{connected} processes.

A process can only engage in one connect operation involving the identical \refarg{procs} array at a time.
However, a process can be simultaneously engaged in multiple connect operations, each involving a different \refarg{procs} array.

As in the case of the \refapi{PMIx_Fence} operation, the \refarg{info} array can be used to pass user-level directives regarding timeout constraints and other options available from the host \ac{RM}.

\adviceuserstart
All processes engaged in a given \refapi{PMIx_Connect} operation must provide the identical \refarg{procs} array as ordering of entries in the array and the method by which those processes are identified (e.g., use of \refconst{PMIX_RANK_WILDCARD} versus listing the individual processes) \textit{may} impact the host environment's algorithm for uniquely identifying an operation.
\adviceuserend

\adviceimplstart
\refapi{PMIx_Connect} and its non-blocking form are both \emph{collective} operations. Accordingly, the \ac{PMIx} server library is required to aggregate participation by local clients, passing the request to the host environment once all local participants have executed the \ac{API}.
\adviceimplend

\advicermstart
The host will receive a single call for each collective operation. It is the responsibility of the host to identify the nodes containing participating processes, execute the collective across all participating nodes, and notify the local \ac{PMIx} server library upon completion of the global collective.
\advicermend


%%%%%%%%%%%%%%%%%%%%%%%%%%%%%%%%%%%%%%%%%%%%%%%%%
\subsection{\code{PMIx_Connect_nb}}
\declareapi{PMIx_Connect_nb}

%%%%
\summary

Nonblocking \refapi{PMIx_Connect_nb} routine.

%%%%
\format

\copySignature{PMIx_Connect_nb}{1.0}{
pmix_status_t \\
PMIx_Connect_nb(const pmix_proc_t procs[], size_t nprocs, \\
\hspace*{16\sigspace}const pmix_info_t info[], size_t ninfo, \\
\hspace*{16\sigspace}pmix_op_cbfunc_t cbfunc, void *cbdata)
}

\begin{arglist}
\argin{procs}{Array of proc structures (array of handles)}
\argin{nprocs}{Number of elements in the \refarg{procs} array (integer)}
\argin{info}{Array of info structures (array of handles)}
\argin{ninfo}{Number of elements in the \refarg{info} array (integer)}
\argin{cbfunc}{Callback function \refapi{pmix_op_cbfunc_t} (function reference)}
\argin{cbdata}{Data to be passed to the callback function (memory reference)}
\end{arglist}

\returnsimplenb

\returnstart
\begin{itemize}
    \item \refconst{PMIX_OPERATION_SUCCEEDED}, indicating that the request was immediately processed and returned \textit{success} - the \refarg{cbfunc} will \textit{not} be called
\end{itemize}
\returnend

\reqattrstart
\ac{PMIx} libraries are not required to directly support any attributes for this function. However, any provided attributes must be passed to the host \ac{SMS} daemon for processing.

\reqattrend

\optattrstart
The following attributes are optional for \ac{PMIx} implementations:

\pasteAttributeItem{PMIX_ALL_CLONES_PARTICIPATE}


The following attributes are optional for host environments that support this operation:

\pasteAttributeItem{PMIX_TIMEOUT}

\optattrend

%%%%
\descr

Nonblocking version of \refapi{PMIx_Connect}. The callback function is called once all processes identified in \refarg{procs} have called either \refapi{PMIx_Connect} or its non-blocking version, \textit{and} the host environment has completed any supporting operations required to meet the terms of the \ac{PMIx} definition of \textit{connected} processes. See the advice provided in the description for \refapi{PMIx_Connect} for more information.


%%%%%%%%%%%%%%%%%%%%%%%%%%%%%%%%%%%%%%%%%%%%%%%%%
\subsection{\code{PMIx_Disconnect}}
\declareapi{PMIx_Disconnect}

%%%%
\summary

Disconnect a previously connected set of processes.

%%%%
\format

\copySignature{PMIx_Disconnect}{1.0}{
pmix_status_t \\
PMIx_Disconnect(const pmix_proc_t procs[], size_t nprocs, \\
\hspace*{16\sigspace}const pmix_info_t info[], size_t ninfo);
}

\begin{arglist}
\argin{procs}{Array of proc structures (array of handles)}
\argin{nprocs}{Number of elements in the \refarg{procs} array (integer)}
\argin{info}{Array of info structures (array of handles)}
\argin{ninfo}{Number of elements in the \refarg{info} array (integer)}
\end{arglist}

\returnstart
\begin{itemize}
    \item the \refconst{PMIX_ERR_INVALID_OPERATION} error indicating that the specified set of \refarg{procs} was not previously \textit{connected} via a call to \refapi{PMIx_Connect} or its non-blocking form.
\end{itemize}
\returnend

\reqattrstart
\ac{PMIx} libraries are not required to directly support any attributes for this function. However, any provided attributes must be passed to the host \ac{SMS} daemon for processing.

\reqattrend

\optattrstart
The following attributes are optional for \ac{PMIx} implementations:

\pasteAttributeItem{PMIX_ALL_CLONES_PARTICIPATE}


The following attributes are optional for host environments that support this operation:

\pasteAttributeItem{PMIX_TIMEOUT}

\optattrend

%%%%
\descr

Disconnect a previously connected set of processes. The function will return once all processes identified in \refarg{procs} have called either \refapi{PMIx_Disconnect} or its non-blocking version, \textit{and} the host environment has completed any required supporting operations.

A process can only engage in one disconnect operation involving the identical \refarg{procs} array at a time.
However, a process can be simultaneously engaged in multiple disconnect operations, each involving a different \refarg{procs} array.

As in the case of the \refapi{PMIx_Fence} operation, the \refarg{info} array can be used to pass user-level directives regarding timeout constraints, and other options available from the host \ac{RM}.

\adviceuserstart
All processes engaged in a given \refapi{PMIx_Disconnect} operation must provide the identical \refarg{procs} array as ordering of entries in the array and the method by which those processes are identified (e.g., use of \refconst{PMIX_RANK_WILDCARD} versus listing the individual processes) \textit{may} impact the host environment's algorithm for uniquely identifying an operation.
\adviceuserend

\adviceimplstart
\refapi{PMIx_Disconnect} and its non-blocking form are both \emph{collective} operations. Accordingly, the \ac{PMIx} server library is required to aggregate participation by local clients, passing the request to the host environment once all local participants have executed the \ac{API}.
\adviceimplend

\advicermstart
The host will receive a single call for each collective operation. The host will receive a single call for each collective operation. It is the responsibility of the host to identify the nodes containing participating processes, execute the collective across all participating nodes, and notify the local \ac{PMIx} server library upon completion of the global collective.

\advicermend


%%%%%%%%%%%%%%%%%%%%%%%%%%%%%%%%%%%%%%%%%%%%%%%%%
\subsection{\code{PMIx_Disconnect_nb}}
\declareapi{PMIx_Disconnect_nb}

%%%%
\summary

Nonblocking \refapi{PMIx_Disconnect} routine.

%%%%
\format

\copySignature{PMIx_Disconnect_nb}{1.0}{
pmix_status_t \\
PMIx_Disconnect_nb(const pmix_proc_t procs[], size_t nprocs, \\
\hspace*{19\sigspace}const pmix_info_t info[], size_t ninfo, \\
\hspace*{19\sigspace}pmix_op_cbfunc_t cbfunc, void *cbdata);
}

\begin{arglist}
\argin{procs}{Array of proc structures (array of handles)}
\argin{nprocs}{Number of elements in the \refarg{procs} array (integer)}
\argin{info}{Array of info structures (array of handles)}
\argin{ninfo}{Number of elements in the \refarg{info} array (integer)}
\argin{cbfunc}{Callback function \refapi{pmix_op_cbfunc_t} (function reference)}
\argin{cbdata}{Data to be passed to the callback function (memory reference)}
\end{arglist}

\returnsimplenb

\returnstart
\begin{itemize}
    \item \refconst{PMIX_OPERATION_SUCCEEDED}, indicating that the request was immediately processed and returned \textit{success} - the \refarg{cbfunc} will \textit{not} be called
\end{itemize}
\returnend

\reqattrstart
\ac{PMIx} libraries are not required to directly support any attributes for this function. However, any provided attributes must be passed to the host \ac{SMS} daemon for processing.

\reqattrend

\optattrstart
The following attributes are optional for \ac{PMIx} implementations:

\pasteAttributeItem{PMIX_ALL_CLONES_PARTICIPATE}


The following attributes are optional for host environments that support this operation:

\pasteAttributeItem{PMIX_TIMEOUT}

\optattrend

%%%%
\descr

Nonblocking \refapi{PMIx_Disconnect} routine. The callback function is called either:

\begin{itemize}
    \item to return the \refconst{PMIX_ERR_INVALID_OPERATION} error indicating that the specified set of \refarg{procs} was not previously \textit{connected} via a call to \refapi{PMIx_Connect} or its non-blocking form;

    \item to return a \ac{PMIx} error constant indicating that the operation failed; or

    \item once all processes identified in \refarg{procs} have called either \refapi{PMIx_Disconnect_nb} or its blocking version, \textit{and} the host environment has completed any required supporting operations.
\end{itemize}

See the advice provided in the description for \refapi{PMIx_Disconnect} for more information.


%%%%%%%%%%%%%%%%%%%%%%%%%%%%%%%%%%%%%%%%%%%%%%%%%
%%%%%%%%%%%%%%%%%%%%%%%%%%%%%%%%%%%%%%%%%%%%%%%%%
\section{Process Locality}
\label{chap:api_proc_mgmt:locality}

The relative locality of processes is often used to optimize their interactions with the hardware and other processes. \ac{PMIx} provides a means by which the host environment can communicate the locality of a given process using the \refapi{PMIx_server_generate_locality_string} to generate an abstracted representation of that value. This provides a human-readable format and allows the client to parse the locality string with a method of its choice that may differ from the one used by the server that generated it.

There are times, however, when relative locality and other \ac{PMIx}-provided
information does not include some element required by the application. In these
instances, the application may need access to the full description of the
local hardware topology. 
\ac{PMIx} is not required to generate a description but offers an abstraction 
method by which users can obtain a pointer to
the description. This transparently enables support for different methods of
sharing the topology between the host environment (which may well have already
generated it prior to the local start of application processes) and the clients -
e.g., through passing of a shared memory region.

%%%%%%%%%%%%%%%%%%%%%%%%%%%%%%%%%%%%%%%%%%%%%%%%%
\subsection{\code{PMIx_Load_topology}}
\declareapi{PMIx_Load_topology}

%%%%
\summary

Load the local hardware topology description

%%%%
\format

\copySignature{PMIx_Load_topology}{4.0}{
pmix_status_t \\
PMIx_Load_topology(pmix_topology_t *topo);
}

\begin{arglist}
\arginout{topo}{Address of a \refstruct{pmix_topology_t} structure where the topology information is to be loaded (handle)}
\end{arglist}

A successful return indicates that the \refarg{topo} was successfully loaded.

\returnsimple

%%%%
\descr

Obtain a pointer to the topology description of the local node. If the
\refarg{source} field of the provided \refstruct{pmix_topology_t} is set, then
the \ac{PMIx} library must return a description from the specified
implementation or else indicate that the implementation is not available by
returning the \refconst{PMIX_ERR_NOT_SUPPORTED} error constant.

The description should be treated as a "read-only" object and attempts to modify it may result in errors.
The \ac{PMIx} library is responsible for performing any required cleanup when the client library finalizes.

\adviceuserstart
It is the responsibility of the user to ensure that the \refarg{topo} argument
is properly initialized prior to calling this \ac{API}, and if no \refarg{source} was specified, to check the
returned \refarg{source} to verify that the returned topology description is
compatible with the user's code.
\adviceuserend

%%%%%%%%%%%%%%%%%%%%%%%%%%%%%%%%%%%%%%%%%%%%%%%%%
\subsection{\code{PMIx_Get_relative_locality}}
\declareapi{PMIx_Get_relative_locality}

%%%%
\summary

Get the relative locality of two local processes given their locality strings.

%%%%
\format

\copySignature{PMIx_Get_relative_locality}{4.0}{
pmix_status_t \\
PMIx_Get_relative_locality(const char *locality1, \\
\hspace*{27\sigspace}const char *locality2, \\
\hspace*{27\sigspace}pmix_locality_t *locality);
}

\begin{arglist}
\argin{locality1}{\refattr{PMIX_LOCALITY_STRING} associated with the first process}
\argin{locality2}{\refattr{PMIX_LOCALITY_STRING} associated with the second process}
\arginout{locality}{Location where the relative locality bitmask is to be constructed (memory reference)}
\end{arglist}

A successful return indicates that the \refarg{locality} was successfully loaded.

\returnsimple

%%%%
\descr

Parse the locality strings of two processes (as returned by \refapi{PMIx_Get} using the \refattr{PMIX_LOCALITY_STRING} key) and set the appropriate \refstruct{pmix_locality_t} locality bits in the provided memory location.

%%%%%%%%%%%%%%%%%%%%%%%%%%%%%%%%%%%%%%%%%%%%%%%%%
\subsubsection{Topology description}
\declarestruct{pmix_topology_t}

The \refstruct{pmix_topology_t} structure contains a (case-insensitive)
string identifying the source of the topology (e.g., "hwloc") and a pointer
to the corresponding implementation-specific topology description.

\copySignature{pmix_topoology_t}{4.0}{
typedef struct pmix_topology \{ \\
\hspace*{4\sigspace}char *source; \\
\hspace*{4\sigspace}void *topology; \\
\} pmix_topoology_t;
}

%%%%%%%%%%%%%%%%%%%%%%%%%%%%%%%%%%%%%%%%%%%%%%%%%
\subsubsection{Topology support macros}

The following macros support the \refstruct{pmix_topology_t} structure.

%%%%
\littleheader{Static initializer for the topology structure}
\declaremacroProvisional{PMIX_TOPOLOGY_STATIC_INIT}

Provide a static initializer for the \refstruct{pmix_topology_t} fields.

\versionMarker{5.0}
\cspecificstart
\begin{codepar}
PMIX_TOPOLOGY_STATIC_INIT
\end{codepar}
\cspecificend


\littleheader{Initialize the topology structure}
\declaremacro{PMIX_TOPOLOGY_CONSTRUCT}

Initialize the \refstruct{pmix_topology_t} fields to \code{NULL}

\copySignature{PMIX_TOPOLOGY_CONSTRUCT}{4.0}{
PMIX_TOPOLOGY_CONSTRUCT(m)
}

\begin{arglist}
\argin{m}{Pointer to the structure to be initialized (pointer to \refstruct{pmix_topology_t})}
\end{arglist}


\littleheader{Destruct a topology structure}
\declareapi{PMIx_Topology_destruct}

%%%%
\summary

Destruct a \refstruct{pmix_topology_t} fields

%%%%
\format

\versionMarker{5.0}
\cspecificstart
\begin{codepar}
void
PMIx_Topology_destruct(pmix_topology_t *topo);
\end{codepar}
\cspecificend

\begin{arglist}
\argin{topo}{Pointer to the structure to be destructed (pointer to \refstruct{pmix_topology_t})}
\end{arglist}

%%%%
\descr

Release any memory storage held by the \refstruct{pmix_topology_t} structure

\littleheader{Create a topology array}
\declaremacro{PMIX_TOPOLOGY_CREATE}

Allocate and initialize a \refstruct{pmix_topology_t} array.

\copySignature{PMIX_TOPOLOGY_CREATE}{4.0}{
PMIX_TOPOLOGY_CREATE(m, n)
}

\begin{arglist}
\arginout{m}{Address where the pointer to the array of \refstruct{pmix_topology_t} structures shall be stored (handle)}
\argin{n}{Number of structures to be allocated (size_t)}
\end{arglist}


%%%%%%%%%%%%%%%%%%%%%%%%%%%%%%%%%%%%%%%%%%%%%%%%%
\subsubsection{Relative locality of two processes}
\declarestruct{pmix_locality_t}
\label{api:proc:locality}

\versionMarker{4.0}
The \refstruct{pmix_locality_t} datatype is a \code{uint16_t} bitmask that
defines the relative locality of two processes on a node. The following
constants represent specific bits in the mask and can be used to test a
locality value using standard bit-test methods.

\begin{constantdesc}
%
\declareconstitemvalue{PMIX_LOCALITY_UNKNOWN}{0x0000}
All bits are set to zero, indicating that the relative locality of the two processes is unknown
%
\declareconstitemvalue{PMIX_LOCALITY_NONLOCAL}{0x0000}
The two processes do not share any common locations
%
\declareconstitemvalue{PMIX_LOCALITY_SHARE_HWTHREAD}{0x0001}
The two processes share at least one hardware thread
%
\declareconstitemvalue{PMIX_LOCALITY_SHARE_CORE}{0x0002}
The two processes share at least one core
%
\declareconstitemvalue{PMIX_LOCALITY_SHARE_L1CACHE}{0x0004}
The two processes share at least an L1 cache
%
\declareconstitemvalue{PMIX_LOCALITY_SHARE_L2CACHE}{0x0008}
The two processes share at least an L2 cache
%
\declareconstitemvalue{PMIX_LOCALITY_SHARE_L3CACHE}{0x0010}
The two processes share at least an L3 cache
%
\declareconstitemvalue{PMIX_LOCALITY_SHARE_PACKAGE}{0x0020}
The two processes share at least a package
%
\declareconstitemvalue{PMIX_LOCALITY_SHARE_NUMA}{0x0040}
The two processes share at least one \ac{NUMA} region
%
\declareconstitemvalue{PMIX_LOCALITY_SHARE_NODE}{0x4000}
The two processes are executing on the same node
%
\end{constantdesc}


Implementers and vendors may choose to extend these definitions as needed to describe a particular system.


%%%%%%%%%%%%%%%%%%%%%%%%%%%%%%%%%%%%%%%%%%%%%%%%%
\subsubsection{Locality keys}

%
\declareAttribute{PMIX_LOCALITY_STRING}{"pmix.locstr"}{char*}{
String describing a process's bound location - referenced using the process's
rank. The string is prefixed by the implementation that created it (e.g.,
"hwloc") followed by a colon. The remainder of the string represents the
corresponding locality as expressed by the underlying implementation. The
entire string must be passed to \refapi{PMIx_Get_relative_locality} for
processing. Note that hosts are only required to provide locality strings for
local client processes - thus, a call to \refapi{PMIx_Get} for the locality
string of a process that returns \refconst{PMIX_ERR_NOT_FOUND} indicates that
the process is not executing on the same node.
}

%%%%%%%%%%%%%%%%%%%%%%%%%%%%%%%%%%%%%%%%%%%%%%%%%
\subsection{\code{PMIx_Parse_cpuset_string}}
\declareapi{PMIx_Parse_cpuset_string}

%%%%
\summary

Parse the \ac{PU} binding bitmap from its string representation.

%%%%
\format

\copySignature{PMIx_Parse_cpuset_string}{4.0}{
pmix_status_t \\
PMIx_Parse_cpuset_string(const char *cpuset_string, \\
\hspace*{25\sigspace}pmix_cpuset_t *cpuset);
}

\begin{arglist}
\argin{cpuset_string}{\refattr{PMIX_CPUSET} string associated with a \ac{PMIx} process}
\argin{cpuset_string}{String returned by the \refapi{PMIx_server_generate_cpuset_string} \ac{API} (handle)}
\arginout{cpuset}{Address of an object where the bitmap is to be stored (memory reference)}
\end{arglist}

A successful return indicates that the \refarg{cpuset} was successfully loaded.

\returnsimple

%%%%
\descr

Parse the string representation of the binding bitmap (as returned by \refapi{PMIx_Get} using the \refattr{PMIX_CPUSET} key) and set the appropriate \ac{PU} binding location information in the provided memory location.
If the \refarg{source} field of \refarg{cpuset} does not match with the underlying source that provided the binding bitmap, an error will be returned.

%%%%%%%%%%%%%%%%%%%%%%%%%%%%%%%%%%%%%%%%%%%%%%%%%
\subsection{\code{PMIx_Get_cpuset}}
\declareapi{PMIx_Get_cpuset}

%%%%
\summary

Get the \ac{PU} binding bitmap of the current process.

%%%%
\format

\copySignature{PMIx_Get_cpuset}{4.0}{
pmix_status_t \\
PMIx_Get_cpuset(pmix_cpuset_t *cpuset, pmix_bind_envelope_t ref);
}

\begin{arglist}
\arginout{cpuset}{Address of an object where the bitmap is to be stored (memory reference)}
\argin{ref}{The binding envelope to be considered when formulating the bitmap (\refstruct{pmix_bind_envelope_t})}
\end{arglist}

A successful return indicates that the \refarg{cpuset} was successfully loaded.

\returnsimple

%%%%
\descr

Obtain and set the appropriate \ac{PU} binding location information in the provided memory location based on the specified binding envelope.

%%%%%%%%%%%%%%%%%%%%%%%%%%%%%%%%%%%%%%%%%%%%%%%%%
\subsubsection{Binding envelope}
\declarestruct{pmix_bind_envelope_t}
\label{api:proc:bindenv}

\versionMarker{4.0}
The \refstruct{pmix_bind_envelope_t} data type
defines the envelope of threads within a possibly multi-threaded process that are to be considered when getting the cpuset associated with the process. Valid values include:

\begin{constantdesc}
%
\declareconstitemvalue{PMIX_CPUBIND_PROCESS}{0}
Use the location of all threads in the possibly multi-threaded process.
%
\declareconstitemvalue{PMIX_CPUBIND_THREAD}{1}
Use only the location of the thread calling the \ac{API}.
%
\end{constantdesc}


%%%%%%%%%%%%%%%%%%%%%%%%%%%%%%%%%%%%%%%%%%%%%%%%%
\subsection{\code{PMIx_Compute_distances}}
\declareapi{PMIx_Compute_distances}

%%%%
\summary

Compute distances from specified process location to local devices.

%%%%
\format

\copySignature{PMIx_Compute_distances}{4.0}{
pmix_status_t \\
PMIx_Compute_distances(pmix_topology_t *topo, \\
\hspace*{23\sigspace}pmix_cpuset_t *cpuset, \\
\hspace*{23\sigspace}pmix_info_t info[], size_t ninfo[], \\
\hspace*{23\sigspace}pmix_device_distance_t *distances[], \\
\hspace*{23\sigspace}size_t *ndist);
}

\begin{arglist}
\argin{topo}{Pointer to the topology description of the node where the process is located (\code{NULL} indicates the local node) (\refstruct{pmix_topology_t})}
\argin{cpuset}{Pointer to the location of the process (\refstruct{pmix_cpuset_t})}
\argin{info}{Array of \refstruct{pmix_info_t} describing the devices whose distance is to be computed (handle)}
\argin{ninfo}{Number of elements in \refarg{info} (integer)}
\arginout{distances}{Pointer to an address where the array of \refstruct{pmix_device_distance_t} structures containing the distances from the caller to the specified devices is to be returned (handle)}
\arginout{ndist}{Pointer to an address where the number of elements in the \refarg{distances} array is to be returned (handle)}
\end{arglist}

\returnsimple

\optattrstart
The following attributes are optional for \ac{PMIx} implementations:

\pasteAttributeItem{PMIX_DEVICE_DISTANCES}
\pasteAttributeItem{PMIX_DEVICE_ID}
\pasteAttributeItem{PMIX_DEVICE_TYPE}

\optattrend

%%%%
\descr

Both the minimum and maximum distance fields in the elements of the array shall be filled with the respective distances between the current process location and the types of devices or specific device identified in the \refarg{info} directives. In the absence of directives, distances to all device types supported by the underlying topology description shall be returned.

\adviceuserstart
A process whose threads are not all bound to the same location may return inconsistent results from calls to this \ac{API} by different threads if the \refconst{PMIX_CPUBIND_THREAD} binding envelope was used when generating the \refarg{cpuset}.
\adviceuserend


%%%%%%%%%%%%%%%%%%%%%%%%%%%%%%%%%%%%%%%%%%%%%%%%%
\subsection{\code{PMIx_Compute_distances_nb}}
\declareapi{PMIx_Compute_distances_nb}

%%%%
\summary

Compute distances from specified process location to local devices.

%%%%
\format

\copySignature{PMIx_Compute_distances_nb}{4.0}{
pmix_status_t \\
PMIx_Compute_distances_nb(pmix_topology_t *topo, \\
\hspace*{26\sigspace}pmix_cpuset_t *cpuset, \\
\hspace*{26\sigspace}pmix_info_t info[], size_t ninfo[], \\
\hspace*{26\sigspace}pmix_device_dist_cbfunc_t cbfunc, \\
\hspace*{26\sigspace}void *cbdata);
}

\begin{arglist}
\argin{topo}{Pointer to the topology description of the node where the process is located (\code{NULL} indicates the local node) (\refstruct{pmix_topology_t})}
\argin{cpuset}{Pointer to the location of the process (\refstruct{pmix_cpuset_t})}
\argin{info}{Array of \refstruct{pmix_info_t} describing the devices whose distance is to be computed (handle)}
\argin{ninfo}{Number of elements in \refarg{info} (integer)}
\argin{cbfunc}{Callback function \refapi{pmix_info_cbfunc_t} (function reference)}
\argin{cbdata}{Data to be passed to the callback function (memory reference)}
\end{arglist}

\returnsimplenb

\optattrstart
The following attributes are optional for \ac{PMIx} implementations:

\pasteAttributeItem{PMIX_DEVICE_DISTANCES}
\pasteAttributeItem{PMIX_DEVICE_ID}
\pasteAttributeItem{PMIX_DEVICE_TYPE}

\optattrend

%%%%
\descr

Non-blocking form of the \refapi{PMIx_Compute_distances} \ac{API}.

%%%%%%%%%%%%%%%%%%%%%%%%%%%%%%%%%%%%%%%%%%%%%%%%%
\subsubsection{Device Distance Callback Function}
\declareapi{pmix_device_dist_cbfunc_t}

%%%%
\summary

The \refapi{pmix_device_dist_cbfunc_t} is used to return an array of device distances.

\copySignature{pmix_device_dist_cbfunc_t}{4.0}{
typedef void (*pmix_device_dist_cbfunc_t) \\
\hspace*{4\sigspace}(pmix_status_t status, \\
\hspace*{5\sigspace}pmix_device_distance_t *dist, \\
\hspace*{5\sigspace}size_t ndist, \\
\hspace*{5\sigspace}void *cbdata, \\
\hspace*{5\sigspace}pmix_release_cbfunc_t release_fn, \\
\hspace*{5\sigspace}void *release_cbdata);
}

\begin{arglist}
\argin{status}{Status associated with the operation (\refstruct{pmix_status_t})}
\argin{dist}{Array of \refstruct{pmix_device_distance_t} returned by the operation (pointer)}
\argin{ndist}{Number of elements in the \argref{dist} array (\code{size_t})}
\argin{cbdata}{Callback data passed to original \ac{API} call (memory reference)}
\argin{release_fn}{Function to be called when done with the \argref{dist} data (function pointer)}
\argin{release_cbdata}{Callback data to be passed to \argref{release_fn} (memory reference)}
\end{arglist}


%%%%
\descr

The \refarg{status} indicates if requested data was found or not.
The array of \refstruct{pmix_device_distance_t} will contain the distance information.

%%%%%%%%%%%%%%%%%%%%%%%%%%%%%%%%%%%%%%%%%%%%%%%%%
\subsection{Device type}
\declarestruct{pmix_device_type_t}
\label{api:proc:devtype}

The \refstruct{pmix_device_type_t} is a \code{uint64_t} bitmask for identifying the type(s) whose distances are being requested, or the type of a specific device being referenced (e.g., in a \refstruct{pmix_device_distance_t} object).

\copySignature{pmix_device_type_t}{1.0}{
typedef uint64_t pmix_device_type_t;
}

The following constants can be used to set a variable of the type \refstruct{pmix_device_type_t}.

\begin{constantdesc}
%
\declareconstitemvalue{PMIX_DEVTYPE_UNKNOWN}{0x00}
The device is of an unknown type - will not be included in returned device distances.
%
\declareconstitemvalue{PMIX_DEVTYPE_BLOCK}{0x01}
Operating system block device, or non-volatile memory device (e.g., "sda" or "dax2.0" on Linux).
%
\declareconstitemvalue{PMIX_DEVTYPE_GPU}{0x02}
Operating system \ac{GPU} device (e.g., "card0" for a Linux \ac{DRM} device).
%
\declareconstitemvalue{PMIX_DEVTYPE_NETWORK}{0x04}
Operating system network device (e.g., the "eth0" interface on Linux).
%
\declareconstitemvalue{PMIX_DEVTYPE_OPENFABRICS}{0x08}
Operating system OpenFabrics device (e.g., an "mlx4_0" InfiniBand \ac{HCA}, or "hfi1_0" Omni-Path interface on Linux).
%
\declareconstitemvalue{PMIX_DEVTYPE_DMA}{0x10}
Operating system \ac{DMA} engine device (e.g., the "dma0chan0" \ac{DMA} channel on Linux).
%
\declareconstitemvalue{PMIX_DEVTYPE_COPROC}{0x20}
Operating system co-processor device (e.g., "mic0" for a Xeon Phi on Linux, "opencl0d0" for a OpenCL device, or "cuda0" for a \ac{CUDA} device).
%
\end{constantdesc}

%%%%%%%%%%%%%%%%%%%%%%%%%%%%%%%%%%%%%%%%%%%%%%%%%
\subsection{Device Distance Structure}
\declarestruct{pmix_device_distance_t}

The \refstruct{pmix_device_distance_t} structure contains the minimum and maximum relative distance from the caller to a given device.

\copySignature{pmix_device_distance_t}{4.0}{
typedef struct pmix_device_distance \{ \\
\hspace*{4\sigspace}char *uuid; \\
\hspace*{4\sigspace}char *osname; \\
\hspace*{4\sigspace}pmix_device_type_t type; \\
\hspace*{4\sigspace}uint16_t mindist; \\
\hspace*{4\sigspace}uint16_t maxdist; \\
\} pmix_device_distance_t;
}

The \refarg{uuid} is a string identifier guaranteed to be unique within the cluster and is typically assembled from discovered device attributes (e.g., the \ac{IP} address of the device). The \refarg{osname} is the local operating system name of the device and is only unique to that node.

The two distance fields provide the minimum and maximum relative distance to the device from the specified location of the process, expressed as a 16-bit integer value where a smaller number indicates that this device is closer to the process than a device with a larger distance value. Note that relative distance values are not necessarily correlated to a physical property - e.g., a device at twice the distance from another device does not necessarily have twice the latency for communication with it.

Relative distances only apply to similar devices and cannot be used to compare devices of different types. Both minimum and maximum distances are provided to support cases where the process may be bound to more than one location, and the locations are at different distances from the device.

A relative distance value of \code{UINT16_MAX} indicates that the distance from the process to the device could not be provided. This may be due to lack of available information (e.g., the \ac{PMIx} library not having access to device locations) or other factors.


%%%%%%%%%%%%%%%%%%%%%%%%%%%%%%%%%%%%%%%%%%%%%%%%%
\subsection{Device distance support macros}
\label{api:netenddist:macros}

The following macros are provided to support the \refstruct{pmix_device_distance_t} structure.

%%%%
\littleheader{Static initializer for the device distance structure}
\declaremacroProvisional{PMIX_DEVICE_DIST_STATIC_INIT}

Provide a static initializer for the \refstruct{pmix_device_distance_t} fields.

\versionMarker{5.0}
\cspecificstart
\begin{codepar}
PMIX_DEVICE_DIST_STATIC_INIT
\end{codepar}
\cspecificend


%%%%
\littleheader{Initialize the device distance structure}
\declaremacro{PMIX_DEVICE_DIST_CONSTRUCT}

Initialize the \refstruct{pmix_device_distance_t} fields.

\copySignature{PMIX_DEVICE_DIST_CONSTRUCT}{4.0}{
PMIX_DEVICE_DIST_CONSTRUCT(m)
}

\begin{arglist}
\argin{m}{Pointer to the structure to be initialized (pointer to \refstruct{pmix_device_distance_t})}
\end{arglist}

%%%%
\littleheader{Destruct the device distance structure}
\declaremacro{PMIX_DEVICE_DIST_DESTRUCT}

Destruct the \refstruct{pmix_device_distance_t} fields.

\copySignature{PMIX_DEVICE_DIST_DESTRUCT}{4.0}{
PMIX_DEVICE_DIST_DESTRUCT(m)
}

\begin{arglist}
\argin{m}{Pointer to the structure to be destructed (pointer to \refstruct{pmix_device_distance_t})}
\end{arglist}

%%%%
\littleheader{Create an device distance array}
\declaremacro{PMIX_DEVICE_DIST_CREATE}

Allocate and initialize a \refstruct{pmix_device_distance_t} array.

\copySignature{PMIX_DEVICE_DIST_CREATE}{4.0}{
PMIX_DEVICE_DIST_CREATE(m, n)
}

\begin{arglist}
\arginout{m}{Address where the pointer to the array of \refstruct{pmix_device_distance_t} structures shall be stored (handle)}
\argin{n}{Number of structures to be allocated (\code{size_t})}
\end{arglist}

%%%%
\littleheader{Release an device distance array}
\declaremacro{PMIX_DEVICE_DIST_FREE}

Release an array of \refstruct{pmix_device_distance_t} structures.

\copySignature{PMIX_DEVICE_DIST_FREE}{4.0}{
PMIX_DEVICE_DIST_FREE(m, n)
}

\begin{arglist}
\argin{m}{Pointer to the array of \refstruct{pmix_device_distance_t} structures (handle)}
\argin{n}{Number of structures in the array (\code{size_t})}
\end{arglist}

%%%%%%%%%%%%%%%%%%%%%%%%%%%%%%%%%%%%%%%%%%%%%%%%%
\subsection{Device distance attributes}
\label{api:netenddist:attrs}

The following attributes can be used to retrieve device distances from the \ac{PMIx} data store. Note that distances stored by the host environment are based on the process location at the time of start of execution and may not reflect changes to location imposed by the process itself.

%
\declareAttribute{PMIX_DEVICE_DISTANCES}{"pmix.dev.dist"}{pmix_data_array_t}{
Return an array of \refstruct{pmix_device_distance_t} containing the minimum and maximum distances of the given process location to all devices of the specified type on the local node.
}
%
\declareAttribute{PMIX_DEVICE_TYPE}{"pmix.dev.type"}{pmix_device_type_t}{
Bitmask specifying the type(s) of device(s) whose information is being requested. Only used as a directive/qualifier.
}
%
\declareAttribute{PMIX_DEVICE_ID}{"pmix.dev.id"}{string}{
System-wide \ac{UUID} or node-local \ac{OS} name of a particular device.
}

%%%%%%%%%%%%%%%%%%%%%%%%%%%%%%%%%%%%%%%%%%%%%%%%%
%%%%%%%%%%%%%%%%%%%%%%%%%%%%%%%%%%%%%%%%%%%%%%%%%
