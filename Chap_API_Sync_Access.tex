%%%%%%%%%%%%%%%%%%%%%%%%%%%%%%%%%%%%%%%%%%%%%%%%%
% Chapter: Synchronization and Data Access Operations
%%%%%%%%%%%%%%%%%%%%%%%%%%%%%%%%%%%%%%%%%%%%%%%%%
\chapter{Synchronization and Data Access Operations}
\label{chap:api_sync_acc}

Applications may need to synchronize their operations at various points in
their execution. Depending on a variety of factors (e.g., the programming
model and where the synchronization point lies), the application may choose to
execute the operation using \ac{PMIx}. This is particularly useful in
situations where communication by other means is not yet available since
\ac{PMIx} relies on the host environment's infrastructure for such operations.

Synchronization operations also offer an opportunity for processes to exchange
data at a known point in their execution. Where required, this can include
information on communication endpoints for subsequent wireup of various
messaging protocols.

This chapter covers both the synchronization and data retrieval functions
provided under the \ac{PMIx} Standard.

%%%%%%%%%%%%%%%%%%%%%%%%%%%%%%%%%%%%%%%%%%%%%%%%%
%%%%%%%%%%%%%%%%%%%%%%%%%%%%%%%%%%%%%%%%%%%%%%%%%
\section{\code{PMIx_Fence}}
\declareapi{PMIx_Fence}

%%%%
\summary

Execute a blocking barrier across the processes identified in the specified array, collecting information posted via \refapi{PMIx_Put} as directed.

%%%%
\format

\copySignature{PMIx_Fence}{1.0}{
pmix_status_t \\
PMIx_Fence(const pmix_proc_t procs[], size_t nprocs, \\
\hspace*{11\sigspace}const pmix_info_t info[], size_t ninfo);
}

\begin{arglist}
\argin{procs}{Array of \refstruct{pmix_proc_t} structures (array of handles)}
\argin{nprocs}{Number of elements in the \refarg{procs} array (integer)}
\argin{info}{Array of info structures (array of handles)}
\argin{ninfo}{Number of elements in the \refarg{info} array (integer)}
\end{arglist}

\returnsimple

\reqattrstart
The following attributes are required to be supported by all \ac{PMIx} libraries:

\pasteAttributeItem{PMIX_COLLECT_DATA}
\pasteAttributeItem{PMIX_COLLECT_GENERATED_JOB_INFO}

\reqattrend

\optattrstart
The following attributes are optional for \ac{PMIx} implementations:

\pasteAttributeItem{PMIX_ALL_CLONES_PARTICIPATE}


The following attributes are optional for host environments:

\pasteAttributeItem{PMIX_TIMEOUT}

\optattrend

%%%%
\descr

Passing a \code{NULL} pointer as the \refarg{procs} parameter indicates that the fence is to span all processes in the client's namespace.
Each provided \refstruct{pmix_proc_t} struct can pass \refconst{PMIX_RANK_WILDCARD} to indicate that all processes in the given namespace are participating.

The \refarg{info} array is used to pass user directives regarding the behavior of the fence operation. Note that for scalability reasons, the default behavior for \refapi{PMIx_Fence} is to not collect data posted by the operation's participants.

\adviceimplstart
\refapi{PMIx_Fence} and its non-blocking form are both \emph{collective} operations. Accordingly, the \ac{PMIx} server library is required to aggregate participation by local clients, passing the request to the host environment once all local participants have executed the \ac{API}.
\adviceimplend

\advicermstart
The host will receive a single call for each collective operation. It is the responsibility of the host to identify the nodes containing participating processes, execute the collective across all participating nodes, and notify the local \ac{PMIx} server library upon completion of the global collective.
\advicermend


%%%%%%%%%%%%%%%%%%%%%%%%%%%%%%%%%%%%%%%%%%%%%%%%%
%%%%%%%%%%%%%%%%%%%%%%%%%%%%%%%%%%%%%%%%%%%%%%%%%
\section{\code{PMIx_Fence_nb}}
\declareapi{PMIx_Fence_nb}

%%%%
\summary

Execute a nonblocking \refapi{PMIx_Fence} across the processes identified in the specified array of processes, collecting information posted via \refapi{PMIx_Put} as directed.

%%%%
\format

\copySignature{PMIx_Fence_nb}{1.0}{
pmix_status_t \\
PMIx_Fence_nb(const pmix_proc_t procs[], size_t nprocs, \\
\hspace*{14\sigspace}const pmix_info_t info[], size_t ninfo, \\
\hspace*{14\sigspace}pmix_op_cbfunc_t cbfunc, void *cbdata);
}

\begin{arglist}
\argin{procs}{Array of \refstruct{pmix_proc_t} structures (array of handles)}
\argin{nprocs}{Number of elements in the \refarg{procs} array (integer)}
\argin{info}{Array of info structures (array of handles)}
\argin{ninfo}{Number of elements in the \refarg{info} array (integer)}
\argin{cbfunc}{Callback function (function reference)}
\argin{cbdata}{Data to be passed to the callback function (memory reference)}
\end{arglist}

\returnsimplenb

\returnstart
\begin{itemize}
    \item \refconst{PMIX_OPERATION_SUCCEEDED}, indicating that the request was immediately processed and returned \textit{success} - the \refarg{cbfunc} will \textit{not} be called. This can occur if the collective involved only processes on the local node.
\end{itemize}
\returnend

\reqattrstart
The following attributes are required to be supported by all \ac{PMIx} libraries:

\pasteAttributeItem{PMIX_COLLECT_DATA}
\pasteAttributeItem{PMIX_COLLECT_GENERATED_JOB_INFO}

\reqattrend

\optattrstart
The following attributes are optional for \ac{PMIx} implementations:

\pasteAttributeItem{PMIX_ALL_CLONES_PARTICIPATE}


The following attributes are optional for host environments that support this operation:

\pasteAttributeItem{PMIX_TIMEOUT}

\optattrend

%%%%
\descr

Nonblocking version of the \refapi{PMIx_Fence} routine. See the \refapi{PMIx_Fence} description for further details.

%%%%%%%%%%%%%%%%%%%%%%%%%%%%%%%%%%%%%%%%%%%%%%%%%
\subsection{Fence-related attributes}

The following attributes are defined specifically to support the fence operation:

%
\declareAttribute{PMIX_COLLECT_DATA}{"pmix.collect"}{bool}{
Collect all data posted by the participants using \refapi{PMIx_Put} that
has been committed via \refapi{PMIx_Commit}, making the collection locally
available to each participant at the end of the operation. By default, this will include all job-level information that was locally generated by \ac{PMIx} servers unless excluded using the \refattr{PMIX_COLLECT_GENERATED_JOB_INFO} attribute.
}
%
\declareAttributeNEW{PMIX_COLLECT_GENERATED_JOB_INFO}{"pmix.collect.gen"}{bool}{
Collect all job-level information (i.e., reserved keys) that was locally generated by \ac{PMIx} servers. Some job-level information (e.g., distance between processes and fabric devices) is best determined on a distributed basis as it primarily pertains to local processes. Should remote processes need to access the information, it can either be obtained collectively using the \refapi{PMIx_Fence} operation with this directive, or can be retrieved one peer at a time using \refapi{PMIx_Get} without first having performed the job-wide collection.
}
%
\declareAttributeNEW{PMIX_ALL_CLONES_PARTICIPATE}{"pmix.clone.part"}{bool}{
All \refterm{clones} of the calling process must participate in the collective operation.
}


%%%%%%%%%%%%%%%%%%%%%%%%%%%%%%%%%%%%%%%%%%%%%%%%%
%%%%%%%%%%%%%%%%%%%%%%%%%%%%%%%%%%%%%%%%%%%%%%%%%
\section{\code{PMIx_Get}}
\declareapi{PMIx_Get}

%%%%
\summary

Retrieve a key/value pair from the client's namespace.

%%%%
\format

\copySignature{PMIx_Get}{1.0}{
pmix_status_t \\
PMIx_Get(const pmix_proc_t *proc, const pmix_key_t key, \\
\hspace*{9\sigspace}const pmix_info_t info[], size_t ninfo, \\
\hspace*{9\sigspace}pmix_value_t **val);
}

\begin{arglist}
\argin{proc}{Process identifier - a \code{NULL} value may be used in place of the caller's ID (handle)}
\argin{key}{Key to retrieve (\refstruct{pmix_key_t})}
\argin{info}{Array of info structures (array of handles)}
\argin{ninfo}{Number of elements in the \refarg{info} array (integer)}
\argout{val}{value (handle)}
\end{arglist}

A successful return indicates that the requested data has been returned in the manner requested (.e.g., in a provided static memory location ).

\returnstart
\begin{itemize}
\item \refconst{PMIX_ERR_BAD_PARAM} A bad parameter was passed to the function call - e.g., the request included the \refattr{PMIX_GET_STATIC_VALUES} directive, but the provided storage location was \code{NULL}
\item \refconst{PMIX_ERR_EXISTS_OUTSIDE_SCOPE} The requested key exists, but was posted in a \emph{scope} (see Section \ref{api:nres:scope}) that does not include the requester.
\item \refconst{PMIX_ERR_NOT_FOUND} The requested data was not available.
\end{itemize}
\returnend

\reqattrstart
The following attributes are required to be supported by all \ac{PMIx} libraries:

\pasteAttributeItem{PMIX_OPTIONAL}
\pasteAttributeItem{PMIX_IMMEDIATE}
\pasteAttributeItem{PMIX_DATA_SCOPE}
\pasteAttributeItem{PMIX_SESSION_INFO}
\pasteAttributeItem{PMIX_JOB_INFO}
\pasteAttributeItem{PMIX_APP_INFO}
\pasteAttributeItem{PMIX_NODE_INFO}
\pasteAttributeItem{PMIX_GET_STATIC_VALUES}
\pasteAttributeItem{PMIX_GET_POINTER_VALUES}
\pasteAttributeItem{PMIX_GET_REFRESH_CACHE}

\reqattrend

\optattrstart
The following attributes are optional for host environments:

\pasteAttributeItem{PMIX_TIMEOUT}

\optattrend

%%%%
\descr

Retrieve information for the specified \refarg{key} associated with the process identified in the given \refstruct{pmix_proc_t}. See Chapters \ref{chap:api_rsvd_keys} and \ref{chap:nrkeys} for details on rules governing retrieval of information. Information will be returned according to provided directives:

\begin{itemize}
    \item In the absence of any directive, the returned \refstruct{pmix_value_t} shall be an allocated memory object. The caller is responsible for releasing the object when done.
    \item If \refattr{PMIX_GET_POINTER_VALUES} is given, then the function shall return a pointer to a \refstruct{pmix_value_t} in the \ac{PMIx} library's memory that contains the requested information.
    \item If \refattr{PMIX_GET_STATIC_VALUES} is given, then the function shall return the information in the provided \refstruct{pmix_value_t} pointer. In this case, the caller must provide storage for the structure and pass the pointer to that storage in the \refarg{val} parameter. If the implementation cannot return a static value, then the call to \refapi{PMIx_Get} must return the \refconst{PMIX_ERR_NOT_SUPPORTED} status.
\end{itemize}

This is a blocking operation - the caller will block until the retrieval rules of Chapters \ref{chap:api_rsvd_keys} or \ref{chap:nrkeys} are met.

The \refarg{info} array is used to pass user directives regarding the get operation.

%%%%%%%%%%%%%%%%%%%%%%%%%%%%%%%%%%%%%%%%%%%%%%%%%
\subsection{\code{PMIx_Get_nb}}
\declareapi{PMIx_Get_nb}

%%%%
\summary

Nonblocking \refapi{PMIx_Get} operation.

%%%%
\format

\copySignature{PMIx_Get_nb}{1.0}{
pmix_status_t \\
PMIx_Get_nb(const pmix_proc_t *proc, const char key[], \\
\hspace*{12\sigspace}const pmix_info_t info[], size_t ninfo, \\
\hspace*{12\sigspace}pmix_value_cbfunc_t cbfunc, void *cbdata);
}

\begin{arglist}
\argin{proc}{Process identifier - a \code{NULL} value may be used in place of the caller's ID (handle)}
\argin{key}{Key to retrieve (string)}
\argin{info}{Array of info structures (array of handles)}
\argin{ninfo}{Number of elements in the \refarg{info} array (integer)}
\argin{cbfunc}{Callback function (function reference)}
\argin{cbdata}{Data to be passed to the callback function (memory reference)}
\end{arglist}

\returnsimplenb

If executed, the status returned in the provided callback function will be one of the following constants:

\begin{itemize}
\item \refconst{PMIX_SUCCESS} The requested data has been returned.
\item \refconst{PMIX_ERR_EXISTS_OUTSIDE_SCOPE} The requested key exists, but was posted in a \emph{scope} (see Section \ref{api:nres:scope}) that does not include the requester.
\item \refconst{PMIX_ERR_NOT_FOUND} The requested data was not available.
\item a non-zero \ac{PMIx} error constant indicating a reason for the request's failure.
\end{itemize}

\reqattrstart
The following attributes are required to be supported by all \ac{PMIx} libraries:

\pasteAttributeItem{PMIX_OPTIONAL}
\pasteAttributeItem{PMIX_IMMEDIATE}
\pasteAttributeItem{PMIX_DATA_SCOPE}
\pasteAttributeItem{PMIX_SESSION_INFO}
\pasteAttributeItem{PMIX_JOB_INFO}
\pasteAttributeItem{PMIX_APP_INFO}
\pasteAttributeItem{PMIX_NODE_INFO}
\pasteAttributeItem{PMIX_GET_POINTER_VALUES}
\pasteAttributeItem{PMIX_GET_REFRESH_CACHE}

\divider

The following attributes are required for host environments that support this operation:

\pasteAttributeItem{PMIX_WAIT}

\reqattrend

\optattrstart
The following attributes are optional for host environments that support this operation:

\pasteAttributeItem{PMIX_TIMEOUT}

\optattrend

%%%%
\descr

The callback function will be executed once the retrieval rules of Chapters \ref{chap:api_rsvd_keys} or \ref{chap:nrkeys} are met.
See \refapi{PMIx_Get} for a full description. Note that the non-blocking form of this function cannot support the \refattr{PMIX_GET_STATIC_VALUES} attribute as the user cannot pass in the required pointer to storage for the result.


%%%%%%%%%%%%%%%%%%%%%%%%%%%%%%%%%%%%%%%%%%%%%%%%%
\subsection{Retrieval attributes}
\label{chap:api_kg:attr}

The following attributes are defined for use by retrieval \acp{API}:

%
\declareAttribute{PMIX_OPTIONAL}{"pmix.optional"}{bool}{
Look only in the client's local data store for the requested value - do not request data from the \ac{PMIx} server if not found.
}
%
\declareAttribute{PMIX_IMMEDIATE}{"pmix.immediate"}{bool}{
Specified operation should immediately return an error from the \ac{PMIx} server if the requested data cannot be found - do not request it from the host \ac{RM}.
}
%
\declareAttributeNEW{PMIX_GET_POINTER_VALUES}{"pmix.get.pntrs"}{bool}{
Request that any pointers in the returned value point directly to values in the key-value store. The user \emph{must not} release any returned data pointers.
}
%
\declareAttributeNEW{PMIX_GET_STATIC_VALUES}{"pmix.get.static"}{bool}{
Request that the data be returned in the provided storage location. The caller is responsible for destructing the \refstruct{pmix_value_t} using the \refmacro{PMIX_VALUE_DESTRUCT} macro when done.
}
%
\declareAttributeNEW{PMIX_GET_REFRESH_CACHE}{"pmix.get.refresh"}{bool}{
When retrieving data for a remote process, refresh the existing local data cache for the process in case new values have been put and committed by the process since the last refresh. Local process information is assumed to be automatically updated upon posting by the process. A \code{NULL} key will cause all values associated with the process to be refreshed - otherwise, only the indicated key will be updated. A process rank of \refconst{PMIX_RANK_WILDCARD} can be used to update job-related information in dynamic environments. The user is responsible for subsequently updating refreshed values they may have cached in their own local memory.
}
%
\declareAttribute{PMIX_DATA_SCOPE}{"pmix.scope"}{pmix_scope_t}{
Scope of the data to be searched in a \refapi{PMIx_Get} call.
}
%
\declareAttribute{PMIX_TIMEOUT}{"pmix.timeout"}{int}{
Time in seconds before the specified operation should time out (zero indicating infinite) and return the \refconst{PMIX_ERR_TIMEOUT} error.
Care should be taken to avoid race conditions caused by multiple layers (client, server, and host) simultaneously timing the operation.
}
%
\declareAttribute{PMIX_WAIT}{"pmix.wait"}{int}{
Caller requests that the \ac{PMIx} server wait until at least the specified number of values are found (a value of zero indicates \emph{all} and is the default).
}


%%%%%%%%%%%%%%%%%%%%%%%%%%%%%%%%%%%%%%%%%%%%%%%%%
