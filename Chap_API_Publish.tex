%%%%%%%%%%%%%%%%%%%%%%%%%%%%%%%%%%%%%%%%%%%%%%%%%
% Chapter: Publish/Lookup Operations
%%%%%%%%%%%%%%%%%%%%%%%%%%%%%%%%%%%%%%%%%%%%%%%%%
\chapter{Publish/Lookup Operations}
\label{chap:pub}

Chapter~\ref{chap:api_rsvd_keys}
and
Section ~\ref{chap:data_sharing:non_rsvd_keys}
present how reserved and non-reserved keys deal with
information that either is associated with a specific process (i.e., the
retrieving process knows the identifier of the process that posted it) or
requires a synchronization operation prior to retrieval (e.g., the case of
globally unique non-reserved keys). However, another requirement exists for an
asynchronous exchange of data where neither the posting nor the retrieving
process is known in advance (e.g. two namespaces that do not share a child-parent relationship).
The \acp{API} defined in this section focus on resolving that specific
situation by allowing processes to publish data that can subsequently be
retrieved solely by referral to its key. Mechanisms for constraining
the scope of availability of the information are also provided as a means for better
targeting of the eventual recipient(s).

Note that no presumption is made regarding how the published information is to be stored, nor as to the entity (host environment or \ac{PMIx} implementation) that shall act as the datastore. The descriptions in the remainder of this chapter shall simply refer to that entity as the \emph{datastore}.

%%%%%%%%%%%%%%%%%%%%%%%%%%%%%%%%%%%%%%%%%%%%%%%%%
%%%%%%%%%%%%%%%%%%%%%%%%%%%%%%%%%%%%%%%%%%%%%%%%%
\section{\code{PMIx_Publish_datastore}}

\declareapiProvisional{PMIx_Publish_datastore}

%%%%
\summary

Publish data for later access via \refapi{PMIx_Lookup_datastore}.

%%%%
\format


\cspecificstart
\begin{codepar}
pmix_status_t \\
PMIx_Publish_datastore(const pmix_info_t pinfo[], size_t npinfo, \\
\hspace*{12\sigspace}pmix_publish_id_t *publish_id, \\
\hspace*{12\sigspace}const pmix_info_t info[], size_t ninfo);
\end{codepar}
\cspecificend

\begin{arglist}
\argin{pinfo}{Array of key value pairs to publish (array of \refstruct{pmix_info_t})}
\argin{npinfo}{Number of elements in the \refarg{pinfo} array (integer)}
\argout{publish_id}{The epoch number associated with this call}
\argin{info}{Array of info structures (array of \refstruct{pmix_info_t})}
\argin{ninfo}{Number of elements in the \refarg{info} array (integer)}
\end{arglist}
%
\returnsimple
%
\reqattrstart
There are no required attributes for this \ac{API}. \ac{PMIx} implementations that do not directly support the operation but are hosted by environments that do support it must pass any attributes that are provided by the client to the host environment for processing. In addition, the \ac{PMIx} library is required to add the \refAttributeItem{PMIX_USERID} and the \refAttributeItem{PMIX_GRPID} attributes of the client process that published the information to the \refarg{info} array passed to the host environment.
\reqattrend

\optattrstart
The following attributes are optional for host environments that support this operation:

\pasteAttributeItem{PMIX_TIMEOUT}
\pasteAttributeItem{PMIX_RANGE}
\pasteAttributeItem{PMIX_PERSISTENCE}
\pasteAttributeItem{PMIX_ACCESS_USERIDS}
\pasteAttributeItem{PMIX_ACCESS_GRPIDS}
\pasteAttributeItem{PMIX_ACCESS_PERMISSIONS}

\optattrend

%%%%
\descr

Publish the data in the \refarg{pinfo} array for subsequent lookup.
By default, data is accessible by all processes in the same session as the publisher.  
The attributes \refconst{PMIX_RANGE}, 
\refconst{PMIX_ACCESS_USERIDS}, \refconst{PMIX_ACCESS_GRPIDS} and \refconst{PMIX_ACCESS_PERMISSIONS} 
can be included in the \refarg{info} array to restrict the processes that can access the published data on systems 
which support these attributes.  
By default, the data will remain published until
all processes in the publishing application terminate.  The attribute \refconst{PMIX_PERSISTENCE} 
can be included in the \refarg{info} array to change if and when data will be automatically 
unpublished for systems which support this attribute (See Section \ref{chap:pub:types:persist})

Each published key is assigned a epoch number when it is published to help identify the relative order
that value assignments are made to a key when a key is assigned multiple values.
All of the keys in the \refarg{pinfo} array will be associated with the same epoch number.  
Within a single process, each call to \refapi{PMIx_Publish_datastore} or \refapi{PMIx_Publish_datastore_nb}
will return a unique value.   
If one call to \refapi{PMIx_Publish_datastore} or \refapi{PMIx_Publish_datastore_nb} returns successfully before 
another call to either \ac{API} is initiated by a process,
the former should assign a smaller epoch number than the 
later call.  In the case where two calls by processes on the same host overlap in time, 
the implementation should make a best effort to determine which call was made first and assign the 
keys of that call a smaller epoch number.  
A key with a greater epoch number be never be visible to other processes before a key with 
a lesser epoch number.

\adviceuserstart
To ensure that one call to \refapi{PMIx_Publish_datastore} call is assigned a smaller value than another 
call across multiple threads or processes it may be necessary to use synchronization primatives. 
\refapi{PMIx_Fence} is a \ac{PMIx} \ac{API} that can be used for synchronizing across multiple processes.
\adviceuserend

\adviceimplstart
The requirement for an implementation to assign an epoch number permits a variety of implementations.  If a centeralized 
datastore is used, the epoch number can be a simple epoch number such as a counter that is incremented on each
\refapi{PMIx_Publish_datastore} and \refapi{PMIx_Publish_datastore_nb} call.  Alternatively, the epoch number can
be a time value if hosts are guaranteed to have clocks with sufficient resolution and synchronization to ensure
that consecutive calls can be distiguished as occuring at different points of time.
\adviceimplend

The returned \refarg{publish_id} is used to associate the published key values with the specific publish
call used to publish them.  A \refstruct{pmix_publish_id_t} can be used to determine the publishing
process (\refstruct{pmix_proc_t}) and the epoch number of a call to publish data.. 
The \refarg{publish_epoch} must be retained for unpublishing and may be transfered to other processes 
either for unpublishing or to uniquely identify a specific value of a key.

The blocking form of this call will block until it has obtained confirmation from the datastore that the data is available for lookup. 
The \refarg{pinfo} array can be released upon return from the blocking function call.

Clients performing a lookup operation with \refapi{PMIx_Lookup_datastore} on
a key will
receive a list of all accessible values published for that key.  The lookup \acp{API} also allows the caller to further
restrict what values are considered accessible, such as restricting which publishing processes to consider.

%%%%%%%%%%%%%%%%%%%%%%%%%%%%%%%%%%%%%%%%%%%%%%%%%
%%%%%%%%%%%%%%%%%%%%%%%%%%%%%%%%%%%%%%%%%%%%%%%%%
\section{\code{PMIx_Publish_datastore_nb}}

\declareapiProvisional{PMIx_Publish_datastore_nb}

%%%%
\summary

Nonblocking \refapi{PMIx_Publish_datastore} routine.

%%%%
\format

\cspecificstart
\begin{codepar}
pmix_status_t \\
PMIx_Publish_datastore_nb(const pmix_info_t pinfo[], size_t npinfo, \\
\hspace*{12\sigspace}const pmix_info_t info[], size_t ninfo, \\
\hspace*{16\sigspace}pmix_publish_datastore_cbfunc_t cbfunc, void *cbdata);
\end{codepar}
\cspecificend

\begin{arglist}
\argin{pinfo}{Array of key value pairs to publish (array of \refstruct{pmix_info_t})}
\argin{npinfo}{Number of elements in the \refarg{pinfo} array (integer)}
\argin{info}{Array of info structures (array of \refstruct{pmix_info_t})}
\argin{ninfo}{Number of elements in the \refarg{info} array (integer)}
\argin{cbfunc}{Callback function \refapi{pmix_publish_datastore_cbfunc_t} (function reference)}
\argin{cbdata}{Data to be passed to the callback function (memory reference)}
\end{arglist}

\returnsimplenb

\reqattrstart
There are no required attributes for this \ac{API}. \ac{PMIx} implementations that do not directly support the operation but are hosted by environments that do support it must pass any attributes that are provided by the client to the host environment for processing. In addition, the \ac{PMIx} library is required to add the \refAttributeItem{PMIX_USERID} and the \refAttributeItem{PMIX_GRPID} attributes of the client process that published the information to the \refarg{info} array passed to the host environment.
\reqattrend

\optattrstart
The following attributes are optional for host environments that support this operation:

\pasteAttributeItem{PMIX_TIMEOUT}
\pasteAttributeItem{PMIX_RANGE}
\pasteAttributeItem{PMIX_PERSISTENCE}
\pasteAttributeItem{PMIX_ACCESS_USERIDS}
\pasteAttributeItem{PMIX_ACCESS_GRPIDS}
\pasteAttributeItem{PMIX_ACCESS_PERMISSIONS}

\optattrend

%%%%
\descr

Nonblocking \refapi{PMIx_Publish_datastore} routine.  The handle to the published data is returned 
as a parameter to the callback function during a successful call.

%%%%%%%%%%%%%%%%%%%%%%%%%%%%%%%%%%%%%%%%%%%%%%%%%
\section{\code{PMIx_Lookup_datastore}}
\declareapiProvisional{PMIx_Lookup_datastore}

%%%%
\summary

Lookup information published by a process or host environment using \refapi{PMIx_Publish_datastore} or \refapi{PMIx_Publish_datastore_nb}.

%%%%
\format

\cspecificstart
\begin{codepar}
pmix_status_t \\
PMIx_Lookup_datastore(pmix_pdsdata_t** data, \\  
\hspace*{12\sigspace}size_t ndata, \\
\hspace*{12\sigspace}const pmix_info_t info[], \\
\hspace*{12\sigspace}size_t ninfo);
\end{codepar}
\cspecificend

\begin{arglist}
\arginout{data}{Array of \refstruct{pmix_pdsdata_t} structures indicating the keys to lookup and providing storage for the results. (array of \refstruct{pmix_pdsdata_t})}
\argin{ndata}{Number of elements in the \refarg{data} array (integer)}
\argin{info}{Array of info structures (array of handles)}
\argin{ninfo}{Number of elements in the \refarg{info} array (integer)}
\end{arglist}

\returnstart
\begin{itemize}
\item \refconst{PMIX_ERR_NOT_FOUND} None of the requested data could be found within the requester's range.  The address pointed to by \refarg{nresults} will be set to 0.

\item \refconst{PMIX_ERR_PARTIAL_SUCCESS} Some of the requested data was found.  
Any key that cannot be found will return with a data type of \refconst{PMIX_UNDEF} in the associated \refarg{value} struct. Note that the specific reason for a particular piece of missing information (e.g., lack of permissions) cannot be communicated back to the requester in this situation.
Only found data will be included in the returned \refarg{data} array. Note that the specific reason for a particular piece of missing information (e.g., lack of permissions or the data has not been published) cannot be communicated back to the requester in this situation.

\end{itemize}
\returnend

\reqattrstart
\ac{PMIx} libraries are not required to directly support any attributes for this function. However, any provided attributes must be passed to the host environment for processing, and the \ac{PMIx} library is required to add the \refAttributeItem{PMIX_USERID} and the \refAttributeItem{PMIX_GRPID} attributes of the client process that is requesting the info.

\reqattrend

\optattrstart
The following attributes are optional for host environments that support this operation:

\pasteAttributeItem{PMIX_TIMEOUT}
\pasteAttributeItem{PMIX_RANGE}
\pasteAttributeItem{PMIX_WAIT}

\optattrend

%%%%
\descr

Lookup information published by a process or host environment using \refapi{PMIx_Publishdatastore_} 
or \refapi{PMIx_Publish_datastore_nb}.
A lookup operation is always performed on a range which can be specified using the directive \refAttributeItem{PMIX_RANGE} or otherwise defaults to \refconst{PMIX_RANGE_SESSION}.

The lookup operation will be constrained to data published to the specified range.
Data is returned per the retrieval rules of Section \ref{chap:pub:retrules}.

The \argref{data} parameter consists of an array of \refstruct{pmix_pdsdata_t} structures with the keys specifying the requested information.
Data will be returned for each \code{key} field in the associated \code{value} and \code{publish_id} fields of this structure.
The length of the arrays \code{publish_id} and \code{value} will be the number of published values matching the requested
key and may be 0.  The \code{value} \refstruct{pmix_data_array_t} will have a \code{type} of \refstruct{pmix_value_t} 
and the \code{publish_id} \refstruct{pmix_data_array_t} will have a \code{type} of \refstruct{pmix_publish_id_t}.

SOLT: TODO: discuss how these are released/freed.

\adviceuserstart
Although this is a blocking function, it will not wait by default for the requested data to be published.
Instead, it will block for the time required by the datastore to lookup its current data and return any found items.
Thus, the caller is responsible for either ensuring that data is published prior to executing a lookup, using \refattr{PMIX_WAIT} to instruct the datastore to wait for the data to be published, or retrying until the requested data is found.
\adviceuserend


%%%%%%%%%%%%%%%%%%%%%%%%%%%%%%%%%%%%%%%%%%%%%%%%%
%%%%%%%%%%%%%%%%%%%%%%%%%%%%%%%%%%%%%%%%%%%%%%%%%
\section{\code{PMIx_Lookup_datastore_nb}}
\declareapi{PMIx_Lookup_datastore_nb}

%%%%
\summary

Nonblocking version of \refapi{PMIx_Lookup_datastore}.

%%%%
\format

\cspecificstart
\begin{codepar}
pmix_status_t \\
PMIx_Lookup_datastore_nb(pmix_key_t keys[], \\
\hspace*{15\sigspace}const pmix_info_t info[], size_t ninfo, \\
\hspace*{15\sigspace}pmix_lookup_datastore_cbfunc_t cbfunc, void *cbdata);
\end{codepar}
\cspecificend

\begin{arglist}
\argin{keys}{\code{NULL}-terminated array of keys (array of strings)}
\argin{info}{Array of info structures (array of handles)}
\argin{ninfo}{Number of elements in the \refarg{info} array (integer)}
\argin{cbfunc}{Callback function (handle)}
\argin{cbdata}{Callback data to be provided to the callback function (pointer)}
\end{arglist}

\returnsimplenb

If executed, the status returned in the provided callback function will be one of the following constants:

\begin{itemize}
\item \refconst{PMIX_SUCCESS} All data was found and has been returned.

\item \refconst{PMIX_ERR_NOT_FOUND} None of the requested data was available within the requester's range. The \refarg{pdata} array in the callback function shall be \code{NULL} and the \refarg{npdata} parameter set to zero.

\item \refconst{PMIX_ERR_PARTIAL_SUCCESS} Some of the requested data was found.
Only found data will be included in the returned \refarg{pdata} array. Note that the specific reason for a particular piece of missing information (e.g., lack of permissions or the data has not been published) cannot be communicated back to the requester in this situation.

\item \refconst{PMIX_ERR_NOT_SUPPORTED} There is no available datastore (either at the host environment or \ac{PMIx} implementation level) on this system that supports this function.

\item \refconst{PMIX_ERR_NO_PERMISSIONS} All of the requested data was found and range restrictions were met for each specified key, but none of the matching data could be returned due to lack of access permissions.

\item a non-zero \ac{PMIx} error constant indicating a reason for the request's failure.
\end{itemize}

\reqattrstart
\ac{PMIx} libraries are not required to directly support any attributes for this function. However, any provided attributes must be passed to the host environment for processing, and the \ac{PMIx} library is required to add the \refAttributeItem{PMIX_USERID} and the \refAttributeItem{PMIX_GRPID} attributes of the client process that is requesting the info.

\reqattrend

\optattrstart
The following attributes are optional for host environments that support this operation:

\pasteAttributeItem{PMIX_TIMEOUT}
\pasteAttributeItem{PMIX_RANGE}
\pasteAttributeItem{PMIX_WAIT}

\optattrend

%%%%
\descr

Non-blocking form of the \refapi{PMIx_Lookup_datastore} function.

%%%%%%%%%%%%%%%%%%%%%%%%%%%%%%%%%%%%%%%%%%%%%%%%%
\section{\code{PMIx_Unpublish_datastore}}
\declareapiProvisional{PMIx_Unpublish_datastore}

%%%%
\summary

Unpublish a list of keys published by the calling process using \refapi{PMIx_Publish_datastore}.

SOLT: TODO: special unpublish all keys with this publish-id

SOLT: TODO: special all keys even with any publish-id I have access to.

SOLT: TODO: publish-id to proc + epoch number + publish_id

SOLT: TODO: declare what an epoch number type is

SOLT: TODO: declare publish_id

SOLT: TODO: macro's for creating/freeing pmix_pdsdata_t's

SOLT: TODO: lookup blocking and non-blocking are so different.  Should we separate keys from return values?


%%%%
\format

\cspecificstart
\begin{codepar}
pmix_status_t \\
PMIx_Unpublish_datastore(const pmix_key_t keys_to_unpublish[], \\
\hspace*{15\sigspace}const pmix_publish_id_t publish_id[], \\
\hspace*{15\sigspace}size_t nkeys, \\
\hspace*{15\sigspace}const pmix_info_t info[], size_t ninfo);
\end{codepar}
\cspecificend

\begin{arglist}
\argin{keys_to_unpublish}{Array of keys to unpublish (array of handles)}
\argin{publish_id}{Array of publish identifiers indicating the publishing call that published the corresponding entry in \refarg{keys_to_unpublish} (array of handles)}
\argin{nkeys}{Number of elements in the \refarg{keys_to_unpublish} and \refarg{publish_id} keys}
\argin{info}{Array of info structures (array of handles)}
\argin{ninfo}{Number of elements in the \refarg{info} array (integer)}
\end{arglist}

\returnsimple

\reqattrstart
\ac{PMIx} libraries are not required to directly support any attributes for this function. However, any provided attributes must be passed to the host environment for processing, and the \ac{PMIx} library is required to add the \refAttributeItem{PMIX_USERID} and the \refAttributeItem{PMIX_GRPID} attributes of the client process that is requesting the operation.

\reqattrend

\optattrstart
The following attributes are optional for host environments that support this operation:

\pasteAttributeItem{PMIX_TIMEOUT}
\pasteAttributeItem{PMIX_RANGE}

\optattrend

%%%%
\descr

Unpublish a list of keys published by the calling process using \refapi{PMIx_Publish_datastore}.
The function will block until the data has been removed by the server (i.e., it keys associated with the handle are no longer visible to to other processes).


%%%%%%%%%%%%%%%%%%%%%%%%%%%%%%%%%%%%%%%%%%%%%%%%%
%%%%%%%%%%%%%%%%%%%%%%%%%%%%%%%%%%%%%%%%%%%%%%%%%
\section{\code{PMIx_Unpublish_datastore_nb}}
\declareapiProvisional{PMIx_Unpublish_datastore_nb}

%%%%
\summary

Nonblocking version of \refapi{PMIx_Unpublish_datastore}.

%%%%
\format

\cspecificstart
\begin{codepar}
pmix_status_t \\
PMIx_Unpublish_datastore_nb(const pmix_key_t keys_to_unpublish[], \\
\hspace*{15\sigspace}const pmix_publish_id_t publish_id[], \\
\hspace*{15\sigspace}size_t nkeys, \\
\hspace*{15\sigspace}const pmix_info_t info[], size_t ninfo, \\
\hspace*{18\sigspace}pmix_op_cbfunc_t cbfunc, void *cbdata);
\end{codepar}
\cspecificend

\begin{arglist}
\argin{keys_to_unpublish}{Array of keys to unpublish (array of handles)}
\argin{publish_id}{Array of publish identifiers indicating the publishing call that published the corresponding entry in \refarg{keys_to_unpublish} (array of handles)}
\argin{nkeys}{Number of elements in the \refarg{keys_to_unpublish} and \refarg{publish_id} keys}
\argin{info}{Array of info structures (array of handles)}
\argin{ninfo}{Number of elements in the \refarg{info} array (integer)}
\argin{cbfunc}{Callback function \refapi{pmix_op_cbfunc_t} (function reference)}
\argin{cbdata}{Data to be passed to the callback function (memory reference)}
\end{arglist}

\returnsimplenb

\returnstart
\begin{itemize}
    \item \refconst{PMIX_OPERATION_SUCCEEDED}, indicating that the request was immediately processed and returned \textit{success} - the \refarg{cbfunc} will \textit{not} be called.
\end{itemize}
\returnend

\reqattrstart
\ac{PMIx} libraries are not required to directly support any attributes for this function. However, any provided attributes must be passed to the host environment for processing, and the \ac{PMIx} library is required to add the \refAttributeItem{PMIX_USERID} and the \refAttributeItem{PMIX_GRPID} attributes of the client process that is requesting the operation.

\reqattrend

\optattrstart
The following attributes are optional for host environments that support this operation:

\pasteAttributeItem{PMIX_TIMEOUT}

\optattrend

%%%%
\descr

Non-blocking form of the \refapi{PMIx_Unpublish} function.
The callback function will be executed once the server confirms removal of the specified data. The \refarg{info} array must be maintained until the callback is provided.


%%%%%%%%%%%%%%%%%%%%%%%%%%%%%%%%%%%%%%%%%%%%%%%%%

%%%%%%%%%%%%%%%%%%%%%%%%%%%%%%%%%%%%%%%%%%%%%%%%%
%%%%%%%%%%%%%%%%%%%%%%%%%%%%%%%%%%%%%%%%%%%%%%%%%
\section{\code{PMIx_Publish}}
\declareapi{PMIx_Publish}

%%%%
\summary

Publish data for later access via \refapi{PMIx_Lookup}.

%%%%
\format

\copySignature{PMIx_Publish}{1.0}{
pmix_status_t \\
PMIx_Publish(const pmix_info_t info[], size_t ninfo);
}

\begin{arglist}
\argin{info}{Array of info structures containing both data to be published and directives (array of handles)}
%% (array of handles is used everywhere, but it is not really an array of handles, so I'm not sure why)
\argin{ninfo}{Number of elements in the \refarg{info} array (integer)}
\end{arglist}

\returnsimple

\reqattrstart
There are no required attributes for this \ac{API}. \ac{PMIx} implementations that do not directly support the operation but are hosted by environments that do support it must pass any attributes that are provided by the client to the host environment for processing. In addition, the \ac{PMIx} library is required to add the \refAttributeItem{PMIX_USERID} and the \refAttributeItem{PMIX_GRPID} attributes of the client process that published the information to the \refarg{info} array passed to the host environment.

\reqattrend

\optattrstart
The following attributes are optional for host environments that support this operation:

\pasteAttributeItem{PMIX_TIMEOUT}
\pasteAttributeItem{PMIX_RANGE}
\pasteAttributeItem{PMIX_PERSISTENCE}
\pasteAttributeItem{PMIX_ACCESS_USERIDS}
\pasteAttributeItem{PMIX_ACCESS_GRPIDS}
\pasteAttributeItem{PMIX_ACCESS_PERMISSIONS}

\optattrend

%%%%
\descr

Publish the data in the \refarg{info} array for subsequent lookup.
By default, the data will be published into the \refconst{PMIX_RANGE_SESSION} range and with \refconst{PMIX_PERSIST_APP} persistence.
Changes to those values, and any additional directives, can be included in the \refstruct{pmix_info_t} array. Attempts to access the data by processes outside of the provided data range shall be rejected. The \refattr{PMIX_PERSISTENCE} attribute instructs the datastore holding the published information as to how long that information is to be retained.

The blocking form of this call will block until it has obtained confirmation from the datastore that the data is available for lookup. The \refarg{info} array can be released upon return from the blocking function call.

Publishing duplicate keys is permitted provided they are published to different
ranges. Custom ranges are considered different if they have different members.
Duplicate keys being published on the same data range shall return the
\refconst{PMIX_ERR_DUPLICATE_KEY} error.

%In some cases, implementations may be incapable of distinguishing which
%info keys in the \refarg{info} array are for publishing and which info keys are
%directives.  To make it clear, it is recommended that the keys to be published
%are designated by passing them as a \refstruct{pmix_data_array_t} using the
%\refattr{PMIX_DATA_TO_PUBLISH} directive.
%If the \refarg{info} array contains a \refattr{PMIX_DATA_TO_PUBLISH} info,
%all other elements of the info array will be treated as directives.
%If the info array does not include a \refattr{PMIX_DATA_TO_PUBLISH} info,
%the implementation should
%distinguish between info array elements that specify keys and directives as follows:
%All standardized directives to the publish call,
%including optional attributes the implementation does not support,
%should be treated as
%directives.  Non-supported directives
%may be ignored as outlined in Section \ref{intro:portability:attributes},
%but should not be treated as data to
%publish.  The implementation may treat any custom (non-standardized) directives it
%supports as directives.  All other \refarg{info} array elements
%should be assumed to be data to be published.
%Since additional directives may be added to the standard and implementations may add support for additional custom directives, the use of \refattr{PMIX_DATA_TO_PUBLISH} is the only reliable way to ensure that
%future implementations will not mis-classify elements of an \refarg{info} array.

%%%%%%%%%%%%%%%%%%%%%%%%%%%%%%%%%%%%%%%%%%%%%%%%%
%%%%%%%%%%%%%%%%%%%%%%%%%%%%%%%%%%%%%%%%%%%%%%%%%
\section{\code{PMIx_Publish_nb}}

\declareapi{PMIx_Publish_nb}

%%%%
\summary

Nonblocking \refapi{PMIx_Publish} routine.

%%%%
\format

\copySignature{PMIx_Publish_nb}{1.0}{
pmix_status_t \\
PMIx_Publish_nb(const pmix_info_t info[], size_t ninfo, \\
\hspace*{16\sigspace}pmix_op_cbfunc_t cbfunc, void *cbdata);
}

\begin{arglist}
\argin{info}{Array of info structures containing both data to be published and directives (array of handles)}
\argin{ninfo}{Number of elements in the \refarg{info} array (integer)}
\argin{cbfunc}{Callback function \refapi{pmix_op_cbfunc_t} (function reference)}
\argin{cbdata}{Data to be passed to the callback function (memory reference)}
\end{arglist}

\returnsimplenb

\returnstart
\begin{itemize}
    \item \refconst{PMIX_OPERATION_SUCCEEDED}, indicating that the request was immediately processed and returned \textit{success} - the \refarg{cbfunc} will \textit{not} be called.
\end{itemize}
\returnend

\reqattrstart
There are no required attributes for this \ac{API}. \ac{PMIx} implementations that do not directly support the operation but are hosted by environments that do support it must pass any attributes that are provided by the client to the host environment for processing. In addition, the \ac{PMIx} library is required to add the \refAttributeItem{PMIX_USERID} and the \refAttributeItem{PMIX_GRPID} attributes of the client process that published the information to the \refarg{info} array passed to the host environment.
\reqattrend

\optattrstart
The following attributes are optional for host environments that support this operation:

\pasteAttributeItem{PMIX_TIMEOUT}
\pasteAttributeItem{PMIX_RANGE}
\pasteAttributeItem{PMIX_PERSISTENCE}
\pasteAttributeItem{PMIX_ACCESS_USERIDS}
\pasteAttributeItem{PMIX_ACCESS_GRPIDS}
\pasteAttributeItem{PMIX_ACCESS_PERMISSIONS}

\optattrend

%%%%
\descr

Nonblocking \refapi{PMIx_Publish} routine.


%%%%%%%%%%%%%%%%%%%%%%%%%%%%%%%%%%%%%%%%%%%%%%%%%
%%%%%%%%%%%%%%%%%%%%%%%%%%%%%%%%%%%%%%%%%%%%%%%%%
\section{Publish-specific constants}

The following constants are defined for use with the \refapi{PMIx_Publish} \acp{API}:

\begin{constantdesc}
%
\declareconstitemvalueNEW{PMIX_ERR_DUPLICATE_KEY}{-53}
The provided key has already been published on the same data range.
%
\end{constantdesc}


%%%%%%%%%%%%%%%%%%%%%%%%%%%%%%%%%%%%%%%%%%%%%%%%%
%%%%%%%%%%%%%%%%%%%%%%%%%%%%%%%%%%%%%%%%%%%%%%%%%
\section{Publish-specific attributes}

The following attributes are defined for use with the \refapi{PMIx_Publish} \acp{API}:

%
\declareAttribute{PMIX_RANGE}{"pmix.range"}{pmix_data_range_t}{
Define constraints on the processes that can access published data or generated events or define constraints on the provider of data when looking up published data.
}
%
\declareAttribute{PMIX_PERSISTENCE}{"pmix.persist"}{pmix_persistence_t}{
Declare how long the datastore shall retain the provided data. The datastore is to delete the data upon reaching the persistence criterion.
}
%
\declareAttributeNEW{PMIX_ACCESS_PERMISSIONS}{"pmix.aperms"}{pmix_data_array_t}{
Define access permissions for the published data. The value shall contain an array of \refstruct{pmix_info_t} structs containing the specified permissions.
}
%
\declareAttributeNEW{PMIX_ACCESS_USERIDS}{"pmix.auids"}{pmix_data_array_t}{
Array of effective \acp{UID} that are allowed to access the published data.
}
%
\declareAttributeNEW{PMIX_ACCESS_GRPIDS}{"pmix.agids"}{pmix_data_array_t}{
Array of effective \acp{GID} that are allowed to access the published data.
}
%

%%%%%%%%%%%%%%%%%%%%%%%%%%%%%%%%%%%%%%%%%%%%%%%%%
%%%%%%%%%%%%%%%%%%%%%%%%%%%%%%%%%%%%%%%%%%%%%%%%%
\section{Publish-Lookup Datatypes}

The following data types are defined for use with the \refapi{PMIx_Publish} \acp{API}.

%%%%%%%%%%%%%%%%%%%%%%%%%%%%%%%%%%%%%%%%%%%%%%%%%
\subsection{Range of Published Data}
\declarestruct{pmix_data_range_t}

\versionMarker{1.0}
The \refstruct{pmix_data_range_t} structure is a \code{uint8_t} type that defines a range for data \textit{published} via the \refapi{PMIx_Publish} \ac{API} and events generated via the \refapi{PMIx_Notify_event}.
The following constants can be used to set a variable of the type \refstruct{pmix_data_range_t}.

\begin{constantdesc}
%
\declareconstitemvalue{PMIX_RANGE_UNDEF}{0}
Undefined range.
%
\declareconstitemvalue{PMIX_RANGE_RM}{1}
Data is intended for the host environment, or lookup is restricted to data published by the host environment.
%
\declareconstitemvalue{PMIX_RANGE_LOCAL}{2}
Published data and generated events are restricted to processes on the same node as the publisher or event creator.  Lookup of data is restricted to data published by processes on the same node as the requester.
%
\declareconstitemvalue{PMIX_RANGE_NAMESPACE}{3}
Published data and generated events are restricted to processes in the same namespace as the publisher or event creator.
Lookup of data is restricted to data published by procesess in the same namespace as the requester.
%
\declareconstitemvalue{PMIX_RANGE_SESSION}{4}
Published data and generated events are restricted to processes in the same session as the publisher or event creator.
Lookup of data is restricted to data published by procesess in the same session as the requester.
%
\declareconstitemvalue{PMIX_RANGE_GLOBAL}{5}
Published data and generated events are available to all processes within the domain of the host environment.
Lookup of data is unrestricted and open to data published by any processes within the domain of the host enivornment as the requester.  This range differs from \refconst{PMIX_RANGE_RM} only on systems which have mechanisms to share events and
publish/lookup data across multiple instances of a host environment.
%
\declareconstitemvalue{PMIX_RANGE_PROC_LOCAL}{7}
Published data and generated events are available only to calling process.
Lookup of data is restricted to data published by the calling process.
%
\declareconstitemvalue{PMIX_RANGE_CUSTOM}{6}
Published data and generated events are restricted to processes
described in the \refstruct{pmix_info_t} associated with this call.
Lookup of data is restricted to data published by the processes described in
in the \refstruct{pmix_info_t}.
%
\declareconstitemvalue{PMIX_RANGE_INVALID}{UINT8_MAX}
Invalid value - typically used to indicate that a range has not yet been set.
%
\end{constantdesc}


%%%%%%%%%%%%%%%%%%%%%%%%%%%%%%%%%%%%%%%%%%%%%%%%%
\subsection{Data Persistence Structure}
\label{chap:pub:types:persist}
\declarestruct{pmix_persistence_t}

\versionMarker{1.0}
The \refstruct{pmix_persistence_t} structure is a \code{uint8_t} type that defines the policy for data published by clients via the \refapi{PMIx_Publish} \ac{API}.
The following constants can be used to set a variable of the type \refstruct{pmix_persistence_t}.

\begin{constantdesc}
%
\declareconstitemvalue{PMIX_PERSIST_INDEF}{0}
Retain data until unpublished.
%
\declareconstitemvalue{PMIX_PERSIST_FIRST_READ}{1}
Retain data until the first access, then the data is deleted.
%
\declareconstitemvalue{PMIX_PERSIST_PROC}{2}
Retain data until the publishing process terminates.
%
\declareconstitemvalue{PMIX_PERSIST_APP}{3}
Retain data until the application terminates.
%
\declareconstitemvalue{PMIX_PERSIST_SESSION}{4}
Retain data until the session/allocation terminates.
%
\declareconstitemvalue{PMIX_PERSIST_INVALID}{UINT8_MAX}
Invalid value - typically used to indicate that a persistence has not yet been set.
%
\end{constantdesc}


%%%%%%%%%%%%%%%%%%%%%%%%%%%%%%%%%%%%%%%%%%%%%%%%%
\subsection{Lookup Related Data Structures}

\declarestruct{pmix_pdata_t}

The \refstruct{pmix_pdata_t} structure is used both to request the lookup of keys and to describe the value and publishing process of any keys that were successfully retrieved.
A request to lookup published values is described by an array of \refstruct{pmix_pdata_t} structures.
Only the key field is used in the lookup request.
The results of the lookup operation are returned in the same array with the proc and value fields set when the key is successfully found.
The value field's data type is set to \refconst{PMIX_UNDEF} in the associated \refarg{value} struct of any key which was not retrieved.
%
\copySignature{pmix_pdata_t}{1.0}{
typedef struct pmix_pdata \{ \\
\hspace*{4\sigspace}pmix_proc_t proc; \\
\hspace*{4\sigspace}pmix_key_t key; \\
\hspace*{4\sigspace}pmix_value_t value; \\
\} pmix_pdata_t;
}

where:
\begin{itemize}
    \item \emph{proc} is the process identifier of the data publisher.
    \item \emph{key} is the string key of the published data.
    \item \emph{value} is the value associated with the \emph{key}.
\end{itemize}

\declarestruct{pmix_pdsdata_t}

The \refstruct{pmix_pds_data_t} structure is used both to request the lookup of keys and to describe the value and publishing process of any keys that were successfully retrieved.
A request to lookup published values is described by an array of \refstruct{pmix_pdsdata_t} structures.
Only the key field is used in the lookup request.
The results of the lookup operation are returned in the same array with the proc and value fields set when the key is successfully found.
The value field's data type is set to \refconst{PMIX_UNDEF} in the associated \refarg{value} struct of any key which was not retrieved.
%
typedef struct pmix_pdsdata \{ \\
\hspace*{4\sigspace}pmix_key_t key; \\
\hspace*{4\sigspace}pmix_data_array_t value; \\
\hspace*{4\sigspace}pmix_data_array_t publish_id; \\
\} pmix_pdata_t;

where:

where:
\begin{itemize}
    \item \emph{key} is the string key of the published data.
    \item \emph{publish_id} is the identifier of the publish call that created the value.
    \item \emph{value} is the value associated with the \emph{key}.
\end{itemize}

%%%%%%%%%%%%%%%%%%%%%%%%%%%%%%%%%%%%%%%%%%%%%%%%%
%%%%%%%%%%%%%%%%%%%%%%%%%%%%%%%%%%%%%%%%%%%%%%%%%
\section{\code{PMIx_Lookup}}
\declareapi{PMIx_Lookup}

%%%%
\summary

Lookup information published by a process or host environment using \refapi{PMIx_Publish} or \refapi{PMIx_Publish_nb}.

%%%%
\format

\copySignature{PMIx_Lookup}{1.0}{
pmix_status_t \\
PMIx_Lookup(pmix_pdata_t data[], size_t ndata, \\
\hspace*{12\sigspace}const pmix_info_t info[], size_t ninfo);
}

\begin{arglist}
\arginout{data}{Array of publishable data structures (array of \refstruct{pmix_pdata_t})}
\argin{ndata}{Number of elements in the \refarg{data} array (integer)}
\argin{info}{Array of info structures (array of \refstruct{pmix_info_t})}
\argin{ninfo}{Number of elements in the \refarg{info} array (integer)}
\end{arglist}

\returnstart
\begin{itemize}
\item \refconst{PMIX_ERR_NOT_FOUND} None of the requested data could be found within the requester's range.

\item \refconst{PMIX_ERR_PARTIAL_SUCCESS} Some of the requested data was found.
Any key that cannot be found will return with a data type of \refconst{PMIX_UNDEF} in the associated \refarg{value} struct. Note that the specific reason for a particular piece of missing information (e.g., lack of permissions) cannot be communicated back to the requester in this situation.

\item \refconst{PMIX_ERR_NO_PERMISSIONS} All requested data was found and range restrictions were met for each specified key, but none of the matching data could be returned due to lack of access permissions.

\end{itemize}
\returnend

\reqattrstart
\ac{PMIx} libraries are not required to directly support any attributes for this function. However, any provided attributes must be passed to the host environment for processing, and the \ac{PMIx} library is required to add the \refAttributeItem{PMIX_USERID} and the \refAttributeItem{PMIX_GRPID} attributes of the client process that is requesting the info.

\reqattrend

\optattrstart
The following attributes are optional for host environments that support this operation:

\pasteAttributeItem{PMIX_TIMEOUT}
\pasteAttributeItem{PMIX_RANGE}
\pasteAttributeItem{PMIX_WAIT}

\optattrend

%%%%
\descr

Lookup information published by a process or host environment using \refapi{PMIx_Publish} or \refapi{PMIx_Publish_nb}.
A lookup operation is always performed on a range which can be specified using the directive \refAttributeItem{PMIX_RANGE} or otherwise defaults to \refconst{PMIX_RANGE_SESSION}.

The lookup operation will be constrained to data published to the specified range.
Data is returned per the retrieval rules of Section \ref{chap:pub:retrules}.

The \argref{data} parameter consists of an array of \refstruct{pmix_pdata_t} structures with the keys specifying the requested information.
Data will be returned for each \code{key} field in the associated \code{value} field of this structure as per the above description of return values. The \code{proc} field in each \refstruct{pmix_pdata_t} structure will contain the namespace/rank of the process that published the data.

\adviceuserstart
Although this is a blocking function, it will not wait by default for the requested data to be published.
Instead, it will block for the time required by the datastore to lookup its current data and return any found items.
Thus, the caller is responsible for either ensuring that data is published prior to executing a lookup, using \refattr{PMIX_WAIT} to instruct the datastore to wait for the data to be published, or retrying until the requested data is found.
\adviceuserend


%%%%%%%%%%%%%%%%%%%%%%%%%%%%%%%%%%%%%%%%%%%%%%%%%
%%%%%%%%%%%%%%%%%%%%%%%%%%%%%%%%%%%%%%%%%%%%%%%%%
\section{\code{PMIx_Lookup_nb}}
\declareapi{PMIx_Lookup_nb}

%%%%
\summary

Nonblocking version of \refapi{PMIx_Lookup}.

%%%%
\format

\copySignature{PMIx_Lookup_nb}{1.0}{
pmix_status_t \\
PMIx_Lookup_nb(char **keys, \\
\hspace*{15\sigspace}const pmix_info_t info[], size_t ninfo, \\
\hspace*{15\sigspace}pmix_lookup_cbfunc_t cbfunc, void *cbdata);
}

\begin{arglist}
\argin{keys}{\code{NULL}-terminated array of keys (array of strings)}
\argin{info}{Array of info structures (array of handles)}
\argin{ninfo}{Number of elements in the \refarg{info} array (integer)}
\argin{cbfunc}{Callback function (handle)}
\argin{cbdata}{Callback data to be provided to the callback function (pointer)}
\end{arglist}

\returnsimplenb

If executed, the status returned in the provided callback function will be one of the following constants:

\begin{itemize}
\item \refconst{PMIX_SUCCESS} All data was found and has been returned.

\item \refconst{PMIX_ERR_NOT_FOUND} None of the requested data was available within the requester's range. The \refarg{pdata} array in the callback function shall be \code{NULL} and the \refarg{npdata} parameter set to zero.

\item \refconst{PMIX_ERR_PARTIAL_SUCCESS} Some of the requested data was found.
Only found data will be included in the returned \refarg{pdata} array. Note that the specific reason for a particular piece of missing information (e.g., lack of permissions or the data has not been published) cannot be communicated back to the requester in this situation.

\item \refconst{PMIX_ERR_NOT_SUPPORTED} There is no available datastore (either at the host environment or \ac{PMIx} implementation level) on this system that supports this function.

\item \refconst{PMIX_ERR_NO_PERMISSIONS} All of the requested data was found and range restrictions were met for each specified key, but none of the matching data could be returned due to lack of access permissions.

\item a non-zero \ac{PMIx} error constant indicating a reason for the request's failure.
\end{itemize}

\reqattrstart
\ac{PMIx} libraries are not required to directly support any attributes for this function. However, any provided attributes must be passed to the host environment for processing, and the \ac{PMIx} library is required to add the \refAttributeItem{PMIX_USERID} and the \refAttributeItem{PMIX_GRPID} attributes of the client process that is requesting the info.

\reqattrend

\optattrstart
The following attributes are optional for host environments that support this operation:

\pasteAttributeItem{PMIX_TIMEOUT}
\pasteAttributeItem{PMIX_RANGE}
\pasteAttributeItem{PMIX_WAIT}

\optattrend

%%%%
\descr

Non-blocking form of the \refapi{PMIx_Lookup} function.

%%%%%%%%%%%%%%%%%%%%%%%%%%%%%%%%%%%%%%%%%%%%%%%%%
\subsubsection{Lookup data structure support macros}

The following macros are provided to support the \refstruct{pmix_pdata_t} structure.

%%%%
\littleheader{Static initializer for the pdata structure}
\declaremacroProvisional{PMIX_LOOKUP_STATIC_INIT}

Provide a static initializer for the \refstruct{pmix_pdata_t} fields.

\versionMarker{4.2}
\cspecificstart
\begin{codepar}
PMIX_LOOKUP_STATIC_INIT
\end{codepar}
\cspecificend


\littleheader{Initialize the pdata structure}
\declaremacro{PMIX_PDATA_CONSTRUCT}

Initialize the \refstruct{pmix_pdata_t} fields

\copySignature{PMIX_PDATA_CONSTRUCT}{1.0}{
PMIX_PDATA_CONSTRUCT(m)
}

\begin{arglist}
\argin{m}{Pointer to the structure to be initialized (pointer to \refstruct{pmix_pdata_t})}
\end{arglist}

\littleheader{Destruct the pdata structure}
\declaremacro{PMIX_PDATA_DESTRUCT}

Destruct the \refstruct{pmix_pdata_t} fields

\copySignature{PMIX_PDATA_DESTRUCT}{1.0}{
PMIX_PDATA_DESTRUCT(m)
}

\begin{arglist}
\argin{m}{Pointer to the structure to be destructed (pointer to \refstruct{pmix_pdata_t})}
\end{arglist}

%%%%%%%%%%%
\littleheader{Create a pdata array}
\declaremacro{PMIX_PDATA_CREATE}

Allocate and initialize an array of \refstruct{pmix_pdata_t} structures

\copySignature{PMIX_PDATA_CREATE}{1.0}{
PMIX_PDATA_CREATE(m, n)
}

\begin{arglist}
\arginout{m}{Address where the pointer to the array of \refstruct{pmix_pdata_t} structures shall be stored (handle)}
\argin{n}{Number of structures to be allocated (\code{size_t})}
\end{arglist}


%%%%%%%%%%%
\littleheader{Free a pdata structure}
\declaremacro{PMIX_PDATA_RELEASE}

Release a \refstruct{pmix_pdata_t} structure

\copySignature{PMIX_PDATA_RELEASE}{4.0}{
PMIX_PDATA_RELEASE(m)
}

\begin{arglist}
\argin{m}{Pointer to a \refstruct{pmix_pdata_t} structure (handle)}
\end{arglist}


%%%%%%%%%%%
\littleheader{Free a pdata array}
\declaremacro{PMIX_PDATA_FREE}

Release an array of \refstruct{pmix_pdata_t} structures

\copySignature{PMIX_PDATA_FREE}{1.0}{
PMIX_PDATA_FREE(m, n)
}

\begin{arglist}
\argin{m}{Pointer to the array of \refstruct{pmix_pdata_t} structures (handle)}
\argin{n}{Number of structures in the array (\code{size_t})}
\end{arglist}

%%%%%%%%%%%
\littleheader{Load a lookup data structure}
\declaremacro{PMIX_PDATA_LOAD}

This macro simplifies the loading of key, process identifier, and data into a \refstruct{pmix_pdata_t} by correctly assigning values to the structure's fields.

\copySignature{PMIX_PDATA_LOAD}{1.0}{
PMIX_PDATA_LOAD(m, p, k, d, t);
}

\begin{arglist}
\argin{m}{Pointer to the \refstruct{pmix_pdata_t} structure into which the key and data are to be loaded (pointer to \refstruct{pmix_pdata_t})}
\argin{p}{Pointer to the \refstruct{pmix_proc_t} structure containing the identifier of the process being referenced (pointer to \refstruct{pmix_proc_t})}
\argin{k}{String key to be loaded - must be less than or equal to \refconst{PMIX_MAX_KEYLEN} in length (handle)}
\argin{d}{Pointer to the data value to be loaded (handle)}
\argin{t}{Type of the provided data value (\refstruct{pmix_data_type_t})}
\end{arglist}

\adviceuserstart
Key, process identifier, and data will all be copied into the \refstruct{pmix_pdata_t} - thus, the source information can be modified or free'd without affecting the copied data once the macro has completed.
\adviceuserend

%%%%%%%%%%%
\littleheader{Transfer a lookup data structure}
\declaremacro{PMIX_PDATA_XFER}

This macro simplifies the transfer of key, process identifier, and data value between two\refstruct{pmix_pdata_t} structures.

\copySignature{PMIX_PDATA_XFER}{2.0}{
PMIX_PDATA_XFER(d, s);
}

\begin{arglist}
\argin{d}{Pointer to the destination \refstruct{pmix_pdata_t} (pointer to \refstruct{pmix_pdata_t})}
\argin{s}{Pointer to the source \refstruct{pmix_pdata_t} (pointer to \refstruct{pmix_pdata_t})}
\end{arglist}

\adviceuserstart
Key, process identifier, and data will all be copied into the destination \refstruct{pmix_pdata_t} - thus, the source \refstruct{pmix_pdata_t} may free'd without affecting the copied data once the macro has completed.
\adviceuserend

%%%%%%%%%%%%%%%%%%%%%%%%%%%%%%%%%%%%%%%%%%%%%%%%%
\section{Retrieval rules for published data}
\label{chap:pub:retrules}

The retrieval rules for published data primarily revolve around enforcing data access permissions and range constraints.
All publish and lookup operations operate on a range. If not specified, the range defaults to \refconst{PMIX_RANGE_SESSION}.
The key being looked up will match with a published key only if all of the following conditions are met:

\begin{enumerate}
    \item The lookup key matches the published key.
    \item The type of range specified by the publisher is the same as the type of range specified by the requester.
    \item The requestor must be a member of the range specified by the publisher.
    \item The publisher must be a member of the range specified by the requestor.
    \item If the publisher specified access permissions, the effective \ac{UID} and \ac{GID} of the requester must meet those requirements.
\end{enumerate}

The status returned by the datastore shall be set to:

\begin{itemize}
\item \refconst{PMIX_SUCCESS} All data was found and is included in the returned information.

\item \refconst{PMIX_ERR_NOT_FOUND} None of the requested data could be found within a requester's range.

\item \refconst{PMIX_ERR_PARTIAL_SUCCESS} Some of the requested data was found.
Only found data will be included in the returned information. Note that the specific reason for a particular piece of missing information (e.g., lack of permissions) cannot be communicated back to the requester in this situation.

\item \refconst {PMIX_ERR_NO_PERMISSIONS} All requested data was found and range restrictions were met for each specified key, but none of the matching data could be returned due to lack of access permissions.

\item a non-zero \ac{PMIx} error constant indicating a reason for the request's failure.
\end{itemize}

%%%%%%%%%%%%%%%%%%%%%%%%%%%%%%%%%%%%%%%%%%%%%%%%%
%%%%%%%%%%%%%%%%%%%%%%%%%%%%%%%%%%%%%%%%%%%%%%%%%
\section{\code{PMIx_Unpublish}}
\declareapi{PMIx_Unpublish}

%%%%
\summary

Unpublish a list of keys published by the calling process.

%%%%
\format

\copySignature{PMIx_Unpublish}{1.0}{
pmix_status_t \\
PMIx_Unpublish(char **keys, \\
\hspace*{15\sigspace}const pmix_info_t info[], size_t ninfo);
}

\begin{arglist}
\argin{keys}{\code{NULL}-terminated array of keys (array of strings)}
\argin{info}{Array of info structures (array of handles)}
\argin{ninfo}{Number of elements in the \refarg{info} array (integer)}
\end{arglist}

\returnsimple

\reqattrstart
\ac{PMIx} libraries are not required to directly support any attributes for this function. However, any provided attributes must be passed to the host environment for processing, and the \ac{PMIx} library is required to add the \refAttributeItem{PMIX_USERID} and the \refAttributeItem{PMIX_GRPID} attributes of the client process that is requesting the operation.

\reqattrend

\optattrstart
The following attributes are optional for host environments that support this operation:

\pasteAttributeItem{PMIX_TIMEOUT}
\pasteAttributeItem{PMIX_RANGE}

\optattrend

%%%%
\descr

Unpublish a list of keys published by the calling process.
The function will block until the data has been removed by the server (i.e., it is safe to publish that key again within the specified range).
A value of \code{NULL} for the \refarg{keys} parameter instructs the server to remove all data published by this process.

By default, the range is assumed to be \refconst{PMIX_RANGE_SESSION}.
Changes to the range, and any additional directives, can be provided in the \refarg{info} array.


%%%%%%%%%%%%%%%%%%%%%%%%%%%%%%%%%%%%%%%%%%%%%%%%%
%%%%%%%%%%%%%%%%%%%%%%%%%%%%%%%%%%%%%%%%%%%%%%%%%
\section{\code{PMIx_Unpublish_nb}}
\declareapi{PMIx_Unpublish_nb}

%%%%
\summary

Nonblocking version of \refapi{PMIx_Unpublish}.

%%%%
\format

\copySignature{PMIx_Unpublish_nb}{1.0}{
pmix_status_t \\
PMIx_Unpublish_nb(char **keys, \\
\hspace*{18\sigspace}const pmix_info_t info[], size_t ninfo, \\
\hspace*{18\sigspace}pmix_op_cbfunc_t cbfunc, void *cbdata);
}

\begin{arglist}
\argin{keys}{\code{NULL}-terminated array of keys (array of strings)}
\argin{info}{Array of info structures (array of handles)}
\argin{ninfo}{Number of elements in the \refarg{info} array (integer)}
\argin{cbfunc}{Callback function \refapi{pmix_op_cbfunc_t} (function reference)}
\argin{cbdata}{Data to be passed to the callback function (memory reference)}
\end{arglist}

\returnsimplenb

\returnstart
\begin{itemize}
    \item \refconst{PMIX_OPERATION_SUCCEEDED}, indicating that the request was immediately processed and returned \textit{success} - the \refarg{cbfunc} will \textit{not} be called.
\end{itemize}
\returnend

\reqattrstart
\ac{PMIx} libraries are not required to directly support any attributes for this function. However, any provided attributes must be passed to the host environment for processing, and the \ac{PMIx} library is required to add the \refAttributeItem{PMIX_USERID} and the \refAttributeItem{PMIX_GRPID} attributes of the client process that is requesting the operation.

\reqattrend

\optattrstart
The following attributes are optional for host environments that support this operation:

\pasteAttributeItem{PMIX_TIMEOUT}
\pasteAttributeItem{PMIX_RANGE}

\optattrend

%%%%
\descr

Non-blocking form of the \refapi{PMIx_Unpublish} function.
The callback function will be executed once the server confirms removal of the specified data. The \refarg{info} array must be maintained until the callback is provided.


%%%%%%%%%%%%%%%%%%%%%%%%%%%%%%%%%%%%%%%%%%%%%%%%%
