%%%%%%%%%%%%%%%%%%%%%%%%%%%%%%%%%%%%%%%%%%%%%%%%%
% Chapter: API Fabric support
%%%%%%%%%%%%%%%%%%%%%%%%%%%%%%%%%%%%%%%%%%%%%%%%%
\chapter{Fabric Support Definitions}
\label{chap:api_fabric}

As the drive for performance continues, interest has grown in both scheduling algorithms that take into account network locality of the allocated resources, and in optimizing collective communication patterns by structuring them to follow fabric topology. Several interfaces have been defined that are specifically intended to support \acp{WLM} (also known as \emph{schedulers}) by providing access to information of potential use to scheduling algorithms - e.g., information on communication costs between different points on the fabric.

In contrast, hierarchical collective operations require each process have global information about both its peers and the fabric. For example, one might aggregate the contribution from all processes on a node, then again across all nodes on a common switch, and finally across all switches. Creating such optimized patterns relies on detailed knowledge of the fabric location of each participant.

\ac{PMIx} supports these efforts by defining datatypes and attributes by which fabric coordinates for processes and devices can be obtained from the host \ac{SMS}. When used in conjunction with other \ac{PMIx} methods described in Chapter \ref{chap:api_server}, this results in the ability of a process to obtain the fabric coordinate of all other processes without incurring additional overhead associated with the publish/exchange of that information.



\section{Fabric Support Events}
\label{api:sched:consts}

The following events are defined for use in fabric-related operations.

\begin{constantdesc}

%
\declareconstitemNEW{PMIX_FABRIC_UPDATE_PENDING}
The \ac{PMIx} server library has been alerted to a change in the fabric that requires updating of one or more registered \refstruct{pmix_fabric_t} objects.

%
\declareconstitemNEW{PMIX_FABRIC_UPDATED}
The \ac{PMIx} server library has completed updating the entries of all affected \refstruct{pmix_fabric_t} objects registered with the library. Access to the entries of those objects may now resume.

%
\declareconstitemNEW{PMIX_FABRIC_UPDATE_ENDPOINTS}
Endpoint assignments have been updated, usually in response to migration
or restart of a process. Clients should use \refapi{PMIx_Get} to update any
internally cached connections.

\end{constantdesc}

%%%%%%%%%%%
\section{Fabric Support Datatypes}

Several datatype definitions have been created to support fabric-related operations and information.

\subsection{Fabric Coordinate Structure}
\declarestruct{pmix_coord_t}

The \refstruct{pmix_coord_t} structure describes the fabric coordinates of a specified process in a given view

\versionMarker{4.0}
\cspecificstart
\begin{codepar}
typedef struct pmix_coord \{
    char *fabric;
    char *plane;
    pmix_coord_view_t view;
    uint32_t *coord;
    size_t dims;
\} pmix_coord_t;
\end{codepar}
\cspecificend

All coordinate values shall be expressed as unsigned integers due to their units being defined in fabric devices and not physical distances. The coordinate is therefore an indicator of connectivity and not relative communication distance.

The fabric and plane fields are assigned by the fabric provider to help the user identify the fabric to which the coordinates refer. Note that providers are not required to assign any particular value to the fields and may choose to leave the fields blank. Example entries include \{"Ethernet", "mgmt"\} or \{"infiniband", "data1"\}.

\adviceimplstart
Note that the \refstruct{pmix_coord_t} structure does not imply nor mandate any requirement on how the coordinate data is to be stored within the \ac{PMIx} library. Implementers are free to store the coordinate in whatever format they choose.
\adviceimplend

A fabric coordinate is usually associated with a given fabric device - e.g., a particular \ac{NIC} and port on a node. Thus, while the fabric coordinate of a device must be unique in a given view, the coordinate may be shared by multiple processes on a node. If the node contains multiple fabric devices, then either the device closest to the binding location of a process shall be used as its coordinate, or (if the process is unbound or its binding is not known) all devices on the node shall be reported as a \refstruct{pmix_data_array_t} of \refstruct{pmix_coord_t} structures.

Nodes with multiple fabric devices can also have those devices configured as multiple \refterm{fabric planes}. In such cases, a given process (even if bound to a specific location) may be associated with a coordinate on each plane. The resulting set of fabric coordinates shall be reported as a \refstruct{pmix_data_array_t} of \refstruct{pmix_coord_t} structures. The caller may request a coordinate from a specific fabric plane by passing the \refattr{PMIX_FABRIC_PLANE} attribute as a directive/qualifier to the \refapi{PMIx_Get} or \refapi{PMIx_Query_info_nb} call.

\subsection{Fabric Coordinate Support Macros}
\label{api:netcoord:macros}

The following macros are provided to support the \refstruct{pmix_coord_t} structure.

%%%%
\littleheader{Initialize the coord structure}
\declaremacro{PMIX_COORD_CONSTRUCT}

Initialize the \refstruct{pmix_coord_t} fields

\versionMarker{4.0}
\cspecificstart
\begin{codepar}
PMIX_COORD_CONSTRUCT(m)
\end{codepar}
\cspecificend

\begin{arglist}
\argin{m}{Pointer to the structure to be initialized (pointer to \refstruct{pmix_coord_t})}
\end{arglist}

%%%%
\littleheader{Destruct the coord structure}
\declaremacro{PMIX_COORD_DESTRUCT}

Destruct the \refstruct{pmix_coord_t} fields

\versionMarker{4.0}
\cspecificstart
\begin{codepar}
PMIX_COORD_DESTRUCT(m)
\end{codepar}
\cspecificend

\begin{arglist}
\argin{m}{Pointer to the structure to be destructed (pointer to \refstruct{pmix_coord_t})}
\end{arglist}

%%%%
\littleheader{Create a coord array}
\declaremacro{PMIX_COORD_CREATE}

Allocate and initialize a \refstruct{pmix_coord_t} array

\versionMarker{4.0}
\cspecificstart
\begin{codepar}
PMIX_COORD_CREATE(m, n)
\end{codepar}
\cspecificend

\begin{arglist}
\arginout{m}{Address where the pointer to the array of \refstruct{pmix_coord_t} structures shall be stored (handle)}
\argin{n}{Number of structures to be allocated (\code{size_t})}
\end{arglist}

%%%%
\littleheader{Release a coord array}
\declaremacro{PMIX_COORD_FREE}

Release an array of \refstruct{pmix_coord_t} structures

\versionMarker{4.0}
\cspecificstart
\begin{codepar}
PMIX_COORD_FREE(m, n)
\end{codepar}
\cspecificend

\begin{arglist}
\argin{m}{Pointer to the array of \refstruct{pmix_coord_t} structures (handle)}
\argin{n}{Number of structures in the array (\code{size_t})}
\end{arglist}


%%%%%%%%%%%%
\subsection{Fabric Coordinate Views}
\declarestruct{pmix_coord_view_t}

\versionMarker{4.0}
\cspecificstart
\begin{codepar}
typedef uint8_t pmix_coord_view_t;
#define PMIX_COORD_VIEW_UNDEF       0x00
#define PMIX_COORD_LOGICAL_VIEW     0x01
#define PMIX_COORD_PHYSICAL_VIEW    0x02
\end{codepar}
\cspecificend

Fabric coordinates can be reported based on different \emph{views} according to user preference at the time of request. The following views have been defined:

\begin{constantdesc}
%
\declareconstitemNEW{PMIX_COORD_VIEW_UNDEF}
The coordinate view has not been defined.
%
\declareconstitemNEW{PMIX_COORD_LOGICAL_VIEW}
The coordinates are provided in a \emph{logical} view, typically given in Cartesian (x,y,z) dimensions, that describes the data flow in the fabric as defined by the arrangement of the hierarchical addressing scheme, fabric segmentation, routing domains, and other similar factors employed by that fabric.
%
\declareconstitemNEW{PMIX_COORD_PHYSICAL_VIEW}
The coordinates are provided in a \emph{physical} view based on the actual wiring diagram of the fabric - i.e., values along each axis reflect the relative position of that interface on the specific fabric cabling.
%
\end{constantdesc}

\adviceimplstart
\ac{PMIx} library implementers are advised to avoid declaring the above constants as actual \code{enum} values in order to allow host environments to add support for possibly proprietary coordinate views.
\adviceimplend

If the requester does not specify a view, coordinates shall default to the \emph{logical} view.


\subsection{Fabric Link State}
\declarestruct{pmix_link_state_t}

The \refstruct{pmix_link_state_t} is a \code{uint32_t} type for fabric link states.

\versionMarker{4.0}
\cspecificstart
\begin{codepar}
typedef uint8_t pmix_link_state_t;
\end{codepar}
\cspecificend

The following constants can be used to set a variable of the type \refstruct{pmix_link_state_t}. All definitions were introduced in version 4 of the standard unless otherwise marked. Valid link state values start at zero.

\begin{constantdesc}
%
\declareconstitemNEW{PMIX_LINK_STATE_UNKNOWN}
The port state is unknown or not applicable.

\declareconstitemNEW{PMIX_LINK_DOWN}
The port is inactive.

\declareconstitemNEW{PMIX_LINK_UP}
The port is active.

\end{constantdesc}

\subsection{Fabric Operation Constants}
\declarestruct{pmix_fabric_operation_t}

\versionMarker{4.0}
The \refstruct{pmix_fabric_operation_t} datatype is an enumerated type for specifying fabric operations used in the \ac{PMIx} server module's \refapi{pmix_server_fabric_fn_t} \ac{API}.

\begin{constantdesc}
%
\declareconstitemNEW{PMIX_FABRIC_REQUEST_INFO}
Request information on a specific fabric - if the fabric isn't specified as per \refapi{PMIx_Fabric_register}, then return information on the system default fabric. Information to be returned is described in \refstruct{pmix_fabric_t}.
%
\declareconstitemNEW{PMIX_FABRIC_UPDATE_INFO}
Update information on a specific fabric - the index of the fabric (\refattr{PMIX_FABRIC_INDEX}) to be updated must be provided.
%
\declareconstitemNEW{PMIX_FABRIC_GET_VERTEX_INFO}
Request information on a specific device within the identified fabric - the index of the device (\refattr{PMIX_FABRIC_DEVICE_INDEX}) and of the fabric (\refattr{PMIX_FABRIC_INDEX}) must be provided. If the device identifier is not specified, then return vertex info on all devices in the fabric. Information to be included on each vertex is described in \refstruct{pmix_fabric_t}.

\adviceuserstart
Requesting information on every device in the fabric may be an expensive operation in terms of both memory footprint and time.
\adviceuserend
%
\declareconstitemNEW{PMIX_FABRIC_GET_DEVICE_INDEX}
Request the fabric-wide index (returned as \refattr{PMIX_FABRIC_DEVICE_INDEX}) for a specific device within the identified fabric based on the provided vertex information. The index of the fabric must be provided.
%
\end{constantdesc}


\subsection{Fabric registration structure}
\declarestruct{pmix_fabric_t}

The \refstruct{pmix_fabric_t} structure is used by a \ac{WLM} to interact with fabric-related \ac{PMIx} interfaces, and to provide information about the fabric for use in scheduling algorithms or other purposes.

\versionMarker{4.0}
\cspecificstart
\begin{codepar}
typedef struct pmix_fabric_s \{
    char *name;
    size_t index;
    pmix_info_t *info;
    size_t ninfo;
    void *module;
\} pmix_fabric_t;;
\end{codepar}
\cspecificend

Note that in this structure:

\begin{itemize}
    \item \refarg{name} is an optional user-supplied string name identifying the fabric being referenced by this struct. If provided, the field must be a \code{NULL}-terminated string composed of standard alphanumeric values supported by common utilities such as \textit{strcmp}.;
    \item \refarg{index} is a \ac{PMIx}-provided number identifying this object;
    \item \refarg{info} is an array of \refstruct{pmix_info_t} containing information (provided by the \ac{PMIx} library) about the fabric;
    \item \refarg{ninfo} is the number of elements in the \refarg{info} array
    \item \refarg{module} points to an opaque object reserved for use by the \ac{PMIx} server library.
\end{itemize}

Note that only the \refarg{name} field is provided by the user - all other fields are provided by the \ac{PMIx} library and must not be modified by the user. The \refarg{info} array contains a varying amount of information depending upon both the \ac{PMIx} implementation and information available from the fabric vendor. At a minimum, it must contain (ordering is arbitrary):

\reqattrstart

\pasteAttributeItem{PMIX_FABRIC_VENDOR}
\pasteAttributeItem{PMIX_FABRIC_IDENTIFIER}
\pasteAttributeItem{PMIX_FABRIC_NUM_VERTICES}

\reqattrend

and may optionally contain one or more of the following:

\optattrstart
\pasteAttributeItem{PMIX_FABRIC_COST_MATRIX}
\pasteAttributeItem{PMIX_FABRIC_GROUPS}
\pasteAttributeItem{PMIX_FABRIC_DIMS}
\pasteAttributeItem{PMIX_FABRIC_PLANE}
\pasteAttributeItem{PMIX_FABRIC_SHAPE}
\pasteAttributeItem{PMIX_FABRIC_SHAPE_STRING}

While unusual due to scaling issues, implementations may include an array of \refattr{PMIX_FABRIC_DEVICE} elements describing the vertex information for each device in the system. Each element shall contain a \refstruct{pmix_data_array_t} of \refstruct{pmix_info_t} values describing the device. Each array may contain one or more of the following (ordering is arbitrary):

\pasteAttributeItem{PMIX_FABRIC_DEVICE_NAME}
\pasteAttributeItem{PMIX_FABRIC_DEVICE_VENDOR}
\pasteAttributeItem{PMIX_FABRIC_DEVICE_ID}
\pasteAttributeItem{PMIX_HOSTNAME}
\pasteAttributeItem{PMIX_FABRIC_DEVICE_DRIVER}
\pasteAttributeItem{PMIX_FABRIC_DEVICE_FIRMWARE}
\pasteAttributeItem{PMIX_FABRIC_DEVICE_ADDRESS}
\pasteAttributeItem{PMIX_FABRIC_DEVICE_MTU}
\pasteAttributeItem{PMIX_FABRIC_DEVICE_SPEED}
\pasteAttributeItem{PMIX_FABRIC_DEVICE_STATE}
\pasteAttributeItem{PMIX_FABRIC_DEVICE_TYPE}
\pasteAttributeItem{PMIX_FABRIC_DEVICE_BUS_TYPE}
\pasteAttributeItem{PMIX_FABRIC_DEVICE_PCI_DEVID}

\optattrend

%%%%
\subsubsection{Initialize the fabric structure}
\declaremacro{PMIX_FABRIC_CONSTRUCT}

Initialize the \refstruct{pmix_fabric_t} fields

\versionMarker{4.0}
\cspecificstart
\begin{codepar}
PMIX_FABRIC_CONSTRUCT(m)
\end{codepar}
\cspecificend

\begin{arglist}
\argin{m}{Pointer to the structure to be initialized (pointer to \refstruct{pmix_fabric_t})}
\end{arglist}


\section{Fabric Support Attributes}
\label{api:sched:attrs}

The following attribute is used by the \ac{PMIx} server library supporting the system's \ac{WLM} to indicate that it wants access to the fabric support functions:

\declareAttributeNEW{PMIX_SERVER_SCHEDULER}{"pmix.srv.sched"}{bool}{
Server requests access to \ac{WLM}-supporting features - passed solely to the \refapi{PMIx_server_init} \ac{API} to indicate that the library is to be initialized for scheduler support.
}

\vspace{\baselineskip}
The following attributes may be returned by calls to the scheduler-related \acp{API} or in response to queries (e.g., \refapi{PMIx_Get} or \refapi{PMIx_Query_info}) made by processes or tools.

\declareAttributeNEW{PMIX_FABRIC_COST_MATRIX}{"pmix.fab.cm"}{pointer}{
Pointer to a two-dimensional square array of point-to-point relative communication costs expressed as \code{uint16_t} values
}

\declareAttributeNEW{PMIX_FABRIC_GROUPS}{"pmix.fab.grps"}{string}{
A string delineating the group membership of nodes in the system, where each fabric group consists of the group number followed by a colon and a comma-delimited list of nodes in that group, with the groups delimited by semi-colons (e.g., 0:node000,node002,node004,node006;1:node001,node003,node005,node007)
}

\declareAttributeNEW{PMIX_FABRIC_VENDOR}{"pmix.fab.vndr"}{string}{
Name of fabric vendor (e.g., Amazon, Mellanox, Cray, Intel)
}

\declareAttributeNEW{PMIX_FABRIC_IDENTIFIER}{"pmix.fab.id"}{string}{
An identifier for the fabric (e.g., MgmtEthernet, Slingshot-11, OmniPath-1)
}

\declareAttributeNEW{PMIX_FABRIC_INDEX}{"pmix.fab.idx"}{size_t}{
The index of the fabric as returned in \refstruct{pmix_fabric_t}
}

\declareAttributeNEW{PMIX_FABRIC_NUM_VERTICES}{"pmix.fab.nverts"}{size_t}{
Total number of fabric devices in the system - corresponds to the number of vertices (i.e., rows and columns) in the cost matrix
}

%
\declareAttributeNEW{PMIX_FABRIC_COORDINATE}{"pmix.fab.coord"}{pmix_data_array_t}{
Array of \refstruct{pmix_coord_t} fabric coordinates of the specified process in the view and/or plane provided by the requester. If only one fabric device has been assigned to the specified process, then the array will contain only one address. Otherwise, the array will contain the coordinates of all devices available to the process in order of least to greatest distance from the process (devices equally distant from the process will be listed in arbitrary order).
}

%
\declareAttributeNEW{PMIX_FABRIC_DIMS}{"pmix.fab.dims"}{uint32_t}{
Number of dimensions in the specified fabric plane/view. If no plane is specified in a request, then the dimensions of all planes in the system will be returned as a \refstruct{pmix_data_array_t} containing an array of \code{uint32_t} values. Default is to provide dimensions in \emph{logical} view.
}

%
\declareAttributeNEW{PMIX_FABRIC_ENDPT}{"pmix.fab.endpt"}{pmix_data_array_t}{
Fabric endpoints for a specified process. As multiple endpoints may be assigned to a given process (e.g., in the case where multiple devices are associated with a package to which the process is bound), the returned values will be provided in a \refstruct{pmix_data_array_t} - the returned data type of the individual values in the array varies by fabric provider.
}

%
\declareAttributeNEW{PMIX_FABRIC_SHAPE}{"pmix.fab.shape"}{pmix_data_array_t*}{
The size of each dimension in the specified fabric plane/view, returned in a \refstruct{pmix_data_array_t} containing an array of \code{uint32_t} values. The size is defined as the number of elements present in that dimension - e.g., the number of devices in one dimension of a physical view of a fabric plane. If no plane is specified, then the shape of each plane in the system will be returned in a \refstruct{pmix_data_array_t} array where each element is itself a two-element array containing the \refattr{PMIX_FABRIC_PLANE} followed by that plane's fabric shape. Default is to provide the shape in \emph{logical} view.
}

\declareAttributeNEW{PMIX_FABRIC_SHAPE_STRING}{"pmix.fab.shapestr"}{string}{
Network shape expressed as a string (e.g., "10x12x2"). If no plane is specified, then the shape of each plane in the system will be returned in a \refstruct{pmix_data_array_t} array where each element is itself a two-element array containing the \refattr{PMIX_FABRIC_PLANE} followed by that plane's fabric shape string. Default is to provide the shape in \emph{logical} view.
}

\declareAttributeNEW{PMIX_SWITCH_PEERS}{"pmix.speers"}{string}{
Comma-delimited string of peer ranks that share the same switch as the process specified in the call to \refapi{PMIx_Get}. Single-device environments will return a string. Multi-device environments will return a \refstruct{pmix_data_array_t} array of results, each element consisting of a two-element array containing the \refattr{PMIX_FABRIC_DEVICE_INDEX} of the local fabric device and a comma-delimited string of peer ranks sharing the switch to which that device is connected.
}

\vspace{\baselineskip}
The following attributes can be used either as a key (e.g., when requesting information via \refapi{PMIx_Get}) or as a modifier to such a request:

%
\declareAttributeNEW{PMIX_FABRIC_PLANE}{"pmix.fab.plane"}{char*}{
ID string of a fabric plane (e.g., CIDR for Ethernet). When used as a modifier in a request for information, specifies the plane whose information is to be returned. When used directly as a key in a request, returns a \refstruct{pmix_data_array_t} of string identifiers for all fabric planes in the system.
}

%
\declareAttributeNEW{PMIX_FABRIC_SWITCH}{"pmix.fab.switch"}{char*}{
ID string of a fabric switch. When used as a modifier in a request for information, specifies the switch whose information is to be returned. When used directly as a key in a request, returns a \refstruct{pmix_data_array_t} of string identifiers for all fabric switches in the system.
}
%
\declareAttributeNEW{PMIX_FABRIC_VIEW}{"pmix.fab.view"}{pmix_coord_view_t}{
Fabric coordinate view to be used for the requested coordinate - see \refstruct{pmix_coord_view_t} for the list of accepted values. This attribute is solely defined as a modifier/qualifier by which the caller can indicate the desired coordinate view for the information being requested and can not be used as a key to a query.
}

\vspace{\baselineskip}
The following attributes are used to describe devices attached to the fabric.

\declareAttributeNEW{PMIX_FABRIC_DEVICE}{"pmix.fabdev"}{\refstruct{pmix_data_array_t}}{
An array of \refstruct{pmix_info_t} describing a particular fabric device using one or more of the attributes defined below.
}

\declareAttributeNEW{PMIX_FABRIC_DEVICE_INDEX}{"pmix.fabdev.idx"}{\code{uint32_t}}{
System-unique index of a particular fabric device.
}

\declareAttributeNEW{PMIX_FABRIC_DEVICE_NAME}{"pmix.fabdev.nm"}{string}{
The operating system name associated with the device. This may be a logical fabric interface name (e.g. eth0 or eno1) or an absolute filename.
}

\declareAttributeNEW{PMIX_FABRIC_DEVICE_VENDOR}{"pmix.fabdev.vndr"}{string}{
Indicates the name of the vendor that distributes the device.
}

\declareAttributeNEW{PMIX_FABRIC_DEVICE_BUS_TYPE}{"pmix.fabdev.btyp"}{string}{
The type of bus to which the device is attached (e.g., "PCI", "GEN-Z").
}

\declareAttributeNEW{PMIX_FABRIC_DEVICE_ID}{"pmix.fabdev.devid"}{string}{
This is a vendor-provided identifier for the device or product.
}

\declareAttributeNEW{PMIX_FABRIC_DEVICE_DRIVER}{"pmix.fabdev.driver"}{string}{
The name of the driver associated with the device
}

\declareAttributeNEW{PMIX_FABRIC_DEVICE_FIRMWARE}{"pmix.fabdev.fmwr"}{string}{
The device’s firmware version
}

\declareAttributeNEW{PMIX_FABRIC_DEVICE_ADDRESS}{"pmix.fabdev.addr"}{string}{
The primary link-level address associated with the device, such as a \ac{MAC} address. If multiple addresses are available, only one will be reported.
}

\declareAttributeNEW{PMIX_FABRIC_DEVICE_MTU}{"pmix.fabdev.mtu"}{size_t}{
The maximum transfer unit of link level frames or packets, in bytes.
}

\declareAttributeNEW{PMIX_FABRIC_DEVICE_SPEED}{"pmix.fabdev.speed"}{size_t}{
The active link data rate, given in bits per second.
}

\declareAttributeNEW{PMIX_FABRIC_DEVICE_STATE}{"pmix.fabdev.state"}{\refstruct{pmix_link_state_t}}{
The last available physical port state for the specified device. Possible values are \refconst{PMIX_LINK_STATE_UNKNOWN}, \refconst{PMIX_LINK_DOWN}, and \refconst{PMIX_LINK_UP}, to indicate if the port state is unknown or not applicable (unknown), inactive (down), or active (up).
}

\declareAttributeNEW{PMIX_FABRIC_DEVICE_TYPE}{"pmix.fabdev.type"}{string}{
Specifies the type of fabric interface currently active on the device, such as Ethernet or InfiniBand.
}

\declareAttributeNEW{PMIX_FABRIC_DEVICE_PCI_DEVID}{"pmix.fabdev.pcidevid"}{string}{
A node-level unique identifier for a \ac{PCI} device. Provided only if the device is located on a \ac{PCI} bus. The identifier is constructed as a four-part tuple delimited by colons comprised of the \ac{PCI} 16-bit domain, 8-bit bus, 8-bit device, and 8-bit function IDs, each expressed in zero-extended hexadecimal form. Thus, an example identifier might be "abc1:0f:23:01". The combination of node identifier (\refattr{PMIX_HOSTNAME} or \refattr{PMIX_NODEID}) and \refattr{PMIX_FABRIC_DEVICE_PCI_DEVID} shall be unique within the system.
}


%%%%%%%%%%%
\section{Fabric Support Functions}

The following \acp{API} allow the \ac{WLM} to request specific services from the fabric subsystem via the \ac{PMIx} library.

\advicermstart
Due to their high cost in terms of execution, memory consumption, and interactions with other \ac{SMS} components (e.g., a fabric manager), it is strongly advised that the underlying implementation of these \acp{API} be restricted to a single \ac{PMIx} server in a system that is supporting the \ac{SMS} component responsible for the scheduling of allocations (i.e., the system \refterm{scheduler}). The \refattr{PMIX_SERVER_SCHEDULER} attribute can be used for this purpose to control the execution path. Clients, tools, and other servers utilizing these functions are advised to have their requests forwarded to the server supporting the scheduler using the \refapi{pmix_server_fabric_fn_t} server module function, as needed.
\advicermend

%%%%%%%%%%%
\subsection{\code{PMIx_Fabric_register}}
\declareapi{PMIx_Fabric_register}

%%%%
\summary

Register for access to fabric-related information.

%%%%
\format

\versionMarker{4.0}
\cspecificstart
\begin{codepar}
pmix_status_t
PMIx_Fabric_register(pmix_fabric_t *fabric,
                     const pmix_info_t directives[],
                     size_t ndirs)
\end{codepar}
\cspecificend

\begin{arglist}
\argin{fabric}{address of a \refstruct{pmix_fabric_t} (backed by storage). User may populate the "name" field at will - \ac{PMIx} does not utilize this field (handle)}
\argin{directives}{an optional array of values indicating desired behaviors and/or fabric to be accessed. If \code{NULL}, then the highest priority available fabric will be used (array of handles)}
\argin{ndirs}{Number of elements in the \refarg{directives} array (integer)}
\end{arglist}

Returns \refconst{PMIX_SUCCESS} or a negative value corresponding to a \ac{PMIx} error constant.

\reqattrstart
The following directives are required to be supported by all \ac{PMIx} libraries to aid users in identifying the fabric whose data is being sought:

\pasteAttributeItem{PMIX_FABRIC_PLANE}
\pasteAttributeItem{PMIX_FABRIC_IDENTIFIER}
\pasteAttributeItem{PMIX_FABRIC_VENDOR}

\reqattrend

%%%%
\descr

Register for access to fabric-related information, including the communication cost matrix. This call must be made prior to requesting information from a fabric. The caller may request access to a particular fabric using the vendor, type, or identifier, or to a specific \refterm{fabric plane} via the \refattr{PMIX_FABRIC_PLANE} attribute - otherwise, the default fabric will be returned.

For performance reasons, the \ac{PMIx} library does not provide thread protection for accessing the information in the \refstruct{pmix_fabric_t} structure. Instead, the \ac{PMIx} implementation shall provide two methods for coordinating updates to the provided fabric information:

\begin{itemize}

    \item Users may periodically poll for updates using the \refapi{PMIx_Fabric_update} \ac{API}

    \item Users may register for \refconst{PMIX_FABRIC_UPDATE_PENDING} events indicating that an update to the cost matrix is pending. When received, users are required to terminate or pause any actions involving access to the cost matrix before returning from the event. Completion of the \refconst{PMIX_FABRIC_UPDATE_PENDING} event handler indicates to the \ac{PMIx} library that the fabric object's entries are available for updating. This may include releasing and re-allocating memory as the number of vertices may have changed (e.g., due to addition or removal of one or more devices). When the update has been completed, the \ac{PMIx} library will generate a \refconst{PMIX_FABRIC_UPDATED} event indicating that it is safe to begin using the updated fabric object(s).

\end{itemize}

There is no requirement that the caller exclusively use either one of these options. For example, the user may choose to both register for fabric update events, but poll for an update prior to some critical operation.

%%%%%%%%%%%
\subsection{\code{PMIx_Fabric_register_nb}}
\declareapi{PMIx_Fabric_register_nb}

%%%%
\summary

Register for access to fabric-related information.

%%%%
\format

\versionMarker{4.0}
\cspecificstart
\begin{codepar}
pmix_status_t
PMIx_Fabric_register_nb(pmix_fabric_t *fabric,
                        const pmix_info_t directives[],
                        size_t ndirs,
                        pmix_op_cbfunc_t cbfunc, void *cbdata)
\end{codepar}
\cspecificend

\begin{arglist}
\argin{fabric}{address of a \refstruct{pmix_fabric_t} (backed by storage). User may populate the "name" field at will - \ac{PMIx} does not utilize this field (handle)}
\argin{directives}{an optional array of values indicating desired behaviors and/or fabric to be accessed. If \code{NULL}, then the highest priority available fabric will be used (array of handles)}
\argin{ndirs}{Number of elements in the \refarg{directives} array (integer)}
\argin{cbfunc}{Callback function \refapi{pmix_op_cbfunc_t} (function reference)}
\argin{cbdata}{Data to be passed to the callback function (memory reference)}
\end{arglist}

Returns one of the following:

\begin{itemize}
\item \refconst{PMIX_SUCCESS} indicating that the request has been accepted for processing and the provided callback function will be executed upon completion of the operation. Note that the library must not invoke the callback function prior to returning from the \ac{API}.
\item a non-zero \ac{PMIx} error constant indicating a reason for the request to have been rejected. In this case, the provided callback function will not be executed
\end{itemize}


%%%%
\descr

Non-blocking form of \refapi{PMIx_Fabric_register}. The caller is not allowed to access the provided \refstruct{pmix_fabric_t} until the callback function has been executed, at which time the fabric information will have been loaded into the provided structure.

%%%%%%%%%%%
\subsection{\code{PMIx_Fabric_update}}
\declareapi{PMIx_Fabric_update}

%%%%
\summary

Update fabric-related information.

%%%%
\format

\versionMarker{4.0}
\cspecificstart
\begin{codepar}
pmix_status_t
PMIx_Fabric_update(pmix_fabric_t *fabric)
\end{codepar}
\cspecificend

\begin{arglist}
\argin{fabric}{address of a \refstruct{pmix_fabric_t} (backed by storage) (handle)}
\end{arglist}

Returns \refconst{PMIX_SUCCESS} or a negative value corresponding to a \ac{PMIx} error constant.

%%%%
\descr

Update fabric-related information. This call can be made at any time to request an update of the fabric information contained in the provided \refstruct{pmix_fabric_t} object. The caller is not allowed to access the provided \refstruct{pmix_fabric_t} until the call has returned.


%%%%%%%%%%%
\subsection{\code{PMIx_Fabric_update_nb}}
\declareapi{PMIx_Fabric_update_nb}

%%%%
\summary

Update fabric-related information.

%%%%
\format

\versionMarker{4.0}
\cspecificstart
\begin{codepar}
pmix_status_t
PMIx_Fabric_update_nb(pmix_fabric_t *fabric,
                      pmix_op_cbfunc_t cbfunc, void *cbdata)
\end{codepar}
\cspecificend

\begin{arglist}
\argin{fabric}{address of a \refstruct{pmix_fabric_t} (handle)}
\argin{cbfunc}{Callback function \refapi{pmix_op_cbfunc_t} (function reference)}
\argin{cbdata}{Data to be passed to the callback function (memory reference)}
\end{arglist}

Returns one of the following:

\begin{itemize}
\item \refconst{PMIX_SUCCESS} indicating that the request has been accepted for processing and the provided callback function will be executed upon completion of the operation. Note that the library must not invoke the callback function prior to returning from the \ac{API}.
\item a non-zero \ac{PMIx} error constant indicating a reason for the request to have been rejected. In this case, the provided callback function will not be executed
\end{itemize}

%%%%
\descr

Non-blocking form of \refapi{PMIx_Fabric_update}. The caller is not allowed to access the provided \refstruct{pmix_fabric_t} until the callback function has been executed.


%%%%%%%%%%%
\subsection{\code{PMIx_Fabric_deregister}}
\declareapi{PMIx_Fabric_deregister}

%%%%
\summary

Deregister a fabric object.

%%%%
\format

\versionMarker{4.0}
\cspecificstart
\begin{codepar}
pmix_status_t PMIx_Fabric_deregister(pmix_fabric_t *fabric)
\end{codepar}
\cspecificend

\begin{arglist}
\argin{fabric}{address of a \refstruct{pmix_fabric_t} (handle)}
\end{arglist}

Returns \refconst{PMIX_SUCCESS} or a negative value corresponding to a \ac{PMIx} error constant.

%%%%
\descr

Deregister a fabric object, providing an opportunity for the \ac{PMIx} library to cleanup any information (e.g., cost matrix) associated with it. Contents of the provided \refstruct{pmix_fabric_t} will be invalidated upon function return.


%%%%%%%%%%%
\subsection{\code{PMIx_Fabric_deregister_nb}}
\declareapi{PMIx_Fabric_deregister_nb}

%%%%
\summary

Deregister a fabric object.

%%%%
\format

\versionMarker{4.0}
\cspecificstart
\begin{codepar}
pmix_status_t PMIx_Fabric_deregister_nb(pmix_fabric_t *fabric,
                            pmix_op_cbfunc_t cbfunc, void *cbdata)
\end{codepar}
\cspecificend

\begin{arglist}
\argin{fabric}{address of a \refstruct{pmix_fabric_t} (handle)}
\argin{cbfunc}{Callback function \refapi{pmix_op_cbfunc_t} (function reference)}
\argin{cbdata}{Data to be passed to the callback function (memory reference)}
\end{arglist}

Returns one of the following:

\begin{itemize}
\item \refconst{PMIX_SUCCESS} indicating that the request has been accepted for processing and the provided callback function will be executed upon completion of the operation. Note that the library must not invoke the callback function prior to returning from the \ac{API}.
\item a non-zero \ac{PMIx} error constant indicating a reason for the request to have been rejected. In this case, the provided callback function will not be executed
\end{itemize}

%%%%
\descr

Non-blocking form of \refapi{PMIx_Fabric_deregister}. Provided \refarg{fabric} must not be accessed until after callback function has been executed.


%%%%%%%%%%%
\subsection{\code{PMIx_Fabric_get_vertex_info}}
\declareapi{PMIx_Fabric_get_vertex_info}

%%%%
\summary

Given a communication cost matrix index for a specified fabric, return the corresponding vertex info.

%%%%
\format

\versionMarker{4.0}
\cspecificstart
\begin{codepar}
pmix_status_t
PMIx_Fabric_get_vertex_info(pmix_fabric_t *fabric, uint32_t index,
                            pmix_info_t **info, size_t *ninfo)
\end{codepar}
\cspecificend

\begin{arglist}
\argin{fabric}{address of a \refstruct{pmix_fabric_t} (handle)}
\argin{index}{vertex index (i.e., communication cost matrix row or column number) (integer)}
\arginout{info}{Address where a pointer to an array of \refstruct{pmix_info_t} containing the results of the query can be returned (memory reference)}
\arginout{ninfo}{Address where the number of elements in \refarg{info} can be returned (handle)}
\end{arglist}

Returns one of the following:

\begin{itemize}
    \item \refconst{PMIX_SUCCESS}, indicating return of a valid value.
    \item \refconst{PMIX_ERR_BAD_PARAM}, indicating that the provided index is out of bounds.
    \item a \ac{PMIx} error constant indicating either an error in the input or that the request failed.
\end{itemize}

%%%%
\descr

Query information about a specified vertex (a.k.a., fabric device) in the system. The returned \refarg{status} indicates if requested data was found or not. The returned array of \refstruct{pmix_info_t} will contain information on the specified vertex - the exact contents will depend on the \ac{PMIx} implementation and the fabric vendor. At a minimum, it must contain sufficient information to uniquely identify the device within the system (ordering is arbitrary):

\reqattrstart
\pasteAttributeItemBegin{PMIX_HOSTNAME} The \refattr{PMIX_NODEID} may be returned in its place, or in addition to the hostname.
\pasteAttributeItemEnd
\pasteAttributeItem{PMIX_FABRIC_DEVICE_NAME}
\pasteAttributeItem{PMIX_FABRIC_DEVICE_VENDOR}
\pasteAttributeItem{PMIX_FABRIC_DEVICE_BUS_TYPE}
\pasteAttributeItemBegin{PMIX_FABRIC_DEVICE_PCI_DEVID} This item should be included if the device bus type is \ac{PCI} - the equivalent should be provided for any other bus type.
\pasteAttributeItemEnd

\reqattrend

The returned array may optionally contain one or more of the following:

\optattrstart
\pasteAttributeItem{PMIX_FABRIC_DEVICE_ID}
\pasteAttributeItem{PMIX_FABRIC_DEVICE_DRIVER}
\pasteAttributeItem{PMIX_FABRIC_DEVICE_FIRMWARE}
\pasteAttributeItem{PMIX_FABRIC_DEVICE_ADDRESS}
\pasteAttributeItem{PMIX_FABRIC_DEVICE_MTU}
\pasteAttributeItem{PMIX_FABRIC_DEVICE_SPEED}
\pasteAttributeItem{PMIX_FABRIC_DEVICE_STATE}
\pasteAttributeItem{PMIX_FABRIC_DEVICE_TYPE}
\optattrend

The caller is responsible for releasing the returned array.



%%%%%%%%%%%
\subsection{\code{PMIx_Fabric_get_vertex_info_nb}}
\declareapi{PMIx_Fabric_get_vertex_info_nb}

%%%%
\summary

Given a communication cost matrix index for a specified fabric, return the corresponding vertex info.

%%%%
\format

\versionMarker{4.0}
\cspecificstart
\begin{codepar}
pmix_status_t
PMIx_Fabric_get_vertex_info_nb(pmix_fabric_t *fabric, uint32_t index,
                               pmix_info_cbfunc_t cbfunc, void *cbdata)
\end{codepar}
\cspecificend

\begin{arglist}
\argin{fabric}{address of a \refstruct{pmix_fabric_t} (handle)}
\argin{index}{vertex index (i.e., communication cost matrix row or column number) (integer)}
\argin{cbfunc}{Callback function \refapi{pmix_info_cbfunc_t} (function reference)}
\argin{cbdata}{Data to be passed to the callback function (memory reference)}
\end{arglist}

Returns one of the following:

\begin{itemize}
\item \refconst{PMIX_SUCCESS} indicating that the request has been accepted for processing and the provided callback function will be executed upon completion of the operation. Note that the library must not invoke the callback function prior to returning from the \ac{API}.
\item a non-zero \ac{PMIx} error constant indicating a reason for the request to have been rejected. In this case, the provided callback function will not be executed
\end{itemize}

%%%%
\descr

Non-blocking form of \refapi{PMIx_Fabric_get_vertex_info}. Data will be returned in the provided callback function.


%%%%%%%%%%%
\subsection{\code{PMIx_Fabric_get_device_index}}
\declareapi{PMIx_Fabric_get_device_index}

%%%%
\summary

Given vertex info, return the corresponding communication cost matrix index.

%%%%
\format

\versionMarker{4.0}
\cspecificstart
\begin{codepar}
pmix_status_t
PMIx_Fabric_get_device_index(pmix_fabric_t *fabric,
                      const pmix_info_t vertex[], size_t ninfo,
                      uint32_t *index)
\end{codepar}
\cspecificend

\begin{arglist}
\argin{fabric}{address of a \refstruct{pmix_fabric_t} (handle)}
\argin{vertex}{array of \refstruct{pmix_info_t} containing info describing the vertex whose index is being queried (handle)}
\argin{ninfo} number of elements in \refarg{vertex}
\argout{index}{pointer to the location where the index is to be returned (memory reference (handle))}
\end{arglist}

Returns one of the following:

\begin{itemize}
    \item \refconst{PMIX_SUCCESS}, indicating return of a valid value.
    \item a \ac{PMIx} error constant indicating either an error in the input or that the request failed.
\end{itemize}


%%%%
\descr

Query the index number of a vertex corresponding to the provided description. The description must provide adequate information to uniquely identify the target vertex. At a minimum, this must include identification of the node hosting the device using either the \refattr{PMIX_HOSTNAME} or \refattr{PMIX_NODEID}, plus a node-level unique identifier for the device (e.g., the \refattr{PMIX_FABRIC_DEVICE_PCI_DEVID} for a \ac{PCI} device).


%%%%%%%%%%%
\subsection{\code{PMIx_Fabric_get_device_index_nb}}
\declareapi{PMIx_Fabric_get_device_index_nb}

%%%%
\summary

Given vertex info, return the corresponding communication cost matrix index.

%%%%
\format

\versionMarker{4.0}
\cspecificstart
\begin{codepar}
pmix_status_t
PMIx_Fabric_get_device_index_nb(pmix_fabric_t *fabric,
                      const pmix_info_t vertex[], size_t ninfo,
                      pmix_info_cbfunc_t cbfunc, void *cbdata)
\end{codepar}
\cspecificend

\begin{arglist}
\argin{fabric}{address of a \refstruct{pmix_fabric_t} (handle)}
\argin{vertex}{array of \refstruct{pmix_info_t} containing info describing the vertex whose index is being queried (handle)}
\argin{ninfo} number of elements in \refarg{vertex}
\argin{cbfunc}{Callback function \refapi{pmix_info_cbfunc_t} (function reference)}
\argin{cbdata}{Data to be passed to the callback function (memory reference)}
\end{arglist}

Returns one of the following:

\begin{itemize}
\item \refconst{PMIX_SUCCESS} indicating that the request has been accepted for processing and the provided callback function will be executed upon completion of the operation. Note that the library must not invoke the callback function prior to returning from the \ac{API}.
\item a non-zero \ac{PMIx} error constant indicating a reason for the request to have been rejected. In this case, the provided callback function will not be executed
\end{itemize}


%%%%
\descr

Non-blocking form of \refapi{PMIx_Fabric_get_device_index}. Index will be returned in the provided callback function via the \refattr{PMIX_FABRIC_INDEX} attribute.

%%%%%%%%%%%%%%%%%%%%%%%%%%%%%%%%%%%%%%%%%%%%%%%%%
